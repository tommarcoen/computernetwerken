% Most entries are taken from the book "Data communications and networking"
% by Behrouz A. Forouzan; published by McGraw-Hill in 2007.
% or Computer networks by Larry L. Peterson and Bruce S. Davie;
% published by Morgan Kaufmann in 2022.
\newglossaryentry{10BASE2}{
   name = {\abbr{10BASE2}},
   description = {The thin coaxial cable implementation of standard Ethernet.}
}

\newglossaryentry{10BASE5}{
   name = {\abbr{10BASE5}},
   description = {The thick coaxial cable implementation of standard Ethernet.}
}

\newglossaryentry{10BASE-T}{
   name = {\abbr{10BASE5}},
   description = {The twisted-pair implementation of standard Ethernet.}
}

\newglossaryentry{100BASE-TX}{
   name = {\abbr{100BASE-TX}},
   description = {The two-wire \acs{UTP} implementation of Fast Ethernet.}
}

\newglossaryentry{1000BASE-LX}{
   name = {\abbr{1000BASE-LX}},
   description = {The two-wire fibre implementation of Gigabit Ethernet using long-wave lasers.}
}

\newglossaryentry{1000BASE-SX}{
   name = {\abbr{1000BASE-SX}},
   description = {The two-wire fibre implementation of Gigabit Ethernet using short-wave lasers.}
}

\newglossaryentry{1000BASE-T}{
   name = {\abbr{1000BASE-T}},
   description = {The four-wire \acs{UTP} implementation of Gigabit Ethernet.}
}

\newglossaryentry{802.3}{
   name = {802.3},
   description = {\acs{IEEE} Ethernet standard}
}

\newglossaryentry{802.11}{
   name = {802.11},
   description = {\acs{IEEE} wireless network standard}
}

\newglossaryentry{AP-gls}{
   name = {\acf{AP}},
   description = {A central base station in a \acf{BSS}.}
}

\newglossaryentry{ACK-gls}{
   name = {acknowledgment (\abbr{ACK})},
   description = {A response sent by the receiver to indicate the succesful receipt of data.}
}

\newglossaryentry{ARP-gls}{
   name = {\acf{ARP}},
   description = {A protocol for obtaining the physical (\acs{MAC}) address of a node when the network (\acs{IP}) address is known.}
}

\newglossaryentry{ARPA-gls}{
   name = {\acf{ARPA}},
   description = {
      One of the research and development organisations within the Department of Defense.
      Responsible for funding the ARPAnet as well as the research that led to the development of the \acs{TCP}/\acs{IP} internet.
      Also known as \abbr{DARPA}, the `D' standing for Defense.
   }
}

\newglossaryentry{authentication}{
   name = {authentication},
   description = {Security protocol by which two suspicious parties prove to each other that they are whom they claim to be.}
}

\newglossaryentry{autonegotiation}{
   name = {autonegotiation},
   description = {A Fast Ethernet feature that allows two devices to negotiate the mode or data rate.}
}

\newglossaryentry{bandwidth}{
   name = {bandwidth},
   description = {
      The difference between the highest and the lowest frequencies of a composite signal.
      It also measures the information-carrying capacity of a line or a network.
   }
}

\newglossaryentry{bss-gls}{
   name = {\acf{BSS}},
   description = {The building block of a wireless \acs{LAN} as defined by the \acs{IEEE} 802.11 standard.}
}

\newglossaryentry{bit}{
   name = {bit},
   description ={Binary digit. The smallest unit of data (0 or 1).}
}

\newglossaryentry{certificate}{
   name = {certificate},
   description = {
      A document digitally signed by one entity that contains the name and public key of another entity.
      Used to distribute public keys.
   }
}

\newglossaryentry{checksum}{
   name = {checksum},
   description = {
      Typically a ones' complement sum over some or all of the bytes of a packet, computed and appended to the packet by the sender.
      The receiver recomputes the checksum and compares it to the one carried in the message.
      Checksums are used to detect errors in a packet and may also be used to verify that the packet has been delivered to the correct host.
      The term \emph{checksum} is also sometimes (imprecisely) used to refer generically to error-detecting codes.
   }
}

\newglossaryentry{CIDR-gls}{
   name = {\acf{CIDR}},
   description = {
      A method of aggregating routes that treats a block of contiguous class~c \acs{IP} addresses as a single network.
   }
}

\newglossaryentry{client}{
   name = {client},
   description = {The requestor of a service in a distributed system.}
}

\newglossaryentry{CDN-gls}{
   name = {\acf{CDN}},
   description = {
      A collection of surrogate web servers, distributed across the internet, that respond to web \acs{HTTP} requests in place of the server.
      The goal of widely distributing the surrogate servers is to have a surrogate close the the client, making it possible to respond to requests more quickly.
   }
}

\newglossaryentry{PXE-gls}{
   name={preboot execution environment},
   description={The \acl{PXE} specification describes a standardized client–server environment that boots a software assembly, retrieved from a network, on \acs{PXE}-enabled clients.
   On the client side it requires only a \acs{PXE}-capable \acf{NIC}, and uses a small set of industry-standard network protocols such as \acs{DHCP} and \acs{TFTP}.}
}