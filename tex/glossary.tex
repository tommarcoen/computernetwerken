% Most entries are taken from the book "Data communications and networking"
% by Behrouz A. Forouzan; published by McGraw-Hill in 2007.
\newglossaryentry{10BASE2}{
   name = {\abbr{10BASE2}},
   description = {The thin coaxial cable implementation of standard Ethernet.}
}

\newglossaryentry{10BASE5}{
   name = {\abbr{10BASE5}},
   description = {The thick coaxial cable implementation of standard Ethernet.}
}

\newglossaryentry{10BASE-T}{
   name = {\abbr{10BASE5}},
   description = {The twisted-pair implementation of standard Ethernet.}
}

\newglossaryentry{100BASE-TX}{
   name = {\abbr{100BASE-TX}},
   description = {The two-wire \acs{UTP} implementation of Fast Ethernet.}
}

\newglossaryentry{1000BASE-LX}{
   name = {\abbr{1000BASE-LX}},
   description = {The two-wire fibre implementation of Gigabit Ethernet using long-wave lasers.}
}

\newglossaryentry{1000BASE-SX}{
   name = {\abbr{1000BASE-SX}},
   description = {The two-wire fibre implementation of Gigabit Ethernet using short-wave lasers.}
}

\newglossaryentry{1000BASE-T}{
   name = {\abbr{1000BASE-T}},
   description = {The four-wire \acs{UTP} implementation of Gigabit Ethernet.}
}

\newglossaryentry{AP-gls}{
   name = {\acf{AP}},
   description = {A central base station in a \acf{BSS}.}
}

\newglossaryentry{ACK-gls}{
   name = {acknowledgment (\abbr{ACK})},
   description = {A response sent by the receiver to indicate the succesful receipt of data.}
}

\newglossaryentry{ARP-gls}{
   name = {\acf{ARP}},
   description = {A protocol for obtaining the physical (\acs{MAC}) address of a node when the network (\acs{IP}) address is known.}
}

\newglossaryentry{ARPA-gls}{
   name = {\acf{ARPA}},
   description = {The government agency that funded the ARPAnet.}
}

\newglossaryentry{autonegotiation}{
   name = {autonegotiation},
   description = {A Fast Ethernet feature that allows two devices to negotiate the mode or data rate.}
}

\newglossaryentry{bandwidth}{
   name = {bandwidth},
   description = {
      The difference between the highest and the lowest frequencies of a composite signal.
      It also measures the information-carrying capacity of a line or a network.
   }
}

\newglossaryentry{bss-gls}{
   name = {\acf{BSS}},
   description = {The building block of a wireless \acs{LAN} as defined by the \acs{IEEE} 802.11 standard.}
}

\newglossaryentry{bit}{
   name = {bit},
   description ={Binary digit. The smallest unit of data (0 or 1).}
}

\newglossaryentry{PXE-gls}{
   name={preboot execution environment},
   description={The \acl{PXE} specification describes a standardized client–server environment that boots a software assembly, retrieved from a network, on \acs{PXE}-enabled clients.
   On the client side it requires only a \acs{PXE}-capable \acf{NIC}, and uses a small set of industry-standard network protocols such as \acs{DHCP} and \acs{TFTP}.}
}