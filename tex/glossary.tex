% Most entries are taken from the book "Data communications and networking"
% by Behrouz A. Forouzan; published by McGraw-Hill in 2007.
% or Computer networks by Larry L. Peterson and Bruce S. Davie;
% published by Morgan Kaufmann in 2022.
\newglossaryentry{10BASE2}{
   name = {\abbr{10BASE2}},
   sort = {0010BASE2},
   description = {The thin coaxial cable implementation of standard Ethernet.}
}

\newglossaryentry{10BASE5}{
   name = {\abbr{10BASE5}},
   sort = {0010BASE5},
   description = {The thick coaxial cable implementation of standard Ethernet.}
}

\newglossaryentry{10BASE-T}{
   name = {\abbr{10BASE-T}},
   sort = {0010BASE-T},
   description = {The twisted-pair implementation of standard Ethernet.}
}

\newglossaryentry{100BASE-TX}{
   name = {\abbr{100BASE-TX}},
   sort = {0100BASE-TX},
   description = {The two-wire \acs{UTP} implementation of Fast Ethernet.}
}

\newglossaryentry{1000BASE-LX}{
   name = {\abbr{1000BASE-LX}},
   sort = {1000BASE-LX},
   description = {The two-wire fibre implementation of Gigabit Ethernet using long-wave lasers.}
}

\newglossaryentry{1000BASE-SX}{
   name = {\abbr{1000BASE-SX}},
   sort = {1000BASE-SX},
   description = {The two-wire fibre implementation of Gigabit Ethernet using short-wave lasers.}
}

\newglossaryentry{1000BASE-T}{
   name = {\abbr{1000BASE-T}},
   sort = {1000BASE-T},
   description = {The four-wire \acs{UTP} implementation of Gigabit Ethernet.}
}

\newglossaryentry{802.3}{
   name = {802.3},
   sort = {0802.03},
   description = {\acs{IEEE} Ethernet standard}
}

\newglossaryentry{802.11}{
   name = {802.11},
   sort = {0802.11},
   description = {\acs{IEEE} wireless network standard}
}

\newglossaryentry{AP-gls}{
   name = {\acf{AP}},
   sort = {access point},
   description = {A central base station in a \acf{BSS}.}
}

\newglossaryentry{ACK-gls}{
   name = {acknowledgment (\abbr{ACK})},
   description = {A response sent by the receiver to indicate the succesful receipt of data.}
}

\newglossaryentry{ARP-gls}{
   name = {\acf{ARP}},
   sort = {Address Resolution Protocol},
   description = {A protocol for obtaining the physical (\acs{MAC}) address of a node when the network (\acs{IP}) address is known.}
}

\newglossaryentry{ARPA-gls}{
   name = {\acf{ARPA}},
   sort = {advanced research project agency},
   description = {
      One of the research and development organisations within the Department of Defense.
      Responsible for funding the ARPAnet as well as the research that led to the development of the \acs{TCP}/\acs{IP} internet.
      Also known as \abbr{DARPA}, the `D' standing for Defense.
   }
}

\newglossaryentry{authentication}{
   name = {authentication},
   description = {Security protocol by which two suspicious parties prove to each other that they are whom they claim to be.}
}

\newglossaryentry{autonegotiation}{
   name = {autonegotiation},
   description = {A Fast Ethernet feature that allows two devices to negotiate the mode or data rate.}
}

\newglossaryentry{AS-gls}{
   name = {\acf{AS}},
   sort = {autonomous system},
   description = {A group of networks and routers, subject to a common authority and using the same intradomain routing protocol.}
}

\newglossaryentry{bandwidth}{
   name = {bandwidth},
   description = {
      The difference between the highest and the lowest frequencies of a composite signal.
      It also measures the information-carrying capacity of a line or a network.
   }
}

\newglossaryentry{baud rate}{
   name = {baud rate},
   description = {
      The number of signal elements transmitted per second.
      A signal element consists of one or more bits.
   }
}

\newglossaryentry{bss-gls}{
   name = {\acf{BSS}},
   sort = {basic service set},
   description = {The building block of a wireless \acs{LAN} as defined by the \acs{IEEE} 802.11 standard.}
}

\newglossaryentry{bit}{
   name = {bit},
   description ={Binary digit. The smallest unit of data (0 or 1).}
}

\newglossaryentry{bit rate}{
   name = {bit rate},
   description = {The number of bits transmitted per second.}
}

\newglossaryentry{BOOTP-gls}{
   name = {\acf{BOOTP}},
   sort = {bootstrap protocol},
   description = {
      The protocol that provides configuration information from a table (file).
      This protocol is deprecated and has been replaced by \acs{DHCP}.
   }
}

\newglossaryentry{broadcast address}{
   name = {broadcast address},
   description = {An address that allows transmission of a message to all nodes of a network.}
}

\newglossaryentry{bus topology}{
   name = {bus topology},
   description = {A network topology in which all computers are attached to a shared medium.}
}

\newglossaryentry{caching}{
   name = {caching},
   description = {The storing of information in small, fast memory.}
}

\newglossaryentry{certificate}{
   name = {certificate},
   description = {
      A document digitally signed by one entity that contains the name and public key of another entity.
      Used to distribute public keys.
   }
}

\newglossaryentry{checksum}{
   name = {checksum},
   description = {
      Typically a ones' complement sum over some or all of the bytes of a packet, computed and appended to the packet by the sender.
      The receiver recomputes the checksum and compares it to the one carried in the message.
      Checksums are used to detect errors in a packet and may also be used to verify that the packet has been delivered to the correct host.
      The term \emph{checksum} is also sometimes (imprecisely) used to refer generically to error-detecting codes.
   }
}

\newglossaryentry{CIDR-gls}{
   name = {\acf{CIDR}},
   sort = {classless interdomain routing},
   description = {
      A method of aggregating routes that treats a block of contiguous class~c \acs{IP} addresses as a single network.
   }
}

\newglossaryentry{client}{
   name = {client},
   description = {The requestor of a service in a distributed system.}
}

\newglossaryentry{CDN-gls}{
   name = {\acf{CDN}},
   sort = {content delivery network},
   description = {
      A collection of surrogate web servers, distributed across the internet, that respond to web \acs{HTTP} requests in place of the server.
      The goal of widely distributing the surrogate servers is to have a surrogate close the the client, making it possible to respond to requests more quickly.
   }
}

\newglossaryentry{datagram}{
   name = {datagram},
   description = {
      The basic transmission unit in the internet architecture.
      A datagram contains all of the inforamtion needed to deliver it to its destination, analogous to a letter in the \abbr{US} postal system.
      Datagram networks are connectionless.
   }
}

\newglossaryentry{distance vector}{
   name = {distance vector},
   description = {
      A lowest-cost-path algorithm used in routing.
      Each node advertises reachability information and associated costs to its immediate neighbours and uses the updates it recevies to construct its forwarding table.
      The routing protocol \acs{RIP} uses a distance-vector algorithm.
      Contrast with \gls{link state}.
   }
}

\newglossaryentry{domain}{
   name = {domain},
   description = {
      Can refer either to a context in the hierarchical \acs{DNS} name space (e.g., the `edu' domain) or to a region of the internet that is treated as a single entity for the purpose of hierarchical routing.
      The latter is equivalent to \gls{AS-gls}.
   }
}

\newglossaryentry{DNS-gls}{
   name = {\acf{DNS}},
   sort = {domain name system},
   description = {
      The distributed naming system of the internet, used to resolve host names into \acs{IP} addresses.
      The \acs{DNS} is implemented by a hierarchy of name servers.
   }
}

\newglossaryentry{default gateway}{
   name = {default gateway},
   description = {The router used for default routing.}
}

\newglossaryentry{default routing}{
   name = {default routing},
   description = {A routing method in which a router (the default gateway) is assigned to receive all packets with no match in the routing table.}
}

\newglossaryentry{distance-vector routing}{
   name = {distance-vector routing},
   description = {A routing method in which each router sends its neighbour a list of networks it can reach and the distance to that network.}
}

\newglossaryentry{dotted-decimal notation}{
   name = {dotted-decimal notation},
   description = {A notation devised to make the \acs{IP} address easier to read; each byte is converted to its decimal equivalent and then set off from its neighbour by a dot.}
}

\newglossaryentry{dynamic routing}{
   name = {dynamic routing},
   description = {Routing in which the routing table entries are updated automatically by the routing protocol.}
}

\newglossaryentry{encapsulation}{
   name = {encapsulation},
   description = {
      The operation, performed by a lower-level protocol, of attaching a protocol-specific header and/or trailer to a message passed down by a higher-level protocol.
      As a message travels down the protocol stack, it gathers a sequence of headers, of which the outermost corresponds to the protocol at the bottom of the stack.
   }
}

\newglossaryentry{end system}{
   name = {end system},
   description = {A sender or receiver of data.}
}

\newglossaryentry{ephemeral port number}{
   name = {ephemeral port number},
   description = {
      A communications endpoint (port) of a transport-layer protocol of the internet protocol suite that is used for only a short period of time for the duration of a communication session.
      Such short-lived ports are allocated automatically within a predefined range of port numbers by the \acs{IP} stack of an operating system (\vref{tab:ephemeral-ports}).
      Ephemeral port numbers are typically used for the client-end of a client--server communication.
   }
}

\newglossaryentry{extended service set}{
   name = {extended service set},
   description = {A wireless \acs{LAN} service, created by two or more \acf{AP}, which appear to users as a single, seamless network.}
}

\newglossaryentry{extranet}{
   name = {extranet},
   description = {A private network that allows authorised access from outside users.}
}

\newglossaryentry{Fast Ethernet}{
   name = {Fast Ethernet},
   description = {Ethernet with a data rate of \SI{100}{\mega\bit\per\second}.}
}

\newglossaryentry{fast retransmission}{
   name = {fast restransmission},
   description = {Retransmission of a segment in \acs{TCP} when three acknowledgments have been received that imply the loss or corruption of that segment.}
}

\newglossaryentry{fibre-optic cable}{
   name = {fibre-optic cable},
   description = {
      A high-bandwidth transmission medium that carries data signals in the form of pulses of light.
      It consists of a thin cylinder of glass or plastic, called the core, surrounded by a concentric layer of glass or plastic called the cladding.
   }
}

\newglossaryentry{firewall}{
   name = {firewall},
   description = {A device (usually a router) installed between the internal network of an organisation and the rest of the internet to provide security.}
}

\newglossaryentry{frame}{
   name = {frame},
   description = {A group of bits representing a block of data.}
}

\newglossaryentry{frequency}{
   name = {frequency},
   description = {
      The number of cycles per second of a periodic signal.
      The frequency is measured in hertz (Hz) which is equal to one event per second.
   }
}

\newglossaryentry{full-duplex mode}{
   name = {full-duplex mode},
   description = {A transmission mode in which both parties can communicate simultaneously.}
}

\newglossaryentry{gateway}{
   name = {gateway},
   description = {A different name for a router.}
}

\newglossaryentry{GHz}{
   name = {GHz},
   description = {gigahertz; \SI{1}{\giga\hertz} = \SI{1e9}{\hertz} or one billion cycles per second.}
}

\newglossaryentry{Gigabit Ethernet}{
   name = {Gigabit Ethernet},
   description = {Ethernet with a data rate of \SI{1000}{\mega\bit\per\second}.}
}

\newglossaryentry{half-duplex mode}{
   name = {half-duplex mode},
   description  = {
      A transmission mode in which communication can be two-way but not at the same time.
   }
}

\newglossaryentry{host}{
   name = {host},
   description = {A computer attached to one or more networks that supports users and runs application programs.}
}

\newglossaryentry{Hz}{
   name = {Hz},
   description = {
      hertz; the unit of frequency in the International System of Units (\acs{SI}).
      See also \emph{frequency}.
   }
}

\newglossaryentry{IPsec}{
   name = {IPsec},
   description = {A collection of protocols designed by the \acs{IETF} to provide security for a packet carried over the internet.}
}

\newglossaryentry{link}{
   name = {link},
   description = {The physical communication pathway that transfers data from one device to another.}
}

\newglossaryentry{link state}{
   name = {link state},
   description = {
      A lowest-cost-path algorithm used in routing.
      Information on directly connected neighbours and current link costs are flooded to all routers;
      each router uses this information to build a view of the network on which to base forwarding decisions.
      The \acs{OSPF} routing protocol uses a link-state algorithm.
      Contrast with \gls{distance vector}.
   }
}

\newglossaryentry{logical address}{
   name = {logical address},
   description = {An address defined in the network layer.}
}

\newglossaryentry{MHz}{
   name = {MHz},
   description = {megahertz; \SI{1}{\mega\hertz} = \SI{1e6}{\hertz} or one million cycles per second.}
}

\newglossaryentry{MTA-gls2}{
   name = {mail transfer agent},
   description = {See \acf{MTA}.}
}

\newglossaryentry{MTU-gls}{
   name = {\acf{MTU}},
   sort = {maximum transmission unit},
   description = {The largest size data unit a specific network can handle.}
}

\newglossaryentry{MTA-gls}{
   name = {\acf{MTA}},
   sort = {message transfer agent},
   description = {An \acs{SMTP} component that transfers the message across the internet.}
}

\newglossaryentry{metric}{
   name = {metric},
   description = {A cost assigned for passing through a network.}
}

\newglossaryentry{network}{
   name = {network},
   description = {A system consisting of interconnected nodes that share data, hardware, and software.}
}

\newglossaryentry{network address}{
   name = {network address},
   description = {
      Ad address that identifies a network to the rest of the internet.
      It is the first address in a block.
   }
}

\newglossaryentry{NAT-gls}{
   name = {\acf{NAT}},
   sort = {network address translation},
   description = {A technologgy that allows a private network to use a set of private addresses for internal communication and a set of public addresses for external communication.}
}

\newglossaryentry{NIC-gls}{
   name = {\acf{NIC}},
   sort = {network-interface card},
   description = {An electronic device, internal or external to a station, that contains circuitry to enable the station to be connected to a network.}
}

\newglossaryentry{node}{
   name = {node},
   description = {
      A generic term used for individual computers that make up a network.
      Nodes include general-purpose computers, switches, and routers.
   }
}

\newglossaryentry{omnidirectional antenna}{
   name = {omnidirectional antenna},
   description = {An antenna that sends out or receives signals in all directions.}
}

\newglossaryentry{optical fibre}{
   name = {optical fibre},
   description = {A thin thread of glass or another transparent material to carry light beams.}
}

\newglossaryentry{OSI model}{
   name = {\acs{OSI} model},
   sort = {OSI model},
   description = {A seven-layer model for data communication defined by the \acf{ISO}.}
}

\newglossaryentry{packet-filter firewall}{
   name = {packet-filter firewall},
   description = {A firewall that forwards or blocks packets based on information in the network-layer and transport-layer headers.}
}

\newglossaryentry{physical address}{
   name = {physical address},
   description = {The address of a device at the data-link layer (\acs{MAC} address).}
}

\newglossaryentry{PPP connection}{
   name = {point-to-point connection},
   description = {A dedicated transmission link between two devices.}
}

\newglossaryentry{port address}{
   name = {port address},
   description = {In \acs{TCP}/\acs{IP} an integer that identifies a process or application.}
}

\newglossaryentry{port number}{
   name = {port number},
   description = {See \gls{port address}.}
}

\newglossaryentry{prefix}{
   name = {prefix},
   description = {The common part of an address range.}
}

\newglossaryentry{PXE-gls}{
   name={preboot execution environment},
   description={The \acl{PXE} specification describes a standardised client--server environment that boots a software assembly, retrieved from a network, on \acs{PXE}-enabled clients.
   On the client side it requires only a \acs{PXE}-capable \acf{NIC}, and uses a small set of industry-standard network protocols such as \acs{DHCP} and \acs{TFTP}.}
}

\newglossaryentry{registrar}{
   name = {registrar},
   description = {An authority to register new \acs{DNS} names.}
}

\newglossaryentry{relay agent}{
   name = {relay agent},
   description = {Foor \acs{BOOTP} and \acs{DHCP}, a device -- usually a router -- that can help send local requests to remote servers.   }
}

\newglossaryentry{root server}{
   name = {root server},
   description = {
      In \acs{DNS}, a server whose zone consists of the whole tree.
      A root server usually does not store any information about domains but delegates its authority to other servers, keeping references to those servers.
   }
}
\newglossaryentry{RTT-gls}{
   name = {\acf{RTT}},
   sort = {round-trip time},
   description = {The time required for a packet to travel from the source to the destination and back again to the source.}
}
\newglossaryentry{router}{
   name = {router},
   description = {
      An internetworking device which operates mainly at the third layer of the \acs{OSI} model (network layer).
      A router connects two or more networks together and forwards packets from one network to another.
   }
}

\newglossaryentry{routing table}{
   name = {routing table},
   description = {
      A table containing information a router needs to forward packets to other networks.
      The information may include the network address, the cost, the address of the next-hop router, and so on.
   }
}

\newglossaryentry{subnet mask}{
   name = {subnet mask},
   description = {The mask used to separate the network number from the host identifier.}
}

\newglossaryentry{switch}{
   name = {switch},
   description = {
      A device connecting multiple communication lines together.
      A switch operates on the second layer of the \acs{OSI} model (the data link layer) but can also operate on higher layers.
   }
}

\newglossaryentry{switched Ethernet}{
   name = {switched Ethernet},
   description = {An Ethernet in which a switch, replacing the hub, can direct a transmission to its destination.}
}

\newglossaryentry{three-way handshake}{
   name = {three-way handshake},
   description = {A sequence of events for connection establishment consisting of a synchronisation packet sent by the first host, an acknowledgement from the second host which also includes its own syncrhonisation information, and finally the acknowledgment from the first host.}
}

\newglossaryentry{throughput}{
   name = {throughtput},
   description = {The number of bits that can pass through a point in one second.}
}

\newglossaryentry{TTL-gls}{
   name = {\acf{TTL}},
   sort = {time-to-live},
   description = {The lifetime of a packet.}
}

\newglossaryentry{unicast routing}{
   name = {unicast routing},
   description = {The sending of a packet to just one destination.}
}

\newglossaryentry{unidirectional antenna}{
   name = {unidirectional antenna},
   description = {An antenna that sends and receives signals in one direction.}
}

\newglossaryentry{VLAN-gls}{
   name = {\acf{VLAN}},
   sort = {virtual local area network},
   description = {A technology that divides a physical \acs{LAN} into virtual workgroups through software methods.}
}
\newglossaryentry{VPN-gls}{
   name = {\acf{VPN}},
   sort = {virtual private network},
   description = {A technology that creates a network that is physically public, but virtually private.}
}

\newglossaryentry{well-known port number}{
   name = {well-known port number},
   description = {A port number that identifies a process on the server.}
}

\newglossaryentry{WAN-gls}{
   name = {\acf{WAN}},
   sort = {wide-area network},
   description = {A network that uses a technology that can span a large geographical distance.}
}

\newglossaryentry{zone}{
   name = {zone},
   description = {In \acs{DNS}, what a server is responsible for or has authority over.}
}