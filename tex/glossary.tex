% Most entries are taken from the book "Data communications and networking"
% by Behrouz A. Forouzan; published by McGraw-Hill in 2007.
\newglossaryentry{10BASE2}{
   name = {\abbr{10BASE2}},
   sort = {0010BASE2},
   description = {The thin coaxial cable implementation of standard Ethernet.}
}

\newglossaryentry{10BASE5}{
   name = {\abbr{10BASE5}},
   sort = {0010BASE5},
   description = {The thick coaxial cable implementation of standard Ethernet.}
}

\newglossaryentry{10BASE-T}{
   name = {\abbr{10BASE-T}},
   sort = {0010BASE-T},
   description = {The twisted-pair implementation of standard Ethernet.}
}

\newglossaryentry{100BASE-TX}{
   name = {\abbr{100BASE-TX}},
   sort = {0100BASE-TX},
   description = {The two-wire \acs{UTP} implementation of Fast Ethernet.}
}

\newglossaryentry{1000BASE-LX}{
   name = {\abbr{1000BASE-LX}},
   sort = {1000BASE-LX},
   description = {The two-wire fibre implementation of Gigabit Ethernet using long-wave lasers.}
}

\newglossaryentry{1000BASE-SX}{
   name = {\abbr{1000BASE-SX}},
   sort = {1000BASE-SX},
   description = {The two-wire fibre implementation of Gigabit Ethernet using short-wave lasers.}
}

\newglossaryentry{1000BASE-T}{
   name = {\abbr{1000BASE-T}},
   sort = {1000BASE-T},
   description = {The four-wire \acs{UTP} implementation of Gigabit Ethernet.}
}

\newglossaryentry{AP-gls}{
   name = {\acf{AP}},
   sort = {access point},
   description = {A central base station in a \acf{BSS}.}
}

\newglossaryentry{ACK-gls}{
   name = {acknowledgment (\abbr{ACK})},
   description = {A response sent by the receiver to indicate the succesful receipt of data.}
}

\newglossaryentry{ARP-gls}{
   name = {\acf{ARP}},
   sort = {ARP},
   description = {A protocol for obtaining the physical (\acs{MAC}) address of a node when the network (\acs{IP}) address is known.}
}

\newglossaryentry{ARPA-gls}{
   name = {\acf{ARPA}},
   sort = {ARPA},
   description = {The government agency that funded the ARPAnet.}
}

\newglossaryentry{autonegotiation}{
   name = {autonegotiation},
   description = {A Fast Ethernet feature that allows two devices to negotiate the mode or data rate.}
}

\newglossaryentry{bandwidth}{
   name = {bandwidth},
   description = {
      The difference between the highest and the lowest frequencies of a composite signal.
      It also measures the information-carrying capacity of a line or a network.
   }
}

\newglossaryentry{baud rate}{
   name = {baud rate},
   description = {
      The number of signal elements transmitted per second.
      A signal element consists of one or more bits.
   }
}

\newglossaryentry{bss-gls}{
   name = {\acf{BSS}},
   sort = {basic service set},
   description = {The building block of a wireless \acs{LAN} as defined by the \acs{IEEE} 802.11 standard.}
}

\newglossaryentry{bit}{
   name = {bit},
   description ={Binary digit. The smallest unit of data (0 or 1).}
}

\newglossaryentry{bit rate}{
   name = {bit rate},
   description = {The number of bits transmitted per second.}
}

\newglossaryentry{BOOTP-gls}{
   name = {\acf{BOOTP}},
   sort = {bootstrap protocol},
   description = {
      The protocol that provides configuration information from a table (file).
      This protocol is deprecated and has been replaced by \acs{DHCP}.
   }
}

\newglossaryentry{broadcast address}{
   name = {broadcast address},
   description = {An address that allows transmission of a message to all nodes of a network.}
}

\newglossaryentry{bus topology}{
   name = {bus topology},
   description = {A network topology in which all computers are attached to a shared medium.}
}

\newglossaryentry{caching}{
   name = {caching},
   description = {The storing of information in small, fast memory.}
}

%\newglossaryentry{datagram}{
%   name = {datagram},
%   description = {The \acf{IP} data unit.}
%}

\newglossaryentry{full-duplex mode}{
   name = {full-duplex mode},
   description = {A transmission mode in which both parties can communicate simultaneously.}
}

\newglossaryentry{gateway}{
   name = {gateway},
   description = {A different name for a router.}
}

\newglossaryentry{half-duplex mode}{
   name = {half-duplex mode}
   description  = {
      A transmission mode in which communication can be two-way but not at the same time.
   }
}
\newglossaryentry{host}{
   name = {host},
   description = {A station or node on a network.}
}

\newglossaryentry{IPsec}{
   name = {IPsec},
   description = {A collection of protocols designed by the \acs{IETF} to provide security for a packet carried over the internet.}
}

\newglossaryentry{link}{
   name = {link},
   description = {The physical communication pathway that transfers data from one device to another.}
}

\newglossaryentry{logical address}{
   name = {logical address},
   description = {An address defined in the network layer.}
}

\newglossaryentry{MTA-gls2}{
   name = {mail transfer agent},
   description = {See \acf{MTA}.}
}

\newglossaryentry{MTU-gls}{
   name = {\acf{MTU}},
   sort = {maximum transmission unit},
   description = {The largest size data unit a specific network can handle.}
}

\newglossaryentry{MTA-gls}{
   name = {\acf{MTA}},
   sort = {message transfer agent},
   description = {An \acs{SMTP} component that transfers the message across the internet.}
   }
}

\newglossaryentry{metric}{
   name = {metric},
   description = {A cost assigned for passing through a network.}
}

\newglossaryentry{network}{
   name = {network},
   description = {A system consisting of interconnected nodes that share data, hardware, and software.}
}

\newglossaryentry{network address}{
   name = {network address},
   description = {
      Ad address that identifies a network to the rest of the internet.
      It is the first address in a block.
   }
}

\newglossaryentry{NAT-gls}{
   name = {\acf{NAT}},
   sort = {network address translation},
   description = {A technologgy that allows a private network to use a set of private addresses for internal communication and a set of public addresses for external communication.}
}

\newglossaryentry{NIC-gls}{
   name = {\acf{NIC}},
   sort = {network-interface card},
   description = {An electronic device, internal or external to a station, that contains circuitry to enable the station to be connected to a network.}
}

\newglossaryentry{node}{
   name = {node},
   description = {An addressable communication device (e.g.~a computer or router) on a network.}
}

\newglossaryentry{omnidirectional antenna}{
   name = {omnidirectional antenna},
   description = {An antenna that sends out or receives signals in all directions.}
}

\newglossaryentry{optical fibre}{
   name = {optical fibre},
   description = {A thin thread of glass or another transparent material to carry light beams.}
}

\newglossaryentry{OSI model}{
   name = {\acs{OSI} model},
   sort = {OSI model},
   description = {A seven-layer model for data communication defined by the \acf{ISO}.}
}

\newglossaryentry{packet-filter firewall}{
   name = {packet-filter firewall},
   description = {A firewall that forwards or blocks packets based on information in the network-layer and transport-layer headers.}
}

\newglossaryentry{physical address}{
   name = {physical address},
   description = {The address of a device at the data-link layer (\acs{MAC} address).}
}

\newglossaryentry{PPP connection}{
   name = {point-to-point connection},
   description = {A dedicated transmission link between two devices.}
}

\newglossaryentry{port address}{
   name = {port address},
   description = {In \acs{TCP}/\acs{IP} an integer that identifies a process or application.}
}

\newglossaryentry{port number}{
   name = {port number},
   description = {See \emph{port address}.}
}

\newglossaryentry{prefix}{
   name = {prefix},
   description = {The common part of an address range.}
}

\newglossaryentry{PXE-gls}{
   name={preboot execution environment},
   description={The \acl{PXE} specification describes a standardized client–server environment that boots a software assembly, retrieved from a network, on \acs{PXE}-enabled clients.
   On the client side it requires only a \acs{PXE}-capable \acf{NIC}, and uses a small set of industry-standard network protocols such as \acs{DHCP} and \acs{TFTP}.}
}

\newglossaryentry{registrar}{
   name = {registrar},
   description = {An authority to register new \acs{DNS} names.}
}

\newglossaryentry{relay agent}{
   name = {relay agent},
   description = {Foor \acs{BOOTP} and \acs{DHCP}, a device -- usually a router -- that can help send local requests to remote servers.   }
}

\newglossaryentry{root serveR}{
   name = {root server},
   description = {
      In \acs{DNS}, a server whose zone consists of the whole tree.
      A root server usually does not store any information about domains but delegates its authority to other servers, keeping references to those servers.
   }
}
\newglossaryentry{RTT-gls}{
   name = {\acf{RTT}},
   sort = {round-trip time},
   description = {The time required for a packet to travel from the source to the destination and back again to the source.}
}
\newglossaryentry{router}{
   name = {router},
   description = {
      An internetworking device which operates mainly at the third layer of the \acs{OSI} model (network layer).
      A router connects two or more networks together and forwards packets from one network to another.
   }
}

\newglossaryentry{routing table}{
   name = {routing table},
   description = {
      A table containing information a router needs to forward packets to other networks.
      The information may include the network address, the cost, the address of the next-hop router, and so on.
   }
}

\newglossaryentry{subnet mask}{
   name = {subnet mask},
   description = {The mask used to separate the network number from the host identifier.}
}

\newglossaryentry{switch}{
   name = {switch},
   description = {
      A device connecting multiple communication lines together.
      A switch operates on the second layer of the \acs{OSI} model (the data link layer) but can also operate on higher layers.
   }
}

\newglossaryentry{switched Ethernet}{
   name = {switched Ethernet},
   description = {An Ethernet in which a switch, replacing the hub, can direct a transmission to its destination.}
}

\newglossaryentry{three-way handshake}{
   name = {three-way handshake},
   description = {A sequence of events for connection establishment consisting of a synchronisation packet sent by the first host, an acknowledgement from the second host which also includes its own syncrhonisation information, and finally the acknowledgment from the first host.}
}

\newglossaryentry{throughput}{
   name = {throughtput},
   description = {The number of bits that can pass through a point in one second.}
}

\newglossaryentry{TTL-gls}{
   name = {\acf{TTL}},
   sort = {time-to-live},
   description = {The lifetime of a packet.}
}

\newglossaryentry{unicast routing}{
   name = {unicast routing},
   description = {The sending of a packet to just one destination.}
}

\newglossaryentry{unidirectional antenna}{
   name = {unidirectional antenna},
   description = {An antenna that sends and receives signals in one direction.}
}

\newglossaryentry{VLAN-gls}{
   name = {\acf{VLAN}},
   sort = {virtual local area network},
   description = {A technology that divides a physical \acs{LAN} into virtual workgroups through software methods.}
}
\newglossaryentry{VPN-gls}{
   name = {\acf{VPN}},
   sort = {virtual private network},
   description = {A technology that creates a network that is physically public, but virtually private.}
}

\newglossaryentry{well-known port number}{
   name = {well-known port number},
   description = {A port number that identifies a process on the server.}
}

\newglossaryentry{WAN-gls}{
   name = {\acf{WAN}},
   sort = {wide-area network},
   description = {A network that uses a technology that can span a large geographical distance.}
}

\newglossaryentry{zone}{
   name = {zone},
   description = {In \acs{DNS}, what a server is responsible for or has authority over.}
}