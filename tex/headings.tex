\let\oldchapter\chapter
\renewcommand{\chapter}[1]{
   \chapter{#1}
   \only<beamer>{\section{#1}}
   {\setbeamercolor{background canvas}{bg=backgroundchapter}
   \mode<beamer>{
   \begin{frame}
   \begin{center}
   \Large\textbf{#1}
   \end{center}
   \end{frame}}
}}

\newcommand{\Chapter}[1]{
\only<article>{\chapter{#1}}
\only<beamer>{\section{#1}}
{\setbeamercolor{background canvas}{bg=backgroundchapter}
\mode<beamer>{
\begin{frame}
\begin{center}
\Large\textbf{#1}
\end{center}
\end{frame}}
}}

\newcommand{\Section}[1]{
\only<article>{\section{#1}}
\only<beamer>{\subsection{#1}}
\mode<beamer>{
{\setbeamercolor{background canvas}{bg=backgroundsection}
\begin{frame}
\begin{center}
\Large\textbf{#1}
\end{center}
\end{frame}}
}}

\NewDocumentCommand{\Paragraph}{om}{
\only<article>{\paragraph{#2\IfNoValueF{#1}{ (#1)}}}
\mode<beamer>{
\begin{frame}
\begin{center}
\Large #2

\IfNoValueF{#1}{{\footnotesize\color{slidesubtext} #1}}
\end{center}
\end{frame}
}}

\newcommand{\slide}[2][]{
\mode<beamer>{
\begin{frame}
\begin{center}
\Large #2

{\footnotesize\color{slidesubtext} #1}
\end{center}
\end{frame}
}}

\newcommand{\titleslide}[2][]{
\mode<beamer>{
{\setbeamercolor{background canvas}{bg=backgroundtitle}
\begin{frame}
\begin{center}
\Large\textsc{#2}

{\footnotesize\color{white} \textit{#1}}
\end{center}
\end{frame}
}
}
}