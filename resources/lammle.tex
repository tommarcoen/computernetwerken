\includegraphics{images/cover.jpg}

\protect\hypertarget{f_00.xhtml}{}{}

\protect\hypertarget{f_01.xhtml}{}{}

\protect\hypertarget{f_01.xhtmlux5cux23Page_iv}{}{}Senior Acquisitions
Editor: Kenyon Brown\\
Development Editor: Kim Wimpsett\\
Technical Editor: Todd Montgomery\\
Production Editor: Christine O'Connor\\
Copy Editor: Judy Flynn\\
Editorial Manager: Mary Beth Wakefield\\
Production Manager: Kathleen Wisor\\
Executive Publisher: Jim Minatel\\
Book Designers: Judy Fung and Bill Gibson\\
Proofreader: Josh Chase, Word One New York\\
Indexer: Johnna vanHoose Dinse\\
Project Coordinator, Cover: Brent Savage\\
Cover Designer: Wiley\\
Cover Image: Getty Images Inc./Jeremy Woodhouse

Copyright © 2016 by John Wiley \& Sons, Inc., Indianapolis, Indiana

Published simultaneously in Canada

ISBN: 978-1-119-28828-2\\
ISBN: 978-1-119-28830-5 (ebk.)\\
ISBN: 978-1-119-28829-9 (ebk.)

Manufactured in the United States of America

No part of this publication may be reproduced, stored in a retrieval
system or transmitted in any form or by any means, electronic,
mechanical, photocopying, recording, scanning or otherwise, except as
permitted under Sections 107 or 108 of the 1976 United States Copyright
Act, without either the prior written permission of the Publisher, or
authorization through payment of the appropriate per-copy fee to the
Copyright Clearance Center, 222 Rosewood Drive, Danvers, MA 01923, (978)
750-8400, fax (978) 646-8600. Requests to the Publisher for permission
should be addressed to the Permissions Department, John Wiley \& Sons,
Inc., 111 River Street, Hoboken, NJ 07030, (201) 748-6011, fax (201)
748-6008, or online at \url{http://www.wiley.com/go/permissions}.

Limit of Liability/Disclaimer of Warranty: The publisher and the author
make no representations or warranties with respect to the accuracy or
completeness of the contents of this work and specifically disclaim all
warranties, including without limitation warranties of fitness for a
particular purpose. No warranty may be created or extended by sales or
promotional materials. The advice and strategies contained herein may
not be suitable for every situation. This work is sold with the
understanding that the publisher is not engaged in rendering legal,
accounting, or other professional services. If professional assistance
is required, the services of a competent professional person should be
sought. Neither the publisher nor the author shall be liable for damages
arising herefrom. The fact that an organization or Web site is referred
to in this work as a citation and/or a potential source of further
information does not mean that the author or the publisher endorses the
information the organization or Web site may provide or recommendations
it may make. Further, readers should be aware that Internet Web sites
listed in this work may have changed or disappeared between when this
work was written and when it is read.

For general information on our other products and services or to obtain
technical support, please contact our Customer Care Department within
the U.S. at (877) 762-2974, outside the U.S. at (317) 572-3993 or fax
(317) 572-4002.

Wiley publishes in a variety of print and electronic formats and by
print-on-demand. Some material included with standard print versions of
this book may not be included in e-books or in print-on-demand. If this
book refers to media such as a CD or DVD that is not included in the
version you purchased, you may download this material at
\url{http://booksupport.wiley.com}. For more information about Wiley
products, visit \href{http://www.wiley.com}{www.wiley.com}.

Library of Congress Control Number: 2016950861

TRADEMARKS: Wiley, the Wiley logo, and the Sybex logo are trademarks or
registered trademarks of John Wiley \& Sons, Inc. and/or its affiliates,
in the United States and other countries, and may not be used without
written permission. CCNA is a registered trademark of Cisco Technology,
Inc. All other trademarks are the property of their respective owners.
John Wiley \& Sons, Inc. is not associated with any product or vendor
mentioned in this book.

\protect\hypertarget{f_02.xhtml}{}{}

\section[{Acknowledgments}]{\texorpdfstring{\protect\hypertarget{f_02.xhtmlux5cux23Page_v}{}{}{Acknowledgments}}{Acknowledgments}}

There are many people who work to put a book together, and as an author,
I dedicated an enormous amount of time to write this book, but it would
have never been published without the dedicated, hard work of many other
people.

Kenyon Brown, my acquisitions editor, is instrumental to my success in
the world of Cisco certification. Ken, I look forward to our continued
progress together in both the print and video markets! My technical
editor, Todd Montgomery, was absolutely amazing to work with and he was
always there to check my work and make suggestions. Thank you! Also,
I've worked with Kim Wimpsett, the development editor, for years now and
she coordinated all the pages you hold in your hands as they flew from
thoughts in my head to the production process.

Christine O'Connor, my production editor, and Judy Flynn, my copyeditor,
were my rock and foundation for formatting and intense editing of every
page in this book. This amazing team gives me the confidence to help
keep me moving during the difficult and very long days, week after week.
How Christine stays so organized with all my changes as well as making
sure every figure is in the right place in the book is still a mystery
to me! You're amazing, Christine! Thank you! Judy understands my writing
style so well now, after doing at least a dozen books with me, that she
even sometimes finds a technical error that may have slipped through as
I was going through the material. Thank you, Judy, for doing such a
great job! I truly thank you both.

\protect\hypertarget{f_03.xhtml}{}{}

\section[{About the
Author}]{\texorpdfstring{\protect\hypertarget{f_03.xhtmlux5cux23Page_vi}{}{}{About
the Author}}{About the Author}}

\textbf{Todd Lammle} is the authority on Cisco certification and
internetworking and is Cisco certified in most Cisco certification
categories. He is a world-renowned author, speaker, trainer, and
consultant. Todd has three decades of experience working with LANs,
WANs, and large enterprise licensed and unlicensed wireless networks,
and lately he's been implementing large Cisco Firepower networks. His
years of real-world experience are evident in his writing; he is not
just an author but an experienced networking engineer with very
practical experience working on the largest networks in the world, at
such companies as Xerox, Hughes Aircraft, Texaco, AAA, Cisco, and
Toshiba, among many others. Todd has published over 60 books, including
the very popular \emph{CCNA: Cisco Certified Network Associate Study
Guide, CCNA Wireless Study Guide, CCNA Data Center Study Guide, and
SSFIPS (Firepower)}, all from Sybex. He runs an international consulting
and training company based in Colorado, Texas, and San Francisco.

You can reach Todd through his forum and blog at
\href{http://www.lammle.com/ccna}{www.lammle.com/ccna}.

\protect\hypertarget{ftoc.xhtml}{}{}

\subsection{\texorpdfstring{\textbf{List of Tables}}{List of Tables}}

\begin{enumerate}
\tightlist
\item
  \protect\hyperlink{f_04.xhtml}{Introduction}

  \begin{enumerate}
  \tightlist
  \item
    \protect\hyperlink{f_04.xhtmlux5cux23table-I-1}{\textbf{Table I.1}}
  \item
    \protect\hyperlink{f_04.xhtmlux5cux23table-I-2}{\textbf{Table I.2}}
  \item
    \protect\hyperlink{f_04.xhtmlux5cux23table-I-3}{\textbf{Table I.3}}
  \item
    \protect\hyperlink{f_04.xhtmlux5cux23table-I-4}{\textbf{Table I.4}}
  \item
    \protect\hyperlink{f_04.xhtmlux5cux23table-I-5}{\textbf{Table I.5}}
  \item
    \protect\hyperlink{f_04.xhtmlux5cux23table-I-6}{\textbf{Table I.6}}
  \item
    \protect\hyperlink{f_04.xhtmlux5cux23table-I-7}{\textbf{Table I.7}}
  \item
    \protect\hyperlink{f_04.xhtmlux5cux23table-I-8}{\textbf{Table I.8}}
  \item
    \protect\hyperlink{f_04.xhtmlux5cux23table-I-9}{\textbf{Table I.9}}
  \item
    \protect\hyperlink{f_04.xhtmlux5cux23table-I-10}{\textbf{Table
    I.10}}
  \item
    \protect\hyperlink{f_04.xhtmlux5cux23table-I-11}{\textbf{Table
    I.11}}
  \item
    \protect\hyperlink{f_04.xhtmlux5cux23table-I-12}{\textbf{Table
    I.12}}
  \item
    \protect\hyperlink{f_04.xhtmlux5cux23table-I-13}{\textbf{Table
    I.13}}
  \item
    \protect\hyperlink{f_04.xhtmlux5cux23table-I-14}{\textbf{Table
    I.14}}
  \item
    \protect\hyperlink{f_04.xhtmlux5cux23table-I-15}{\textbf{Table
    I.15}}
  \item
    \protect\hyperlink{f_04.xhtmlux5cux23table-I-16}{\textbf{Table
    I.16}}
  \item
    \protect\hyperlink{f_04.xhtmlux5cux23table-I-17}{\textbf{Table
    I.17}}
  \end{enumerate}
\item
  \protect\hyperlink{c02.xhtml}{Chapter 2}

  \begin{enumerate}
  \tightlist
  \item
    \protect\hyperlink{c02.xhtmlux5cux23table02-1}{\textbf{Table 2.1}}
  \item
    \protect\hyperlink{c02.xhtmlux5cux23table02-2}{\textbf{Table 2.2}}
  \item
    \protect\hyperlink{c02.xhtmlux5cux23table02-3}{\textbf{Table 2.3}}
  \end{enumerate}
\item
  \protect\hyperlink{c03.xhtml}{Chapter 3}

  \begin{enumerate}
  \tightlist
  \item
    \protect\hyperlink{c03.xhtmlux5cux23table03-1}{\textbf{Table 3.1}}
  \item
    \protect\hyperlink{c03.xhtmlux5cux23table03-2}{\textbf{Table 3.2}}
  \item
    \protect\hyperlink{c03.xhtmlux5cux23table03-3}{\textbf{Table 3.3}}
  \item
    \protect\hyperlink{c03.xhtmlux5cux23table03-4}{\textbf{Table 3.4}}
  \item
    \protect\hyperlink{c03.xhtmlux5cux23table03-5}{\textbf{Table 3.5}}
  \end{enumerate}
\item
  \protect\hyperlink{c04.xhtml}{Chapter 4}

  \begin{enumerate}
  \tightlist
  \item
    \protect\hyperlink{c04.xhtmlux5cux23table04-1}{\textbf{Table 4.1}}
  \item
    \protect\hyperlink{c04.xhtmlux5cux23table04-2}{\textbf{Table 4.2}}
  \item
    \protect\hyperlink{c04.xhtmlux5cux23table04-3}{\textbf{Table 4.3}}
  \end{enumerate}
\item
  \protect\hyperlink{c05.xhtml}{Chapter 5}

  \begin{enumerate}
  \tightlist
  \item
    \protect\hyperlink{c05.xhtmlux5cux23table05-1}{\textbf{Table 5.1}}
  \end{enumerate}
\item
  \protect\hyperlink{c06.xhtml}{Chapter 6}

  \begin{enumerate}
  \tightlist
  \item
    \protect\hyperlink{c06.xhtmlux5cux23table6-1}{\textbf{Table 6.1}}
  \item
    \protect\hyperlink{c06.xhtmlux5cux23table6-2}{\textbf{Table 6.2}}
  \item
    \protect\hyperlink{c06.xhtmlux5cux23table6-3}{\textbf{Table 6.3}}
  \end{enumerate}
\item
  \protect\hyperlink{c07.xhtml}{Chapter 7}

  \begin{enumerate}
  \tightlist
  \item
    \protect\hyperlink{c07.xhtmlux5cux23table7-1}{\textbf{Table 7.1}}
  \item
    \protect\hyperlink{c07.xhtmlux5cux23table7-2}{\textbf{Table 7.2}}
  \item
    \protect\hyperlink{c07.xhtmlux5cux23table7-3}{\textbf{Table 7.3}}
  \end{enumerate}
\item
  \protect\hyperlink{c08.xhtml}{Chapter 8}

  \begin{enumerate}
  \tightlist
  \item
    \protect\hyperlink{c08.xhtmlux5cux23table8-1}{\textbf{Table 8.1}}
  \item
    \protect\hyperlink{c08.xhtmlux5cux23table8-2}{\textbf{Table 8.2}}
  \item
    \protect\hyperlink{c08.xhtmlux5cux23table8-3}{\textbf{Table 8.3}}
  \end{enumerate}
\item
  \protect\hyperlink{c09.xhtml}{Chapter 9}

  \begin{enumerate}
  \tightlist
  \item
    \protect\hyperlink{c09.xhtmlux5cux23table9-1}{\textbf{Table 9.1}}
  \item
    \protect\hyperlink{c09.xhtmlux5cux23table9-2}{\textbf{Table 9.2}}
  \end{enumerate}
\item
  \protect\hyperlink{c12.xhtml}{Chapter 12}

  \begin{enumerate}
  \tightlist
  \item
    \protect\hyperlink{c12.xhtmlux5cux23table12-1}{\textbf{Table 12.1}}
  \end{enumerate}
\item
  \protect\hyperlink{c13.xhtml}{Chapter 13}

  \begin{enumerate}
  \tightlist
  \item
    \protect\hyperlink{c13.xhtmlux5cux23table13-1}{\textbf{Table 13.1}}
  \item
    \protect\hyperlink{c13.xhtmlux5cux23table13-2}{\textbf{Table 13.2}}
  \item
    \protect\hyperlink{c13.xhtmlux5cux23table13-3}{\textbf{Table 13.3}}
  \end{enumerate}
\item
  \protect\hyperlink{c14.xhtml}{Chapter 14}

  \begin{enumerate}
  \tightlist
  \item
    \protect\hyperlink{c14.xhtmlux5cux23table14-1}{\textbf{Table 14.1}}
  \item
    \protect\hyperlink{c14.xhtmlux5cux23table14-2}{\textbf{Table 14.2}}
  \end{enumerate}
\item
  \protect\hyperlink{c15.xhtml}{Chapter 15}

  \begin{enumerate}
  \tightlist
  \item
    \protect\hyperlink{c15.xhtmlux5cux23table15-1}{\textbf{Table 15.1}}
  \end{enumerate}
\item
  \protect\hyperlink{c17.xhtml}{Chapter 17}

  \begin{enumerate}
  \tightlist
  \item
    \protect\hyperlink{c17.xhtmlux5cux23table17-1}{\textbf{Table 17.1}}
  \item
    \protect\hyperlink{c17.xhtmlux5cux23table17-2}{\textbf{Table 17.2}}
  \end{enumerate}
\item
  \protect\hyperlink{c18.xhtml}{Chapter 18}

  \begin{enumerate}
  \tightlist
  \item
    \protect\hyperlink{c18.xhtmlux5cux23table18-1}{\textbf{Table 18.1}}
  \item
    \protect\hyperlink{c18.xhtmlux5cux23table18-2}{\textbf{Table 18.2}}
  \item
    \protect\hyperlink{c18.xhtmlux5cux23table18-3}{\textbf{Table 18.3}}
  \end{enumerate}
\item
  \protect\hyperlink{c19.xhtml}{Chapter 19}

  \begin{enumerate}
  \tightlist
  \item
    \protect\hyperlink{c19.xhtmlux5cux23table19-1}{\textbf{Table 19.1}}
  \end{enumerate}
\item
  \protect\hyperlink{c21.xhtml}{Chapter 21}

  \begin{enumerate}
  \tightlist
  \item
    \protect\hyperlink{c21.xhtmlux5cux23table21-1}{\textbf{Table 21.1}}
  \end{enumerate}
\end{enumerate}

\subsection{\texorpdfstring{\textbf{List of
Illustrations}}{List of Illustrations}}

\begin{enumerate}
\tightlist
\item
  \protect\hyperlink{f_04.xhtml}{Introduction}

  \begin{enumerate}
  \tightlist
  \item
    \protect\hyperlink{f_04.xhtmlux5cux23figure01-1}{\textbf{Figure I.1}
    The Cisco certification path.}
  \end{enumerate}
\item
  \protect\hyperlink{c01.xhtml}{Chapter 1}

  \begin{enumerate}
  \tightlist
  \item
    \protect\hyperlink{c01.xhtmlux5cux23figure01-1}{\textbf{Figure 1.1}
    A very basic network}
  \item
    \protect\hyperlink{c01.xhtmlux5cux23figure01-2}{\textbf{Figure 1.2}
    A switch can break up collision domains.}
  \item
    \protect\hyperlink{c01.xhtmlux5cux23figure01-3}{\textbf{Figure 1.3}
    Routers create an internetwork.}
  \item
    \protect\hyperlink{c01.xhtmlux5cux23figure01-4}{\textbf{Figure 1.4}
    Internetworking devices}
  \item
    \protect\hyperlink{c01.xhtmlux5cux23figure01-5}{\textbf{Figure 1.5}
    Switched networks creating an internetwork}
  \item
    \protect\hyperlink{c01.xhtmlux5cux23figure01-6}{\textbf{Figure 1.6}
    Other devices typically found in our internetworks today.}
  \item
    \protect\hyperlink{c01.xhtmlux5cux23figure01-7}{\textbf{Figure 1.7}
    The upper layers}
  \item
    \protect\hyperlink{c01.xhtmlux5cux23figure01-8}{\textbf{Figure 1.8}
    The lower layers}
  \item
    \protect\hyperlink{c01.xhtmlux5cux23figure01-9}{\textbf{Figure 1.9}
    OSI layer functions}
  \item
    \protect\hyperlink{c01.xhtmlux5cux23figure01-10}{\textbf{Figure
    1.10} Establishing a connection-oriented session}
  \item
    \protect\hyperlink{c01.xhtmlux5cux23figure01-11}{\textbf{Figure
    1.11} Transmitting segments with flow control}
  \item
    \protect\hyperlink{c01.xhtmlux5cux23figure01-12}{\textbf{Figure
    1.12} Windowing}
  \item
    \protect\hyperlink{c01.xhtmlux5cux23figure01-13}{\textbf{Figure
    1.13} Transport layer reliable delivery}
  \item
    \protect\hyperlink{c01.xhtmlux5cux23figure01-14}{\textbf{Figure
    1.14} Routing table used in a router}
  \item
    \protect\hyperlink{c01.xhtmlux5cux23figure01-15}{\textbf{Figure
    1.15} A router in an internetwork. Each router LAN interface is a
    broadcast domain. Routers break up broadcast domains by default and
    provide WAN services.}
  \item
    \protect\hyperlink{c01.xhtmlux5cux23figure01-16}{\textbf{Figure
    1.16} Data Link layer}
  \item
    \protect\hyperlink{c01.xhtmlux5cux23figure01-17}{\textbf{Figure
    1.17} A switch in an internetwork}
  \item
    \protect\hyperlink{c01.xhtmlux5cux23figure01-18}{\textbf{Figure
    1.18} A hub in a network}
  \item
    \protect\hyperlink{c01.xhtmlux5cux23figure01-19}{\textbf{Figure
    1.19} Physical vs. Logical Topolgies}
  \end{enumerate}
\item
  \protect\hyperlink{c02.xhtml}{Chapter 2}

  \begin{enumerate}
  \tightlist
  \item
    \protect\hyperlink{c02.xhtmlux5cux23figure02-1}{\textbf{Figure 2.1}
    Legacy collision domain design}
  \item
    \protect\hyperlink{c02.xhtmlux5cux23figure02-2}{\textbf{Figure 2.2}
    A typical network you'd see today}
  \item
    \protect\hyperlink{c02.xhtmlux5cux23figure02-3}{\textbf{Figure 2.3}
    A router creates broadcast domain boundaries.}
  \item
    \protect\hyperlink{c02.xhtmlux5cux23figure02-4}{\textbf{Figure 2.4}
    CSMA/CD}
  \item
    \protect\hyperlink{c02.xhtmlux5cux23figure02-5}{\textbf{Figure 2.5}
    Half-duplex example}
  \item
    \protect\hyperlink{c02.xhtmlux5cux23figure02-6}{\textbf{Figure 2.6}
    Full-duplex example}
  \item
    \protect\hyperlink{c02.xhtmlux5cux23figure02-7}{\textbf{Figure 2.7}
    Ethernet addressing using MAC addresses}
  \item
    \protect\hyperlink{c02.xhtmlux5cux23figure02-8}{\textbf{Figure 2.8}
    Typical Ethernet frame format}
  \item
    \protect\hyperlink{c02.xhtmlux5cux23figure02-9}{\textbf{Figure 2.9}
    Category 5 Enhanced UTP cable}
  \item
    \protect\hyperlink{c02.xhtmlux5cux23figure02-10}{\textbf{Figure
    2.10} Straight-through Ethernet cable}
  \item
    \protect\hyperlink{c02.xhtmlux5cux23figure02-11}{\textbf{Figure
    2.11} Crossover Ethernet cable}
  \item
    \protect\hyperlink{c02.xhtmlux5cux23figure02-12}{\textbf{Figure
    2.12} Typical uses for straight-through and cross-over Ethernet
    cables}
  \item
    \protect\hyperlink{c02.xhtmlux5cux23figure02-13}{\textbf{Figure
    2.13} UTP Gigabit crossover Ethernet cable}
  \item
    \protect\hyperlink{c02.xhtmlux5cux23figure02-14}{\textbf{Figure
    2.14} Rolled Ethernet cable}
  \item
    \protect\hyperlink{c02.xhtmlux5cux23figure02-15}{\textbf{Figure
    2.15} Configuring your console emulation program}
  \item
    \protect\hyperlink{c02.xhtmlux5cux23figure02-16}{\textbf{Figure
    2.16} A Cisco 2960 console connections}
  \item
    \protect\hyperlink{c02.xhtmlux5cux23figure02-17}{\textbf{Figure
    2.17} RJ45 UTP cable question \#1}
  \item
    \protect\hyperlink{c02.xhtmlux5cux23figure02-18}{\textbf{Figure
    2.18} RJ45 UTP cable question \#2}
  \item
    \protect\hyperlink{c02.xhtmlux5cux23figure02-19}{\textbf{Figure
    2.19} Typical fiber cable.}
  \item
    \protect\hyperlink{c02.xhtmlux5cux23figure02-20}{\textbf{Figure
    2.20} Multimode and single-mode fibers}
  \item
    \protect\hyperlink{c02.xhtmlux5cux23figure02-21}{\textbf{Figure
    2.21} Data encapsulation}
  \item
    \protect\hyperlink{c02.xhtmlux5cux23figure02-22}{\textbf{Figure
    2.22} PDU and layer addressing}
  \item
    \protect\hyperlink{c02.xhtmlux5cux23figure02-23}{\textbf{Figure
    2.23} Port numbers at the Transport layer}
  \item
    \protect\hyperlink{c02.xhtmlux5cux23figure02-24}{\textbf{Figure
    2.24} The Cisco hierarchical model}
  \end{enumerate}
\item
  \protect\hyperlink{c03.xhtml}{Chapter 3}

  \begin{enumerate}
  \tightlist
  \item
    \protect\hyperlink{c03.xhtmlux5cux23figure03-1}{\textbf{Figure 3.1}
    The DoD and OSI models}
  \item
    \protect\hyperlink{c03.xhtmlux5cux23figure03-2}{\textbf{Figure 3.2}
    The TCP/IP protocol suite}
  \item
    \protect\hyperlink{c03.xhtmlux5cux23figure03-3}{\textbf{Figure 3.3}
    Telnet}
  \item
    \protect\hyperlink{c03.xhtmlux5cux23figure03-4}{\textbf{Figure 3.4}
    Secure Shell}
  \item
    \protect\hyperlink{c03.xhtmlux5cux23figure03-5}{\textbf{Figure 3.5}
    FTP}
  \item
    \protect\hyperlink{c03.xhtmlux5cux23figure03-6}{\textbf{Figure 3.6}
    TFTP}
  \item
    \protect\hyperlink{c03.xhtmlux5cux23figure03-7}{\textbf{Figure 3.7}
    SNMP}
  \item
    \protect\hyperlink{c03.xhtmlux5cux23figure03-8}{\textbf{Figure 3.8}
    HTTP}
  \item
    \protect\hyperlink{c03.xhtmlux5cux23figure03-9}{\textbf{Figure 3.9}
    NTP}
  \item
    \protect\hyperlink{c03.xhtmlux5cux23figure03-10}{\textbf{Figure
    3.10} DNS}
  \item
    \protect\hyperlink{c03.xhtmlux5cux23figure03-11}{\textbf{Figure
    3.11} DHCP client four-step process}
  \item
    \protect\hyperlink{c03.xhtmlux5cux23figure03-12}{\textbf{Figure
    3.12} TCP segment format}
  \item
    \protect\hyperlink{c03.xhtmlux5cux23figure03-13}{\textbf{Figure
    3.13} UDP segment}
  \item
    \protect\hyperlink{c03.xhtmlux5cux23figure03-14}{\textbf{Figure
    3.14} Port numbers for TCP and UDP}
  \item
    \protect\hyperlink{c03.xhtmlux5cux23figure03-15}{\textbf{Figure
    3.15} IP header}
  \item
    \protect\hyperlink{c03.xhtmlux5cux23figure03-16}{\textbf{Figure
    3.16} The Protocol field in an IP header}
  \item
    \protect\hyperlink{c03.xhtmlux5cux23figure03-17}{\textbf{Figure
    3.17} ICMP error message is sent to the sending host from the remote
    router.}
  \item
    \protect\hyperlink{c03.xhtmlux5cux23figure03-18}{\textbf{Figure
    3.18} ICMP in action}
  \item
    \protect\hyperlink{c03.xhtmlux5cux23figure03-19}{\textbf{Figure
    3.19} Local ARP broadcast}
  \item
    \protect\hyperlink{c03.xhtmlux5cux23figure03-20}{\textbf{Figure
    3.20} Summary of the three classes of networks}
  \item
    \protect\hyperlink{c03.xhtmlux5cux23figure03-21}{\textbf{Figure
    3.21} Local layer 2 broadcasts}
  \item
    \protect\hyperlink{c03.xhtmlux5cux23figure03-22}{\textbf{Figure
    3.22} Layer 3 broadcasts}
  \item
    \protect\hyperlink{c03.xhtmlux5cux23figure03-23}{\textbf{Figure
    3.23} Unicast address}
  \item
    \protect\hyperlink{c03.xhtmlux5cux23figure03-24}{\textbf{Figure
    3.24} EIGRP multicast example}
  \end{enumerate}
\item
  \protect\hyperlink{c04.xhtml}{Chapter 4}

  \begin{enumerate}
  \tightlist
  \item
    \protect\hyperlink{c04.xhtmlux5cux23figure04-1}{\textbf{Figure 4.1}
    One network}
  \item
    \protect\hyperlink{c04.xhtmlux5cux23figure04-2}{\textbf{Figure 4.2}
    Multiple networks connected together}
  \item
    \protect\hyperlink{c04.xhtmlux5cux23figure04-3}{\textbf{Figure 4.3}
    Implementing a Class C /25 logical network}
  \item
    \protect\hyperlink{c04.xhtmlux5cux23figure04-4}{\textbf{Figure 4.4}
    Implementing a class C /26 (with three networks)}
  \item
    \protect\hyperlink{c04.xhtmlux5cux23figure04-5}{\textbf{Figure 4.5}
    Implementing a Class C /27 logical network}
  \end{enumerate}
\item
  \protect\hyperlink{c05.xhtml}{Chapter 5}

  \begin{enumerate}
  \tightlist
  \item
    \protect\hyperlink{c05.xhtmlux5cux23figure05-1}{\textbf{Figure 5.1}
    Typical classful network}
  \item
    \protect\hyperlink{c05.xhtmlux5cux23figure05-2}{\textbf{Figure 5.2}
    Classless network design}
  \item
    \protect\hyperlink{c05.xhtmlux5cux23figure05-3}{\textbf{Figure 5.3}
    The VLSM table}
  \item
    \protect\hyperlink{c05.xhtmlux5cux23figure05-4}{\textbf{Figure 5.4}
    VLSM network example 1}
  \item
    \protect\hyperlink{c05.xhtmlux5cux23figure05-5}{\textbf{Figure 5.5}
    VLSM table example 1}
  \item
    \protect\hyperlink{c05.xhtmlux5cux23figure05-6}{\textbf{Figure 5.6}
    VLSM network example 2}
  \item
    \protect\hyperlink{c05.xhtmlux5cux23figure05-7}{\textbf{Figure 5.7}
    VLSM table example 2}
  \item
    \protect\hyperlink{c05.xhtmlux5cux23figure05-8}{\textbf{Figure 5.8}
    VLSM design example 1}
  \item
    \protect\hyperlink{c05.xhtmlux5cux23figure05-9}{\textbf{Figure 5.9}
    Solution to VLSM design example 1}
  \item
    \protect\hyperlink{c05.xhtmlux5cux23figure05-10}{\textbf{Figure
    5.10} VLSM design example 2}
  \item
    \protect\hyperlink{c05.xhtmlux5cux23figure05-11}{\textbf{Figure
    5.11} Solution to VLSM design example 2}
  \item
    \protect\hyperlink{c05.xhtmlux5cux23figure05-12}{\textbf{Figure
    5.12} Summary address used in an internetwork}
  \item
    \protect\hyperlink{c05.xhtmlux5cux23figure05-13}{\textbf{Figure
    5.13} Summarization example 4}
  \item
    \protect\hyperlink{c05.xhtmlux5cux23figure05-14}{\textbf{Figure
    5.14} Summarization example 5}
  \item
    \protect\hyperlink{c05.xhtmlux5cux23figure05-15}{\textbf{Figure
    5.15} Basic IP troubleshooting}
  \item
    \protect\hyperlink{c05.xhtmlux5cux23figure05-16}{\textbf{Figure
    5.16} IP address problem 1}
  \item
    \protect\hyperlink{c05.xhtmlux5cux23figure05-17}{\textbf{Figure
    5.17} IP address problem 2}
  \item
    \protect\hyperlink{c05.xhtmlux5cux23figure05-18}{\textbf{Figure
    5.18} Find the valid host \#1}
  \item
    \protect\hyperlink{c05.xhtmlux5cux23figure05-19}{\textbf{Figure
    5.19} Find the valid host \#2}
  \item
    \protect\hyperlink{c05.xhtmlux5cux23figure05-20}{\textbf{Figure
    5.20} Find the valid host address \#3}
  \item
    \protect\hyperlink{c05.xhtmlux5cux23figure05-21}{\textbf{Figure
    5.21} Find the valid subnet mask}
  \end{enumerate}
\item
  \protect\hyperlink{c06.xhtml}{Chapter 6}

  \begin{enumerate}
  \tightlist
  \item
    \protect\hyperlink{c06.xhtmlux5cux23figure6-1}{\textbf{Figure 6.1} A
    Cisco 2960 switch}
  \item
    \protect\hyperlink{c06.xhtmlux5cux23figure6-2}{\textbf{Figure 6.2} A
    new Cisco 1900 router}
  \item
    \protect\hyperlink{c06.xhtmlux5cux23figure6-3}{\textbf{Figure 6.3} A
    typical WAN connection. Clocking is typically provided by a DCE
    network to routers. In nonproduction environments, a DCE network is
    not always present.}
  \item
    \protect\hyperlink{c06.xhtmlux5cux23figure6-4}{\textbf{Figure 6.4}
    Providing clocking on a nonproduction network}
  \item
    \protect\hyperlink{c06.xhtmlux5cux23figure6-5}{\textbf{Figure 6.5}
    Where do you configure clocking? Use the \texttt{show\ controllers}
    command on each router's serial interface to find out.}
  \item
    \protect\hyperlink{c06.xhtmlux5cux23figure6-6}{\textbf{Figure 6.6}
    By looking at R1, the \texttt{show\ controllers} command reveals
    that R1 and R2 can't communicate.}
  \end{enumerate}
\item
  \protect\hyperlink{c07.xhtml}{Chapter 7}

  \begin{enumerate}
  \tightlist
  \item
    \protect\hyperlink{c07.xhtmlux5cux23figure7-1}{\textbf{Figure 7.1}
    Router bootup process}
  \item
    \protect\hyperlink{c07.xhtmlux5cux23figure7-2}{\textbf{Figure 7.2}
    DHCP configuration example on a switch}
  \item
    \protect\hyperlink{c07.xhtmlux5cux23figure7-3}{\textbf{Figure 7.3}
    Configuring a DHCP relay}
  \item
    \protect\hyperlink{c07.xhtmlux5cux23figure7-4}{\textbf{Figure 7.4}
    Messages sent to a syslog server}
  \item
    \protect\hyperlink{c07.xhtmlux5cux23figure7-5}{\textbf{Figure 7.5}
    Synchronizing time information}
  \item
    \protect\hyperlink{c07.xhtmlux5cux23figure7-6}{\textbf{Figure 7.6}
    Cisco Discovery Protocol}
  \item
    \protect\hyperlink{c07.xhtmlux5cux23figure7-7}{\textbf{Figure 7.7}
    Documenting a network topology using CDP}
  \item
    \protect\hyperlink{c07.xhtmlux5cux23figure7-8}{\textbf{Figure 7.8}
    Network topology documented}
  \end{enumerate}
\item
  \protect\hyperlink{c08.xhtml}{Chapter 8}

  \begin{enumerate}
  \tightlist
  \item
    \protect\hyperlink{c08.xhtmlux5cux23figure8-1}{\textbf{Figure 8.1}
    Copying an IOS from a router to a TFTP host}
  \end{enumerate}
\item
  \protect\hyperlink{c09.xhtml}{Chapter 9}

  \begin{enumerate}
  \tightlist
  \item
    \protect\hyperlink{c09.xhtmlux5cux23figure9-1}{\textbf{Figure 9.1} A
    simple routing example}
  \item
    \protect\hyperlink{c09.xhtmlux5cux23figure9-2}{\textbf{Figure 9.2}
    IP routing example using two hosts and one router}
  \item
    \protect\hyperlink{c09.xhtmlux5cux23figure9-3}{\textbf{Figure 9.3}
    Frame used from Host A to the Lab\_A router when Host B is pinged}
  \item
    \protect\hyperlink{c09.xhtmlux5cux23figure9-4}{\textbf{Figure 9.4}
    IP routing example 1}
  \item
    \protect\hyperlink{c09.xhtmlux5cux23figure9-5}{\textbf{Figure 9.5}
    IP routing example 2}
  \item
    \protect\hyperlink{c09.xhtmlux5cux23figure9-6}{\textbf{Figure 9.6}
    Basic IP routing using MAC and IP addresses}
  \item
    \protect\hyperlink{c09.xhtmlux5cux23figure9-7}{\textbf{Figure 9.7}
    Testing basic routing knowledge}
  \item
    \protect\hyperlink{c09.xhtmlux5cux23figure9-8}{\textbf{Figure 9.8}
    Configuring IP routing}
  \item
    \protect\hyperlink{c09.xhtmlux5cux23figure9-9}{\textbf{Figure 9.9}
    Our internetwork}
  \end{enumerate}
\item
  \protect\hyperlink{c10.xhtml}{Chapter 10}

  \begin{enumerate}
  \tightlist
  \item
    \protect\hyperlink{c10.xhtmlux5cux23figure10-1}{\textbf{Figure 10.1}
    Empty forward/filter table on a switch}
  \item
    \protect\hyperlink{c10.xhtmlux5cux23figure10-2}{\textbf{Figure 10.2}
    How switches learn hosts' locations}
  \item
    \protect\hyperlink{c10.xhtmlux5cux23figure10-3}{\textbf{Figure 10.3}
    Forward/filter table}
  \item
    \protect\hyperlink{c10.xhtmlux5cux23figure10-4}{\textbf{Figure 10.4}
    Forward/filter table answer}
  \item
    \protect\hyperlink{c10.xhtmlux5cux23figure10-5}{\textbf{Figure 10.5}
    ``Port security'' on a switch port restricts port access by MAC
    address.}
  \item
    \protect\hyperlink{c10.xhtmlux5cux23figure10-6}{\textbf{Figure 10.6}
    Protecting a PC in a lobby}
  \item
    \protect\hyperlink{c10.xhtmlux5cux23figure10-7}{\textbf{Figure 10.7}
    Broadcast storm}
  \item
    \protect\hyperlink{c10.xhtmlux5cux23figure10-8}{\textbf{Figure 10.8}
    Multiple frame copies}
  \item
    \protect\hyperlink{c10.xhtmlux5cux23figure10-9}{\textbf{Figure 10.9}
    A Cisco Catalyst switch}
  \item
    \protect\hyperlink{c10.xhtmlux5cux23figure10-10}{\textbf{Figure
    10.10} Our switched network}
  \end{enumerate}
\item
  \protect\hyperlink{c11.xhtml}{Chapter 11}

  \begin{enumerate}
  \tightlist
  \item
    \protect\hyperlink{c11.xhtmlux5cux23figure11-1}{\textbf{Figure 11.1}
    Flat network structure}
  \item
    \protect\hyperlink{c11.xhtmlux5cux23figure11-2}{\textbf{Figure 11.2}
    The benefit of a switched network}
  \item
    \protect\hyperlink{c11.xhtmlux5cux23figure11-3}{\textbf{Figure 11.3}
    One switch, one LAN: Before VLANs, there were no separations between
    hosts.}
  \item
    \protect\hyperlink{c11.xhtmlux5cux23figure11-4}{\textbf{Figure 11.4}
    One switch, two virtual LANs (\emph{logical} separation between
    hosts): Still physically one switch, but this switch acts as many
    separate devices.}
  \item
    \protect\hyperlink{c11.xhtmlux5cux23figure11-5}{\textbf{Figure 11.5}
    Access ports}
  \item
    \protect\hyperlink{c11.xhtmlux5cux23figure11-6}{\textbf{Figure 11.6}
    VLANs can span across multiple switches by using trunk links, which
    carry traffic for multiple VLANs.}
  \item
    \protect\hyperlink{c11.xhtmlux5cux23figure11-7}{\textbf{Figure 11.7}
    IEEE 802.1q encapsulation with and without the 802.1q tag}
  \item
    \protect\hyperlink{c11.xhtmlux5cux23figure11-8}{\textbf{Figure 11.8}
    Router connecting three VLANs together for inter-VLAN communication,
    one router interface for each VLAN}
  \item
    \protect\hyperlink{c11.xhtmlux5cux23figure11-9}{\textbf{Figure 11.9}
    Router on a stick: single router interface connecting all three
    VLANs together for inter-VLAN communication}
  \item
    \protect\hyperlink{c11.xhtmlux5cux23figure11-10}{\textbf{Figure
    11.10} A router creates logical interfaces.}
  \item
    \protect\hyperlink{c11.xhtmlux5cux23figure11-11}{\textbf{Figure
    11.11} With IVR, routing runs on the backplane of the switch, and it
    appears to the hosts that a router is present.}
  \item
    \protect\hyperlink{c11.xhtmlux5cux23figure11-12}{\textbf{Figure
    11.12} Configuring inter-VLAN example 1}
  \item
    \protect\hyperlink{c11.xhtmlux5cux23figure11-13}{\textbf{Figure
    11.13} Inter-VLAN example 2}
  \item
    \protect\hyperlink{c11.xhtmlux5cux23figure11-14}{\textbf{Figure
    11.14} Inter-VLAN example 3}
  \item
    \protect\hyperlink{c11.xhtmlux5cux23figure11-15}{\textbf{Figure
    11.15} Inter-VLAN example 4}
  \item
    \protect\hyperlink{c11.xhtmlux5cux23figure11-16}{\textbf{Figure
    11.16} Inter-VLAN routing with a multilayer switch}
  \end{enumerate}
\item
  \protect\hyperlink{c12.xhtml}{Chapter 12}

  \begin{enumerate}
  \tightlist
  \item
    \protect\hyperlink{c12.xhtmlux5cux23figure12-1}{\textbf{Figure 12.1}
    A typical secured network}
  \item
    \protect\hyperlink{c12.xhtmlux5cux23figure12-2}{\textbf{Figure 12.2}
    IP access list example with three LANs and a WAN connection}
  \item
    \protect\hyperlink{c12.xhtmlux5cux23figure12-3}{\textbf{Figure 12.3}
    IP standard access list example 2}
  \item
    \protect\hyperlink{c12.xhtmlux5cux23figure12-4}{\textbf{Figure 12.4}
    IP standard access list example 3}
  \item
    \protect\hyperlink{c12.xhtmlux5cux23figure12-5}{\textbf{Figure 12.5}
    Extended ACL example 1}
  \item
    \protect\hyperlink{c12.xhtmlux5cux23figure12-6}{\textbf{Figure 12.6}
    Extended ACL example 3}
  \end{enumerate}
\item
  \protect\hyperlink{c13.xhtml}{Chapter 13}

  \begin{enumerate}
  \tightlist
  \item
    \protect\hyperlink{c13.xhtmlux5cux23figure13-1}{\textbf{Figure 13.1}
    Where to configure NAT}
  \item
    \protect\hyperlink{c13.xhtmlux5cux23figure13-2}{\textbf{Figure 13.2}
    Basic NAT translation}
  \item
    \protect\hyperlink{c13.xhtmlux5cux23figure13-3}{\textbf{Figure 13.3}
    NAT overloading example (PAT)}
  \item
    \protect\hyperlink{c13.xhtmlux5cux23figure13-4}{\textbf{Figure 13.4}
    NAT example}
  \item
    \protect\hyperlink{c13.xhtmlux5cux23figure13-5}{\textbf{Figure 13.5}
    Another NAT example}
  \item
    \protect\hyperlink{c13.xhtmlux5cux23figure13-6}{\textbf{Figure 13.6}
    Last NAT example}
  \end{enumerate}
\item
  \protect\hyperlink{c14.xhtml}{Chapter 14}

  \begin{enumerate}
  \tightlist
  \item
    \protect\hyperlink{c14.xhtmlux5cux23figure14-1}{\textbf{Figure 14.1}
    IPv6 address example}
  \item
    \protect\hyperlink{c14.xhtmlux5cux23figure14-2}{\textbf{Figure 14.2}
    IPv6 global unicast addresses}
  \item
    \protect\hyperlink{c14.xhtmlux5cux23figure14-3}{\textbf{Figure 14.3}
    IPv6 link local FE80::/10: The first 10 bits define the address
    type.}
  \item
    \protect\hyperlink{c14.xhtmlux5cux23figure14-4}{\textbf{Figure 14.4}
    EUI-64 interface ID assignment}
  \item
    \protect\hyperlink{c14.xhtmlux5cux23figure14-5}{\textbf{Figure 14.5}
    Two steps to IPv6 autoconfiguration}
  \item
    \protect\hyperlink{c14.xhtmlux5cux23figure14-6}{\textbf{Figure 14.6}
    IPv6 autoconfiguration example}
  \item
    \protect\hyperlink{c14.xhtmlux5cux23figure14-7}{\textbf{Figure 14.7}
    IPv6 header}
  \item
    \protect\hyperlink{c14.xhtmlux5cux23figure14-8}{\textbf{Figure 14.8}
    ICMPv6}
  \item
    \protect\hyperlink{c14.xhtmlux5cux23figure14-9}{\textbf{Figure 14.9}
    Router solicitation (RS) and router advertisement (RA)}
  \item
    \protect\hyperlink{c14.xhtmlux5cux23figure14-10}{\textbf{Figure
    14.10} Neighbor solicitation (NS) and neighbor advertisement (NA)}
  \item
    \protect\hyperlink{c14.xhtmlux5cux23figure14-11}{\textbf{Figure
    14.11} Duplicate address detection (DAD)}
  \item
    \protect\hyperlink{c14.xhtmlux5cux23figure14-12}{\textbf{Figure
    14.12} IPv6 static and default routing}
  \item
    \protect\hyperlink{c14.xhtmlux5cux23figure14-13}{\textbf{Figure
    14.13} Our internetwork}
  \end{enumerate}
\item
  \protect\hyperlink{c15.xhtml}{Chapter 15}

  \begin{enumerate}
  \tightlist
  \item
    \protect\hyperlink{c15.xhtmlux5cux23figure15-1}{\textbf{Figure 15.1}
    VTP modes}
  \item
    \protect\hyperlink{c15.xhtmlux5cux23figure15-2}{\textbf{Figure 15.2}
    A switched network with switching loops}
  \item
    \protect\hyperlink{c15.xhtmlux5cux23figure15-3}{\textbf{Figure 15.3}
    A switched network with STP}
  \item
    \protect\hyperlink{c15.xhtmlux5cux23figure15-4}{\textbf{Figure 15.4}
    STP operations}
  \item
    \protect\hyperlink{c15.xhtmlux5cux23figure15-5}{\textbf{Figure 15.5}
    STP operations}
  \item
    \protect\hyperlink{c15.xhtmlux5cux23figure15-6}{\textbf{Figure 15.6}
    STP operations}
  \item
    \protect\hyperlink{c15.xhtmlux5cux23figure15-7}{\textbf{Figure 15.7}
    STP operations}
  \item
    \protect\hyperlink{c15.xhtmlux5cux23figure15-8}{\textbf{Figure 15.8}
    Common STP example}
  \item
    \protect\hyperlink{c15.xhtmlux5cux23figure15-9}{\textbf{Figure 15.9}
    PVST+ provides efficient root bridge selection.}
  \item
    \protect\hyperlink{c15.xhtmlux5cux23figure15-10}{\textbf{Figure
    15.10} PVST+ unique bridge ID}
  \item
    \protect\hyperlink{c15.xhtmlux5cux23figure15-11}{\textbf{Figure
    15.11} RSTP example 1}
  \item
    \protect\hyperlink{c15.xhtmlux5cux23figure15-12}{\textbf{Figure
    15.12} RSTP example 1 answer}
  \item
    \protect\hyperlink{c15.xhtmlux5cux23figure15-13}{\textbf{Figure
    15.13} RSTP example 2}
  \item
    \protect\hyperlink{c15.xhtmlux5cux23figure15-14}{\textbf{Figure
    15.14} RSTP example 2, answer 1}
  \item
    \protect\hyperlink{c15.xhtmlux5cux23figure15-15}{\textbf{Figure
    15.15} RSTP example 2, answer 2}
  \item
    \protect\hyperlink{c15.xhtmlux5cux23figure15-16}{\textbf{Figure
    15.16} Our simple three-switch network}
  \item
    \protect\hyperlink{c15.xhtmlux5cux23figure15-17}{\textbf{Figure
    15.17} STP stopping loops}
  \item
    \protect\hyperlink{c15.xhtmlux5cux23figure15-18}{\textbf{Figure
    15.18} STP failure}
  \item
    \protect\hyperlink{c15.xhtmlux5cux23figure15-19}{\textbf{Figure
    15.19} PortFast}
  \item
    \protect\hyperlink{c15.xhtmlux5cux23figure15-20}{\textbf{Figure
    15.20} Before and after port channels}
  \item
    \protect\hyperlink{c15.xhtmlux5cux23figure15-21}{\textbf{Figure
    15.21} EtherChannel example}
  \end{enumerate}
\item
  \protect\hyperlink{c16.xhtml}{Chapter 16}

  \begin{enumerate}
  \tightlist
  \item
    \protect\hyperlink{c16.xhtmlux5cux23figure16-1}{\textbf{Figure 16.1}
    Mitigating threats at the access layer}
  \item
    \protect\hyperlink{c16.xhtmlux5cux23figure16-2}{\textbf{Figure 16.2}
    DHCP snooping and DAI}
  \item
    \protect\hyperlink{c16.xhtmlux5cux23figure16-3}{\textbf{Figure 16.3}
    Identity-based networking}
  \item
    \protect\hyperlink{c16.xhtmlux5cux23figure16-4}{\textbf{Figure 16.4}
    SNMP GET and TRAP messages}
  \item
    \protect\hyperlink{c16.xhtmlux5cux23figure16-5}{\textbf{Figure 16.5}
    Cisco's MIB OIDs}
  \item
    \protect\hyperlink{c16.xhtmlux5cux23figure16-6}{\textbf{Figure 16.6}
    Default gateway}
  \item
    \protect\hyperlink{c16.xhtmlux5cux23figure16-7}{\textbf{Figure 16.7}
    Proxy ARP}
  \item
    \protect\hyperlink{c16.xhtmlux5cux23figure16-8}{\textbf{Figure 16.8}
    FHRPs use a virtual router with a virtual IP address and virtual MAC
    address.}
  \item
    \protect\hyperlink{c16.xhtmlux5cux23figure16-9}{\textbf{Figure 16.9}
    HSRP active and standby routers}
  \item
    \protect\hyperlink{c16.xhtmlux5cux23figure16-10}{\textbf{Figure
    16.10} Example of HSRP active and standby routers swapping
    interfaces}
  \item
    \protect\hyperlink{c16.xhtmlux5cux23figure16-11}{\textbf{Figure
    16.11} HSRP Hellos}
  \item
    \protect\hyperlink{c16.xhtmlux5cux23figure16-12}{\textbf{Figure
    16.12} Interface tracking setup}
  \item
    \protect\hyperlink{c16.xhtmlux5cux23figure16-13}{\textbf{Figure
    16.13} HSRP configuration and verification}
  \item
    \protect\hyperlink{c16.xhtmlux5cux23figure16-14}{\textbf{Figure
    16.14} HSRP load balancing per VLAN}
  \end{enumerate}
\item
  \protect\hyperlink{c17.xhtml}{Chapter 17}

  \begin{enumerate}
  \tightlist
  \item
    \protect\hyperlink{c17.xhtmlux5cux23figure17-1}{\textbf{Figure 17.1}
    EIGRP neighbor discovery}
  \item
    \protect\hyperlink{c17.xhtmlux5cux23figure17-2}{\textbf{Figure 17.2}
    Advertised distance}
  \item
    \protect\hyperlink{c17.xhtmlux5cux23figure17-3}{\textbf{Figure 17.3}
    Feasible distance}
  \item
    \protect\hyperlink{c17.xhtmlux5cux23figure17-4}{\textbf{Figure 17.4}
    The tables used by EIGRP}
  \item
    \protect\hyperlink{c17.xhtmlux5cux23figure17-5}{\textbf{Figure 17.5}
    Configuring our little internetwork with EIGRP}
  \item
    \protect\hyperlink{c17.xhtmlux5cux23figure17-6}{\textbf{Figure 17.6}
    Discontiguous networks}
  \item
    \protect\hyperlink{c17.xhtmlux5cux23figure17-7}{\textbf{Figure 17.7}
    EIGRP route selection process}
  \item
    \protect\hyperlink{c17.xhtmlux5cux23figure17-8}{\textbf{Figure 17.8}
    Split horizon in action, part 1}
  \item
    \protect\hyperlink{c17.xhtmlux5cux23figure17-9}{\textbf{Figure 17.9}
    Split horizon in action, part 2}
  \item
    \protect\hyperlink{c17.xhtmlux5cux23figure17-10}{\textbf{Figure
    17.10} Troubleshooting scenario}
  \item
    \protect\hyperlink{c17.xhtmlux5cux23figure17-11}{\textbf{Figure
    17.11} Configuring EIGRPv6 on our internetwork}
  \end{enumerate}
\item
  \protect\hyperlink{c18.xhtml}{Chapter 18}

  \begin{enumerate}
  \tightlist
  \item
    \protect\hyperlink{c18.xhtmlux5cux23figure18-1}{\textbf{Figure 18.1}
    OSPF design example. An OSPF hierarchical design minimizes routing
    table entries and keeps the impact of any topology changes contained
    within a specific area.}
  \item
    \protect\hyperlink{c18.xhtmlux5cux23figure18-2}{\textbf{Figure 18.2}
    The Hello protocol}
  \item
    \protect\hyperlink{c18.xhtmlux5cux23figure18-3}{\textbf{Figure 18.3}
    Sample OSPF wildcard configuration}
  \item
    \protect\hyperlink{c18.xhtmlux5cux23figure18-4}{\textbf{Figure 18.4}
    Our new network layout}
  \item
    \protect\hyperlink{c18.xhtmlux5cux23figure18-5}{\textbf{Figure 18.5}
    Adding a non-OSPF network to the LA router}
  \item
    \protect\hyperlink{c18.xhtmlux5cux23figure18-6}{\textbf{Figure 18.6}
    OSPF router ID (RID)}
  \end{enumerate}
\item
  \protect\hyperlink{c19.xhtml}{Chapter 19}

  \begin{enumerate}
  \tightlist
  \item
    \protect\hyperlink{c19.xhtmlux5cux23figure19-1}{\textbf{Figure 19.1}
    OSPF single-area network: All routers flood the network with
    link-state information to all other routers within the same area.}
  \item
    \protect\hyperlink{c19.xhtmlux5cux23figure19-2}{\textbf{Figure 19.2}
    OSPF multi-area network: All routers flood the network only within
    their area.}
  \item
    \protect\hyperlink{c19.xhtmlux5cux23figure19-3}{\textbf{Figure 19.3}
    Router roles: Routers within an area are called internal routers.}
  \item
    \protect\hyperlink{c19.xhtmlux5cux23figure19-4}{\textbf{Figure 19.4}
    Type 1 Link-State Advertisements}
  \item
    \protect\hyperlink{c19.xhtmlux5cux23figure19-5}{\textbf{Figure 19.5}
    Basic LSA types}
  \item
    \protect\hyperlink{c19.xhtmlux5cux23figure19-6}{\textbf{Figure 19.6}
    OSPF neighbor states, part 1}
  \item
    \protect\hyperlink{c19.xhtmlux5cux23figure19-7}{\textbf{Figure 19.7}
    OSPF router neighbor states, part 2}
  \item
    \protect\hyperlink{c19.xhtmlux5cux23figure19-8}{\textbf{Figure 19.8}
    Our internetwork}
  \item
    \protect\hyperlink{c19.xhtmlux5cux23figure19-9}{\textbf{Figure 19.9}
    Our internetwork}
  \item
    \protect\hyperlink{c19.xhtmlux5cux23figure19-10}{\textbf{Figure
    19.10} Our internetwork with dual links}
  \item
    \protect\hyperlink{c19.xhtmlux5cux23figure19-11}{\textbf{Figure
    19.11} Configuring OSPFv3}
  \end{enumerate}
\item
  \protect\hyperlink{c20.xhtml}{Chapter 20}

  \begin{enumerate}
  \tightlist
  \item
    \protect\hyperlink{c20.xhtmlux5cux23figure20-1}{\textbf{Figure 20.1}
    Troubleshooting scenario}
  \item
    \protect\hyperlink{c20.xhtmlux5cux23figure20-2}{\textbf{Figure 20.2}
    Using SPAN for troubleshooting}
  \item
    \protect\hyperlink{c20.xhtmlux5cux23figure20-3}{\textbf{Figure 20.3}
    Extended ACLs}
  \item
    \protect\hyperlink{c20.xhtmlux5cux23figure20-4}{\textbf{Figure 20.4}
    IPv6 troubleshooting scenario}
  \item
    \protect\hyperlink{c20.xhtmlux5cux23figure20-5}{\textbf{Figure 20.5}
    Router solicitation (RS) and router advertisement (RA)}
  \item
    \protect\hyperlink{c20.xhtmlux5cux23figure20-6}{\textbf{Figure 20.6}
    Neighbor solicitation (NS) and neighbor advertisement (NA)}
  \item
    \protect\hyperlink{c20.xhtmlux5cux23figure20-7}{\textbf{Figure 20.7}
    VLAN connectivity}
  \end{enumerate}
\item
  \protect\hyperlink{c21.xhtml}{Chapter 21}

  \begin{enumerate}
  \tightlist
  \item
    \protect\hyperlink{c21.xhtmlux5cux23figure21-1}{\textbf{Figure 21.1}
    Hub-and-spoke}
  \item
    \protect\hyperlink{c21.xhtmlux5cux23figure21-2}{\textbf{Figure 21.2}
    Fully meshed topology}
  \item
    \protect\hyperlink{c21.xhtmlux5cux23figure21-3}{\textbf{Figure 21.3}
    Partially meshed topology}
  \item
    \protect\hyperlink{c21.xhtmlux5cux23figure21-4}{\textbf{Figure 21.4}
    WAN terms}
  \item
    \protect\hyperlink{c21.xhtmlux5cux23figure21-5}{\textbf{Figure 21.5}
    WAN connection types}
  \item
    \protect\hyperlink{c21.xhtmlux5cux23figure21-6}{\textbf{Figure 21.6}
    Branch WAN challenges}
  \item
    \protect\hyperlink{c21.xhtmlux5cux23figure21-7}{\textbf{Figure 21.7}
    Intelligent WAN}
  \item
    \protect\hyperlink{c21.xhtmlux5cux23figure21-8}{\textbf{Figure 21.8}
    IWAN four technology pillars}
  \item
    \protect\hyperlink{c21.xhtmlux5cux23figure21-9}{\textbf{Figure 21.9}
    DTE-DCE-DTE WAN connection: Clocking is typically provided by the
    DCE network to routers. In nonproduction environments, a DCE network
    is not always present.}
  \item
    \protect\hyperlink{c21.xhtmlux5cux23figure21-10}{\textbf{Figure
    21.10} Cisco's HDLC frame format: Each vendor's HDLC has a
    proprietary data field to support multiprotocol environments.}
  \item
    \protect\hyperlink{c21.xhtmlux5cux23figure21-11}{\textbf{Figure
    21.11} Configuring Cisco's HDLC proprietary WAN encapsulation}
  \item
    \protect\hyperlink{c21.xhtmlux5cux23figure21-12}{\textbf{Figure
    21.12} Point-to-Point Protocol stack}
  \item
    \protect\hyperlink{c21.xhtmlux5cux23figure21-13}{\textbf{Figure
    21.13} PPP session establishment}
  \item
    \protect\hyperlink{c21.xhtmlux5cux23figure21-14}{\textbf{Figure
    21.14} PPP authentication example}
  \item
    \protect\hyperlink{c21.xhtmlux5cux23figure21-15}{\textbf{Figure
    21.15} Failed PPP authentication}
  \item
    \protect\hyperlink{c21.xhtmlux5cux23figure21-16}{\textbf{Figure
    21.16} Mismatched WAN encapsulations}
  \item
    \protect\hyperlink{c21.xhtmlux5cux23figure21-17}{\textbf{Figure
    21.17} Mismatched IP addresses}
  \item
    \protect\hyperlink{c21.xhtmlux5cux23figure21-18}{\textbf{Figure
    21.18} MLP between Corp and SF routers}
  \item
    \protect\hyperlink{c21.xhtmlux5cux23figure21-19}{\textbf{Figure
    21.19} PPPoE with ADSL}
  \item
    \protect\hyperlink{c21.xhtmlux5cux23figure21-20}{\textbf{Figure
    21.20} Example of using a VPN}
  \item
    \protect\hyperlink{c21.xhtmlux5cux23figure21-21}{\textbf{Figure
    21.21} Enterprise-managed VPNs}
  \item
    \protect\hyperlink{c21.xhtmlux5cux23figure21-22}{\textbf{Figure
    21.22} Provider-managed VPNs}
  \item
    \protect\hyperlink{c21.xhtmlux5cux23figure21-23}{\textbf{Figure
    21.23} Generic Routing Encapsulation (GRE) tunnel structure}
  \item
    \protect\hyperlink{c21.xhtmlux5cux23figure21-24}{\textbf{Figure
    21.24} Example of GRE configuration}
  \item
    \protect\hyperlink{c21.xhtmlux5cux23figure21-25}{\textbf{Figure
    21.25} Example of EBGP lay layout}
  \end{enumerate}
\item
  \protect\hyperlink{c22.xhtml}{Chapter 22}

  \begin{enumerate}
  \tightlist
  \item
    \protect\hyperlink{c22.xhtmlux5cux23figure22-1}{\textbf{Figure 22.1}
    Switch stacking}
  \item
    \protect\hyperlink{c22.xhtmlux5cux23figure22-2}{\textbf{Figure 22.2}
    Cloud computing is on-demand.}
  \item
    \protect\hyperlink{c22.xhtmlux5cux23figure22-3}{\textbf{Figure 22.3}
    Advantages of cloud computing}
  \item
    \protect\hyperlink{c22.xhtmlux5cux23figure22-4}{\textbf{Figure 22.4}
    Cloud computing service}
  \item
    \protect\hyperlink{c22.xhtmlux5cux23figure22-5}{\textbf{Figure 22.5}
    The SDN architecture}
  \item
    \protect\hyperlink{c22.xhtmlux5cux23figure22-6}{\textbf{Figure 22.6}
    Southbound interfaces}
  \item
    \protect\hyperlink{c22.xhtmlux5cux23figure22-7}{\textbf{Figure 22.7}
    Northbound interfaces}
  \item
    \protect\hyperlink{c22.xhtmlux5cux23figure22-8}{\textbf{Figure 22.8}
    Where APIC-EM fits in the SDN stack}
  \item
    \protect\hyperlink{c22.xhtmlux5cux23figure22-9}{\textbf{Figure 22.9}
    APIC-Enterprise Module}
  \item
    \protect\hyperlink{c22.xhtmlux5cux23figure22-10}{\textbf{Figure
    22.10} APIC-Enterprise Module path trace sample}
  \item
    \protect\hyperlink{c22.xhtmlux5cux23figure22-11}{\textbf{Figure
    22.11} APIC-Enterprise Module IWAN}
  \item
    \protect\hyperlink{c22.xhtmlux5cux23figure22-12}{\textbf{Figure
    22.12} Traffic characteristics}
  \item
    \protect\hyperlink{c22.xhtmlux5cux23figure22-13}{\textbf{Figure
    22.13} Trust boundaries}
  \item
    \protect\hyperlink{c22.xhtmlux5cux23figure22-14}{\textbf{Figure
    22.14} Policing and shaping rate limiters}
  \item
    \protect\hyperlink{c22.xhtmlux5cux23figure22-15}{\textbf{Figure
    22.15} Congestion management}
  \item
    \protect\hyperlink{c22.xhtmlux5cux23figure22-16}{\textbf{Figure
    22.16} Queuing mechanisms}
  \item
    \protect\hyperlink{c22.xhtmlux5cux23figure22-17}{\textbf{Figure
    22.17} Congestion avoidance}
  \end{enumerate}
\end{enumerate}

\subsection{Guide}

\begin{enumerate}
\tightlist
\item
  \href{fcover.xhtml}{Cover}
\item
\item
  \protect\hyperlink{p01.xhtml}{Part}
\end{enumerate}

\subsection{Pages}

\begin{enumerate}
\tightlist
\item
  \protect\hyperlink{f_01.xhtmlux5cux23Page_iv}{iv}
\item
  \protect\hyperlink{f_02.xhtmlux5cux23Page_v}{v}
\item
  \protect\hyperlink{f_03.xhtmlux5cux23Page_vi}{vi}
\item
  \protect\hyperlink{f_04.xhtmlux5cux23Page_xxv}{xxv}
\item
  \protect\hyperlink{f_04.xhtmlux5cux23Page_xxvi}{xxvi}
\item
  \protect\hyperlink{f_04.xhtmlux5cux23Page_xxvii}{xxvii}
\item
  \protect\hyperlink{f_04.xhtmlux5cux23Page_xxviii}{xxviii}
\item
  \protect\hyperlink{f_04.xhtmlux5cux23Page_xxix}{xxix}
\item
  \protect\hyperlink{f_04.xhtmlux5cux23Page_xxx}{xxx}
\item
  \protect\hyperlink{f_04.xhtmlux5cux23Page_xxxi}{xxxi}
\item
  \protect\hyperlink{f_04.xhtmlux5cux23Page_xxxii}{xxxii}
\item
  \protect\hyperlink{f_04.xhtmlux5cux23Page_xxxiii}{xxxiii}
\item
  \protect\hyperlink{f_04.xhtmlux5cux23Page_xxxiv}{xxxiv}
\item
  \protect\hyperlink{f_04.xhtmlux5cux23Page_xxxv}{xxxv}
\item
  \protect\hyperlink{f_04.xhtmlux5cux23Page_xxxvi}{xxxvi}
\item
  \protect\hyperlink{f_04.xhtmlux5cux23Page_xxxvii}{xxxvii}
\item
  \protect\hyperlink{f_04.xhtmlux5cux23Page_xxxviii}{xxxviii}
\item
  \protect\hyperlink{f_04.xhtmlux5cux23Page_xxxix}{xxxix}
\item
  \protect\hyperlink{f_04.xhtmlux5cux23Page_xl}{xl}
\item
  \protect\hyperlink{f_04.xhtmlux5cux23Page_xli}{xli}
\item
  \protect\hyperlink{f_04.xhtmlux5cux23Page_xlii}{xlii}
\item
  \protect\hyperlink{f_04.xhtmlux5cux23Page_xliii}{xliii}
\item
  \protect\hyperlink{f_04.xhtmlux5cux23Page_xliv}{xliv}
\item
  \protect\hyperlink{f_04.xhtmlux5cux23Page_xlv}{xlv}
\item
  \protect\hyperlink{f_04.xhtmlux5cux23Page_xlvi}{xlvi}
\item
  \protect\hyperlink{f_04.xhtmlux5cux23Page_xlvii}{xlvii}
\item
  \protect\hyperlink{f_04.xhtmlux5cux23Page_xlviii}{xlviii}
\item
  \protect\hyperlink{f_04.xhtmlux5cux23Page_xlix}{xlix}
\item
  \protect\hyperlink{f_04.xhtmlux5cux23Page_l}{l}
\item
  \protect\hyperlink{f_04.xhtmlux5cux23Page_li}{li}
\item
  \protect\hyperlink{f_04.xhtmlux5cux23Page_lii}{lii}
\item
  \protect\hyperlink{f_04.xhtmlux5cux23Page_liii}{liii}
\item
  \protect\hyperlink{f_04.xhtmlux5cux23Page_liv}{liv}
\item
  \protect\hyperlink{f_04.xhtmlux5cux23Page_lv}{lv}
\item
  \protect\hyperlink{f_04.xhtmlux5cux23Page_lvii}{lvii}
\item
  \protect\hyperlink{f_04.xhtmlux5cux23Page_lviii}{lviii}
\item
  \protect\hyperlink{f_04.xhtmlux5cux23Page_lix}{lix}
\item
  \protect\hyperlink{f_04.xhtmlux5cux23Page_lx}{lx}
\item
  \protect\hyperlink{f_04.xhtmlux5cux23Page_lxi}{lxi}
\item
  \protect\hyperlink{f_04.xhtmlux5cux23Page_lxii}{lxii}
\item
  \protect\hyperlink{f_04.xhtmlux5cux23Page_lxiii}{lxiii}
\item
  \protect\hyperlink{f_04.xhtmlux5cux23Page_lxiv}{lxiv}
\item
  \protect\hyperlink{f_04.xhtmlux5cux23Page_lxv}{lxv}
\item
  \protect\hyperlink{f_05.xhtmlux5cux23Page_l}{l}
\item
  \protect\hyperlink{f_05.xhtmlux5cux23Page_li}{li}
\item
  \protect\hyperlink{f_05.xhtmlux5cux23Page_lii}{lii}
\item
  \protect\hyperlink{f_05.xhtmlux5cux23Page_liii}{liii}
\item
  \protect\hyperlink{f_05.xhtmlux5cux23Page_liv}{liv}
\item
  \protect\hyperlink{f_05.xhtmlux5cux23Page_lv}{lv}
\item
  \protect\hyperlink{f_05.xhtmlux5cux23Page_lvii}{lvii}
\item
  \protect\hyperlink{f_05.xhtmlux5cux23Page_lviii}{lviii}
\item
  \protect\hyperlink{f_05.xhtmlux5cux23Page_lix}{lix}
\item
  \protect\hyperlink{f_05.xhtmlux5cux23Page_lx}{lx}
\item
  \protect\hyperlink{f_06.xhtmlux5cux23Page_lxi}{lxi}
\item
  \protect\hyperlink{f_06.xhtmlux5cux23Page_lxii}{lxii}
\item
  \protect\hyperlink{f_06.xhtmlux5cux23Page_lxiii}{lxiii}
\item
  \protect\hyperlink{f_06.xhtmlux5cux23Page_lxiv}{lxiv}
\item
  \protect\hyperlink{f_06.xhtmlux5cux23Page_lxv}{lxv}
\item
  \protect\hyperlink{p01.xhtmlux5cux23Page_1}{1}
\item
  \protect\hyperlink{c01.xhtmlux5cux23Page_3}{3}
\item
  \protect\hyperlink{c01.xhtmlux5cux23Page_4}{4}
\item
  \protect\hyperlink{c01.xhtmlux5cux23Page_5}{5}
\item
  \protect\hyperlink{c01.xhtmlux5cux23Page_6}{6}
\item
  \protect\hyperlink{c01.xhtmlux5cux23Page_7}{7}
\item
  \protect\hyperlink{c01.xhtmlux5cux23Page_8}{8}
\item
  \protect\hyperlink{c01.xhtmlux5cux23Page_9}{9}
\item
  \protect\hyperlink{c01.xhtmlux5cux23Page_10}{10}
\item
  \protect\hyperlink{c01.xhtmlux5cux23Page_11}{11}
\item
  \protect\hyperlink{c01.xhtmlux5cux23Page_12}{12}
\item
  \protect\hyperlink{c01.xhtmlux5cux23Page_13}{13}
\item
  \protect\hyperlink{c01.xhtmlux5cux23Page_14}{14}
\item
  \protect\hyperlink{c01.xhtmlux5cux23Page_15}{15}
\item
  \protect\hyperlink{c01.xhtmlux5cux23Page_16}{16}
\item
  \protect\hyperlink{c01.xhtmlux5cux23Page_17}{17}
\item
  \protect\hyperlink{c01.xhtmlux5cux23Page_18}{18}
\item
  \protect\hyperlink{c01.xhtmlux5cux23Page_19}{19}
\item
  \protect\hyperlink{c01.xhtmlux5cux23Page_20}{20}
\item
  \protect\hyperlink{c01.xhtmlux5cux23Page_21}{21}
\item
  \protect\hyperlink{c01.xhtmlux5cux23Page_22}{22}
\item
  \protect\hyperlink{c01.xhtmlux5cux23Page_23}{23}
\item
  \protect\hyperlink{c01.xhtmlux5cux23Page_24}{24}
\item
  \protect\hyperlink{c01.xhtmlux5cux23Page_25}{25}
\item
  \protect\hyperlink{c01.xhtmlux5cux23Page_26}{26}
\item
  \protect\hyperlink{c01.xhtmlux5cux23Page_27}{27}
\item
  \protect\hyperlink{c01.xhtmlux5cux23Page_28}{28}
\item
  \protect\hyperlink{c01.xhtmlux5cux23Page_29}{29}
\item
  \protect\hyperlink{c01.xhtmlux5cux23Page_30}{30}
\item
  \protect\hyperlink{c01.xhtmlux5cux23Page_31}{31}
\item
  \protect\hyperlink{c01.xhtmlux5cux23Page_32}{32}
\item
  \protect\hyperlink{c01.xhtmlux5cux23Page_33}{33}
\item
  \protect\hyperlink{c01.xhtmlux5cux23Page_34}{34}
\item
  \protect\hyperlink{c01.xhtmlux5cux23Page_35}{35}
\item
  \protect\hyperlink{c01.xhtmlux5cux23Page_36}{36}
\item
  \protect\hyperlink{c01.xhtmlux5cux23Page_37}{37}
\item
  \protect\hyperlink{c01.xhtmlux5cux23Page_38}{38}
\item
  \protect\hyperlink{c01.xhtmlux5cux23Page_39}{39}
\item
  \protect\hyperlink{c01.xhtmlux5cux23Page_40}{40}
\item
  \protect\hyperlink{c01.xhtmlux5cux23Page_41}{41}
\item
  \protect\hyperlink{c02.xhtmlux5cux23Page_41}{41}
\item
  \protect\hyperlink{c02.xhtmlux5cux23Page_42}{42}
\item
  \protect\hyperlink{c02.xhtmlux5cux23Page_43}{43}
\item
  \protect\hyperlink{c02.xhtmlux5cux23Page_44}{44}
\item
  \protect\hyperlink{c02.xhtmlux5cux23Page_45}{45}
\item
  \protect\hyperlink{c02.xhtmlux5cux23Page_46}{46}
\item
  \protect\hyperlink{c02.xhtmlux5cux23Page_47}{47}
\item
  \protect\hyperlink{c02.xhtmlux5cux23Page_48}{48}
\item
  \protect\hyperlink{c02.xhtmlux5cux23Page_49}{49}
\item
  \protect\hyperlink{c02.xhtmlux5cux23Page_50}{50}
\item
  \protect\hyperlink{c02.xhtmlux5cux23Page_51}{51}
\item
  \protect\hyperlink{c02.xhtmlux5cux23Page_52}{52}
\item
  \protect\hyperlink{c02.xhtmlux5cux23Page_53}{53}
\item
  \protect\hyperlink{c02.xhtmlux5cux23Page_54}{54}
\item
  \protect\hyperlink{c02.xhtmlux5cux23Page_55}{55}
\item
  \protect\hyperlink{c02.xhtmlux5cux23Page_56}{56}
\item
  \protect\hyperlink{c02.xhtmlux5cux23Page_57}{57}
\item
  \protect\hyperlink{c02.xhtmlux5cux23Page_58}{58}
\item
  \protect\hyperlink{c02.xhtmlux5cux23Page_59}{59}
\item
  \protect\hyperlink{c02.xhtmlux5cux23Page_60}{60}
\item
  \protect\hyperlink{c02.xhtmlux5cux23Page_61}{61}
\item
  \protect\hyperlink{c02.xhtmlux5cux23Page_62}{62}
\item
  \protect\hyperlink{c02.xhtmlux5cux23Page_63}{63}
\item
  \protect\hyperlink{c02.xhtmlux5cux23Page_64}{64}
\item
  \protect\hyperlink{c02.xhtmlux5cux23Page_65}{65}
\item
  \protect\hyperlink{c02.xhtmlux5cux23Page_66}{66}
\item
  \protect\hyperlink{c02.xhtmlux5cux23Page_67}{67}
\item
  \protect\hyperlink{c02.xhtmlux5cux23Page_68}{68}
\item
  \protect\hyperlink{c02.xhtmlux5cux23Page_69}{69}
\item
  \protect\hyperlink{c02.xhtmlux5cux23Page_70}{70}
\item
  \protect\hyperlink{c02.xhtmlux5cux23Page_71}{71}
\item
  \protect\hyperlink{c02.xhtmlux5cux23Page_72}{72}
\item
  \protect\hyperlink{c02.xhtmlux5cux23Page_73}{73}
\item
  \protect\hyperlink{c02.xhtmlux5cux23Page_74}{74}
\item
  \protect\hyperlink{c02.xhtmlux5cux23Page_75}{75}
\item
  \protect\hyperlink{c02.xhtmlux5cux23Page_76}{76}
\item
  \protect\hyperlink{c02.xhtmlux5cux23Page_77}{77}
\item
  \protect\hyperlink{c02.xhtmlux5cux23Page_78}{78}
\item
  \protect\hyperlink{c02.xhtmlux5cux23Page_79}{79}
\item
  \protect\hyperlink{c02.xhtmlux5cux23Page_80}{80}
\item
  \protect\hyperlink{c02.xhtmlux5cux23Page_81}{81}
\item
  \protect\hyperlink{c02.xhtmlux5cux23Page_82}{82}
\item
  \protect\hyperlink{c02.xhtmlux5cux23Page_83}{83}
\item
  \protect\hyperlink{c03.xhtmlux5cux23Page_85}{85}
\item
  \protect\hyperlink{c03.xhtmlux5cux23Page_86}{86}
\item
  \protect\hyperlink{c03.xhtmlux5cux23Page_87}{87}
\item
  \protect\hyperlink{c03.xhtmlux5cux23Page_88}{88}
\item
  \protect\hyperlink{c03.xhtmlux5cux23Page_89}{89}
\item
  \protect\hyperlink{c03.xhtmlux5cux23Page_90}{90}
\item
  \protect\hyperlink{c03.xhtmlux5cux23Page_91}{91}
\item
  \protect\hyperlink{c03.xhtmlux5cux23Page_92}{92}
\item
  \protect\hyperlink{c03.xhtmlux5cux23Page_93}{93}
\item
  \protect\hyperlink{c03.xhtmlux5cux23Page_94}{94}
\item
  \protect\hyperlink{c03.xhtmlux5cux23Page_95}{95}
\item
  \protect\hyperlink{c03.xhtmlux5cux23Page_96}{96}
\item
  \protect\hyperlink{c03.xhtmlux5cux23Page_97}{97}
\item
  \protect\hyperlink{c03.xhtmlux5cux23Page_98}{98}
\item
  \protect\hyperlink{c03.xhtmlux5cux23Page_99}{99}
\item
  \protect\hyperlink{c03.xhtmlux5cux23Page_100}{100}
\item
  \protect\hyperlink{c03.xhtmlux5cux23Page_101}{101}
\item
  \protect\hyperlink{c03.xhtmlux5cux23Page_102}{102}
\item
  \protect\hyperlink{c03.xhtmlux5cux23Page_103}{103}
\item
  \protect\hyperlink{c03.xhtmlux5cux23Page_104}{104}
\item
  \protect\hyperlink{c03.xhtmlux5cux23Page_105}{105}
\item
  \protect\hyperlink{c03.xhtmlux5cux23Page_106}{106}
\item
  \protect\hyperlink{c03.xhtmlux5cux23Page_107}{107}
\item
  \protect\hyperlink{c03.xhtmlux5cux23Page_108}{108}
\item
  \protect\hyperlink{c03.xhtmlux5cux23Page_109}{109}
\item
  \protect\hyperlink{c03.xhtmlux5cux23Page_110}{110}
\item
  \protect\hyperlink{c03.xhtmlux5cux23Page_111}{111}
\item
  \protect\hyperlink{c03.xhtmlux5cux23Page_112}{112}
\item
  \protect\hyperlink{c03.xhtmlux5cux23Page_113}{113}
\item
  \protect\hyperlink{c03.xhtmlux5cux23Page_114}{114}
\item
  \protect\hyperlink{c03.xhtmlux5cux23Page_115}{115}
\item
  \protect\hyperlink{c03.xhtmlux5cux23Page_116}{116}
\item
  \protect\hyperlink{c03.xhtmlux5cux23Page_117}{117}
\item
  \protect\hyperlink{c03.xhtmlux5cux23Page_118}{118}
\item
  \protect\hyperlink{c03.xhtmlux5cux23Page_119}{119}
\item
  \protect\hyperlink{c03.xhtmlux5cux23Page_120}{120}
\item
  \protect\hyperlink{c03.xhtmlux5cux23Page_121}{121}
\item
  \protect\hyperlink{c03.xhtmlux5cux23Page_122}{122}
\item
  \protect\hyperlink{c03.xhtmlux5cux23Page_123}{123}
\item
  \protect\hyperlink{c03.xhtmlux5cux23Page_124}{124}
\item
  \protect\hyperlink{c03.xhtmlux5cux23Page_125}{125}
\item
  \protect\hyperlink{c03.xhtmlux5cux23Page_126}{126}
\item
  \protect\hyperlink{c03.xhtmlux5cux23Page_127}{127}
\item
  \protect\hyperlink{c03.xhtmlux5cux23Page_128}{128}
\item
  \protect\hyperlink{c03.xhtmlux5cux23Page_129}{129}
\item
  \protect\hyperlink{c03.xhtmlux5cux23Page_130}{130}
\item
  \protect\hyperlink{c03.xhtmlux5cux23Page_131}{131}
\item
  \protect\hyperlink{c03.xhtmlux5cux23Page_132}{132}
\item
  \protect\hyperlink{c03.xhtmlux5cux23Page_133}{133}
\item
  \protect\hyperlink{c03.xhtmlux5cux23Page_134}{134}
\item
  \protect\hyperlink{c04.xhtmlux5cux23Page_135}{135}
\item
  \protect\hyperlink{c04.xhtmlux5cux23Page_136}{136}
\item
  \protect\hyperlink{c04.xhtmlux5cux23Page_137}{137}
\item
  \protect\hyperlink{c04.xhtmlux5cux23Page_138}{138}
\item
  \protect\hyperlink{c04.xhtmlux5cux23Page_139}{139}
\item
  \protect\hyperlink{c04.xhtmlux5cux23Page_140}{140}
\item
  \protect\hyperlink{c04.xhtmlux5cux23Page_141}{141}
\item
  \protect\hyperlink{c04.xhtmlux5cux23Page_142}{142}
\item
  \protect\hyperlink{c04.xhtmlux5cux23Page_143}{143}
\item
  \protect\hyperlink{c04.xhtmlux5cux23Page_144}{144}
\item
  \protect\hyperlink{c04.xhtmlux5cux23Page_145}{145}
\item
  \protect\hyperlink{c04.xhtmlux5cux23Page_146}{146}
\item
  \protect\hyperlink{c04.xhtmlux5cux23Page_147}{147}
\item
  \protect\hyperlink{c04.xhtmlux5cux23Page_148}{148}
\item
  \protect\hyperlink{c04.xhtmlux5cux23Page_149}{149}
\item
  \protect\hyperlink{c04.xhtmlux5cux23Page_150}{150}
\item
  \protect\hyperlink{c04.xhtmlux5cux23Page_151}{151}
\item
  \protect\hyperlink{c04.xhtmlux5cux23Page_152}{152}
\item
  \protect\hyperlink{c04.xhtmlux5cux23Page_153}{153}
\item
  \protect\hyperlink{c04.xhtmlux5cux23Page_154}{154}
\item
  \protect\hyperlink{c04.xhtmlux5cux23Page_155}{155}
\item
  \protect\hyperlink{c04.xhtmlux5cux23Page_156}{156}
\item
  \protect\hyperlink{c04.xhtmlux5cux23Page_157}{157}
\item
  \protect\hyperlink{c04.xhtmlux5cux23Page_158}{158}
\item
  \protect\hyperlink{c04.xhtmlux5cux23Page_159}{159}
\item
  \protect\hyperlink{c04.xhtmlux5cux23Page_160}{160}
\item
  \protect\hyperlink{c04.xhtmlux5cux23Page_161}{161}
\item
  \protect\hyperlink{c04.xhtmlux5cux23Page_162}{162}
\item
  \protect\hyperlink{c04.xhtmlux5cux23Page_163}{163}
\item
  \protect\hyperlink{c04.xhtmlux5cux23Page_164}{164}
\item
  \protect\hyperlink{c04.xhtmlux5cux23Page_165}{165}
\item
  \protect\hyperlink{c04.xhtmlux5cux23Page_166}{166}
\item
  \protect\hyperlink{c04.xhtmlux5cux23Page_167}{167}
\item
  \protect\hyperlink{c04.xhtmlux5cux23Page_168}{168}
\item
  \protect\hyperlink{c04.xhtmlux5cux23Page_169}{169}
\item
  \protect\hyperlink{c04.xhtmlux5cux23Page_170}{170}
\item
  \protect\hyperlink{c04.xhtmlux5cux23Page_171}{171}
\item
  \protect\hyperlink{c04.xhtmlux5cux23Page_172}{172}
\item
  \protect\hyperlink{c04.xhtmlux5cux23Page_173}{173}
\item
  \protect\hyperlink{c04.xhtmlux5cux23Page_174}{174}
\item
  \protect\hyperlink{c05.xhtmlux5cux23Page_175}{175}
\item
  \protect\hyperlink{c05.xhtmlux5cux23Page_176}{176}
\item
  \protect\hyperlink{c05.xhtmlux5cux23Page_177}{177}
\item
  \protect\hyperlink{c05.xhtmlux5cux23Page_178}{178}
\item
  \protect\hyperlink{c05.xhtmlux5cux23Page_179}{179}
\item
  \protect\hyperlink{c05.xhtmlux5cux23Page_180}{180}
\item
  \protect\hyperlink{c05.xhtmlux5cux23Page_181}{181}
\item
  \protect\hyperlink{c05.xhtmlux5cux23Page_182}{182}
\item
  \protect\hyperlink{c05.xhtmlux5cux23Page_183}{183}
\item
  \protect\hyperlink{c05.xhtmlux5cux23Page_184}{184}
\item
  \protect\hyperlink{c05.xhtmlux5cux23Page_185}{185}
\item
  \protect\hyperlink{c05.xhtmlux5cux23Page_186}{186}
\item
  \protect\hyperlink{c05.xhtmlux5cux23Page_187}{187}
\item
  \protect\hyperlink{c05.xhtmlux5cux23Page_188}{188}
\item
  \protect\hyperlink{c05.xhtmlux5cux23Page_189}{189}
\item
  \protect\hyperlink{c05.xhtmlux5cux23Page_190}{190}
\item
  \protect\hyperlink{c05.xhtmlux5cux23Page_191}{191}
\item
  \protect\hyperlink{c05.xhtmlux5cux23Page_192}{192}
\item
  \protect\hyperlink{c05.xhtmlux5cux23Page_193}{193}
\item
  \protect\hyperlink{c05.xhtmlux5cux23Page_194}{194}
\item
  \protect\hyperlink{c05.xhtmlux5cux23Page_195}{195}
\item
  \protect\hyperlink{c05.xhtmlux5cux23Page_196}{196}
\item
  \protect\hyperlink{c05.xhtmlux5cux23Page_197}{197}
\item
  \protect\hyperlink{c05.xhtmlux5cux23Page_198}{198}
\item
  \protect\hyperlink{c05.xhtmlux5cux23Page_199}{199}
\item
  \protect\hyperlink{c05.xhtmlux5cux23Page_200}{200}
\item
  \protect\hyperlink{c05.xhtmlux5cux23Page_201}{201}
\item
  \protect\hyperlink{c05.xhtmlux5cux23Page_202}{202}
\item
  \protect\hyperlink{c05.xhtmlux5cux23Page_203}{203}
\item
  \protect\hyperlink{c05.xhtmlux5cux23Page_204}{204}
\item
  \protect\hyperlink{c06.xhtmlux5cux23Page_205}{205}
\item
  \protect\hyperlink{c06.xhtmlux5cux23Page_206}{206}
\item
  \protect\hyperlink{c06.xhtmlux5cux23Page_207}{207}
\item
  \protect\hyperlink{c06.xhtmlux5cux23Page_208}{208}
\item
  \protect\hyperlink{c06.xhtmlux5cux23Page_209}{209}
\item
  \protect\hyperlink{c06.xhtmlux5cux23Page_210}{210}
\item
  \protect\hyperlink{c06.xhtmlux5cux23Page_211}{211}
\item
  \protect\hyperlink{c06.xhtmlux5cux23Page_212}{212}
\item
  \protect\hyperlink{c06.xhtmlux5cux23Page_213}{213}
\item
  \protect\hyperlink{c06.xhtmlux5cux23Page_214}{214}
\item
  \protect\hyperlink{c06.xhtmlux5cux23Page_215}{215}
\item
  \protect\hyperlink{c06.xhtmlux5cux23Page_216}{216}
\item
  \protect\hyperlink{c06.xhtmlux5cux23Page_217}{217}
\item
  \protect\hyperlink{c06.xhtmlux5cux23Page_218}{218}
\item
  \protect\hyperlink{c06.xhtmlux5cux23Page_219}{219}
\item
  \protect\hyperlink{c06.xhtmlux5cux23Page_220}{220}
\item
  \protect\hyperlink{c06.xhtmlux5cux23Page_221}{221}
\item
  \protect\hyperlink{c06.xhtmlux5cux23Page_222}{222}
\item
  \protect\hyperlink{c06.xhtmlux5cux23Page_223}{223}
\item
  \protect\hyperlink{c06.xhtmlux5cux23Page_224}{224}
\item
  \protect\hyperlink{c06.xhtmlux5cux23Page_225}{225}
\item
  \protect\hyperlink{c06.xhtmlux5cux23Page_226}{226}
\item
  \protect\hyperlink{c06.xhtmlux5cux23Page_227}{227}
\item
  \protect\hyperlink{c06.xhtmlux5cux23Page_228}{228}
\item
  \protect\hyperlink{c06.xhtmlux5cux23Page_229}{229}
\item
  \protect\hyperlink{c06.xhtmlux5cux23Page_230}{230}
\item
  \protect\hyperlink{c06.xhtmlux5cux23Page_231}{231}
\item
  \protect\hyperlink{c06.xhtmlux5cux23Page_232}{232}
\item
  \protect\hyperlink{c06.xhtmlux5cux23Page_233}{233}
\item
  \protect\hyperlink{c06.xhtmlux5cux23Page_234}{234}
\item
  \protect\hyperlink{c06.xhtmlux5cux23Page_235}{235}
\item
  \protect\hyperlink{c06.xhtmlux5cux23Page_236}{236}
\item
  \protect\hyperlink{c06.xhtmlux5cux23Page_237}{237}
\item
  \protect\hyperlink{c06.xhtmlux5cux23Page_238}{238}
\item
  \protect\hyperlink{c06.xhtmlux5cux23Page_239}{239}
\item
  \protect\hyperlink{c06.xhtmlux5cux23Page_240}{240}
\item
  \protect\hyperlink{c06.xhtmlux5cux23Page_241}{241}
\item
  \protect\hyperlink{c06.xhtmlux5cux23Page_242}{242}
\item
  \protect\hyperlink{c06.xhtmlux5cux23Page_243}{243}
\item
  \protect\hyperlink{c06.xhtmlux5cux23Page_244}{244}
\item
  \protect\hyperlink{c06.xhtmlux5cux23Page_245}{245}
\item
  \protect\hyperlink{c06.xhtmlux5cux23Page_246}{246}
\item
  \protect\hyperlink{c06.xhtmlux5cux23Page_247}{247}
\item
  \protect\hyperlink{c06.xhtmlux5cux23Page_248}{248}
\item
  \protect\hyperlink{c06.xhtmlux5cux23Page_249}{249}
\item
  \protect\hyperlink{c06.xhtmlux5cux23Page_250}{250}
\item
  \protect\hyperlink{c06.xhtmlux5cux23Page_251}{251}
\item
  \protect\hyperlink{c06.xhtmlux5cux23Page_252}{252}
\item
  \protect\hyperlink{c06.xhtmlux5cux23Page_253}{253}
\item
  \protect\hyperlink{c06.xhtmlux5cux23Page_254}{254}
\item
  \protect\hyperlink{c06.xhtmlux5cux23Page_255}{255}
\item
  \protect\hyperlink{c06.xhtmlux5cux23Page_256}{256}
\item
  \protect\hyperlink{c06.xhtmlux5cux23Page_257}{257}
\item
  \protect\hyperlink{c06.xhtmlux5cux23Page_258}{258}
\item
  \protect\hyperlink{c06.xhtmlux5cux23Page_259}{259}
\item
  \protect\hyperlink{c06.xhtmlux5cux23Page_260}{260}
\item
  \protect\hyperlink{c06.xhtmlux5cux23Page_261}{261}
\item
  \protect\hyperlink{c06.xhtmlux5cux23Page_262}{262}
\item
  \protect\hyperlink{c06.xhtmlux5cux23Page_263}{263}
\item
  \protect\hyperlink{c06.xhtmlux5cux23Page_264}{264}
\item
  \protect\hyperlink{c06.xhtmlux5cux23Page_265}{265}
\item
  \protect\hyperlink{c06.xhtmlux5cux23Page_266}{266}
\item
  \protect\hyperlink{c06.xhtmlux5cux23Page_267}{267}
\item
  \protect\hyperlink{c06.xhtmlux5cux23Page_268}{268}
\item
  \protect\hyperlink{c06.xhtmlux5cux23Page_269}{269}
\item
  \protect\hyperlink{c06.xhtmlux5cux23Page_270}{270}
\item
  \protect\hyperlink{c06.xhtmlux5cux23Page_271}{271}
\item
  \protect\hyperlink{c07.xhtmlux5cux23Page_273}{273}
\item
  \protect\hyperlink{c07.xhtmlux5cux23Page_274}{274}
\item
  \protect\hyperlink{c07.xhtmlux5cux23Page_275}{275}
\item
  \protect\hyperlink{c07.xhtmlux5cux23Page_276}{276}
\item
  \protect\hyperlink{c07.xhtmlux5cux23Page_277}{277}
\item
  \protect\hyperlink{c07.xhtmlux5cux23Page_278}{278}
\item
  \protect\hyperlink{c07.xhtmlux5cux23Page_279}{279}
\item
  \protect\hyperlink{c07.xhtmlux5cux23Page_280}{280}
\item
  \protect\hyperlink{c07.xhtmlux5cux23Page_281}{281}
\item
  \protect\hyperlink{c07.xhtmlux5cux23Page_282}{282}
\item
  \protect\hyperlink{c07.xhtmlux5cux23Page_283}{283}
\item
  \protect\hyperlink{c07.xhtmlux5cux23Page_284}{284}
\item
  \protect\hyperlink{c07.xhtmlux5cux23Page_285}{285}
\item
  \protect\hyperlink{c07.xhtmlux5cux23Page_286}{286}
\item
  \protect\hyperlink{c07.xhtmlux5cux23Page_287}{287}
\item
  \protect\hyperlink{c07.xhtmlux5cux23Page_288}{288}
\item
  \protect\hyperlink{c07.xhtmlux5cux23Page_289}{289}
\item
  \protect\hyperlink{c07.xhtmlux5cux23Page_290}{290}
\item
  \protect\hyperlink{c07.xhtmlux5cux23Page_291}{291}
\item
  \protect\hyperlink{c07.xhtmlux5cux23Page_292}{292}
\item
  \protect\hyperlink{c07.xhtmlux5cux23Page_293}{293}
\item
  \protect\hyperlink{c07.xhtmlux5cux23Page_294}{294}
\item
  \protect\hyperlink{c07.xhtmlux5cux23Page_295}{295}
\item
  \protect\hyperlink{c07.xhtmlux5cux23Page_296}{296}
\item
  \protect\hyperlink{c07.xhtmlux5cux23Page_297}{297}
\item
  \protect\hyperlink{c07.xhtmlux5cux23Page_298}{298}
\item
  \protect\hyperlink{c07.xhtmlux5cux23Page_299}{299}
\item
  \protect\hyperlink{c07.xhtmlux5cux23Page_300}{300}
\item
  \protect\hyperlink{c07.xhtmlux5cux23Page_301}{301}
\item
  \protect\hyperlink{c07.xhtmlux5cux23Page_302}{302}
\item
  \protect\hyperlink{c07.xhtmlux5cux23Page_303}{303}
\item
  \protect\hyperlink{c07.xhtmlux5cux23Page_304}{304}
\item
  \protect\hyperlink{c07.xhtmlux5cux23Page_305}{305}
\item
  \protect\hyperlink{c07.xhtmlux5cux23Page_306}{306}
\item
  \protect\hyperlink{c07.xhtmlux5cux23Page_307}{307}
\item
  \protect\hyperlink{c07.xhtmlux5cux23Page_308}{308}
\item
  \protect\hyperlink{c07.xhtmlux5cux23Page_309}{309}
\item
  \protect\hyperlink{c07.xhtmlux5cux23Page_310}{310}
\item
  \protect\hyperlink{c07.xhtmlux5cux23Page_311}{311}
\item
  \protect\hyperlink{c07.xhtmlux5cux23Page_312}{312}
\item
  \protect\hyperlink{c07.xhtmlux5cux23Page_313}{313}
\item
  \protect\hyperlink{c07.xhtmlux5cux23Page_314}{314}
\item
  \protect\hyperlink{c07.xhtmlux5cux23Page_315}{315}
\item
  \protect\hyperlink{c07.xhtmlux5cux23Page_316}{316}
\item
  \protect\hyperlink{c07.xhtmlux5cux23Page_317}{317}
\item
  \protect\hyperlink{c07.xhtmlux5cux23Page_318}{318}
\item
  \protect\hyperlink{c07.xhtmlux5cux23Page_319}{319}
\item
  \protect\hyperlink{c07.xhtmlux5cux23Page_320}{320}
\item
  \protect\hyperlink{c07.xhtmlux5cux23Page_321}{321}
\item
  \protect\hyperlink{c07.xhtmlux5cux23Page_322}{322}
\item
  \protect\hyperlink{c08.xhtmlux5cux23Page_323}{323}
\item
  \protect\hyperlink{c08.xhtmlux5cux23Page_324}{324}
\item
  \protect\hyperlink{c08.xhtmlux5cux23Page_325}{325}
\item
  \protect\hyperlink{c08.xhtmlux5cux23Page_326}{326}
\item
  \protect\hyperlink{c08.xhtmlux5cux23Page_327}{327}
\item
  \protect\hyperlink{c08.xhtmlux5cux23Page_328}{328}
\item
  \protect\hyperlink{c08.xhtmlux5cux23Page_329}{329}
\item
  \protect\hyperlink{c08.xhtmlux5cux23Page_330}{330}
\item
  \protect\hyperlink{c08.xhtmlux5cux23Page_331}{331}
\item
  \protect\hyperlink{c08.xhtmlux5cux23Page_332}{332}
\item
  \protect\hyperlink{c08.xhtmlux5cux23Page_333}{333}
\item
  \protect\hyperlink{c08.xhtmlux5cux23Page_334}{334}
\item
  \protect\hyperlink{c08.xhtmlux5cux23Page_335}{335}
\item
  \protect\hyperlink{c08.xhtmlux5cux23Page_336}{336}
\item
  \protect\hyperlink{c08.xhtmlux5cux23Page_337}{337}
\item
  \protect\hyperlink{c08.xhtmlux5cux23Page_338}{338}
\item
  \protect\hyperlink{c08.xhtmlux5cux23Page_339}{339}
\item
  \protect\hyperlink{c08.xhtmlux5cux23Page_340}{340}
\item
  \protect\hyperlink{c08.xhtmlux5cux23Page_341}{341}
\item
  \protect\hyperlink{c08.xhtmlux5cux23Page_342}{342}
\item
  \protect\hyperlink{c08.xhtmlux5cux23Page_343}{343}
\item
  \protect\hyperlink{c08.xhtmlux5cux23Page_344}{344}
\item
  \protect\hyperlink{c08.xhtmlux5cux23Page_345}{345}
\item
  \protect\hyperlink{c08.xhtmlux5cux23Page_346}{346}
\item
  \protect\hyperlink{c08.xhtmlux5cux23Page_347}{347}
\item
  \protect\hyperlink{c08.xhtmlux5cux23Page_348}{348}
\item
  \protect\hyperlink{c08.xhtmlux5cux23Page_349}{349}
\item
  \protect\hyperlink{c08.xhtmlux5cux23Page_350}{350}
\item
  \protect\hyperlink{c08.xhtmlux5cux23Page_351}{351}
\item
  \protect\hyperlink{c08.xhtmlux5cux23Page_352}{352}
\item
  \protect\hyperlink{c08.xhtmlux5cux23Page_353}{353}
\item
  \protect\hyperlink{c08.xhtmlux5cux23Page_354}{354}
\item
  \protect\hyperlink{c08.xhtmlux5cux23Page_355}{355}
\item
  \protect\hyperlink{c09.xhtmlux5cux23Page_357}{357}
\item
  \protect\hyperlink{c09.xhtmlux5cux23Page_358}{358}
\item
  \protect\hyperlink{c09.xhtmlux5cux23Page_359}{359}
\item
  \protect\hyperlink{c09.xhtmlux5cux23Page_360}{360}
\item
  \protect\hyperlink{c09.xhtmlux5cux23Page_361}{361}
\item
  \protect\hyperlink{c09.xhtmlux5cux23Page_362}{362}
\item
  \protect\hyperlink{c09.xhtmlux5cux23Page_363}{363}
\item
  \protect\hyperlink{c09.xhtmlux5cux23Page_364}{364}
\item
  \protect\hyperlink{c09.xhtmlux5cux23Page_365}{365}
\item
  \protect\hyperlink{c09.xhtmlux5cux23Page_366}{366}
\item
  \protect\hyperlink{c09.xhtmlux5cux23Page_367}{367}
\item
  \protect\hyperlink{c09.xhtmlux5cux23Page_368}{368}
\item
  \protect\hyperlink{c09.xhtmlux5cux23Page_369}{369}
\item
  \protect\hyperlink{c09.xhtmlux5cux23Page_370}{370}
\item
  \protect\hyperlink{c09.xhtmlux5cux23Page_371}{371}
\item
  \protect\hyperlink{c09.xhtmlux5cux23Page_372}{372}
\item
  \protect\hyperlink{c09.xhtmlux5cux23Page_373}{373}
\item
  \protect\hyperlink{c09.xhtmlux5cux23Page_374}{374}
\item
  \protect\hyperlink{c09.xhtmlux5cux23Page_375}{375}
\item
  \protect\hyperlink{c09.xhtmlux5cux23Page_376}{376}
\item
  \protect\hyperlink{c09.xhtmlux5cux23Page_377}{377}
\item
  \protect\hyperlink{c09.xhtmlux5cux23Page_378}{378}
\item
  \protect\hyperlink{c09.xhtmlux5cux23Page_379}{379}
\item
  \protect\hyperlink{c09.xhtmlux5cux23Page_380}{380}
\item
  \protect\hyperlink{c09.xhtmlux5cux23Page_381}{381}
\item
  \protect\hyperlink{c09.xhtmlux5cux23Page_382}{382}
\item
  \protect\hyperlink{c09.xhtmlux5cux23Page_383}{383}
\item
  \protect\hyperlink{c09.xhtmlux5cux23Page_384}{384}
\item
  \protect\hyperlink{c09.xhtmlux5cux23Page_385}{385}
\item
  \protect\hyperlink{c09.xhtmlux5cux23Page_386}{386}
\item
  \protect\hyperlink{c09.xhtmlux5cux23Page_387}{387}
\item
  \protect\hyperlink{c09.xhtmlux5cux23Page_388}{388}
\item
  \protect\hyperlink{c09.xhtmlux5cux23Page_389}{389}
\item
  \protect\hyperlink{c09.xhtmlux5cux23Page_390}{390}
\item
  \protect\hyperlink{c09.xhtmlux5cux23Page_391}{391}
\item
  \protect\hyperlink{c09.xhtmlux5cux23Page_392}{392}
\item
  \protect\hyperlink{c09.xhtmlux5cux23Page_393}{393}
\item
  \protect\hyperlink{c09.xhtmlux5cux23Page_394}{394}
\item
  \protect\hyperlink{c09.xhtmlux5cux23Page_395}{395}
\item
  \protect\hyperlink{c09.xhtmlux5cux23Page_396}{396}
\item
  \protect\hyperlink{c09.xhtmlux5cux23Page_397}{397}
\item
  \protect\hyperlink{c09.xhtmlux5cux23Page_398}{398}
\item
  \protect\hyperlink{c09.xhtmlux5cux23Page_399}{399}
\item
  \protect\hyperlink{c09.xhtmlux5cux23Page_400}{400}
\item
  \protect\hyperlink{c09.xhtmlux5cux23Page_401}{401}
\item
  \protect\hyperlink{c09.xhtmlux5cux23Page_402}{402}
\item
  \protect\hyperlink{c09.xhtmlux5cux23Page_403}{403}
\item
  \protect\hyperlink{c09.xhtmlux5cux23Page_404}{404}
\item
  \protect\hyperlink{c09.xhtmlux5cux23Page_405}{405}
\item
  \protect\hyperlink{c09.xhtmlux5cux23Page_406}{406}
\item
  \protect\hyperlink{c09.xhtmlux5cux23Page_407}{407}
\item
  \protect\hyperlink{c09.xhtmlux5cux23Page_408}{408}
\item
  \protect\hyperlink{c09.xhtmlux5cux23Page_409}{409}
\item
  \protect\hyperlink{c10.xhtmlux5cux23Page_411}{411}
\item
  \protect\hyperlink{c10.xhtmlux5cux23Page_412}{412}
\item
  \protect\hyperlink{c10.xhtmlux5cux23Page_413}{413}
\item
  \protect\hyperlink{c10.xhtmlux5cux23Page_414}{414}
\item
  \protect\hyperlink{c10.xhtmlux5cux23Page_415}{415}
\item
  \protect\hyperlink{c10.xhtmlux5cux23Page_416}{416}
\item
  \protect\hyperlink{c10.xhtmlux5cux23Page_417}{417}
\item
  \protect\hyperlink{c10.xhtmlux5cux23Page_418}{418}
\item
  \protect\hyperlink{c10.xhtmlux5cux23Page_419}{419}
\item
  \protect\hyperlink{c10.xhtmlux5cux23Page_420}{420}
\item
  \protect\hyperlink{c10.xhtmlux5cux23Page_421}{421}
\item
  \protect\hyperlink{c10.xhtmlux5cux23Page_422}{422}
\item
  \protect\hyperlink{c10.xhtmlux5cux23Page_423}{423}
\item
  \protect\hyperlink{c10.xhtmlux5cux23Page_424}{424}
\item
  \protect\hyperlink{c10.xhtmlux5cux23Page_425}{425}
\item
  \protect\hyperlink{c10.xhtmlux5cux23Page_426}{426}
\item
  \protect\hyperlink{c10.xhtmlux5cux23Page_427}{427}
\item
  \protect\hyperlink{c10.xhtmlux5cux23Page_428}{428}
\item
  \protect\hyperlink{c10.xhtmlux5cux23Page_429}{429}
\item
  \protect\hyperlink{c10.xhtmlux5cux23Page_430}{430}
\item
  \protect\hyperlink{c10.xhtmlux5cux23Page_431}{431}
\item
  \protect\hyperlink{c10.xhtmlux5cux23Page_432}{432}
\item
  \protect\hyperlink{c10.xhtmlux5cux23Page_433}{433}
\item
  \protect\hyperlink{c10.xhtmlux5cux23Page_434}{434}
\item
  \protect\hyperlink{c10.xhtmlux5cux23Page_435}{435}
\item
  \protect\hyperlink{c10.xhtmlux5cux23Page_436}{436}
\item
  \protect\hyperlink{c10.xhtmlux5cux23Page_437}{437}
\item
  \protect\hyperlink{c10.xhtmlux5cux23Page_438}{438}
\item
  \protect\hyperlink{c10.xhtmlux5cux23Page_439}{439}
\item
  \protect\hyperlink{c10.xhtmlux5cux23Page_440}{440}
\item
  \protect\hyperlink{c10.xhtmlux5cux23Page_441}{441}
\item
  \protect\hyperlink{c10.xhtmlux5cux23Page_442}{442}
\item
  \protect\hyperlink{c11.xhtmlux5cux23Page_443}{443}
\item
  \protect\hyperlink{c11.xhtmlux5cux23Page_444}{444}
\item
  \protect\hyperlink{c11.xhtmlux5cux23Page_445}{445}
\item
  \protect\hyperlink{c11.xhtmlux5cux23Page_446}{446}
\item
  \protect\hyperlink{c11.xhtmlux5cux23Page_447}{447}
\item
  \protect\hyperlink{c11.xhtmlux5cux23Page_448}{448}
\item
  \protect\hyperlink{c11.xhtmlux5cux23Page_449}{449}
\item
  \protect\hyperlink{c11.xhtmlux5cux23Page_450}{450}
\item
  \protect\hyperlink{c11.xhtmlux5cux23Page_451}{451}
\item
  \protect\hyperlink{c11.xhtmlux5cux23Page_452}{452}
\item
  \protect\hyperlink{c11.xhtmlux5cux23Page_453}{453}
\item
  \protect\hyperlink{c11.xhtmlux5cux23Page_454}{454}
\item
  \protect\hyperlink{c11.xhtmlux5cux23Page_455}{455}
\item
  \protect\hyperlink{c11.xhtmlux5cux23Page_456}{456}
\item
  \protect\hyperlink{c11.xhtmlux5cux23Page_457}{457}
\item
  \protect\hyperlink{c11.xhtmlux5cux23Page_458}{458}
\item
  \protect\hyperlink{c11.xhtmlux5cux23Page_459}{459}
\item
  \protect\hyperlink{c11.xhtmlux5cux23Page_460}{460}
\item
  \protect\hyperlink{c11.xhtmlux5cux23Page_461}{461}
\item
  \protect\hyperlink{c11.xhtmlux5cux23Page_462}{462}
\item
  \protect\hyperlink{c11.xhtmlux5cux23Page_463}{463}
\item
  \protect\hyperlink{c11.xhtmlux5cux23Page_464}{464}
\item
  \protect\hyperlink{c11.xhtmlux5cux23Page_465}{465}
\item
  \protect\hyperlink{c11.xhtmlux5cux23Page_466}{466}
\item
  \protect\hyperlink{c11.xhtmlux5cux23Page_467}{467}
\item
  \protect\hyperlink{c11.xhtmlux5cux23Page_468}{468}
\item
  \protect\hyperlink{c11.xhtmlux5cux23Page_469}{469}
\item
  \protect\hyperlink{c11.xhtmlux5cux23Page_470}{470}
\item
  \protect\hyperlink{c11.xhtmlux5cux23Page_471}{471}
\item
  \protect\hyperlink{c11.xhtmlux5cux23Page_472}{472}
\item
  \protect\hyperlink{c11.xhtmlux5cux23Page_473}{473}
\item
  \protect\hyperlink{c11.xhtmlux5cux23Page_474}{474}
\item
  \protect\hyperlink{c11.xhtmlux5cux23Page_475}{475}
\item
  \protect\hyperlink{c11.xhtmlux5cux23Page_476}{476}
\item
  \protect\hyperlink{c11.xhtmlux5cux23Page_477}{477}
\item
  \protect\hyperlink{c11.xhtmlux5cux23Page_478}{478}
\item
  \protect\hyperlink{c11.xhtmlux5cux23Page_479}{479}
\item
  \protect\hyperlink{c11.xhtmlux5cux23Page_480}{480}
\item
  \protect\hyperlink{c11.xhtmlux5cux23Page_481}{481}
\item
  \protect\hyperlink{c11.xhtmlux5cux23Page_482}{482}
\item
  \protect\hyperlink{c12.xhtmlux5cux23Page_483}{483}
\item
  \protect\hyperlink{c12.xhtmlux5cux23Page_484}{484}
\item
  \protect\hyperlink{c12.xhtmlux5cux23Page_485}{485}
\item
  \protect\hyperlink{c12.xhtmlux5cux23Page_486}{486}
\item
  \protect\hyperlink{c12.xhtmlux5cux23Page_487}{487}
\item
  \protect\hyperlink{c12.xhtmlux5cux23Page_488}{488}
\item
  \protect\hyperlink{c12.xhtmlux5cux23Page_489}{489}
\item
  \protect\hyperlink{c12.xhtmlux5cux23Page_490}{490}
\item
  \protect\hyperlink{c12.xhtmlux5cux23Page_491}{491}
\item
  \protect\hyperlink{c12.xhtmlux5cux23Page_492}{492}
\item
  \protect\hyperlink{c12.xhtmlux5cux23Page_493}{493}
\item
  \protect\hyperlink{c12.xhtmlux5cux23Page_494}{494}
\item
  \protect\hyperlink{c12.xhtmlux5cux23Page_495}{495}
\item
  \protect\hyperlink{c12.xhtmlux5cux23Page_496}{496}
\item
  \protect\hyperlink{c12.xhtmlux5cux23Page_497}{497}
\item
  \protect\hyperlink{c12.xhtmlux5cux23Page_498}{498}
\item
  \protect\hyperlink{c12.xhtmlux5cux23Page_499}{499}
\item
  \protect\hyperlink{c12.xhtmlux5cux23Page_500}{500}
\item
  \protect\hyperlink{c12.xhtmlux5cux23Page_501}{501}
\item
  \protect\hyperlink{c12.xhtmlux5cux23Page_502}{502}
\item
  \protect\hyperlink{c12.xhtmlux5cux23Page_503}{503}
\item
  \protect\hyperlink{c12.xhtmlux5cux23Page_504}{504}
\item
  \protect\hyperlink{c12.xhtmlux5cux23Page_505}{505}
\item
  \protect\hyperlink{c12.xhtmlux5cux23Page_506}{506}
\item
  \protect\hyperlink{c12.xhtmlux5cux23Page_507}{507}
\item
  \protect\hyperlink{c12.xhtmlux5cux23Page_508}{508}
\item
  \protect\hyperlink{c12.xhtmlux5cux23Page_509}{509}
\item
  \protect\hyperlink{c12.xhtmlux5cux23Page_510}{510}
\item
  \protect\hyperlink{c12.xhtmlux5cux23Page_511}{511}
\item
  \protect\hyperlink{c12.xhtmlux5cux23Page_512}{512}
\item
  \protect\hyperlink{c12.xhtmlux5cux23Page_513}{513}
\item
  \protect\hyperlink{c12.xhtmlux5cux23Page_514}{514}
\item
  \protect\hyperlink{c12.xhtmlux5cux23Page_515}{515}
\item
  \protect\hyperlink{c12.xhtmlux5cux23Page_516}{516}
\item
  \protect\hyperlink{c12.xhtmlux5cux23Page_517}{517}
\item
  \protect\hyperlink{c12.xhtmlux5cux23Page_518}{518}
\item
  \protect\hyperlink{c12.xhtmlux5cux23Page_519}{519}
\item
  \protect\hyperlink{c12.xhtmlux5cux23Page_520}{520}
\item
  \protect\hyperlink{c13.xhtmlux5cux23Page_521}{521}
\item
  \protect\hyperlink{c13.xhtmlux5cux23Page_522}{522}
\item
  \protect\hyperlink{c13.xhtmlux5cux23Page_523}{523}
\item
  \protect\hyperlink{c13.xhtmlux5cux23Page_524}{524}
\item
  \protect\hyperlink{c13.xhtmlux5cux23Page_525}{525}
\item
  \protect\hyperlink{c13.xhtmlux5cux23Page_526}{526}
\item
  \protect\hyperlink{c13.xhtmlux5cux23Page_527}{527}
\item
  \protect\hyperlink{c13.xhtmlux5cux23Page_528}{528}
\item
  \protect\hyperlink{c13.xhtmlux5cux23Page_529}{529}
\item
  \protect\hyperlink{c13.xhtmlux5cux23Page_530}{530}
\item
  \protect\hyperlink{c13.xhtmlux5cux23Page_531}{531}
\item
  \protect\hyperlink{c13.xhtmlux5cux23Page_532}{532}
\item
  \protect\hyperlink{c13.xhtmlux5cux23Page_533}{533}
\item
  \protect\hyperlink{c13.xhtmlux5cux23Page_534}{534}
\item
  \protect\hyperlink{c13.xhtmlux5cux23Page_535}{535}
\item
  \protect\hyperlink{c13.xhtmlux5cux23Page_536}{536}
\item
  \protect\hyperlink{c13.xhtmlux5cux23Page_537}{537}
\item
  \protect\hyperlink{c13.xhtmlux5cux23Page_538}{538}
\item
  \protect\hyperlink{c13.xhtmlux5cux23Page_539}{539}
\item
  \protect\hyperlink{c13.xhtmlux5cux23Page_540}{540}
\item
  \protect\hyperlink{c13.xhtmlux5cux23Page_541}{541}
\item
  \protect\hyperlink{c13.xhtmlux5cux23Page_542}{542}
\item
  \protect\hyperlink{c13.xhtmlux5cux23Page_543}{543}
\item
  \protect\hyperlink{c13.xhtmlux5cux23Page_544}{544}
\item
  \protect\hyperlink{c13.xhtmlux5cux23Page_545}{545}
\item
  \protect\hyperlink{c14.xhtmlux5cux23Page_547}{547}
\item
  \protect\hyperlink{c14.xhtmlux5cux23Page_548}{548}
\item
  \protect\hyperlink{c14.xhtmlux5cux23Page_549}{549}
\item
  \protect\hyperlink{c14.xhtmlux5cux23Page_550}{550}
\item
  \protect\hyperlink{c14.xhtmlux5cux23Page_551}{551}
\item
  \protect\hyperlink{c14.xhtmlux5cux23Page_552}{552}
\item
  \protect\hyperlink{c14.xhtmlux5cux23Page_553}{553}
\item
  \protect\hyperlink{c14.xhtmlux5cux23Page_554}{554}
\item
  \protect\hyperlink{c14.xhtmlux5cux23Page_555}{555}
\item
  \protect\hyperlink{c14.xhtmlux5cux23Page_556}{556}
\item
  \protect\hyperlink{c14.xhtmlux5cux23Page_557}{557}
\item
  \protect\hyperlink{c14.xhtmlux5cux23Page_558}{558}
\item
  \protect\hyperlink{c14.xhtmlux5cux23Page_559}{559}
\item
  \protect\hyperlink{c14.xhtmlux5cux23Page_560}{560}
\item
  \protect\hyperlink{c14.xhtmlux5cux23Page_561}{561}
\item
  \protect\hyperlink{c14.xhtmlux5cux23Page_562}{562}
\item
  \protect\hyperlink{c14.xhtmlux5cux23Page_563}{563}
\item
  \protect\hyperlink{c14.xhtmlux5cux23Page_564}{564}
\item
  \protect\hyperlink{c14.xhtmlux5cux23Page_565}{565}
\item
  \protect\hyperlink{c14.xhtmlux5cux23Page_566}{566}
\item
  \protect\hyperlink{c14.xhtmlux5cux23Page_567}{567}
\item
  \protect\hyperlink{c14.xhtmlux5cux23Page_568}{568}
\item
  \protect\hyperlink{c14.xhtmlux5cux23Page_569}{569}
\item
  \protect\hyperlink{c14.xhtmlux5cux23Page_570}{570}
\item
  \protect\hyperlink{c14.xhtmlux5cux23Page_571}{571}
\item
  \protect\hyperlink{c14.xhtmlux5cux23Page_572}{572}
\item
  \protect\hyperlink{c14.xhtmlux5cux23Page_573}{573}
\item
  \protect\hyperlink{c14.xhtmlux5cux23Page_574}{574}
\item
  \protect\hyperlink{c14.xhtmlux5cux23Page_575}{575}
\item
  \protect\hyperlink{c14.xhtmlux5cux23Page_576}{576}
\item
  \protect\hyperlink{c14.xhtmlux5cux23Page_577}{577}
\item
  \protect\hyperlink{c14.xhtmlux5cux23Page_578}{578}
\item
  \protect\hyperlink{c14.xhtmlux5cux23Page_579}{579}
\item
  \protect\hyperlink{c14.xhtmlux5cux23Page_580}{580}
\item
  \protect\hyperlink{p02.xhtmlux5cux23Page_581}{581}
\item
  \protect\hyperlink{c15.xhtmlux5cux23Page_583}{583}
\item
  \protect\hyperlink{c15.xhtmlux5cux23Page_584}{584}
\item
  \protect\hyperlink{c15.xhtmlux5cux23Page_585}{585}
\item
  \protect\hyperlink{c15.xhtmlux5cux23Page_586}{586}
\item
  \protect\hyperlink{c15.xhtmlux5cux23Page_587}{587}
\item
  \protect\hyperlink{c15.xhtmlux5cux23Page_588}{588}
\item
  \protect\hyperlink{c15.xhtmlux5cux23Page_589}{589}
\item
  \protect\hyperlink{c15.xhtmlux5cux23Page_590}{590}
\item
  \protect\hyperlink{c15.xhtmlux5cux23Page_591}{591}
\item
  \protect\hyperlink{c15.xhtmlux5cux23Page_592}{592}
\item
  \protect\hyperlink{c15.xhtmlux5cux23Page_593}{593}
\item
  \protect\hyperlink{c15.xhtmlux5cux23Page_594}{594}
\item
  \protect\hyperlink{c15.xhtmlux5cux23Page_595}{595}
\item
  \protect\hyperlink{c15.xhtmlux5cux23Page_596}{596}
\item
  \protect\hyperlink{c15.xhtmlux5cux23Page_597}{597}
\item
  \protect\hyperlink{c15.xhtmlux5cux23Page_598}{598}
\item
  \protect\hyperlink{c15.xhtmlux5cux23Page_599}{599}
\item
  \protect\hyperlink{c15.xhtmlux5cux23Page_600}{600}
\item
  \protect\hyperlink{c15.xhtmlux5cux23Page_601}{601}
\item
  \protect\hyperlink{c15.xhtmlux5cux23Page_602}{602}
\item
  \protect\hyperlink{c15.xhtmlux5cux23Page_603}{603}
\item
  \protect\hyperlink{c15.xhtmlux5cux23Page_604}{604}
\item
  \protect\hyperlink{c15.xhtmlux5cux23Page_605}{605}
\item
  \protect\hyperlink{c15.xhtmlux5cux23Page_606}{606}
\item
  \protect\hyperlink{c15.xhtmlux5cux23Page_607}{607}
\item
  \protect\hyperlink{c15.xhtmlux5cux23Page_608}{608}
\item
  \protect\hyperlink{c15.xhtmlux5cux23Page_609}{609}
\item
  \protect\hyperlink{c15.xhtmlux5cux23Page_610}{610}
\item
  \protect\hyperlink{c15.xhtmlux5cux23Page_611}{611}
\item
  \protect\hyperlink{c15.xhtmlux5cux23Page_612}{612}
\item
  \protect\hyperlink{c15.xhtmlux5cux23Page_613}{613}
\item
  \protect\hyperlink{c15.xhtmlux5cux23Page_614}{614}
\item
  \protect\hyperlink{c15.xhtmlux5cux23Page_615}{615}
\item
  \protect\hyperlink{c15.xhtmlux5cux23Page_616}{616}
\item
  \protect\hyperlink{c15.xhtmlux5cux23Page_617}{617}
\item
  \protect\hyperlink{c15.xhtmlux5cux23Page_618}{618}
\item
  \protect\hyperlink{c15.xhtmlux5cux23Page_619}{619}
\item
  \protect\hyperlink{c15.xhtmlux5cux23Page_620}{620}
\item
  \protect\hyperlink{c15.xhtmlux5cux23Page_621}{621}
\item
  \protect\hyperlink{c15.xhtmlux5cux23Page_622}{622}
\item
  \protect\hyperlink{c15.xhtmlux5cux23Page_623}{623}
\item
  \protect\hyperlink{c15.xhtmlux5cux23Page_624}{624}
\item
  \protect\hyperlink{c15.xhtmlux5cux23Page_625}{625}
\item
  \protect\hyperlink{c15.xhtmlux5cux23Page_626}{626}
\item
  \protect\hyperlink{c15.xhtmlux5cux23Page_627}{627}
\item
  \protect\hyperlink{c15.xhtmlux5cux23Page_628}{628}
\item
  \protect\hyperlink{c15.xhtmlux5cux23Page_629}{629}
\item
  \protect\hyperlink{c15.xhtmlux5cux23Page_630}{630}
\item
  \protect\hyperlink{c15.xhtmlux5cux23Page_631}{631}
\item
  \protect\hyperlink{c15.xhtmlux5cux23Page_632}{632}
\item
  \protect\hyperlink{c15.xhtmlux5cux23Page_633}{633}
\item
  \protect\hyperlink{c15.xhtmlux5cux23Page_634}{634}
\item
  \protect\hyperlink{c15.xhtmlux5cux23Page_635}{635}
\item
  \protect\hyperlink{c15.xhtmlux5cux23Page_636}{636}
\item
  \protect\hyperlink{c15.xhtmlux5cux23Page_637}{637}
\item
  \protect\hyperlink{c15.xhtmlux5cux23Page_638}{638}
\item
  \protect\hyperlink{c15.xhtmlux5cux23Page_639}{639}
\item
  \protect\hyperlink{c15.xhtmlux5cux23Page_640}{640}
\item
  \protect\hyperlink{c15.xhtmlux5cux23Page_641}{641}
\item
  \protect\hyperlink{c15.xhtmlux5cux23Page_642}{642}
\item
  \protect\hyperlink{c15.xhtmlux5cux23Page_643}{643}
\item
  \protect\hyperlink{c15.xhtmlux5cux23Page_644}{644}
\item
  \protect\hyperlink{c15.xhtmlux5cux23Page_645}{645}
\item
  \protect\hyperlink{c15.xhtmlux5cux23Page_646}{646}
\item
  \protect\hyperlink{c15.xhtmlux5cux23Page_647}{647}
\item
  \protect\hyperlink{c16.xhtmlux5cux23Page_649}{649}
\item
  \protect\hyperlink{c16.xhtmlux5cux23Page_650}{650}
\item
  \protect\hyperlink{c16.xhtmlux5cux23Page_651}{651}
\item
  \protect\hyperlink{c16.xhtmlux5cux23Page_652}{652}
\item
  \protect\hyperlink{c16.xhtmlux5cux23Page_653}{653}
\item
  \protect\hyperlink{c16.xhtmlux5cux23Page_654}{654}
\item
  \protect\hyperlink{c16.xhtmlux5cux23Page_655}{655}
\item
  \protect\hyperlink{c16.xhtmlux5cux23Page_656}{656}
\item
  \protect\hyperlink{c16.xhtmlux5cux23Page_657}{657}
\item
  \protect\hyperlink{c16.xhtmlux5cux23Page_658}{658}
\item
  \protect\hyperlink{c16.xhtmlux5cux23Page_659}{659}
\item
  \protect\hyperlink{c16.xhtmlux5cux23Page_660}{660}
\item
  \protect\hyperlink{c16.xhtmlux5cux23Page_661}{661}
\item
  \protect\hyperlink{c16.xhtmlux5cux23Page_662}{662}
\item
  \protect\hyperlink{c16.xhtmlux5cux23Page_663}{663}
\item
  \protect\hyperlink{c16.xhtmlux5cux23Page_664}{664}
\item
  \protect\hyperlink{c16.xhtmlux5cux23Page_665}{665}
\item
  \protect\hyperlink{c16.xhtmlux5cux23Page_666}{666}
\item
  \protect\hyperlink{c16.xhtmlux5cux23Page_667}{667}
\item
  \protect\hyperlink{c16.xhtmlux5cux23Page_668}{668}
\item
  \protect\hyperlink{c16.xhtmlux5cux23Page_669}{669}
\item
  \protect\hyperlink{c16.xhtmlux5cux23Page_670}{670}
\item
  \protect\hyperlink{c16.xhtmlux5cux23Page_671}{671}
\item
  \protect\hyperlink{c16.xhtmlux5cux23Page_672}{672}
\item
  \protect\hyperlink{c16.xhtmlux5cux23Page_673}{673}
\item
  \protect\hyperlink{c16.xhtmlux5cux23Page_674}{674}
\item
  \protect\hyperlink{c16.xhtmlux5cux23Page_675}{675}
\item
  \protect\hyperlink{c16.xhtmlux5cux23Page_676}{676}
\item
  \protect\hyperlink{c16.xhtmlux5cux23Page_677}{677}
\item
  \protect\hyperlink{c16.xhtmlux5cux23Page_678}{678}
\item
  \protect\hyperlink{c16.xhtmlux5cux23Page_679}{679}
\item
  \protect\hyperlink{c16.xhtmlux5cux23Page_680}{680}
\item
  \protect\hyperlink{c17.xhtmlux5cux23Page_681}{681}
\item
  \protect\hyperlink{c17.xhtmlux5cux23Page_682}{682}
\item
  \protect\hyperlink{c17.xhtmlux5cux23Page_683}{683}
\item
  \protect\hyperlink{c17.xhtmlux5cux23Page_684}{684}
\item
  \protect\hyperlink{c17.xhtmlux5cux23Page_685}{685}
\item
  \protect\hyperlink{c17.xhtmlux5cux23Page_686}{686}
\item
  \protect\hyperlink{c17.xhtmlux5cux23Page_687}{687}
\item
  \protect\hyperlink{c17.xhtmlux5cux23Page_688}{688}
\item
  \protect\hyperlink{c17.xhtmlux5cux23Page_689}{689}
\item
  \protect\hyperlink{c17.xhtmlux5cux23Page_690}{690}
\item
  \protect\hyperlink{c17.xhtmlux5cux23Page_691}{691}
\item
  \protect\hyperlink{c17.xhtmlux5cux23Page_692}{692}
\item
  \protect\hyperlink{c17.xhtmlux5cux23Page_693}{693}
\item
  \protect\hyperlink{c17.xhtmlux5cux23Page_694}{694}
\item
  \protect\hyperlink{c17.xhtmlux5cux23Page_695}{695}
\item
  \protect\hyperlink{c17.xhtmlux5cux23Page_696}{696}
\item
  \protect\hyperlink{c17.xhtmlux5cux23Page_697}{697}
\item
  \protect\hyperlink{c17.xhtmlux5cux23Page_698}{698}
\item
  \protect\hyperlink{c17.xhtmlux5cux23Page_699}{699}
\item
  \protect\hyperlink{c17.xhtmlux5cux23Page_700}{700}
\item
  \protect\hyperlink{c17.xhtmlux5cux23Page_701}{701}
\item
  \protect\hyperlink{c17.xhtmlux5cux23Page_702}{702}
\item
  \protect\hyperlink{c17.xhtmlux5cux23Page_703}{703}
\item
  \protect\hyperlink{c17.xhtmlux5cux23Page_704}{704}
\item
  \protect\hyperlink{c17.xhtmlux5cux23Page_705}{705}
\item
  \protect\hyperlink{c17.xhtmlux5cux23Page_706}{706}
\item
  \protect\hyperlink{c17.xhtmlux5cux23Page_707}{707}
\item
  \protect\hyperlink{c17.xhtmlux5cux23Page_708}{708}
\item
  \protect\hyperlink{c17.xhtmlux5cux23Page_709}{709}
\item
  \protect\hyperlink{c17.xhtmlux5cux23Page_710}{710}
\item
  \protect\hyperlink{c17.xhtmlux5cux23Page_711}{711}
\item
  \protect\hyperlink{c17.xhtmlux5cux23Page_712}{712}
\item
  \protect\hyperlink{c17.xhtmlux5cux23Page_713}{713}
\item
  \protect\hyperlink{c17.xhtmlux5cux23Page_714}{714}
\item
  \protect\hyperlink{c17.xhtmlux5cux23Page_715}{715}
\item
  \protect\hyperlink{c17.xhtmlux5cux23Page_716}{716}
\item
  \protect\hyperlink{c17.xhtmlux5cux23Page_717}{717}
\item
  \protect\hyperlink{c17.xhtmlux5cux23Page_718}{718}
\item
  \protect\hyperlink{c17.xhtmlux5cux23Page_719}{719}
\item
  \protect\hyperlink{c17.xhtmlux5cux23Page_720}{720}
\item
  \protect\hyperlink{c17.xhtmlux5cux23Page_721}{721}
\item
  \protect\hyperlink{c17.xhtmlux5cux23Page_722}{722}
\item
  \protect\hyperlink{c17.xhtmlux5cux23Page_723}{723}
\item
  \protect\hyperlink{c17.xhtmlux5cux23Page_724}{724}
\item
  \protect\hyperlink{c17.xhtmlux5cux23Page_725}{725}
\item
  \protect\hyperlink{c17.xhtmlux5cux23Page_726}{726}
\item
  \protect\hyperlink{c17.xhtmlux5cux23Page_727}{727}
\item
  \protect\hyperlink{c17.xhtmlux5cux23Page_728}{728}
\item
  \protect\hyperlink{c17.xhtmlux5cux23Page_729}{729}
\item
  \protect\hyperlink{c17.xhtmlux5cux23Page_730}{730}
\item
  \protect\hyperlink{c17.xhtmlux5cux23Page_731}{731}
\item
  \protect\hyperlink{c17.xhtmlux5cux23Page_732}{732}
\item
  \protect\hyperlink{c17.xhtmlux5cux23Page_733}{733}
\item
  \protect\hyperlink{c17.xhtmlux5cux23Page_734}{734}
\item
  \protect\hyperlink{c17.xhtmlux5cux23Page_735}{735}
\item
  \protect\hyperlink{c17.xhtmlux5cux23Page_736}{736}
\item
  \protect\hyperlink{c17.xhtmlux5cux23Page_737}{737}
\item
  \protect\hyperlink{c17.xhtmlux5cux23Page_738}{738}
\item
  \protect\hyperlink{c17.xhtmlux5cux23Page_739}{739}
\item
  \protect\hyperlink{c17.xhtmlux5cux23Page_740}{740}
\item
  \protect\hyperlink{c17.xhtmlux5cux23Page_741}{741}
\item
  \protect\hyperlink{c17.xhtmlux5cux23Page_742}{742}
\item
  \protect\hyperlink{c17.xhtmlux5cux23Page_743}{743}
\item
  \protect\hyperlink{c18.xhtmlux5cux23Page_745}{745}
\item
  \protect\hyperlink{c18.xhtmlux5cux23Page_746}{746}
\item
  \protect\hyperlink{c18.xhtmlux5cux23Page_747}{747}
\item
  \protect\hyperlink{c18.xhtmlux5cux23Page_748}{748}
\item
  \protect\hyperlink{c18.xhtmlux5cux23Page_749}{749}
\item
  \protect\hyperlink{c18.xhtmlux5cux23Page_750}{750}
\item
  \protect\hyperlink{c18.xhtmlux5cux23Page_751}{751}
\item
  \protect\hyperlink{c18.xhtmlux5cux23Page_752}{752}
\item
  \protect\hyperlink{c18.xhtmlux5cux23Page_753}{753}
\item
  \protect\hyperlink{c18.xhtmlux5cux23Page_754}{754}
\item
  \protect\hyperlink{c18.xhtmlux5cux23Page_755}{755}
\item
  \protect\hyperlink{c18.xhtmlux5cux23Page_756}{756}
\item
  \protect\hyperlink{c18.xhtmlux5cux23Page_757}{757}
\item
  \protect\hyperlink{c18.xhtmlux5cux23Page_758}{758}
\item
  \protect\hyperlink{c18.xhtmlux5cux23Page_759}{759}
\item
  \protect\hyperlink{c18.xhtmlux5cux23Page_760}{760}
\item
  \protect\hyperlink{c18.xhtmlux5cux23Page_761}{761}
\item
  \protect\hyperlink{c18.xhtmlux5cux23Page_762}{762}
\item
  \protect\hyperlink{c18.xhtmlux5cux23Page_763}{763}
\item
  \protect\hyperlink{c18.xhtmlux5cux23Page_764}{764}
\item
  \protect\hyperlink{c18.xhtmlux5cux23Page_765}{765}
\item
  \protect\hyperlink{c18.xhtmlux5cux23Page_766}{766}
\item
  \protect\hyperlink{c18.xhtmlux5cux23Page_767}{767}
\item
  \protect\hyperlink{c18.xhtmlux5cux23Page_768}{768}
\item
  \protect\hyperlink{c18.xhtmlux5cux23Page_769}{769}
\item
  \protect\hyperlink{c18.xhtmlux5cux23Page_770}{770}
\item
  \protect\hyperlink{c18.xhtmlux5cux23Page_771}{771}
\item
  \protect\hyperlink{c18.xhtmlux5cux23Page_772}{772}
\item
  \protect\hyperlink{c18.xhtmlux5cux23Page_773}{773}
\item
  \protect\hyperlink{c18.xhtmlux5cux23Page_774}{774}
\item
  \protect\hyperlink{c18.xhtmlux5cux23Page_775}{775}
\item
  \protect\hyperlink{c18.xhtmlux5cux23Page_776}{776}
\item
  \protect\hyperlink{c18.xhtmlux5cux23Page_777}{777}
\item
  \protect\hyperlink{c18.xhtmlux5cux23Page_778}{778}
\item
  \protect\hyperlink{c18.xhtmlux5cux23Page_779}{779}
\item
  \protect\hyperlink{c18.xhtmlux5cux23Page_780}{780}
\item
  \protect\hyperlink{c18.xhtmlux5cux23Page_781}{781}
\item
  \protect\hyperlink{c19.xhtmlux5cux23Page_783}{783}
\item
  \protect\hyperlink{c19.xhtmlux5cux23Page_784}{784}
\item
  \protect\hyperlink{c19.xhtmlux5cux23Page_785}{785}
\item
  \protect\hyperlink{c19.xhtmlux5cux23Page_786}{786}
\item
  \protect\hyperlink{c19.xhtmlux5cux23Page_787}{787}
\item
  \protect\hyperlink{c19.xhtmlux5cux23Page_788}{788}
\item
  \protect\hyperlink{c19.xhtmlux5cux23Page_789}{789}
\item
  \protect\hyperlink{c19.xhtmlux5cux23Page_790}{790}
\item
  \protect\hyperlink{c19.xhtmlux5cux23Page_791}{791}
\item
  \protect\hyperlink{c19.xhtmlux5cux23Page_792}{792}
\item
  \protect\hyperlink{c19.xhtmlux5cux23Page_793}{793}
\item
  \protect\hyperlink{c19.xhtmlux5cux23Page_794}{794}
\item
  \protect\hyperlink{c19.xhtmlux5cux23Page_795}{795}
\item
  \protect\hyperlink{c19.xhtmlux5cux23Page_796}{796}
\item
  \protect\hyperlink{c19.xhtmlux5cux23Page_797}{797}
\item
  \protect\hyperlink{c19.xhtmlux5cux23Page_798}{798}
\item
  \protect\hyperlink{c19.xhtmlux5cux23Page_799}{799}
\item
  \protect\hyperlink{c19.xhtmlux5cux23Page_800}{800}
\item
  \protect\hyperlink{c19.xhtmlux5cux23Page_801}{801}
\item
  \protect\hyperlink{c19.xhtmlux5cux23Page_802}{802}
\item
  \protect\hyperlink{c19.xhtmlux5cux23Page_803}{803}
\item
  \protect\hyperlink{c19.xhtmlux5cux23Page_804}{804}
\item
  \protect\hyperlink{c19.xhtmlux5cux23Page_805}{805}
\item
  \protect\hyperlink{c19.xhtmlux5cux23Page_806}{806}
\item
  \protect\hyperlink{c19.xhtmlux5cux23Page_807}{807}
\item
  \protect\hyperlink{c19.xhtmlux5cux23Page_808}{808}
\item
  \protect\hyperlink{c19.xhtmlux5cux23Page_809}{809}
\item
  \protect\hyperlink{c19.xhtmlux5cux23Page_810}{810}
\item
  \protect\hyperlink{c19.xhtmlux5cux23Page_811}{811}
\item
  \protect\hyperlink{c19.xhtmlux5cux23Page_812}{812}
\item
  \protect\hyperlink{c19.xhtmlux5cux23Page_813}{813}
\item
  \protect\hyperlink{c19.xhtmlux5cux23Page_814}{814}
\item
  \protect\hyperlink{c19.xhtmlux5cux23Page_815}{815}
\item
  \protect\hyperlink{c19.xhtmlux5cux23Page_816}{816}
\item
  \protect\hyperlink{c19.xhtmlux5cux23Page_817}{817}
\item
  \protect\hyperlink{c19.xhtmlux5cux23Page_818}{818}
\item
  \protect\hyperlink{c19.xhtmlux5cux23Page_819}{819}
\item
  \protect\hyperlink{c19.xhtmlux5cux23Page_820}{820}
\item
  \protect\hyperlink{c19.xhtmlux5cux23Page_821}{821}
\item
  \protect\hyperlink{c19.xhtmlux5cux23Page_822}{822}
\item
  \protect\hyperlink{c19.xhtmlux5cux23Page_823}{823}
\item
  \protect\hyperlink{c19.xhtmlux5cux23Page_824}{824}
\item
  \protect\hyperlink{c19.xhtmlux5cux23Page_825}{825}
\item
  \protect\hyperlink{c19.xhtmlux5cux23Page_826}{826}
\item
  \protect\hyperlink{c19.xhtmlux5cux23Page_827}{827}
\item
  \protect\hyperlink{c19.xhtmlux5cux23Page_828}{828}
\item
  \protect\hyperlink{c19.xhtmlux5cux23Page_829}{829}
\item
  \protect\hyperlink{c19.xhtmlux5cux23Page_830}{830}
\item
  \protect\hyperlink{c20.xhtmlux5cux23Page_831}{831}
\item
  \protect\hyperlink{c20.xhtmlux5cux23Page_832}{832}
\item
  \protect\hyperlink{c20.xhtmlux5cux23Page_833}{833}
\item
  \protect\hyperlink{c20.xhtmlux5cux23Page_834}{834}
\item
  \protect\hyperlink{c20.xhtmlux5cux23Page_835}{835}
\item
  \protect\hyperlink{c20.xhtmlux5cux23Page_836}{836}
\item
  \protect\hyperlink{c20.xhtmlux5cux23Page_837}{837}
\item
  \protect\hyperlink{c20.xhtmlux5cux23Page_838}{838}
\item
  \protect\hyperlink{c20.xhtmlux5cux23Page_839}{839}
\item
  \protect\hyperlink{c20.xhtmlux5cux23Page_840}{840}
\item
  \protect\hyperlink{c20.xhtmlux5cux23Page_841}{841}
\item
  \protect\hyperlink{c20.xhtmlux5cux23Page_842}{842}
\item
  \protect\hyperlink{c20.xhtmlux5cux23Page_843}{843}
\item
  \protect\hyperlink{c20.xhtmlux5cux23Page_844}{844}
\item
  \protect\hyperlink{c20.xhtmlux5cux23Page_845}{845}
\item
  \protect\hyperlink{c20.xhtmlux5cux23Page_846}{846}
\item
  \protect\hyperlink{c20.xhtmlux5cux23Page_847}{847}
\item
  \protect\hyperlink{c20.xhtmlux5cux23Page_848}{848}
\item
  \protect\hyperlink{c20.xhtmlux5cux23Page_849}{849}
\item
  \protect\hyperlink{c20.xhtmlux5cux23Page_850}{850}
\item
  \protect\hyperlink{c20.xhtmlux5cux23Page_851}{851}
\item
  \protect\hyperlink{c20.xhtmlux5cux23Page_852}{852}
\item
  \protect\hyperlink{c20.xhtmlux5cux23Page_853}{853}
\item
  \protect\hyperlink{c20.xhtmlux5cux23Page_854}{854}
\item
  \protect\hyperlink{c20.xhtmlux5cux23Page_855}{855}
\item
  \protect\hyperlink{c20.xhtmlux5cux23Page_856}{856}
\item
  \protect\hyperlink{c20.xhtmlux5cux23Page_857}{857}
\item
  \protect\hyperlink{c20.xhtmlux5cux23Page_858}{858}
\item
  \protect\hyperlink{c20.xhtmlux5cux23Page_859}{859}
\item
  \protect\hyperlink{c20.xhtmlux5cux23Page_860}{860}
\item
  \protect\hyperlink{c20.xhtmlux5cux23Page_861}{861}
\item
  \protect\hyperlink{c20.xhtmlux5cux23Page_862}{862}
\item
  \protect\hyperlink{c20.xhtmlux5cux23Page_863}{863}
\item
  \protect\hyperlink{c20.xhtmlux5cux23Page_864}{864}
\item
  \protect\hyperlink{c20.xhtmlux5cux23Page_865}{865}
\item
  \protect\hyperlink{c20.xhtmlux5cux23Page_866}{866}
\item
  \protect\hyperlink{c20.xhtmlux5cux23Page_867}{867}
\item
  \protect\hyperlink{c20.xhtmlux5cux23Page_868}{868}
\item
  \protect\hyperlink{c20.xhtmlux5cux23Page_869}{869}
\item
  \protect\hyperlink{c20.xhtmlux5cux23Page_870}{870}
\item
  \protect\hyperlink{c20.xhtmlux5cux23Page_871}{871}
\item
  \protect\hyperlink{c20.xhtmlux5cux23Page_872}{872}
\item
  \protect\hyperlink{c20.xhtmlux5cux23Page_873}{873}
\item
  \protect\hyperlink{c20.xhtmlux5cux23Page_874}{874}
\item
  \protect\hyperlink{c20.xhtmlux5cux23Page_875}{875}
\item
  \protect\hyperlink{c20.xhtmlux5cux23Page_876}{876}
\item
  \protect\hyperlink{c20.xhtmlux5cux23Page_877}{877}
\item
  \protect\hyperlink{c20.xhtmlux5cux23Page_878}{878}
\item
  \protect\hyperlink{c20.xhtmlux5cux23Page_879}{879}
\item
  \protect\hyperlink{c21.xhtmlux5cux23Page_881}{881}
\item
  \protect\hyperlink{c21.xhtmlux5cux23Page_882}{882}
\item
  \protect\hyperlink{c21.xhtmlux5cux23Page_883}{883}
\item
  \protect\hyperlink{c21.xhtmlux5cux23Page_884}{884}
\item
  \protect\hyperlink{c21.xhtmlux5cux23Page_885}{885}
\item
  \protect\hyperlink{c21.xhtmlux5cux23Page_886}{886}
\item
  \protect\hyperlink{c21.xhtmlux5cux23Page_887}{887}
\item
  \protect\hyperlink{c21.xhtmlux5cux23Page_888}{888}
\item
  \protect\hyperlink{c21.xhtmlux5cux23Page_889}{889}
\item
  \protect\hyperlink{c21.xhtmlux5cux23Page_890}{890}
\item
  \protect\hyperlink{c21.xhtmlux5cux23Page_891}{891}
\item
  \protect\hyperlink{c21.xhtmlux5cux23Page_892}{892}
\item
  \protect\hyperlink{c21.xhtmlux5cux23Page_893}{893}
\item
  \protect\hyperlink{c21.xhtmlux5cux23Page_894}{894}
\item
  \protect\hyperlink{c21.xhtmlux5cux23Page_895}{895}
\item
  \protect\hyperlink{c21.xhtmlux5cux23Page_896}{896}
\item
  \protect\hyperlink{c21.xhtmlux5cux23Page_897}{897}
\item
  \protect\hyperlink{c21.xhtmlux5cux23Page_898}{898}
\item
  \protect\hyperlink{c21.xhtmlux5cux23Page_899}{899}
\item
  \protect\hyperlink{c21.xhtmlux5cux23Page_900}{900}
\item
  \protect\hyperlink{c21.xhtmlux5cux23Page_901}{901}
\item
  \protect\hyperlink{c21.xhtmlux5cux23Page_902}{902}
\item
  \protect\hyperlink{c21.xhtmlux5cux23Page_903}{903}
\item
  \protect\hyperlink{c21.xhtmlux5cux23Page_904}{904}
\item
  \protect\hyperlink{c21.xhtmlux5cux23Page_905}{905}
\item
  \protect\hyperlink{c21.xhtmlux5cux23Page_906}{906}
\item
  \protect\hyperlink{c21.xhtmlux5cux23Page_907}{907}
\item
  \protect\hyperlink{c21.xhtmlux5cux23Page_908}{908}
\item
  \protect\hyperlink{c21.xhtmlux5cux23Page_909}{909}
\item
  \protect\hyperlink{c21.xhtmlux5cux23Page_910}{910}
\item
  \protect\hyperlink{c21.xhtmlux5cux23Page_911}{911}
\item
  \protect\hyperlink{c21.xhtmlux5cux23Page_912}{912}
\item
  \protect\hyperlink{c21.xhtmlux5cux23Page_913}{913}
\item
  \protect\hyperlink{c21.xhtmlux5cux23Page_914}{914}
\item
  \protect\hyperlink{c21.xhtmlux5cux23Page_915}{915}
\item
  \protect\hyperlink{c21.xhtmlux5cux23Page_916}{916}
\item
  \protect\hyperlink{c21.xhtmlux5cux23Page_917}{917}
\item
  \protect\hyperlink{c21.xhtmlux5cux23Page_918}{918}
\item
  \protect\hyperlink{c21.xhtmlux5cux23Page_919}{919}
\item
  \protect\hyperlink{c21.xhtmlux5cux23Page_920}{920}
\item
  \protect\hyperlink{c21.xhtmlux5cux23Page_921}{921}
\item
  \protect\hyperlink{c21.xhtmlux5cux23Page_922}{922}
\item
  \protect\hyperlink{c21.xhtmlux5cux23Page_923}{923}
\item
  \protect\hyperlink{c21.xhtmlux5cux23Page_924}{924}
\item
  \protect\hyperlink{c21.xhtmlux5cux23Page_925}{925}
\item
  \protect\hyperlink{c21.xhtmlux5cux23Page_926}{926}
\item
  \protect\hyperlink{c21.xhtmlux5cux23Page_927}{927}
\item
  \protect\hyperlink{c21.xhtmlux5cux23Page_928}{928}
\item
  \protect\hyperlink{c21.xhtmlux5cux23Page_929}{929}
\item
  \protect\hyperlink{c21.xhtmlux5cux23Page_930}{930}
\item
  \protect\hyperlink{c21.xhtmlux5cux23Page_931}{931}
\item
  \protect\hyperlink{c21.xhtmlux5cux23Page_932}{932}
\item
  \protect\hyperlink{c21.xhtmlux5cux23Page_933}{933}
\item
  \protect\hyperlink{c21.xhtmlux5cux23Page_934}{934}
\item
  \protect\hyperlink{c21.xhtmlux5cux23Page_935}{935}
\item
  \protect\hyperlink{c21.xhtmlux5cux23Page_936}{936}
\item
  \protect\hyperlink{c21.xhtmlux5cux23Page_937}{937}
\item
  \protect\hyperlink{c21.xhtmlux5cux23Page_938}{938}
\item
  \protect\hyperlink{c21.xhtmlux5cux23Page_939}{939}
\item
  \protect\hyperlink{c21.xhtmlux5cux23Page_940}{940}
\item
  \protect\hyperlink{c21.xhtmlux5cux23Page_941}{941}
\item
  \protect\hyperlink{c21.xhtmlux5cux23Page_942}{942}
\item
  \protect\hyperlink{c21.xhtmlux5cux23Page_943}{943}
\item
  \protect\hyperlink{c21.xhtmlux5cux23Page_944}{944}
\item
  \protect\hyperlink{c21.xhtmlux5cux23Page_945}{945}
\item
  \protect\hyperlink{c22.xhtmlux5cux23Page_947}{947}
\item
  \protect\hyperlink{c22.xhtmlux5cux23Page_948}{948}
\item
  \protect\hyperlink{c22.xhtmlux5cux23Page_949}{949}
\item
  \protect\hyperlink{c22.xhtmlux5cux23Page_950}{950}
\item
  \protect\hyperlink{c22.xhtmlux5cux23Page_951}{951}
\item
  \protect\hyperlink{c22.xhtmlux5cux23Page_952}{952}
\item
  \protect\hyperlink{c22.xhtmlux5cux23Page_953}{953}
\item
  \protect\hyperlink{c22.xhtmlux5cux23Page_954}{954}
\item
  \protect\hyperlink{c22.xhtmlux5cux23Page_955}{955}
\item
  \protect\hyperlink{c22.xhtmlux5cux23Page_956}{956}
\item
  \protect\hyperlink{c22.xhtmlux5cux23Page_957}{957}
\item
  \protect\hyperlink{c22.xhtmlux5cux23Page_958}{958}
\item
  \protect\hyperlink{c22.xhtmlux5cux23Page_959}{959}
\item
  \protect\hyperlink{c22.xhtmlux5cux23Page_960}{960}
\item
  \protect\hyperlink{c22.xhtmlux5cux23Page_961}{961}
\item
  \protect\hyperlink{c22.xhtmlux5cux23Page_962}{962}
\item
  \protect\hyperlink{c22.xhtmlux5cux23Page_963}{963}
\item
  \protect\hyperlink{c22.xhtmlux5cux23Page_964}{964}
\item
  \protect\hyperlink{c22.xhtmlux5cux23Page_965}{965}
\item
  \protect\hyperlink{c22.xhtmlux5cux23Page_966}{966}
\item
  \protect\hyperlink{c22.xhtmlux5cux23Page_967}{967}
\item
  \protect\hyperlink{c22.xhtmlux5cux23Page_968}{968}
\item
  \protect\hyperlink{c22.xhtmlux5cux23Page_969}{969}
\item
  \protect\hyperlink{c22.xhtmlux5cux23Page_970}{970}
\item
  \protect\hyperlink{c22.xhtmlux5cux23Page_971}{971}
\item
  \protect\hyperlink{c22.xhtmlux5cux23Page_972}{972}
\item
  \protect\hyperlink{c22.xhtmlux5cux23Page_973}{973}
\item
  \protect\hyperlink{c22.xhtmlux5cux23Page_974}{974}
\item
  \protect\hyperlink{c22.xhtmlux5cux23Page_975}{975}
\item
  \protect\hyperlink{c22.xhtmlux5cux23Page_976}{976}
\item
  \protect\hyperlink{b01.xhtmlux5cux23Page_977}{977}
\item
  \protect\hyperlink{b01.xhtmlux5cux23Page_978}{978}
\item
  \protect\hyperlink{b01.xhtmlux5cux23Page_979}{979}
\item
  \protect\hyperlink{b01.xhtmlux5cux23Page_980}{980}
\item
  \protect\hyperlink{b01.xhtmlux5cux23Page_981}{981}
\item
  \protect\hyperlink{b01.xhtmlux5cux23Page_982}{982}
\item
  \protect\hyperlink{b01.xhtmlux5cux23Page_983}{983}
\item
  \protect\hyperlink{b01.xhtmlux5cux23Page_984}{984}
\item
  \protect\hyperlink{b01.xhtmlux5cux23Page_985}{985}
\item
  \protect\hyperlink{b01.xhtmlux5cux23Page_986}{986}
\item
  \protect\hyperlink{b01.xhtmlux5cux23Page_987}{987}
\item
  \protect\hyperlink{b01.xhtmlux5cux23Page_988}{988}
\item
  \protect\hyperlink{b01.xhtmlux5cux23Page_989}{989}
\item
  \protect\hyperlink{b01.xhtmlux5cux23Page_990}{990}
\item
  \protect\hyperlink{b01.xhtmlux5cux23Page_991}{991}
\item
  \protect\hyperlink{b01.xhtmlux5cux23Page_992}{992}
\item
  \protect\hyperlink{b01.xhtmlux5cux23Page_993}{993}
\item
  \protect\hyperlink{b01.xhtmlux5cux23Page_994}{994}
\item
  \protect\hyperlink{b01.xhtmlux5cux23Page_995}{995}
\item
  \protect\hyperlink{b01.xhtmlux5cux23Page_996}{996}
\item
  \protect\hyperlink{b02.xhtmlux5cux23Page_997}{997}
\item
  \protect\hyperlink{b02.xhtmlux5cux23Page_998}{998}
\item
  \protect\hyperlink{b02.xhtmlux5cux23Page_999}{999}
\item
  \protect\hyperlink{b02.xhtmlux5cux23Page_1000}{1000}
\item
  \protect\hyperlink{b02.xhtmlux5cux23Page_1001}{1001}
\item
  \protect\hyperlink{b02.xhtmlux5cux23Page_1002}{1002}
\item
  \protect\hyperlink{b02.xhtmlux5cux23Page_1003}{1003}
\item
  \protect\hyperlink{b02.xhtmlux5cux23Page_1004}{1004}
\item
  \protect\hyperlink{b02.xhtmlux5cux23Page_1005}{1005}
\item
  \protect\hyperlink{b02.xhtmlux5cux23Page_1006}{1006}
\item
  \protect\hyperlink{b02.xhtmlux5cux23Page_1007}{1007}
\item
  \protect\hyperlink{b02.xhtmlux5cux23Page_1008}{1008}
\item
  \protect\hyperlink{b02.xhtmlux5cux23Page_1009}{1009}
\item
  \protect\hyperlink{b02.xhtmlux5cux23Page_1010}{1010}
\item
  \protect\hyperlink{b02.xhtmlux5cux23Page_1011}{1011}
\item
  \protect\hyperlink{b02.xhtmlux5cux23Page_1012}{1012}
\item
  \protect\hyperlink{b02.xhtmlux5cux23Page_1013}{1013}
\item
  \protect\hyperlink{b02.xhtmlux5cux23Page_1014}{1014}
\item
  \protect\hyperlink{b02.xhtmlux5cux23Page_1015}{1015}
\item
  \protect\hyperlink{b02.xhtmlux5cux23Page_1016}{1016}
\item
  \protect\hyperlink{b02.xhtmlux5cux23Page_1017}{1017}
\item
  \protect\hyperlink{b02.xhtmlux5cux23Page_1018}{1018}
\item
  \protect\hyperlink{b02.xhtmlux5cux23Page_1019}{1019}
\item
  \protect\hyperlink{b02.xhtmlux5cux23Page_1020}{1020}
\item
  \protect\hyperlink{b02.xhtmlux5cux23Page_1021}{1021}
\item
  \protect\hyperlink{b02.xhtmlux5cux23Page_1022}{1022}
\item
  \protect\hyperlink{b02.xhtmlux5cux23Page_1023}{1023}
\item
  \protect\hyperlink{b02.xhtmlux5cux23Page_1024}{1024}
\item
  \protect\hyperlink{b02.xhtmlux5cux23Page_1025}{1025}
\item
  \protect\hyperlink{b02.xhtmlux5cux23Page_1026}{1026}
\item
  \protect\hyperlink{b02.xhtmlux5cux23Page_1027}{1027}
\item
  \protect\hyperlink{b02.xhtmlux5cux23Page_1028}{1028}
\item
  \protect\hyperlink{b02.xhtmlux5cux23Page_1029}{1029}
\item
  \protect\hyperlink{b02.xhtmlux5cux23Page_1030}{1030}
\item
  \protect\hyperlink{b02.xhtmlux5cux23Page_1031}{1031}
\item
  \protect\hyperlink{b02.xhtmlux5cux23Page_1032}{1032}
\item
  \protect\hyperlink{b02.xhtmlux5cux23Page_1033}{1033}
\item
  \protect\hyperlink{b02.xhtmlux5cux23Page_1034}{1034}
\item
  \protect\hyperlink{b02.xhtmlux5cux23Page_1035}{1035}
\item
  \protect\hyperlink{b03.xhtmlux5cux23Page_1037}{1037}
\item
  \protect\hyperlink{b03.xhtmlux5cux23Page_1038}{1038}
\item
  \protect\hyperlink{b03.xhtmlux5cux23Page_1039}{1039}
\item
  \protect\hyperlink{b03.xhtmlux5cux23Page_1040}{1040}
\item
  \protect\hyperlink{b03.xhtmlux5cux23Page_1041}{1041}
\item
  \protect\hyperlink{b03.xhtmlux5cux23Page_1042}{1042}
\item
  \protect\hyperlink{b03.xhtmlux5cux23Page_1043}{1043}
\item
  \protect\hyperlink{b03.xhtmlux5cux23Page_1044}{1044}
\item
  \protect\hyperlink{b03.xhtmlux5cux23Page_1045}{1045}
\item
  \protect\hyperlink{b03.xhtmlux5cux23Page_1046}{1046}
\item
  \protect\hyperlink{b04.xhtmlux5cux23Page_1047}{1047}
\end{enumerate}

\protect\hypertarget{f_04.xhtml}{}{}

\section[{Introduction}]{\texorpdfstring{\protect\hypertarget{f_04.xhtmlux5cux23Page_xxv}{}{}{Introduction}}{Introduction}}

Welcome to the exciting world of Cisco certification! If you've picked
up this book because you want to improve yourself and your life with a
better, more satisfying, and secure job, you've done the right thing.
Whether you're striving to enter the thriving, dynamic IT sector or
seeking to enhance your skill set and advance your position within it,
being Cisco certified can seriously stack the odds in your favor to help
you attain your goals!

Cisco certifications are powerful instruments of success that also
markedly improve your grasp of all things internetworking. As you
progress through this book, you'll gain a complete understanding of
networking that reaches far beyond Cisco devices. By the end of this
book, you'll comprehensively know how disparate network topologies and
technologies work together to form the fully operational networks that
are vital to today's very way of life in the developed world. The
knowledge and expertise you'll gain here is essential for and relevant
to every networking job and is why Cisco certifications are in such high
demand---even at companies with few Cisco devices!

Although it's now common knowledge that Cisco rules routing and
switching, the fact that it also rocks the security, collaboration, data
center, wireless and service provider worlds is also well recognized.
And Cisco certifications reach way beyond the popular but less extensive
certifications like those offered by CompTIA and Microsoft to equip you
with indispensable insight into today's vastly complex networking realm.
Essentially, by deciding to become Cisco certified, you're proudly
announcing that you want to become an unrivaled networking expert---a
goal that this book will get you well on your way to achieving.
Congratulations in advance on the beginning of your brilliant future!

\begin{center}\rule{0.5\linewidth}{0.5pt}\end{center}

\includegraphics{images/note.png}For up-to-the-minute updates covering
additions or modifications to the Cisco certification exams, as well as
additional study tools, review questions, videos, and bonus materials,
be sure to visit the Todd Lammle websites and forum at
\href{http://www.lammle.com/ccna}{www.lammle.com/ccna}.

\begin{center}\rule{0.5\linewidth}{0.5pt}\end{center}

\subsection{Cisco's Network Certifications}

It used to be that to secure the holy grail of Cisco
certifications---the CCIE---you passed only one written test before
being faced with a grueling, formidable hands-on lab. This intensely
daunting, all-or-nothing approach made it nearly impossible to succeed
and predictably didn't work out too well for most people. Cisco
responded to this issue by creating a series of new certifications,
which not only made it easier to eventually win the highly coveted CCIE
prize, it gave employers a way to accurately rate and measure the skill
levels of prospective and current employees. This exciting paradigm
shift in Cisco's certification path truly opened doors that few were
allowed through before!

\protect\hypertarget{f_04.xhtmlux5cux23Page_xxvi}{}{}Beginning in 1998,
obtaining the Cisco Certified Network Associate (CCNA) certification was
the first milestone in the Cisco certification climb, as well as the
official prerequisite to each of the more advanced levels. But that
changed in 2007, when Cisco announced the Cisco Certified Entry Network
Technician (CCENT) certification. And then in May 2016, Cisco once again
proclaimed updates to the CCENT and CCNA Routing and Switching (R/S)
tests. Now the Cisco certification process looks like
\protect\hyperlink{f_04.xhtmlux5cux23figure01-1}{Figure I.1}.

\begin{figure}
\centering
\includegraphics{images/f01.jpg}
\caption{{\protect\hyperlink{f_04.xhtmlux5cux23figureanchor01-1}{\textbf{Figure
I.1}} The Cisco certification path.}}
\end{figure}

\begin{center}\rule{0.5\linewidth}{0.5pt}\end{center}

\includegraphics{images/note.png}I have included only the most popular
tracks in \protect\hyperlink{f_04.xhtmlux5cux23figure01-1}{Figure I.1}.
In addition to the ones in this image, there are also tracks for Design,
Service Provider, Service Provider Operations, and Video.

\begin{center}\rule{0.5\linewidth}{0.5pt}\end{center}

The Cisco R/S path is by far the most popular and could very well remain
so, but soon you'll see the Data Center path become more and more of a
focus as companies migrate to data center technologies. The Security and
Collaboration tracks also actually does provide a good job opportunity,
and an even newer one that is becoming more popular is the Industrial
CCNA. Still, understanding the foundation of R/S before attempting any
other certification track is something I highly recommend.

Even so, and as the figure shows, you only need your CCENT certification
to get underway for most of the tracks. Also, note that there are a few
other certification tracks you can go down that are not shown in the
figure, although they're not as popular as the ones shown.

\subsection{Cisco Certified Entry Network Technician (CCENT)}

Don't be fooled by the oh-so-misleading name of this first certification
because it absolutely isn't entry level! Okay---maybe entry level for
Cisco's certification path, but definitely not for someone without
experience trying to break into the highly lucrative yet challenging IT
job market! For the uninitiated, the CompTIA A+ and Network+
certifications
aren't\protect\hypertarget{f_04.xhtmlux5cux23Page_xxvii}{}{} official
prerequisites, but know that Cisco does expect you to have that type and
level of experience before embarking on your Cisco certification
journey.

All of this gets us to 2016, when the climb to Cisco supremacy just got
much harder again. The innocuous-sounding siren's call of the CCENT can
lure you to some serious trouble if you're not prepared, because it's
actually much harder than the old CCNA ever was. This will rapidly
become apparent once you start studying, but be encouraged! The fact
that the certification process is getting harder really works better for
you in the long run, because that which is harder to obtain only becomes
that much more valuable when you finally do, right? Yes, indeed!

Another important factor to keep in mind is that the Interconnection
Cisco Network Devices Part 1 (ICND1) exam, which is the required exam
for the CCENT certification, costs \$150 per attempt and it's anything
but easy to pass! The good news is that Part 1 of this book (Chapters
1-14) will guide you step-by-step in building a strong foundation in
routing and switching technologies. You really need to build on a strong
technical foundation and stay away from exam cram type books, suspicious
online material, and the like. They can help somewhat, but understand
that you'll pass the Cisco certification exams only if you have a strong
foundation and that you'll get that solid foundation only by reading as
much as you can, performing the written labs and review questions in
this book, and practicing lots and lots of hands-on labs. Additional
practice exam questions, videos, and labs are offered on my website, and
what seems like a million other sites offer additional material that can
help you study.

However, there is one way to skip the CCENT exam and still meet the
prerequisite before moving on to any other certification track, and that
path is through the CCNA R/S Composite exam. First, I'll discuss the
Interconnecting Cisco Network Devices Part 2 (ICND2) exam, and then I'll
tell you about the CCNA Composite exam, which will provide you, when
successful, with both the CCENT and the CCNA R/S certification.

\subsection{Cisco Certified Network Associate Routing and Switching
(CCNA R/S)}

Once you have achieved your CCENT certification, you can take the ICND2
(200-105) exam in order to achieve your CCNA R/S certification, which is
the most popular certification Cisco has by far because it's the most
sought-after certification of all employers.

As with the CCENT, the ICND2 exam is also \$150 per attempt---although
thinking you can just skim a book and pass any of these exams would
probably be a really expensive mistake! The CCENT/CCNA exams are
extremely hard and cover a lot of material, so you have to really know
your stuff. Taking a Cisco class or spending months with hands-on
experience is definitely a requirement to succeed when faced with this
monster!

And once you have your CCNA, you don't have to stop there---you can
choose to continue and achieve an even higher certification, called the
Cisco Certified Network Professional (CCNP). There are various ones, as
shown in Figure NaN.1. The CCNP R/S is still the most popular, with
Voice certifications coming in at a close second. And I've got to tell
you that the Data Center certification will be catching up fast. Also
good to know is\protect\hypertarget{f_04.xhtmlux5cux23Page_xxviii}{}{}
that anyone with a CCNP R/S has all the skills and knowledge needed to
attempt the notoriously dreaded but coveted CCIE R/S lab. But just
becoming a CCNA R/S can land you that job you've dreamed about and
that's what this book is all about: helping you to get and keep a great
job!

Still, why take two exams to get your CCNA if you don't have to? Cisco
still has the CCNA Composite (200-125) exam that, if passed, will land
you with your CCENT and your CCNA R/S via only one test priced at only
\$250. Some people like the one-test approach, and some people like the
two-test approach. Part 2 of this book (Chapters 15-22) covers the ICND2
exam topics.

\subsection{Why Become a CCENT and CCNA R/S?}

Cisco, like Microsoft and other vendors that provide certification, has
created the certification process to give administrators a set of skills
and to equip prospective employers with a way to measure those skills or
match certain criteria. And as you probably know, becoming a CCNA R/S is
certainly the initial, key step on a successful journey toward a new,
highly rewarding, and sustainable networking career.

The CCNA program was created to provide a solid introduction not only to
the Cisco Internetwork Operating System (IOS) and Cisco hardware but
also to internetworking in general, making it helpful to you in areas
that are not exclusively Cisco's. And regarding today's certification
process, it's not unrealistic that network managers---even those without
Cisco equipment---require Cisco certifications for their job applicants.

Rest assured that if you make it through the CCNA and are still
interested in Cisco and internetworking, you're headed down a path to
certain success!

\subsection{What Skills Do You Need to Become a CCNA R/S?}

This ICND1 exam (100-105) tests a candidate for the knowledge and skills
required to successfully install, operate, and troubleshoot a small
branch office network. The exam includes questions on the operation of
IP data networks, LAN switching technologies, IPv6, IP routing
technologies, IP services, network device security, and basic
troubleshooting. The ICND2 exam (exam 200-105) tests a candidate for the
knowledge and skills required to successfully install, operate, and
troubleshoot a small- to medium-size enterprise branch network. The exam
includes questions on LAN switching technologies, IP routing
technologies, security, troubleshooting, and WAN technologies.

\subsection{How Do You Become a CCNA R/S}

If you want to go straight for our CCNA R/S and take only one exam, all
you have to do is pass the CCNA Composite exam (200-125). Oh, but don't
you wish it were that easy? True, it's just one test, but it's a
whopper, and to pass it you must possess enough knowledge to understand
what the test writers are saying, and you need to know everything I
mentioned previously, in the sections on the ICND1 and ICND2 exams! Hey,
it's hard, but it can be done!

\protect\hypertarget{f_04.xhtmlux5cux23Page_xxix}{}{}What does the CCNA
Composite exam (200-125) cover? Pretty much the same topics covered in
the ICND1 and ICND2 exams. Candidates can prepare for this exam by
taking the Todd Lammle authorized Cisco boot camps. 200-125 tests a
candidate's knowledge and skills required to install, operate, and
troubleshoot a small- to medium-size enterprise branch network.

While you can take the Composite exam to get your CCNA, it's good to
know that Cisco offers the two-step process I discussed earlier in this
Introduction. And this book covers both those exams too! It may be
easier than taking that one ginormous exam for you, but don't think the
two-test method is easy. It takes work! However, it can be done; you
just need to stick with your studies.

The two-test method involves passing the following:

\begin{itemize}
\tightlist
\item
  Exam 100-105: Interconnecting Cisco Networking Devices Part 1 (ICND1)
\item
  Exam 200-105: Interconnecting Cisco Networking Devices Part 2 (ICND2)
\end{itemize}

I can't stress this point enough: It's critical that you have some
hands-on experience with Cisco routers. If you can get a hold of some
basic routers and switches, you're set, but if you can't, I've worked
hard to provide hundreds of configuration examples throughout this book
to help network administrators, or people who want to become network
administrators, learn the skills they need to pass the CCENT and CCNA
R/S exams.

\begin{center}\rule{0.5\linewidth}{0.5pt}\end{center}

\includegraphics{images/note.png}For Cisco certification hands-on
training with CCSI Todd Lammle, please see:
\href{http://www.lammle.com/ccna}{www.lammle.com/ccna}. Each student
will get hands-on experience by configuring at least three routers and
two switches.no sharing of equipment!

\begin{center}\rule{0.5\linewidth}{0.5pt}\end{center}

\subsection{What Does This Book Cover?}

This book covers everything you need to know to pass the ICND1 (100-105)
and ICND2 (200-105) exams, as well as the CCNA Composite (200-125) exam.
But regardless of which path you choose, as I've said, taking plenty of
time to study and practice with routers or a router simulator is the
real key to success.

You will learn the following information in this book:

\textbf{Chapter 1: Internetworking}~~ Chapters 1.14 map to the ICND1
exam. In Chapter 1, you will learn the basics of the Open Systems
Interconnection (OSI) model the way Cisco wants you to learn it. There
are written labs and plenty of review questions to help you. Do not even
think of skipping the fundamental written labs in this chapter!

\textbf{Chapter 2: Ethernet Networking and Data Encapsulation}~~ This
chapter will provide you with the Ethernet foundation you need in order
to pass both the CCENT and CCNA exams. Data encapsulation is discussed
in detail in this chapter as well. And as with the other chapters, this
chapter includes written labs and review questions to help you.

\protect\hypertarget{f_04.xhtmlux5cux23Page_xxx}{}{}\textbf{Chapter 3:
Introduction to TCP/IP}~~ This chapter provides you with the background
necessary for success on the exam, as well as in the real world with a
thorough presentation of TCP/IP. This in-depth chapter covers the very
beginnings of the Internet Protocol stack and goes all the way to IP
addressing and understanding the difference between a network address
and a broadcast address before finally ending with network
troubleshooting.

\textbf{Chapter 4: Easy Subnetting}~~ You'll actually be able to subnet
a network in your head after reading this chapter if you really want to!
And you'll find plenty of help in this chapter as long as you don't skip
the written labs and review questions at the end.

\textbf{Chapter 5: VLSMs, Summarization, and Troubleshooting TCP/IP}
Here, you'll find out all about variable length subnet masks (VLSMs) and
how to design a network using VLSMs. This chapter will finish with
summarization techniques and configurations. As with Chapter 4, plenty
of help is there for you if you don't skip the written lab and review
questions.

\textbf{Chapter 6: Cisco's Internetworking Operating System (IOS)}~~
This chapter introduces you to the Cisco Internetworking Operating
System (IOS) and command-line interface (CLI). In this chapter you'll
learn how to turn on a router and configure the basics of the IOS,
including setting passwords, banners, and more. Hands-on labs will help
you gain a firm grasp of the concepts taught in the chapter. Before you
go through the hands-on labs, be sure to complete the written lab and
review questions.

\textbf{Chapter 7: Managing a Cisco Internetwork}~~ This chapter
provides you with the management skills needed to run a Cisco IOS
network. Backing up and restoring the IOS, as well as router
configuration, are covered, as are the troubleshooting tools necessary
to keep a network up and running. As always, before tackling the
hands-on labs in this chapter, complete the written labs and review
questions.

\textbf{Chapter 8: Managing Cisco Devices}~~ This chapter describes the
boot process of Cisco routers, the configuration register, and how to
manage Cisco IOS files. The chapter finishes with a section on Cisco's
new licensing strategy for IOS. Hands-on and written labs, along with
review questions, will help you build a strong foundation for the
objectives covered in this chapter.

\textbf{Chapter 9: IP Routing}~~ This is a fun chapter because we will
begin to build our network, add IP addresses, and route data between
routers. You will also learn about static, default, and dynamic routing
using RIP and RIPv2. Hands-on labs, a written lab, and the review
questions will help you fully nail down IP routing.

\textbf{Chapter 10: Layer 2 Switching} This chapter sets you up with the
solid background you need on layer 2 switching, how switches perform
address learning and make forwarding and filtering decisions. In
addition, switch port security with MAC addresses is covered in detail.
As always, go through the hands-on labs, written lab, and review
questions to make sure you've really got layer 2 switching down!

\textbf{Chapter 11: VLANs and Inter-VLAN Routing} Here I cover virtual
VLANs and how to use them in your internetwork. This chapter covers the
nitty-gritty of VLANs and the different concepts and protocols used with
VLANs. I'll also guide you
through\protect\hypertarget{f_04.xhtmlux5cux23Page_xxxi}{}{}
troubleshooting techniques in this all-important chapter. The hands-on
labs, written lab, and review questions are there to reinforce the VLAN
material.

\textbf{Chapter 12: Security}~~ This chapter covers security and access
lists, which are created on routers to filter the network. IP standard,
extended, and named access lists are covered in detail. Written and
hands-on labs, along with review questions, will help you study for the
security and access-list portion of the Cisco exams.

\textbf{Chapter 13: Network Address Translation (NAT)}~~ New
information, commands, troubleshooting, and detailed hands-on labs will
help you nail the NAT CCENT objectives.

\textbf{Chapter 14: Internet Protocol Version 6 (IPv6)}~~ This is a fun
chapter chock-full of some great information. IPv6 is not the big, bad
scary creature that most people think it is, and it's a really important
objective on the latest exam, so study this chapter carefully---don't
just skim it. And make sure you hit those hands-on labs hard!

\textbf{Chapter 15: Enhanced Switched Technologies}~~ Chapter 15 is the
first chapter of Part 2 of this book, which maps to the ICND2 exam. This
chapter will start off with STP protocols and dive into the
fundamentals, covering the modes, as well as the various flavors of STP.
VLANs, trunks, and troubleshooting are covered as well. EtherChannel
technologies, configuration, and verification are also covered. There
are hands-on labs, a written lab, and plenty of review questions to help
you. Do not even think of skipping the fundamental written and hands-on
labs in this chapter!

\textbf{Chapter 16: Network Device Management and Security Managing
Cisco Devices}~~ This chapter describes the boot process of Cisco
routers, the configuration register, and how to manage Cisco IOS files.
The chapter finishes with a section on Cisco's new licensing strategy
for its IOS. Hands-on and written labs, along with review questions,
will help you build a strong foundation for the objectives covered in
this chapterhow to mitigate threats at the access layer using various
security techniques. AAA with RADIUIS and TACACS+, SNMP and HSRP are
also covered in this chapter. Don't skip the hands-on labs that are
included, as well as a written lab and review questions at the end of
the chapter.

\textbf{Chapter 17: Enhanced IGRP} EIGRP was not covered in the ICND1
(CCENT) chapters, so this is a full chapter on nothing but EIGRP and
EIGRPv6. There are lots of examples, including configuration,
verification, and troubleshooting labs, with both IP and with IPv6.
Great hands-on labs are included, as well as a written lab and review
questions.

\textbf{Chapter 18: Open Shortest Path First (OSPF)}~~ Chapter 9 dives
into more complex dynamic routing by covering OSPF routing. The written
lab, hands-on labs, and review questions will help you master this vital
routing protocol.

\textbf{Chapter 19: Multi-Area OSPF}~~ The ICND1 (CCENT) portion of this
book had a large chapter on OSPF, so before reading this chapter, be
sure you have the CCENT objectives down pat with a strong OSPF
foundation. This chapter will take off where that ICND1 chapter left off
and add multi-area networks along with advanced configurations and then
finish with OSPv3. Hands-on labs, a written lab, and challenging review
questions await you at the end of the chapter.

\protect\hypertarget{f_04.xhtmlux5cux23Page_xxxii}{}{}\textbf{Chapter
20: Troubleshooting IP, IPv6, and VLANs}~~ I want to say this is the
most important chapter in the book, but that's hard to say. You can
decide that yourself when you take the exam! Be sure to go through all
the troubleshooting steps for IP, IPv6, and VLANs. The hands-on labs for
this chapter will be included in the free bonus material and dynamic
labs that I'll write and change as needed. Don't skip the written lab
and review questions.

\textbf{Chapter 21: Wide Area Networks}~~ This is the longest, and last,
chapter in the book. It covers multiple protocols in depth, especially
HDLC, PPP, and Frame Relay, along with a discussion on many other
technologies. Good troubleshooting examples are provided in the PPP and
Frame Relay configuration sections, and these cannot be skipped!
Hands-on labs meant to focus squarely on the objectives are included at
the end of the chapter, as well as a written lab and challenging review
questions.

\textbf{Chapter 22: Evolution of Intelligent Networks}~~ I saved the
hardest chapter for last. What makes this chapter challenging is that
there is no configuration section to you really need to dive deep into
the cloud, APIC-EM and QoS sections with an open and ready mind. I stuck
as close to the objectives as possible in order to help you ace the
exam. The written lab and review questions are spot on for the
objectives.

\textbf{Appendix A: Answers to Written Labs}~~ This appendix contains
the answers to the book's written labs.

\textbf{Appendix B: Answers to Chapter Review Questions}~~ This appendix
provides the answers to the end-of-chapter review questions.

\textbf{Appendix C: Disabling and Configuring Network Services} Appendix
C takes a look at the basic services you should disable on your routers
to make your network less of a target for denial of service (DoS)
attacks and break-in attempts.

\begin{center}\rule{0.5\linewidth}{0.5pt}\end{center}

\includegraphics{images/tip.png}Be sure to check the announcements
section of my forum to find out how to download bonus material I created
specifically for this book.

\begin{center}\rule{0.5\linewidth}{0.5pt}\end{center}

\subsection{What's Available Online?}

I have worked hard to provide some really great tools to help you with
your certification process. All of the following tools, most of them
available at
\href{http://www.wiley.com/go/sybextestprep}{www.wiley.com/go/sybextestprep},
should be loaded on your workstation when you're studying for the test.
As a fantastic bonus, I was able to add to the download link a preview
section from my CCNA video series! Please understand that these are not
the full versions, but they're still a great value for you included free
with this book.

\textbf{Test Preparation Software}~~ The test preparation software
prepares you to pass the ICND1 and ICND2 exams and the CCNA R/S
Composite exam. You'll find all the review and assessment questions from
the book plus additional practice exam questions that appear exclusively
from the downloadable study tools.

\protect\hypertarget{f_04.xhtmlux5cux23Page_xxxiii}{}{}\textbf{Electronic
Flashcards}~~ The companion study tools include over 200 flashcards
specifically written to hit you hard, so don't get discouraged if you
don't ace your way through them at first! They're there to ensure that
you're really ready for the exam. And no worries---armed with the review
questions, practice exams, and flashcards, you'll be more than prepared
when exam day comes!

\textbf{Glossary}~~ A complete glossary of CCENT, ICND2, CCNA R/S and
Cisco routing terms is available at
\href{http://www.wiley.com/go/sybextestprep}{www.wiley.com/go/sybextestprep}.

\textbf{Todd Lammle Bonus Material and Labs}~~ Be sure to check the
announcement section of my forum at
\href{http://www.lammle.com/ccna}{www.lammle.com/ccna} for directions on
how to download all the latest bonus material created specifically to
help you study for your ICND1, ICND2, and CCNA R/S exams.

\textbf{Todd Lammle Videos}~~ I have created a full CCNA series of
videos that can be purchased at
\href{http://www.lammle.com/ccna}{www.lammle.com/ccna}

\subsection{How to Use This Book}

If you want a solid foundation for the serious effort of preparing for
the Interconnecting Cisco Network Devices Part 1 and 2 exams, or the
CCNA R/S Composite exam, then look no further. I've spent hundreds of
hours putting together this book with the sole intention of helping you
to pass the Cisco exams, as well as really learn how to correctly
configure Cisco routers and switches!

This book is loaded with valuable information, and you will get the most
out of your study time if you understand why the book is organized the
way it is.

So to maximize your benefit from this book, I recommend the following
study method:

\begin{enumerate}
\tightlist
\item
  Take the assessment test that's provided at the end of this
  introduction. (The answers are at the end of the test.) It's okay if
  you don't know any of the answers; that's why you bought this book!
  Carefully read over the explanations for any questions you get wrong
  and note the chapters in which the material relevant to them is
  covered. This information should help you plan your study strategy.
\item
  Study each chapter carefully, making sure you fully understand the
  information and the test objectives listed at the beginning of each
  one. Pay extra-close attention to any chapter that includes material
  covered in questions you missed.
\item
  Complete the written labs at the end of each chapter. (Answers to
  these appear in Appendix A.) Do not skip these written exercises
  because they directly relate to the Cisco exams and what you must
  glean from the chapters in which they appear. Do not just skim these
  labs! Make sure you completely understand the reason for each correct
  answer.
\item
  Complete all hands-on labs in each chapter, referring to the text of
  the chapter so that you understand the reason for each step you take.
  Try to get your hands on some real equipment, but if you don't have
  Cisco equipment available, try the LammleSim IOS version, which you
  can use for the hands-on labs found only in this book. These labs will
  equip you with everything you need for all your Cisco certification
  goals.
\item
  \protect\hypertarget{f_04.xhtmlux5cux23Page_xxxiv}{}{}Answer all of
  the review questions related to each chapter. (The answers appear in
  Appendix B.) Note the questions that confuse you, and study the topics
  they cover again until the concepts are crystal clear. And again---do
  not just skim these questions! Make sure you fully comprehend the
  reason for each correct answer. Remember that these will not be the
  exact questions you will find on the exam, but they're written to help
  you understand the chapter material and ultimately pass the exam!
\item
  Try your hand at the practice questions that are exclusive to this
  book. The questions can be found only at
  \href{http://www.wiley.com/go/sybextestprep}{www.wiley.com/go/sybextestprep}.
  And be sure to check out
  \href{http://www.lammle.com/ccna}{www.lammle.com/ccna} for the most
  up-to-date Cisco exam prep questions, videos, Todd Lammle boot camps,
  and more.
\item
  Test yourself using all the flashcards, which are also found on the
  download link. These are brand-new and updated flashcards to help you
  prepare for the CCNA R/S exam and a wonderful study tool!
\end{enumerate}

To learn every bit of the material covered in this book, you'll have to
apply yourself regularly, and with discipline. Try to set aside the same
time period every day to study, and select a comfortable and quiet place
to do so. I'm confident that if you work hard, you'll be surprised at
how quickly you learn this material!

If you follow these steps and really study---\emph{doing hands-on labs
every single day} in addition to using the review questions, the
practice exams, the Todd Lammle video sections, and the electronic
flashcards, as well as all the written labs---it would actually be hard
to fail the Cisco exams. But understand that studying for the Cisco
exams is a lot like getting in shape---if you do not go to the gym every
day, it's not going to happen!

\subsection{Where Do You Take the Exams?}

You may take the ICND1, ICND2, or CCNA R/S Composite or any Cisco exam
at any of the Pearson VUE authorized testing centers. For information,
check \href{http://www.vue.com}{www.vue.com} or call 877-404-EXAM
(3926).

To register for a Cisco exam, follow these steps:

\begin{enumerate}
\tightlist
\item
  Determine the number of the exam you want to take. (The ICND1 exam
  number is 100-105, ICND2 is 100-205, and CCNA R/S Composite is
  200-125.)
\item
  Register with the nearest Pearson VUE testing center. At this point,
  you will be asked to pay in advance for the exam. At the time of this
  writing, the ICND1 and ICND2 exams are \$150, and the CCNA R/S
  Composite exam is \$250. The exams must be taken within one year of
  payment. You can schedule exams up to six weeks in advance or as late
  as the day you want to take it---but if you fail a Cisco exam, you
  must wait five days before you will be allowed to retake it. If
  something comes up and you need to cancel or reschedule your exam
  appointment, contact Pearson VUE at least 24 hours in advance.
\item
  \protect\hypertarget{f_04.xhtmlux5cux23Page_xxxv}{}{}When you schedule
  the exam, you'll get instructions regarding all appointment and
  cancellation procedures, the ID requirements, and information about
  the testing-center location.
\end{enumerate}

\subsection{Tips for Taking Your Cisco Exams}

The Cisco exams contain about 40-50 questions and must be completed in
about 90 minutes or less. This information can change per exam. You must
get a score of about 85 percent to pass this exam, but again, each exam
can be different.

Many questions on the exam have answer choices that at first glance look
identical---especially the syntax questions! So remember to read through
the choices carefully because close just doesn't cut it. If you get
commands in the wrong order or forget one measly character, you'll get
the question wrong. So, to practice, do the hands-on exercises at the
end of this book's chapters over and over again until they feel natural
to you.

Also, never forget that the right answer is the Cisco answer. In many
cases, more than one appropriate answer is presented, but the
\emph{correct} answer is the one that Cisco recommends. On the exam, you
will always be told to pick one, two, or three options, never "choose
all that apply." The Cisco exam may include the following test formats:

\begin{itemize}
\tightlist
\item
  Multiple-choice single answer
\item
  Multiple-choice multiple answer
\item
  Drag-and-drop
\item
  Router simulations
\end{itemize}

Cisco proctored exams will not show the steps to follow in completing a
router interface configuration, but they do allow partial command
responses. For example, \texttt{show\ run,\ sho\ running}, or
\texttt{sh\ running-config} would be acceptable.

Here are some general tips for exam success:

\begin{itemize}
\tightlist
\item
  Arrive early at the exam center so you can relax and review your study
  materials.
\item
  Read the questions \emph{carefully}. Don't jump to conclusions. Make
  sure you're clear about exactly what each question asks. "Read twice,
  answer once," is what I always tell my students.
\item
  When answering multiple-choice questions that you're not sure about,
  use the process of elimination to get rid of the obviously incorrect
  answers first. Doing this greatly improves your odds if you need to
  make an educated guess.
\item
  You can no longer move forward and backward through the Cisco exams,
  so doublecheck your answer before clicking Next since you can't change
  your mind.
\end{itemize}

After you complete an exam, you'll get immediate, online notification of
your pass or fail status, a printed examination score report that
indicates your pass or fail status, and your exam results by section.
(The test administrator will give you the printed score report.) Test
scores are automatically forwarded to Cisco within five working days
after you take the test, so you don't need to send your score to them.
If you pass the exam, you'll receive confirmation from Cisco, typically
within two to four weeks, sometimes a bit longer.

\subsection[Objective Map for CCNA Routing and Switching Certification
Exam]{\texorpdfstring{\protect\hypertarget{f_04.xhtmlux5cux23Page_xxxvi}{}{}Objective
Map for CCNA Routing and Switching Certification
Exam}{Objective Map for CCNA Routing and Switching Certification Exam}}

We've provided this objective map to help you locate where objectives
for the CCNA Routing and Switching certification exams are covered in
each chapter. Please refer to it when you want to find an objective
quickly.

\subsubsection{ICND1 Exam Objectives}

Exam objectives are subject to change at any time without prior notice
and at Cisco's sole discretion. Please visit Cisco's certification
website
(\href{http://www.cisco.com/web/learning}{www.cisco.com/web/learning})
for the latest information on the ICND1 Exam 100-105.

{\protect\hypertarget{f_04.xhtmlux5cux23table-I-1}{}{}\textbf{Table I.1}
20\% 1.0 Network Fundamentals}

\begin{longtable}[]{@{}ll@{}}
\toprule
Objective & Chapter(s)\tabularnewline
\midrule
\endhead
1.1 Compare and contrast OSI and TCP/IP models &
\protect\hyperlink{c03.xhtml}{3}\tabularnewline
1.2 Compare and contrast TCP and UDP protocols &
\protect\hyperlink{c03.xhtml}{3}\tabularnewline
1.3 Describe the impact of infrastructure components in an enterprise
network & \protect\hyperlink{c01.xhtml}{1}\tabularnewline
1.3.a Firewalls & \protect\hyperlink{c01.xhtml}{1}\tabularnewline
1.3.b Access points & \protect\hyperlink{c01.xhtml}{1}\tabularnewline
1.3.c Wireless controllers &
\protect\hyperlink{c01.xhtml}{1}\tabularnewline
1.4 Compare and contrast collapsed core and three-tier architectures &
\protect\hyperlink{c02.xhtml}{2}\tabularnewline
1.5 Compare and contrast network topologies &
\protect\hyperlink{c01.xhtml}{1}\tabularnewline
1.5.a Star & \protect\hyperlink{c01.xhtml}{1}\tabularnewline
1.5.b Mesh & \protect\hyperlink{c01.xhtml}{1}\tabularnewline
1.5.c Hybrid & \protect\hyperlink{c01.xhtml}{1}\tabularnewline
1.6 Select the appropriate cabling type based on implementation
requirements & \protect\hyperlink{c02.xhtml}{2}\tabularnewline
1.7 Apply troubleshooting methodologies to resolve problems &
\protect\hyperlink{c03.xhtml}{3},\protect\hyperlink{c05.xhtml}{5}\tabularnewline
1.7.a Perform fault isolation and document &
\protect\hyperlink{c03.xhtml}{3},\protect\hyperlink{c05.xhtml}{5}\tabularnewline
1.7.b Resolve or escalate &
\protect\hyperlink{c03.xhtml}{3},\protect\hyperlink{c05.xhtml}{5}\tabularnewline
1.7.c Verify and monitor resolution &
\protect\hyperlink{c03.xhtml}{3},\protect\hyperlink{c05.xhtml}{5}\tabularnewline
1.8 Configure, verify, and troubleshoot IPv4 addressing and subnetting &
\protect\hyperlink{c04.xhtml}{4},\protect\hyperlink{c05.xhtml}{5}\tabularnewline
1.9 Compare and contrast IPv4 address types &
\protect\hyperlink{c03.xhtml}{3}\tabularnewline
1.9.a Unicast & \protect\hyperlink{c03.xhtml}{3}\tabularnewline
1.9.b Broadcast & \protect\hyperlink{c03.xhtml}{3}\tabularnewline
1.9.c Multicast & \protect\hyperlink{c03.xhtml}{3}\tabularnewline
\protect\hypertarget{f_04.xhtmlux5cux23Page_xxxvii}{}{}1.10 Describe the
need for private IPv4 addressing &
\protect\hyperlink{c03.xhtml}{3}\tabularnewline
1.11 Identify the appropriate IPv6 addressing scheme to satisfy
addressing requirements in a LAN/WAN environment &
\protect\hyperlink{c14.xhtml}{14}\tabularnewline
1.12 Configure, verify, and troubleshoot IPv6 addressing &
\protect\hyperlink{c14.xhtml}{14}\tabularnewline
1.13 Configure and verify IPv6 Stateless Address Auto Configuration &
\protect\hyperlink{c14.xhtml}{14}\tabularnewline
1.14 Compare and contrast IPv6 address types &
\protect\hyperlink{c14.xhtml}{14}\tabularnewline
1.14.a Global unicast & \protect\hyperlink{c14.xhtml}{14}\tabularnewline
1.14.b Unique local & \protect\hyperlink{c14.xhtml}{14}\tabularnewline
1.14.c Link local & \protect\hyperlink{c14.xhtml}{14}\tabularnewline
1.14.d Multicast & \protect\hyperlink{c14.xhtml}{14}\tabularnewline
1.14.e Modified EUI 64 &
\protect\hyperlink{c14.xhtml}{14}\tabularnewline
1.14.f Autoconfiguration &
\protect\hyperlink{c14.xhtml}{14}\tabularnewline
1.14.g Anycast & \protect\hyperlink{c14.xhtml}{14}\tabularnewline
\bottomrule
\end{longtable}

{\protect\hypertarget{f_04.xhtmlux5cux23table-I-2}{}{}\textbf{Table I.2}
26\% 2.0 LAN Switching Fundamentals}

\begin{longtable}[]{@{}ll@{}}
\toprule
Objective & Chapter(s)\tabularnewline
\midrule
\endhead
2.1 Describe and verify switching concepts &
\protect\hyperlink{c10.xhtml}{10}\tabularnewline
2.1.a MAC learning and aging &
\protect\hyperlink{c10.xhtml}{10}\tabularnewline
2.1.b Frame switching & \protect\hyperlink{c10.xhtml}{10}\tabularnewline
2.1.c Frame flooding & \protect\hyperlink{c10.xhtml}{10}\tabularnewline
2.1.d MAC address table &
\protect\hyperlink{c10.xhtml}{10}\tabularnewline
2.2 Interpret Ethernet frame format &
\protect\hyperlink{c02.xhtml}{2}\tabularnewline
2.3 Troubleshoot interface and cable issues (collisions, errors, duplex,
speed) & \protect\hyperlink{c06.xhtml}{6}\tabularnewline
2.4 Configure, verify, and troubleshoot VLANs (normal range) spanning
multiple switches & \protect\hyperlink{c11.xhtml}{11}\tabularnewline
2.4.a Access ports (data and voice) &
\protect\hyperlink{c11.xhtml}{11}\tabularnewline
2.4.b Default VLAN & \protect\hyperlink{c11.xhtml}{11}\tabularnewline
2.5 Configure, verify, and troubleshoot interswitch connectivity &
\protect\hyperlink{c11.xhtml}{11}\tabularnewline
2.5.a Trunk ports & \protect\hyperlink{c11.xhtml}{11}\tabularnewline
2.5.b 802.1Q & \protect\hyperlink{c11.xhtml}{11}\tabularnewline
2.5.c Native VLAN & \protect\hyperlink{c11.xhtml}{11}\tabularnewline
2.6 Configure and verify Layer 2 protocols &
\protect\hyperlink{c07.xhtml}{7}\tabularnewline
2.6.a Cisco Discovery Protocol &
\protect\hyperlink{c07.xhtml}{7}\tabularnewline
2.6.b LLDP & \protect\hyperlink{c07.xhtml}{7}\tabularnewline
2.7 Configure, verify, and troubleshoot port security &
\protect\hyperlink{c10.xhtml}{10}\tabularnewline
\protect\hypertarget{f_04.xhtmlux5cux23Page_xxxviii}{}{}2.7.a Static &
\protect\hyperlink{c10.xhtml}{10}\tabularnewline
2.7.b Dynamic & \protect\hyperlink{c10.xhtml}{10}\tabularnewline
2.7.c Sticky & \protect\hyperlink{c10.xhtml}{10}\tabularnewline
2.7.d Max MAC addresses &
\protect\hyperlink{c10.xhtml}{10}\tabularnewline
2.7.e Violation actions &
\protect\hyperlink{c10.xhtml}{10}\tabularnewline
2.7.f Err-disable recovery &
\protect\hyperlink{c10.xhtml}{10}\tabularnewline
\bottomrule
\end{longtable}

{\protect\hypertarget{f_04.xhtmlux5cux23table-I-3}{}{}\textbf{Table I.3}
25\% 3.0 Routing Fundamentals}

\begin{longtable}[]{@{}ll@{}}
\toprule
Objective & Chapter(s)\tabularnewline
\midrule
\endhead
3.1 Describe the routing concepts &
\protect\hyperlink{c09.xhtml}{9}\tabularnewline
3.1.a Packet handling along the path through a network &
\protect\hyperlink{c09.xhtml}{9}\tabularnewline
3.1.b Forwarding decision based on route lookup &
\protect\hyperlink{c09.xhtml}{9}\tabularnewline
3.1.c Frame rewrite & \protect\hyperlink{c09.xhtml}{9}\tabularnewline
3.2 Interpret the components of routing table &
\protect\hyperlink{c09.xhtml}{9}\tabularnewline
3.2.a Prefix & \protect\hyperlink{c09.xhtml}{9}\tabularnewline
3.2.b Network mask & \protect\hyperlink{c09.xhtml}{9}\tabularnewline
3.2.c Next hop & \protect\hyperlink{c09.xhtml}{9}\tabularnewline
3.2.d Routing protocol code &
\protect\hyperlink{c09.xhtml}{9}\tabularnewline
3.2.e Administrative distance &
\protect\hyperlink{c09.xhtml}{9}\tabularnewline
3.2.f Metric & \protect\hyperlink{c09.xhtml}{9}\tabularnewline
3.2.g Gateway of last resort &
\protect\hyperlink{c09.xhtml}{9}\tabularnewline
3.3 Describe how a routing table is populated by different routing
information sources & \protect\hyperlink{c09.xhtml}{9}\tabularnewline
3.3.a Admin distance & \protect\hyperlink{c09.xhtml}{9}\tabularnewline
3.4 Configure, verify, and troubleshoot inter-VLAN routing &
\protect\hyperlink{c11.xhtml}{11}\tabularnewline
3.4.a Router on a stick &
\protect\hyperlink{c11.xhtml}{11}\tabularnewline
3.5 Compare and contrast static routing and dynamic routing &
\protect\hyperlink{c09.xhtml}{9}\tabularnewline
3.6 Configure, verify, and troubleshoot IPv4 and IPv6 static routing &
\protect\hyperlink{c09.xhtml}{9}\tabularnewline
3.6.a Default route &
\protect\hyperlink{c09.xhtml}{9},\protect\hyperlink{c14.xhtml}{14}\tabularnewline
3.6.b Network route & \protect\hyperlink{c09.xhtml}{9}\tabularnewline
3.6.c Host route & \protect\hyperlink{c09.xhtml}{9}\tabularnewline
3.6.d Floating static & \protect\hyperlink{c09.xhtml}{9}\tabularnewline
3.7 Configure, verify, and troubleshoot RIPv2 for IPv4 (excluding
authentication, filtering, manual summarization, redistribution) &
\protect\hyperlink{c09.xhtml}{9}\tabularnewline
\bottomrule
\end{longtable}

{\protect\hypertarget{f_04.xhtmlux5cux23table-I-4}{}{}\textbf{Table I.4}
15\% 4.0 Infrastructure Services}

\begin{longtable}[]{@{}ll@{}}
\toprule
Objective & Chapter(s)\tabularnewline
\midrule
\endhead
\protect\hypertarget{f_04.xhtmlux5cux23Page_xxxix}{}{}4.1 Describe DNS
lookup operation & \protect\hyperlink{c07.xhtml}{7}\tabularnewline
4.2 Troubleshoot client connectivity issues involving DNS &
\protect\hyperlink{c07.xhtml}{7}\tabularnewline
4.3 Configure and verify DHCP on a router (excluding static
reservations) & \protect\hyperlink{c07.xhtml}{7}\tabularnewline
4.3.a Server & \protect\hyperlink{c07.xhtml}{7}\tabularnewline
4.3.b Relay & \protect\hyperlink{c07.xhtml}{7}\tabularnewline
4.3.c Client & \protect\hyperlink{c07.xhtml}{7}\tabularnewline
4.3.d TFTP, DNS, and gateway options &
\protect\hyperlink{c07.xhtml}{7}\tabularnewline
4.4 Troubleshoot client- and router-based DHCP connectivity issues &
\protect\hyperlink{c07.xhtml}{7}\tabularnewline
4.5 Configure and verify NTP operating in client/server mode &
\protect\hyperlink{c07.xhtml}{7}\tabularnewline
4.6 Configure, verify, and troubleshoot IPv4 standard numbered and named
access list for routed interfaces &
\protect\hyperlink{c12.xhtml}{12}\tabularnewline
4.7 Configure, verify, and troubleshoot inside source NAT &
\protect\hyperlink{c13.xhtml}{13}\tabularnewline
4.7.a Static & \protect\hyperlink{c13.xhtml}{13}\tabularnewline
4.7.b Pool & \protect\hyperlink{c13.xhtml}{13}\tabularnewline
4.7.c PAT & \protect\hyperlink{c13.xhtml}{13}\tabularnewline
\bottomrule
\end{longtable}

{\protect\hypertarget{f_04.xhtmlux5cux23table-I-5}{}{}\textbf{Table I.5}
14\% 5.0 Infrastructure Maintenance}

\begin{longtable}[]{@{}ll@{}}
\toprule
Objective & Chapter(s)\tabularnewline
\midrule
\endhead
5.1 Configure and verify device-monitoring using syslog &
\protect\hyperlink{c07.xhtml}{7}\tabularnewline
5.2 Configure and verify device management &
\protect\hyperlink{c07.xhtml}{7},\protect\hyperlink{c08.xhtml}{8}\tabularnewline
5.2.a Backup and restore device configuration &
\protect\hyperlink{c07.xhtml}{7}\tabularnewline
5.2.b Using Cisco Discovery Protocol and LLDP for device discovery &
\protect\hyperlink{c07.xhtml}{7}\tabularnewline
5.2.c Licensing & \protect\hyperlink{c08.xhtml}{8}\tabularnewline
5.2.d Logging & \protect\hyperlink{c07.xhtml}{7}\tabularnewline
5.2.e Timezone & \protect\hyperlink{c07.xhtml}{7}\tabularnewline
5.2.f Loopback & \protect\hyperlink{c07.xhtml}{7}\tabularnewline
5.3 Configure and verify initial device configuration &
\protect\hyperlink{c06.xhtml}{6}\tabularnewline
5.4 Configure, verify, and troubleshoot basic device hardening &
\protect\hyperlink{c06.xhtml}{6}\tabularnewline
5.4.a Local authentication &
\protect\hyperlink{c06.xhtml}{6}\tabularnewline
5.4.b Secure password & \protect\hyperlink{c06.xhtml}{6}\tabularnewline
5.4.c Access to device & \protect\hyperlink{c06.xhtml}{6}\tabularnewline
5.4.c. (i) Source address &
\protect\hyperlink{c06.xhtml}{6}\tabularnewline
5.4.c. (ii) Telnet/SSH & \protect\hyperlink{c06.xhtml}{6}\tabularnewline
5.4.d Login banner & \protect\hyperlink{c06.xhtml}{6}\tabularnewline
\protect\hypertarget{f_04.xhtmlux5cux23Page_xl}{}{}5.5 Perform device
maintenance &
\protect\hyperlink{c06.xhtml}{6},\protect\hyperlink{c08.xhtml}{8}\tabularnewline
5.5.a Cisco IOS upgrades and recovery (SCP, FTP, TFTP, and MD5 verify) &
\protect\hyperlink{c08.xhtml}{8}\tabularnewline
5.5.b Password recovery and configuration register &
\protect\hyperlink{c08.xhtml}{8}\tabularnewline
5.5.c File system management &
\protect\hyperlink{c08.xhtml}{8}\tabularnewline
5.6 Use Cisco IOS tools to troubleshoot and resolve problems &
\protect\hyperlink{c06.xhtml}{6}\tabularnewline
5.6.a Ping and traceroute with extended option &
\protect\hyperlink{c06.xhtml}{6}\tabularnewline
5.6.b Terminal monitor & \protect\hyperlink{c06.xhtml}{6}\tabularnewline
5.6.c Log events & \protect\hyperlink{c06.xhtml}{6}\tabularnewline
\bottomrule
\end{longtable}

\subsection{ICND2 Exam Objectives}

Exam objectives are subject to change at any time without prior notice
and at Cisco's sole discretion. Please visit Cisco's certification
website
(\href{http://www.cisco.com/web/learning}{www.cisco.com/web/learning})
for the latest information on the ICND2 Exam 200-105.

{\protect\hypertarget{f_04.xhtmlux5cux23table-I-6}{}{}\textbf{Table I.6}
26\% 1.0 LAN Switching Technologies}

\begin{longtable}[]{@{}ll@{}}
\toprule
Objective & Chapter(s)\tabularnewline
\midrule
\endhead
1.1 Configure, verify, and troubleshoot VLANs (normal/extended range)
spanning multiple switches &
\protect\hyperlink{c15.xhtml}{15}\tabularnewline
1.1.a Access ports (data and voice) &
\protect\hyperlink{c15.xhtml}{15}\tabularnewline
1.1.b Default VLAN & \protect\hyperlink{c15.xhtml}{15}\tabularnewline
1.2 Configure, verify, and troubleshoot interswitch connectivity &
\protect\hyperlink{c15.xhtml}{15}\tabularnewline
1.2.a Add and remove VLANs on a trunk &
\protect\hyperlink{c15.xhtml}{15}\tabularnewline
1.2.b DTP and VTP (v1\&v2) &
\protect\hyperlink{c15.xhtml}{15}\tabularnewline
1.3 Configure, verify, and troubleshoot STP protocols &
\protect\hyperlink{c15.xhtml}{15}\tabularnewline
1.3.a STP mode (PVST+ and RPVST+) &
\protect\hyperlink{c15.xhtml}{15}\tabularnewline
1.3.b STP root bridge selection &
\protect\hyperlink{c15.xhtml}{15}\tabularnewline
1.4 Configure, verify, and troubleshoot STP-related optional features &
\protect\hyperlink{c15.xhtml}{15}\tabularnewline
1.4.a PortFast & \protect\hyperlink{c15.xhtml}{15}\tabularnewline
1.4.b BPDU guard & \protect\hyperlink{c15.xhtml}{15}\tabularnewline
1.5 Configure, verify, and troubleshoot (Layer 2/Layer 3) EtherChannel &
\protect\hyperlink{c15.xhtml}{15}\tabularnewline
1.5.a Static & \protect\hyperlink{c15.xhtml}{15}\tabularnewline
1.5.b PAGP & \protect\hyperlink{c15.xhtml}{15}\tabularnewline
1.5.c LACP & \protect\hyperlink{c15.xhtml}{15}\tabularnewline
1.6 Describe the benefits of switch stacking and chassis aggregation &
\protect\hyperlink{c22.xhtml}{22}\tabularnewline
1.7 Describe common access layer threat mitigation techniques &
\protect\hyperlink{c15.xhtml}{15},\protect\hyperlink{c16.xhtml}{16},\protect\hyperlink{c20.xhtml}{20}\tabularnewline
\protect\hypertarget{f_04.xhtmlux5cux23Page_xli}{}{}1.7.a 802.1x &
\protect\hyperlink{c16.xhtml}{16}\tabularnewline
1.7.b DHCP snooping & \protect\hyperlink{c16.xhtml}{16}\tabularnewline
1.7.c Nondefault native VLAN & 15, 20\tabularnewline
\bottomrule
\end{longtable}

{\protect\hypertarget{f_04.xhtmlux5cux23table-I-7}{}{}\textbf{Table I.7}
29\% 2.0 Routing Technologies}

\begin{longtable}[]{@{}ll@{}}
\toprule
Objective & Chapter(s)\tabularnewline
\midrule
\endhead
2.1 Configure, verify, and troubleshoot Inter-VLAN routing 1 &
\protect\hyperlink{c15.xhtml}{15}\tabularnewline
2.1.a Router on a stick 1 &
\protect\hyperlink{c15.xhtml}{15}\tabularnewline
2.1.b SVI 1 & \protect\hyperlink{c15.xhtml}{15}\tabularnewline
2.2 Compare and contrast distance vector and link-state routing
protocols &
\protect\hyperlink{c17.xhtml}{17},\protect\hyperlink{c18.xhtml}{18},\protect\hyperlink{c19.xhtml}{19}\tabularnewline
2.3 Compare and contrast interior and exterior routing protocols &
\protect\hyperlink{c17.xhtml}{17},\protect\hyperlink{c18.xhtml}{18},\protect\hyperlink{c19.xhtml}{19}\tabularnewline
2.4 Configure, verify, and troubleshoot single area and multiarea OSPFv2
for IPv4 (excluding authentication, filtering, manual summarization,
redistribution, stub, virtual-link, and LSAs) &
\protect\hyperlink{c18.xhtml}{18},\protect\hyperlink{c19.xhtml}{19}\tabularnewline
2.5 Configure, verify, and troubleshoot single area and multiarea OSPFv3
for IPv6 (excluding authentication, filtering, manual summarization,
redistribution, stub, virtual-link, and LSAs) &
\protect\hyperlink{c18.xhtml}{18},
\protect\hyperlink{c19.xhtml}{19}\tabularnewline
2.6 Configure, verify, and troubleshoot EIGRP for IPv4 (excluding
authentication, filtering, manual summarization, redistribution, stub) &
\protect\hyperlink{c17.xhtml}{17}\tabularnewline
2.7 Configure, verify, and troubleshoot EIGRP for IPv6 (excluding
authentication, filtering, manual summarization, redistribution, stub) &
\protect\hyperlink{c17.xhtml}{17}\tabularnewline
\bottomrule
\end{longtable}

{\protect\hypertarget{f_04.xhtmlux5cux23table-I-8}{}{}\textbf{Table I.8}
16\% 3.0 WAN Technologies}

\begin{longtable}[]{@{}ll@{}}
\toprule
Objective & Chapter(s)\tabularnewline
\midrule
\endhead
3.1 Configure and verify PPP and MLPPP on WAN interfaces using local
authentication & \protect\hyperlink{c21.xhtml}{21}\tabularnewline
3.2 Configure, verify, and troubleshoot PPPoE client-side interfaces
using local authentication &
\protect\hyperlink{c21.xhtml}{21}\tabularnewline
3.3 Configure, verify, and troubleshoot GRE tunnel connectivity &
\protect\hyperlink{c21.xhtml}{21}\tabularnewline
3.4 Describe WAN topology options &
\protect\hyperlink{c21.xhtml}{21}\tabularnewline
3.4.a Point-to-point & \protect\hyperlink{c21.xhtml}{21}\tabularnewline
3.4.b Hub and spoke & \protect\hyperlink{c21.xhtml}{21}\tabularnewline
3.4.c Full mesh & \protect\hyperlink{c21.xhtml}{21}\tabularnewline
3.4.d Single vs dual-homed &
\protect\hyperlink{c21.xhtml}{21}\tabularnewline
3.5 Describe WAN access connectivity options &
\protect\hyperlink{c21.xhtml}{21}\tabularnewline
3.5.a MPLS & \protect\hyperlink{c21.xhtml}{21}\tabularnewline
\protect\hypertarget{f_04.xhtmlux5cux23Page_xlii}{}{}3.5.b MetroEthernet
& \protect\hyperlink{c21.xhtml}{21}\tabularnewline
3.5.c Broadband PPPoE & \protect\hyperlink{c21.xhtml}{21}\tabularnewline
3.5.d Internet VPN (DMVPN, site-to-site VPN, client VPN) &
\protect\hyperlink{c21.xhtml}{21}\tabularnewline
3.6 Configure and verify single-homed branch connectivity using eBGP
IPv4 (limited to peering and route advertisement using Network command
only) & \protect\hyperlink{c21.xhtml}{21}\tabularnewline
\bottomrule
\end{longtable}

{\protect\hypertarget{f_04.xhtmlux5cux23table-I-9}{}{}\textbf{Table I.9}
14\% 4.0 Infrastructure Services}

\begin{longtable}[]{@{}ll@{}}
\toprule
Objective & Chapter(s)\tabularnewline
\midrule
\endhead
4.1 Configure, verify, and troubleshoot basic HSRP &
\protect\hyperlink{c16.xhtml}{16}\tabularnewline
4.1.a Priority & \protect\hyperlink{c16.xhtml}{16}\tabularnewline
4.1.b Preemption & \protect\hyperlink{c16.xhtml}{16}\tabularnewline
4.1.c Version & \protect\hyperlink{c16.xhtml}{16}\tabularnewline
4.2 Describe the effects of cloud resources on enterprise network
architecture & \protect\hyperlink{c22.xhtml}{22}\tabularnewline
4.2.a Traffic path to internal and external cloud services &
\protect\hyperlink{c22.xhtml}{22}\tabularnewline
4.2.b Virtual services &
\protect\hyperlink{c22.xhtml}{22}\tabularnewline
4.2.c Basic virtual network infrastructure &
\protect\hyperlink{c22.xhtml}{22}\tabularnewline
4.3 Describe basic QoS concepts &
\protect\hyperlink{c22.xhtml}{22}\tabularnewline
4.3.a Marking & \protect\hyperlink{c22.xhtml}{22}\tabularnewline
4.3.b Device trust & \protect\hyperlink{c22.xhtml}{22}\tabularnewline
4.3.c Prioritization & \protect\hyperlink{c22.xhtml}{22}\tabularnewline
4.3.c. (i) Voice 4.3.c. (ii) Video 4.3.c. (iii) Data &
\protect\hyperlink{c22.xhtml}{22}\tabularnewline
4.3.d Shaping & \protect\hyperlink{c22.xhtml}{22}\tabularnewline
4.3.e Policing & \protect\hyperlink{c22.xhtml}{22}\tabularnewline
4.3.f Congestion management &
\protect\hyperlink{c22.xhtml}{22}\tabularnewline
4.4 Configure, verify, and troubleshoot IPv4 and IPv6 access list for
traffic filtering & \protect\hyperlink{c20.xhtml}{20}\tabularnewline
4.4.a Standard & \protect\hyperlink{c20.xhtml}{20}\tabularnewline
4.4.b Extended & \protect\hyperlink{c20.xhtml}{20}\tabularnewline
4.4.c Named & \protect\hyperlink{c20.xhtml}{20}\tabularnewline
4.5 Verify ACLs using the APIC-EM Path Trace ACL analysis tool &
\protect\hyperlink{c22.xhtml}{22}\tabularnewline
\bottomrule
\end{longtable}

{\protect\hypertarget{f_04.xhtmlux5cux23table-I-10}{}{}\textbf{Table
I.10} 15\% 5.0 Infrastructure Maintenance}

\begin{longtable}[]{@{}ll@{}}
\toprule
Objective & Chapter(s)\tabularnewline
\midrule
\endhead
\protect\hypertarget{f_04.xhtmlux5cux23Page_xliii}{}{}5.1 Configure and
verify device-monitoring protocols &
\protect\hyperlink{c16.xhtml}{16}\tabularnewline
5.1.a SNMPv2 & \protect\hyperlink{c16.xhtml}{16}\tabularnewline
5.1.b SNMPv3 & \protect\hyperlink{c16.xhtml}{16}\tabularnewline
5.2 Troubleshoot network connectivity issues using ICMP echo-based IP
SLA & \protect\hyperlink{c20.xhtml}{20}\tabularnewline
5.3 Use local SPAN to troubleshoot and resolve problems &
\protect\hyperlink{c20.xhtml}{20}\tabularnewline
5.4 Describe device management using AAA with TACACS+ and RADIUS &
\protect\hyperlink{c16.xhtml}{16}\tabularnewline
5.5 Describe network programmability in enterprise network architecture
& \protect\hyperlink{c22.xhtml}{22}\tabularnewline
5.5.a Function of a controller &
\protect\hyperlink{c22.xhtml}{22}\tabularnewline
5.5.b Separation of control plane and data plane &
\protect\hyperlink{c22.xhtml}{22}\tabularnewline
5.5.c Northbound and southbound APIs &
\protect\hyperlink{c22.xhtml}{22}\tabularnewline
5.6 Troubleshoot basic Layer 3 end-to-end connectivity issues &
\protect\hyperlink{c22.xhtml}{22}\tabularnewline
\bottomrule
\end{longtable}

\subsection{CCNA Exam Objectives (Composite Exam)}

Exam objectives are subject to change at any time without prior notice
and at Cisco's sole discretion. Please visit Cisco's certification
website
(\href{http://www.cisco.com/web/learning}{www.cisco.com/web/learning})
for the latest information on the CCNA Exam 200-125.

{\protect\hypertarget{f_04.xhtmlux5cux23table-I-11}{}{}\textbf{Table
I.11} 15\% 1.0 Network Fundamentals}

\begin{longtable}[]{@{}ll@{}}
\toprule
Objective & Chapter(s)\tabularnewline
\midrule
\endhead
1.1 Compare and contrast OSI and TCP/IP models &
\protect\hyperlink{c03.xhtml}{3}\tabularnewline
1.2 Compare and contrast TCP and UDP protocols &
\protect\hyperlink{c03.xhtml}{3}\tabularnewline
1.3 Describe the impact of infrastructure components in an enterprise
network & \protect\hyperlink{c01.xhtml}{1}\tabularnewline
1.3.a Firewalls & \protect\hyperlink{c01.xhtml}{1}\tabularnewline
1.3.b Access points & \protect\hyperlink{c01.xhtml}{1}\tabularnewline
1.3.c Wireless controllers &
\protect\hyperlink{c01.xhtml}{1}\tabularnewline
1.4 Describe the effects of cloud resources on enterprise network
architecture & \protect\hyperlink{c22.xhtml}{22}\tabularnewline
1.4.a Traffic path to internal and external cloud services &
\protect\hyperlink{c22.xhtml}{22}\tabularnewline
1.4.b Virtual services &
\protect\hyperlink{c22.xhtml}{22}\tabularnewline
1.4.c Basic virtual network infrastructure &
\protect\hyperlink{c22.xhtml}{22}\tabularnewline
1.5 Compare and contrast collapsed core and three-tier architectures &
\protect\hyperlink{c02.xhtml}{2}\tabularnewline
1.6 Compare and contrast network topologies &
\protect\hyperlink{c01.xhtml}{1}\tabularnewline
1.6.a Star & \protect\hyperlink{c01.xhtml}{1}\tabularnewline
1.6.b Mesh & \protect\hyperlink{c01.xhtml}{1}\tabularnewline
\protect\hypertarget{f_04.xhtmlux5cux23Page_xliv}{}{}1.6.c Hybrid &
\protect\hyperlink{c01.xhtml}{1}\tabularnewline
1.7 Select the appropriate cabling type based on implementation
requirements & \protect\hyperlink{c02.xhtml}{2}\tabularnewline
1.8 Apply troubleshooting methodologies to resolve problems &
\protect\hyperlink{c03.xhtml}{3},\protect\hyperlink{c05.xhtml}{5}\tabularnewline
1.8.a Perform and document fault isolation &
\protect\hyperlink{c03.xhtml}{3},\protect\hyperlink{c05.xhtml}{5}\tabularnewline
1.8.b Resolve or escalate &
\protect\hyperlink{c03.xhtml}{3},\protect\hyperlink{c05.xhtml}{5}\tabularnewline
1.8.c Verify and monitor resolution &
\protect\hyperlink{c03.xhtml}{3},\protect\hyperlink{c05.xhtml}{5}\tabularnewline
1.9 Configure, verify, and troubleshoot IPv4 addressing and subnetting &
\protect\hyperlink{c04.xhtml}{4},\protect\hyperlink{c05.xhtml}{5}\tabularnewline
1.10 Compare and contrast IPv4 address types &
\protect\hyperlink{c03.xhtml}{3}\tabularnewline
1.10.a Unicast & \protect\hyperlink{c03.xhtml}{3}\tabularnewline
1.10.b Broadcast & \protect\hyperlink{c03.xhtml}{3}\tabularnewline
1.10.c Multicast & \protect\hyperlink{c03.xhtml}{3}\tabularnewline
1.11 Describe the need for private IPv4 addressing &
\protect\hyperlink{c03.xhtml}{3}\tabularnewline
1.12 Identify the appropriate IPv6 addressing scheme to satisfy
addressing requirements in a LAN/WAN environment &
\protect\hyperlink{c14.xhtml}{14}\tabularnewline
1.13 Configure, verify, and troubleshoot IPv6 addressing &
\protect\hyperlink{c14.xhtml}{14}\tabularnewline
1.14 Configure and verify IPv6 Stateless Address Auto Configuration &
\protect\hyperlink{c14.xhtml}{14}\tabularnewline
1.15 Compare and contrast IPv6 address types &
\protect\hyperlink{c14.xhtml}{14}\tabularnewline
1.15.a Global unicast & \protect\hyperlink{c14.xhtml}{14}\tabularnewline
1.15.b Unique local & \protect\hyperlink{c14.xhtml}{14}\tabularnewline
1.15.c Link local & \protect\hyperlink{c14.xhtml}{14}\tabularnewline
1.15.d Multicast & \protect\hyperlink{c14.xhtml}{14}\tabularnewline
1.15.e Modified EUI 64 &
\protect\hyperlink{c14.xhtml}{14}\tabularnewline
1.15.f Autoconfiguration &
\protect\hyperlink{c14.xhtml}{14}\tabularnewline
1.15.g Anycast & \protect\hyperlink{c14.xhtml}{14}\tabularnewline
\bottomrule
\end{longtable}

{\protect\hypertarget{f_04.xhtmlux5cux23table-I-12}{}{}\textbf{Table
I.12} 21\% 2.0 LAN Switching Technologies}

\begin{longtable}[]{@{}ll@{}}
\toprule
Objective & Chapter(s)\tabularnewline
\midrule
\endhead
2.1 Describe and verify switching concepts &
\protect\hyperlink{c10.xhtml}{10}\tabularnewline
2.1.a MAC learning and aging &
\protect\hyperlink{c10.xhtml}{10}\tabularnewline
2.1.b Frame switching & \protect\hyperlink{c10.xhtml}{10}\tabularnewline
2.1.c Frame flooding & \protect\hyperlink{c10.xhtml}{10}\tabularnewline
2.1.d MAC address table &
\protect\hyperlink{c10.xhtml}{10}\tabularnewline
2.2 Interpret Ethernet frame format &
\protect\hyperlink{c02.xhtml}{2}\tabularnewline
2.3 Troubleshoot interface and cable issues (collisions, errors, duplex,
speed) & \protect\hyperlink{c06.xhtml}{6}\tabularnewline
2.4 Configure, verify, and troubleshoot VLANs (normal/extended range)
spanning multiple switches &
\protect\hyperlink{c11.xhtml}{11}\tabularnewline
\protect\hypertarget{f_04.xhtmlux5cux23Page_xlv}{}{}2.4.a Access ports
(data and voice) & \protect\hyperlink{c11.xhtml}{11}\tabularnewline
2.4.b Default VLAN & \protect\hyperlink{c11.xhtml}{11}\tabularnewline
2.5 Configure, verify, and troubleshoot interswitch connectivity &
\protect\hyperlink{c11.xhtml}{11}\tabularnewline
2.5.a Trunk ports & \protect\hyperlink{c11.xhtml}{11}\tabularnewline
2.5.b Add and remove VLANs on a trunk &
\protect\hyperlink{c15.xhtml}{15}\tabularnewline
2.5.c DTP, VTP (v1\&v2), and 802.1Q &
\protect\hyperlink{c15.xhtml}{15}\tabularnewline
2.5.d Native VLAN & \protect\hyperlink{c11.xhtml}{11}\tabularnewline
2.6 Configure, verify, and troubleshoot STP protocols &
\protect\hyperlink{c15.xhtml}{15}\tabularnewline
2.6.a STP mode (PVST+ and RPVST+) &
\protect\hyperlink{c15.xhtml}{15}\tabularnewline
2.6.b STP root bridge selection &
\protect\hyperlink{c15.xhtml}{15}\tabularnewline
2.7 Configure, verify and troubleshoot STP related optional features &
\protect\hyperlink{c15.xhtml}{15}\tabularnewline
2.7.a PortFast & \protect\hyperlink{c15.xhtml}{15}\tabularnewline
2.7.b BPDU guard & \protect\hyperlink{c15.xhtml}{15}\tabularnewline
2.8 Configure and verify Layer 2 protocols &
\protect\hyperlink{c07.xhtml}{7}\tabularnewline
2.8.a Cisco Discovery Protocol &
\protect\hyperlink{c07.xhtml}{7}\tabularnewline
2.8.b LLDP & \protect\hyperlink{c07.xhtml}{7}\tabularnewline
2.9 Configure, verify, and troubleshoot (Layer 2/Layer 3) EtherChannel &
\protect\hyperlink{c15.xhtml}{15}\tabularnewline
2.9.a Static & \protect\hyperlink{c15.xhtml}{15}\tabularnewline
2.9.b PAGP & \protect\hyperlink{c15.xhtml}{15}\tabularnewline
2.9.c LACP & \protect\hyperlink{c15.xhtml}{15}\tabularnewline
2.10 Describe the benefits of switch stacking and chassis aggregation &
\protect\hyperlink{c22.xhtml}{22}\tabularnewline
\bottomrule
\end{longtable}

{\protect\hypertarget{f_04.xhtmlux5cux23table-I-13}{}{}\textbf{Table
I.13} 23\% 3.0 Routing Technologies}

\begin{longtable}[]{@{}ll@{}}
\toprule
Objective & Chapter(s)\tabularnewline
\midrule
\endhead
3.1 Describe the routing concepts &
\protect\hyperlink{c09.xhtml}{9}\tabularnewline
3.1.a Packet handling along the path through a network &
\protect\hyperlink{c09.xhtml}{9}\tabularnewline
3.1.b Forwarding decision based on route lookup &
\protect\hyperlink{c09.xhtml}{9}\tabularnewline
3.1.c Frame rewrite & \protect\hyperlink{c09.xhtml}{9}\tabularnewline
3.2 Interpret the components of a routing table &
\protect\hyperlink{c09.xhtml}{9}\tabularnewline
3.2.a Prefix & \protect\hyperlink{c09.xhtml}{9}\tabularnewline
3.2.b Network mask & \protect\hyperlink{c09.xhtml}{9}\tabularnewline
3.2.c Next hop & \protect\hyperlink{c09.xhtml}{9}\tabularnewline
3.2.d Routing protocol code &
\protect\hyperlink{c09.xhtml}{9}\tabularnewline
3.2.e Administrative distance &
\protect\hyperlink{c09.xhtml}{9}\tabularnewline
3.2.f Metric & \protect\hyperlink{c09.xhtml}{9}\tabularnewline
\protect\hypertarget{f_04.xhtmlux5cux23Page_xlvi}{}{}3.2.g Gateway of
last resort & \protect\hyperlink{c09.xhtml}{9}\tabularnewline
3.3 Describe how a routing table is populated by different routing
information sources & \protect\hyperlink{c09.xhtml}{9}\tabularnewline
3.3.a Admin distance & \protect\hyperlink{c09.xhtml}{9}\tabularnewline
3.4 Configure, verify, and troubleshoot inter-VLAN routing &
\protect\hyperlink{c11.xhtml}{11},\protect\hyperlink{c15.xhtml}{15}\tabularnewline
3.4.a Router on a stick &
\protect\hyperlink{c11.xhtml}{11},\protect\hyperlink{c15.xhtml}{15}\tabularnewline
3.4.b SVI & \protect\hyperlink{c15.xhtml}{15}\tabularnewline
3.5 Compare and contrast static routing and dynamic routing &
\protect\hyperlink{c09.xhtml}{9}\tabularnewline
3.6 Compare and contrast distance vector and link state routing
protocols &
\protect\hyperlink{c17.xhtml}{17},\protect\hyperlink{c18.xhtml}{18},\protect\hyperlink{c19.xhtml}{19}\tabularnewline
3.7 Compare and contrast interior and exterior routing protocols &
\protect\hyperlink{c18.xhtml}{18},\protect\hyperlink{c19.xhtml}{19}\tabularnewline
3.8 Configure, verify, and troubleshoot IPv4 and IPv6 static routing &
\protect\hyperlink{c09.xhtml}{9}\tabularnewline
3.8.a Default route &
\protect\hyperlink{c09.xhtml}{9},\protect\hyperlink{c14.xhtml}{14}\tabularnewline
3.8.b Network route & \protect\hyperlink{c09.xhtml}{9}\tabularnewline
3.8.c Host route & \protect\hyperlink{c09.xhtml}{9}\tabularnewline
3.8.d Floating static & \protect\hyperlink{c09.xhtml}{9}\tabularnewline
3.9 Configure, verify, and troubleshoot single area and multi-area
OSPFv2 for IPv4 (excluding authentication, filtering, manual
summarization, redistribution, stub, virtual-link, and LSAs) &
\protect\hyperlink{c04.xhtml}{4},\protect\hyperlink{c05.xhtml}{5}\tabularnewline
3.10 Configure, verify, and troubleshoot single area and multi-area
OSPFv3 for IPv6 (excluding authentication, filtering, manual
summarization, redistribution, stub, virtual-link, and LSAs) &
\protect\hyperlink{c04.xhtml}{4},
\protect\hyperlink{c05.xhtml}{5}\tabularnewline
3.11 Configure, verify, and troubleshoot EIGRP for IPv4 (excluding
authentication, filtering, manual summarization, redistribution, stub) &
\protect\hyperlink{c03.xhtml}{3}\tabularnewline
3.12 Configure, verify, and troubleshoot EIGRP for IPv6 (excluding
authentication, filtering, manual summarization, redistribution, stub) &
\protect\hyperlink{c03.xhtml}{3}\tabularnewline
3.13 Configure, verify, and troubleshoot RIPv2 for IPv4 (excluding
authentication, filtering, manual summarization, redistribution) &
\protect\hyperlink{c09.xhtml}{9}\tabularnewline
3.14 Troubleshoot basic Layer 3 end-to-end connectivity issues &
\protect\hyperlink{c07.xhtml}{7}\tabularnewline
\bottomrule
\end{longtable}

{\protect\hypertarget{f_04.xhtmlux5cux23table-I-14}{}{}\textbf{Table
I.14} 10\% 4.0 WAN Technologies}

\begin{longtable}[]{@{}ll@{}}
\toprule
Objective & Chapter(s)\tabularnewline
\midrule
\endhead
4.1 Configure and verify PPP and MLPPP on WAN interfaces using local
authentication & \protect\hyperlink{c21.xhtml}{21}\tabularnewline
4.2 Configure, verify, and troubleshoot PPPoE client-side interfaces
using local authentication &
\protect\hyperlink{c21.xhtml}{21}\tabularnewline
4.3 Configure, verify, and troubleshoot GRE tunnel connectivity &
\protect\hyperlink{c21.xhtml}{21}\tabularnewline
4.4 Describe WAN topology options &
\protect\hyperlink{c21.xhtml}{21}\tabularnewline
4.4.a Point-to-point & \protect\hyperlink{c21.xhtml}{21}\tabularnewline
\protect\hypertarget{f_04.xhtmlux5cux23Page_xlvii}{}{}4.4.b Hub and
spoke & \protect\hyperlink{c21.xhtml}{21}\tabularnewline
4.4.c Full mesh & \protect\hyperlink{c21.xhtml}{21}\tabularnewline
4.4.d Single vs dual-homed &
\protect\hyperlink{c21.xhtml}{21}\tabularnewline
4.5 Describe WAN access connectivity options &
\protect\hyperlink{c21.xhtml}{21}\tabularnewline
4.5.a MPLS & \protect\hyperlink{c21.xhtml}{21}\tabularnewline
4.5.b Metro Ethernet & \protect\hyperlink{c21.xhtml}{21}\tabularnewline
4.5.c Broadband PPPoE & \protect\hyperlink{c21.xhtml}{21}\tabularnewline
4.5.d Internet VPN (DMVPN, site-to-site VPN, client VPN) &
\protect\hyperlink{c21.xhtml}{21}\tabularnewline
4.6 Configure and verify single-homed branch connectivity using eBGP
IPv4 (limited to peering and route advertisement using Network command
only) & \protect\hyperlink{c21.xhtml}{21}\tabularnewline
4.7 Describe basic QoS concepts &
\protect\hyperlink{c22.xhtml}{22}\tabularnewline
4.7.a Marking & \protect\hyperlink{c22.xhtml}{22}\tabularnewline
4.7.b Device trust & \protect\hyperlink{c22.xhtml}{22}\tabularnewline
4.7.c Prioritization & \protect\hyperlink{c22.xhtml}{22}\tabularnewline
4.7.c. (i) Voice & \protect\hyperlink{c22.xhtml}{22}\tabularnewline
4.7.c. (ii) Video & \protect\hyperlink{c22.xhtml}{22}\tabularnewline
4.7.c. (iii) Data & \protect\hyperlink{c22.xhtml}{22}\tabularnewline
4.7.d Shaping & \protect\hyperlink{c22.xhtml}{22}\tabularnewline
4.7.e Policing & \protect\hyperlink{c22.xhtml}{22}\tabularnewline
4.7.f Congestion management &
\protect\hyperlink{c22.xhtml}{22}\tabularnewline
\bottomrule
\end{longtable}

{\protect\hypertarget{f_04.xhtmlux5cux23table-I-15}{}{}\textbf{Table
I.15} 10\% 5.0 Infrastructure Services}

\begin{longtable}[]{@{}ll@{}}
\toprule
Objective & Chapter(s)\tabularnewline
\midrule
\endhead
5.1 Describe DNS lookup operation &
\protect\hyperlink{c07.xhtml}{7}\tabularnewline
5.2 Troubleshoot client connectivity issues involving DNS &
\protect\hyperlink{c07.xhtml}{7}\tabularnewline
5.3 Configure and verify DHCP on a router (excluding static
reservations) & \protect\hyperlink{c07.xhtml}{7}\tabularnewline
5.3.a Server & \protect\hyperlink{c07.xhtml}{7}\tabularnewline
5.3.b Relay & \protect\hyperlink{c07.xhtml}{7}\tabularnewline
5.3.c Client & \protect\hyperlink{c07.xhtml}{7}\tabularnewline
5.3.d TFTP, DNS, and gateway options &
\protect\hyperlink{c07.xhtml}{7}\tabularnewline
5.4 Troubleshoot client- and router-based DHCP connectivity issues &
\protect\hyperlink{c07.xhtml}{7}\tabularnewline
5.5 Configure, verify, and troubleshoot basic HSRP &
\protect\hyperlink{c16.xhtml}{16}\tabularnewline
5.5.a Priority & \protect\hyperlink{c16.xhtml}{16}\tabularnewline
5.5.b Preemption & \protect\hyperlink{c16.xhtml}{16}\tabularnewline
5.5.c Version & \protect\hyperlink{c16.xhtml}{16}\tabularnewline
\protect\hypertarget{f_04.xhtmlux5cux23Page_xlviii}{}{}5.6 Configure,
verify, and troubleshoot inside source NAT &
\protect\hyperlink{c13.xhtml}{13}\tabularnewline
5.6.a Static & \protect\hyperlink{c13.xhtml}{13}\tabularnewline
5.6.b Pool & \protect\hyperlink{c13.xhtml}{13}\tabularnewline
5.6.c PAT & \protect\hyperlink{c13.xhtml}{13}\tabularnewline
5.7 Configure and verify NTP operating in a client/server mode &
\protect\hyperlink{c07.xhtml}{7}\tabularnewline
\bottomrule
\end{longtable}

{\protect\hypertarget{f_04.xhtmlux5cux23table-I-16}{}{}\textbf{Table
I.16} 11\% 6.0 Infrastructure Security}

\begin{longtable}[]{@{}ll@{}}
\toprule
Objective & Chapter(s)\tabularnewline
\midrule
\endhead
6.1 Configure, verify, and troubleshoot port security &
\protect\hyperlink{c10.xhtml}{10}\tabularnewline
6.1.a Static & \protect\hyperlink{c10.xhtml}{10}\tabularnewline
6.1.b Dynamic & \protect\hyperlink{c10.xhtml}{10}\tabularnewline
6.1.c Sticky & \protect\hyperlink{c10.xhtml}{10}\tabularnewline
6.1.d Max MAC addresses &
\protect\hyperlink{c10.xhtml}{10}\tabularnewline
6.1.e Violation actions &
\protect\hyperlink{c10.xhtml}{10}\tabularnewline
6.1.f Err-disable recovery &
\protect\hyperlink{c10.xhtml}{10}\tabularnewline
6.2 Describe common access layer threat mitigation techniques &
\protect\hyperlink{c15.xhtml}{15},\protect\hyperlink{c16.xhtml}{16},\protect\hyperlink{c20.xhtml}{20}\tabularnewline
6.2.a 802.1x & \protect\hyperlink{c16.xhtml}{16}\tabularnewline
6.2.b DHCP snooping & \protect\hyperlink{c16.xhtml}{16}\tabularnewline
6.2.c Nondefault native VLAN &
\protect\hyperlink{c15.xhtml}{15},\protect\hyperlink{c20.xhtml}{20}\tabularnewline
6.3 Configure, verify, and troubleshoot IPv4 and IPv6 access list for
traffic filtering & \protect\hyperlink{c20.xhtml}{20}\tabularnewline
6.3.a Standard & \protect\hyperlink{c20.xhtml}{20}\tabularnewline
6.3.b Extended & \protect\hyperlink{c20.xhtml}{20}\tabularnewline
6.3.c Named & \protect\hyperlink{c20.xhtml}{20}\tabularnewline
6.4 Verify ACLs using the APIC-EM Path Trace ACL Analysis tool &
\protect\hyperlink{c22.xhtml}{22}\tabularnewline
6.5 Configure, verify, and troubleshoot basic device hardening &
\protect\hyperlink{c06.xhtml}{6}\tabularnewline
6.5.a Local authentication &
\protect\hyperlink{c06.xhtml}{6}\tabularnewline
6.5.b Secure password & \protect\hyperlink{c06.xhtml}{6}\tabularnewline
6.5.c Access to device & \protect\hyperlink{c06.xhtml}{6}\tabularnewline
6.5.c. (i) Source address &
\protect\hyperlink{c06.xhtml}{6}\tabularnewline
6.5.c. (ii) Telnet/SSH & \protect\hyperlink{c06.xhtml}{6}\tabularnewline
6.5.d Login banner & \protect\hyperlink{c06.xhtml}{6}\tabularnewline
6.6 Describe device security using AAA with TACACS+ and RADIUS &
\protect\hyperlink{c16.xhtml}{16}\tabularnewline
\bottomrule
\end{longtable}

{\protect\hypertarget{f_04.xhtmlux5cux23table-I-17}{}{}\protect\hypertarget{f_04.xhtmlux5cux23Page_xlix}{}{}\textbf{Table
I.17} 10\% 7.0 Infrastructure Management}

\begin{longtable}[]{@{}ll@{}}
\toprule
Objective & Chapter(s)\tabularnewline
\midrule
\endhead
7.1 Configure and verify device-monitoring protocols &
\protect\hyperlink{c16.xhtml}{16}\tabularnewline
7.1.a SNMPv2 & \protect\hyperlink{c16.xhtml}{16}\tabularnewline
7.1.b SNMPv3 & \protect\hyperlink{c16.xhtml}{16}\tabularnewline
7.1.c Syslog &
\protect\hyperlink{c07.xhtml}{7},\protect\hyperlink{c16.xhtml}{16}\tabularnewline
7.2 Troubleshoot network connectivity issues using ICMP echo-based IP
SLA & \protect\hyperlink{c20.xhtml}{20}\tabularnewline
7.3 Configure and verify device management &
\protect\hyperlink{c07.xhtml}{7},\protect\hyperlink{c08.xhtml}{8}\tabularnewline
7.3.a Backup and restore device configuration &
\protect\hyperlink{c07.xhtml}{7}\tabularnewline
7.3.b Using Cisco Discovery Protocol or LLDP for device discovery &
\protect\hyperlink{c07.xhtml}{7}\tabularnewline
7.3.c Licensing & \protect\hyperlink{c08.xhtml}{8}\tabularnewline
7.3.d Logging & \protect\hyperlink{c07.xhtml}{7}\tabularnewline
7.3.e Timezone & \protect\hyperlink{c07.xhtml}{7}\tabularnewline
7.3.f Loopback & \protect\hyperlink{c07.xhtml}{7}\tabularnewline
7.4 Configure and verify initial device configuration &
\protect\hyperlink{c06.xhtml}{6}\tabularnewline
7.5 Perform device maintenance &
\protect\hyperlink{c06.xhtml}{6},\protect\hyperlink{c08.xhtml}{8}\tabularnewline
7.5.a Cisco IOS upgrades and recovery (SCP, FTP, TFTP, and MD5 verify) &
\protect\hyperlink{c08.xhtml}{8}\tabularnewline
7.5.b Password recovery and configuration register &
\protect\hyperlink{c08.xhtml}{8}\tabularnewline
7.5.c File system management &
\protect\hyperlink{c08.xhtml}{8}\tabularnewline
7.6 Use Cisco IOS tools to troubleshoot and resolve problems &
\protect\hyperlink{c06.xhtml}{6}\tabularnewline
7.6.a Ping and traceroute with extended option &
\protect\hyperlink{c06.xhtml}{6}\tabularnewline
7.6.b Terminal monitor & \protect\hyperlink{c06.xhtml}{6}\tabularnewline
7.6.c Log events & \protect\hyperlink{c06.xhtml}{6}\tabularnewline
7.6.d Local SPAN &
\protect\hyperlink{c06.xhtml}{6},\protect\hyperlink{c20.xhtml}{20}\tabularnewline
7.7 Describe network programmability in enterprise network architecture
& \protect\hyperlink{c22.xhtml}{22}\tabularnewline
7.7.a Function of a controller &
\protect\hyperlink{c22.xhtml}{22}\tabularnewline
7.7.b Separation of control plane and data plane &
\protect\hyperlink{c22.xhtml}{22}\tabularnewline
7.7.c Northbound and southbound APIs &
\protect\hyperlink{c22.xhtml}{22}\tabularnewline
\bottomrule
\end{longtable}

*******************

\subsection[Assessment
Test]{\texorpdfstring{\protect\hypertarget{f_04.xhtmlux5cux23Page_l}{}{}Assessment
Test}{Assessment Test}}

\begin{enumerate}
\item
  What is the \texttt{sys-id-ext} field in a BPDU used for?

  \begin{itemize}
  \tightlist
  \item
    It is a 4-bit field inserted into an Ethernet frame to define
    trunking information between switches.
  \item
    It is a 12-bit field inserted into an Ethernet frame to define VLANs
    in an STP instance.
  \item
    It is a 4-bit field inserted into an non-Ethernet frame to define
    EtherChannel options.
  \item
    It is a 12-bit field inserted into an Ethernet frame to define STP
    root bridges.
  \end{itemize}
\item
  You have four RSTP PVST+ links between switches and want to aggregate
  the bandwidth. What solution will you use?

  \begin{itemize}
  \tightlist
  \item
    EtherChannel
  \item
    PortFast
  \item
    BPDU Channel
  \item
    VLANs
  \item
    EtherBundle
  \end{itemize}
\item
  What configuration parameters must be configured the same between
  switches for LACP to form a channel? (Choose three.)

  \begin{itemize}
  \tightlist
  \item
    Virtual MAC address
  \item
    Port speeds
  \item
    Duplex
  \item
    PortFast enabled
  \item
    Allowed VLAN information
  \end{itemize}
\item
  You reload a router with a configuration register setting of 0x2101.
  What will the router do when it reloads?

  \begin{itemize}
  \tightlist
  \item
    The router enters setup mode.
  \item
    The router enters ROM monitor mode.
  \item
    The router boots the mini-IOS in ROM.
  \item
    The router expands the first IOS in flash memory into RAM.
  \end{itemize}
\item
  Which of the following commands provides the product ID and serial
  number of a router?

  \begin{itemize}
  \tightlist
  \item
    \texttt{show\ license}
  \item
    \texttt{show\ license\ feature}
  \item
    \texttt{show\ version}
  \item
    \texttt{show\ license\ udi}
  \end{itemize}
\item
  \protect\hypertarget{f_04.xhtmlux5cux23Page_li}{}{}Which command
  allows you to view the technology options and licenses that are
  supported on your router along with several status variables?

  \begin{itemize}
  \tightlist
  \item
    \texttt{show\ license}
  \item
    \texttt{show\ license\ feature}
  \item
    \texttt{show\ license\ udi}
  \item
    \texttt{show\ version}
  \end{itemize}
\item
  Which of the following services provide the operating system and the
  network?

  \begin{itemize}
  \tightlist
  \item
    IaaS
  \item
    PaaS
  \item
    SaaS
  \item
    none of the above
  \end{itemize}
\item
  You want to send a console message to a syslog server, but you only
  want to send status messages of 3 and lower. Which of the following
  commands will you use?

  \begin{itemize}
  \tightlist
  \item
    \texttt{logging\ trap\ emergencies}
  \item
    \texttt{logging\ trap\ errors}
  \item
    \texttt{logging\ trap\ debugging}
  \item
    \texttt{logging\ trap\ notifications}
  \item
    \texttt{logging\ trap\ critical}
  \item
    \texttt{logging\ trap\ warnings}
  \item
    \texttt{logging\ trap\ alerts}
  \end{itemize}
\item
  When stacking switches, which is true? (Choose 2)

  \begin{itemize}
  \tightlist
  \item
    The stack is managed as multiple objects, and has a single
    management IP address
  \item
    The stack is managed as a single object, and has a single management
    IP address
  \item
    The master switch is chosen when you configure the first switches
    master algorithm to on
  \item
    The master switch is elected form one of the stack member switches
  \end{itemize}
\item
  You need to connect to a remote IPv6 server in your virtual server
  farm. You can connect to the IPv4 servers, but not the critical IPv6
  server you desperately need. Based on the following output, what could
  your problem be?

\begin{verbatim}
C:\>ipconfig
Connection-specific DNS Suffix . : localdomain
IPv6 Address. . . . . . . . . . . : 2001:db8:3c4d:3:ac3b:2ef:1823:8938
Temporary IPv6 Address. . . . . . : 2001:db8:3c4d:3:2f33:44dd:211:1c3d
Link-local IPv6 Address . . . . . : fe80::ac3b:2ef:1823:8938%11
IPv4 Address. . . . . . . . . . . : 10.1.1.10
Subnet Mask . . . . . . . . . . . : 255.255.255.0
Default Gateway . . . . . . . . . : 10.1.1.1
\end{verbatim}

  \begin{itemize}
  \tightlist
  \item
    The global address is in the wrong subnet.
  \item
    The IPv6 default gateway has not been configured or received from
    the router.
  \item
    \protect\hypertarget{f_04.xhtmlux5cux23Page_lii}{}{}The link-local
    address has not been resolved so the host cannot communicate to the
    router.
  \item
    There are two IPv6 global addresses configured. One must be removed
    from the configuration.
  \end{itemize}
\item
  What command is used to view the IPv6-to-MAC-address resolution table
  on a Cisco router?

  \begin{itemize}
  \tightlist
  \item
    \texttt{show\ ip\ arp}
  \item
    \texttt{show\ ipv6\ arp}
  \item
    \texttt{show\ ip\ neighbors}
  \item
    \texttt{show\ ipv6\ neighbors}
  \item
    \texttt{show\ arp}
  \end{itemize}
\item
  An IPv6 ARP entry is listed as with a status of REACH. What can you
  conclude about the IPv6-to-MAC-address mapping?

  \begin{itemize}
  \tightlist
  \item
    The interface has communicated with the neighbor address and the
    mapping is current.
  \item
    The interface has not communicated within the neighbor reachable
    time frame.
  \item
    The ARP entry has timed out.
  \item
    IPv6 can reach the neighbor address but the addresses has not yet
    been resolved.
  \end{itemize}
\item
  Serial0/1 goes down. How will EIGRP send packets to the 10.1.1.0
  network?

\begin{verbatim}
Corp#show ip eigrp topology
[output cut]
P 10.1.1.0/24, 2 successors, FD is 2681842
	via 10.1.2.2 (2681842/2169856), Serial0/0
	via 10.1.3.1 (2973467/2579243), Serial0/2
	via 10.1.3.3 (2681842/2169856), Serial0/1
\end{verbatim}

  \begin{itemize}
  \tightlist
  \item
    EIGRP will put the 10.1.1.0 network into active mode.
  \item
    EIGRP will drop all packets destined for 10.1.1.0.
  \item
    EIGRP will just keep sending packets out s0/0.
  \item
    EIGRP will use s0/2 as the successor and keep routing to 10.1.1.0.
  \end{itemize}
\item
  What command produced the following output?

\begin{verbatim}
via FE80::201:C9FF:FED0:3301 (29110112/33316), Serial0/0/0
via FE80::209:7CFF:FE51:B401 (4470112/42216), Serial0/0/1
via FE80::209:7CFF:FE51:B401 (2170112/2816), Serial0/0/2
\end{verbatim}

  \begin{itemize}
  \tightlist
  \item
    \texttt{show\ ip\ protocols}
  \item
    \texttt{show\ ipv6\ protocols}
  \item
    \texttt{show\ ip\ eigrp\ neighbors}
  \item
    \texttt{show\ ipv6\ eigrp\ neighbors}
  \item
    \texttt{show\ ip\ eigrp\ topology}
  \item
    \texttt{show\ ipv6\ eigrp\ topology}
  \end{itemize}
\item
  \protect\hypertarget{f_04.xhtmlux5cux23Page_liii}{}{}You need to
  troubleshoot an adjacency between two EIGRP configured routers? What
  should you look for? (Choose four.)

  \begin{itemize}
  \tightlist
  \item
    Verify the AS numbers.
  \item
    Verify that you have the proper interfaces enabled for EIGRP.
  \item
    Make sure there are no mismatched K-values.
  \item
    Check your passive interface settings.
  \item
    Make sure your remote routers are not connected to the Internet.
  \item
    If authentication is configured, make sure all routers use different
    passwords.
  \end{itemize}
\item
  You have two OSPF directly configured routers that are not forming an
  adjacency. What should you check? (Choose three.)

  \begin{itemize}
  \tightlist
  \item
    Process ID
  \item
    Hello and dead timers
  \item
    Link cost
  \item
    Area
  \item
    IP address/subnet mask
  \end{itemize}
\item
  When do two adjacent routers-enter the 2WAY state?

  \begin{itemize}
  \tightlist
  \item
    After both routers have received Hello information
  \item
    After they have exchanged topology databases
  \item
    When they connect only to a DR or BDR
  \item
    When they need to exchange RID information
  \end{itemize}
\item
  Which type of LSAs are generated by ABRs and referred to summary link
  advertisements (SLAs)?

  \begin{itemize}
  \tightlist
  \item
    Type 1
  \item
    Type 2
  \item
    Type 3
  \item
    Type 4
  \item
    Type 5
  \end{itemize}
\item
  Which of the following is not provided by the AH portion of IPsec?

  \begin{itemize}
  \tightlist
  \item
    Integrity
  \item
    Confidentiality
  \item
    Authenticity
  \item
    Anti-reply
  \end{itemize}
\item
  Which statement about GRE is not true?

  \begin{itemize}
  \tightlist
  \item
    GRE is stateless and has no flow control.
  \item
    GRE has security.
  \item
    \protect\hypertarget{f_04.xhtmlux5cux23Page_liv}{}{}GRE has
    additional overhead for tunneled packets, at least 24 bytes.
  \item
    GRE uses a protocol-type field in the GRE header so any layer 3
    protocol can be used through the tunnel.
  \end{itemize}
\item
  Which QoS mechanism will drop traffic if a session uses more than the
  allotted bandwidth?

  \begin{itemize}
  \tightlist
  \item
    Congestion management
  \item
    Shaping
  \item
    Policing
  \item
    Marking
  \end{itemize}
\item
  IPv6 unicast routing is running on the Corp router. Which of the
  following addresses would show up with the show ipv6 int brief
  command?

\begin{verbatim}
Corp#sh int f0/0
FastEthernet0/0 is up, line protocol is up
	Hardware is AmdFE, address is 000d.bd3b.0d80 (bia 000d.bd3b.0d80)
[output cut]
\end{verbatim}

  \begin{itemize}
  \tightlist
  \item
    \texttt{FF02::3c3d:0d:bdff:fe3b:0d80}
  \item
    \texttt{FE80::3c3d:2d:bdff:fe3b:0d80}
  \item
    \texttt{FE80::3c3d:0d:bdff:fe3b:0d80}
  \item
    \texttt{FE80::3c3d:2d:ffbd:3bfe:0d80}
  \end{itemize}
\item
  A host sends a type of NDP message providing the MAC address that was
  requested. Which type of NDP was sent?

  \begin{itemize}
  \tightlist
  \item
    NA
  \item
    RS
  \item
    RA
  \item
    NS
  \end{itemize}
\item
  Each field in an IPv6 address is how many bits long?

  \begin{itemize}
  \tightlist
  \item
    4
  \item
    16
  \item
    32
  \item
    128
  \end{itemize}
\item
  To enable OSPFv3, which of the following would you use?

  \begin{itemize}
  \tightlist
  \item
    Router(config-if)\#\textbf{ipv6 ospf 10 area 0.0.0.0}
  \item
    Router(config-if)\#\textbf{ipv6 router rip 1}
  \item
    Router(config)\#\textbf{ipv6 router eigrp 10}
  \item
    Router(config-rtr)\#\textbf{no shutdown}
  \item
    Router(config-if)\#\textbf{ospf ipv6 10 area 0}
  \end{itemize}
\item
  \protect\hypertarget{f_04.xhtmlux5cux23Page_lv}{}{}What does the
  command \texttt{routerA(config)\#line\ cons\ 0} allow you to perform
  next?

  \begin{itemize}
  \tightlist
  \item
    Set the Telnet password.
  \item
    Shut down the router.
  \item
    Set your console password.
  \item
    Disable console connections.
  \end{itemize}
\item
  Which two statements describe the IP address 10.16.3.65/23? (Choose
  two.)

  \begin{itemize}
  \tightlist
  \item
    The subnet address is 10.16.3.0 255.255.254.0.
  \item
    The lowest host address in the subnet is 10.16.2.1 255.255.254.0.
  \item
    The last valid host address in the subnet is 10.16.2.254
    255.255.254.0.
  \item
    The broadcast address of the subnet is 10.16.3.255 255.255.254.0.
  \item
    The network is not subnetted.
  \end{itemize}
\item
  On which interface do you configure an IP address for a switch?

  \begin{itemize}
  \tightlist
  \item
    \texttt{int\ fa0/0}
  \item
    \texttt{int\ vty\ 0\ 15}
  \item
    \texttt{int\ vlan\ 1}
  \item
    \texttt{int\ s/0/0}
  \end{itemize}
\item
  Which of the following is the valid host range for the subnet on which
  the IP address 192.168.168.188 255.255.255.192 resides?

  \begin{itemize}
  \tightlist
  \item
    192.168.168.129--190
  \item
    192.168.168.129--191
  \item
    192.168.168.128--190
  \item
    192.168.168.128--192
  \end{itemize}
\item
  Which of the following is considered to be the inside host's address
  after translation?

  \begin{itemize}
  \tightlist
  \item
    Inside local
  \item
    Outside local
  \item
    Inside global
  \item
    Outside global
  \end{itemize}
\item
  Your inside locals are not being translated to the inside global
  addresses. Which of the following commands will show you if your
  inside globals are allowed to use the NAT pool?

\begin{verbatim}
ip nat pool Corp 198.18.41.129 198.18.41.134 netmask 255.255.255.248
ip nat inside source list 100 int pool Corp overload
\end{verbatim}

  \begin{itemize}
  \tightlist
  \item
    \texttt{debug\ ip\ nat}
  \item
    \texttt{show\ access-list}
  \item
    \texttt{show\ ip\ nat\ translation}
  \item
    \texttt{show\ ip\ nat\ statistics}
  \end{itemize}
\item
  How many collision domains are created when you segment a network with
  a 12-port switch?

  \begin{itemize}
  \tightlist
  \item
    1
  \item
    2
  \item
    5
  \item
    12
  \end{itemize}
\item
  Which of the following commands will allow you to set your Telnet
  password on a Cisco router?

  \begin{itemize}
  \tightlist
  \item
    \texttt{line\ telnet\ 0\ 4}
  \item
    \texttt{line\ aux\ 0\ 4}
  \item
    \texttt{line\ vty\ 0\ 4}
  \item
    \texttt{line\ con\ 0}
  \end{itemize}
\item
  Which router command allows you to view the entire contents of all
  access lists?

  \begin{itemize}
  \tightlist
  \item
    \texttt{show\ all\ access-lists}
  \item
    \texttt{show\ access-lists}
  \item
    \texttt{show\ ip\ interface}
  \item
    \texttt{show\ interface}
  \end{itemize}
\item
  What does a VLAN do?

  \begin{itemize}
  \tightlist
  \item
    Acts as the fastest port to all servers
  \item
    Provides multiple collision domains on one switch port
  \item
    Breaks up broadcast domains in a layer 2 switch internetwork
  \item
    Provides multiple broadcast domains within a single collision domain
  \end{itemize}
\item
  If you wanted to delete the configuration stored in NVRAM, choose the
  best answer for the Cisco objectives.

  \begin{itemize}
  \tightlist
  \item
    \texttt{erase\ startup}
  \item
    \texttt{delete\ running}
  \item
    \texttt{erase\ flash}
  \item
    \texttt{erase\ running}
  \end{itemize}
\item
  Which protocol is used to send a destination network unknown message
  back to originating hosts?

  \begin{itemize}
  \tightlist
  \item
    TCP
  \item
    ARP
  \item
    ICMP
  \item
    BootP
  \end{itemize}
\item
  \protect\hypertarget{f_04.xhtmlux5cux23Page_lvii}{}{}Which class of IP
  address provides 15 bits for subnetting?

  \begin{itemize}
  \tightlist
  \item
    A
  \item
    B
  \item
    C
  \item
    D
  \end{itemize}
\item
  There are three possible routes for a router to reach a destination
  network. The first route is from OSPF with a metric of 782. The second
  route is from RIPv2 with a metric of 4. The third is from EIGRP with a
  composite metric of 20514560. Which route will be installed by the
  router in its routing table?

  \begin{itemize}
  \tightlist
  \item
    RIPv2
  \item
    EIGRP
  \item
    OSPF
  \item
    All three
  \end{itemize}
\item
  Which one of the following is true regarding VLANs?

  \begin{itemize}
  \tightlist
  \item
    Two VLANs are configured by default on all Cisco switches.
  \item
    VLANs only work if you have a complete Cisco switched internetwork.
    No off-brand switches are allowed.
  \item
    You should not have more than 10 switches in the same VTP domain.
  \item
    You need to have a trunk link configured between switches in order
    to send information about more than one VLAN down the link.
  \end{itemize}
\item
  Which two of the following commands will place network 10.2.3.0/24
  into area 0? (Choose two.)

  \begin{itemize}
  \tightlist
  \item
    \texttt{router\ eigrp\ 10}
  \item
    \texttt{router\ ospf\ 10}
  \item
    \texttt{router\ rip}
  \item
    \texttt{network\ 10.0.0.0}
  \item
    \texttt{network\ 10.2.3.0\ 255.255.255.0\ area\ 0}
  \item
    \texttt{network\ 10.2.3.0\ 0.0.0.255\ area0}
  \item
    \texttt{network\ 10.2.3.0\ 0.0.0.255\ area\ 0}
  \end{itemize}
\item
  How many broadcast domains are created when you segment a network with
  a 12-port switch?

  \begin{itemize}
  \tightlist
  \item
    1
  \item
    2
  \item
    5
  \item
    12
  \end{itemize}
\item
  \protect\hypertarget{f_04.xhtmlux5cux23Page_lviii}{}{}If routers in a
  single area are configured with the same priority value, what value
  does a router use for the OSPF router ID in the absence of a loopback
  interface?

  \begin{itemize}
  \tightlist
  \item
    The lowest IP address of any physical interface
  \item
    The highest IP address of any physical interface
  \item
    The lowest IP address of any logical interface
  \item
    The highest IP address of any logical interface
  \end{itemize}
\item
  What protocols are used to configure trunking on a switch? (Choose
  two.)

  \begin{itemize}
  \tightlist
  \item
    VLAN Trunking Protocol
  \item
    VLAN
  \item
    802.1q
  \item
    ISL
  \end{itemize}
\item
  What is a stub network?

  \begin{itemize}
  \tightlist
  \item
    A network with more than one exit point
  \item
    A network with more than one exit and entry point
  \item
    A network with only one entry and no exit point
  \item
    A network that has only one entry and exit point
  \end{itemize}
\item
  Where is a hub specified in the OSI model?

  \begin{itemize}
  \tightlist
  \item
    Session layer
  \item
    Physical layer
  \item
    Data Link layer
  \item
    Application layer
  \end{itemize}
\item
  What are the two main types of access control lists (ACLs)? (Choose
  two.)

  \begin{itemize}
  \tightlist
  \item
    Standard
  \item
    IEEE
  \item
    Extended
  \item
    Specialized
  \end{itemize}
\item
  Which of the following is the best summarization of the following
  networks: 192.168.128.0 through 192.168.159.0?

  \begin{itemize}
  \tightlist
  \item
    192.168.0.0/24
  \item
    192.168.128.0/16
  \item
    192.168.128.0/19
  \item
    192.168.128.0/20
  \end{itemize}
\item
  \protect\hypertarget{f_04.xhtmlux5cux23Page_lix}{}{}What command is
  used to create a backup configuration?

  \begin{itemize}
  \tightlist
  \item
    \texttt{copy\ running\ backup}
  \item
    \texttt{copy\ running-config\ startup-config}
  \item
    \texttt{config\ mem}
  \item
    \texttt{wr\ net}
  \end{itemize}
\item
  1000Base-T is which IEEE standard?

  \begin{itemize}
  \tightlist
  \item
    802.3f
  \item
    802.3z
  \item
    802.3ab
  \item
    802.3ae
  \end{itemize}
\item
  Which protocol does DHCP use at the Transport layer?

  \begin{itemize}
  \tightlist
  \item
    IP
  \item
    TCP
  \item
    UDP
  \item
    ARP
  \end{itemize}
\item
  If your router is facilitating a CSU/DSU, which of the following
  commands do you need to use to provide the router with a 64000 bps
  serial link?

  \begin{itemize}
  \tightlist
  \item
    \texttt{RouterA(config)\#bandwidth\ 64}
  \item
    \texttt{RouterA(config-if)\#bandwidth\ 64000}
  \item
    \texttt{RouterA(config)\#clockrate\ 64000}
  \item
    \texttt{RouterA(config-if)\#clock\ rate\ 64}
  \item
    \texttt{RouterA(config-if)\#clock\ rate\ 64000}
  \end{itemize}
\item
  Which command is used to determine if an access list is enabled on a
  particular interface?

  \begin{itemize}
  \tightlist
  \item
    \texttt{show\ access-lists}
  \item
    \texttt{show\ interface}
  \item
    \texttt{show\ ip\ interface}
  \item
    \texttt{show\ interface\ access-lists}
  \end{itemize}
\item
  Which of the following statements is true with regard to ISL and
  802.1q?

  \begin{itemize}
  \tightlist
  \item
    802.1q encapsulates the frame with control information; ISL inserts
    an ISL field along with tag control information.
  \item
    802.1q is Cisco proprietary.
  \item
    ISL encapsulates the frame with control information; 802.1q inserts
    an 802.1q field along with tag control information.
  \item
    ISL is a standard.
  \end{itemize}
\item
  \protect\hypertarget{f_04.xhtmlux5cux23Page_lx}{}{}The protocol data
  unit (PDU) encapsulation is completed in which order?

  \begin{itemize}
  \tightlist
  \item
    Bits, frames, packets, segments, data
  \item
    Data, bits, segments, frames, packets
  \item
    Data, segments, packets, frames, bits
  \item
    Packets, frames, bits, segments, data
  \end{itemize}
\item
  Based on the configuration shown below, what statement is true?

\begin{verbatim}
S1(config)#ip routing
S1(config)#int vlan 10
S1(config-if)#ip address 192.168.10.1 255.255.255.0
S1(config-if)#int vlan 20
S1(config-if)#ip address 192.168.20.1 255.255.255.0
\end{verbatim}

  \begin{itemize}
  \tightlist
  \item
    This is a multilayer switch.
  \item
    The two VLANs are in the same subnet.
  \item
    Encapsulation must be configured.
  \item
    VLAN 10 is the management VLAN.
  \end{itemize}
\end{enumerate}

******************

\subsection[Answers to Assessment
Test]{\texorpdfstring{\protect\hypertarget{f_04.xhtmlux5cux23Page_lxi}{}{}Answers
to Assessment Test}{Answers to Assessment Test}}

\begin{enumerate}
\tightlist
\item
  B. To allow for the PVST+ to operate, there's a field inserted into
  the BPDU to accommodate the extended system ID so that PVST+ can have
  a root bridge configured on a per-STP instance. The extended system ID
  (VLAN ID) is a 12-bit field, and we can even see what this field is
  carrying via show spanning-tree command output. See Chapter 15 for
  more information.
\item
  A. Cisco's EtherChannel can bundle up to eight ports between switches
  to provide resiliency and more bandwidth between switches. See Chapter
  15 for more information.
\item
  B, C, E. All the ports on both sides of every link must be configured
  exactly the same between switches or it will not work. Speed, duplex,
  and allowed VLANs must match. See Chapter 15 for more information.
\item
  C. 2100 boots the router into ROM monitor mode, 2101 loads the
  mini-IOS from ROM, and 2102 is the default and loads the IOS from
  flash. See Chapter 8 for more information.
\item
  D. The \texttt{show\ license\ udi} command displays the unique device
  identifier (UDI) of the router, which comprises the product ID (PID)
  and serial number of the router. See Chapter 8 for more information.
\item
  B. The \texttt{show\ license} feature command allows you to view the
  technology package licenses and feature licenses that are supported on
  your router along with several status variables related to software
  activation and licensing, both licensed and unlicensed features. See
  Chapter 8 for more information.
\item
  C, D, F. The SDN architecture slightly differs from the architecture
  of traditional networks. It comprises three stacked layers: Data,
  Control and Application. See Chapter 8 for more information.
\item
  B. There are eight different trap levels. If you choose, for example
  level 3, level 0 through level 3 messages will be displayed. See
  Chapter 8 for more information.
\item
  B, D. Each stack of switches has a single IP address and is managed as
  a single object. This single IP management applies to activities such
  as fault detection, VLAN creation and modification, security, and QoS
  controls. Each stack has only one configuration file, which is
  distributed to each member in the stack. When you add a new switch to
  the stack, the master switch automatically configures the unit with
  the currently running IOS image and the configuration of the stack.
  You do not have to do anything to bring up the switch before it is
  ready to operate. See chapter 22 for more information.
\item
  B. There is no IPv6 default gateway listed in the output, which will
  be the link-local address of the router interface, sent to the host as
  a router advertisement. Until this host receives the router address,
  the host will communicate with IPv6 only on the local subnet. See
  Chapter 20 for more information.
\item
  D. The command \texttt{show\ ipv6\ neighbors} provides the ARP cache
  for on a router. See Chapter 20 for more information.
\item
  \protect\hypertarget{f_04.xhtmlux5cux23Page_lxii}{}{}A. If the state
  is STALE when the interface has not communicated within the neighbor
  reachable time frame. The next time the neighbor communicates, the
  state will be REACH. See Chapter 20 for more information.
\item
  C. There are two successor routes, so by default, EIGRP was
  load-balancing out s0/0 and s0/1. When s0/1 goes down, EIGRP will just
  keep forwarding traffic out the second link s0/0. s0/1 will be removed
  from the routing table. See Chapter 17 for more information.
\item
  F. There isn't a lot to go on from with the output, but the only
  commands that provide the FD and AD are \texttt{show\ ip\ eigrp}
  topology and \texttt{show\ ipv6\ eigrp\ topology}. The addresses in
  the output are link-local IPv6 addresses, so our answer is the latter.
  See Chapter 17 for more information.
\item
  A, B, C, D. Cisco has documented steps, according to the objectives,
  that you must go through when troubleshooting an adjacency. See
  Chapter 18 for more information.
\item
  B, D, E. In order for two OSPF routers to create an adjacency, the
  Hello and dead timers must match, and they must both be configured
  into the same area, as well as being in the same subnet. See Chapter
  18 for more information.
\item
  A. The process starts by sending out Hello packets. Every listening
  router will then add the originating router to the neighbor database.
  The responding routers will reply with all of their Hello information
  so that the originating router can add them to its own neighbor table.
  At this point, we will have reached the 2WAY state---only certain
  routers will advance beyond to this. See Chapter 19 for more
  information.
\item
  C. Referred to as summary link advertisements (SLAs), Type 3 LSAs are
  generated by area border routers. These ABRs send Type 3 LSAs toward
  the area external to the one where they were generated. See Chapter 19
  for more information.
\item
  B. Authentication Header (AH) provides authentication of either all or
  part of the IP packet through the addition of a header that is
  calculated based on the values in the packet, but it doesn't offer any
  encryption services. See Chapter 21 for more information.
\item
  B. Generic Routing Encapsulation (GRE) has no built-in security
  mechanisms. See Chapter 21 for more information.
\item
  C. When traffic exceeds the allocated rate, the policer can take one
  of two actions. It can either drop traffic or re-mark it to another
  class of service. The new class usually has a higher drop probability.
  See Chapter 21 for more information.
\item
  B. This can be a hard question if you don't remember to invert the 7th
  bit of the first octet in the MAC address! Always look for the 7th bit
  when studying for the Cisco R/S, and when using eui-64, invert it. The
  eui-64 autoconfiguration then inserts an FF:FE in the middle of the
  48-bit MAC address to create a unique IPv6 address. See Chapter 14 for
  more information.
\item
  A. The NDP neighbor advertisement (NA) contains the MAC address. A
  neighbor solicitation (NS) was initially sent asking for the MAC
  address. See Chapter 14 for more information.
\item
  \protect\hypertarget{f_04.xhtmlux5cux23Page_lxiii}{}{}B. Each field in
  an IPv6 address is 16 bits long. An IPv6 address is a total of 128
  bits. See Chapter 14 for more information.
\item
  A. To enable OSPFv3, you enable the protocol at the interface level,
  as with RIPng. The command string is area-id. It's important to
  understand that area \texttt{0} and area \texttt{0.0.0.0} both
  describe area \texttt{0}. See Chapter 19 for more information.
\item
  C. The command line console \texttt{0} places you at a prompt where
  you can then set your console user-mode password. See Chapter 6 for
  more information.
\item
  B, D. The mask 255.255.254.0 (/23) used with a Class A address means
  that there are 15 subnet bits and 9 host bits. The block size in the
  third octet is 2 (256--254). So this makes the subnets in the
  interesting octet 0, 2, 4, 6, etc., all the way to 254. The host
  10.16.3.65 is in the 2.0 subnet. The next subnet is 4.0, so the
  broadcast address for the 2.0 subnet is 3.255. The valid host
  addresses are 2.1 through 3.254. See Chapter 4 for more information.
\item
  C. The IP address is configured under a logical interface, called a
  management domain or VLAN 1, by default. See Chapter 10 for more
  information.
\item
  A. 256 -- 192 = 64, so 64 is our block size. Just count in increments
  of 64 to find our subnet: 64 + 64 = 128. 128 + 64 = 192. The subnet is
  128, the broadcast address is 191, and the valid host range is the
  numbers in between, or 129--190. See Chapter 4 for more information.
\item
  C. An inside global address is considered to be the IP address of the
  host on the private network after translation. See Chapter 13 for more
  information.
\item
  B. Once you create your pool, the command ip nat inside source must be
  used to say which inside locals are allowed to use the pool. In this
  question, we need to see if access list 100 is configured correctly,
  if at all, so \texttt{show\ access-list} is the best answer. See
  Chapter 13 for more information.
\item
  D. Layer 2 switching creates individual collision domains per port.
  See Chapter 1 for more information.
\item
  C. The command line vty 0 4 places you in a prompt that will allow you
  to set or change your Telnet password. See Chapter 6 for more
  information.
\item
  B. To see the contents of all access lists, use the
  \texttt{show\ access-lists} command. See Chapter 12 for more
  information.
\item
  C. VLANs break up broadcast domains at layer 2. See Chapter 11 for
  more information.
\item
  A. The command \texttt{erase\ startup-config} deletes the
  configuration stored in NVRAM. See Chapter 6 for more information.
\item
  C. ICMP is the protocol at the Network layer that is used to send
  messages back to an originating router. See Chapter 3 for more
  information.
\item
  A. Class A addressing provides 22 bits for host subnetting. Class B
  provides 16 bits, but only 14 are available for subnetting. Class C
  provides only 6 bits for subnetting. See Chapter 3 for more
  information.
\item
  \protect\hypertarget{f_04.xhtmlux5cux23Page_lxiv}{}{}B. Only the EIGRP
  route will be placed in the routing table because EIGRP has the lowest
  administrative distance (AD), and that is always used before metrics.
  See Chapter 8 for more information.
\item
  D. Switches send information about only one VLAN down a link unless it
  is configured as a trunk link. See Chapter 11 for more information.
\item
  B, G. To enable OSPF, you must first start OSPF using a process ID.
  The number is irrelevant; just choose a number from 1 to 65,535 and
  you're good to go. After you start the OSPF process, you must
  configure interfaces on which to activate OSPF using the network
  command with wildcards and specification of an area. Option F is wrong
  because there must be a space after the parameter area and before you
  list the area number. See Chapter 9 for more information.
\item
  A. By default, switches break up collision domains on a per-port basis
  but are one large broadcast domain. See Chapter 1 for more
  information.
\item
  B. At the moment of OSPF process startup, the highest IP address on
  any active interface will be the router ID (RID) of the router. If you
  have a loopback interface configured (logical interface), then that
  will override the interface IP address and become the RID of the
  router automatically. See Chapter 18 for more information.
\item
  C, D. VLAN Trunking Protocol (VTP) is not right because it has nothing
  to do with trunking except that it sends VLAN information across a
  trunk link. 802.1q and ISL encapsulations are used to configure
  trunking on a port. See Chapter 11 for more information.
\item
  D. Stub networks have only one connection to an internetwork. Default
  routes should be set on a stub network or network loops may occur;
  however, there are exceptions to this rule. See Chapter 9 for more
  information.
\item
  B. Hubs regenerate electrical signals, which are specified at the
  Physical layer. See Chapter 1 for more information.
\item
  A, C. Standard and extended access control lists (ACLs) are used to
  configure security on a router. See Chapter 12 for more information.
\item
  C. If you start at 192.168.128.0 and go through 192.168.159.0, you can
  see that this is a block of 32 in the third octet. Since the network
  address is always the first one in the range, the summary address is
  192.168.128.0. What mask provides a block of 32 in the third octet?
  The answer is 255.255.224.0, or /19. See Chapter 5 for more
  information.
\item
  B. The command to back up the configuration on a router is
  \texttt{copy\ running-config\ startup-config}. See Chapter 7 for more
  information.
\item
  C. IEEE 802.3ab is the standard for 1 Gbps on twisted-pair. See
  Chapter 2 for more information.
\item
  C. User Datagram Protocol is a connection network service at the
  Transport layer, and DHCP uses this connectionless service. See
  Chapter 3 for more information
\item
  \protect\hypertarget{f_04.xhtmlux5cux23Page_lxv}{}{}E. The clock rate
  command is two words, and the speed of the line is in bits per second
  (bps). See Chapter 6 for more information.
\item
  C. The \texttt{show\ ip\ interface} command will \texttt{show} you if
  any interfaces have an outbound or inbound access list set. See
  Chapter 12 for more information.
\item
  C. Unlike ISL, which encapsulates the frame with control information,
  802.1q inserts an 802.1q field along with tag control information. See
  Chapter 11 for more information.
\item
  C. The PDU encapsulation method defines how data is encoded as it goes
  through each layer of the TCP/IP model. Data is segmented at the
  Transport later, packets created at the Network layer, frames at the
  Data Link layer, and finally, the Physical layer encodes the 1s and 0s
  into a digital signal. See Chapter 2 for more information.
\item
  A. With a multilayer switch, enable IP routing and create one logical
  interface for each VLAN using the interface vlan number command and
  you're now doing inter-VLAN routing on the backplane of the switch!
  See Chapter 11 for more information.
\end{enumerate}

\protect\hypertarget{f_05.xhtml}{}{}

\section[Assessment
Test]{\texorpdfstring{\protect\hypertarget{f_05.xhtmlux5cux23Page_l}{}{}Assessment
Test}{Assessment Test}}

\begin{enumerate}
\item
  What is the \texttt{sys-id-ext} field in a BPDU used for?

  \begin{itemize}
  \tightlist
  \item
    It is a 4-bit field inserted into an Ethernet frame to define
    trunking information between switches.
  \item
    It is a 12-bit field inserted into an Ethernet frame to define VLANs
    in an STP instance.
  \item
    It is a 4-bit field inserted into an non-Ethernet frame to define
    EtherChannel options.
  \item
    It is a 12-bit field inserted into an Ethernet frame to define STP
    root bridges.
  \end{itemize}
\item
  You have four RSTP PVST+ links between switches and want to aggregate
  the bandwidth. What solution will you use?

  \begin{itemize}
  \tightlist
  \item
    EtherChannel
  \item
    PortFast
  \item
    BPDU Channel
  \item
    VLANs
  \item
    EtherBundle
  \end{itemize}
\item
  What configuration parameters must be configured the same between
  switches for LACP to form a channel? (Choose three.)

  \begin{itemize}
  \tightlist
  \item
    Virtual MAC address
  \item
    Port speeds
  \item
    Duplex
  \item
    PortFast enabled
  \item
    Allowed VLAN information
  \end{itemize}
\item
  You reload a router with a configuration register setting of 0x2101.
  What will the router do when it reloads?

  \begin{itemize}
  \tightlist
  \item
    The router enters setup mode.
  \item
    The router enters ROM monitor mode.
  \item
    The router boots the mini-IOS in ROM.
  \item
    The router expands the first IOS in flash memory into RAM.
  \end{itemize}
\item
  Which of the following commands provides the product ID and serial
  number of a router?

  \begin{itemize}
  \tightlist
  \item
    \texttt{show\ license}
  \item
    \texttt{show\ license\ feature}
  \item
    \texttt{show\ version}
  \item
    \texttt{show\ license\ udi}
  \end{itemize}
\item
  \protect\hypertarget{f_05.xhtmlux5cux23Page_li}{}{}Which command
  allows you to view the technology options and licenses that are
  supported on your router along with several status variables?

  \begin{itemize}
  \tightlist
  \item
    \texttt{show\ license}
  \item
    \texttt{show\ license\ feature}
  \item
    \texttt{show\ license\ udi}
  \item
    \texttt{show\ version}
  \end{itemize}
\item
  Which of the following services provide the operating system and the
  network?

  \begin{itemize}
  \tightlist
  \item
    IaaS
  \item
    PaaS
  \item
    SaaS
  \item
    none of the above
  \end{itemize}
\item
  You want to send a console message to a syslog server, but you only
  want to send status messages of 3 and lower. Which of the following
  commands will you use?

  \begin{itemize}
  \tightlist
  \item
    \texttt{logging\ trap\ emergencies}
  \item
    \texttt{logging\ trap\ errors}
  \item
    \texttt{logging\ trap\ debugging}
  \item
    \texttt{logging\ trap\ notifications}
  \item
    \texttt{logging\ trap\ critical}
  \item
    \texttt{logging\ trap\ warnings}
  \item
    \texttt{logging\ trap\ alerts}
  \end{itemize}
\item
  When stacking switches, which is true? (Choose 2)

  \begin{itemize}
  \tightlist
  \item
    The stack is managed as multiple objects, and has a single
    management IP address
  \item
    The stack is managed as a single object, and has a single management
    IP address
  \item
    The master switch is chosen when you configure the first switches
    master algorithm to on
  \item
    The master switch is elected form one of the stack member switches
  \end{itemize}
\item
  You need to connect to a remote IPv6 server in your virtual server
  farm. You can connect to the IPv4 servers, but not the critical IPv6
  server you desperately need. Based on the following output, what could
  your problem be?

\begin{verbatim}
C:\>ipconfig
Connection-specific DNS Suffix . : localdomain
IPv6 Address. . . . . . . . . . . : 2001:db8:3c4d:3:ac3b:2ef:1823:8938
Temporary IPv6 Address. . . . . . : 2001:db8:3c4d:3:2f33:44dd:211:1c3d
Link-local IPv6 Address . . . . . : fe80::ac3b:2ef:1823:8938%11
IPv4 Address. . . . . . . . . . . : 10.1.1.10
Subnet Mask . . . . . . . . . . . : 255.255.255.0
Default Gateway . . . . . . . . . : 10.1.1.1
\end{verbatim}

  \begin{itemize}
  \tightlist
  \item
    The global address is in the wrong subnet.
  \item
    The IPv6 default gateway has not been configured or received from
    the router.
  \item
    \protect\hypertarget{f_05.xhtmlux5cux23Page_lii}{}{}The link-local
    address has not been resolved so the host cannot communicate to the
    router.
  \item
    There are two IPv6 global addresses configured. One must be removed
    from the configuration.
  \end{itemize}
\item
  What command is used to view the IPv6-to-MAC-address resolution table
  on a Cisco router?

  \begin{itemize}
  \tightlist
  \item
    \texttt{show\ ip\ arp}
  \item
    \texttt{show\ ipv6\ arp}
  \item
    \texttt{show\ ip\ neighbors}
  \item
    \texttt{show\ ipv6\ neighbors}
  \item
    \texttt{show\ arp}
  \end{itemize}
\item
  An IPv6 ARP entry is listed as with a status of REACH. What can you
  conclude about the IPv6-to-MAC-address mapping?

  \begin{itemize}
  \tightlist
  \item
    The interface has communicated with the neighbor address and the
    mapping is current.
  \item
    The interface has not communicated within the neighbor reachable
    time frame.
  \item
    The ARP entry has timed out.
  \item
    IPv6 can reach the neighbor address but the addresses has not yet
    been resolved.
  \end{itemize}
\item
  Serial0/1 goes down. How will EIGRP send packets to the 10.1.1.0
  network?

\begin{verbatim}
Corp#show ip eigrp topology
[output cut]
P 10.1.1.0/24, 2 successors, FD is 2681842
	via 10.1.2.2 (2681842/2169856), Serial0/0
	via 10.1.3.1 (2973467/2579243), Serial0/2
	via 10.1.3.3 (2681842/2169856), Serial0/1
\end{verbatim}

  \begin{itemize}
  \tightlist
  \item
    EIGRP will put the 10.1.1.0 network into active mode.
  \item
    EIGRP will drop all packets destined for 10.1.1.0.
  \item
    EIGRP will just keep sending packets out s0/0.
  \item
    EIGRP will use s0/2 as the successor and keep routing to 10.1.1.0.
  \end{itemize}
\item
  What command produced the following output?

\begin{verbatim}
via FE80::201:C9FF:FED0:3301 (29110112/33316), Serial0/0/0
via FE80::209:7CFF:FE51:B401 (4470112/42216), Serial0/0/1
via FE80::209:7CFF:FE51:B401 (2170112/2816), Serial0/0/2
\end{verbatim}

  \begin{itemize}
  \tightlist
  \item
    \texttt{show\ ip\ protocols}
  \item
    \texttt{show\ ipv6\ protocols}
  \item
    \texttt{show\ ip\ eigrp\ neighbors}
  \item
    \texttt{show\ ipv6\ eigrp\ neighbors}
  \item
    \texttt{show\ ip\ eigrp\ topology}
  \item
    \texttt{show\ ipv6\ eigrp\ topology}
  \end{itemize}
\item
  \protect\hypertarget{f_05.xhtmlux5cux23Page_liii}{}{}You need to
  troubleshoot an adjacency between two EIGRP configured routers? What
  should you look for? (Choose four.)

  \begin{itemize}
  \tightlist
  \item
    Verify the AS numbers.
  \item
    Verify that you have the proper interfaces enabled for EIGRP.
  \item
    Make sure there are no mismatched K-values.
  \item
    Check your passive interface settings.
  \item
    Make sure your remote routers are not connected to the Internet.
  \item
    If authentication is configured, make sure all routers use different
    passwords.
  \end{itemize}
\item
  You have two OSPF directly configured routers that are not forming an
  adjacency. What should you check? (Choose three.)

  \begin{itemize}
  \tightlist
  \item
    Process ID
  \item
    Hello and dead timers
  \item
    Link cost
  \item
    Area
  \item
    IP address/subnet mask
  \end{itemize}
\item
  When do two adjacent routers-enter the 2WAY state?

  \begin{itemize}
  \tightlist
  \item
    After both routers have received Hello information
  \item
    After they have exchanged topology databases
  \item
    When they connect only to a DR or BDR
  \item
    When they need to exchange RID information
  \end{itemize}
\item
  Which type of LSAs are generated by ABRs and referred to summary link
  advertisements (SLAs)?

  \begin{itemize}
  \tightlist
  \item
    Type 1
  \item
    Type 2
  \item
    Type 3
  \item
    Type 4
  \item
    Type 5
  \end{itemize}
\item
  Which of the following is not provided by the AH portion of IPsec?

  \begin{itemize}
  \tightlist
  \item
    Integrity
  \item
    Confidentiality
  \item
    Authenticity
  \item
    Anti-reply
  \end{itemize}
\item
  Which statement about GRE is not true?

  \begin{itemize}
  \tightlist
  \item
    GRE is stateless and has no flow control.
  \item
    GRE has security.
  \item
    \protect\hypertarget{f_05.xhtmlux5cux23Page_liv}{}{}GRE has
    additional overhead for tunneled packets, at least 24 bytes.
  \item
    GRE uses a protocol-type field in the GRE header so any layer 3
    protocol can be used through the tunnel.
  \end{itemize}
\item
  Which QoS mechanism will drop traffic if a session uses more than the
  allotted bandwidth?

  \begin{itemize}
  \tightlist
  \item
    Congestion management
  \item
    Shaping
  \item
    Policing
  \item
    Marking
  \end{itemize}
\item
  IPv6 unicast routing is running on the Corp router. Which of the
  following addresses would show up with the show ipv6 int brief
  command?

\begin{verbatim}
Corp#sh int f0/0
FastEthernet0/0 is up, line protocol is up
	Hardware is AmdFE, address is 000d.bd3b.0d80 (bia 000d.bd3b.0d80)
[output cut]
\end{verbatim}

  \begin{itemize}
  \tightlist
  \item
    \texttt{FF02::3c3d:0d:bdff:fe3b:0d80}
  \item
    \texttt{FE80::3c3d:2d:bdff:fe3b:0d80}
  \item
    \texttt{FE80::3c3d:0d:bdff:fe3b:0d80}
  \item
    \texttt{FE80::3c3d:2d:ffbd:3bfe:0d80}
  \end{itemize}
\item
  A host sends a type of NDP message providing the MAC address that was
  requested. Which type of NDP was sent?

  \begin{itemize}
  \tightlist
  \item
    NA
  \item
    RS
  \item
    RA
  \item
    NS
  \end{itemize}
\item
  Each field in an IPv6 address is how many bits long?

  \begin{itemize}
  \tightlist
  \item
    4
  \item
    16
  \item
    32
  \item
    128
  \end{itemize}
\item
  To enable OSPFv3, which of the following would you use?

  \begin{itemize}
  \tightlist
  \item
    Router(config-if)\#\textbf{ipv6 ospf 10 area 0.0.0.0}
  \item
    Router(config-if)\#\textbf{ipv6 router rip 1}
  \item
    Router(config)\#\textbf{ipv6 router eigrp 10}
  \item
    Router(config-rtr)\#\textbf{no shutdown}
  \item
    Router(config-if)\#\textbf{ospf ipv6 10 area 0}
  \end{itemize}
\item
  \protect\hypertarget{f_05.xhtmlux5cux23Page_lv}{}{}What does the
  command \texttt{routerA(config)\#line\ cons\ 0} allow you to perform
  next?

  \begin{itemize}
  \tightlist
  \item
    Set the Telnet password.
  \item
    Shut down the router.
  \item
    Set your console password.
  \item
    Disable console connections.
  \end{itemize}
\item
  Which two statements describe the IP address 10.16.3.65/23? (Choose
  two.)

  \begin{itemize}
  \tightlist
  \item
    The subnet address is 10.16.3.0 255.255.254.0.
  \item
    The lowest host address in the subnet is 10.16.2.1 255.255.254.0.
  \item
    The last valid host address in the subnet is 10.16.2.254
    255.255.254.0.
  \item
    The broadcast address of the subnet is 10.16.3.255 255.255.254.0.
  \item
    The network is not subnetted.
  \end{itemize}
\item
  On which interface do you configure an IP address for a switch?

  \begin{itemize}
  \tightlist
  \item
    \texttt{int\ fa0/0}
  \item
    \texttt{int\ vty\ 0\ 15}
  \item
    \texttt{int\ vlan\ 1}
  \item
    \texttt{int\ s/0/0}
  \end{itemize}
\item
  Which of the following is the valid host range for the subnet on which
  the IP address 192.168.168.188 255.255.255.192 resides?

  \begin{itemize}
  \tightlist
  \item
    192.168.168.129--190
  \item
    192.168.168.129--191
  \item
    192.168.168.128--190
  \item
    192.168.168.128--192
  \end{itemize}
\item
  Which of the following is considered to be the inside host's address
  after translation?

  \begin{itemize}
  \tightlist
  \item
    Inside local
  \item
    Outside local
  \item
    Inside global
  \item
    Outside global
  \end{itemize}
\item
  Your inside locals are not being translated to the inside global
  addresses. Which of the following commands will show you if your
  inside globals are allowed to use the NAT pool?

\begin{verbatim}
ip nat pool Corp 198.18.41.129 198.18.41.134 netmask 255.255.255.248
ip nat inside source list 100 int pool Corp overload
\end{verbatim}

  \begin{itemize}
  \tightlist
  \item
    \texttt{debug\ ip\ nat}
  \item
    \texttt{show\ access-list}
  \item
    \texttt{show\ ip\ nat\ translation}
  \item
    \texttt{show\ ip\ nat\ statistics}
  \end{itemize}
\item
  How many collision domains are created when you segment a network with
  a 12-port switch?

  \begin{itemize}
  \tightlist
  \item
    1
  \item
    2
  \item
    5
  \item
    12
  \end{itemize}
\item
  Which of the following commands will allow you to set your Telnet
  password on a Cisco router?

  \begin{itemize}
  \tightlist
  \item
    \texttt{line\ telnet\ 0\ 4}
  \item
    \texttt{line\ aux\ 0\ 4}
  \item
    \texttt{line\ vty\ 0\ 4}
  \item
    \texttt{line\ con\ 0}
  \end{itemize}
\item
  Which router command allows you to view the entire contents of all
  access lists?

  \begin{itemize}
  \tightlist
  \item
    \texttt{show\ all\ access-lists}
  \item
    \texttt{show\ access-lists}
  \item
    \texttt{show\ ip\ interface}
  \item
    \texttt{show\ interface}
  \end{itemize}
\item
  What does a VLAN do?

  \begin{itemize}
  \tightlist
  \item
    Acts as the fastest port to all servers
  \item
    Provides multiple collision domains on one switch port
  \item
    Breaks up broadcast domains in a layer 2 switch internetwork
  \item
    Provides multiple broadcast domains within a single collision domain
  \end{itemize}
\item
  If you wanted to delete the configuration stored in NVRAM, choose the
  best answer for the Cisco objectives.

  \begin{itemize}
  \tightlist
  \item
    \texttt{erase\ startup}
  \item
    \texttt{delete\ running}
  \item
    \texttt{erase\ flash}
  \item
    \texttt{erase\ running}
  \end{itemize}
\item
  Which protocol is used to send a destination network unknown message
  back to originating hosts?

  \begin{itemize}
  \tightlist
  \item
    TCP
  \item
    ARP
  \item
    ICMP
  \item
    BootP
  \end{itemize}
\item
  \protect\hypertarget{f_05.xhtmlux5cux23Page_lvii}{}{}Which class of IP
  address provides 15 bits for subnetting?

  \begin{itemize}
  \tightlist
  \item
    A
  \item
    B
  \item
    C
  \item
    D
  \end{itemize}
\item
  There are three possible routes for a router to reach a destination
  network. The first route is from OSPF with a metric of 782. The second
  route is from RIPv2 with a metric of 4. The third is from EIGRP with a
  composite metric of 20514560. Which route will be installed by the
  router in its routing table?

  \begin{itemize}
  \tightlist
  \item
    RIPv2
  \item
    EIGRP
  \item
    OSPF
  \item
    All three
  \end{itemize}
\item
  Which one of the following is true regarding VLANs?

  \begin{itemize}
  \tightlist
  \item
    Two VLANs are configured by default on all Cisco switches.
  \item
    VLANs only work if you have a complete Cisco switched internetwork.
    No off-brand switches are allowed.
  \item
    You should not have more than 10 switches in the same VTP domain.
  \item
    You need to have a trunk link configured between switches in order
    to send information about more than one VLAN down the link.
  \end{itemize}
\item
  Which two of the following commands will place network 10.2.3.0/24
  into area 0? (Choose two.)

  \begin{itemize}
  \tightlist
  \item
    \texttt{router\ eigrp\ 10}
  \item
    \texttt{router\ ospf\ 10}
  \item
    \texttt{router\ rip}
  \item
    \texttt{network\ 10.0.0.0}
  \item
    \texttt{network\ 10.2.3.0\ 255.255.255.0\ area\ 0}
  \item
    \texttt{network\ 10.2.3.0\ 0.0.0.255\ area0}
  \item
    \texttt{network\ 10.2.3.0\ 0.0.0.255\ area\ 0}
  \end{itemize}
\item
  How many broadcast domains are created when you segment a network with
  a 12-port switch?

  \begin{itemize}
  \tightlist
  \item
    1
  \item
    2
  \item
    5
  \item
    12
  \end{itemize}
\item
  \protect\hypertarget{f_05.xhtmlux5cux23Page_lviii}{}{}If routers in a
  single area are configured with the same priority value, what value
  does a router use for the OSPF router ID in the absence of a loopback
  interface?

  \begin{itemize}
  \tightlist
  \item
    The lowest IP address of any physical interface
  \item
    The highest IP address of any physical interface
  \item
    The lowest IP address of any logical interface
  \item
    The highest IP address of any logical interface
  \end{itemize}
\item
  What protocols are used to configure trunking on a switch? (Choose
  two.)

  \begin{itemize}
  \tightlist
  \item
    VLAN Trunking Protocol
  \item
    VLAN
  \item
    802.1q
  \item
    ISL
  \end{itemize}
\item
  What is a stub network?

  \begin{itemize}
  \tightlist
  \item
    A network with more than one exit point
  \item
    A network with more than one exit and entry point
  \item
    A network with only one entry and no exit point
  \item
    A network that has only one entry and exit point
  \end{itemize}
\item
  Where is a hub specified in the OSI model?

  \begin{itemize}
  \tightlist
  \item
    Session layer
  \item
    Physical layer
  \item
    Data Link layer
  \item
    Application layer
  \end{itemize}
\item
  What are the two main types of access control lists (ACLs)? (Choose
  two.)

  \begin{itemize}
  \tightlist
  \item
    Standard
  \item
    IEEE
  \item
    Extended
  \item
    Specialized
  \end{itemize}
\item
  Which of the following is the best summarization of the following
  networks: 192.168.128.0 through 192.168.159.0?

  \begin{itemize}
  \tightlist
  \item
    192.168.0.0/24
  \item
    192.168.128.0/16
  \item
    192.168.128.0/19
  \item
    192.168.128.0/20
  \end{itemize}
\item
  \protect\hypertarget{f_05.xhtmlux5cux23Page_lix}{}{}What command is
  used to create a backup configuration?

  \begin{itemize}
  \tightlist
  \item
    \texttt{copy\ running\ backup}
  \item
    \texttt{copy\ running-config\ startup-config}
  \item
    \texttt{config\ mem}
  \item
    \texttt{wr\ net}
  \end{itemize}
\item
  1000Base-T is which IEEE standard?

  \begin{itemize}
  \tightlist
  \item
    802.3f
  \item
    802.3z
  \item
    802.3ab
  \item
    802.3ae
  \end{itemize}
\item
  Which protocol does DHCP use at the Transport layer?

  \begin{itemize}
  \tightlist
  \item
    IP
  \item
    TCP
  \item
    UDP
  \item
    ARP
  \end{itemize}
\item
  If your router is facilitating a CSU/DSU, which of the following
  commands do you need to use to provide the router with a 64000 bps
  serial link?

  \begin{itemize}
  \tightlist
  \item
    \texttt{RouterA(config)\#bandwidth\ 64}
  \item
    \texttt{RouterA(config-if)\#bandwidth\ 64000}
  \item
    \texttt{RouterA(config)\#clockrate\ 64000}
  \item
    \texttt{RouterA(config-if)\#clock\ rate\ 64}
  \item
    \texttt{RouterA(config-if)\#clock\ rate\ 64000}
  \end{itemize}
\item
  Which command is used to determine if an access list is enabled on a
  particular interface?

  \begin{itemize}
  \tightlist
  \item
    \texttt{show\ access-lists}
  \item
    \texttt{show\ interface}
  \item
    \texttt{show\ ip\ interface}
  \item
    \texttt{show\ interface\ access-lists}
  \end{itemize}
\item
  Which of the following statements is true with regard to ISL and
  802.1q?

  \begin{itemize}
  \tightlist
  \item
    802.1q encapsulates the frame with control information; ISL inserts
    an ISL field along with tag control information.
  \item
    802.1q is Cisco proprietary.
  \item
    ISL encapsulates the frame with control information; 802.1q inserts
    an 802.1q field along with tag control information.
  \item
    ISL is a standard.
  \end{itemize}
\item
  \protect\hypertarget{f_05.xhtmlux5cux23Page_lx}{}{}The protocol data
  unit (PDU) encapsulation is completed in which order?

  \begin{itemize}
  \tightlist
  \item
    Bits, frames, packets, segments, data
  \item
    Data, bits, segments, frames, packets
  \item
    Data, segments, packets, frames, bits
  \item
    Packets, frames, bits, segments, data
  \end{itemize}
\item
  Based on the configuration shown below, what statement is true?

\begin{verbatim}
S1(config)#ip routing
S1(config)#int vlan 10
S1(config-if)#ip address 192.168.10.1 255.255.255.0
S1(config-if)#int vlan 20
S1(config-if)#ip address 192.168.20.1 255.255.255.0
\end{verbatim}

  \begin{itemize}
  \tightlist
  \item
    This is a multilayer switch.
  \item
    The two VLANs are in the same subnet.
  \item
    Encapsulation must be configured.
  \item
    VLAN 10 is the management VLAN.
  \end{itemize}
\end{enumerate}

\protect\hypertarget{f_06.xhtml}{}{}

\section[Answers to Assessment
Test]{\texorpdfstring{\protect\hypertarget{f_06.xhtmlux5cux23Page_lxi}{}{}Answers
to Assessment Test}{Answers to Assessment Test}}

\begin{enumerate}
\tightlist
\item
  B. To allow for the PVST+ to operate, there's a field inserted into
  the BPDU to accommodate the extended system ID so that PVST+ can have
  a root bridge configured on a per-STP instance. The extended system ID
  (VLAN ID) is a 12-bit field, and we can even see what this field is
  carrying via show spanning-tree command output. See Chapter 15 for
  more information.
\item
  A. Cisco's EtherChannel can bundle up to eight ports between switches
  to provide resiliency and more bandwidth between switches. See Chapter
  15 for more information.
\item
  B, C, E. All the ports on both sides of every link must be configured
  exactly the same between switches or it will not work. Speed, duplex,
  and allowed VLANs must match. See Chapter 15 for more information.
\item
  C. 2100 boots the router into ROM monitor mode, 2101 loads the
  mini-IOS from ROM, and 2102 is the default and loads the IOS from
  flash. See Chapter 8 for more information.
\item
  D. The \texttt{show\ license\ udi} command displays the unique device
  identifier (UDI) of the router, which comprises the product ID (PID)
  and serial number of the router. See Chapter 8 for more information.
\item
  B. The \texttt{show\ license} feature command allows you to view the
  technology package licenses and feature licenses that are supported on
  your router along with several status variables related to software
  activation and licensing, both licensed and unlicensed features. See
  Chapter 8 for more information.
\item
  C, D, F. The SDN architecture slightly differs from the architecture
  of traditional networks. It comprises three stacked layers: Data,
  Control and Application. See Chapter 8 for more information.
\item
  B. There are eight different trap levels. If you choose, for example
  level 3, level 0 through level 3 messages will be displayed. See
  Chapter 8 for more information.
\item
  B, D. Each stack of switches has a single IP address and is managed as
  a single object. This single IP management applies to activities such
  as fault detection, VLAN creation and modification, security, and QoS
  controls. Each stack has only one configuration file, which is
  distributed to each member in the stack. When you add a new switch to
  the stack, the master switch automatically configures the unit with
  the currently running IOS image and the configuration of the stack.
  You do not have to do anything to bring up the switch before it is
  ready to operate. See chapter 22 for more information.
\item
  B. There is no IPv6 default gateway listed in the output, which will
  be the link-local address of the router interface, sent to the host as
  a router advertisement. Until this host receives the router address,
  the host will communicate with IPv6 only on the local subnet. See
  Chapter 20 for more information.
\item
  D. The command \texttt{show\ ipv6\ neighbors} provides the ARP cache
  for on a router. See Chapter 20 for more information.
\item
  \protect\hypertarget{f_06.xhtmlux5cux23Page_lxii}{}{}A. If the state
  is STALE when the interface has not communicated within the neighbor
  reachable time frame. The next time the neighbor communicates, the
  state will be REACH. See Chapter 20 for more information.
\item
  C. There are two successor routes, so by default, EIGRP was
  load-balancing out s0/0 and s0/1. When s0/1 goes down, EIGRP will just
  keep forwarding traffic out the second link s0/0. s0/1 will be removed
  from the routing table. See Chapter 17 for more information.
\item
  F. There isn't a lot to go on from with the output, but the only
  commands that provide the FD and AD are \texttt{show\ ip\ eigrp}
  topology and \texttt{show\ ipv6\ eigrp\ topology}. The addresses in
  the output are link-local IPv6 addresses, so our answer is the latter.
  See Chapter 17 for more information.
\item
  A, B, C, D. Cisco has documented steps, according to the objectives,
  that you must go through when troubleshooting an adjacency. See
  Chapter 18 for more information.
\item
  B, D, E. In order for two OSPF routers to create an adjacency, the
  Hello and dead timers must match, and they must both be configured
  into the same area, as well as being in the same subnet. See Chapter
  18 for more information.
\item
  A. The process starts by sending out Hello packets. Every listening
  router will then add the originating router to the neighbor database.
  The responding routers will reply with all of their Hello information
  so that the originating router can add them to its own neighbor table.
  At this point, we will have reached the 2WAY state---only certain
  routers will advance beyond to this. See Chapter 19 for more
  information.
\item
  C. Referred to as summary link advertisements (SLAs), Type 3 LSAs are
  generated by area border routers. These ABRs send Type 3 LSAs toward
  the area external to the one where they were generated. See Chapter 19
  for more information.
\item
  B. Authentication Header (AH) provides authentication of either all or
  part of the IP packet through the addition of a header that is
  calculated based on the values in the packet, but it doesn't offer any
  encryption services. See Chapter 21 for more information.
\item
  B. Generic Routing Encapsulation (GRE) has no built-in security
  mechanisms. See Chapter 21 for more information.
\item
  C. When traffic exceeds the allocated rate, the policer can take one
  of two actions. It can either drop traffic or re-mark it to another
  class of service. The new class usually has a higher drop probability.
  See Chapter 21 for more information.
\item
  B. This can be a hard question if you don't remember to invert the 7th
  bit of the first octet in the MAC address! Always look for the 7th bit
  when studying for the Cisco R/S, and when using eui-64, invert it. The
  eui-64 autoconfiguration then inserts an FF:FE in the middle of the
  48-bit MAC address to create a unique IPv6 address. See Chapter 14 for
  more information.
\item
  A. The NDP neighbor advertisement (NA) contains the MAC address. A
  neighbor solicitation (NS) was initially sent asking for the MAC
  address. See Chapter 14 for more information.
\item
  \protect\hypertarget{f_06.xhtmlux5cux23Page_lxiii}{}{}B. Each field in
  an IPv6 address is 16 bits long. An IPv6 address is a total of 128
  bits. See Chapter 14 for more information.
\item
  A. To enable OSPFv3, you enable the protocol at the interface level,
  as with RIPng. The command string is area-id. It's important to
  understand that area \texttt{0} and area \texttt{0.0.0.0} both
  describe area \texttt{0}. See Chapter 19 for more information.
\item
  C. The command line console \texttt{0} places you at a prompt where
  you can then set your console user-mode password. See Chapter 6 for
  more information.
\item
  B, D. The mask 255.255.254.0 (/23) used with a Class A address means
  that there are 15 subnet bits and 9 host bits. The block size in the
  third octet is 2 (256--254). So this makes the subnets in the
  interesting octet 0, 2, 4, 6, etc., all the way to 254. The host
  10.16.3.65 is in the 2.0 subnet. The next subnet is 4.0, so the
  broadcast address for the 2.0 subnet is 3.255. The valid host
  addresses are 2.1 through 3.254. See Chapter 4 for more information.
\item
  C. The IP address is configured under a logical interface, called a
  management domain or VLAN 1, by default. See Chapter 10 for more
  information.
\item
  A. 256 -- 192 = 64, so 64 is our block size. Just count in increments
  of 64 to find our subnet: 64 + 64 = 128. 128 + 64 = 192. The subnet is
  128, the broadcast address is 191, and the valid host range is the
  numbers in between, or 129--190. See Chapter 4 for more information.
\item
  C. An inside global address is considered to be the IP address of the
  host on the private network after translation. See Chapter 13 for more
  information.
\item
  B. Once you create your pool, the command ip nat inside source must be
  used to say which inside locals are allowed to use the pool. In this
  question, we need to see if access list 100 is configured correctly,
  if at all, so \texttt{show\ access-list} is the best answer. See
  Chapter 13 for more information.
\item
  D. Layer 2 switching creates individual collision domains per port.
  See Chapter 1 for more information.
\item
  C. The command line vty 0 4 places you in a prompt that will allow you
  to set or change your Telnet password. See Chapter 6 for more
  information.
\item
  B. To see the contents of all access lists, use the
  \texttt{show\ access-lists} command. See Chapter 12 for more
  information.
\item
  C. VLANs break up broadcast domains at layer 2. See Chapter 11 for
  more information.
\item
  A. The command \texttt{erase\ startup-config} deletes the
  configuration stored in NVRAM. See Chapter 6 for more information.
\item
  C. ICMP is the protocol at the Network layer that is used to send
  messages back to an originating router. See Chapter 3 for more
  information.
\item
  A. Class A addressing provides 22 bits for host subnetting. Class B
  provides 16 bits, but only 14 are available for subnetting. Class C
  provides only 6 bits for subnetting. See Chapter 3 for more
  information.
\item
  \protect\hypertarget{f_06.xhtmlux5cux23Page_lxiv}{}{}B. Only the EIGRP
  route will be placed in the routing table because EIGRP has the lowest
  administrative distance (AD), and that is always used before metrics.
  See Chapter 8 for more information.
\item
  D. Switches send information about only one VLAN down a link unless it
  is configured as a trunk link. See Chapter 11 for more information.
\item
  B, G. To enable OSPF, you must first start OSPF using a process ID.
  The number is irrelevant; just choose a number from 1 to 65,535 and
  you're good to go. After you start the OSPF process, you must
  configure interfaces on which to activate OSPF using the network
  command with wildcards and specification of an area. Option F is wrong
  because there must be a space after the parameter area and before you
  list the area number. See Chapter 9 for more information.
\item
  A. By default, switches break up collision domains on a per-port basis
  but are one large broadcast domain. See Chapter 1 for more
  information.
\item
  B. At the moment of OSPF process startup, the highest IP address on
  any active interface will be the router ID (RID) of the router. If you
  have a loopback interface configured (logical interface), then that
  will override the interface IP address and become the RID of the
  router automatically. See Chapter 18 for more information.
\item
  C, D. VLAN Trunking Protocol (VTP) is not right because it has nothing
  to do with trunking except that it sends VLAN information across a
  trunk link. 802.1q and ISL encapsulations are used to configure
  trunking on a port. See Chapter 11 for more information.
\item
  D. Stub networks have only one connection to an internetwork. Default
  routes should be set on a stub network or network loops may occur;
  however, there are exceptions to this rule. See Chapter 9 for more
  information.
\item
  B. Hubs regenerate electrical signals, which are specified at the
  Physical layer. See Chapter 1 for more information.
\item
  A, C. Standard and extended access control lists (ACLs) are used to
  configure security on a router. See Chapter 12 for more information.
\item
  C. If you start at 192.168.128.0 and go through 192.168.159.0, you can
  see that this is a block of 32 in the third octet. Since the network
  address is always the first one in the range, the summary address is
  192.168.128.0. What mask provides a block of 32 in the third octet?
  The answer is 255.255.224.0, or /19. See Chapter 5 for more
  information.
\item
  B. The command to back up the configuration on a router is
  \texttt{copy\ running-config\ startup-config}. See Chapter 7 for more
  information.
\item
  C. IEEE 802.3ab is the standard for 1 Gbps on twisted-pair. See
  Chapter 2 for more information.
\item
  C. User Datagram Protocol is a connection network service at the
  Transport layer, and DHCP uses this connectionless service. See
  Chapter 3 for more information
\item
  \protect\hypertarget{f_06.xhtmlux5cux23Page_lxv}{}{}E. The clock rate
  command is two words, and the speed of the line is in bits per second
  (bps). See Chapter 6 for more information.
\item
  C. The \texttt{show\ ip\ interface} command will \texttt{show} you if
  any interfaces have an outbound or inbound access list set. See
  Chapter 12 for more information.
\item
  C. Unlike ISL, which encapsulates the frame with control information,
  802.1q inserts an 802.1q field along with tag control information. See
  Chapter 11 for more information.
\item
  C. The PDU encapsulation method defines how data is encoded as it goes
  through each layer of the TCP/IP model. Data is segmented at the
  Transport later, packets created at the Network layer, frames at the
  Data Link layer, and finally, the Physical layer encodes the 1s and 0s
  into a digital signal. See Chapter 2 for more information.
\item
  A. With a multilayer switch, enable IP routing and create one logical
  interface for each VLAN using the interface vlan number command and
  you're now doing inter-VLAN routing on the backplane of the switch!
  See Chapter 11 for more information.
\end{enumerate}

\protect\hypertarget{p01.xhtml}{}{}

\section[{Part 1}\\
{ICND1}]{\texorpdfstring{\protect\hypertarget{p01.xhtmlux5cux23p01}{}{}\protect\hypertarget{p01.xhtmlux5cux23Page_1}{}{}{Part
1}\\
{ICND1}}{Part 1 ICND1}}

\protect\hypertarget{c01.xhtml}{}{}

\section[{Chapter 1}\\
{Internetworking~}]{\texorpdfstring{\protect\hypertarget{c01.xhtmlux5cux23c01}{}{}\protect\hypertarget{c01.xhtmlux5cux23Page_3}{}{}{Chapter
1}\\
{Internetworking~}}{Chapter 1 Internetworking~}}

\begin{center}\rule{0.5\linewidth}{0.5pt}\end{center}

\subsection{THE FOLLOWING ICND1 EXAM TOPICS ARE COVERED IN THIS
CHAPTER:}

\begin{enumerate}
\tightlist
\item
  \includegraphics{images/right.png} \textbf{Network Fundamentals}

  \begin{enumerate}
  \tightlist
  \item
    \includegraphics{images/squ.png} 1.3 Describe the impact of
    infrastructure components in an enterprise network

    \begin{enumerate}
    \tightlist
    \item
      \includegraphics{images/squ.png} 1.3.a Firewalls
    \item
      \includegraphics{images/squ.png} 1.3.b Access points
    \item
      \includegraphics{images/squ.png} 1.3.c Wireless controllers
    \end{enumerate}
  \item
    \includegraphics{images/squ.png} 1.5 Compare and contrast network
    topologies

    \begin{enumerate}
    \tightlist
    \item
      \includegraphics{images/squ.png} 1.5.a Star
    \item
      \includegraphics{images/squ.png} 1.5.b Mesh
    \item
      \includegraphics{images/squ.png} 1.5.c Hybrid
    \end{enumerate}
  \end{enumerate}
\end{enumerate}

\protect\hypertarget{c01.xhtmlux5cux23Page_4}{}{}\includegraphics{images/intro.png}
Welcome to the exciting world of internetworking. This first chapter
will serve as an internetworking review by focusing on how to connect
networks together using Cisco routers and switches, and I've written it
with the assumption that you have some simple basic networking
knowledge. The emphasis of this review will be on the Cisco CCENT and/or
CCNA Routing and Switching (CCNA R/S) objectives, on which you'll need a
solid grasp in order to succeed in getting your certifications.

Let's start by defining exactly what an internetwork is: You create an
internetwork when you connect two or more networks via a router and
configure a logical network addressing scheme with a protocol such as IP
or IPv6.

We'll also dissect the Open Systems Interconnection (OSI) model, and
I'll describe each part of it to you in detail because you really need
complete, reliable knowledge of it. Understanding the OSI model is key
for the solid foundation you'll need to build upon with the more
advanced Cisco networking knowledge gained as you become increasingly
more skilled.

The OSI model has seven hierarchical layers that were developed to
enable different networks to communicate reliably between disparate
systems. Since this book is centering upon all things CCNA, it's crucial
for you to understand the OSI model as Cisco sees it, so that's how I'll
be presenting the seven layers to you.

After you finish reading this chapter, you'll encounter review questions
and written labs. These are given to you to really lock the information
from this chapter into your memory. So don't skip them!

\begin{center}\rule{0.5\linewidth}{0.5pt}\end{center}

\includegraphics{images/note.png} To find up-to-the-minute updates for
this chapter, please see
\href{http://www.lammle.com/ccna}{www.lammle.com/ccna} or the book's web
page via \href{http://www.sybex.com/go/ccna}{www.sybex.com/go/ccna}.

\begin{center}\rule{0.5\linewidth}{0.5pt}\end{center}

\subsection[Internetworking
Basics]{\texorpdfstring{\protect\hypertarget{c01.xhtmlux5cux23c01-sec-1}{}{}Internetworking
Basics}{Internetworking Basics}}

Before exploring internetworking models and the OSI model's
specifications, you need to grasp the big picture and the answer to this
burning question: Why is it so important to learn Cisco internetworking
anyway?

Networks and networking have grown exponentially over the past 20 years,
and understandably so. They've had to evolve at light speed just to keep
up with huge increases in basic, mission-critical user needs (e.g., the
simple sharing of data and printers) as well as greater burdens like
multimedia remote presentations and conferencing. Unless everyone
\protect\hypertarget{c01.xhtmlux5cux23Page_5}{}{}who needs to share
network resources is located in the same office space---an increasingly
uncommon situation---the challenge is to connect relevant networks so
all users can share the wealth of whatever services and resources are
required.

\protect\hyperlink{c01.xhtmlux5cux23figure01-1}{Figure 1.1} shows a
basic \emph{local area network (LAN)} that's connected using a
\emph{hub}, which is basically just an antiquated device that connects
wires together. Keep in mind that a simple network like this would be
considered one collision domain and one broadcast domain. No worries if
you have no idea what I mean by that because coming up soon, I'm going
to talk about collision and broadcast domains enough to make you dream
about them!

\begin{figure}
\centering
\includegraphics{images/c01f001.jpg}
\caption{{\protect\hyperlink{c01.xhtmlux5cux23figureanchor01-1}{\textbf{FIGURE
1.1}} A very basic network}}
\end{figure}

Things really can't get much simpler than this. And yes, though you can
still find this configuration in some home networks, even many of those
as well as the smallest business networks are more complicated today. As
we move through this book, I'll just keep building upon this tiny
network a bit at a time until we arrive at some really nice, robust, and
current network designs---the types that will help you get your
certification and a job!

But as I said, we'll get there one step at a time, so let's get back to
the network shown in
\protect\hyperlink{c01.xhtmlux5cux23figure01-1}{Figure 1.1} with this
scenario: Bob wants to send Sally a file, and to complete that goal in
this kind of network, he'll simply broadcast that he's looking for her,
which is basically just shouting out over the network. Think of it like
this: Bob walks out of his house and yells down a street called Chaos
Court in order to contact Sally. This might work if Bob and Sally were
the only ones living there, but not so much if it's crammed with homes
and all the others living there are always hollering up and down the
street to their neighbors just like Bob. Nope, Chaos Court would
absolutely live up to its name, with all those residents going off
whenever they felt like it---and believe it or not, our networks
actually still work this way to a degree! So, given a choice, would you
stay in Chaos Court, or would you pull up stakes and move on over to a
nice new modern community called Broadway Lanes, which offers plenty of
amenities and room for your home plus future additions all on nice, wide
streets that can easily handle all present and future traffic? If you
chose the latter, good choice\ldots{} so did Sally, and she now lives a
much quieter life, getting letters (packets) from Bob instead of a
headache!

The scenario I just described brings me to the basic point of what this
book and the Cisco certification objectives are really all about. My
goal of showing you how to create efficient networks and segment them
correctly in order to minimize all the chaotic yelling and screaming
going on in them is a universal theme throughout my CCENT and CCNA
series books. It's just inevitable that you'll have to break up a large
network into a bunch of smaller
\protect\hypertarget{c01.xhtmlux5cux23Page_6}{}{}ones at some point to
match a network's equally inevitable growth, and as that expansion
occurs, user response time simultaneously dwindles to a frustrating
crawl. But if you master the vital technology and skills I have in store
for you in this series, you'll be well equipped to rescue your network
and its users by creating an efficient new network neighborhood to give
them key amenities like the bandwidth they need to meet their evolving
demands.

And this is no joke; most of us think of growth as good---and it can
be---but as many of us experience daily when commuting to work, school,
etc., it can also mean your LAN's traffic congestion can reach critical
mass and grind to a complete halt! Again, the solution to this problem
begins with breaking up a massive network into a number of smaller
ones---something called \emph{network segmentation}. This concept is a
lot like planning a new community or modernizing an existing one. More
streets are added, complete with new intersections and traffic signals,
plus post offices are built with official maps documenting all those
street names and directions on how to get to each. You'll need to effect
new laws to keep order to it all and provide a police station to protect
this nice new neighborhood as well. In a networking neighborhood
environment, all of this is carried out using devices like
\emph{routers, switches}, and \emph{bridges}.

So let's take a look at our new neighborhood now, because the word has
gotten out; many more hosts have moved into it, so it's time to upgrade
that new high-capacity infrastructure that we promised to handle the
increase in population.
\protect\hyperlink{c01.xhtmlux5cux23figure01-2}{Figure 1.2} shows a
network that's been segmented with a switch, making each network segment
that connects to the switch its own separate collision domain. Doing
this results in a lot less yelling!

\begin{figure}
\centering
\includegraphics{images/c01f002.jpg}
\caption{{\protect\hyperlink{c01.xhtmlux5cux23figureanchor01-2}{\textbf{FIGURE
1.2}} A switch can break up collision domains.}}
\end{figure}

This is a great start, but I really want you to make note of the fact
that this network is still one, single broadcast domain, meaning that
we've really only decreased our screaming and yelling, not eliminated
it. For example, if there's some sort of vital announcement that
everyone in our neighborhood needs to hear about, it will definitely
still get loud! You can see that the hub used in
\protect\hyperlink{c01.xhtmlux5cux23figure01-2}{Figure 1.2} just
extended the one collision domain from the switch port. The result is
that John received the data from Bob but, happily, Sally did not. This
is good because Bob intended to talk with John directly, and if he had
needed to send a broadcast instead, everyone, including Sally, would
have received it, possibly causing unnecessary congestion.

Here's a list of some of the things that commonly cause LAN traffic
congestion:

\begin{enumerate}
\tightlist
\item
  Too many hosts in a collision or broadcast domain
\item
  Broadcast storms
\item
  \protect\hypertarget{c01.xhtmlux5cux23Page_7}{}{} Too much multicast
  traffic
\item
  Low bandwidth
\item
  Adding hubs for connectivity to the network
\item
  A bunch of ARP broadcasts
\end{enumerate}

Take another look at
\protect\hyperlink{c01.xhtmlux5cux23figure01-2}{Figure 1.2} and make
sure you see that I extended the main hub from
\protect\hyperlink{c01.xhtmlux5cux23figure01-1}{Figure 1.1} to a switch
in \protect\hyperlink{c01.xhtmlux5cux23figure01-2}{Figure 1.2}. I did
that because hubs don't segment a network; they just connect network
segments. Basically, it's an inexpensive way to connect a couple of PCs,
and again, that's great for home use and troubleshooting, but that's
about it!

As our planned community starts to grow, we'll need to add more streets
with traffic control, and even some basic security. We'll achieve this
by adding routers because these convenient devices are used to connect
networks and route packets of data from one network to another. Cisco
became the de facto standard for routers because of its unparalleled
selection of high-quality router products and fantastic service. So
never forget that by default, routers are basically employed to
efficiently break up a \emph{broadcast domain}---the set of all devices
on a network segment, which are allowed to ``hear'' all broadcasts sent
out on that specific segment.

\protect\hyperlink{c01.xhtmlux5cux23figure01-3}{Figure 1.3} depicts a
router in our growing network, creating an internetwork and breaking up
broadcast domains.

\begin{figure}
\centering
\includegraphics{images/c01f003.jpg}
\caption{{\protect\hyperlink{c01.xhtmlux5cux23figureanchor01-3}{\textbf{FIGURE
1.3}} Routers create an internetwork.}}
\end{figure}

The network in \protect\hyperlink{c01.xhtmlux5cux23figure01-3}{Figure
1.3} is actually a pretty cool little network. Each host is connected to
its own collision domain because of the switch, and the router has
created two broadcast domains. So now our Sally is happily living in
peace in a completely different neighborhood, no longer subjected to
Bob's incessant shouting! If Bob wants to talk with Sally, he has to
send a packet with a destination address using her IP address---he
cannot broadcast for her!

But there's more\ldots{} routers provide connections to \emph{wide area
network (WAN)} services as well via a serial interface for WAN
connections---specifically, a V.35 physical interface on a Cisco router.

Let me make sure you understand why breaking up a broadcast domain is so
important. When a host or server sends a network broadcast, every device
on the network must read and process that broadcast---unless you have a
router. When the router's interface receives this broadcast, it can
respond by basically saying, ``Thanks, but no thanks,'' and discard the
broadcast without forwarding it on to other networks. Even though
routers are known for breaking up broadcast domains by default, it's
important to remember that they break up collision domains as well.

\protect\hypertarget{c01.xhtmlux5cux23Page_8}{}{}There are two
advantages to using routers in your network:

\begin{enumerate}
\tightlist
\item
  They don't forward broadcasts by default.
\item
  They can filter the network based on layer 3 (Network layer)
  information such as an IP address.
\end{enumerate}

Here are four ways a router functions in your network:

\begin{enumerate}
\tightlist
\item
  Packet switching
\item
  Packet filtering
\item
  Internetwork communication
\item
  Path selection
\end{enumerate}

I'll tell you all about the various layers later in this chapter, but
for now, it's helpful to think of routers as layer 3 switches. Unlike
plain-vanilla layer 2 switches, which forward or filter frames, routers
(layer 3 switches) use logical addressing and provide an important
capacity called \emph{packet switching}. Routers can also provide packet
filtering via access lists, and when routers connect two or more
networks together and use logical addressing (IP or IPv6), you then have
an \emph{internetwork}. Finally, routers use a routing table, which is
essentially a map of the internetwork, to make best path selections for
getting data to its proper destination and properly forward packets to
remote networks.

Conversely, we don't use layer 2 switches to create internetworks
because they don't break up broadcast domains by default. Instead,
they're employed to add functionality to a network LAN. The main purpose
of these switches is to make a LAN work better---to optimize its
performance---providing more bandwidth for the LAN's users. Also, these
switches don't forward packets to other networks like routers do.
Instead, they only ``switch'' frames from one port to another within the
switched network. And don't worry, even though you're probably thinking,
``Wait---what are frames and packets?'' I promise to completely fill you
in later in this chapter. For now, think of a packet as a package
containing data.

Okay, so by default, switches break up collision domains, but what are
these things? \emph{Collision domain} is an Ethernet term used to
describe a network scenario in which one device sends a packet out on a
network segment and every other device on that same segment is forced to
pay attention no matter what. This isn't very efficient because if a
different device tries to transmit at the same time, a collision will
occur, requiring both devices to retransmit, one at a time---not good!
This happens a lot in a hub environment, where each host segment
connects to a hub that represents only one collision domain and a single
broadcast domain. By contrast, each and every port on a switch
represents its own collision domain, allowing network traffic to flow
much more smoothly.

\begin{center}\rule{0.5\linewidth}{0.5pt}\end{center}

\includegraphics{images/note.png} Switches create separate collision
domains within a single broadcast domain. Routers provide a separate
broadcast domain for each interface. Don't let this ever confuse you!

\begin{center}\rule{0.5\linewidth}{0.5pt}\end{center}

The term \emph{bridging} was introduced before routers and switches were
implemented, so it's pretty common to hear people referring to switches
as bridges. That's because bridges and
\protect\hypertarget{c01.xhtmlux5cux23Page_9}{}{}switches basically do
the same thing---break up collision domains on a LAN. Note to self that
you cannot buy a physical bridge these days, only LAN switches, which
use bridging technologies. This does not mean that you won't still hear
Cisco and others refer to LAN switches as multiport bridges now and
then.

But does it mean that a switch is just a multiple-port bridge with more
brainpower? Well, pretty much, only there are still some key
differences. Switches do provide a bridging function, but they do that
with greatly enhanced management ability and features. Plus, most
bridges had only 2 or 4 ports, which is severely limiting. Of course, it
was possible to get your hands on a bridge with up to 16 ports, but
that's nothing compared to the hundreds of ports available on some
switches!

\begin{center}\rule{0.5\linewidth}{0.5pt}\end{center}

\includegraphics{images/note.png} You would use a bridge in a network to
reduce collisions within broadcast domains and to increase the number of
collision domains in your network. Doing this provides more bandwidth
for users. And never forget that using hubs in your Ethernet network can
contribute to congestion. As always, plan your network design carefully!

\begin{center}\rule{0.5\linewidth}{0.5pt}\end{center}

\protect\hyperlink{c01.xhtmlux5cux23figure01-4}{Figure 1.4} shows how a
network would look with all these internetwork devices in place.
Remember, a router doesn't just break up broadcast domains for every LAN
interface, it breaks up collision domains too.

\begin{figure}
\centering
\includegraphics{images/c01f004.jpg}
\caption{{\protect\hyperlink{c01.xhtmlux5cux23figureanchor01-4}{\textbf{FIGURE
1.4}} Internetworking devices}}
\end{figure}

\protect\hypertarget{c01.xhtmlux5cux23Page_10}{}{}Looking at
\protect\hyperlink{c01.xhtmlux5cux23figure01-4}{Figure 1.4}, did you
notice that the router has the center stage position and connects each
physical network together? I'm stuck with using this layout because of
the ancient bridges and hubs involved. I really hope you don't run
across a network like this, but it's still really important to
understand the strategic ideas that this figure represents!

See that bridge up at the top of our internetwork shown in
\protect\hyperlink{c01.xhtmlux5cux23figure01-4}{Figure 1.4}? It's there
to connect the hubs to a router. The bridge breaks up collision domains,
but all the hosts connected to both hubs are still crammed into the same
broadcast domain. That bridge also created only three collision domains,
one for each port, which means that each device connected to a hub is in
the same collision domain as every other device connected to that same
hub. This is really lame and to be avoided if possible, but it's still
better than having one collision domain for all hosts! So don't do this
at home; it's a great museum piece and a wonderful example of what not
to do, but this inefficient design would be terrible for use in today's
networks! It does show us how far we've come though, and again, the
foundational concepts it illustrates are really important for you to
get.

And I want you to notice something else: The three interconnected hubs
at the bottom of the figure also connect to the router. This setup
creates one collision domain and one broadcast domain and makes that
bridged network, with its two collision domains, look majorly better by
contrast!

\begin{center}\rule{0.5\linewidth}{0.5pt}\end{center}

\includegraphics{images/note.png} Don't misunderstand\ldots{}
bridges/switches are used to segment networks, but they will not isolate
broadcast or multicast packets.

\begin{center}\rule{0.5\linewidth}{0.5pt}\end{center}

The best network connected to the router is the LAN switched network on
the left. Why? Because each port on that switch breaks up collision
domains. But it's not all good---all devices are still in the same
broadcast domain. Do you remember why this can be really bad? Because
all devices must listen to all broadcasts transmitted, that's why! And
if your broadcast domains are too large, the users have less bandwidth
and are required to process more broadcasts. Network response time
eventually will slow to a level that could cause riots and strikes, so
it's important to keep your broadcast domains small in the vast majority
of networks today.

Once there are only switches in our example network, things really
change a lot! \protect\hyperlink{c01.xhtmlux5cux23figure01-5}{Figure
1.5} demonstrates a network you'll typically stumble upon today.

\begin{figure}
\centering
\includegraphics{images/c01f005.jpg}
\caption{{\protect\hyperlink{c01.xhtmlux5cux23figureanchor01-5}{\textbf{FIGURE
1.5}} Switched networks creating an internetwork}}
\end{figure}

\protect\hypertarget{c01.xhtmlux5cux23Page_11}{}{}Here I've placed the
LAN switches at the center of this network world, with the router
connecting the logical networks. If I went ahead and implemented this
design, I'll have created something called virtual LANs, or VLANs, which
are used when you logically break up broadcast domains in a layer 2,
switched network. It's really important to understand that even in a
switched network environment, you still need a router to provide
communication between VLANs. Don't forget that!

Still, clearly the best network design is the one that's perfectly
configured to meet the business requirements of the specific company or
client it serves, and it's usually one in which LAN switches exist in
harmony with routers strategically placed in the network. It's my hope
that this book will help you understand the basics of routers and
switches so you can make solid, informed decisions on a case-by-case
basis and be able to achieve that goal! But I digress\ldots{}

So let's go back to
\protect\hyperlink{c01.xhtmlux5cux23figure01-4}{Figure 1.4} now for a
minute and really scrutinize it because I want to ask you this question:
How many collision domains and broadcast domains are really there in
this internetwork? I hope you answered nine collision domains and three
broadcast domains! The broadcast domains are definitely the easiest to
spot because only routers break up broadcast domains by default, and
since there are three interface connections, that gives you three
broadcast domains. But do you see the nine collision domains? Just in
case that's a no, I'll explain. The all-hub network at the bottom is one
collision domain; the bridge network on top equals three collision
domains. Add in the switch network of five collision domains---one for
each switch port---and you get a total of nine!

While we're at this, in
\protect\hyperlink{c01.xhtmlux5cux23figure01-5}{Figure 1.5}, each port
on the switch is a separate collision domain, and each VLAN would be a
separate broadcast domain. So how many collision domains do you see
here? I'm counting 12---remember that connections between the switches
are considered a collision domain! Since the figure doesn't show any
VLAN information, we can assume the default of one broadcast domain is
in place.

Before we move on to Internetworking Models, let's take a look at a few
more network devices that we'll find in pretty much every network today
as shown in \protect\hyperlink{c01.xhtmlux5cux23figure01-6}{Figure 1.6}.

\begin{figure}
\centering
\includegraphics{images/c01f006.jpg}
\caption{{\protect\hyperlink{c01.xhtmlux5cux23figureanchor01-6}{\textbf{FIGURE
1.6}} Other devices typically found in our internetworks today.}}
\end{figure}

\protect\hypertarget{c01.xhtmlux5cux23Page_12}{}{}Taking off from the
switched network in
\protect\hyperlink{c01.xhtmlux5cux23figure01-5}{Figure 1.5}, you'll find
WLAN devices, including AP's and wireless controllers, and firewalls.
You'd be hard pressed not to find these devices in your networks today.

Let's look closer at these devices:

\begin{enumerate}
\tightlist
\item
  WLAN devices: These devices connect wireless devices such as
  computers, printers, and tablets to the network. Since pretty much
  every device manufactured today has a wireless NIC, you just need to
  configure a basic access point (AP) to connect to a traditional wired
  network.
\item
  Access Points or APs: These devices allow wireless devices to connect
  to a wired network and extend a collision domain from a switch, and
  are typically in their own broadcast domain or what we'll refer to as
  a Virtual LAN (VLAN). An AP can be a simple standalone device, but
  today they are usually managed by wireless controllers either in house
  or through the internet.
\item
  WLAN Controllers: These are the devices that network administrators or
  network operations centers use to manage access points in medium to
  large to extremely large quantities. The WLAN controller automatically
  handles the configuration of wireless access points and was typically
  used only in larger enterprise systems. However, with Cisco's
  acquisition of Meraki systems, you can easily manage a small to medium
  sized wireless network via the cloud using their simple to configure
  web controller system.
\item
  Firewalls: These devices are network security systems that monitor and
  control the incoming and outgoing network traffic based on
  predetermined security rules, and is usually an Intrusion Protection
  System (IPS). Cisco Adaptive Security Appliance (ASA) firewall
  typically establishes a barrier between a trusted, secure internal
  network and the Internet, which is not secure or trusted. Cisco's new
  acquisition of Sourcefire put them in the top of the market with Next
  Generation Firewalls (NGFW) and Next Generation IPS (NGIPS), which
  Cisco now just calls Firepower. Cisco new Firepower runs on dedicated
  appliances, Cisco's ASA's, ISR routers and even on Meraki products.
\end{enumerate}

\begin{center}\rule{0.5\linewidth}{0.5pt}\end{center}

\subsubsection[\hfill\break
Should I Replace My Existing 10/100 Mbps
Switches?]{\texorpdfstring{\protect\includegraphics{images/earth.png}\\
Should I Replace My Existing 10/100 Mbps
Switches?}{ Should I Replace My Existing 10/100 Mbps Switches?}}

Let's say you're a network administrator at a large company. The boss
comes to you and says that he got your requisition to buy a bunch of new
switches but he's really freaking out about the price tag! Should you
push it---do you really need to go this far?

Absolutely! Make your case and go for it because the newest switches add
really huge capacity to a network that older 10/100 Mbps switches just
can't touch. And yes, five-year-old switches are considered pretty
Pleistocene these days. But in reality, most of us just don't have an
unlimited budget to buy all new gigabit switches; however, 10/100
switches are just not good enough in today's networks.

\protect\hypertarget{c01.xhtmlux5cux23Page_13}{}{}Another good question:
Do you really need low-latency 1 Gbps or better switch ports for all
your users, servers, and other devices? Yes, you \emph{absolutely} need
new higher-end switches! This is because servers and hosts are no longer
the bottlenecks of our internetworks, our routers and switches
are---especially legacy ones. We now need gigabit on the desktop and on
every router interface; 10 Gbps is now the minimum between switch
uplinks, so go to 40 or even 100 Gbps as uplinks if you can afford it.

Go ahead. Put in that requisition for all new switches. You'll be a hero
before long!

\begin{center}\rule{0.5\linewidth}{0.5pt}\end{center}

Okay, so now that you've gotten a pretty thorough introduction to
internetworking and the various devices that populate an internetwork,
it's time to head into exploring the internetworking models.

\subsection[Internetworking
Models]{\texorpdfstring{\protect\hypertarget{c01.xhtmlux5cux23c01-sec-2}{}{}Internetworking
Models}{Internetworking Models}}

First a little history: When networks first came into being, computers
could typically communicate only with computers from the same
manufacturer. For example, companies ran either a complete DECnet
solution or an IBM solution, never both together. In the late 1970s, the
\emph{Open Systems Interconnection (OSI) reference model} was created by
the International Organization for Standardization (ISO) to break
through this barrier.

The OSI model was meant to help vendors create interoperable network
devices and software in the form of protocols so that different vendor
networks could work in peaceable accord with each other. Like world
peace, it'll probably never happen completely, but it's still a great
goal!

Anyway the OSI model is the primary architectural model for networks. It
describes how data and network information are communicated from an
application on one computer through the network media to an application
on another computer. The OSI reference model breaks this approach into
layers.

Coming up, I'll explain the layered approach to you plus how we can use
it to help us troubleshoot our internetworks.

\begin{center}\rule{0.5\linewidth}{0.5pt}\end{center}

\includegraphics{images/tip.png} Goodness! ISO, OSI, and soon you'll
hear about IOS! Just remember that the ISO created the OSI and that
Cisco created the Internetworking Operating System (IOS), which is what
this book is all-so-about.

\begin{center}\rule{0.5\linewidth}{0.5pt}\end{center}

\subsubsection[The Layered
Approach]{\texorpdfstring{\protect\hypertarget{c01.xhtmlux5cux23c01-sec-3}{}{}The
Layered Approach}{The Layered Approach}}

Understand that a \emph{reference model} is a conceptual blueprint of
how communications should take place. It addresses all the processes
required for effective communication and divides them into logical
groupings called \emph{layers}. When a communication system is designed
in this manner, it's known as a hierarchical or \emph{layered
architecture}.

\protect\hypertarget{c01.xhtmlux5cux23Page_14}{}{}Think of it like this:
You and some friends want to start a company. One of the first things
you'll do is sort out every task that must be done and decide who will
do what. You would move on to determine the order in which you would
like everything to be done with careful consideration of how all your
specific operations relate to each other. You would then organize
everything into departments (e.g., sales, inventory, and shipping), with
each department dealing with its specific responsibilities and keeping
its own staff busy enough to focus on their own particular area of the
enterprise.

In this scenario, departments are a metaphor for the layers in a
communication system. For things to run smoothly, the staff of each
department has to trust in and rely heavily upon those in the others to
do their jobs well. During planning sessions, you would take notes,
recording the entire process to guide later discussions and clarify
standards of operation, thereby creating your business blueprint---your
own reference model.

And once your business is launched, your department heads, each armed
with the part of the blueprint relevant to their own department, will
develop practical ways to implement their distinct tasks. These
practical methods, or protocols, will then be compiled into a standard
operating procedures manual and followed closely because each procedure
will have been included for different reasons, delimiting their various
degrees of importance and implementation. All of this will become vital
if you form a partnership or acquire another company because then it
will be really important that the new company's business model is
compatible with yours!

Models happen to be really important to software developers too. They
often use a reference model to understand computer communication
processes so they can determine which functions should be accomplished
on a given layer. This means that if someone is creating a protocol for
a certain layer, they only need to be concerned with their target
layer's function. Software that maps to another layer's protocols and is
specifically designed to be deployed there will handle additional
functions. The technical term for this idea is \emph{binding}. The
communication processes that are related to each other are bound, or
grouped together, at a particular layer.

\subsubsection[Advantages of Reference
Models]{\texorpdfstring{\protect\hypertarget{c01.xhtmlux5cux23c01-sec-4}{}{}Advantages
of Reference Models}{Advantages of Reference Models}}

The OSI model is hierarchical, and there are many advantages that can be
applied to any layered model, but as I said, the OSI model's primary
purpose is to allow different vendors' networks to interoperate.

Here's a list of some of the more important benefits of using the OSI
layered model:

\begin{enumerate}
\tightlist
\item
  It divides the network communication process into smaller and simpler
  components, facilitating component development, design, and
  troubleshooting.
\item
  It allows multiple-vendor development through the standardization of
  network components.
\item
  It encourages industry standardization by clearly defining what
  functions occur at each layer of the model.
\item
  It allows various types of network hardware and software to
  communicate.
\item
  It prevents changes in one layer from affecting other layers to
  expedite development.
\end{enumerate}

\subsection[The OSI Reference
Model]{\texorpdfstring{\protect\hypertarget{c01.xhtmlux5cux23c01-sec-5}{}{}\protect\hypertarget{c01.xhtmlux5cux23Page_15}{}{}The
OSI Reference Model}{The OSI Reference Model}}

One of best gifts the OSI specifications gives us is paving the way for
the data transfer between disparate hosts running different operating
systems, like Unix hosts, Windows machines, Macs, smartphones, and so
on.

And remember, the OSI is a logical model, not a physical one. It's
essentially a set of guidelines that developers can use to create and
implement applications to run on a network. It also provides a framework
for creating and implementing networking standards, devices, and
internetworking schemes.

The OSI has seven different layers, divided into two groups. The top
three layers define how the applications within the end stations will
communicate with each other as well as with users. The bottom four
layers define how data is transmitted end to end.

\protect\hyperlink{c01.xhtmlux5cux23figure01-7}{Figure 1.7} shows the
three upper layers and their functions.

\begin{figure}
\centering
\includegraphics{images/c01f007.jpg}
\caption{{\protect\hyperlink{c01.xhtmlux5cux23figureanchor01-7}{\textbf{FIGURE
1.7}} The upper layers}}
\end{figure}

When looking at \protect\hyperlink{c01.xhtmlux5cux23figure01-6}{Figure
1.6}, understand that users interact with the computer at the
Application layer and also that the upper layers are responsible for
applications communicating between hosts. None of the upper layers knows
anything about networking or network addresses because that's the
responsibility of the four bottom layers.

In \protect\hyperlink{c01.xhtmlux5cux23figure01-8}{Figure 1.8}, which
shows the four lower layers and their functions, you can see that it's
these four bottom layers that define how data is transferred through
physical media like wire, cable, fiber optics, switches, and routers.
These bottom layers also determine how to rebuild a data stream from a
transmitting host to a destination host's application.

\begin{figure}
\centering
\includegraphics{images/c01f008.jpg}
\caption{{\protect\hyperlink{c01.xhtmlux5cux23figureanchor01-8}{\textbf{FIGURE
1.8}} The lower layers}}
\end{figure}

\protect\hypertarget{c01.xhtmlux5cux23Page_16}{}{}The following network
devices operate at all seven layers of the OSI model:

\begin{enumerate}
\tightlist
\item
  \emph{Network management stations (NMSs)}
\item
  Web and application servers
\item
  Gateways (not default gateways)
\item
  Servers
\item
  Network hosts
\end{enumerate}

Basically, the ISO is pretty much the Emily Post of the network protocol
world. Just as Ms. Post wrote the book setting the standards---or
protocols---for human social interaction, the ISO developed the OSI
reference model as the precedent and guide for an open network protocol
set. Defining the etiquette of communication models, it remains the most
popular means of comparison for protocol suites today.

The OSI reference model has the following seven layers:

\begin{enumerate}
\tightlist
\item
  Application layer (layer 7)
\item
  Presentation layer (layer 6)
\item
  Session layer (layer 5)
\item
  Transport layer (layer 4)
\item
  Network layer (layer 3)
\item
  Data Link layer (layer 2)
\item
  Physical layer (layer 1)
\end{enumerate}

Some people like to use a mnemonic to remember the seven layers, such as
All People Seem To Need Data Processing.
\protect\hyperlink{c01.xhtmlux5cux23figure01-9}{Figure 1.9} shows a
summary of the functions defined at each layer of the OSI model.

\begin{figure}
\centering
\includegraphics{images/c01f009.jpg}
\caption{{\protect\hyperlink{c01.xhtmlux5cux23figureanchor01-9}{\textbf{FIGURE
1.9}} OSI layer functions}}
\end{figure}

I've separated the seven-layer model into three different functions: the
upper layers, the middle layers, and the bottom layers. The upper layers
communicate with the user interface
\protect\hypertarget{c01.xhtmlux5cux23Page_17}{}{}and application, the
middle layers do reliable communication and routing to a remote network,
and the bottom layers communicate to the local network.

With this in hand, you're now ready to explore each layer's function in
detail!

\subsubsection[The Application
Layer]{\texorpdfstring{\protect\hypertarget{c01.xhtmlux5cux23c01-sec-6}{}{}The
Application Layer}{The Application Layer}}

The \emph{Application layer} of the OSI model marks the spot where users
actually communicate to the computer and comes into play only when it's
clear that access to the network will be needed soon. Take the case of
Internet Explorer (IE). You could actually uninstall every trace of
networking components like TCP/IP, the NIC card, and so on and still use
IE to view a local HTML document. But things would get ugly if you tried
to do things like view a remote HTML document that must be retrieved
because IE and other browsers act on these types of requests by
attempting to access the Application layer. So basically, the
Application layer is working as the interface between the actual
application program and the next layer down by providing ways for the
application to send information down through the protocol stack. This
isn't actually part of the layered structure, because browsers don't
live in the Application layer, but they interface with it as well as the
relevant protocols when asked to access remote resources.

Identifying and confirming the communication partner's availability and
verifying the required resources to permit the specified type of
communication to take place also occurs at the Application layer. This
is important because, like the lion's share of browser functions,
computer applications sometimes need more than desktop resources. It's
more typical than you would think for the communicating components of
several network applications to come together to carry out a requested
function. Here are a few good examples of these kinds of events:

\begin{enumerate}
\tightlist
\item
  File transfers
\item
  Email
\item
  Enabling remote access
\item
  Network management activities
\item
  Client/server processes
\item
  Information location
\end{enumerate}

Many network applications provide services for communication over
enterprise networks, but for present and future internetworking, the
need is fast developing to reach beyond the limits of current physical
networking.

\begin{center}\rule{0.5\linewidth}{0.5pt}\end{center}

\includegraphics{images/note.png} The Application layer works as the
interface between actual application programs. This means end-user
programs like Microsoft Word don't reside at the Application layer, they
interface with the Application layer protocols. Later, in Chapter 3,
``Introduction to TCP/IP,'' I'll talk in detail about a few important
programs that actually reside at the Application layer, like Telnet,
FTP, and TFTP.

\begin{center}\rule{0.5\linewidth}{0.5pt}\end{center}

\subsubsection[The Presentation
Layer]{\texorpdfstring{\protect\hypertarget{c01.xhtmlux5cux23c01-sec-7}{}{}\protect\hypertarget{c01.xhtmlux5cux23Page_18}{}{}The
Presentation Layer}{The Presentation Layer}}

The \emph{Presentation layer} gets its name from its purpose: It
presents data to the Application layer and is responsible for data
translation and code formatting. Think of it as the OSI model's
translator, providing coding and conversion services. One very effective
way of ensuring a successful data transfer is to convert the data into a
standard format before transmission. Computers are configured to receive
this generically formatted data and then reformat it back into its
native state to read it. An example of this type of translation service
occurs when translating old Extended Binary Coded Decimal Interchange
Code (EBCDIC) data to ASCII, the American Standard Code for Information
Interchange (often pronounced ``askee''). So just remember that by
providing translation services, the Presentation layer ensures that data
transferred from the Application layer of one system can be read by the
Application layer of another one.

With this in mind, it follows that the OSI would include protocols that
define how standard data should be formatted, so key functions like data
compression, decompression, encryption, and decryption are also
associated with this layer. Some Presentation layer standards are
involved in multimedia operations as well.

\subsubsection[The Session
Layer]{\texorpdfstring{\protect\hypertarget{c01.xhtmlux5cux23c01-sec-8}{}{}The
Session Layer}{The Session Layer}}

The \emph{Session layer} is responsible for setting up, managing, and
dismantling sessions between Presentation layer entities and keeping
user data separate. Dialog control between devices also occurs at this
layer.

Communication between hosts' various applications at the Session layer,
as from a client to a server, is coordinated and organized via three
different modes: \emph{simplex}, \emph{half-duplex}, and
\emph{full-duplex}. Simplex is simple one-way communication, kind of
like saying something and not getting a reply. Half-duplex is actual
two-way communication, but it can take place in only one direction at a
time, preventing the interruption of the transmitting device. It's like
when pilots and ship captains communicate over their radios, or even a
walkie-talkie. But full-duplex is exactly like a real conversation where
devices can transmit and receive at the same time, much like two people
arguing or interrupting each other during a telephone conversation.

\subsubsection[The Transport
Layer]{\texorpdfstring{\protect\hypertarget{c01.xhtmlux5cux23c01-sec-9}{}{}The
Transport Layer}{The Transport Layer}}

The \emph{Transport layer} segments and reassembles data into a single
data stream. Services located at this layer take all the various data
received from upper-layer applications, then combine it into the same,
concise data stream. These protocols provide end-to-end data transport
services and can establish a logical connection between the sending host
and destination host on an internetwork.

A pair of well-known protocols called TCP and UDP are integral to this
layer, but no worries if you're not already familiar with them because
I'll bring you up to speed later, in Chapter 3. For now, understand that
although both work at the Transport layer, TCP is known as a reliable
service but UDP is not. This distinction gives application developers
\protect\hypertarget{c01.xhtmlux5cux23Page_19}{}{}more options because
they have a choice between the two protocols when they are designing
products for this layer.

The Transport layer is responsible for providing mechanisms for
multiplexing upper-layer applications, establishing sessions, and
tearing down virtual circuits. It can also hide the details of
network-dependent information from the higher layers as well as provide
transparent data transfer.

\begin{center}\rule{0.5\linewidth}{0.5pt}\end{center}

\includegraphics{images/note.png} The term \emph{reliable networking}
can be used at the Transport layer. Reliable networking requires that
acknowledgments, sequencing, and flow control will all be used.

\begin{center}\rule{0.5\linewidth}{0.5pt}\end{center}

The Transport layer can be either connectionless or connection-oriented,
but because Cisco really wants you to understand the connection-oriented
function of the Transport layer, I'm going to go into that in more
detail here.

\paragraph{Connection-Oriented Communication}

For reliable transport to occur, a device that wants to transmit must
first establish a connection-oriented communication session with a
remote device---its peer system---known as a \emph{call setup} or a
\emph{three-way handshake}. Once this process is complete, the data
transfer occurs, and when it's finished, a call termination takes place
to tear down the virtual circuit.

\protect\hyperlink{c01.xhtmlux5cux23figure01-10}{Figure 1.10} depicts a
typical reliable session taking place between sending and receiving
systems. In it, you can see that both hosts' application programs begin
by notifying their individual operating systems that a connection is
about to be initiated. The two operating systems communicate by sending
messages over the network confirming that the transfer is approved and
that both sides are ready for it to take place. After all of this
required synchronization takes place, a connection is fully established
and the data transfer begins. And by the way, it's really helpful to
understand that this virtual circuit setup is often referred to as
overhead!

\begin{figure}
\centering
\includegraphics{images/c01f010.jpg}
\caption{{\protect\hyperlink{c01.xhtmlux5cux23figureanchor01-10}{\textbf{FIGURE
1.10}} Establishing a connection-oriented session}}
\end{figure}

\protect\hypertarget{c01.xhtmlux5cux23Page_20}{}{}Okay, now while the
information is being transferred between hosts, the two machines
periodically check in with each other, communicating through their
protocol software to ensure that all is going well and that the data is
being received properly.

Here's a summary of the steps in the connection-oriented session---that
three-way handshake---pictured in
\protect\hyperlink{c01.xhtmlux5cux23figure01-9}{Figure 1.9}:

\begin{enumerate}
\tightlist
\item
  The first ``connection agreement'' segment is a request for
  \emph{synchronization (SYN)}.
\item
  The next segments \emph{acknowledge (ACK)} the request and establish
  connection parameters---the rules---between hosts. These segments
  request that the receiver's sequencing is synchronized here as well so
  that a bidirectional connection can be formed.
\item
  The final segment is also an acknowledgment, which notifies the
  destination host that the connection agreement has been accepted and
  that the actual connection has been established. Data transfer can now
  begin.
\end{enumerate}

Sounds pretty simple, but things don't always flow so smoothly.
Sometimes during a transfer, congestion can occur because a high-speed
computer is generating data traffic a lot faster than the network itself
can process it! And a whole bunch of computers simultaneously sending
datagrams through a single gateway or destination can also jam things up
pretty badly. In the latter case, a gateway or destination can become
congested even though no single source caused the problem. Either way,
the problem is basically akin to a freeway bottleneck---too much traffic
for too small a capacity. It's not usually one car that's the problem;
it's just that there are way too many cars on that freeway at once!

But what actually happens when a machine receives a flood of datagrams
too quickly for it to process? It stores them in a memory section called
a \emph{buffer}. Sounds great; it's just that this buffering action can
solve the problem only if the datagrams are part of a small burst. If
the datagram deluge continues, eventually exhausting the device's
memory, its flood capacity will be exceeded and it will dump any and all
additional datagrams it receives just like an inundated overflowing
bucket!

\paragraph{Flow Control}

Since floods and losing data can both be tragic, we have a fail-safe
solution in place known as \emph{flow control}. Its job is to ensure
data integrity at the Transport layer by allowing applications to
request reliable data transport between systems. Flow control prevents a
sending host on one side of the connection from overflowing the buffers
in the receiving host. Reliable data transport employs a
connection-oriented communications session between systems, and the
protocols involved ensure that the following will be achieved:

\begin{enumerate}
\tightlist
\item
  The segments delivered are acknowledged back to the sender upon their
  reception.
\item
  Any segments not acknowledged are retransmitted.
\item
  Segments are sequenced back into their proper order upon arrival at
  their destination.
\item
  A manageable data flow is maintained in order to avoid congestion,
  overloading, or worse, data loss.
\end{enumerate}

\protect\hypertarget{c01.xhtmlux5cux23Page_21}{}{}

\begin{center}\rule{0.5\linewidth}{0.5pt}\end{center}

\includegraphics{images/note.png} The purpose of flow control is to
provide a way for the receiving device to control the amount of data
sent by the sender.

\begin{center}\rule{0.5\linewidth}{0.5pt}\end{center}

Because of the transport function, network flood control systems really
work well. Instead of dumping and losing data, the Transport layer can
issue a ``not ready'' indicator to the sender, or potential source of
the flood. This mechanism works kind of like a stoplight, signaling the
sending device to stop transmitting segment traffic to its overwhelmed
peer. After the peer receiver processes the segments already in its
memory reservoir---its buffer---it sends out a ``ready'' transport
indicator. When the machine waiting to transmit the rest of its
datagrams receives this ``go'' indicator, it resumes its transmission.
The process is pictured in
\protect\hyperlink{c01.xhtmlux5cux23figure01-11}{Figure 1.11}.

\begin{figure}
\centering
\includegraphics{images/c01f011.jpg}
\caption{{\protect\hyperlink{c01.xhtmlux5cux23figureanchor01-11}{\textbf{FIGURE
1.11}} Transmitting segments with flow control}}
\end{figure}

In a reliable, connection-oriented data transfer, datagrams are
delivered to the receiving host hopefully in the same sequence they're
transmitted. A failure will occur if any data segments are lost,
duplicated, or damaged along the way---a problem solved by having the
receiving host acknowledge that it has received each and every data
segment.

A service is considered connection-oriented if it has the following
characteristics:

\begin{enumerate}
\tightlist
\item
  A virtual circuit, or ``three-way handshake,'' is set up.
\item
  It uses sequencing.
\item
  It uses acknowledgments.
\item
  It uses flow control.
\end{enumerate}

\begin{center}\rule{0.5\linewidth}{0.5pt}\end{center}

\includegraphics{images/note.png} The types of flow control are
buffering, windowing, and congestion avoidance.

\begin{center}\rule{0.5\linewidth}{0.5pt}\end{center}

\paragraph[Windowing]{\texorpdfstring{\protect\hypertarget{c01.xhtmlux5cux23Page_22}{}{}Windowing}{Windowing}}

Ideally, data throughput happens quickly and efficiently. And as you can
imagine, it would be painfully slow if the transmitting machine had to
actually wait for an acknowledgment after sending each and every
segment! The quantity of data segments, measured in bytes, that the
transmitting machine is allowed to send without receiving an
acknowledgment is called a \emph{window}.

\begin{center}\rule{0.5\linewidth}{0.5pt}\end{center}

\includegraphics{images/note.png} Windows are used to control the amount
of outstanding, unacknowledged data segments.

\begin{center}\rule{0.5\linewidth}{0.5pt}\end{center}

The size of the window controls how much information is transferred from
one end to the other before an acknowledgement is required. While some
protocols quantify information depending on the number of packets,
TCP/IP measures it by counting the number of bytes.

As you can see in
\protect\hyperlink{c01.xhtmlux5cux23figure01-12}{Figure 1.12}, there are
two window sizes---one set to 1 and one set to 3.

\begin{figure}
\centering
\includegraphics{images/c01f012.jpg}
\caption{{\protect\hyperlink{c01.xhtmlux5cux23figureanchor01-12}{\textbf{FIGURE
1.12}} Windowing}}
\end{figure}

If you've configured a window size of 1, the sending machine will wait
for an acknowledgment for each data segment it transmits before
transmitting another one but will allow three to be transmitted before
receiving an acknowledgement if the window size is set to 3.

In this simplified example, both the sending and receiving machines are
workstations. Remember that in reality, the transmission isn't based on
simple numbers but in the amount of bytes that can be sent!

\begin{center}\rule{0.5\linewidth}{0.5pt}\end{center}

\includegraphics{images/note.png} If a receiving host fails to receive
all the bytes that it should acknowledge, the host can improve the
communication session by decreasing the window size.

\begin{center}\rule{0.5\linewidth}{0.5pt}\end{center}

\paragraph[Acknowledgments]{\texorpdfstring{\protect\hypertarget{c01.xhtmlux5cux23Page_23}{}{}Acknowledgments}{Acknowledgments}}

Reliable data delivery ensures the integrity of a stream of data sent
from one machine to the other through a fully functional data link. It
guarantees that the data won't be duplicated or lost. This is achieved
through something called \emph{positive acknowledgment with
retransmission}---a technique that requires a receiving machine to
communicate with the transmitting source by sending an acknowledgment
message back to the sender when it receives data. The sender documents
each segment measured in bytes, then sends and waits for this
acknowledgment before sending the next segment. Also important is that
when it sends a segment, the transmitting machine starts a timer and
will retransmit if it expires before it gets an acknowledgment back from
the receiving end.
\protect\hyperlink{c01.xhtmlux5cux23figure01-13}{Figure 1.13} shows the
process I just described.

\begin{figure}
\centering
\includegraphics{images/c01f013.jpg}
\caption{{\protect\hyperlink{c01.xhtmlux5cux23figureanchor01-13}{\textbf{FIGURE
1.13}} Transport layer reliable delivery}}
\end{figure}

In the figure, the sending machine transmits segments 1, 2, and 3. The
receiving node acknowledges that it has received them by requesting
segment 4 (what it is expecting next). When it receives the
acknowledgment, the sender then transmits segments 4, 5, and 6. If
segment 5 doesn't make it to the destination, the receiving node
acknowledges that event with a request for the segment to be re-sent.
The sending machine will then resend the lost segment and wait for an
acknowledgment, which it must receive in order to move on to the
transmission of segment 7.

The Transport layer, working in tandem with the Session layer, also
separates the data from different applications, an activity known as
\emph{session multiplexing}, and it happens when a client connects to a
server with multiple browser sessions open. This is exactly what's
taking place when you go someplace online like Amazon and click multiple
links, opening them simultaneously to get information when comparison
shopping. The client data from each browser session must be separate
when the server application receives it, which is pretty slick
technologically speaking, and it's the Transport layer to the rescue for
that juggling act!

\subsubsection[The Network
Layer]{\texorpdfstring{\protect\hypertarget{c01.xhtmlux5cux23c01-sec-10}{}{}\protect\hypertarget{c01.xhtmlux5cux23Page_24}{}{}The
Network Layer}{The Network Layer}}

The \emph{Network layer}, or layer 3, manages device addressing, tracks
the location of devices on the network, and determines the best way to
move data. This means that it's up to the Network layer to transport
traffic between devices that aren't locally attached. Routers, which are
layer 3 devices, are specified at this layer and provide the routing
services within an internetwork.

Here's how that works: first, when a packet is received on a router
interface, the destination IP address is checked. If the packet isn't
destined for that particular router, it will look up the destination
network address in the routing table. Once the router chooses an exit
interface, the packet will be sent to that interface to be framed and
sent out on the local network. If the router can't find an entry for the
packet's destination network in the routing table, the router drops the
packet.

Data and route update packets are the two types of packets used at the
Network layer:

\textbf{Data Packets} These are used to transport user data through the
internetwork. Protocols used to support data traffic are called routed
protocols, and IP and IPv6 are key examples. I'll cover IP addressing in
Chapter 3, ``Introduction to TCP/IP,'' and Chapter 4, ``Easy
Subnetting,'' and I'll cover IPv6 in Chapter 14, ``Internet Protocol
Version 6 (IPv6).''

\textbf{Route Update Packets} These packets are used to update
neighboring routers about the networks connected to all routers within
the internetwork. Protocols that send route update packets are called
routing protocols; the most critical ones for CCNA are RIPv2, EIGRP, and
OSPF. Route update packets are used to help build and maintain routing
tables.

\protect\hyperlink{c01.xhtmlux5cux23figure01-14}{Figure 1.14} shows an
example of a routing table. The routing table each router keeps and
refers to includes the following information:

\begin{figure}
\centering
\includegraphics{images/c01f014.jpg}
\caption{{\protect\hyperlink{c01.xhtmlux5cux23figureanchor01-14}{\textbf{FIGURE
1.14}} Routing table used in a router}}
\end{figure}

\textbf{\protect\hypertarget{c01.xhtmlux5cux23Page_25}{}{}Network
Addresses} Protocol-specific network addresses. A router must maintain a
routing table for individual routing protocols because each routed
protocol keeps track of a network with a different addressing scheme.
For example, the routing tables for IP and IPv6 are completely
different, so the router keeps a table for each one. Think of it as a
street sign in each of the different languages spoken by the American,
Spanish, and French people living on a street; the street sign would
read Cat/Gato/Chat.

\textbf{Interface} The exit interface a packet will take when destined
for a specific network.

\textbf{Metric} The distance to the remote network. Different routing
protocols use different ways of computing this distance. I'm going to
cover routing protocols thoroughly in Chapter 9, ``IP Routing.'' For
now, know that some routing protocols like the Routing Information
Protocol, or RIP, use hop count, which refers to the number of routers a
packet passes through en route to a remote network. Others use
bandwidth, delay of the line, or even tick count (1/18 of a second) to
determine the best path for data to get to a given destination.

And as I mentioned earlier, routers break up broadcast domains, which
means that by default, broadcasts aren't forwarded through a router. Do
you remember why this is a good thing? Routers also break up collision
domains, but you can also do that using layer 2 (Data Link layer)
switches. Because each interface in a router represents a separate
network, it must be assigned unique network identification numbers, and
each host on the network connected to that router must use the same
network number. \protect\hyperlink{c01.xhtmlux5cux23figure01-15}{Figure
1.15} shows how a router works in an internetwork.

\begin{figure}
\centering
\includegraphics{images/c01f015.jpg}
\caption{{\protect\hyperlink{c01.xhtmlux5cux23figureanchor01-15}{\textbf{FIGURE
1.15}} A router in an internetwork. Each router LAN interface is a
broadcast domain. Routers break up broadcast domains by default and
provide WAN services.}}
\end{figure}

Here are some router characteristics that you should never forget:

\begin{enumerate}
\tightlist
\item
  Routers, by default, will not forward any broadcast or multicast
  packets.
\item
  Routers use the logical address in a Network layer header to determine
  the next-hop router to forward the packet to.
\item
  Routers can use access lists, created by an administrator, to control
  security based on the types of packets allowed to enter or exit an
  interface.
\item
  Routers can provide layer 2 bridging functions if needed and can
  simultaneously route through the same interface.
\item
  Layer 3 devices---in this case, routers---provide connections between
  \emph{virtual LANs (VLANs)}.
\item
  Routers can provide \emph{quality of service (QoS)} for specific types
  of network traffic.
\end{enumerate}

\subsubsection[The Data Link
Layer]{\texorpdfstring{\protect\hypertarget{c01.xhtmlux5cux23c01-sec-11}{}{}\protect\hypertarget{c01.xhtmlux5cux23Page_26}{}{}The
Data Link Layer}{The Data Link Layer}}

The \emph{Data Link layer} provides for the physical transmission of
data and handles error notification, network topology, and flow control.
This means that the Data Link layer will ensure that messages are
delivered to the proper device on a LAN using hardware addresses and
will translate messages from the Network layer into bits for the
Physical layer to transmit.

The Data Link layer formats the messages, each called a \emph{data
frame}, and adds a customized header containing the hardware destination
and source address. This added information forms a sort of capsule that
surrounds the original message in much the same way that engines,
navigational devices, and other tools were attached to the lunar modules
of the Apollo project. These various pieces of equipment were useful
only during certain stages of space flight and were stripped off the
module and discarded when their designated stage was completed. The
process of data traveling through networks is similar.

\protect\hyperlink{c01.xhtmlux5cux23figure01-16}{Figure 1.16} shows the
Data Link layer with the Ethernet and IEEE specifications. When you
check it out, notice that the IEEE 802.2 standard is used in conjunction
with and adds functionality to the other IEEE standards. (You'll read
more about the important IEEE 802 standards used with the Cisco
objectives in Chapter 2, ``Ethernet Networking and Data
Encapsulation.'')

\begin{figure}
\centering
\includegraphics{images/c01f016.jpg}
\caption{{\protect\hyperlink{c01.xhtmlux5cux23figureanchor01-16}{\textbf{FIGURE
1.16}} Data Link layer}}
\end{figure}

It's important for you to understand that routers, which work at the
Network layer, don't care at all about where a particular host is
located. They're only concerned about where networks are located and the
best way to reach them---including remote ones. Routers are totally
obsessive when it comes to networks, which in this case is a good thing!
It's the Data Link layer that's responsible for the actual unique
identification of each device that resides on a local network.

For a host to send packets to individual hosts on a local network as
well as transmit packets between routers, the Data Link layer uses
hardware addressing. Each time a packet is sent between routers, it's
framed with control information at the Data Link layer, but that
information is stripped off at the receiving router and only the
original packet is left completely intact. This framing of the packet
continues for each hop until the packet is finally delivered to the
correct receiving host. It's really important to understand that the
packet itself is never altered along the route; it's only encapsulated
with the type of control information required for it to be properly
passed on to the different media types.

\protect\hypertarget{c01.xhtmlux5cux23Page_27}{}{}The IEEE Ethernet Data
Link layer has two sublayers:

\textbf{Media Access Control (MAC)} Defines how packets are placed on
the media. Contention for media access is ``first come/first served''
access where everyone shares the same bandwidth---hence the name.
Physical addressing is defined here as well as logical topologies.
What's a logical topology? It's the signal path through a physical
topology. Line discipline, error notification (but not correction), the
ordered delivery of frames, and optional flow control can also be used
at this sublayer.

\textbf{Logical Link Control (LLC)} Responsible for identifying Network
layer protocols and then encapsulating them. An LLC header tells the
Data Link layer what to do with a packet once a frame is received. It
works like this: a host receives a frame and looks in the LLC header to
find out where the packet is destined---for instance, the IP protocol at
the Network layer. The LLC can also provide flow control and sequencing
of control bits.

The switches and bridges I talked about near the beginning of the
chapter both work at the Data Link layer and filter the network using
hardware (MAC) addresses. I'll talk about these next.

\begin{center}\rule{0.5\linewidth}{0.5pt}\end{center}

\includegraphics{images/note.png} As data is encoded with control
information at each layer of the OSI model, the data is named with
something called a protocol data unit (PDU). At the Transport layer, the
PDU is called a segment, at the Network layer it's a packet, at the Data
Link a frame, and at the Physical layer it's called bits. This method of
naming the data at each layer is covered thoroughly in Chapter 2.

\begin{center}\rule{0.5\linewidth}{0.5pt}\end{center}

\paragraph{Switches and Bridges at the Data Link Layer}

Layer 2 switching is considered hardware-based bridging because it uses
specialized hardware called an \emph{application-specific integrated
circuit (ASIC)}. ASICs can run up to high gigabit speeds with very low
latency rates.

\begin{center}\rule{0.5\linewidth}{0.5pt}\end{center}

\includegraphics{images/note.png} \emph{Latency} is the time measured
from when a frame enters a port to when it exits a port.

\begin{center}\rule{0.5\linewidth}{0.5pt}\end{center}

Bridges and switches read each frame as it passes through the network.
The layer 2 device then puts the source hardware address in a filter
table and keeps track of which port the frame was received on. This
information (logged in the bridge's or switch's filter table) is what
helps the machine determine the location of the specific sending device.
\protect\hyperlink{c01.xhtmlux5cux23figure01-17}{Figure 1.17} shows a
switch in an internetwork and how John is sending packets to the
Internet and Sally doesn't hear his frames because she is in a different
collision domain. The destination frame goes directly to the default
gateway router, and Sally doesn't see John's traffic, much to her
relief.

\protect\hypertarget{c01.xhtmlux5cux23Page_28}{}{}

\begin{figure}
\centering
\includegraphics{images/c01f017.jpg}
\caption{{\protect\hyperlink{c01.xhtmlux5cux23figureanchor01-17}{\textbf{FIGURE
1.17}} A switch in an internetwork}}
\end{figure}

The real estate business is all about location, location, location, and
it's the same way for both layer 2 and layer 3 devices. Though both need
to be able to negotiate the network, it's crucial to remember that
they're concerned with very different parts of it. Primarily, layer 3
machines (such as routers) need to locate specific networks, whereas
layer 2 machines (switches and bridges) need to eventually locate
specific devices. So, networks are to routers as individual devices are
to switches and bridges. And routing tables that ``map'' the
internetwork are for routers as filter tables that ``map'' individual
devices are for switches and bridges.

After a filter table is built on the layer 2 device, it will forward
frames only to the segment where the destination hardware address is
located. If the destination device is on the same segment as the frame,
the layer 2 device will block the frame from going to any other
segments. If the destination is on a different segment, the frame can be
transmitted only to that segment. This is called \emph{transparent
bridging}.

When a switch interface receives a frame with a destination hardware
address that isn't found in the device's filter table, it will forward
the frame to all connected segments. If the unknown device that was sent
the ``mystery frame'' replies to this forwarding action, the switch
updates its filter table regarding that device's location. But in the
event the destination address of the transmitting frame is a broadcast
address, the switch will forward all broadcasts to every connected
segment by default.

All devices that the broadcast is forwarded to are considered to be in
the same broadcast domain. This can be a problem because layer 2 devices
propagate layer 2 broadcast storms that can seriously choke performance,
and the only way to stop a broadcast storm from propagating through an
internetwork is with a layer 3 device---a router!

The biggest benefit of using switches instead of hubs in your
internetwork is that each switch port is actually its own collision
domain. Remember that a hub creates one large collision domain, which is
not a good thing! But even armed with a switch, you still don't get to
just break up broadcast domains by default because neither switches nor
bridges will do that. They'll simply forward all broadcasts instead.

Another benefit of LAN switching over hub-centered implementations is
that each device on every segment plugged into a switch can transmit
simultaneously. Well, at least they can as long as there's only one host
on each \protect\hypertarget{c01.xhtmlux5cux23Page_29}{}{}port and there
isn't a hub plugged into a switch port! As you might have guessed, this
is because hubs allow only one device per network segment to communicate
at a time.

\subsubsection[The Physical
Layer]{\texorpdfstring{\protect\hypertarget{c01.xhtmlux5cux23c01-sec-12}{}{}The
Physical Layer}{The Physical Layer}}

Finally arriving at the bottom, we find that the \emph{Physical layer}
does two things: it sends bits and receives bits. Bits come only in
values of 1 or 0---a Morse code with numerical values. The Physical
layer communicates directly with the various types of actual
communication media. Different kinds of media represent these bit values
in different ways. Some use audio tones, while others employ \emph{state
transitions}---changes in voltage from high to low and low to high.
Specific protocols are needed for each type of media to describe the
proper bit patterns to be used, how data is encoded into media signals,
and the various qualities of the physical media's attachment interface.

The Physical layer specifies the electrical, mechanical, procedural, and
functional requirements for activating, maintaining, and deactivating a
physical link between end systems. This layer is also where you identify
the interface between the \emph{data terminal equipment (DTE)} and the
\emph{data communication equipment (DCE)}. (Some old phone-company
employees still call DCE ``data circuit-terminating equipment.'') The
DCE is usually located at the service provider, while the DTE is the
attached device. The services available to the DTE are most often
accessed via a modem or \emph{channel service unit/data service unit
(CSU/DSU)}.

The Physical layer's connectors and different physical topologies are
defined by the OSI as standards, allowing disparate systems to
communicate. The Cisco exam objectives are interested only in the IEEE
Ethernet standards.

\paragraph{Hubs at the Physical Layer}

A hub is really a multiple-port repeater. A repeater receives a digital
signal, reamplifies or regenerates that signal, then forwards the signal
out the other port without looking at any data. A hub does the same
thing across all active ports: any digital signal received from a
segment on a hub port is regenerated or reamplified and transmitted out
all other ports on the hub. This means all devices plugged into a hub
are in the same collision domain as well as in the same broadcast
domain. \protect\hyperlink{c01.xhtmlux5cux23figure01-18}{Figure 1.18}
shows a hub in a network and how when one host transmits, all other
hosts must stop and listen.

\begin{figure}
\centering
\includegraphics{images/c01f018.jpg}
\caption{{\protect\hyperlink{c01.xhtmlux5cux23figureanchor01-18}{\textbf{FIGURE
1.18}} A hub in a network}}
\end{figure}

\protect\hypertarget{c01.xhtmlux5cux23Page_30}{}{}Hubs, like repeaters,
don't examine any of the traffic as it enters or before it's transmitted
out to the other parts of the physical media. And every device connected
to the hub, or hubs, must listen if a device transmits. A physical star
network, where the hub is a central device and cables extend in all
directions out from it, is the type of topology a hub creates. Visually,
the design really does resemble a star, whereas Ethernet networks run a
logical bus topology, meaning that the signal has to run through the
network from end to end.

\begin{center}\rule{0.5\linewidth}{0.5pt}\end{center}

\includegraphics{images/note.png} Hubs and repeaters can be used to
enlarge the area covered by a single LAN segment, but I really do not
recommend going with this configuration! LAN switches are affordable for
almost every situation and will make you much happier.

\begin{center}\rule{0.5\linewidth}{0.5pt}\end{center}

\paragraph{Topologies at the Physical layer}

One last thing I want to discuss at the Physical layer is topologies,
both physical and logical. Understand that every type of network has
both a physical and a logical topology.

\begin{enumerate}
\tightlist
\item
  The physical topology of a network refers to the physical layout of
  the devices, but mostly the cabling and cabling layout.
\item
  The logical topology defines the logical path on which the signal will
  travel on the physical topology.
\end{enumerate}

\protect\hyperlink{c01.xhtmlux5cux23figure01-19}{Figure 1.19} shows the
four types of topologies.

\begin{figure}
\centering
\includegraphics{images/c01f019.jpg}
\caption{{\protect\hyperlink{c01.xhtmlux5cux23figureanchor01-19}{\textbf{FIGURE
1.19}} Physical vs. Logical Topolgies}}
\end{figure}

Here are the topology types, although the most common, and pretty much
the only network we use today is a physical star, logical bus
technology, which is considered a hybrid topology (think Ethernet):

\begin{enumerate}
\tightlist
\item
  Bus: In a bus topology, every workstation is connected to a single
  cable, meaning every host is directly connected to every other
  workstation in the network.
\item
  Ring: In a ring topology, computers and other network devices are
  cabled together in a way that the last device is connected to the
  first to form a circle or ring.
\item
  Star: The most common physical topology is a star topology, which is
  your Ethernet switching physical layout. A central cabling device
  (switch) connects the computers and other network devices together.
  This category includes star and extended star topologies. Physical
  connection is commonly made using twisted-pair wiring.
\item
  \protect\hypertarget{c01.xhtmlux5cux23Page_31}{}{} Mesh: In a mesh
  topology, every network device is cabled together with connection to
  each other. Redundant links increase reliability and self-healing. The
  physical connection is commonly made using fiber or twisted-pair
  wiring.
\item
  Hybrid: Ethernet uses a physical star layout (cables come from all
  directions), and the signal travels end-to-end, like a bus route.
\end{enumerate}

\subsection[Summary]{\texorpdfstring{\protect\hypertarget{c01.xhtmlux5cux23c01-sec-13}{}{}Summary}{Summary}}

Whew! I know this seemed like the chapter that wouldn't end, but it
did---and you made it through! You're now armed with a ton of
fundamental information; you're ready to build upon it and are well on
your way to certification.

I started by discussing simple, basic networking and the differences
between collision and broadcast domains.

I then discussed the OSI model---the seven-layer model used to help
application developers design applications that can run on any type of
system or network. Each layer has its special jobs and select
responsibilities within the model to ensure that solid, effective
communications do, in fact, occur. I provided you with complete details
of each layer and discussed how Cisco views the specifications of the
OSI model.

In addition, each layer in the OSI model specifies different types of
devices, and I described the different devices used at each layer.

Remember that hubs are Physical layer devices and repeat the digital
signal to all segments except the one from which it was received.
Switches segment the network using hardware addresses and break up
collision domains. Routers break up broadcast domains as well as
collision domains and use logical addressing to send packets through an
internetwork.

\subsection[Exam
Essentials]{\texorpdfstring{\protect\hypertarget{c01.xhtmlux5cux23c01-sec-14}{}{}Exam
Essentials}{Exam Essentials}}

\textbf{Identify the possible causes of LAN traffic congestion.} Too
many hosts in a broadcast domain, broadcast storms, multicasting, and
low bandwidth are all possible causes of LAN traffic congestion.

\textbf{Describe the difference between a collision domain and a
broadcast domain.} \emph{Collision domain} is an Ethernet term used to
describe a network collection of devices in which one particular device
sends a packet on a network segment, forcing every other device on that
same segment to pay attention to it. With a broadcast domain, a set of
all devices on a network hears all broadcasts sent on all segments.

\textbf{Differentiate a MAC address and an IP address and describe how
and when each address type is used in a network.} A MAC address is a
hexadecimal number identifying the physical connection of a host. MAC
addresses are said to operate on layer 2 of the OSI model. IP addresses,
which can be expressed in binary or decimal format, are logical
identifiers that are said to be on layer 3 of the OSI model. Hosts on
the same physical segment locate
one\protect\hypertarget{c01.xhtmlux5cux23Page_32}{}{} another with MAC
addresses, while IP addresses are used when they reside on different LAN
segments or subnets.

\textbf{Understand the difference between a hub, a bridge, a switch, and
a router.} A hub creates one collision domain and one broadcast domain.
A bridge breaks up collision domains but creates one large broadcast
domain. They use hardware addresses to filter the network. Switches are
really just multiple-port bridges with more intelligence; they break up
collision domains but create one large broadcast domain by default.
Bridges and switches use hardware addresses to filter the network.
Routers break up broadcast domains (and collision domains) and use
logical addressing to filter the network.

\textbf{Identify the functions and advantages of routers.} Routers
perform packet switching, filtering, and path selection, and they
facilitate internetwork communication. One advantage of routers is that
they reduce broadcast traffic.

\textbf{Differentiate connection-oriented and connectionless network
services and describe how each is handled during network
communications.} Connection-oriented services use acknowledgments and
flow control to create a reliable session. More overhead is used than in
a connectionless network service. Connectionless services are used to
send data with no acknowledgments or flow control. This is considered
unreliable.

\textbf{Define the OSI layers, understand the function of each, and
describe how devices and networking protocols can be mapped to each
layer.} You must remember the seven layers of the OSI model and what
function each layer provides. The Application, Presentation, and Session
layers are upper layers and are responsible for communicating from a
user interface to an application. The Transport layer provides
segmentation, sequencing, and virtual circuits. The Network layer
provides logical network addressing and routing through an internetwork.
The Data Link layer provides framing and placing of data on the network
medium. The Physical layer is responsible for taking 1s and 0s and
encoding them into a digital signal for transmission on the network
segment.

\subsection[Written
Labs]{\texorpdfstring{\protect\hypertarget{c01.xhtmlux5cux23c01-sec-15}{}{}Written
Labs}{Written Labs}}

In this section, you'll complete the following labs to make sure you've
got the information and concepts contained within them fully dialed in:

\begin{enumerate}
\tightlist
\item
  Lab 1.1: OSI Questions
\item
  Lab 1.2: Defining the OSI Layers and Devices
\item
  Lab 1.3: Identifying Collision and Broadcast Domains
\end{enumerate}

You can find the answers to these labs in Appendix A, ``Answers to
Written Labs.''

\subsubsection[Written Lab 1.1: OSI
Questions]{\texorpdfstring{\protect\hypertarget{c01.xhtmlux5cux23c01-sec-16}{}{}Written
Lab 1.1: OSI Questions}{Written Lab 1.1: OSI Questions}}

Answer the following questions about the OSI model:

\begin{enumerate}
\tightlist
\item
  Which layer chooses and determines the availability of communicating
  partners along with the resources necessary to make the connection,
  coordinates partnering
  \protect\hypertarget{c01.xhtmlux5cux23Page_33}{}{}applications, and
  forms a consensus on procedures for controlling data integrity and
  error recovery?
\item
  Which layer is responsible for converting data packets from the Data
  Link layer into electrical signals?
\item
  At which layer is routing implemented, enabling connections and path
  selection between two end systems?
\item
  Which layer defines how data is formatted, presented, encoded, and
  converted for use on the network?
\item
  Which layer is responsible for creating, managing, and terminating
  sessions between applications?
\item
  Which layer ensures the trustworthy transmission of data across a
  physical link and is primarily concerned with physical addressing,
  line discipline, network topology, error notification, ordered
  delivery of frames, and flow control?
\item
  Which layer is used for reliable communication between end nodes over
  the network and provides mechanisms for establishing, maintaining, and
  terminating virtual circuits; transport-fault detection and recovery;
  and controlling the flow of information?
\item
  Which layer provides logical addressing that routers will use for path
  determination?
\item
  Which layer specifies voltage, wire speed, and cable pinouts and moves
  bits between devices?
\item
  Which layer combines bits into bytes and bytes into frames, uses MAC
  addressing, and provides error detection?
\item
  Which layer is responsible for keeping the data from different
  applications separate on the network?
\item
  Which layer is represented by frames?
\item
  Which layer is represented by segments?
\item
  Which layer is represented by packets?
\item
  Which layer is represented by bits?
\item
  Rearrange the following in order of encapsulation:

  \begin{enumerate}
  \tightlist
  \item
    Packets
  \item
    Frames
  \item
    Bits
  \item
    Segments
  \end{enumerate}
\item
  Which layer segments and reassembles data into a data stream?
\item
  Which layer provides the physical transmission of the data and handles
  error notification, network topology, and flow control?
\item
  Which layer manages logical device addressing, tracks the location of
  devices on the internetwork, and determines the best way to move data?
\item
  What is the bit length and expression form of a MAC address?
\end{enumerate}

\subsubsection[Written Lab 1.2: Defining the OSI Layers and
Devices]{\texorpdfstring{\protect\hypertarget{c01.xhtmlux5cux23c01-sec-17}{}{}\protect\hypertarget{c01.xhtmlux5cux23Page_34}{}{}Written
Lab 1.2: Defining the OSI Layers and
Devices}{Written Lab 1.2: Defining the OSI Layers and Devices}}

Fill in the blanks with the appropriate layer of the OSI or hub, switch,
or router device.

\begin{longtable}[]{@{}ll@{}}
\toprule
\textbf{Description} & \textbf{Device or OSI Layer}\tabularnewline
\midrule
\endhead
This device sends and receives information about the Network layer.
&\tabularnewline
This layer creates a virtual circuit before transmitting between two end
stations. &\tabularnewline
This device uses hardware addresses to filter a network.
&\tabularnewline
Ethernet is defined at these layers. &\tabularnewline
This layer supports flow control, sequencing, and acknowledgments.
&\tabularnewline
This device can measure the distance to a remote network.
&\tabularnewline
Logical addressing is used at this layer. &\tabularnewline
Hardware addresses are defined at this layer. &\tabularnewline
This device creates one collision domain and one broadcast domain.
&\tabularnewline
This device creates many smaller collision domains, but the network is
still one large broadcast domain. &\tabularnewline
This device can never run full-duplex. &\tabularnewline
This device breaks up collision domains and broadcast domains.
&\tabularnewline
\bottomrule
\end{longtable}

\subsubsection[Written Lab 1.3: Identifying Collision and Broadcast
Domains]{\texorpdfstring{\protect\hypertarget{c01.xhtmlux5cux23c01-sec-18}{}{}Written
Lab 1.3: Identifying Collision and Broadcast
Domains}{Written Lab 1.3: Identifying Collision and Broadcast Domains}}

\begin{enumerate}
\tightlist
\item
  In the following exhibit, identify the number of collision domains and
  broadcast domains in each specified device. Each device is represented
  by a letter:

  \begin{enumerate}
  \def\labelenumii{\Alph{enumii}.}
  \tightlist
  \item
    Hub
  \item
    Bridge
  \item
    Switch
  \item
    Router
  \end{enumerate}
\end{enumerate}

\protect\hypertarget{c01.xhtmlux5cux23Page_35}{}{}

\begin{figure}
\centering
\includegraphics{images/c01f020.jpg}
\caption{}
\end{figure}

\subsection[Review
Questions]{\texorpdfstring{\protect\hypertarget{c01.xhtmlux5cux23c01-sec-19}{}{}\protect\hypertarget{c01.xhtmlux5cux23Page_36}{}{}Review
Questions}{Review Questions}}

\begin{center}\rule{0.5\linewidth}{0.5pt}\end{center}

\includegraphics{images/note.png} The following questions are designed
to test your understanding of this chapter's material. For more
information on how to get additional questions, please see
\href{http://www.lammle.com/ccna}{www.lammle.com/ccna}.

\begin{center}\rule{0.5\linewidth}{0.5pt}\end{center}

You can find the answers to these questions in Appendix B, ``Answers to
Review Questions.''

\begin{enumerate}
\item
  Which of the following statements is/are true with regard to the
  device shown here? (Choose all that apply.)

  \begin{figure}
  \centering
  \includegraphics{images/c01f021.jpg}
  \caption{}
  \end{figure}

  \begin{enumerate}
  \def\labelenumii{\Alph{enumii}.}
  \tightlist
  \item
    It includes one collision domain and one broadcast domain.
  \item
    It includes 10 collision domains and 10 broadcast domains.
  \item
    It includes 10 collision domains and one broadcast domain.
  \item
    It includes one collision domain and 10 broadcast domains.
  \end{enumerate}
\item
  With respect to the OSI model, which one of the following is the
  correct statement about PDUs?

  \begin{enumerate}
  \def\labelenumii{\Alph{enumii}.}
  \tightlist
  \item
    A segment contains IP addresses.
  \item
    A packet contains IP addresses.
  \item
    A segment contains MAC addresses.
  \item
    A packet contains MAC addresses.
  \end{enumerate}
\item
  You are the Cisco administrator for your company. A new branch office
  is opening and you are selecting the necessary hardware to support the
  network. There will be two groups of computers, each organized by
  department. The Sales group computers will be assigned IP addresses
  ranging from 192.168.1.2 to 192.168.1.50. The Accounting group will be
  assigned IP addresses ranging from 10.0.0.2 to 10.0.0.50. What type of
  device should you select to connect the two groups of computers so
  that data communication can occur?

  \begin{enumerate}
  \def\labelenumii{\Alph{enumii}.}
  \tightlist
  \item
    Hub
  \item
    Switch
  \item
    Router
  \item
    Bridge
  \end{enumerate}
\item
  The most effective way to mitigate congestion on a LAN would be to
  \_\_\_\_\_\_\_\_.

  \begin{enumerate}
  \def\labelenumii{\Alph{enumii}.}
  \tightlist
  \item
    Upgrade the network cards
  \item
    Change the cabling to CAT 6
  \item
    \protect\hypertarget{c01.xhtmlux5cux23Page_37}{}{}Replace the hubs
    with switches
  \item
    Upgrade the CPUs in the routers
  \end{enumerate}
\item
  In the following work area, draw a line from the OSI model layer to
  its PDU.

  \begin{figure}
  \centering
  \includegraphics{images/c01f022.jpg}
  \caption{}
  \end{figure}
\item
  What is a function of the WLAN Controller?

  \begin{enumerate}
  \def\labelenumii{\Alph{enumii}.}
  \tightlist
  \item
    To monitor and control the incoming and outgoing network traffic
  \item
    To automatically handle the configuration of wireless access points
  \item
    To allow wireless devices to connect to a wired network
  \item
    To connect networks and intelligently choose the best paths between
    networks
  \end{enumerate}
\item
  You need to provide network connectivity to 150 client computers that
  will reside in the same subnetwork, and each client computer must be
  allocated dedicated bandwidth. Which device should you use to
  accomplish the task?

  \begin{enumerate}
  \def\labelenumii{\Alph{enumii}.}
  \tightlist
  \item
    Hub
  \item
    Switch
  \item
    Router
  \item
    Bridge
  \end{enumerate}
\item
  In the following work area, draw a line from the OSI model layer
  definition on the left to its description on the right.

  \begin{longtable}[]{@{}ll@{}}
  \toprule
  \endhead
  Layer & Description\tabularnewline
  Transport & Framing\tabularnewline
  Physical & End-to-end connection\tabularnewline
  Data Link & Routing\tabularnewline
  Network & Conversion to bits\tabularnewline
  \bottomrule
  \end{longtable}
\item
  What is the function of a firewall?

  \begin{enumerate}
  \def\labelenumii{\Alph{enumii}.}
  \tightlist
  \item
    To automatically handle the configuration of wireless access points
  \item
    To allow wireless devices to connect to a wired network
  \item
    To monitor and control the incoming and outgoing network traffic
  \item
    To connect networks and intelligently choose the best paths between
    networks
  \end{enumerate}
\item
  \protect\hypertarget{c01.xhtmlux5cux23Page_38}{}{}Which layer in the
  OSI reference model is responsible for determining the availability of
  the receiving program and checking to see whether enough resources
  exist for that communication?

  \begin{enumerate}
  \def\labelenumii{\Alph{enumii}.}
  \tightlist
  \item
    Transport
  \item
    Network
  \item
    Presentation
  \item
    Application
  \end{enumerate}
\item
  \protect\hypertarget{c01.xhtmlux5cux23Page_39}{}{}Which of the
  following correctly describe steps in the OSI data encapsulation
  process? (Choose two.)

  \begin{enumerate}
  \def\labelenumii{\Alph{enumii}.}
  \tightlist
  \item
    The Transport layer divides a data stream into segments and may add
    reliability and flow control information.
  \item
    The Data Link layer adds physical source and destination addresses
    and an FCS to the segment.
  \item
    Packets are created when the Network layer encapsulates a frame with
    source and destination host addresses and protocol-related control
    information.
  \item
    Packets are created when the Network layer adds layer 3 addresses
    and control information to a segment.
  \item
    The Presentation layer translates bits into voltages for
    transmission across the physical link.
  \end{enumerate}
\item
  Which of the following layers of the OSI model was later subdivided
  into two layers?

  \begin{enumerate}
  \def\labelenumii{\Alph{enumii}.}
  \tightlist
  \item
    Presentation
  \item
    Transport
  \item
    Data Link
  \item
    Physical
  \end{enumerate}
\item
  What is a function of an access point (AP)?

  \begin{enumerate}
  \def\labelenumii{\Alph{enumii}.}
  \tightlist
  \item
    To monitor and control the incoming and outgoing network traffic
  \item
    To automatically handle the configuration of wireless access point
  \item
    To allow wireless devices to connect to a wired network
  \item
    To connect networks and intelligently choose the best paths between
    networks
  \end{enumerate}
\item
  A\_\_\_\_\_\_\_\_\_is an example of a device that operates only at the
  physical layer.

  \begin{enumerate}
  \def\labelenumii{\Alph{enumii}.}
  \tightlist
  \item
    Hub
  \item
    Switch
  \item
    Router
  \item
    Bridge
  \end{enumerate}
\item
  Which of the following is \emph{not} a benefit of using a reference
  model?

  \begin{enumerate}
  \def\labelenumii{\Alph{enumii}.}
  \tightlist
  \item
    It divides the network communication process into smaller and
    simpler components.
  \item
    It encourages industry standardization.
  \item
    \protect\hypertarget{c01.xhtmlux5cux23Page_40}{}{}It enforces
    consistency across vendors.
  \item
    It allows various types of network hardware and software to
    communicate.
  \end{enumerate}
\item
  Which of the following statements is not true with regard to routers?

  \begin{enumerate}
  \def\labelenumii{\Alph{enumii}.}
  \tightlist
  \item
    They forward broadcasts by default.
  \item
    They can filter the network based on Network layer information.
  \item
    They perform path selection.
  \item
    They perform packet switching.
  \end{enumerate}
\item
  Switches break up\_\_\_\_\_\_\_domains, and routers break
  up\_\_\_\_\_\_\_domains.

  \begin{enumerate}
  \def\labelenumii{\Alph{enumii}.}
  \tightlist
  \item
    broadcast, broadcast
  \item
    collision, collision
  \item
    collision, broadcast
  \item
    broadcast, collision
  \end{enumerate}
\item
  How many collision domains are present in the following diagram?

  \begin{figure}
  \centering
  \includegraphics{images/c01f023.jpg}
  \caption{}
  \end{figure}

  \begin{enumerate}
  \def\labelenumii{\Alph{enumii}.}
  \tightlist
  \item
    8
  \item
    9
  \item
    10
  \item
    11
  \end{enumerate}
\item
  \protect\hypertarget{c01.xhtmlux5cux23Page_41}{}{}Which of the
  following layers of the OSI model is not involved in defining how the
  applications within the end stations will communicate with each other
  as well as with users?

  \begin{enumerate}
  \def\labelenumii{\Alph{enumii}.}
  \tightlist
  \item
    Transport
  \item
    Application
  \item
    Presentation
  \item
    Session
  \end{enumerate}
\item
  Which of the following is the \emph{only} device that operates at all
  layers of the OSI model?

  \begin{enumerate}
  \def\labelenumii{\Alph{enumii}.}
  \tightlist
  \item
    Network host
  \item
    Switch
  \item
    Router
  \item
    Bridge
  \end{enumerate}
\end{enumerate}

\protect\hypertarget{c02.xhtml}{}{}

\section[{Chapter 2}\\
{Ethernet Networking and Data
Encapsulation~}]{\texorpdfstring{\protect\hypertarget{c02.xhtmlux5cux23c02}{}{}\protect\hypertarget{c02.xhtmlux5cux23Page_41}{}{}{Chapter
2}\\
{Ethernet Networking and Data
Encapsulation~}}{Chapter 2 Ethernet Networking and Data Encapsulation~}}

\begin{center}\rule{0.5\linewidth}{0.5pt}\end{center}

\subsection{THE FOLLOWING ICND1 EXAM TOPICS ARE COVERED IN THIS
CHAPTER:}

\begin{enumerate}
\tightlist
\item
  \includegraphics{images/right.png} \textbf{Network Fundamentals}

  \begin{enumerate}
  \tightlist
  \item
    \includegraphics{images/squ.png} 1.6 Select the appropriate cabling
    type based on implementation requirements
  \item
    \includegraphics{images/squ.png} 1.4 Compare and contrast collapsed
    core and three-tier architectures
  \end{enumerate}
\item
  \includegraphics{images/right.png} \textbf{LAN Switching Technologies}

  \begin{enumerate}
  \tightlist
  \item
    \includegraphics{images/squ.png} 2.2 Interpret Ethernet frame format
  \end{enumerate}
\end{enumerate}

\protect\hypertarget{c02.xhtmlux5cux23Page_42}{}{}\includegraphics{images/intro.png}
Before we begin exploring a set of key foundational topics like the
TCP/IP DoD model, IP addressing, subnetting, and routing in the upcoming
chapters, I really want you to grasp the big picture of LANs
conceptually. The role Ethernet plays in today's networks as well as
what Media Access Control (MAC) addresses are and how they are used are
two more critical networking basics you'll want a solid understanding of
as well.

We'll cover these important subjects and more in this chapter, beginning
with Ethernet basics and the way MAC addresses are used on an Ethernet
LAN, and then we'll focus in on the actual protocols used with Ethernet
at the Data Link layer. To round out this discussion, you'll also learn
about some very important Ethernet specifications.

You know by now that there are a whole bunch of different devices
specified at the various layers of the OSI model and that it's essential
to be really familiar with the many types of cables and connectors
employed to hook them up to the network correctly. I'll review the types
of cabling used with Cisco devices in this chapter, demonstrate how to
connect to a router or switch, plus show you how to connect a router or
switch via a console connection.

I'll also introduce you to a vital process of encoding data as it makes
its way down the OSI stack, known as encapsulation.

I'm not nagging at all here---okay, maybe just a little, but promise
that you'll actually work through the four written labs and 20 review
questions I added to the end of this chapter just for you. You'll be so
happy you did because they're written strategically to make sure all the
important material covered in this chapter gets locked in, vault-tight
into your memory. So don't skip them!

\begin{center}\rule{0.5\linewidth}{0.5pt}\end{center}

\includegraphics{images/note.png} To find up-to-the-minute updates for
this chapter, please see \texttt{www.lammle.com/ccna} or the book's web
page via \texttt{www.sybex.com/go/ccna}.

\begin{center}\rule{0.5\linewidth}{0.5pt}\end{center}

\subsection[Ethernet Networks in
Review]{\texorpdfstring{\protect\hypertarget{c02.xhtmlux5cux23c02-sec-1}{}{}Ethernet
Networks in Review}{Ethernet Networks in Review}}

\emph{Ethernet} is a contention-based media access method that allows
all hosts on a network to share the same link's bandwidth. Some reasons
it's so popular are that Ethernet is really pretty simple to implement
and it makes troubleshooting fairly straightforward as well. Ethernet is
also readily scalable, meaning that it eases the process of integrating
new \protect\hypertarget{c02.xhtmlux5cux23Page_43}{}{}technologies into
an existing network infrastructure, like upgrading from Fast Ethernet to
Gigabit Ethernet.

Ethernet uses both Data Link and Physical layer specifications, so
you'll be presented with information relative to both layers, which
you'll need to effectively implement, troubleshoot, and maintain an
Ethernet network.

\subsubsection[Collision
Domain]{\texorpdfstring{\protect\hypertarget{c02.xhtmlux5cux23c02-sec-2}{}{}Collision
Domain}{Collision Domain}}

In Chapter 1, ``Internetworking,'' you learned that the Ethernet term
\emph{collision domain} refers to a network scenario wherein one device
sends a frame out on a physical network segment forcing every other
device on the same segment to pay attention to it. This is bad because
if two devices on a single physical segment just happen to transmit
simultaneously, it will cause a collision and require these devices to
retransmit. Think of a collision event as a situation where each
device's digital signals totally interfere with one another on the wire.
\protect\hyperlink{c02.xhtmlux5cux23figure02-1}{Figure 2.1} shows an
old, legacy network that's a single collision domain where only one host
can transmit at a time.

\begin{figure}
\centering
\includegraphics{images/c02f001.jpg}
\caption{{\protect\hyperlink{c02.xhtmlux5cux23figureanchor02-1}{\textbf{FIGURE
2.1}} Legacy collision domain design}}
\end{figure}

The hosts connected to each hub are in the same collision domain, so if
one of them transmits, all the others must take the time to listen for
and read the digital signal. It is easy to see how collisions can be a
serious drag on network performance, so I'll show you how to
strategically avoid them soon!

Okay---take another look at the network pictured in
\protect\hyperlink{c02.xhtmlux5cux23figure02-1}{Figure 2.1}. True, it
has only one collision domain, but worse, it's also a single broadcast
domain---what a mess! Let's check out an example, in
\protect\hyperlink{c02.xhtmlux5cux23figure02-2}{Figure 2.2}, of a
typical network design still used today and see if it's any better.

\protect\hypertarget{c02.xhtmlux5cux23Page_44}{}{}

\begin{figure}
\centering
\includegraphics{images/c02f002.jpg}
\caption{{\protect\hyperlink{c02.xhtmlux5cux23figureanchor02-2}{\textbf{FIGURE
2.2}} A typical network you'd see today}}
\end{figure}

Because each port off a switch is a single collision domain, we gain
more bandwidth for users, which is a great start. But switches don't
break up broadcast domains by default, so this is still only one
broadcast domain, which is not so good. This can work in a really small
network, but to expand it at all, we would need to break up the network
into smaller broadcast domains or our users won't get enough bandwidth!
And you're probably wondering about that device in the lower-right
corner, right? Well, that's a \emph{wireless access point}, which is
sometimes referred as an AP (which stands for access point). It's a
wireless device that allows hosts to connect wirelessly using the IEEE
802.11 specification and I added it to the figure to demonstrate how
these devices can be used to extend a collision domain. But still,
understand that APs don't actually segment the network, they only extend
them, meaning our LAN just got a lot bigger, with an unknown amount of
hosts that are all still part of one measly broadcast domain! This
clearly demonstrates why it's so important to understand exactly what a
broadcast domain is, and now is a great time to talk about them in
detail.

\subsubsection[Broadcast
Domain]{\texorpdfstring{\protect\hypertarget{c02.xhtmlux5cux23c02-sec-3}{}{}Broadcast
Domain}{Broadcast Domain}}

Let me start by giving you the formal definition: \emph{broadcast
domain} refers to a group of devices on a specific network segment that
hear all the broadcasts sent out on that specific network segment.

But even though a broadcast domain is usually a boundary delimited by
physical media like switches and routers, the term can also refer to a
logical division of a network segment, where all hosts can communicate
via a Data Link layer, hardware address broadcast.

\protect\hyperlink{c02.xhtmlux5cux23figure02-3}{Figure 2.3} shows how a
router would create a broadcast domain boundary.

Here you can see there are two router interfaces giving us two broadcast
domains, and I count 10 switch segments, meaning we've got 10 collision
domains.

The design depicted in
\protect\hyperlink{c02.xhtmlux5cux23figure02-3}{Figure 2.3} is still in
use today, and routers will be around for a long time, but in the
latest, modern switched networks, it's important to create small
broadcast domains. We achieve this by building virtual LANs (VLANs)
within \protect\hypertarget{c02.xhtmlux5cux23Page_45}{}{}our switched
networks, which I'll demonstrate shortly. Without employing VLANs in
today's switched environments, there wouldn't be much bandwidth
available to individual users. Switches break up collision domains with
each port, which is awesome, but they're still only one broadcast domain
by default! It's also one more reason why it's extremely important to
design our networks very carefully.

\begin{figure}
\centering
\includegraphics{images/c02f003.jpg}
\caption{{\protect\hyperlink{c02.xhtmlux5cux23figureanchor02-3}{\textbf{FIGURE
2.3}} A router creates broadcast domain boundaries.}}
\end{figure}

And key to carefully planning your network design is never to allow
broadcast domains to grow too large and get out of control. Both
collision and broadcast domains can easily be controlled with routers
and VLANs, so there's just no excuse to allow user bandwidth to slow to
a painful crawl when there are plenty of tools in your arsenal to
prevent the suffering!

An important reason for this book's existence is to ensure that you
really get the foundational basics of Cisco networks nailed down so you
can effectively design, implement, configure, troubleshoot, and even
dazzle colleagues and superiors with elegant designs that lavish your
users with all the bandwidth their hearts could possibly desire.

To make it to the top of that mountain, you need more than just the
basic story, so let's move on to explore the collision detection
mechanism used in half-duplex Ethernet.

\subsubsection[CSMA/CD]{\texorpdfstring{\protect\hypertarget{c02.xhtmlux5cux23c02-sec-4}{}{}CSMA/CD}{CSMA/CD}}

Ethernet networking uses a protocol called \emph{Carrier Sense Multiple
Access with Collision Detection (CSMA/CD)}, which helps devices share
the bandwidth evenly while preventing two devices from transmitting
simultaneously on the same network medium. CSMA/CD was actually created
to overcome the problem of the collisions that occur when packets are
transmitted from different nodes at the same time. And trust me---good
collision management is crucial, because when a node transmits in a
CSMA/CD network, all the other nodes on the network receive and examine
that transmission. Only switches and routers can effectively prevent a
transmission from propagating throughout the entire network!

So, how does the CSMA/CD protocol work? Let's start by taking a look at
\protect\hyperlink{c02.xhtmlux5cux23figure02-4}{Figure 2.4}.

\protect\hypertarget{c02.xhtmlux5cux23Page_46}{}{}

\begin{figure}
\centering
\includegraphics{images/c02f004.jpg}
\caption{{\protect\hyperlink{c02.xhtmlux5cux23figureanchor02-4}{\textbf{FIGURE
2.4}} CSMA/CD}}
\end{figure}

When a host wants to transmit over the network, it first checks for the
presence of a digital signal on the wire. If all is clear and no other
host is transmitting, the host will then proceed with its transmission.

But it doesn't stop there. The transmitting host constantly monitors the
wire to make sure no other hosts begin transmitting. If the host detects
another signal on the wire, it sends out an extended jam signal that
causes all nodes on the segment to stop sending data---think busy
signal.

The nodes respond to that jam signal by waiting a bit before attempting
to transmit again. Backoff algorithms determine when the colliding
stations can retransmit. If collisions keep occurring after 15 tries,
the nodes attempting to transmit will then time out. Half-duplex can be
pretty messy!

When a collision occurs on an Ethernet LAN, the following happens:

\begin{enumerate}
\tightlist
\item
  A jam signal informs all devices that a collision occurred.
\item
  The collision invokes a random backoff algorithm.
\item
  Each device on the Ethernet segment stops transmitting for a short
  time until its backoff timer expires.
\item
  All hosts have equal priority to transmit after the timers have
  expired.
\end{enumerate}

\protect\hypertarget{c02.xhtmlux5cux23Page_47}{}{}The ugly effects of
having a CSMA/CD network sustain heavy collisions are delay, low
throughput, and congestion.

\begin{center}\rule{0.5\linewidth}{0.5pt}\end{center}

\includegraphics{images/note.png} Backoff on an Ethernet network is the
retransmission delay that's enforced when a collision occurs. When that
happens, a host will resume transmission only after the forced time
delay has expired. Keep in mind that after the backoff has elapsed, all
stations have equal priority to transmit data.

\begin{center}\rule{0.5\linewidth}{0.5pt}\end{center}

At this point, let's take a minute to talk about Ethernet in detail at
both the Data Link layer (layer 2) and the Physical layer (layer 1).

\subsubsection[Half- and Full-Duplex
Ethernet]{\texorpdfstring{\protect\hypertarget{c02.xhtmlux5cux23c02-sec-5}{}{}Half-
and Full-Duplex Ethernet}{Half- and Full-Duplex Ethernet}}

Half-duplex Ethernet is defined in the original IEEE 802.3 Ethernet
specification, which differs a bit from how Cisco describes things.
Cisco says Ethernet uses only one wire pair with a digital signal
running in both directions on the wire. Even though the IEEE
specifications discuss the half-duplex process somewhat differently,
it's not actually a full-blown technical disagreement. Cisco is really
just talking about a general sense of what's happening with Ethernet.

Half-duplex also uses the CSMA/CD protocol I just discussed to help
prevent collisions and to permit retransmitting if one occurs. If a hub
is attached to a switch, it must operate in half-duplex mode because the
end stations must be able to detect collisions.
\protect\hyperlink{c02.xhtmlux5cux23figure02-5}{Figure 2.5} shows a
network with four hosts connected to a hub.

\begin{figure}
\centering
\includegraphics{images/c02f005.jpg}
\caption{{\protect\hyperlink{c02.xhtmlux5cux23figureanchor02-5}{\textbf{FIGURE
2.5}} Half-duplex example}}
\end{figure}

The problem here is that we can only run half-duplex, and if two hosts
communicate at the same time there will be a collision. Also,
half-duplex Ethernet is only about 30 to 40 percent efficient because a
large 100Base-T network will usually only give you 30 to 40 Mbps, at
most, due to overhead.

But full-duplex Ethernet uses two pairs of wires at the same time
instead of a single wire pair like half-duplex. And full-duplex uses a
point-to-point connection between the transmitter of the transmitting
device and the receiver of the receiving device. This means
\protect\hypertarget{c02.xhtmlux5cux23Page_48}{}{}that full-duplex data
transfers happen a lot faster when compared to half-duplex transfers.
Also, because the transmitted data is sent on a different set of wires
than the received data, collisions won't happen.
\protect\hyperlink{c02.xhtmlux5cux23figure02-6}{Figure 2.6} shows four
hosts connected to a switch, plus a hub. Definitely try not to use hubs
if you can help it!

\begin{figure}
\centering
\includegraphics{images/c02f006.jpg}
\caption{{\protect\hyperlink{c02.xhtmlux5cux23figureanchor02-6}{\textbf{FIGURE
2.6}} Full-duplex example}}
\end{figure}

Theoretically all hosts connected to the switch in
\protect\hyperlink{c02.xhtmlux5cux23figure02-6}{Figure 2.6} can
communicate at the same time because they can run full-duplex. Just keep
in mind that the switch port connecting to the hub as well as the hosts
connecting to that hub must run at half-duplex.

The reason you don't need to worry about collisions is because now it's
like a freeway with multiple lanes instead of the single-lane road
provided by half-duplex. Full-duplex Ethernet is supposed to offer 100
percent efficiency in both directions---for example, you can get 20 Mbps
with a 10 Mbps Ethernet running full-duplex, or 200 Mbps for Fast
Ethernet. But this rate is known as an aggregate rate, which translates
as ``you're supposed to get'' 100 percent efficiency. No guarantees, in
networking as in life!

You can use full-duplex Ethernet in at least the following six
situations:

\begin{enumerate}
\tightlist
\item
  With a connection from a switch to a host
\item
  With a connection from a switch to a switch
\item
  With a connection from a host to a host
\item
  With a connection from a switch to a router
\item
  With a connection from a router to a router
\item
  With a connection from a router to a host
\end{enumerate}

\begin{center}\rule{0.5\linewidth}{0.5pt}\end{center}

\includegraphics{images/note.png} Full-duplex Ethernet requires a
point-to-point connection when only two nodes are present. You can run
full-duplex with just about any device except a hub.

\begin{center}\rule{0.5\linewidth}{0.5pt}\end{center}

Now this may be a little confusing because this begs the question that
if it's capable of all that speed, why wouldn't it actually deliver?
Well, when a full-duplex Ethernet port is powered on, it first connects
to the remote end and then negotiates with the other end of the Fast
Ethernet link. This is called an \emph{auto-detect mechanism}. This
mechanism first
\protect\hypertarget{c02.xhtmlux5cux23Page_49}{}{}decides on the
exchange capability, which means it checks to see if it can run at 10,
100, or even 1000 Mbps. It then checks to see if it can run full-duplex,
and if it can't, it will run half-duplex.

\begin{center}\rule{0.5\linewidth}{0.5pt}\end{center}

\includegraphics{images/note.png} Remember that half-duplex Ethernet
shares a collision domain and provides a lower effective throughput than
full-duplex Ethernet, which typically has a private per-port collision
domain plus a higher effective throughput.

\begin{center}\rule{0.5\linewidth}{0.5pt}\end{center}

Last, remember these important points:

\begin{enumerate}
\tightlist
\item
  There are no collisions in full-duplex mode.
\item
  A dedicated switch port is required for each full-duplex node.
\item
  The host network card and the switch port must be capable of operating
  in full-duplex mode.
\item
  The default behavior of 10Base-T and 100Base-T hosts is 10 Mbps
  half-duplex if the autodetect mechanism fails, so it is always good
  practice to set the speed and duplex of each port on a switch if you
  can.
\end{enumerate}

Now let's take a look at how Ethernet works at the Data Link layer.

\subsubsection[Ethernet at the Data Link
Layer]{\texorpdfstring{\protect\hypertarget{c02.xhtmlux5cux23c02-sec-6}{}{}Ethernet
at the Data Link Layer}{Ethernet at the Data Link Layer}}

Ethernet at the Data Link layer is responsible for Ethernet addressing,
commonly referred to as MAC or hardware addressing. Ethernet is also
responsible for framing packets received from the Network layer and
preparing them for transmission on the local network through the
Ethernet contention-based media access method.

\paragraph{Ethernet Addressing}

Here's where we get into how Ethernet addressing works. It uses the
\emph{Media Access Control (MAC)} address burned into each and every
Ethernet network interface card (NIC). The MAC, or hardware, address is
a 48-bit (6-byte) address written in a hexadecimal format.

\protect\hyperlink{c02.xhtmlux5cux23figure02-7}{Figure 2.7} shows the
48-bit MAC addresses and how the bits are divided.

\begin{figure}
\centering
\includegraphics{images/c02f007.jpg}
\caption{{\protect\hyperlink{c02.xhtmlux5cux23figureanchor02-7}{\textbf{FIGURE
2.7}} Ethernet addressing using MAC addresses}}
\end{figure}

\protect\hypertarget{c02.xhtmlux5cux23Page_50}{}{}The
\emph{organizationally unique identifier (OUI)} is assigned by the IEEE
to an organization. It's composed of 24 bits, or 3 bytes, and it in turn
assigns a globally administered address also made up of 24 bits, or 3
bytes, that's supposedly unique to each and every adapter an
organization manufactures. Surprisingly, there's no guarantee when it
comes to that unique claim! Okay, now look closely at the figure. The
high-order bit is the Individual/Group (I/G) bit. When it has a value of
0, we can assume that the address is the MAC address of a device and
that it may well appear in the source portion of the MAC header. When
it's a 1, we can assume that the address represents either a broadcast
or multicast address in Ethernet.

The next bit is the Global/Local bit, sometimes called the G/L bit or
U/L bit, where \emph{U} means \emph{universal}. When set to 0, this bit
represents a globally administered address, as assigned by the IEEE, but
when it's a 1, it represents a locally governed and administered
address. The low-order 24 bits of an Ethernet address represent a
locally administered or manufacturer-assigned code. This portion
commonly starts with 24 0s for the first card made and continues in
order until there are 24 1s for the last (16,777,216th) card made.
You'll find that many manufacturers use these same six hex digits as the
last six characters of their serial number on the same card.

Let's stop for a minute and go over some addressing schemes important in
the Ethernet world.

\paragraph{Binary to Decimal and Hexadecimal Conversion}

Before we get into working with the TCP/IP protocol and IP addressing,
which we'll do in Chapter 3, ``Introduction to TCP/IP,'' it's really
important for you to truly grasp the differences between binary,
decimal, and hexadecimal numbers and how to convert one format into the
other.

We'll start with binary numbering, which is really pretty simple. The
digits used are limited to either a 1 or a 0, and each digit is called a
\emph{bit}, which is short for \emph{binary digit}. Typically, you group
either 4 or 8 bits together, with these being referred to as a nibble
and a byte, respectively.

The interesting thing about binary numbering is how the value is
represented in a decimal format---the typical decimal format being the
base-10 number scheme that we've all used since kindergarten. The binary
numbers are placed in a value spot, starting at the right and moving
left, with each spot having double the value of the previous spot.

\protect\hyperlink{c02.xhtmlux5cux23table02-1}{Table 2.1} shows the
decimal values of each bit location in a nibble and a byte. Remember, a
nibble is 4 bits and a byte is 8 bits.

{\protect\hyperlink{c02.xhtmlux5cux23tableanchor02-1}{\textbf{TABLE
2.1}} Binary values}

\begin{longtable}[]{@{}ll@{}}
\toprule
Nibble Values & Byte Values\tabularnewline
\midrule
\endhead
8 4 2 1 & 128 64 32 16 8 4 2 1\tabularnewline
\bottomrule
\end{longtable}

\protect\hypertarget{c02.xhtmlux5cux23Page_51}{}{}What all this means is
that if a one digit (1) is placed in a value spot, then the nibble or
byte takes on that decimal value and adds it to any other value spots
that have a 1. If a zero (0) is placed in a bit spot, you don't count
that value.

Let me clarify this a little. If we have a 1 placed in each spot of our
nibble, we would then add up 8 + 4 + 2 + 1 to give us a maximum value of
15. Another example for our nibble values would be 1001, meaning that
the 8 bit and the 1 bit are turned on, which equals a decimal value of
9. If we have a nibble binary value of 0110, then our decimal value
would be 6, because the 4 and 2 bits are turned on.

But the \emph{byte} decimal values can add up to a number that's
significantly higher than 15. This is how: If we counted every bit as a
one (1), then the byte binary value would look like the following
example because, remember, 8 bits equal a byte:

11111111

We would then count up every bit spot because each is turned on. It
would look like this, which demonstrates the maximum value of a byte:

128 + 64 + 32 + 16 + 8 + 4 + 2 + 1 = 255

There are plenty of other decimal values that a binary number can equal.
Let's work through a few examples:

10010110

Which bits are on? The 128, 16, 4, and 2 bits are on, so we'll just add
them up: 128 + 16 + 4 + 2 = 150.

01101100

Which bits are on? The 64, 32, 8, and 4 bits are on, so we just need to
add them up: 64 + 32 + 8 + 4 = 108.

11101000

Which bits are on? The 128, 64, 32, and 8 bits are on, so just add the
values up: 128 + 64 + 32 + 8 = 232.

I highly recommend that you memorize
\protect\hyperlink{c02.xhtmlux5cux23table02-2}{Table 2.2} before braving
the IP sections in Chapter 3, ``Introduction to TCP/IP,'' and Chapter 4,
``Easy Subnetting''!

{\protect\hyperlink{c02.xhtmlux5cux23tableanchor02-2}{\textbf{TABLE
2.2}} Binary to decimal memorization chart}

\begin{longtable}[]{@{}ll@{}}
\toprule
Binary Value & Decimal Value\tabularnewline
\midrule
\endhead
10000000 & 128\tabularnewline
11000000 & 192\tabularnewline
11100000 & 224\tabularnewline
11110000 & 240\tabularnewline
\protect\hypertarget{c02.xhtmlux5cux23Page_52}{}{}11111000 &
248\tabularnewline
11111100 & 252\tabularnewline
11111110 & 254\tabularnewline
11111111 & 255\tabularnewline
\bottomrule
\end{longtable}

Hexadecimal addressing is completely different than binary or
decimal---it's converted by reading nibbles, not bytes. By using a
nibble, we can convert these bits to hex pretty simply. First,
understand that the hexadecimal addressing scheme uses only the
characters 0 through 9. Because the numbers 10, 11, 12, and so on can't
be used (because they are two-digit numbers), the letters \emph{A},
\emph{B}, \emph{C}, \emph{D}, \emph{E}, and \emph{F} are used instead to
represent 10, 11, 12, 13, 14, and 15, respectively.

\begin{center}\rule{0.5\linewidth}{0.5pt}\end{center}

\includegraphics{images/note.png} \emph{Hex} is short for
\emph{hexadecimal}, which is a numbering system that uses the first six
letters of the alphabet, \emph{A} through \emph{F}, to extend beyond the
available 10 characters in the decimal system. These values are not case
sensitive.

\begin{center}\rule{0.5\linewidth}{0.5pt}\end{center}

\protect\hyperlink{c02.xhtmlux5cux23table02-3}{Table 2.3} shows both the
binary value and the decimal value for each hexadecimal digit.

{\protect\hyperlink{c02.xhtmlux5cux23tableanchor02-3}{\textbf{TABLE
2.3}} Hex to binary to decimal chart}

\begin{longtable}[]{@{}lll@{}}
\toprule
Hexadecimal Value & Binary Value & Decimal Value\tabularnewline
\midrule
\endhead
0 & 0000 & 0\tabularnewline
1 & 0001 & 1\tabularnewline
2 & 0010 & 2\tabularnewline
3 & 0011 & 3\tabularnewline
4 & 0100 & 4\tabularnewline
5 & 0101 & 5\tabularnewline
6 & 0110 & 6\tabularnewline
7 & 0111 & 7\tabularnewline
\protect\hypertarget{c02.xhtmlux5cux23Page_53}{}{}8 & 1000 &
8\tabularnewline
9 & 1001 & 9\tabularnewline
A & 1010 & 10\tabularnewline
B & 1011 & 11\tabularnewline
C & 1100 & 12\tabularnewline
D & 1101 & 13\tabularnewline
E & 1110 & 14\tabularnewline
F & 1111 & 15\tabularnewline
\bottomrule
\end{longtable}

Did you notice that the first 10 hexadecimal digits (0--9) are the same
value as the decimal values? If not, look again because this handy fact
makes those values super easy to convert!

Now suppose you have something like this: 0x6A. This is important
because sometimes Cisco likes to put \emph{0x} in front of characters so
you know that they are a hex value. It doesn't have any other special
meaning. So what are the binary and decimal values? All you have to
remember is that each hex character is one nibble and that two hex
characters joined together make a byte. To figure out the binary value,
put the hex characters into two nibbles and then join them together into
a byte. Six equals 0110, and A, which is 10 in hex, equals 1010, so the
complete byte would be 01101010.

To convert from binary to hex, just take the byte and break it into
nibbles. Let me clarify this.

Say you have the binary number 01010101. First, break it into
nibbles---0101 and 0101---with the value of each nibble being 5 since
the 1 and 4 bits are on. This makes the hex answer 0x55. And in decimal
format, the binary number is 01010101, which converts to 64 + 16 + 4 + 1
= 85.

Here's another binary number:

11001100

Your answer would be 1100 = 12 and 1100 = 12, so therefore, it's
converted to CC in hex. The decimal conversion answer would be 128 + 64
+ 8 + 4 = 204.

One more example, then we need to get working on the Physical layer.
Suppose you had the following binary number:

10110101

The hex answer would be 0xB5, since 1011 converts to B and 0101 converts
to 5 in hex value. The decimal equivalent is 128 + 32 + 16 + 4 + 1 =
181.

\protect\hypertarget{c02.xhtmlux5cux23Page_54}{}{}

\begin{center}\rule{0.5\linewidth}{0.5pt}\end{center}

\includegraphics{images/note.png} Make sure you check out Written Lab
2.1 for more practice with binary/decimal/hex conversion!

\begin{center}\rule{0.5\linewidth}{0.5pt}\end{center}

\paragraph{Ethernet Frames}

The Data Link layer is responsible for combining bits into bytes and
bytes into frames. Frames are used at the Data Link layer to encapsulate
packets handed down from the Network layer for transmission on a type of
media access.

The function of Ethernet stations is to pass data frames between each
other using a group of bits known as a MAC frame format. This provides
error detection from a \emph{cyclic redundancy check (CRC)}. But
remember---this is error detection, not error correction. An example of
a typical Ethernet frame used today is shown in
\protect\hyperlink{c02.xhtmlux5cux23figure02-8}{Figure 2.8}.

\begin{figure}
\centering
\includegraphics{images/c02f008.jpg}
\caption{{\protect\hyperlink{c02.xhtmlux5cux23figureanchor02-8}{\textbf{FIGURE
2.8}} Typical Ethernet frame format}}
\end{figure}

\begin{center}\rule{0.5\linewidth}{0.5pt}\end{center}

\includegraphics{images/note.png} Encapsulating a frame within a
different type of frame is called \emph{tunneling}.

\begin{center}\rule{0.5\linewidth}{0.5pt}\end{center}

Following are the details of the various fields in the typical Ethernet
frame type:

\textbf{Preamble} An alternating 1,0 pattern provides a 5 MHz clock at
the start of each packet, which allows the receiving devices to lock the
incoming bit stream.

\textbf{Start Frame Delimiter (SFD)/Synch} The preamble is seven octets
and the SFD is one octet (synch). The SFD is 10101011, where the last
pair of 1s allows the receiver to come into the alternating 1,0 pattern
somewhere in the middle and still sync up to detect the beginning of the
data.

\textbf{Destination Address (DA)} This transmits a 48-bit value using
the least significant bit (LSB) first. The DA is used by receiving
stations to determine whether an incoming packet is addressed to a
particular node. The destination address can be an individual address or
a broadcast or multicast MAC address. Remember that a broadcast is all
1s---all \emph{F}s in hex---and is sent to all devices. A multicast is
sent only to a similar subset of nodes on a network.

\textbf{Source Address (SA)} The SA is a 48-bit MAC address used to
identify the transmitting device, and it uses the least significant bit
first. Broadcast and multicast address formats are illegal within the SA
field.

\textbf{Length or Type} 802.3 uses a Length field, but the Ethernet\_II
frame uses a Type field to identify the Network layer protocol. The old,
original 802.3 cannot identify the upper-layer protocol and must be used
with a proprietary LAN---IPX, for example.

\textbf{\protect\hypertarget{c02.xhtmlux5cux23Page_55}{}{}Data} This is
a packet sent down to the Data Link layer from the Network layer. The
size can vary from 46 to 1,500 bytes.

\textbf{Frame Check Sequence (FCS)} FCS is a field at the end of the
frame that's used to store the cyclic redundancy check (CRC) answer. The
CRC is a mathematical algorithm that's run when each frame is built
based on the data in the frame. When a receiving host receives the frame
and runs the CRC, the answer should be the same. If not, the frame is
discarded, assuming errors have occurred.

Let's pause here for a minute and take a look at some frames caught on
my trusty network analyzer. You can see that the frame below has only
three fields: Destination, Source, and Type, which is shown as Protocol
Type on this particular analyzer:

\begin{verbatim}
Destination:   00:60:f5:00:1f:27
Source:        00:60:f5:00:1f:2c
Protocol Type: 08-00 IP
\end{verbatim}

This is an Ethernet\_II frame. Notice that the Type field is IP, or
08-00, mostly just referred to as 0x800 in hexadecimal.

The next frame has the same fields, so it must be an Ethernet\_II frame
as well:

\begin{verbatim}
Destination:   ff:ff:ff:ff:ff:ff Ethernet Broadcast
Source:        02:07:01:22:de:a4
Protocol Type: 08-00 IP
\end{verbatim}

Did you notice that this frame was a broadcast? You can tell because the
destination hardware address is all 1s in binary, or all \emph{F}s in
hexadecimal.

Let's take a look at one more Ethernet\_II frame. I'll talk about this
next example again when we use IPv6 in Chapter 14, ``Internet Protocol
Version 6 (IPv6),'' but you can see that the Ethernet frame is the same
Ethernet\_II frame used with the IPv4 routed protocol. The Type field
has 0x86dd when the frame is carrying IPv6 data, and when we have IPv4
data, the frame uses 0x0800 in the protocol field:

\begin{verbatim}
Destination: IPv6-Neighbor-Discovery_00:01:00:03 (33:33:00:01:00:03)
Source: Aopen_3e:7f:dd (00:01:80:3e:7f:dd)
Type: IPv6 (0x86dd)
\end{verbatim}

This is the beauty of the Ethernet\_II frame. Because of the Type field,
we can run any Network layer routed protocol and the frame will carry
the data because it can identify the Network layer protocol!

\subsubsection[Ethernet at the Physical
Layer]{\texorpdfstring{\protect\hypertarget{c02.xhtmlux5cux23c02-sec-7}{}{}Ethernet
at the Physical Layer}{Ethernet at the Physical Layer}}

Ethernet was first implemented by a group called DIX, which stands for
Digital, Intel, and Xerox. They created and implemented the first
Ethernet LAN specification, which the IEEE used to create the IEEE 802.3
committee. This was a 10 Mbps network that ran on coax and then
eventually twisted-pair and fiber physical media.

\protect\hypertarget{c02.xhtmlux5cux23Page_56}{}{}The IEEE extended the
802.3 committee to three new committees known as 802.3u (Fast Ethernet),
802.3ab (Gigabit Ethernet on category 5), and then finally one more,
802.3ae (10 Gbps over fiber and coax). There are more standards evolving
almost daily, such as the new 100 Gbps Ethernet (802.3ba)!

When designing your LAN, it's really important to understand the
different types of Ethernet media available to you. Sure, it would be
great to run Gigabit Ethernet to each desktop and 10 Gbps between
switches, but you would need to figure out how to justify the cost of
that network today! However, if you mix and match the different types of
Ethernet media methods currently available, you can come up with a
cost-effective network solution that works really great.

The \emph{EIA/TIA} (Electronic Industries Alliance and the newer
Telecommunications Industry Association) is the standards body that
creates the Physical layer specifications for Ethernet. The EIA/TIA
specifies that Ethernet use a \emph{registered jack (RJ) connector} on
\emph{unshielded twisted-pair (UTP)} cabling (RJ45). But the industry is
moving toward simply calling this an 8-pin modular connector.

Every Ethernet cable type that's specified by the EIA/TIA has inherent
attenuation, which is defined as the loss of signal strength as it
travels the length of a cable and is measured in decibels (dB). The
cabling used in corporate and home markets is measured in categories. A
higher-quality cable will have a higher-rated category and lower
attenuation. For example, category 5 is better than category 3 because
category 5 cables have more wire twists per foot and therefore less
crosstalk. Crosstalk is the unwanted signal interference from adjacent
pairs in the cable.

Here is a list of some of the most common IEEE Ethernet standards,
starting with 10 Mbps Ethernet:

\textbf{10Base-T (IEEE 802.3)} 10 Mbps using category 3 unshielded
twisted pair (UTP) wiring for runs up to 100 meters. Unlike with the
10Base-2 and 10Base-5 networks, each device must connect into a hub or
switch, and you can have only one host per segment or wire. It uses an
RJ45 connector (8-pin modular connector) with a physical star topology
and a logical bus.

\textbf{100Base-TX (IEEE 802.3u)} 100Base-TX, most commonly known as
Fast Ethernet, uses EIA/TIA category 5, 5E, or 6 UTP two-pair wiring.
One user per segment; up to 100 meters long. It uses an RJ45 connector
with a physical star topology and a logical bus.

\textbf{100Base-FX (IEEE 802.3u)} Uses fiber cabling 62.5/125-micron
multimode fiber. Point-to-point topology; up to 412 meters long. It uses
ST and SC connectors, which are media-interface connectors.

\textbf{1000Base-CX (IEEE 802.3z)} Copper twisted-pair, called twinax,
is a balanced coaxial pair that can run only up to 25 meters and uses a
special 9-pin connector known as the High Speed Serial Data Connector
(HSSDC). This is used in Cisco's new Data Center technologies.

\textbf{\protect\hypertarget{c02.xhtmlux5cux23Page_57}{}{}1000Base-T
(IEEE 802.3ab)} Category 5, four-pair UTP wiring up to 100 meters long
and up to 1 Gbps.

\textbf{1000Base-SX (IEEE 802.3z)} The implementation of 1 Gigabit
Ethernet running over multimode fiber-optic cable instead of copper
twisted-pair cable, using short wavelength laser. Multimode fiber (MMF)
using 62.5- and 50-micron core; uses an 850 nanometer (nm) laser and can
go up to 220 meters with 62.5-micron, 550 meters with 50-micron.

\textbf{1000Base-LX (IEEE 802.3z)} Single-mode fiber that uses a
9-micron core and 1300 nm laser and can go from 3 kilometers up to 10
kilometers.

\textbf{1000Base-ZX (Cisco standard)} 1000BaseZX, or 1000Base-ZX, is a
Cisco specified standard for Gigabit Ethernet communication. 1000BaseZX
operates on ordinary single-mode fiber-optic links with spans up to 43.5
miles (70 km).

\textbf{10GBase-T (802.3.an)} 10GBase-T is a standard proposed by the
IEEE 802.3an committee to provide 10 Gbps connections over conventional
UTP cables, (category 5e, 6, or 7 cables). 10GBase-T allows the
conventional RJ45 used for Ethernet LANs and can support signal
transmission at the full 100-meter distance specified for LAN wiring.

\begin{center}\rule{0.5\linewidth}{0.5pt}\end{center}

\includegraphics{images/tip.png} If you want to implement a network
medium that is not susceptible to electromagnetic interference (EMI),
fiber-optic cable provides a more secure, long-distance cable that is
not susceptible to EMI at high speeds.

\begin{center}\rule{0.5\linewidth}{0.5pt}\end{center}

Armed with the basics covered so far in this chapter, you're equipped to
go to the next level and put Ethernet to work using various Ethernet
cabling.

\begin{center}\rule{0.5\linewidth}{0.5pt}\end{center}

\subsubsection[\hfill\break
Interference or Host Distance
Issue?]{\texorpdfstring{\protect\includegraphics{images/earth.png}\\
Interference or Host Distance
Issue?}{ Interference or Host Distance Issue?}}

Quite a few years ago, I was consulting at a very large aerospace
company in the Los Angeles area. In the very busy warehouse, they had
hundreds of hosts providing many different services to the various
departments working in that area.

However, a small group of hosts had been experiencing intermittent
outages that no one could explain since most hosts in the same area had
no problems whatsoever. So I decided to take a crack at this problem and
see what I could find.

First, I traced the backbone connection from the main switch to multiple
switches in the warehouse area. Assuming that the hosts with the issues
were connected to the same switch, I traced each cable, and much to my
surprise they were connected to various switches! Now my interest really
peaked because the simplest issue had been eliminated right off the bat.
It wasn't a simple switch problem!

\protect\hypertarget{c02.xhtmlux5cux23Page_58}{}{}I continued to trace
each cable one by one, and this is what I found:

\begin{figure}
\centering
\includegraphics{images/c02f009.jpg}
\caption{}
\end{figure}

As I drew this network out, I noticed that they had many repeaters in
place, which isn't a cause for immediate suspicion since bandwidth was
not their biggest requirement here. So I looked deeper still. At this
point, I decided to measure the distance of one of the intermittent
hosts connecting to their hub/repeater.

This is what I measured. Can you see the problem?

\begin{figure}
\centering
\includegraphics{images/c02f010.jpg}
\caption{}
\end{figure}

\protect\hypertarget{c02.xhtmlux5cux23Page_59}{}{}Having a hub or
repeater in your network isn't a problem, unless you need better
bandwidth (which they didn't in this case), but the distance was! It's
not always easy to tell how far away a host is from its connection in an
extremely large area, so these hosts ended up having a connection past
the 100-meter Ethernet specification, which created a problem for the
hosts not cabled correctly. Understand that this didn't stop the hosts
from completely working, but the workers felt the hosts stopped working
when they were at their most stressful point of the day. Sure, that
makes sense, because whenever my host stops working, that becomes my
most stressful part of the day!

\begin{center}\rule{0.5\linewidth}{0.5pt}\end{center}

\subsection[Ethernet
Cabling]{\texorpdfstring{\protect\hypertarget{c02.xhtmlux5cux23c02-sec-8}{}{}Ethernet
Cabling}{Ethernet Cabling}}

A discussion about Ethernet cabling is an important one, especially if
you are planning on taking the Cisco exams. You need to really
understand the following three types of cables:

\begin{enumerate}
\tightlist
\item
  Straight-through cable
\item
  Crossover cable
\item
  Rolled cable
\end{enumerate}

We will look at each in the following sections, but first, let's take a
look at the most common Ethernet cable used today, the category 5
Enhanced Unshielded Twisted Pair (UTP), shown in
\protect\hyperlink{c02.xhtmlux5cux23figure02-9}{Figure 2.9}.

\begin{figure}
\centering
\includegraphics{images/c02f011.jpg}
\caption{{\protect\hyperlink{c02.xhtmlux5cux23figureanchor02-9}{\textbf{FIGURE
2.9}} Category 5 Enhanced UTP cable}}
\end{figure}

The category 5 Enhanced UTP cable can handle speeds up to a gigabit with
a distance of up to 100 meters. Typically we'd use this cable for 100
Mbps and category 6 for a gigabit, but the category 5 Enhanced is rated
for gigabit speeds and category 6 is rated for 10 Gbps!

\subsubsection[Straight-Through
Cable]{\texorpdfstring{\protect\hypertarget{c02.xhtmlux5cux23c02-sec-9}{}{}Straight-Through
Cable}{Straight-Through Cable}}

The \emph{straight-through cable} is used to connect the following
devices:

\begin{enumerate}
\tightlist
\item
  Host to switch or hub
\item
  Router to switch or hub
\end{enumerate}

\protect\hypertarget{c02.xhtmlux5cux23Page_60}{}{}Four wires are used in
straight-through cable to connect Ethernet devices. It's relatively
simple to create this type, and
\protect\hyperlink{c02.xhtmlux5cux23figure02-10}{Figure 2.10} shows the
four wires used in a straight-through Ethernet cable.

\begin{figure}
\centering
\includegraphics{images/c02f012.jpg}
\caption{{\protect\hyperlink{c02.xhtmlux5cux23figureanchor02-10}{\textbf{FIGURE
2.10}} Straight-through Ethernet cable}}
\end{figure}

Notice that only pins 1, 2, 3, and 6 are used. Just connect 1 to 1, 2 to
2, 3 to 3, and 6 to 6 and you'll be up and networking in no time.
However, remember that this would be a 10/100 Mbps Ethernet-only cable
and wouldn't work with gigabit, voice, or other LAN or WAN technology.

\subsubsection[Crossover
Cable]{\texorpdfstring{\protect\hypertarget{c02.xhtmlux5cux23c02-sec-10}{}{}Crossover
Cable}{Crossover Cable}}

The \emph{crossover cable} can be used to connect the following devices:

\begin{enumerate}
\tightlist
\item
  Switch to switch
\item
  Hub to hub
\item
  Host to host
\item
  Hub to switch
\item
  Router direct to host
\item
  Router to router
\end{enumerate}

The same four wires used in the straight-through cable are used in this
cable---we just connect different pins together.
\protect\hyperlink{c02.xhtmlux5cux23figure02-11}{Figure 2.11} shows how
the four wires are used in a crossover Ethernet cable.

\protect\hypertarget{c02.xhtmlux5cux23Page_61}{}{}

\begin{figure}
\centering
\includegraphics{images/c02f013.jpg}
\caption{{\protect\hyperlink{c02.xhtmlux5cux23figureanchor02-11}{\textbf{FIGURE
2.11}} Crossover Ethernet cable}}
\end{figure}

Notice that instead of connecting 1 to 1, 2 to 2, and so on, here we
connect pins 1 to 3 and 2 to 6 on each side of the cable.
\protect\hyperlink{c02.xhtmlux5cux23figure02-12}{Figure 2.12} shows some
typical uses of straight-through and crossover cables.

\begin{figure}
\centering
\includegraphics{images/c02f014.jpg}
\caption{{\protect\hyperlink{c02.xhtmlux5cux23figureanchor02-12}{\textbf{FIGURE
2.12}} Typical uses for straight-through and cross-over Ethernet
cables}}
\end{figure}

The crossover examples in
\protect\hyperlink{c02.xhtmlux5cux23figure02-12}{Figure 2.12} are switch
port to switch port, router Ethernet port to router Ethernet port, and
router Ethernet port to PC Ethernet port. For the straight-through
examples I used PC Ethernet to switch port and router Ethernet port to
switch port.

\begin{center}\rule{0.5\linewidth}{0.5pt}\end{center}

\includegraphics{images/tip.png} It's very possible to connect a
straight-through cable between two switches, and it will start working
because of autodetect mechanisms called auto-mdix. But be advised that
the CCNA objectives do not typically consider autodetect mechanisms
valid between devices!

\begin{center}\rule{0.5\linewidth}{0.5pt}\end{center}

\paragraph[UTP Gigabit Wiring
(1000Base-T)]{\texorpdfstring{\protect\hypertarget{c02.xhtmlux5cux23Page_62}{}{}UTP
Gigabit Wiring (1000Base-T)}{UTP Gigabit Wiring (1000Base-T)}}

In the previous examples of 10Base-T and 100Base-T UTP wiring, only two
wire pairs were used, but that is not good enough for Gigabit UTP
transmission.

1000Base-T UTP wiring
(\protect\hyperlink{c02.xhtmlux5cux23figure02-13}{Figure 2.13}) requires
four wire pairs and uses more advanced electronics so that each and
every pair in the cable can transmit simultaneously. Even so, gigabit
wiring is almost identical to my earlier 10/100 example, except that
we'll use the other two pairs in the cable.

\begin{figure}
\centering
\includegraphics{images/c02f015.jpg}
\caption{{\protect\hyperlink{c02.xhtmlux5cux23figureanchor02-13}{\textbf{FIGURE
2.13}} UTP Gigabit crossover Ethernet cable}}
\end{figure}

For a straight-through cable it's still 1 to 1, 2 to 2, and so on up to
pin 8. And in creating the gigabit crossover cable, you'd still cross 1
to 3 and 2 to 6, but you would add 4 to 7 and 5 to 8---pretty
straightforward!

\subsubsection[Rolled
Cable]{\texorpdfstring{\protect\hypertarget{c02.xhtmlux5cux23c02-sec-11}{}{}Rolled
Cable}{Rolled Cable}}

Although \emph{rolled cable} isn't used to connect any Ethernet
connections together, you can use a rolled Ethernet cable to connect a
host EIA-TIA 232 interface to a router console serial communication
(COM) port.

If you have a Cisco router or switch, you would use this cable to
connect your PC, Mac, or a device like an iPad to the Cisco hardware.
Eight wires are used in this cable to connect serial devices, although
not all eight are used to send information, just as in Ethernet
networking. \protect\hyperlink{c02.xhtmlux5cux23figure02-14}{Figure
2.14} shows the eight wires used in a rolled cable.

\begin{figure}
\centering
\includegraphics{images/c02f016.jpg}
\caption{{\protect\hyperlink{c02.xhtmlux5cux23figureanchor02-14}{\textbf{FIGURE
2.14}} Rolled Ethernet cable}}
\end{figure}

\protect\hypertarget{c02.xhtmlux5cux23Page_63}{}{}These are probably the
easiest cables to make because you just cut the end off on one side of a
straight-through cable, turn it over, and put it back on---with a new
connector, of course!

Okay, once you have the correct cable connected from your PC to the
Cisco router or switch console port, you can start your emulation
program such as PuTTY or SecureCRT to create a console connection and
configure the device. Set the configuration as shown in
\protect\hyperlink{c02.xhtmlux5cux23figure02-15}{Figure 2.15}.

\begin{figure}
\centering
\includegraphics{images/c02f017.jpg}
\caption{{\protect\hyperlink{c02.xhtmlux5cux23figureanchor02-15}{\textbf{FIGURE
2.15}} Configuring your console emulation program}}
\end{figure}

Notice that Baud Rate is set to 9600, Data Bits to 8, Parity to None,
and no Flow Control options are set. At this point, you can click
Connect and press the Enter key and you should be connected to your
Cisco device console port.

\protect\hyperlink{c02.xhtmlux5cux23figure02-16}{Figure 2.16} shows a
nice new 2960 switch with two console ports.

\begin{figure}
\centering
\includegraphics{images/c02f018.jpg}
\caption{{\protect\hyperlink{c02.xhtmlux5cux23figureanchor02-16}{\textbf{FIGURE
2.16}} A Cisco 2960 console connections}}
\end{figure}

Notice there are two console connections on this new switch---a typical
original RJ45 connection and the newer mini type-B USB console. Remember
that the new USB port supersedes the RJ45 port if you just happen to
plug into both at the same time, and the USB port can have speeds up to
115,200 Kbps, which is awesome if you have to use Xmodem to
\protect\hypertarget{c02.xhtmlux5cux23Page_64}{}{}update an IOS. I've
even seen some cables that work on iPhones and iPads and allow them to
connect to these mini USB ports!

Now that you've seen the various RJ45 unshielded twisted-pair (UTP)
cables, what type of cable is used between the switches in
\protect\hyperlink{c02.xhtmlux5cux23figure02-17}{Figure 2.17}?

\begin{figure}
\centering
\includegraphics{images/c02f019.jpg}
\caption{{\protect\hyperlink{c02.xhtmlux5cux23figureanchor02-17}{\textbf{FIGURE
2.17}} RJ45 UTP cable question \#1}}
\end{figure}

In order for host A to ping host B, you need a crossover cable to
connect the two switches together. But what types of cables are used in
the network shown in
\protect\hyperlink{c02.xhtmlux5cux23figure02-18}{Figure 2.18}?

\begin{figure}
\centering
\includegraphics{images/c02f020.jpg}
\caption{{\protect\hyperlink{c02.xhtmlux5cux23figureanchor02-18}{\textbf{FIGURE
2.18}} RJ45 UTP cable question \#2}}
\end{figure}

In \protect\hyperlink{c02.xhtmlux5cux23figure02-18}{Figure 2.18},
there's a whole menu of cables in use. For the connection between the
switches, we'd obviously use a crossover cable like we saw in
\protect\hyperlink{c02.xhtmlux5cux23figure02-13}{Figure 2.13}. The
trouble is that you must understand that we have a console connection
that uses a rolled cable. Plus, the connection from the router to the
switch is a straight-through cable, as is true for the hosts to the
switches. Keep in mind that if we had a serial connection, which we
don't, we would use a V.35 to connect us to a WAN.

\subsubsection[Fiber
Optic]{\texorpdfstring{\protect\hypertarget{c02.xhtmlux5cux23c02-sec-12}{}{}Fiber
Optic}{Fiber Optic}}

Fiber-optic cabling has been around for a long time and has some solid
standards. The cable allows for very fast transmission of data, is made
of glass (or even plastic!), is very thin, and works as a waveguide to
transmit light between two ends of the fiber. Fiber optics has been used
to go very long distances, as in intercontinental connections, but it is
\protect\hypertarget{c02.xhtmlux5cux23Page_65}{}{}becoming more and more
popular in Ethernet LAN networks due to the fast speeds available and
because, unlike UTP, it's immune to interference like cross-talk.

Some main components of this cable are the core and the cladding. The
core will hold the light and the cladding confines the light in the
core. The tighter the cladding, the smaller the core, and when the core
is small, less light will be sent, but it can go faster and farther!

In \protect\hyperlink{c02.xhtmlux5cux23figure02-19}{Figure 2.19} you can
see that there is a 9-micron core, which is very small and can be
measured against a human hair, which is 50 microns.

\begin{figure}
\centering
\includegraphics{images/c02f021.jpg}
\caption{{\protect\hyperlink{c02.xhtmlux5cux23figureanchor02-19}{\textbf{FIGURE
2.19}} Typical fiber cable.}}
\end{figure}

Dimensions are in um (10\textsuperscript{--6} meters). Not to scale.

The cladding is 125 microns, which is actually a fiber standard that
allows manufacturers to make connectors for all fiber cables. The last
piece of this cable is the buffer, which is there to protect the
delicate glass.

There are two major types of fiber optics: single-mode and multimode.
\protect\hyperlink{c02.xhtmlux5cux23figure02-20}{Figure 2.20} shows the
differences between multimode and single-mode fibers.

\begin{figure}
\centering
\includegraphics{images/c02f022.jpg}
\caption{{\protect\hyperlink{c02.xhtmlux5cux23figureanchor02-20}{\textbf{FIGURE
2.20}} Multimode and single-mode fibers}}
\end{figure}

\protect\hypertarget{c02.xhtmlux5cux23Page_66}{}{}Single-mode is more
expensive, has a tighter cladding, and can go much farther distances
than multimode. The difference comes in the tightness of the cladding,
which makes a smaller core, meaning that only one mode of light will
propagate down the fiber. Multimode is looser and has a larger core so
it allows multiple light particles to travel down the glass. These
particles have to be put back together at the receiving end, so distance
is less than that with single-mode fiber, which allows only very few
light particles to travel down the fiber.

There are about 70 different connectors for fiber, and Cisco uses a few
different types. Looking back at
\protect\hyperlink{c02.xhtmlux5cux23figure02-16}{Figure 2.16}, the two
bottom ports are referred to as Small Form-Factor Pluggables, or SFPs.

\subsection[Data
Encapsulation]{\texorpdfstring{\protect\hypertarget{c02.xhtmlux5cux23c02-sec-13}{}{}Data
Encapsulation}{Data Encapsulation}}

When a host transmits data across a network to another device, the data
goes through a process called \emph{encapsulation} and is wrapped with
protocol information at each layer of the OSI model. Each layer
communicates only with its peer layer on the receiving device.

To communicate and exchange information, each layer uses \emph{protocol
data units (PDUs)}. These hold the control information attached to the
data at each layer of the model. They are usually attached to the header
in front of the data field but can also be at the trailer, or end, of
it.

Each PDU attaches to the data by encapsulating it at each layer of the
OSI model, and each has a specific name depending on the information
provided in each header. This PDU information is read only by the peer
layer on the receiving device. After its read, it's stripped off and the
data is then handed to the next layer up.

\protect\hyperlink{c02.xhtmlux5cux23figure02-21}{Figure 2.21} shows the
PDUs and how they attach control information to each layer. This figure
demonstrates how the upper-layer user data is converted for transmission
on the network. The data stream is then handed down to the Transport
layer, which sets up a virtual circuit to the receiving device by
sending over a synch packet. Next, the data stream is broken up into
smaller pieces, and a Transport layer header is created and attached to
the header of the data field; now the piece of data is called a
\emph{segment} (a PDU). Each segment can be sequenced so the data stream
can be put back together on the receiving side exactly as it was
transmitted.

\begin{figure}
\centering
\includegraphics{images/c02f023.jpg}
\caption{{\protect\hyperlink{c02.xhtmlux5cux23figureanchor02-21}{\textbf{FIGURE
2.21}} Data encapsulation}}
\end{figure}

\protect\hypertarget{c02.xhtmlux5cux23Page_67}{}{}Each segment is then
handed to the Network layer for network addressing and routing through
the internetwork. Logical addressing (for example, IP and IPv6) is used
to get each segment to the correct network. The Network layer protocol
adds a control header to the segment handed down from the Transport
layer, and what we have now is called a \emph{packet} or
\emph{datagram}. Remember that the Transport and Network layers work
together to rebuild a data stream on a receiving host, but it's not part
of their work to place their PDUs on a local network segment---which is
the only way to get the information to a router or host.

It's the Data Link layer that's responsible for taking packets from the
Network layer and placing them on the network medium (cable or
wireless). The Data Link layer encapsulates each packet in a
\emph{frame}, and the frame's header carries the hardware addresses of
the source and destination hosts. If the destination device is on a
remote network, then the frame is sent to a router to be routed through
an internetwork. Once it gets to the destination network, a new frame is
used to get the packet to the destination host.

To put this frame on the network, it must first be put into a digital
signal. Since a frame is really a logical group of 1s and 0s, the
physical layer is responsible for encoding these digits into a digital
signal, which is read by devices on the same local network. The
receiving devices will synchronize on the digital signal and extract
(decode) the 1s and 0s from the digital signal. At this point, the
devices reconstruct the frames, run a CRC, and then check their answer
against the answer in the frame's FCS field. If it matches, the packet
is pulled from the frame and what's left of the frame is discarded. This
process is called \emph{de-encapsulation}. The packet is handed to the
Network layer, where the address is checked. If the address matches, the
segment is pulled from the packet and what's left of the packet is
discarded. The segment is processed at the Transport layer, which
rebuilds the data stream and acknowledges to the transmitting station
that it received each piece. It then happily hands the data stream to
the upper-layer application.

At a transmitting device, the data encapsulation method works like this:

\begin{enumerate}
\tightlist
\item
  User information is converted to data for transmission on the network.
\item
  Data is converted to segments, and a reliable connection is set up
  between the transmitting and receiving hosts.
\item
  Segments are converted to packets or datagrams, and a logical address
  is placed in the header so each packet can be routed through an
  internetwork.
\item
  Packets or datagrams are converted to frames for transmission on the
  local network. Hardware (Ethernet) addresses are used to uniquely
  identify hosts on a local network segment.
\item
  Frames are converted to bits, and a digital encoding and clocking
  scheme is used.
\end{enumerate}

To explain this in more detail using the layer addressing, I'll use
\protect\hyperlink{c02.xhtmlux5cux23figure02-22}{Figure 2.22}.

Remember that a data stream is handed down from the upper layer to the
Transport layer. As technicians, we really don't care who the data
stream comes from because that's really a programmer's problem. Our job
is to rebuild the data stream reliably and hand it to the upper layers
on the receiving device.

\protect\hypertarget{c02.xhtmlux5cux23Page_68}{}{}

\begin{figure}
\centering
\includegraphics{images/c02f024.jpg}
\caption{{\protect\hyperlink{c02.xhtmlux5cux23figureanchor02-22}{\textbf{FIGURE
2.22}} PDU and layer addressing}}
\end{figure}

Before we go further in our discussion of
\protect\hyperlink{c02.xhtmlux5cux23figure02-22}{Figure 2.22}, let's
discuss port numbers and make sure you understand them. The Transport
layer uses port numbers to define both the virtual circuit and the
upper-layer processes, as you can see from
\protect\hyperlink{c02.xhtmlux5cux23figure02-23}{Figure 2.23}.

\begin{figure}
\centering
\includegraphics{images/c02f025.jpg}
\caption{{\protect\hyperlink{c02.xhtmlux5cux23figureanchor02-23}{\textbf{FIGURE
2.23}} Port numbers at the Transport layer}}
\end{figure}

When using a connection-oriented protocol like TCP, the Transport layer
takes the data stream, makes segments out of it, and establishes a
reliable session by creating a virtual circuit. It then sequences
(numbers) each segment and uses acknowledgments and flow control. If
you're using TCP, the virtual circuit is defined by the source and
destination port number plus the source and destination IP address and
called a socket. Understand that the host just makes this up, starting
at port number 1024 because 0 through 1023 are reserved for well-known
port numbers. The destination port number defines the upper-layer
process or application that the data stream is handed to when the data
stream is reliably rebuilt on the receiving host.

\protect\hypertarget{c02.xhtmlux5cux23Page_69}{}{}Now that you
understand port numbers and how they are used at the Transport layer,
let's go back to \protect\hyperlink{c02.xhtmlux5cux23figure02-22}{Figure
2.22}. Once the Transport layer header information is added to the piece
of data, it becomes a segment that's handed down to the Network layer
along with the destination IP address. As you know, the destination IP
address was handed down from the upper layers to the Transport layer
with the data stream and was identified via name resolution at the upper
layers---probably with DNS.

The Network layer adds a header and adds the logical addressing such as
IP addresses to the front of each segment. Once the header is added to
the segment, the PDU is called a packet. The packet has a protocol field
that describes where the segment came from (either UDP or TCP) so it can
hand the segment to the correct protocol at the Transport layer when it
reaches the receiving host.

The Network layer is responsible for finding the destination hardware
address that dictates where the packet should be sent on the local
network. It does this by using the Address Resolution Protocol
(ARP)---something I'll talk about more in Chapter 3. IP at the Network
layer looks at the destination IP address and compares that address to
its own source IP address and subnet mask. If it turns out to be a local
network request, the hardware address of the local host is requested via
an ARP request. If the packet is destined for a host on a remote
network, IP will look for the IP address of the default gateway (router)
instead.

The packet, along with the destination hardware address of either the
local host or default gateway, is then handed down to the Data Link
layer. The Data Link layer will add a header to the front of the packet
and the piece of data then becomes a frame. It's called a frame because
both a header and a trailer are added to the packet, which makes it look
like it's within bookends---a frame---as shown in
\protect\hyperlink{c02.xhtmlux5cux23figure02-22}{Figure 2.22}. The frame
uses an Ether-Type field to describe which protocol the packet came from
at the Network layer. Now a cyclic redundancy check is run on the frame,
and the answer to the CRC is placed in the Frame Check Sequence field
found in the trailer of the frame.

The frame is now ready to be handed down, one bit at a time, to the
Physical layer, which will use bit-timing rules to encode the data in a
digital signal. Every device on the network segment will receive the
digital signal and synchronize with the clock and extract the 1s and 0s
from the digital signal to build a frame. After the frame is rebuilt, a
CRC is run to make sure the frame is in proper order. If everything
turns out to be all good, the hosts will check the destination MAC and
IP addresses to see if the frame is for them.

If all this is making your eyes cross and your brain freeze, don't
freak. I'll be going over exactly how data is encapsulated and routed
through an internetwork later, in Chapter 9, ``IP Routing.''

\subsection[The Cisco Three-Layer Hierarchical
Model]{\texorpdfstring{\protect\hypertarget{c02.xhtmlux5cux23c02-sec-14}{}{}The
Cisco Three-Layer Hierarchical
Model}{The Cisco Three-Layer Hierarchical Model}}

Most of us were exposed to hierarchy early in life. Anyone with older
siblings learned what it was like to be at the bottom of the hierarchy.
Regardless of where you first discovered the concept of hierarchy, most
of us experience it in many aspects of our lives. It's \emph{hierarchy}
\protect\hypertarget{c02.xhtmlux5cux23Page_70}{}{}that helps us
understand where things belong, how things fit together, and what
functions go where. It brings order to otherwise complex models. If you
want a pay raise, for instance, hierarchy dictates that you ask your
boss, not your subordinate, because that's the person whose role it is
to grant or deny your request. So basically, understanding hierarchy
helps us discern where we should go to get what we need.

Hierarchy has many of the same benefits in network design that it does
in other areas of life. When used properly, it makes networks more
predictable and helps us define which areas should perform certain
functions. Likewise, you can use tools such as access lists at certain
levels in hierarchical networks and avoid them at others.

Let's face it: Large networks can be extremely complicated, with
multiple protocols, detailed configurations, and diverse technologies.
Hierarchy helps us summarize a complex collection of details into an
understandable model, bringing order from the chaos. Then, as specific
configurations are needed, the model dictates the appropriate manner in
which to apply them.

The Cisco hierarchical model can help you design, implement, and
maintain a scalable, reliable, cost-effective hierarchical internetwork.
Cisco defines three layers of hierarchy, as shown in
\protect\hyperlink{c02.xhtmlux5cux23figure02-24}{Figure 2.24}, each with
specific functions.

\begin{figure}
\centering
\includegraphics{images/c02f026.jpg}
\caption{{\protect\hyperlink{c02.xhtmlux5cux23figureanchor02-24}{\textbf{FIGURE
2.24}} The Cisco hierarchical model}}
\end{figure}

Each layer has specific responsibilities. Keep in mind that the three
layers are logical and are not necessarily physical devices. Consider
the OSI model, another logical hierarchy. Its seven layers describe
functions but not necessarily protocols, right? Sometimes a protocol
\protect\hypertarget{c02.xhtmlux5cux23Page_71}{}{}maps to more than one
layer of the OSI model, and sometimes multiple protocols communicate
within a single layer. In the same way, when we build physical
implementations of hierarchical networks, we may have many devices in a
single layer, or there may be a single device performing functions at
two layers. Just remember that the definition of the layers is logical,
not physical!

So let's take a closer look at each of the layers now.

\subsubsection[The Core
Layer]{\texorpdfstring{\protect\hypertarget{c02.xhtmlux5cux23c02-sec-15}{}{}The
Core Layer}{The Core Layer}}

The \emph{core layer} is literally the core of the network. At the top
of the hierarchy, the core layer is responsible for transporting large
amounts of traffic both reliably and quickly. The only purpose of the
network's core layer is to switch traffic as fast as possible. The
traffic transported across the core is common to a majority of users.
But remember that user data is processed at the distribution layer,
which forwards the requests to the core if needed.

If there's a failure in the core, \emph{every single user} can be
affected! This is why fault tolerance at this layer is so important. The
core is likely to see large volumes of traffic, so speed and latency are
driving concerns here. Given the function of the core, we can now
consider some design specifics. Let's start with some things we don't
want to do:

\begin{enumerate}
\tightlist
\item
  Never do anything to slow down traffic. This includes making sure you
  don't use access lists, perform routing between virtual local area
  networks, or implement packet filtering.
\item
  Don't support workgroup access here.
\item
  Avoid expanding the core (e.g., adding routers when the internetwork
  grows). If performance becomes an issue in the core, give preference
  to upgrades over expansion.
\end{enumerate}

Here's a list of things that we want to achieve as we design the core:

\begin{enumerate}
\tightlist
\item
  Design the core for high reliability. Consider data-link technologies
  that facilitate both speed and redundancy, like Gigabit Ethernet with
  redundant links or even 10 Gigabit Ethernet.
\item
  Design with speed in mind. The core should have very little latency.
\item
  Select routing protocols with lower convergence times. Fast and
  redundant data-link connectivity is no help if your routing tables are
  shot!
\end{enumerate}

\subsubsection[The Distribution
Layer]{\texorpdfstring{\protect\hypertarget{c02.xhtmlux5cux23c02-sec-16}{}{}The
Distribution Layer}{The Distribution Layer}}

The \emph{distribution layer} is sometimes referred to as the
\emph{workgroup layer} and is the communication point between the access
layer and the core. The primary functions of the distribution layer are
to provide routing, filtering, and WAN access and to determine how
packets can access the core, if needed. The distribution layer must
determine the fastest way that network service requests are
handled---for example, how a file request is forwarded to a server.
After the distribution layer determines the best path, it forwards the
request to the core layer if necessary. The core layer then quickly
transports the request to the correct service.

\protect\hypertarget{c02.xhtmlux5cux23Page_72}{}{}The distribution layer
is where we want to implement policies for the network because we are
allowed a lot of flexibility in defining network operation here. There
are several things that should generally be handled at the distribution
layer:

\begin{enumerate}
\tightlist
\item
  Routing
\item
  Implementing tools (such as access lists), packet filtering, and
  queuing
\item
  Implementing security and network policies, including address
  translation and firewalls
\item
  Redistributing between routing protocols, including static routing
\item
  Routing between VLANs and other workgroup support functions
\item
  Defining broadcast and multicast domains
\end{enumerate}

Key things to avoid at the distribution layer are those that are limited
to functions that exclusively belong to one of the other layers!

\subsubsection[The Access
Layer]{\texorpdfstring{\protect\hypertarget{c02.xhtmlux5cux23c02-sec-17}{}{}The
Access Layer}{The Access Layer}}

The \emph{access layer} controls user and workgroup access to
internetwork resources. The access layer is sometimes referred to as the
\emph{desktop layer}. The network resources most users need will be
available locally because the distribution layer handles any traffic for
remote services.

The following are some of the functions to be included at the access
layer:

\begin{enumerate}
\tightlist
\item
  Continued (from distribution layer) use of access control and policies
\item
  Creation of separate collision domains (microsegmentation/switches)
\item
  Workgroup connectivity into the distribution layer
\item
  Device connectivity
\item
  Resiliency and security services
\item
  Advanced technology capabilities (voice/video, etc.)
\end{enumerate}

Technologies like Gigabit or Fast Ethernet switching are frequently seen
in the access layer.

I can't stress this enough---just because there are three separate
levels does not imply three separate devices! There could be fewer or
there could be more. After all, this is a \emph{layered} approach.

\subsection[Summary]{\texorpdfstring{\protect\hypertarget{c02.xhtmlux5cux23c02-sec-18}{}{}Summary}{Summary}}

In this chapter, you learned the fundamentals of Ethernet networking,
how hosts communicate on a network. You discovered how CSMA/CD works in
an Ethernet half-duplex network.

I also talked about the differences between half- and full-duplex modes,
and we discussed the collision detection mechanism called CSMA/CD.

\protect\hypertarget{c02.xhtmlux5cux23Page_73}{}{}I described the common
Ethernet cable types used in today's networks in this chapter as well,
and by the way, you'd be wise to study that section really well!

Important enough to not gloss over, this chapter provided an
introduction to encapsulation. Encapsulation is the process of encoding
data as it goes down the OSI stack.

Last, I covered the Cisco three-layer hierarchical model. I described in
detail the three layers and how each is used to help design and
implement a Cisco internetwork.

\subsection[Exam
Essentials]{\texorpdfstring{\protect\hypertarget{c02.xhtmlux5cux23c02-sec-19}{}{}Exam
Essentials}{Exam Essentials}}

\textbf{Describe the operation of Carrier Sense Multiple Access with
Collision Detection (CSMA/CD).} CSMA/CD is a protocol that helps devices
share the bandwidth evenly without having two devices transmit at the
same time on the network medium. Although it does not eliminate
collisions, it helps to greatly reduce them, which reduces
retransmissions, resulting in a more efficient transmission of data for
all devices.

\textbf{Differentiate half-duplex and full-duplex communication and
define the requirements to utilize each method.} Full-duplex Ethernet
uses two pairs of wires at the same time instead of one wire pair like
half-duplex. Full-duplex allows for sending and receiving at the same
time, using different wires to eliminate collisions, while half-duplex
can send or receive but not at the same time and still can suffer
collisions. To use full-duplex, the devices at both ends of the cable
must be capable of and configured to perform full-duplex.

\textbf{Describe the sections of a MAC address and the information
contained in each section .} The MAC, or hardware, address is a 48-bit
(6-byte) address written in a hexadecimal format. The first 24 bits, or
3 bytes, are called the organizationally unique identifier (OUI), which
is assigned by the IEEE to the manufacturer of the NIC. The balance of
the number uniquely identifies the NIC.

\textbf{Identify the binary and hexadecimal equivalent of a decimal
number.} Any number expressed in one format can also be expressed in the
other two. The ability to perform this conversion is critical to
understanding IP addressing and subnetting. Be sure to go through the
written labs covering binary to decimal to hexadecimal conversion.

\textbf{Identify the fields in the Data Link portion of an Ethernet
frame.} The fields in the Data Link portion of a frame include the
preamble, Start Frame Delimiter, destination MAC address, source MAC
address, Length or Type, Data, and Frame Check Sequence.

\textbf{Identify the IEEE physical standards for Ethernet cabling.}
These standards describe the capabilities and physical characteristics
of various cable types and include but are not limited to 10Base-2,
10Base-5, and 10Base-T.

\textbf{Differentiate types of Ethernet cabling and identify their
proper application.} The three types of cables that can be created from
an Ethernet cable are straight-through (to connect a PC's or router's
Ethernet interface to a hub or switch), crossover (to connect hub to
hub, hub to switch, switch to switch, or PC to PC), and rolled (for a
console connection from a PC to a router or switch).

\textbf{\protect\hypertarget{c02.xhtmlux5cux23Page_74}{}{}Describe the
data encapsulation process and the role it plays in packet creation.}
Data encapsulation is a process whereby information is added to the
frame from each layer of the OSI model. This is also called packet
creation. Each layer communicates only with its peer layer on the
receiving device.

\textbf{Understand how to connect a console cable from a PC to a router
and switch.} Take a rolled cable and connect it from the COM port of the
host to the console port of a router. Start your emulations program such
as putty or SecureCRT and set the bits per second to 9600 and flow
control to None.

\textbf{Identify the layers in the Cisco three-layer model and describe
the ideal function of each layer.} The three layers in the Cisco
hierarchical model are the core (responsible for transporting large
amounts of traffic both reliably and quickly), distribution (provides
routing, filtering, and WAN access), and access (workgroup connectivity
into the distribution layer).

\subsection[Written
Labs]{\texorpdfstring{\protect\hypertarget{c02.xhtmlux5cux23c02-sec-20}{}{}Written
Labs}{Written Labs}}

In this section, you'll complete the following labs to make sure you've
got the information and concepts contained within them fully dialed in:

\begin{enumerate}
\tightlist
\item
  Lab 2.1: Binary/Decimal/Hexadecimal Conversion
\item
  Lab 2.2: CSMA/CD Operations
\item
  Lab 2.3: Cabling
\item
  Lab 2.4: Encapsulation
\end{enumerate}

You can find the answers to these labs in Appendix A, ``Answers to
Written Labs.''

\subsubsection[Written Lab 2.1: Binary/Decimal/Hexadecimal
Conversion]{\texorpdfstring{\protect\hypertarget{c02.xhtmlux5cux23c02-sec-21}{}{}Written
Lab 2.1: Binary/Decimal/Hexadecimal
Conversion}{Written Lab 2.1: Binary/Decimal/Hexadecimal Conversion}}

\begin{enumerate}
\item
  Convert from decimal IP address to binary format.

  Complete the following table to express 192.168.10.15 in binary
  format.

  \begin{longtable}[]{@{}lllllllll@{}}
  \toprule
  \endhead
  \textbf{128} & \textbf{64} & \textbf{32} & \textbf{16} & \textbf{8} &
  \textbf{4} & \textbf{2} & \textbf{1} & \textbf{Binary}\tabularnewline
  & & & & & & & &\tabularnewline
  \bottomrule
  \end{longtable}

  \protect\hypertarget{c02.xhtmlux5cux23Page_75}{}{}Complete the
  following table to express 172.16.20.55 in binary format.

  \begin{longtable}[]{@{}lllllllll@{}}
  \toprule
  \endhead
  \textbf{128} & \textbf{64} & \textbf{32} & \textbf{16} & \textbf{8} &
  \textbf{4} & \textbf{2} & \textbf{1} & \textbf{Binary}\tabularnewline
  & & & & & & & &\tabularnewline
  \bottomrule
  \end{longtable}

  Complete the following table to express 10.11.12.99 in binary format.

  \begin{longtable}[]{@{}lllllllll@{}}
  \toprule
  \endhead
  \textbf{128} & \textbf{64} & \textbf{32} & \textbf{16} & \textbf{8} &
  \textbf{4} & \textbf{2} & \textbf{1} & \textbf{Binary}\tabularnewline
  & & & & & & & &\tabularnewline
  \bottomrule
  \end{longtable}
\item
  Convert the following from binary format to decimal IP address.

  Complete the following table to express
  11001100.00110011.10101010.01010101 in decimal IP address format.

  \begin{longtable}[]{@{}lllllllll@{}}
  \toprule
  \endhead
  \textbf{128} & \textbf{64} & \textbf{32} & \textbf{16} & \textbf{8} &
  \textbf{4} & \textbf{2} & \textbf{1} & \textbf{Decimal}\tabularnewline
  & & & & & & & &\tabularnewline
  \bottomrule
  \end{longtable}

  \protect\hypertarget{c02.xhtmlux5cux23Page_76}{}{}Complete the
  following table to express 11000110.11010011.00111001.11010001 in
  decimal IP address format.

  \begin{longtable}[]{@{}lllllllll@{}}
  \toprule
  \endhead
  \textbf{128} & \textbf{64} & \textbf{32} & \textbf{16} & \textbf{8} &
  \textbf{4} & \textbf{2} & \textbf{1} & \textbf{Decimal}\tabularnewline
  & & & & & & & &\tabularnewline
  \bottomrule
  \end{longtable}

  Complete the following table to express
  10000100.11010010.10111000.10100110 in decimal IP address format.

  \begin{longtable}[]{@{}lllllllll@{}}
  \toprule
  \endhead
  \textbf{128} & \textbf{64} & \textbf{32} & \textbf{16} & \textbf{8} &
  \textbf{4} & \textbf{2} & \textbf{1} & \textbf{Decimal}\tabularnewline
  & & & & & & & &\tabularnewline
  \bottomrule
  \end{longtable}
\item
  Convert the following from binary format to hexadecimal.

  Complete the following table to express
  11011000.00011011.00111101.01110110 in hexadecimal.

  \begin{longtable}[]{@{}lllllllll@{}}
  \toprule
  \endhead
  \textbf{128} & \textbf{64} & \textbf{32} & \textbf{16} & \textbf{8} &
  \textbf{4} & \textbf{2} & \textbf{1} &
  \textbf{Hexadecimal}\tabularnewline
  & & & & & & & &\tabularnewline
  \bottomrule
  \end{longtable}

  \protect\hypertarget{c02.xhtmlux5cux23Page_77}{}{}Complete the
  following table to express 11001010.11110101.10000011.11101011 in
  hexadecimal.

  \begin{longtable}[]{@{}lllllllll@{}}
  \toprule
  \endhead
  \textbf{128} & \textbf{64} & \textbf{32} & \textbf{16} & \textbf{8} &
  \textbf{4} & \textbf{2} & \textbf{1} &
  \textbf{Hexadecimal}\tabularnewline
  & & & & & & & &\tabularnewline
  \bottomrule
  \end{longtable}

  Complete the following table to express
  10000100.11010010.01000011.10110011 in hexadecimal.

  \begin{longtable}[]{@{}lllllllll@{}}
  \toprule
  \endhead
  \textbf{128} & \textbf{64} & \textbf{32} & \textbf{16} & \textbf{8} &
  \textbf{4} & \textbf{2} & \textbf{1} &
  \textbf{Hexadecimal}\tabularnewline
  & & & & & & & &\tabularnewline
  \bottomrule
  \end{longtable}
\end{enumerate}

\subsubsection[Written Lab 2.2: CSMA/CD
Operations]{\texorpdfstring{\protect\hypertarget{c02.xhtmlux5cux23c02-sec-22}{}{}Written
Lab 2.2: CSMA/CD Operations}{Written Lab 2.2: CSMA/CD Operations}}

Carrier Sense Multiple Access with Collision Detection (CSMA/CD) helps
to minimize collisions in the network, thereby increasing data
transmission efficiency. Place the following steps of its operation in
the order in which they occur after a collision.

\begin{enumerate}
\tightlist
\item
  All hosts have equal priority to transmit after the timers have
  expired.
\item
  Each device on the Ethernet segment stops transmitting for a short
  time until the timers expire.
\item
  The collision invokes a random backoff algorithm.
\item
  A jam signal informs all devices that a collision occurred.
\end{enumerate}

\subsubsection[Written Lab 2.3:
Cabling]{\texorpdfstring{\protect\hypertarget{c02.xhtmlux5cux23c02-sec-23}{}{}\protect\hypertarget{c02.xhtmlux5cux23Page_78}{}{}Written
Lab 2.3: Cabling}{Written Lab 2.3: Cabling}}

For each of the following situations, determine whether a
straight-through, crossover, or rolled cable would be used.

\begin{enumerate}
\tightlist
\item
  Host to host
\item
  Host to switch or hub
\item
  Router direct to host
\item
  Switch to switch
\item
  Router to switch or hub
\item
  Hub to hub
\item
  Hub to switch
\item
  Host to a router console serial communication (COM) port
\end{enumerate}

\subsubsection[Written Lab 2.4:
Encapsulation]{\texorpdfstring{\protect\hypertarget{c02.xhtmlux5cux23c02-sec-24}{}{}Written
Lab 2.4: Encapsulation}{Written Lab 2.4: Encapsulation}}

Place the following steps of the encapsulation process in the proper
order.

\begin{enumerate}
\tightlist
\item
  Packets or datagrams are converted to frames for transmission on the
  local network. Hardware (Ethernet) addresses are used to uniquely
  identify hosts on a local network segment.
\item
  Segments are converted to packets or datagrams, and a logical address
  is placed in the header so each packet can be routed through an
  internetwork.
\item
  User information is converted to data for transmission on the network.
\item
  Frames are converted to bits, and a digital encoding and clocking
  scheme is used.
\item
  Data is converted to segments, and a reliable connection is set up
  between the transmitting and receiving hosts.
\end{enumerate}

\subsection[Review
Questions]{\texorpdfstring{\protect\hypertarget{c02.xhtmlux5cux23c02-sec-25}{}{}\protect\hypertarget{c02.xhtmlux5cux23Page_79}{}{}Review
Questions}{Review Questions}}

\begin{center}\rule{0.5\linewidth}{0.5pt}\end{center}

\includegraphics{images/note.png} The following questions are designed
to test your understanding of this chapter's material. For more
information on how to get additional questions, please see
\texttt{www.lammle.com/ccna}.

\begin{center}\rule{0.5\linewidth}{0.5pt}\end{center}

You can find the answers to these questions in Appendix B, ``Answers to
Review Questions.''

\begin{enumerate}
\item
  In the accompanying graphic, what is the name for the section of the
  MAC address marked as unknown?

  \begin{figure}
  \centering
  \includegraphics{images/c02f027.jpg}
  \caption{}
  \end{figure}

  \begin{enumerate}
  \def\labelenumii{\Alph{enumii}.}
  \tightlist
  \item
    IOS
  \item
    OSI
  \item
    ISO
  \item
    OUI
  \end{enumerate}
\item
  \_\_\_\_\_\_\_\_\_\_on an Ethernet network is the retransmission delay
  that's enforced when a collision occurs.

  \begin{enumerate}
  \def\labelenumii{\Alph{enumii}.}
  \tightlist
  \item
    Backoff
  \item
    Carrier sense
  \item
    Forward delay
  \item
    Jamming
  \end{enumerate}
\item
  On which type of device could the situation shown in the diagram
  occur?

  \begin{figure}
  \centering
  \includegraphics{images/c02f028.jpg}
  \caption{}
  \end{figure}

  \begin{enumerate}
  \def\labelenumii{\Alph{enumii}.}
  \tightlist
  \item
    \protect\hypertarget{c02.xhtmlux5cux23Page_80}{}{}Hub
  \item
    Switch
  \item
    Router
  \item
    Bridge
  \end{enumerate}
\item
  In the Ethernet II frame shown here, what is the function of the
  section labeled ``FCS''?

  \begin{figure}
  \centering
  \includegraphics{images/c02f029.jpg}
  \caption{}
  \end{figure}

  \begin{enumerate}
  \def\labelenumii{\Alph{enumii}.}
  \tightlist
  \item
    Allows the receiving devices to lock the incoming bit stream.
  \item
    Error detection
  \item
    Identifies the upper-layer protocol
  \item
    Identifies the transmitting device
  \end{enumerate}
\item
  A network interface port has collision detection and carrier sensing
  enabled on a shared twisted-pair network. From this statement, what is
  known about the network interface port?

  \begin{enumerate}
  \def\labelenumii{\Alph{enumii}.}
  \tightlist
  \item
    This is a 10 Mbps switch port.
  \item
    This is a 100 Mb/s switch port.
  \item
    This is an Ethernet port operating at half-duplex.
  \item
    This is an Ethernet port operating at full-duplex.
  \item
    This is a port on a network interface card in a PC.
  \end{enumerate}
\item
  For what two purposes does the Ethernet protocol use physical
  addresses? (Choose two.)

  \begin{enumerate}
  \def\labelenumii{\Alph{enumii}.}
  \tightlist
  \item
    To uniquely identify devices at layer 2
  \item
    To allow communication with devices on a different network
  \item
    To differentiate a layer 2 frame from a layer 3 packet
  \item
    To establish a priority system to determine which device gets to
    transmit first
  \item
    To allow communication between different devices on the same network
  \item
    To allow detection of a remote device when its physical address is
    unknown
  \end{enumerate}
\item
  Between which systems could you use a cable that uses the pinout
  pattern shown here?

  \begin{figure}
  \centering
  \includegraphics{images/c02f030.jpg}
  \caption{}
  \end{figure}

  \begin{enumerate}
  \def\labelenumii{\Alph{enumii}.}
  \tightlist
  \item
    \protect\hypertarget{c02.xhtmlux5cux23Page_81}{}{}With a connection
    from a switch to a switch
  \item
    With a connection from a router to a router
  \item
    With a connection from a host to a host
  \item
    With a connection from a host to a switch
  \end{enumerate}
\item
  In an Ethernet network, under what two scenarios can devices transmit?
  (Choose two.)

  \begin{enumerate}
  \def\labelenumii{\Alph{enumii}.}
  \tightlist
  \item
    When they receive a special token
  \item
    When there is a carrier
  \item
    When they detect that no other devices are sending
  \item
    When the medium is idle
  \item
    When the server grants access
  \end{enumerate}
\item
  What type of cable uses the pinout shown here?

  \begin{figure}
  \centering
  \includegraphics{images/c02f031.jpg}
  \caption{}
  \end{figure}

  \begin{enumerate}
  \def\labelenumii{\Alph{enumii}.}
  \tightlist
  \item
    Fiber optic
  \item
    Crossover Gigabit Ethernet cable
  \item
    Straight-through Fast Ethernet
  \item
    Coaxial
  \end{enumerate}
\item
  When configuring a terminal emulation program, which of the following
  is an incorrect setting?

  \begin{enumerate}
  \def\labelenumii{\Alph{enumii}.}
  \tightlist
  \item
    Bit rate: 9600
  \item
    Parity: None
  \item
    Flow control: None
  \item
    Data bits: 1
  \end{enumerate}
\item
  Which part of a MAC address indicates whether the address is a locally
  or globally administered address?

  \begin{enumerate}
  \def\labelenumii{\Alph{enumii}.}
  \tightlist
  \item
    FCS
  \item
    I/G bit
  \item
    OUI
  \item
    U/L bit
  \end{enumerate}
\item
  \protect\hypertarget{c02.xhtmlux5cux23Page_82}{}{}What cable type uses
  the pinout arrangement shown below?

  \begin{figure}
  \centering
  \includegraphics{images/c02f032.jpg}
  \caption{}
  \end{figure}

  \begin{enumerate}
  \def\labelenumii{\Alph{enumii}.}
  \tightlist
  \item
    Fiber optic
  \item
    Rolled
  \item
    Straight-through
  \item
    Crossover
  \end{enumerate}
\item
  Which of the following is \emph{not} one of the actions taken in the
  operation of CSMA/CD when a collision occurs?

  \begin{enumerate}
  \def\labelenumii{\Alph{enumii}.}
  \tightlist
  \item
    A jam signal informs all devices that a collision occurred.
  \item
    The collision invokes a random backoff algorithm on the systems
    involved in the collision.
  \item
    Each device on the Ethernet segment stops transmitting for a short
    time until its backoff timer expires.
  \item
    All hosts have equal priority to transmit after the timers have
    expired.
  \end{enumerate}
\item
  Which of the following statements is \emph{false} with regard to
  Ethernet?

  \begin{enumerate}
  \def\labelenumii{\Alph{enumii}.}
  \tightlist
  \item
    There are very few collisions in full-duplex mode.
  \item
    A dedicated switch port is required for each full-duplex node.
  \item
    The host network card and the switch port must be capable of
    operating in full-duplex mode to use full-duplex.
  \item
    The default behavior of 10Base-T and 100Base-T hosts is 10 Mbps
    half-duplex if the autodetect mechanism fails.
  \end{enumerate}
\item
  In the following diagram, identify the cable types required for
  connections A and B.

  \begin{figure}
  \centering
  \includegraphics{images/c02f033.jpg}
  \caption{}
  \end{figure}

  \begin{enumerate}
  \def\labelenumii{\Alph{enumii}.}
  \tightlist
  \item
    A= crossover, B= crossover
  \item
    A= crossover, B= straight-through
  \item
    A= straight-through, B= straight-through
  \item
    A= straight-through, B= crossover
  \end{enumerate}
\item
  \protect\hypertarget{c02.xhtmlux5cux23Page_83}{}{}In the following
  image, match the cable type to the standard with which it goes.

  \begin{longtable}[]{@{}ll@{}}
  \toprule
  \endhead
  1000Base-T & IEEE 802.3u\tabularnewline
  1000Base-SX & IEEE 802.3\tabularnewline
  10Base-T & IEEE 802.3ab\tabularnewline
  100Base-TX & IEEE 802.3z\tabularnewline
  \bottomrule
  \end{longtable}
\item
  The cable used to connect to the console port on a router or switch is
  called a \_\_\_\_\_\_\_\_\_cable.

  \begin{enumerate}
  \def\labelenumii{\Alph{enumii}.}
  \tightlist
  \item
    Crossover
  \item
    Rollover
  \item
    Straight-through
  \item
    Full-duplex
  \end{enumerate}
\item
  Which of the following items does a socket comprise?

  \begin{enumerate}
  \def\labelenumii{\Alph{enumii}.}
  \tightlist
  \item
    IP address and MAC address
  \item
    IP address and port number
  \item
    Port number and MAC address
  \item
    MAC address and DLCI
  \end{enumerate}
\item
  Which of the following hexadecimal numbers converts to 28 in decimal?

  \begin{enumerate}
  \def\labelenumii{\Alph{enumii}.}
  \tightlist
  \item
    1c
  \item
    12
  \item
    15
  \item
    ab
  \end{enumerate}
\item
  What cable type is shown in the following graphic?

  \begin{figure}
  \centering
  \includegraphics{images/c02f034.jpg}
  \caption{}
  \end{figure}

  \begin{enumerate}
  \def\labelenumii{\Alph{enumii}.}
  \tightlist
  \item
    Fiber optic
  \item
    Rollover
  \item
    Coaxial
  \item
    Full-duplex
  \end{enumerate}
\end{enumerate}

\protect\hypertarget{c03.xhtml}{}{}

\section[{Chapter 3}\\
{Introduction to
TCP/IP}]{\texorpdfstring{\protect\hypertarget{c03.xhtmlux5cux23c03}{}{}\protect\hypertarget{c03.xhtmlux5cux23Page_85}{}{}{Chapter
3}\\
{Introduction to TCP/IP}}{Chapter 3 Introduction to TCP/IP}}

\begin{center}\rule{0.5\linewidth}{0.5pt}\end{center}

\subsection{THE FOLLOWING ICND1 EXAM TOPICS ARE COVERED IN THIS
CHAPTER:}

\begin{enumerate}
\tightlist
\item
  \includegraphics{images/right.png} \textbf{Network Fundamentals}

  \begin{enumerate}
  \tightlist
  \item
    \includegraphics{images/squ.png} 1.1 Compare and contrast OSI and
    TCP/IP models
  \item
    \includegraphics{images/squ.png} 1.2 Compare and contrast TCP and
    UDP protocols
  \item
    \includegraphics{images/squ.png} 1.7 Apply troubleshooting
    methodologies to resolve problems
  \item
    \includegraphics{images/squ.png} 1.7.a Perform fault isolation and
    document
  \item
    \includegraphics{images/squ.png} 1.7.b Resolve or escalate
  \item
    \includegraphics{images/squ.png} 1.7.c Verify and monitor resolution
  \item
    \includegraphics{images/squ.png} 1.9 Compare and contrast IPv4
    address types

    \begin{enumerate}
    \tightlist
    \item
      \includegraphics{images/squ.png} 1.9.a Unicast
    \item
      \includegraphics{images/squ.png} 1.9.b Broadcast
    \item
      \includegraphics{images/squ.png} 1.9.c Multicast
    \end{enumerate}
  \item
    \includegraphics{images/squ.png} 1.10 Describe the need for private
    IPv4 addressing
  \end{enumerate}
\end{enumerate}

\protect\hypertarget{c03.xhtmlux5cux23Page_86}{}{}\includegraphics{images/intro.png}
The \emph{Transmission Control Protocol/Internet Protocol (TCP/IP)}
suite was designed and implemented by the Department of Defense (DoD) to
ensure and preserve data integrity as well as maintain communications in
the event of catastrophic war. So it follows that if designed and
implemented correctly, a TCP/IP network can be a secure, dependable and
resilient one. In this chapter, I'll cover the protocols of TCP/IP, and
throughout this book, you'll learn how to create a solid TCP/IP network
with Cisco routers and switches.

We'll begin by exploring the DoD's version of TCP/IP, then compare that
version and its protocols with the OSI reference model that we discussed
earlier.

Once you understand the protocols and processes used at the various
levels of the DoD model, we'll take the next logical step by delving
into the world of IP addressing and the different classes of IP
addresses used in networks today.

\begin{center}\rule{0.5\linewidth}{0.5pt}\end{center}

\includegraphics{images/note.png} Subnetting is so vital, it will be
covered in its own chapter, Chapter 4, ``Easy Subnetting.''

\begin{center}\rule{0.5\linewidth}{0.5pt}\end{center}

Because having a good grasp of the various IPv4 address types is
critical to understanding IP addressing, subnetting, and variable length
subnet masks (VLSMs), we'll explore these key topics in detail, ending
this chapter by discussing the various types of IPv4 addresses that
you'll need to have down for the exam.

I'm not going to cover Internet Protocol version 6 in this chapter
because we'll get into that later, in Chapter 14, ``Internet Protocol
Version 6 (IPv6).'' And just so you know, you'll simply see Internet
Protocol version 4 written as just IP, rarely as IPv4.

\begin{center}\rule{0.5\linewidth}{0.5pt}\end{center}

\includegraphics{images/note.png} To find up-to-the-minute updates for
this chapter, please see
\href{http://www.lammle.com/ccna}{www.lammle.com/ccna} or the book's web
page via \href{http://www.sybex.com/go/ccna}{www.sybex.com/go/ccna}.

\begin{center}\rule{0.5\linewidth}{0.5pt}\end{center}

\subsection[Introducing
TCP/IP]{\texorpdfstring{\protect\hypertarget{c03.xhtmlux5cux23c03-sec-1}{}{}Introducing
TCP/IP}{Introducing TCP/IP}}

TCP/IP is at the very core of all things networking, so I really want to
ensure that you have a comprehensive and functional command of it. I'll
start by giving you the whole TCP/IP backstory, including its inception,
and then move on to describe the important technical goals as defined by
its original architects. And of course I'll include how TCP/IP compares
to the theoretical OSI model.

\subsubsection[A Brief History of
TCP/IP]{\texorpdfstring{\protect\hypertarget{c03.xhtmlux5cux23c03-sec-2}{}{}\protect\hypertarget{c03.xhtmlux5cux23Page_87}{}{}A
Brief History of TCP/IP}{A Brief History of TCP/IP}}

TCP first came on the scene way back in 1973, and in 1978, it was
divided into two distinct protocols: TCP and IP. Later, in 1983, TCP/IP
replaced the Network Control Protocol (NCP) and was authorized as the
official means of data transport for anything connecting to ARPAnet, the
Internet's ancestor. The DoD's Advanced Research Projects Agency (ARPA)
created this ancient network way back in 1957 in a cold war reaction to
the Soviet's launching of \emph{Sputnik}. Also in 1983, ARPA was
redubbed DARPA and divided into ARPAnet and MILNET until both were
finally dissolved in~1990.

It may be counterintuitive, but most of the development work on TCP/IP
happened at UC Berkeley in Northern California, where a group of
scientists were simultaneously working on the Berkeley version of UNIX,
which soon became known as the Berkeley Software Distribution (BSD)
series of UNIX versions. Of course, because TCP/IP worked so well, it
was packaged into subsequent releases of BSD Unix and offered to other
universities and institutions if they bought the distribution tape. So
basically, BSD Unix bundled with TCP/IP began as shareware in the world
of academia. As a result, it became the foundation for the tremendous
success and unprecedented growth of today's Internet as well as smaller,
private and corporate intranets.

As usual, what started as a small group of TCP/IP aficionados evolved,
and as it did, the US government created a program to test any new
published standards and make sure they passed certain criteria. This was
to protect TCP/IP's integrity and to ensure that no developer changed
anything too dramatically or added any proprietary features. It's this
very quality---this open-systems approach to the TCP/IP family of
protocols---that sealed its popularity because this quality guarantees a
solid connection between myriad hardware and software platforms with no
strings attached.

\subsection[TCP/IP and the DoD
Model]{\texorpdfstring{\protect\hypertarget{c03.xhtmlux5cux23c03-sec-3}{}{}TCP/IP
and the DoD Model}{TCP/IP and the DoD Model}}

The DoD model is basically a condensed version of the OSI model that
comprises four instead of seven layers:

\begin{enumerate}
\tightlist
\item
  Process/Application layer
\item
  Host-to-Host layer or Transport layer
\item
  Internet layer
\item
  Network Access layer or Link layer
\end{enumerate}

\protect\hyperlink{c03.xhtmlux5cux23figure03-1}{Figure 3.1} offers a
comparison of the DoD model and the OSI reference model. As you can see,
the two are similar in concept, but each has a different number of
layers with different names. Cisco may at times use different names for
the same layer, such as both ``Host-to-Host'' and Transport'' at the
layer above the Internet layer, as well as ``Network Access'' and
``Link'' used to describe the bottom layer.

\protect\hypertarget{c03.xhtmlux5cux23Page_88}{}{}

\begin{figure}
\centering
\includegraphics{images/c03f001.jpg}
\caption{{\protect\hyperlink{c03.xhtmlux5cux23figureanchor03-1}{\textbf{FIGURE
3.1}} The DoD and OSI models}}
\end{figure}

\begin{center}\rule{0.5\linewidth}{0.5pt}\end{center}

\includegraphics{images/note.png} When the different protocols in the IP
stack are discussed, the layers of the OSI and DoD models are
interchangeable. In other words, be prepared for the exam objectives to
call the Host-to-Host layer the Transport layer!

\begin{center}\rule{0.5\linewidth}{0.5pt}\end{center}

A vast array of protocols join forces at the DoD model's
\emph{Process/Application layer}. These processes integrate the various
activities and duties spanning the focus of the OSI's corresponding top
three layers (Application, Presentation, and Session). We'll focus on a
few of the most important applications found in the CCNA objectives. In
short, the Process/Application layer defines protocols for node-to-node
application communication and controls user-interface specifications.

The \emph{Host-to-Host layer or Transport layer} parallels the functions
of the OSI's Transport layer, defining protocols for setting up the
level of transmission service for applications. It tackles issues like
creating reliable end-to-end communication and ensuring the error-free
delivery of data. It handles packet sequencing and maintains data
integrity.

The \emph{Internet layer} corresponds to the OSI's Network layer,
designating the protocols relating to the logical transmission of
packets over the entire network. It takes care of the addressing of
hosts by giving them an IP (Internet Protocol) address and handles the
routing of packets among multiple networks.

At the bottom of the DoD model, the \emph{Network Access layer or Link
layer} implements the data exchange between the host and the network.
The equivalent of the Data Link and Physical layers of the OSI model,
the Network Access layer oversees hardware addressing and defines
protocols for the physical transmission of data. The reason TCP/IP
became so popular is because there were no set physical layer
specifications, so it could run on any existing or future physical
network!

The DoD and OSI models are alike in design and concept and have similar
functions in similar layers.
\protect\hyperlink{c03.xhtmlux5cux23figure03-2}{Figure 3.2} shows the
TCP/IP protocol suite and how its protocols relate to the DoD model
layers.

\protect\hypertarget{c03.xhtmlux5cux23Page_89}{}{}

\begin{figure}
\centering
\includegraphics{images/c03f002.jpg}
\caption{{\protect\hyperlink{c03.xhtmlux5cux23figureanchor03-2}{\textbf{FIGURE
3.2}} The TCP/IP protocol suite}}
\end{figure}

In the following sections, we will look at the different protocols in
more detail, beginning with those found at the Process/Application
layer.

\subsubsection[The Process/Application Layer
Protocols]{\texorpdfstring{\protect\hypertarget{c03.xhtmlux5cux23c03-sec-4}{}{}The
Process/Application Layer
Protocols}{The Process/Application Layer Protocols}}

Coming up, I'll describe the different applications and services
typically used in IP networks, and although there are many more
protocols defined here, we'll focus in on the protocols most relevant to
the CCNA objectives. Here's a list of the protocols and applications
we'll cover in this section:

\begin{enumerate}
\tightlist
\item
  Telnet
\item
  SSH
\item
  FTP
\item
  TFTP
\item
  SNMP
\item
  HTTP
\item
  HTTPS
\item
  NTP
\item
  DNS
\item
  DHCP/BootP
\item
  APIPA
\end{enumerate}

\paragraph{Telnet}

\emph{Telnet} was one of the first Internet standards, developed in
1969, and is the chameleon of protocols---its specialty is terminal
emulation. It allows a user on a remote client machine, called the
Telnet client, to access the resources of another machine, the Telnet
server, in order to access a command-line interface. Telnet achieves
this by pulling a fast one on the Telnet
\protect\hypertarget{c03.xhtmlux5cux23Page_90}{}{}server and making the
client machine appear as though it were a terminal directly attached to
the local network. This projection is actually a software image---a
virtual terminal that can interact with the chosen remote host. A
drawback is that there are no encryption techniques available within the
Telnet protocol, so everything must be sent in clear text, including
passwords! \protect\hyperlink{c03.xhtmlux5cux23figure03-3}{Figure 3.3}
shows an example of a Telnet client trying to connect to a Telnet
server.

\begin{figure}
\centering
\includegraphics{images/c03f003.jpg}
\caption{{\protect\hyperlink{c03.xhtmlux5cux23figureanchor03-3}{\textbf{FIGURE
3.3}} Telnet}}
\end{figure}

These emulated terminals are of the text-mode type and can execute
defined procedures such as displaying menus that give users the
opportunity to choose options and access the applications on the duped
server. Users begin a Telnet session by running the Telnet client
software and then logging into the Telnet server. Telnet uses an 8-bit,
byte-oriented data connection over TCP, which makes it very thorough.
It's still in use today because it is so simple and easy to use, with
very low overhead, but again, with everything sent in clear text, it's
not recommended in production.

\paragraph{Secure Shell (SSH)}

\emph{Secure Shell (SSH)} protocol sets up a secure session that's
similar to Telnet over a standard TCP/IP connection and is employed for
doing things like logging into systems, running programs on remote
systems, and moving files from one system to another. And it does all of
this while maintaining an encrypted connection.
\protect\hyperlink{c03.xhtmlux5cux23figure03-4}{Figure 3.4} shows a SSH
client trying to connect to a SSH server. The client must send the data
encrypted!

You can think of it as the new-generation protocol that's now used in
place of the antiquated and very unused \texttt{rsh} and
\texttt{rlogin}---even Telnet.

\paragraph{File Transfer Protocol (FTP)}

\emph{File Transfer Protocol (FTP)} actually lets us transfer files, and
it can accomplish this between any two machines using it. But FTP isn't
just a protocol; it's also a program. Operating as a protocol, FTP is
used by applications. As a program, it's employed by users to perform
file tasks by hand. FTP also allows for access to both directories and
files and can accomplish certain types of directory operations, such as
relocating into different ones
(\protect\hyperlink{c03.xhtmlux5cux23figure03-5}{Figure 3.5}).

\protect\hypertarget{c03.xhtmlux5cux23Page_91}{}{}

\begin{figure}
\centering
\includegraphics{images/c03f004.jpg}
\caption{{\protect\hyperlink{c03.xhtmlux5cux23figureanchor03-4}{\textbf{FIGURE
3.4}} Secure Shell}}
\end{figure}

\begin{figure}
\centering
\includegraphics{images/c03f005.jpg}
\caption{{\protect\hyperlink{c03.xhtmlux5cux23figureanchor03-5}{\textbf{FIGURE
3.5}} FTP}}
\end{figure}

But accessing a host through FTP is only the first step. Users must then
be subjected to an authentication login that's usually secured with
passwords and usernames implemented by system administrators to restrict
access. You can get around this somewhat by adopting the username
\emph{anonymous}, but you'll be limited in what you'll be able to
access.

Even when employed by users manually as a program, FTP's functions are
limited to listing and manipulating directories, typing file contents,
and copying files between hosts. It can't execute remote files as
programs.

\paragraph{Trivial File Transfer Protocol (TFTP)}

\emph{Trivial File Transfer Protocol (TFTP)} is the stripped-down, stock
version of FTP, but it's the protocol of choice if you know exactly what
you want and where to find it because it's fast and so easy to use!

\protect\hypertarget{c03.xhtmlux5cux23Page_92}{}{}But TFTP doesn't offer
the abundance of functions that FTP does because it has no
directory-browsing abilities, meaning that it can only send and receive
files (\protect\hyperlink{c03.xhtmlux5cux23figure03-6}{Figure 3.6}).
Still, it's heavily used for managing file systems on Cisco devices, as
I'll show you in Chapter 7, ``Managing a Cisco Internetwork.''

\begin{figure}
\centering
\includegraphics{images/c03f006.jpg}
\caption{{\protect\hyperlink{c03.xhtmlux5cux23figureanchor03-6}{\textbf{FIGURE
3.6}} TFTP}}
\end{figure}

This compact little protocol also skimps in the data department, sending
much smaller blocks of data than FTP. Also, there's no authentication as
with FTP, so it's even more insecure, and few sites support it because
of the inherent security risks.

\begin{center}\rule{0.5\linewidth}{0.5pt}\end{center}

\subsubsection[\hfill\break
When Should You Use
FTP?]{\texorpdfstring{\protect\includegraphics{images/earth.png}\\
When Should You Use FTP?}{ When Should You Use FTP?}}

Let's say everyone at your San Francisco office needs a 50 GB file
emailed to them right away. What do you do? Many email servers would
reject that email due to size limits (a lot of ISPs don't allow files
larger than 5 MB or 10 MB to be emailed), and even if there are no size
limits on the server, it would still take a while to send this huge
file. FTP to the rescue!

If you need to give someone a large file or you need to get a large file
from someone, FTP is a nice choice. To use FTP, you would need to set up
an FTP server on the Internet so that the files can be shared.

Besides resolving size issues, FTP is faster than email. In addition,
because it uses TCP and is connection-oriented, if the session dies, FTP
can sometimes start up where it left off. Try that with your email
client!

\begin{center}\rule{0.5\linewidth}{0.5pt}\end{center}

\paragraph[Simple Network Management Protocol
(SNMP)]{\texorpdfstring{\protect\hypertarget{c03.xhtmlux5cux23Page_93}{}{}Simple
Network Management Protocol
(SNMP)}{Simple Network Management Protocol (SNMP)}}

\emph{Simple Network Management Protocol (SNMP)} collects and
manipulates valuable network information, as you can see in
\protect\hyperlink{c03.xhtmlux5cux23figure03-7}{Figure 3.7}. It gathers
data by polling the devices on the network from a network management
station (NMS) at fixed or random intervals, requiring them to disclose
certain information, or even asking for certain information from the
device. In addition, network devices can inform the NMS station about
problems as they occur so the network administrator is alerted.

\begin{figure}
\centering
\includegraphics{images/c03f007.jpg}
\caption{{\protect\hyperlink{c03.xhtmlux5cux23figureanchor03-7}{\textbf{FIGURE
3.7}} SNMP}}
\end{figure}

When all is well, SNMP receives something called a \emph{baseline}---a
report delimiting the operational traits of a healthy network. This
protocol can also stand as a watchdog over the network, quickly
notifying managers of any sudden turn of events. These network watchdogs
are called \emph{agents}, and when aberrations occur, agents send an
alert called a \emph{trap} to the management station.

\begin{center}\rule{0.5\linewidth}{0.5pt}\end{center}

\subsubsection{SNMP Versions 1, 2, and 3}

SNMP versions 1 and 2 are pretty much obsolete. This doesn't mean you
won't see them in a network now and then, but you'll only come across v1
rarely, if ever. SNMPv2 provided improvements, especially in
performance. But one of the best additions was called GETBULK, which
allowed a host to retrieve a large amount of data at once. Even so, v2
never really caught on in the networking world and SNMPv3 is now the
standard. Unlike v1, which used only UDP, v3 uses both TCP and UDP and
added even more security, message integrity, authentication, and
encryption.

\begin{center}\rule{0.5\linewidth}{0.5pt}\end{center}

\paragraph{Hypertext Transfer Protocol (HTTP)}

All those snappy websites comprising a mélange of graphics, text, links,
ads, and so on rely on the \emph{Hypertext Transfer Protocol (HTTP)} to
make it all possible
(\protect\hyperlink{c03.xhtmlux5cux23figure03-8}{Figure 3.8}). It's used
to manage communications between web browsers and web servers and opens
the right resource when you click a link, wherever that resource may
actually reside.

\protect\hypertarget{c03.xhtmlux5cux23Page_94}{}{}

\begin{figure}
\centering
\includegraphics{images/c03f008.jpg}
\caption{{\protect\hyperlink{c03.xhtmlux5cux23figureanchor03-8}{\textbf{FIGURE
3.8}} HTTP}}
\end{figure}

In order for a browser to display a web page, it must find the exact
server that has the right web page, plus the exact details that identify
the information requested. This information must be then be sent back to
the browser. Nowadays, it's highly doubtful that a web server would have
only one page to display!

Your browser can understand what you need when you enter a Uniform
Resource Locator (URL), which we usually refer to as a web address, such
as, for example, \url{http://www.lammle.com/forum} and
\url{http://www.lammle.com/blog}.

So basically, each URL defines the protocol used to transfer data, the
name of the server, and the particular web page on that server.

\paragraph{Hypertext Transfer Protocol Secure (HTTPS)}

\emph{Hypertext Transfer Protocol Secure (HTTPS)} is also known as
Secure Hypertext Transfer Protocol. It uses Secure Sockets Layer (SSL).
Sometimes you'll see it referred to as SHTTP or S-HTTP, which were
slightly different protocols, but since Microsoft supported HTTPS, it
became the de facto standard for securing web communication. But no
matter---as indicated, it's a secure version of HTTP that arms you with
a whole bunch of security tools for keeping transactions between a web
browser and a server secure.

It's what your browser needs to fill out forms, sign in, authenticate,
and encrypt an HTTP message when you do things online like make a
reservation, access your bank, or buy something.

\paragraph{Network Time Protocol (NTP)}

Kudos to Professor David Mills of the University of Delaware for coming
up with this handy protocol that's used to synchronize the clocks on our
computers to one standard time source (typically, an atomic clock).
\emph{Network Time Protocol (NTP)} works by synchronizing devices to
ensure that all computers on a given network agree on the time
(\protect\hyperlink{c03.xhtmlux5cux23figure03-9}{Figure 3.9}).

This may sound pretty simple, but it's very important because so many of
the transactions done today are time and date stamped. Think about
databases---a server can get messed up pretty badly and even crash if
it's out of sync with the machines connected to it by even mere seconds!
You can't have a transaction entered by a machine at, say, 1:50 a.m.
when the server records that transaction as having occurred at 1:45 a.m.
So basically, NTP works to prevent a ``back to the future \emph{sans}
DeLorean'' scenario from bringing down the network---very important
indeed!

\protect\hypertarget{c03.xhtmlux5cux23Page_95}{}{}

\begin{figure}
\centering
\includegraphics{images/c03f009.jpg}
\caption{{\protect\hyperlink{c03.xhtmlux5cux23figureanchor03-9}{\textbf{FIGURE
3.9}} NTP}}
\end{figure}

I'll tell you a lot more about NTP in Chapter 7, including how to
configure this protocol in a Cisco environment.

\paragraph{Domain Name Service (DNS)}

\emph{Domain Name Service (DNS)} resolves hostnames---specifically,
Internet names, such as \href{http://www.lammle.com}{www.lammle.com}.
But you don't have to actually use DNS. You just type in the IP address
of any device you want to communicate with and find the IP address of a
URL by using the Ping program. For example,
\texttt{\textgreater{}ping\ www.cisco.com} will return the IP address
resolved by DNS.

An IP address identifies hosts on a network and the Internet as well,
but DNS was designed to make our lives easier. Think about this: What
would happen if you wanted to move your web page to a different service
provider? The IP address would change and no one would know what the new
one is. DNS allows you to use a domain name to specify an IP address.
You can change the IP address as often as you want and no one will know
the difference.

To resolve a DNS address from a host, you'd typically type in the URL
from your favorite browser, which would hand the data to the Application
layer interface to be transmitted on the network. The application would
look up the DNS address and send a UDP request to your DNS server to
resolve the name
(\protect\hyperlink{c03.xhtmlux5cux23figure03-10}{Figure 3.10}).

If your first DNS server doesn't know the answer to the query, then the
DNS server forwards a TCP request to its root DNS server. Once the query
is resolved, the answer is transmitted back to the originating host,
which means the host can now request the information from the correct
web server.

DNS is used to resolve a \emph{fully qualified domain name (FQDN)}---for
example, \href{http://www.lammle.com}{www.lammle.com} or
\texttt{todd.lammle.com}. An FQDN is a hierarchy that can logically
locate a system based on its domain identifier.

If you want to resolve the name \emph{todd}, you either must type in the
FQDN of \texttt{todd.lammle.com} or have a device such as a PC or router
add the suffix for you. For example, on a Cisco router, you can use the
command \texttt{ip\ domain-name\ lammle.com} to append each request with
the \texttt{lammle.com} domain. If you don't do that, you'll have to
type in the FQDN to get DNS to resolve the name.

\protect\hypertarget{c03.xhtmlux5cux23Page_96}{}{}

\begin{figure}
\centering
\includegraphics{images/c03f010.jpg}
\caption{{\protect\hyperlink{c03.xhtmlux5cux23figureanchor03-10}{\textbf{FIGURE
3.10}} DNS}}
\end{figure}

\begin{center}\rule{0.5\linewidth}{0.5pt}\end{center}

\includegraphics{images/tip.png} An important thing to remember about
DNS is that if you can ping a device with an IP address but cannot use
its FQDN, then you might have some type of DNS configuration failure.

\begin{center}\rule{0.5\linewidth}{0.5pt}\end{center}

\paragraph{Dynamic Host Configuration Protocol (DHCP)/Bootstrap Protocol
(BootP)}

\emph{Dynamic Host Configuration Protocol (DHCP)} assigns IP addresses
to hosts. It allows for easier administration and works well in small to
very large network environments. Many types of hardware can be used as a
DHCP server, including a Cisco router.

DHCP differs from BootP in that BootP assigns an IP address to a host
but the host's hardware address must be entered manually in a BootP
table. You can think of DHCP as a dynamic BootP. But remember that BootP
is also used to send an operating system that a host can boot from. DHCP
can't do that.

But there's still a lot of information a DHCP server can provide to a
host when the host is requesting an IP address from the DHCP server.
Here's a list of the most common types of information a DHCP server can
provide:

\begin{enumerate}
\tightlist
\item
  IP address
\item
  Subnet mask
\item
  Domain name
\item
  Default gateway (routers)
\item
  DNS server address
\item
  WINS server address
\end{enumerate}

\protect\hypertarget{c03.xhtmlux5cux23Page_97}{}{}A client that sends
out a DHCP Discover message in order to receive an IP address sends out
a broadcast at both layer 2 and layer 3.

\begin{enumerate}
\tightlist
\item
  The layer 2 broadcast is all \emph{F}s in hex, which looks like this:
  ff:ff:ff:ff:ff:ff.
\item
  The layer 3 broadcast is 255.255.255.255, which means all networks and
  all hosts.
\end{enumerate}

DHCP is connectionless, which means it uses User Datagram Protocol (UDP)
at the Transport layer, also known as the Host-to-Host layer, which
we'll talk about later.

Seeing is believing, so here's an example of output from my analyzer
showing the layer 2 and layer 3 broadcasts:

\begin{verbatim}
Ethernet II, Src: 0.0.0.0 (00:0b:db:99:d3:5e),Dst: Broadcast(ff:ff:ff:ff:ff:ff)
Internet Protocol, Src: 0.0.0.0 (0.0.0.0),Dst: 255.255.255.255(255.255.255.255)
\end{verbatim}

The Data Link and Network layers are both sending out ``all hands''
broadcasts saying, ``Help---I don't know my IP address!''

\begin{center}\rule{0.5\linewidth}{0.5pt}\end{center}

\includegraphics{images/note.png} DHCP will be discussed in more detail,
including configuration on a Cisco router and switch, in Chapter 7,
``Managing a Cisco Internetwork,'' and Chapter 9, ``IP Routing.''

\begin{center}\rule{0.5\linewidth}{0.5pt}\end{center}

\protect\hyperlink{c03.xhtmlux5cux23figure03-11}{Figure 3.11} shows the
process of a client/server relationship using a DHCP connection.

\begin{figure}
\centering
\includegraphics{images/c03f011.jpg}
\caption{{\protect\hyperlink{c03.xhtmlux5cux23figureanchor03-11}{\textbf{FIGURE
3.11}} DHCP client four-step process}}
\end{figure}

This is the four-step process a client takes to receive an IP address
from a DHCP server:

\begin{enumerate}
\tightlist
\item
  \protect\hypertarget{c03.xhtmlux5cux23Page_98}{}{} The DHCP client
  broadcasts a DHCP Discover message looking for a DHCP server (Port
  67).
\item
  The DHCP server that received the DHCP Discover message sends a layer
  2 unicast DHCP Offer message back to the host.
\item
  The client then broadcasts to the server a DHCP Request message asking
  for the offered IP address and possibly other information.
\item
  The server finalizes the exchange with a unicast DHCP Acknowledgment
  message.
\end{enumerate}

\subparagraph{DHCP Conflicts}

A DHCP address conflict occurs when two hosts use the same IP address.
This sounds bad, and it is! We'll never even have to discuss this
problem once we get to the chapter on IPv6!

During IP address assignment, a DHCP server checks for conflicts using
the Ping program to test the availability of the address before it's
assigned from the pool. If no host replies, then the DHCP server assumes
that the IP address is not already allocated. This helps the server know
that it's providing a good address, but what about the host? To provide
extra protection against that terrible IP conflict issue, the host can
broadcast for its own address!

A host uses something called a gratuitous ARP to help avoid a possible
duplicate address. The DHCP client sends an ARP broadcast out on the
local LAN or VLAN using its newly assigned address to solve conflicts
before they occur.

So, if an IP address conflict is detected, the address is removed from
the DHCP pool (scope), and it's really important to remember that the
address will not be assigned to a host until the administrator resolves
the conflict by hand!

\begin{center}\rule{0.5\linewidth}{0.5pt}\end{center}

\includegraphics{images/note.png} Please see Chapter 9, ``IP Routing,''
to check out a DHCP configuration on a Cisco router and also to find out
what happens when a DHCP client is on one side of a router but the DHCP
server is on the other side on a different network!

\begin{center}\rule{0.5\linewidth}{0.5pt}\end{center}

\paragraph{Automatic Private IP Addressing (APIPA)}

Okay, so what happens if you have a few hosts connected together with a
switch or hub and you don't have a DHCP server? You can add IP
information by hand, known as \emph{static IP addressing}, but later
Windows operating systems provide a feature called Automatic Private IP
Addressing (APIPA). With APIPA, clients can automatically self-configure
an IP address and subnet mask---basic IP information that hosts use to
communicate---when a DHCP server isn't available. The IP address range
for APIPA is 169.254.0.1 through 169.254.255.254. The client also
configures itself with a default Class B subnet mask of 255.255.0.0.

But when you're in your corporate network working and you have a DHCP
server running, and your host shows that it's using this IP address
range, it means that either your
\protect\hypertarget{c03.xhtmlux5cux23Page_99}{}{}DHCP client on the
host is not working or the server is down or can't be reached due to
some network issue. Believe me---I don't know anyone who's seen a host
in this address range and has been happy about it!

Now, let's take a look at the Transport layer, or what the DoD calls the
Host-to-Host layer.

\subsubsection[The Host-to-Host or Transport Layer
Protocols]{\texorpdfstring{\protect\hypertarget{c03.xhtmlux5cux23c03-sec-5}{}{}The
Host-to-Host or Transport Layer
Protocols}{The Host-to-Host or Transport Layer Protocols}}

The main purpose of the Host-to-Host layer is to shield the upper-layer
applications from the complexities of the network. This layer says to
the upper layer, ``Just give me your data stream, with any instructions,
and I'll begin the process of getting your information ready to send.''

Coming up, I'll introduce you to the two protocols at this layer:

\begin{enumerate}
\tightlist
\item
  Transmission Control Protocol (TCP)
\item
  User Datagram Protocol (UDP)
\end{enumerate}

In addition, we'll look at some of the key host-to-host protocol
concepts, as well as the port numbers.

\begin{center}\rule{0.5\linewidth}{0.5pt}\end{center}

\includegraphics{images/note.png} Remember, this is still considered
layer 4, and Cisco really likes the way layer 4 can use acknowledgments,
sequencing, and flow control.

\begin{center}\rule{0.5\linewidth}{0.5pt}\end{center}

\paragraph{Transmission Control Protocol (TCP)}

\emph{Transmission Control Protocol (TCP)} takes large blocks of
information from an application and breaks them into segments. It
numbers and sequences each segment so that the destination's TCP stack
can put the segments back into the order the application intended. After
these segments are sent on the transmitting host, TCP waits for an
acknowledgment of the receiving end's TCP virtual circuit session,
retransmitting any segments that aren't acknowledged.

Before a transmitting host starts to send segments down the model, the
sender's TCP stack contacts the destination's TCP stack to establish a
connection. This creates a \emph{virtual circuit}, and this type of
communication is known as \emph{connection-oriented}. During this
initial handshake, the two TCP layers also agree on the amount of
information that's going to be sent before the recipient's TCP sends
back an acknowledgment. With everything agreed upon in advance, the path
is paved for reliable communication to take place.

TCP is a full-duplex, connection-oriented, reliable, and accurate
protocol, but establishing all these terms and conditions, in addition
to error checking, is no small task. TCP is very complicated, and so not
surprisingly, it's costly in terms of network overhead. And since
today's networks are much more reliable than those of yore, this added
reliability is often unnecessary. Most programmers use TCP because it
removes a lot of programming work, but for real-time video and VoIP,
\emph{User Datagram Protocol (UDP)} is often better because using it
results in less overhead.

\subparagraph[TCP Segment
Format]{\texorpdfstring{\protect\hypertarget{c03.xhtmlux5cux23Page_100}{}{}TCP
Segment Format}{TCP Segment Format}}

Since the upper layers just send a data stream to the protocols in the
Transport layers, I'll use
\protect\hyperlink{c03.xhtmlux5cux23figure03-12}{Figure 3.12} to
demonstrate how TCP segments a data stream and prepares it for the
Internet layer. When the Internet layer receives the data stream, it
routes the segments as packets through an internetwork. The segments are
handed to the receiving host's Host-to-Host layer protocol, which
rebuilds the data stream for the upper-layer applications or protocols.

\begin{figure}
\centering
\includegraphics{images/c03f012.jpg}
\caption{{\protect\hyperlink{c03.xhtmlux5cux23figureanchor03-12}{\textbf{FIGURE
3.12}} TCP segment format}}
\end{figure}

\protect\hyperlink{c03.xhtmlux5cux23figure03-12}{Figure 3.12} shows the
TCP segment format and shows the different fields within the TCP header.
This isn't important to memorize for the Cisco exam objectives, but you
need to understand it well because it's really good foundational
information.

The TCP header is 20 bytes long, or up to 24 bytes with options. You
need to understand what each field in the TCP segment is in order to
build a strong educational foundation:

\textbf{Source port} This is the port number of the application on the
host sending the data, which I'll talk about more thoroughly a little
later in this chapter.

\textbf{Destination port} This is the port number of the application
requested on the destination host.

\textbf{Sequence number} A number used by TCP that puts the data back in
the correct order or retransmits missing or damaged data during a
process called sequencing.

\textbf{Acknowledgment number} The value is the TCP octet that is
expected next.

\textbf{Header length} The number of 32-bit words in the TCP header,
which indicates where the data begins. The TCP header (even one
including options) is an integral number of 32 bits in length.

\textbf{Reserved} Always set to zero.

\textbf{Code bits/flags} Controls functions used to set up and terminate
a session.

\textbf{Window} The window size the sender is willing to accept, in
octets.

\textbf{Checksum} The cyclic redundancy check (CRC), used because TCP
doesn't trust the lower layers and checks everything. The CRC checks the
header and data fields.

\textbf{\protect\hypertarget{c03.xhtmlux5cux23Page_101}{}{}Urgent} A
valid field only if the Urgent pointer in the code bits is set. If so,
this value indicates the offset from the current sequence number, in
octets, where the segment of non-urgent data begins.

\textbf{Options} May be 0, meaning that no options have to be present,
or a multiple of 32 bits. However, if any options are used that do not
cause the option field to total a multiple of 32 bits, padding of 0s
must be used to make sure the data begins on a 32-bit boundary. These
boundaries are known as words.

\textbf{Data} Handed down to the TCP protocol at the Transport layer,
which includes the upper-layer headers.

Let's take a look at a TCP segment copied from a network analyzer:

\begin{verbatim}
TCP - Transport Control Protocol
 Source Port:      5973
 Destination Port: 23
 Sequence Number:  1456389907
 Ack Number:       1242056456
 Offset:           5
 Reserved:         %000000
 Code:             %011000
      Ack is valid
      Push Request
 Window:           61320
 Checksum:         0x61a6
 Urgent Pointer:   0
 No TCP Options
 TCP Data Area:
 vL.5.+.5.+.5.+.5  76 4c 19 35 11 2b 19 35 11 2b 19 35 11
  2b 19 35 +. 11 2b 19
Frame Check Sequence: 0x0d00000f
\end{verbatim}

Did you notice that everything I talked about earlier is in the segment?
As you can see from the number of fields in the header, TCP creates a
lot of overhead. Again, this is why application developers may opt for
efficiency over reliability to save overhead and go with UDP instead.
It's also defined at the Transport layer as an alternative to TCP.

\paragraph{User Datagram Protocol (UDP)}

\emph{User Datagram Protocol (UDP)} is basically the scaled-down economy
model of TCP, which is why UDP is sometimes referred to as a thin
protocol. Like a thin person on a park bench, a thin protocol doesn't
take up a lot of room---or in this case, require much bandwidth on a
network.

UDP doesn't offer all the bells and whistles of TCP either, but it does
do a fabulous job of transporting information that doesn't require
reliable delivery, using far less network resources. (UDP is covered
thoroughly in Request for Comments 768.)

\protect\hypertarget{c03.xhtmlux5cux23Page_102}{}{}So clearly, there are
times that it's wise for developers to opt for UDP rather than TCP, one
of them being when reliability is already taken care of at the
Process/Application layer. Network File System (NFS) handles its own
reliability issues, making the use of TCP both impractical and
redundant. But ultimately, it's up to the application developer to opt
for using UDP or TCP, not the user who wants to transfer data faster!

UDP does \emph{not} sequence the segments and does not care about the
order in which the segments arrive at the destination. UDP just sends
the segments off and forgets about them. It doesn't follow through,
check up on them, or even allow for an acknowledgment of safe
arrival---complete abandonment. Because of this, it's referred to as an
unreliable protocol. This does not mean that UDP is ineffective, only
that it doesn't deal with reliability issues at all.

Furthermore, UDP doesn't create a virtual circuit, nor does it contact
the destination before delivering information to it. Because of this,
it's also considered a \emph{connectionless} protocol. Since UDP assumes
that the application will use its own reliability method, it doesn't use
any itself. This presents an application developer with a choice when
running the Internet Protocol stack: TCP for reliability or UDP for
faster transfers.

It's important to know how this process works because if the segments
arrive out of order, which is commonplace in IP networks, they'll simply
be passed up to the next layer in whatever order they were received.
This can result in some seriously garbled data! On the other hand, TCP
sequences the segments so they get put back together in exactly the
right order, which is something UDP just can't do.

\subparagraph{UDP Segment Format}

\protect\hyperlink{c03.xhtmlux5cux23figure03-13}{Figure 3.13} clearly
illustrates UDP's markedly lean overhead as compared to TCP's hungry
requirements. Look at the figure carefully---can you see that UDP
doesn't use windowing or provide for acknowledgments in the UDP header?

\begin{figure}
\centering
\includegraphics{images/c03f013.jpg}
\caption{{\protect\hyperlink{c03.xhtmlux5cux23figureanchor03-13}{\textbf{FIGURE
3.13}} UDP segment}}
\end{figure}

It's important for you to understand what each field in the UDP segment
is:

\textbf{Source port} Port number of the application on the host sending
the data

\textbf{Destination port} Port number of the application requested on
the destination host

\textbf{Length} Length of UDP header and UDP data

\textbf{Checksum} Checksum of both the UDP header and UDP data fields

\textbf{Data} Upper-layer data

UDP, like TCP, doesn't trust the lower layers and runs its own CRC.
Remember that the Frame Check Sequence (FCS) is the field that houses
the CRC, which is why you can see the FCS information.

\protect\hypertarget{c03.xhtmlux5cux23Page_103}{}{}The following shows a
UDP segment caught on a network analyzer:

\begin{verbatim}
UDP - User Datagram Protocol
 Source Port:      1085
 Destination Port: 5136
 Length:           41
 Checksum:         0x7a3c
 UDP Data Area:
 ..Z......00 01 5a 96 00 01 00 00 00 00 00 11 0000 00
...C..2._C._C  2e 03 00 43 02 1e 32 0a 00 0a 00 80 43 00 80
Frame Check Sequence: 0x00000000
\end{verbatim}

Notice that low overhead! Try to find the sequence number, ack number,
and window size in the UDP segment. You can't because they just aren't
there!

\paragraph{Key Concepts of Host-to-Host Protocols}

Since you've now seen both a connection-oriented (TCP) and
connectionless (UDP) protocol in action, it's a good time to summarize
the two here. \protect\hyperlink{c03.xhtmlux5cux23table03-1}{Table 3.1}
highlights some of the key concepts about these two protocols for you to
memorize.

{\protect\hyperlink{c03.xhtmlux5cux23tableanchor03-1}{\textbf{TABLE
3.1}} Key features of TCP and UDP}

\begin{longtable}[]{@{}ll@{}}
\toprule
\textgreater TCP & \textgreater UDP\tabularnewline
\midrule
\endhead
Sequenced & Unsequenced\tabularnewline
Reliable & Unreliable\tabularnewline
Connection-oriented & Connectionless\tabularnewline
Virtual circuit & Low overhead\tabularnewline
Acknowledgments & No acknowledgment\tabularnewline
Windowing flow control & No windowing or flow control of any
type\tabularnewline
\bottomrule
\end{longtable}

And if all this isn't quite clear yet, a telephone analogy will really
help you understand how TCP works. Most of us know that before you speak
to someone on a phone, you must first establish a connection with that
other person no matter where they are. This is akin to establishing a
virtual circuit with the TCP protocol. If you were giving someone
important information during your conversation, you might say things
like, ``You know? or ``Did you get that?'' Saying things like this is a
lot like a TCP acknowledgment---it's designed to get you verification.
From time to time, especially on mobile phones, people ask, ``Are you
still\protect\hypertarget{c03.xhtmlux5cux23Page_104}{}{} there?'' People
end their conversations with a ``Goodbye'' of some kind, putting closure
on the phone call, which you can think of as tearing down the virtual
circuit that was created for your communication session. TCP performs
these types of functions.

Conversely, using UDP is more like sending a postcard. To do that, you
don't need to contact the other party first, you simply write your
message, address the postcard, and send it off. This is analogous to
UDP's connectionless orientation. Since the message on the postcard is
probably not a matter of life or death, you don't need an acknowledgment
of its receipt. Similarly, UDP does not involve acknowledgments.

Let's take a look at another figure, one that includes TCP, UDP, and the
applications associated to each protocol:
\protect\hyperlink{c03.xhtmlux5cux23figure03-14}{Figure 3.14} (discussed
in the next section).

\begin{figure}
\centering
\includegraphics{images/c03f014.jpg}
\caption{{\protect\hyperlink{c03.xhtmlux5cux23figureanchor03-14}{\textbf{FIGURE
3.14}} Port numbers for TCP and UDP}}
\end{figure}

\paragraph{Port Numbers}

TCP and UDP must use \emph{port numbers} to communicate with the upper
layers because these are what keep track of different conversations
crossing the network simultaneously. Originating-source port numbers are
dynamically assigned by the source host and will equal some number
starting at 1024. Port number 1023 and below are defined in RFC 3232 (or
just see \href{http://www.iana.org}{www.iana.org}), which discusses what
we call well-known port numbers.

Virtual circuits that don't use an application with a well-known port
number are assigned port numbers randomly from a specific range instead.
These port numbers identify the source and destination application or
process in the TCP segment.

\begin{center}\rule{0.5\linewidth}{0.5pt}\end{center}

\includegraphics{images/note.png} The Requests for Comments (RFCs) form
a series of notes about the Internet (originally the ARPAnet) started in
1969. These notes discuss many aspects of computer communication,
focusing on networking protocols, procedures, programs, and concepts,
but they also include meeting notes, opinions, and sometimes even humor.
You can find the RFCs by visiting \texttt{www.iana.org}.

\begin{center}\rule{0.5\linewidth}{0.5pt}\end{center}

\protect\hyperlink{c03.xhtmlux5cux23figure03-14}{Figure 3.14}
illustrates how both TCP and UDP use port numbers. I'll cover the
different port numbers that can be used next:

\begin{enumerate}
\tightlist
\item
  Numbers below 1024 are considered well-known port numbers and are
  defined in RFC 3232.
\item
  Numbers 1024 and above are used by the upper layers to set up sessions
  with other hosts and by TCP and UDP to use as source and destination
  addresses in the segment.
\end{enumerate}

\subparagraph[TCP Session: Source
Port]{\texorpdfstring{\protect\hypertarget{c03.xhtmlux5cux23Page_105}{}{}TCP
Session: Source Port}{TCP Session: Source Port}}

Let's take a minute to check out analyzer output showing a TCP session I
captured with my analyzer software session now:

\begin{verbatim}
TCP - Transport Control Protocol
 Source Port:      5973
 Destination Port: 23
 Sequence Number:  1456389907
 Ack Number:       1242056456
 Offset:           5
 Reserved:         %000000
 Code:             %011000
      Ack is valid
      Push Request
 Window:           61320
 Checksum:         0x61a6
 Urgent Pointer:   0
 No TCP Options
 TCP Data Area:
 vL.5.+.5.+.5.+.5  76 4c 19 35 11 2b 19 35 11 2b 19 35 11
  2b 19 35 +. 11 2b 19
Frame Check Sequence: 0x0d00000f
\end{verbatim}

Notice that the source host makes up the source port, which in this case
is 5973. The destination port is 23, which is used to tell the receiving
host the purpose of the intended connection (Telnet).

By looking at this session, you can see that the source host makes up
the source port by using numbers from 1024 to 65535. But why does the
source make up a port number? To differentiate between sessions with
different hosts because how would a server know where information is
coming from if it didn't have a different number from a sending host?
TCP and the upper layers don't use hardware and logical addresses to
understand the sending host's address as the Data Link and Network layer
protocols do. Instead, they use port numbers.

\subparagraph{TCP Session: Destination Port}

You'll sometimes look at an analyzer and see that only the source port
is above 1024 and the destination port is a well-known port, as shown in
the following trace:

\begin{verbatim}
TCP - Transport Control Protocol
 Source Port:      1144
 Destination Port: 80 World Wide Web HTTP
 Sequence Number:  9356570
 Ack Number:       0
 Offset:           7
 Reserved:         %000000
 Code:             %000010
      Synch Sequence
 Window:           8192
 Checksum:         0x57E7
 Urgent Pointer:   0
 TCP Options:
  Option Type: 2 Maximum Segment Size
    Length:    4
    MSS:       536
  Option Type: 1 No Operation
  Option Type: 1 No Operation
  Option Type: 4
    Length:    2
    Opt Value:
  No More HTTP Data
Frame Check Sequence: 0x43697363
\end{verbatim}

And sure enough, the source port is over 1024, but the destination port
is 80, indicating an HTTP service. The server, or receiving host, will
change the destination port if it needs to.

In the preceding trace, a ``SYN'' packet is sent to the destination
device. This Synch (as shown in the output) sequence is what's used to
inform the remote destination device that it wants to create a session.

\subparagraph{TCP Session: Syn Packet Acknowledgment}

The next trace shows an acknowledgment to the SYN packet:

\begin{verbatim}
TCP - Transport Control Protocol
 Source Port:      80 World Wide Web HTTP
 Destination Port: 1144
 Sequence Number:  2873580788
 Ack Number:       9356571
 Offset:           6
 Reserved:         %000000
 Code:             %010010
      Ack is valid
      Synch Sequence
 Window:           8576
 Checksum:         0x5F85
 Urgent Pointer:   0
 TCP Options:
  Option Type: 2 Maximum Segment Size
    Length:    4
    MSS:       1460
  No More HTTP Data
Frame Check Sequence: 0x6E203132
\end{verbatim}

Notice the \texttt{Ack\ is\ valid}, which means that the source port was
accepted and the device agreed to create a virtual circuit with the
originating host.

And here again, you can see that the response from the server shows that
the source is 80 and the destination is the 1144 sent from the
originating host---all's well!

\protect\hyperlink{c03.xhtmlux5cux23table03-2}{Table 3.2} gives you a
list of the typical applications used in the TCP/IP suite by showing
their well-known port numbers and the Transport layer protocols used by
each application or process. It's really key to memorize this table.

{\protect\hyperlink{c03.xhtmlux5cux23tableanchor03-2}{Table 3.2} Key
protocols that use TCP and UDP}

\begin{longtable}[]{@{}ll@{}}
\toprule
\textgreater TCP & \textgreater UDP\tabularnewline
\midrule
\endhead
Telnet 23 & SNMP 161\tabularnewline
SMTP 25 & TFTP 69\tabularnewline
HTTP 80 & DNS 53\tabularnewline
FTP 20, 21 & BooTPS/DHCP 67\tabularnewline
DNS 53 &\tabularnewline
HTTPS 443 & NTP 123\tabularnewline
SSH 22 &\tabularnewline
POP3 110 &\tabularnewline
IMAP4 143 &\tabularnewline
\bottomrule
\end{longtable}

Notice that DNS uses both TCP and UDP. Whether it opts for one or the
other depends on what it's trying to do. Even though it's not the only
application that can use both protocols, it's certainly one that you
should make sure to remember in your studies.

\begin{center}\rule{0.5\linewidth}{0.5pt}\end{center}

\includegraphics{images/note.png} What makes TCP reliable is sequencing,
acknowledgments, and flow control (windowing). UDP does not have
reliability.

\begin{center}\rule{0.5\linewidth}{0.5pt}\end{center}

\protect\hypertarget{c03.xhtmlux5cux23Page_108}{}{}Okay---I want to
discuss one more item before we move down to the Internet
layer---session multiplexing. Session multiplexing is used by both TCP
and UDP and basically allows a single computer, with a single IP
address, to have multiple sessions occurring simultaneously. Say you go
to \href{http://www.lammle.com}{www.lammle.com} and are browsing and
then you click a link to another page. Doing this opens another session
to your host. Now you go to
\href{http://www.lammle.com/forum}{www.lammle.com/forum} from another
window and that site opens a window as well. Now you have three sessions
open using one IP address because the Session layer is sorting the
separate requests based on the Transport layer port number. This is the
job of the Session layer: to keep application layer data separate!

\subsubsection[The Internet Layer
Protocols]{\texorpdfstring{\protect\hypertarget{c03.xhtmlux5cux23c03-sec-6}{}{}The
Internet Layer Protocols}{The Internet Layer Protocols}}

In the DoD model, there are two main reasons for the Internet layer's
existence: routing and providing a single network interface to the upper
layers.

None of the other upper- or lower-layer protocols have any functions
relating to routing---that complex and important task belongs entirely
to the Internet layer. The Internet layer's second duty is to provide a
single network interface to the upper-layer protocols. Without this
layer, application programmers would need to write ``hooks'' into every
one of their applications for each different Network Access protocol.
This would not only be a pain in the neck, but it would lead to
different versions of each application---one for Ethernet, another one
for wireless, and so on. To prevent this, IP provides one single network
interface for the upper-layer protocols. With that mission accomplished,
it's then the job of IP and the various Network Access protocols to get
along and work together.

All network roads don't lead to Rome---they lead to IP. And all the
other protocols at this layer, as well as all those at the upper layers,
use it. Never forget that. All paths through the DoD model go through
IP. Here's a list of the important protocols at the Internet layer that
I'll cover individually in detail coming up:

\begin{enumerate}
\tightlist
\item
  Internet Protocol (IP)
\item
  Internet Control Message Protocol (ICMP)
\item
  Address Resolution Protocol (ARP)
\end{enumerate}

\paragraph{Internet Protocol (IP)}

\emph{Internet Protocol (IP)} essentially is the Internet layer. The
other protocols found here merely exist to support it. IP holds the big
picture and could be said to ``see all,'' because it's aware of all the
interconnected networks. It can do this because all the machines on the
network have a software, or logical, address called an IP address, which
we'll explore more thoroughly later in this chapter.

For now, understand that IP looks at each packet's address. Then, using
a routing table, it decides where a packet is to be sent next, choosing
the best path to send it upon. The protocols of the Network Access layer
at the bottom of the DoD model don't possess IP's enlightened scope of
the entire network; they deal only with physical links (local networks).

\protect\hypertarget{c03.xhtmlux5cux23Page_109}{}{}Identifying devices
on networks requires answering these two questions: Which network is it
on? And what is its ID on that network? The first answer is the
\emph{software address}, or \emph{logical address}. You can think of
this as the part of the address that specifies the correct street. The
second answer is the hardware address, which goes a step further to
specify the correct mailbox. All hosts on a network have a logical ID
called an IP address. This is the software, or logical, address and
contains valuable encoded information, greatly simplifying the complex
task of routing. (IP is discussed in RFC 791.)

IP receives segments from the Host-to-Host layer and fragments them into
datagrams (packets) if necessary. IP then reassembles datagrams back
into segments on the receiving side. Each datagram is assigned the IP
address of the sender and that of the recipient. Each router or switch
(layer 3 device) that receives a datagram makes routing decisions based
on the packet's destination IP address.

\protect\hyperlink{c03.xhtmlux5cux23figure03-15}{Figure 3.15} shows an
IP header. This will give you a picture of what the IP protocol has to
go through every time user data that is destined for a remote network is
sent from the upper layers.

\begin{figure}
\centering
\includegraphics{images/c03f015.jpg}
\caption{{\protect\hyperlink{c03.xhtmlux5cux23figureanchor03-15}{\textbf{FIGURE
3.15}} IP header}}
\end{figure}

The following fields make up the IP header:

\textbf{Version} IP version number.

\textbf{Header length} Header length (HLEN) in 32-bit words.

\textbf{Priority and Type of Service} Type of Service tells how the
datagram should be handled. The first 3 bits are the priority bits, now
called the differentiated services bits.

\textbf{Total length} Length of the packet, including header and data.

\textbf{Identification} Unique IP-packet value used to differentiate
fragmented packets from different datagrams.

\textbf{Flags} Specifies whether fragmentation should occur.

\textbf{Fragment offset} Provides fragmentation and reassembly if the
packet is too large to put in a frame. It also allows different maximum
transmission units (MTUs) on the Internet.

\textbf{Time To Live} The time to live (TTL) is set into a packet when
it is originally generated. If it doesn't get to where it's supposed to
go before the TTL expires, boom---it's gone. This stops IP packets from
continuously circling the network looking for a home.

\textbf{\protect\hypertarget{c03.xhtmlux5cux23Page_110}{}{}Protocol}
Port of upper-layer protocol; for example, TCP is port 6 or UDP is port
17. Also supports Network layer protocols, like ARP and ICMP, and can be
referred to as the Type field in some analyzers. We'll talk about this
field more in a minute.

\textbf{Header checksum} Cyclic redundancy check (CRC) on header only.

\textbf{Source IP address} 32-bit IP address of sending station.

\textbf{Destination IP address} 32-bit IP address of the station this
packet is destined for.

\textbf{Options} Used for network testing, debugging, security, and
more.

\textbf{Data} After the IP option field, will be the upper-layer data.

Here's a snapshot of an IP packet caught on a network analyzer. Notice
that all the header information discussed previously appears here:

\begin{verbatim}
IP Header - Internet Protocol Datagram
 Version:             4
 Header Length:       5
 Precedence:          0
 Type of Service:     %000
 Unused:              %00
 Total Length:        187
 Identifier:          22486
 Fragmentation Flags: %010 Do Not Fragment
 Fragment Offset:     0
 Time To Live:        60
 IP Type:             0x06 TCP
 Header Checksum:     0xd031
 Source IP Address:   10.7.1.30
 Dest. IP Address:    10.7.1.10
 No Internet Datagram Options
\end{verbatim}

The Type field is typically a Protocol field, but this analyzer sees it
as an IP Type field. This is important. If the header didn't carry the
protocol information for the next layer, IP wouldn't know what to do
with the data carried in the packet. The preceding example clearly tells
IP to hand the segment to TCP.

\protect\hyperlink{c03.xhtmlux5cux23figure03-16}{Figure 3.16}
demonstrates how the Network layer sees the protocols at the Transport
layer when it needs to hand a packet up to the upper-layer protocols.

\protect\hyperlink{c03.xhtmlux5cux23figure03-16}{Figure 3.16} The
Protocol field in an IP header

\begin{figure}
\centering
\includegraphics{images/c03f016.jpg}
\caption{{\protect\hyperlink{c03.xhtmlux5cux23figureanchor03-16}{\textbf{FIGURE
3.16}} The Protocol field in an IP header}}
\end{figure}

\protect\hypertarget{c03.xhtmlux5cux23Page_111}{}{}In this example, the
Protocol field tells IP to send the data to either TCP port 6 or UDP
port 17. But it will be UDP or TCP only if the data is part of a data
stream headed for an upper-layer service or application. It could just
as easily be destined for Internet Control Message Protocol (ICMP),
Address Resolution Protocol (ARP), or some other type of Network layer
protocol.

\protect\hyperlink{c03.xhtmlux5cux23table03-3}{Table 3.3} is a list of
some other popular protocols that can be specified in the Protocol
field.

{\protect\hyperlink{c03.xhtmlux5cux23tableanchor03-3}{Table 3.3}
Possible protocols found in the Protocol field of an IP header}

\begin{longtable}[]{@{}ll@{}}
\toprule
\textgreater Protocol & \textgreater Protocol Number\tabularnewline
\midrule
\endhead
ICMP & 1\tabularnewline
IP in IP (tunneling) & 4\tabularnewline
TCP & 6\tabularnewline
UDP & 17\tabularnewline
EIGRP & 88\tabularnewline
OSPF & 89\tabularnewline
IPv6 & 41\tabularnewline
GRE & 47\tabularnewline
Layer 2 tunnel (L2TP) & 115\tabularnewline
\bottomrule
\end{longtable}

\begin{center}\rule{0.5\linewidth}{0.5pt}\end{center}

\includegraphics{images/note.png} You can find a complete list of
Protocol field numbers at
\href{http://www.iana.org/assignments/protocol-numbers}{www.iana.org/assignments/protocol-numbers}.

\begin{center}\rule{0.5\linewidth}{0.5pt}\end{center}

\paragraph{Internet Control Message Protocol (ICMP)}

\emph{Internet Control Message Protocol (ICMP)} works at the Network
layer and is used by IP for many different services. ICMP is basically a
management protocol and messaging service provider for IP. Its messages
are carried as IP datagrams. RFC 1256 is an annex to ICMP, which gives
hosts extended capability in discovering routes to gateways.

ICMP packets have the following characteristics:

\begin{enumerate}
\tightlist
\item
  They can provide hosts with information about network problems.
\item
  They are encapsulated within IP datagrams.
\end{enumerate}

\protect\hypertarget{c03.xhtmlux5cux23Page_112}{}{}The following are
some common events and messages that ICMP relates to:

\textbf{Destination unreachable} If a router can't send an IP datagram
any further, it uses ICMP to send a message back to the sender, advising
it of the situation. For example, take a look at
\protect\hyperlink{c03.xhtmlux5cux23figure03-17}{Figure 3.17}, which
shows that interface e0 of the Lab\_B router is down.

\begin{figure}
\centering
\includegraphics{images/c03f017.jpg}
\caption{{\protect\hyperlink{c03.xhtmlux5cux23figureanchor03-17}{\textbf{FIGURE
3.17}} ICMP error message is sent to the sending host from the remote
router.}}
\end{figure}

When Host A sends a packet destined for Host B, the Lab\_B router will
send an ICMP destination unreachable message back to the sending device,
which is Host A in this example.

\textbf{Buffer full/source quench} If a router's memory buffer for
receiving incoming datagrams is full, it will use ICMP to send out this
message alert until the congestion abates.

\textbf{Hops/time exceeded} Each IP datagram is allotted a certain
number of routers, called hops, to pass through. If it reaches its limit
of hops before arriving at its destination, the last router to receive
that datagram deletes it. The executioner router then uses ICMP to send
an obituary message, informing the sending machine of the demise of its
datagram.

\textbf{Ping} Packet Internet Groper (Ping) uses ICMP echo request and
reply messages to check the physical and logical connectivity of
machines on an internetwork.

\textbf{Traceroute} Using ICMP time-outs, Traceroute is used to discover
the path a packet takes as it traverses an internetwork.

\begin{center}\rule{0.5\linewidth}{0.5pt}\end{center}

\includegraphics{images/note.png} Traceroute is usually just called
trace. Microsoft Windows uses tracert to allow you to verify address
configurations in your internetwork.

\begin{center}\rule{0.5\linewidth}{0.5pt}\end{center}

The following data is from a network analyzer catching an ICMP echo
request:

\begin{verbatim}
Flags:         0x00
 Status:        0x00
 Packet Length: 78
 Timestamp:     14:04:25.967000 12/20/03
Ethernet Header
 Destination: 00:a0:24:6e:0f:a8
 Source:      00:80:c7:a8:f0:3d
 Ether-Type:  08-00 IP
IP Header - Internet Protocol Datagram
 Version:             4
 Header Length:       5
 Precedence:          0
 Type of Service:     %000
 Unused:              %00
 Total Length:        60
 Identifier:          56325
 Fragmentation Flags: %000
 Fragment Offset:     0
 Time To Live:        32
 IP Type:             0x01 ICMP
 Header Checksum:     0x2df0
 Source IP Address:   100.100.100.2
 Dest. IP Address:    100.100.100.1
 No Internet Datagram Options
ICMP - Internet Control Messages Protocol
 ICMP Type:       8 Echo Request
 Code:            0
 Checksum:        0x395c
 Identifier:      0x0300
 Sequence Number: 4352
 ICMP Data Area:
 abcdefghijklmnop  61 62 63 64 65 66 67 68 69 6a 6b 6c 6d 6e 6f 70
 qrstuvwabcdefghi  71 72 73 74 75 76 77 61 62 63 64 65 66 67 68 69
Frame Check Sequence: 0x00000000
\end{verbatim}

Notice anything unusual? Did you catch the fact that even though ICMP
works at the Internet (Network) layer, it still uses IP to do the Ping
request? The Type field in the IP header is \texttt{0x01}, which
specifies that the data we're carrying is owned by the ICMP protocol.
Remember, just as all roads lead to Rome, all segments or data
\emph{must} go through IP!

\begin{center}\rule{0.5\linewidth}{0.5pt}\end{center}

\includegraphics{images/note.png} The Ping program uses the alphabet in
the data portion of the packet as a payload, typically around 100 bytes
by default, unless, of course, you are pinging from a Windows device,
which thinks the alphabet stops at the letter \emph{W} (and doesn't
include \emph{X}, \emph{Y}, or \emph{Z}) and then starts at \emph{A}
again. Go figure!

\begin{center}\rule{0.5\linewidth}{0.5pt}\end{center}

If you remember reading about the Data Link layer and the different
frame types in Chapter 2, ``Ethernet Networking and Data
Encapsulation,'' you should be able to look at the preceding trace and
tell what type of Ethernet frame this is. The only fields are
\protect\hypertarget{c03.xhtmlux5cux23Page_114}{}{}destination hardware
address, source hardware address, and Ether-Type. The only frame that
uses an Ether-Type field exclusively is an Ethernet\_II frame.

We'll move on soon, but before we get into the ARP protocol, let's take
another look at ICMP in action.
\protect\hyperlink{c03.xhtmlux5cux23figure03-18}{Figure 3.18} shows an
internetwork---it has a router, so it's an internetwork, right?

\begin{figure}
\centering
\includegraphics{images/c03f018.jpg}
\caption{{\protect\hyperlink{c03.xhtmlux5cux23figureanchor03-18}{\textbf{FIGURE
3.18}} ICMP in action}}
\end{figure}

Server 1 (10.1.2.2) telnets to 10.1.1.5 from a DOS prompt. What do you
think Server 1 will receive as a response? Server 1 will send the Telnet
data to the default gateway, which is the router, and the router will
drop the packet because there isn't a network 10.1.1.0 in the routing
table. Because of this, Server 1 will receive an ICMP destination
unreachable back from the router.

\paragraph{Address Resolution Protocol (ARP)}

\emph{Address Resolution Protocol (ARP)} finds the hardware address of a
host from a known IP address. Here's how it works: When IP has a
datagram to send, it must inform a Network Access protocol, such as
Ethernet or wireless, of the destination's hardware address on the local
network. Remember that it has already been informed by upper-layer
protocols of the destination's IP address. If IP doesn't find the
destination host's hardware address in the ARP cache, it uses ARP to
find this information.

As IP's detective, ARP interrogates the local network by sending out a
broadcast asking the machine with the specified IP address to reply with
its hardware address. So basically, ARP translates the software (IP)
address into a hardware address---for example, the destination machine's
Ethernet adapter address---and from it, deduces its whereabouts on the
LAN by broadcasting for this address.
\protect\hyperlink{c03.xhtmlux5cux23figure03-19}{Figure 3.19} shows how
an ARP broadcast looks to a local network.

\protect\hypertarget{c03.xhtmlux5cux23Page_115}{}{}

\begin{figure}
\centering
\includegraphics{images/c03f019.jpg}
\caption{{\protect\hyperlink{c03.xhtmlux5cux23figureanchor03-19}{\textbf{FIGURE
3.19}} Local ARP broadcast}}
\end{figure}

\begin{center}\rule{0.5\linewidth}{0.5pt}\end{center}

\includegraphics{images/note.png} ARP resolves IP addresses to Ethernet
(MAC) addresses.

\begin{center}\rule{0.5\linewidth}{0.5pt}\end{center}

The following trace shows an ARP broadcast---notice that the destination
hardware address is unknown and is all \emph{F}s in hex (all 1s in
binary)---and is a hardware address broadcast:

\begin{verbatim}
 Flags:         0x00
 Status:        0x00
 Packet Length: 64
 Timestamp:     09:17:29.574000 12/06/03
Ethernet Header
 Destination:   FF:FF:FF:FF:FF:FF Ethernet Broadcast
 Source:        00:A0:24:48:60:A5
 Protocol Type: 0x0806 IP ARP
ARP - Address Resolution Protocol
 Hardware:                1 Ethernet (10Mb)
 Protocol:                0x0800 IP
 Hardware Address Length: 6
 Protocol Address Length: 4
 Operation:               1 ARP Request
 Sender Hardware Address: 00:A0:24:48:60:A5
 Sender Internet Address: 172.16.10.3
 Target Hardware Address: 00:00:00:00:00:00 (ignored)
 Target Internet Address: 172.16.10.10
Extra bytes (Padding):
 ................ 0A 0A 0A 0A 0A 0A 0A 0A 0A 0A 0A 0A 0A
  0A 0A 0A 0A 0A
Frame Check Sequence: 0x00000000
\end{verbatim}

\subsection[IP
Addressing]{\texorpdfstring{\protect\hypertarget{c03.xhtmlux5cux23c03-sec-7}{}{}IP
Addressing}{IP Addressing}}

One of the most important topics in any discussion of TCP/IP is IP
addressing. An \emph{IP address} is a numeric identifier assigned to
each machine on an IP network. It designates the specific location of a
device on the network.

An IP address is a software address, not a hardware address---the latter
is hard-coded on a network interface card (NIC) and used for finding
hosts on a local network. IP addressing was designed to allow hosts on
one network to communicate with a host on a different network regardless
of the type of LANs the hosts are participating in.

Before we get into the more complicated aspects of IP addressing, you
need to understand some of the basics. First I'm going to explain some
of the fundamentals of IP addressing and its terminology. Then you'll
learn about the hierarchical IP addressing scheme and private IP
addresses.

\subsubsection[IP
Terminology]{\texorpdfstring{\protect\hypertarget{c03.xhtmlux5cux23c03-sec-8}{}{}IP
Terminology}{IP Terminology}}

Throughout this chapter you're being introduced to several important
terms that are vital to understanding the Internet Protocol. Here are a
few to get you started:

\textbf{Bit} A bit is one digit, either a 1 or a 0.

\textbf{Byte} A byte is 7 or 8 bits, depending on whether parity is
used. For the rest of this chapter, always assume a byte is 8 bits.

\textbf{Octet} An octet, made up of 8 bits, is just an ordinary 8-bit
binary number. In this chapter, the terms \emph{byte} and \emph{octet}
are completely interchangeable.

\textbf{Network address} This is the designation used in routing to send
packets to a remote network---for example, 10.0.0.0, 172.16.0.0, and
192.168.10.0.

\textbf{Broadcast address} The address used by applications and hosts to
send information to all nodes on a network is called the broadcast
address. Examples of layer 3 broadcasts include 255.255.255.255, which
is any network, all nodes; 172.16.255.255, which is all subnets and
hosts on network 172.16.0.0; and 10.255.255.255, which broadcasts to all
subnets and hosts on network 10.0.0.0.

\subsubsection[The Hierarchical IP Addressing
Scheme]{\texorpdfstring{\protect\hypertarget{c03.xhtmlux5cux23c03-sec-9}{}{}\protect\hypertarget{c03.xhtmlux5cux23Page_117}{}{}The
Hierarchical IP Addressing
Scheme}{The Hierarchical IP Addressing Scheme}}

An IP address consists of 32 bits of information. These bits are divided
into four sections, referred to as octets or bytes, with each containing
1 byte (8 bits). You can depict an IP address using one of three
methods:

\begin{enumerate}
\tightlist
\item
  Dotted-decimal, as in 172.16.30.56
\item
  Binary, as in 10101100.00010000.00011110.00111000
\item
  Hexadecimal, as in AC.10.1E.38
\end{enumerate}

All these examples represent the same IP address. Pertaining to IP
addressing, hexadecimal isn't used as often as dotted-decimal or binary,
but you still might find an IP address stored in hexadecimal in some
programs.

The 32-bit IP address is a structured or hierarchical address, as
opposed to a flat or nonhierarchical address. Although either type of
addressing scheme could have been used, \emph{hierarchical addressing}
was chosen for a good reason. The advantage of this scheme is that it
can handle a large number of addresses, namely 4.3 billion (a 32-bit
address space with two possible values for each position---either 0 or
1---gives you 2\textsuperscript{32}, or 4,294,967,296). The disadvantage
of the flat addressing scheme, and the reason it's not used for IP
addressing, relates to routing. If every address were unique, all
routers on the Internet would need to store the address of each and
every machine on the Internet. This would make efficient routing
impossible, even if only a fraction of the possible addresses were used!

The solution to this problem is to use a two- or three-level
hierarchical addressing scheme that is structured by network and host or
by network, subnet, and host.

This two- or three-level scheme can also be compared to a telephone
number. The first section, the area code, designates a very large area.
The second section, the prefix, narrows the scope to a local calling
area. The final segment, the customer number, zooms in on the specific
connection. IP addresses use the same type of layered structure. Rather
than all 32 bits being treated as a unique identifier, as in flat
addressing, a part of the address is designated as the network address
and the other part is designated as either the subnet and host or just
the node address.

Next, we'll cover IP network addressing and the different classes of
address we can use to address our networks.

\paragraph{Network Addressing}

The \emph{network address} (which can also be called the network number)
uniquely identifies each network. Every machine on the same network
shares that network address as part of its IP address. For example, in
the IP address 172.16.30.56, 172.16 is the network address.

The \emph{node address} is assigned to, and uniquely identifies, each
machine on a network. This part of the address must be unique because it
identifies a particular machine---an individual--- as opposed to a
network, which is a group. This number can also be referred to as a
\emph{host address}. In the sample IP address 172.16.30.56, the 30.56
specifies the node address.

The designers of the Internet decided to create classes of networks
based on network size. For the small number of networks possessing a
very large number of nodes, they
\protect\hypertarget{c03.xhtmlux5cux23Page_118}{}{}created the rank
\emph{Class A network}. At the other extreme is the \emph{Class C
network}, which is reserved for the numerous networks with a small
number of nodes. The class distinction for networks between very large
and very small is predictably called the \emph{Class B network}.

Subdividing an IP address into a network and node address is determined
by the class designation of one's network.
\protect\hyperlink{c03.xhtmlux5cux23figure03-20}{Figure 3.20} summarizes
the three classes of networks used to address hosts---a subject I'll
explain in much greater detail throughout this chapter.

\begin{figure}
\centering
\includegraphics{images/c03f020.jpg}
\caption{{\protect\hyperlink{c03.xhtmlux5cux23figureanchor03-20}{\textbf{FIGURE
3.20}} Summary of the three classes of networks}}
\end{figure}

To ensure efficient routing, Internet designers defined a mandate for
the leading-bits section of the address for each different network
class. For example, since a router knows that a Class A network address
always starts with a 0, the router might be able to speed a packet on
its way after reading only the first bit of its address. This is where
the address schemes define the difference between a Class A, a Class B,
and a Class C address. Coming up, I'll discuss the differences between
these three classes, followed by a discussion of the Class D and Class E
addresses. Classes A, B, and C are the only ranges that are used to
address hosts in our networks.

\subparagraph{Network Address Range: Class A}

The designers of the IP address scheme decided that the first bit of the
first byte in a Class A network address must always be off, or 0. This
means a Class A address must be between 0 and 127 in the first byte,
inclusive.

Consider the following network address:

\begin{verbatim}
0xxxxxxx
\end{verbatim}

If we turn the other 7 bits all off and then turn them all on, we'll
find the Class A range of network addresses:

\begin{verbatim}
00000000 = 0
01111111 = 127
\end{verbatim}

So, a Class A network is defined in the first octet between 0 and 127,
and it can't be less or more. Understand that 0 and 127 are not valid in
a Class A network because they're reserved addresses, which I'll explain
soon.

\subparagraph[Network Address Range: Class
B]{\texorpdfstring{\protect\hypertarget{c03.xhtmlux5cux23Page_119}{}{}Network
Address Range: Class B}{Network Address Range: Class B}}

In a Class B network, the RFCs state that the first bit of the first
byte must always be turned on but the second bit must always be turned
off. If you turn the other 6 bits all off and then all on, you will find
the range for a Class B network:

\begin{verbatim}
10000000 = 128
10111111 = 191
\end{verbatim}

As you can see, a Class B network is defined when the first byte is
configured from 128 to 191.

\subparagraph{Network Address Range: Class C}

For Class C networks, the RFCs define the first 2 bits of the first
octet as always turned on, but the third bit can never be on. Following
the same process as the previous classes, convert from binary to decimal
to find the range. Here's the range for a Class C network:

\begin{verbatim}
11000000 = 192
11011111 = 223
\end{verbatim}

So, if you see an IP address that starts at 192 and goes to 223, you'll
know it is a Class C IP address.

\subparagraph{Network Address Ranges: Classes D and E}

The addresses between 224 to 255 are reserved for Class D and E
networks. Class D (224--239) is used for multicast addresses and Class E
(240--255) for scientific purposes, but I'm not going into these types
of addresses because they are beyond the scope of knowledge you need to
gain from this book.

\subparagraph{Network Addresses: Special Purpose}

Some IP addresses are reserved for special purposes, so network
administrators can't ever assign these addresses to nodes.
\protect\hyperlink{c03.xhtmlux5cux23table03-4}{Table 3.4} lists the
members of this exclusive little club and the reasons why they're
included in it.

{\protect\hyperlink{c03.xhtmlux5cux23tableanchor03-4}{Table 3.4}
Reserved IP addresses}

\begin{longtable}[]{@{}ll@{}}
\toprule
\textgreater Address & \textgreater Function\tabularnewline
\midrule
\endhead
Network address of all 0s & Interpreted to mean ``this network or
segment.''\tabularnewline
Network address of all 1s & Interpreted to mean ``all
networks.''\tabularnewline
Network 127.0.0.1 & Reserved for loopback tests. Designates the local
node and allows that node to send a test packet to itself without
generating network traffic.\tabularnewline
\protect\hypertarget{c03.xhtmlux5cux23Page_120}{}{}Node address of all
0s & Interpreted to mean ``network address'' or any host on a specified
network.\tabularnewline
Node address of all 1s & Interpreted to mean ``all nodes'' on the
specified network; for example, 128.2.255.255 means ``all nodes'' on
network 128.2 (Class B address).\tabularnewline
Entire IP address set to all 0s & Used by Cisco routers to designate the
default route. Could also mean ``any network.''\tabularnewline
Entire IP address set to all 1s (same as 255.255.255.255) & Broadcast to
all nodes on the current network; sometimes called an ``all 1s
broadcast'' or local broadcast.\tabularnewline
\bottomrule
\end{longtable}

\paragraph{Class A Addresses}

In a Class A network address, the first byte is assigned to the network
address and the three remaining bytes are used for the node addresses.
The Class A format is as follows:

\begin{verbatim}
network.node.node.node
\end{verbatim}

For example, in the IP address 49.22.102.70, the 49 is the network
address and 22.102.70 is the node address. Every machine on this
particular network would have the distinctive network address of 49.

Class A network addresses are 1 byte long, with the first bit of that
byte reserved and the 7 remaining bits available for manipulation
(addressing). As a result, the maximum number of Class A networks that
can be created is 128. Why? Because each of the 7 bit positions can be
either a 0 or a 1, thus 2\textsuperscript{7}, or 128.

To complicate matters further, the network address of all 0s (0000 0000)
is reserved to designate the default route (see
\protect\hyperlink{c03.xhtmlux5cux23table03-4}{Table 3.4} in the
previous section). Additionally, the address 127, which is reserved for
diagnostics, can't be used either, which means that you can really only
use the numbers 1 to 126 to designate Class A network addresses. This
means the actual number of usable Class A network addresses is 128 minus
2, or 126.

\begin{center}\rule{0.5\linewidth}{0.5pt}\end{center}

\includegraphics{images/note.png} The IP address 127.0.0.1 is used to
test the IP stack on an individual node and cannot be used as a valid
host address. However, the loopback address creates a shortcut method
for TCP/IP applications and services that run on the same device to
communicate with each other.

\begin{center}\rule{0.5\linewidth}{0.5pt}\end{center}

Each Class A address has 3 bytes (24-bit positions) for the node address
of a machine. This means there are 2\textsuperscript{24} ---or
16,777,216---unique combinations and, therefore, precisely
\protect\hypertarget{c03.xhtmlux5cux23Page_121}{}{}that many possible
unique node addresses for each Class A network. Because node addresses
with the two patterns of all 0s and all 1s are reserved, the actual
maximum usable number of nodes for a Class A network is
2\textsuperscript{24} minus 2, which equals 16,777,214. Either way,
that's a huge number of hosts on a single network segment!

\subparagraph{Class A Valid Host IDs}

Here's an example of how to figure out the valid host IDs in a Class A
network address:

\begin{enumerate}
\tightlist
\item
  All host bits off is the network address: 10.0.0.0.
\item
  All host bits on is the broadcast address: 10.255.255.255.
\end{enumerate}

The valid hosts are the numbers in between the network address and the
broadcast address: 10.0.0.1 through 10.255.255.254. Notice that 0s and
255s can be valid host IDs. All you need to remember when trying to find
valid host addresses is that the host bits can't all be turned off or on
at the same time.

\paragraph{Class B Addresses}

In a Class B network address, the first 2 bytes are assigned to the
network address and the remaining 2 bytes are used for node addresses.
The format is as follows:

\begin{verbatim}
network.network.node.node
\end{verbatim}

For example, in the IP address 172.16.30.56, the network address is
172.16 and the node address is 30.56.

With a network address being 2 bytes (8 bits each), you get
2\textsuperscript{16} unique combinations. But the Internet designers
decided that all Class B network addresses should start with the binary
digit 1, then 0. This leaves 14 bit positions to manipulate, therefore
16,384, or 2\textsuperscript{14} unique Class B network addresses.

A Class B address uses 2 bytes for node addresses. This is
2\textsuperscript{16} minus the two reserved patterns of all 0s and all
1s for a total of 65,534 possible node addresses for each Class B
network.

\subparagraph{Class B Valid Host IDs}

Here's an example of how to find the valid hosts in a Class B network:

\begin{enumerate}
\tightlist
\item
  All host bits turned off is the network address: 172.16.0.0.
\item
  All host bits turned on is the broadcast address: 172.16.255.255.
\end{enumerate}

The valid hosts would be the numbers in between the network address and
the broadcast address: 172.16.0.1 through 172.16.255.254.

\paragraph{Class C Addresses}

The first 3 bytes of a Class C network address are dedicated to the
network portion of the address, with only 1 measly byte remaining for
the node address. Here's the format:

\begin{verbatim}
network.network.network.node
\end{verbatim}

Using the example IP address 192.168.100.102, the network address is
192.168.100 and the node address is 102.

\protect\hypertarget{c03.xhtmlux5cux23Page_122}{}{}In a Class C network
address, the first three bit positions are always the binary 110. The
calculation is as follows: 3 bytes, or 24 bits, minus 3 reserved
positions leaves 21 positions. Hence, there are 2\textsuperscript{21},
or 2,097,152, possible Class C networks.

Each unique Class C network has 1 byte to use for node addresses. This
leads to 2\textsuperscript{8}, or 256, minus the two reserved patterns
of all 0s and all 1s, for a total of 254 node addresses for each Class C
network.

\subparagraph{Class C Valid Host IDs}

Here's an example of how to find a valid host ID in a Class C network:

\begin{enumerate}
\tightlist
\item
  All host bits turned off is the network ID: 192.168.100.0.
\item
  All host bits turned on is the broadcast address: 192.168.100.255.
\end{enumerate}

The valid hosts would be the numbers in between the network address and
the broadcast address: 192.168.100.1 through 192.168.100.254.

\subsubsection[Private IP Addresses (RFC
1918)]{\texorpdfstring{\protect\hypertarget{c03.xhtmlux5cux23c03-sec-10}{}{}Private
IP Addresses (RFC 1918)}{Private IP Addresses (RFC 1918)}}

The people who created the IP addressing scheme also created private IP
addresses. These addresses can be used on a private network, but they're
not routable through the Internet. This is designed for the purpose of
creating a measure of well-needed security, but it also conveniently
saves valuable IP address space.

If every host on every network was required to have real routable IP
addresses, we would have run out of IP addresses to hand out years ago.
But by using private IP addresses, ISPs, corporations, and home users
only need a relatively tiny group of bona fide IP addresses to connect
their networks to the Internet. This is economical because they can use
private IP addresses on their inside networks and get along just fine.

To accomplish this task, the ISP and the corporation---the end user, no
matter who they are---need to use something called \emph{Network Address
Translation (NAT)}, which basically takes a private IP address and
converts it for use on the Internet. NAT is covered in Chapter 13,
``Network Address Translation (NAT).'' Many people can use the same real
IP address to transmit out onto the Internet. Doing things this way
saves megatons of address space---good for us all!

The reserved private addresses are listed in
\protect\hyperlink{c03.xhtmlux5cux23table03-5}{Table 3.5}.

{\protect\hyperlink{c03.xhtmlux5cux23tableanchor03-5}{Table 3.5}
Reserved IP address space}

\begin{longtable}[]{@{}ll@{}}
\toprule
\textgreater Address Class & \textgreater Reserved Address
Space\tabularnewline
\midrule
\endhead
Class A & 10.0.0.0 through 10.255.255.255\tabularnewline
Class B & 172.16.0.0 through 172.31.255.255\tabularnewline
Class C & 192.168.0.0 through 192.168.255.255\tabularnewline
\bottomrule
\end{longtable}

\protect\hypertarget{c03.xhtmlux5cux23Page_123}{}{}

\begin{center}\rule{0.5\linewidth}{0.5pt}\end{center}

\includegraphics{images/note.png} You must know your private address
space to become Cisco certified!

\begin{center}\rule{0.5\linewidth}{0.5pt}\end{center}

\begin{center}\rule{0.5\linewidth}{0.5pt}\end{center}

\paragraph{So, What Private IP Address Should I Use?}

That's a really great question: Should you use Class A, Class B, or even
Class C private addressing when setting up your network? Let's take Acme
Corporation in SF as an example. This company is moving into a new
building and needs a whole new network. It has 14 departments, with
about 70 users in each. You could probably squeeze one or two Class C
addresses to use, or maybe you could use a Class B, or even a Class A
just for fun.

The rule of thumb in the consulting world is, when you're setting up a
corporate network--- regardless of how small it is---you should use a
Class A network address because it gives you the most flexibility and
growth options. For example, if you used the 10.0.0.0 network address
with a /24 mask, then you'd have 65,536 networks, each with 254 hosts.
Lots of room for growth with that network!

But if you're setting up a home network, you'd opt for a Class C address
because it is the easiest for people to understand and configure. Using
the default Class C mask gives you one network with 254 hosts---plenty
for a home network.

With the Acme Corporation, a nice 10.1.×.0 with a /24 mask (the × is the
subnet for each department) makes this easy to design, install, and
troubleshoot.

\begin{center}\rule{0.5\linewidth}{0.5pt}\end{center}

\subsection[IPv4 Address
Types]{\texorpdfstring{\protect\hypertarget{c03.xhtmlux5cux23c03-sec-11}{}{}IPv4
Address Types}{IPv4 Address Types}}

Most people use the term \emph{broadcast} as a generic term, and most of
the time, we understand what they mean---but not always! For example,
you might say, ``The host broadcasted through a router to a DHCP
server,'' but, well, it's pretty unlikely that this would ever really
happen. What you probably mean---using the correct technical
jargon---is, ``The DHCP client broadcasted for an IP address and a
router then forwarded this as a unicast packet to the DHCP server.'' Oh,
and remember that with IPv4, broadcasts are pretty important, but with
IPv6, there aren't any broadcasts sent at all---now there's something to
look forward to reading about in Chapter 14!

Okay, I've referred to IP addresses throughout the preceding chapters
and now all throughout this chapter, and even showed you some examples.
But I really haven't gone into the different terms and uses associated
with them yet, and it's about time I did. So here are the address types
that I'd like to define for you:

\textbf{Loopback (localhost)} Used to test the IP stack on the local
computer. Can be any address from 127.0.0.1 through 127.255.255.254.

\textbf{\protect\hypertarget{c03.xhtmlux5cux23Page_124}{}{}Layer 2
broadcasts} These are sent to all nodes on a LAN.

\textbf{Broadcasts (layer 3)} These are sent to all nodes on the
network.

\textbf{Unicast} This is an address for a single interface, and these
are used to send packets to a single destination host.

\textbf{Multicast} These are packets sent from a single source and
transmitted to many devices on different networks. Referred to as
``one-to-many.''

\subsubsection[Layer 2
Broadcasts]{\texorpdfstring{\protect\hypertarget{c03.xhtmlux5cux23c03-sec-12}{}{}Layer
2 Broadcasts}{Layer 2 Broadcasts}}

First, understand that layer 2 broadcasts are also known as hardware
broadcasts---they only go out on a LAN, but they don't go past the LAN
boundary (router).

The typical hardware address is 6 bytes (48 bits) and looks something
like 45:AC:24:E3:60:A5. The broadcast would be all 1s in binary, which
would be all \emph{F}s in hexadecimal, as in ff:ff:ff:ff:ff:ff and shown
in \protect\hyperlink{c03.xhtmlux5cux23figure03-21}{Figure 3.21}.

\begin{figure}
\centering
\includegraphics{images/c03f021.jpg}
\caption{{\protect\hyperlink{c03.xhtmlux5cux23figureanchor03-21}{\textbf{FIGURE
3.21}} Local layer 2 broadcasts}}
\end{figure}

Every network interface card (NIC) will receive and read the frame,
including the router, since this was a layer 2 broadcast, but the router
would never, ever forward this!

\subsubsection[Layer 3
Broadcasts]{\texorpdfstring{\protect\hypertarget{c03.xhtmlux5cux23c03-sec-13}{}{}Layer
3 Broadcasts}{Layer 3 Broadcasts}}

Then there are the plain old broadcast addresses at layer 3. Broadcast
messages are meant to reach all hosts on a broadcast domain. These are
the network broadcasts that have all host bits on.

Here's an example that you're already familiar with: The network address
of 172.16.0.0 255.255.0.0 would have a broadcast address of
172.16.255.255---all host bits on. Broadcasts can also be ``any network
and all hosts,'' as indicated by 255.255.255.255, and shown in
\protect\hyperlink{c03.xhtmlux5cux23figure03-22}{Figure 3.22}.

\protect\hypertarget{c03.xhtmlux5cux23Page_125}{}{}

\begin{figure}
\centering
\includegraphics{images/c03f022.jpg}
\caption{{\protect\hyperlink{c03.xhtmlux5cux23figureanchor03-22}{\textbf{FIGURE
3.22}} Layer 3 broadcasts}}
\end{figure}

In \protect\hyperlink{c03.xhtmlux5cux23figure03-22}{Figure 3.22}, all
hosts on the LAN will get this broadcast on their NIC, including the
router, but by default the router would never forward this packet.

\subsubsection[Unicast
Address]{\texorpdfstring{\protect\hypertarget{c03.xhtmlux5cux23c03-sec-14}{}{}Unicast
Address}{Unicast Address}}

A unicast is defined as a single IP address that's assigned to a network
interface card and is the destination IP address in a packet---in other
words, it's used for directing packets to a specific host.

In \protect\hyperlink{c03.xhtmlux5cux23figure03-23}{Figure 3.23}, both
the MAC address and the destination IP address are for a single NIC on
the network. All hosts on the broadcast domain would receive this frame
and accept it. Only the destination NIC of 10.1.1.2 would accept the
packet; the other NICs would discard the packet.

\begin{figure}
\centering
\includegraphics{images/c03f023.jpg}
\caption{{\protect\hyperlink{c03.xhtmlux5cux23figureanchor03-23}{\textbf{FIGURE
3.23}} Unicast address}}
\end{figure}

\subsubsection[Multicast
Address]{\texorpdfstring{\protect\hypertarget{c03.xhtmlux5cux23c03-sec-15}{}{}\protect\hypertarget{c03.xhtmlux5cux23Page_126}{}{}Multicast
Address}{Multicast Address}}

Multicast is a different beast entirely. At first glance, it appears to
be a hybrid of unicast and broadcast communication, but that isn't quite
the case. Multicast does allow point-to-multipoint communication, which
is similar to broadcasts, but it happens in a different manner. The crux
of \emph{multicast} is that it enables multiple recipients to receive
messages without flooding the messages to all hosts on a broadcast
domain. However, this is not the default behavior---it's what we
\emph{can} do with multicasting if it's configured correctly!

Multicast works by sending messages or data to IP \emph{multicast group}
addresses. Unlike with broadcasts, which aren't forwarded, routers then
forward copies of the packet out to every interface that has hosts
\emph{subscribed} to that group address. This is where multicast differs
from broadcast messages---with multicast communication, copies of
packets, in theory, are sent only to subscribed hosts. For example, when
I say in theory, I mean that the hosts will receive a multicast packet
destined for 224.0.0.10. This is an EIGRP packet, and only a router
running the EIGRP protocol will read these. All hosts on the broadcast
LAN, and Ethernet is a broadcast multi-access LAN technology, will pick
up the frame, read the destination address, then immediately discard the
frame unless they're in the multicast group. This saves PC processing,
not LAN bandwidth. Be warned though---multicasting can cause some
serious LAN congestion if it's not implemented carefully!
\protect\hyperlink{c03.xhtmlux5cux23figure03-24}{Figure 3.24} shows a
Cisco router sending an EIGRP multicast packet on the local LAN and only
the other Cisco router will accept and read this packet.

\begin{figure}
\centering
\includegraphics{images/c03f024.jpg}
\caption{{\protect\hyperlink{c03.xhtmlux5cux23figureanchor03-24}{\textbf{FIGURE
3.24}} EIGRP multicast example}}
\end{figure}

There are several different groups that users or applications can
subscribe to. The range of multicast addresses starts with 224.0.0.0 and
goes through 239.255.255.255. As you can see, this range of addresses
falls within IP Class D address space based on classful IP assignment.

\subsection[Summary]{\texorpdfstring{\protect\hypertarget{c03.xhtmlux5cux23c03-sec-16}{}{}\protect\hypertarget{c03.xhtmlux5cux23Page_127}{}{}Summary}{Summary}}

If you made it this far and understood everything the first time
through, you should be extremely proud of yourself! We really covered a
lot of ground in this chapter, but understand that the information in it
is critical to being able to navigate well through the rest of this
book.

If you didn't get a complete understanding the first time around, don't
stress. It really wouldn't hurt you to read this chapter more than once.
There is still a lot of ground to cover, so make sure you've got this
material all nailed down. That way, you'll be ready for more, and just
so you know, there's a lot more! What we're doing up to this point is
building a solid foundation to build upon as you advance.

With that in mind, after you learned about the DoD model, the layers,
and associated protocols, you learned about the oh-so-important topic of
IP addressing. I discussed in detail the difference between each address
class, how to find a network address and broadcast address, and what
denotes a valid host address range. I can't stress enough how important
it is for you to have this critical information unshakably understood
before moving on to Chapter 4!

Since you've already come this far, there's no reason to stop now and
waste all those brainwaves and new neural connections. So don't
stop---go through the written labs and review questions at the end of
this chapter and make sure you understand each answer's explanation. The
best is yet to come!

\subsection[Exam
Essentials]{\texorpdfstring{\protect\hypertarget{c03.xhtmlux5cux23c03-sec-17}{}{}Exam
Essentials}{Exam Essentials}}

\textbf{Differentiate between the DoD and the OSI network models.} The
DoD model is a condensed version of the OSI model, composed of four
layers instead of seven, but is nonetheless like the OSI model in that
it can be used to describe packet creation and devices and protocols can
be mapped to its layers.

\textbf{Identify Process/Application layer protocols.} Telnet is a
terminal emulation program that allows you to log into a remote host and
run programs. File Transfer Protocol (FTP) is a connection-oriented
service that allows you to transfer files. Trivial FTP (TFTP) is a
connectionless file transfer program. Simple Mail Transfer Protocol
(SMTP) is a sendmail program.

\textbf{Identify Host-to-Host layer protocols.} Transmission Control
Protocol (TCP) is a connection-oriented protocol that provides reliable
network service by using acknowledgments and flow control. User Datagram
Protocol (UDP) is a connectionless protocol that provides low overhead
and is considered unreliable.

\textbf{Identify Internet layer protocols.} Internet Protocol (IP) is a
connectionless protocol that provides network address and routing
through an internetwork. Address Resolution Protocol (ARP) finds a
hardware address from a known IP address. Reverse ARP (RARP) finds an IP
address from a known hardware address. Internet Control Message Protocol
(ICMP) provides diagnostics and destination unreachable messages.

\textbf{\protect\hypertarget{c03.xhtmlux5cux23Page_128}{}{}Describe the
functions of DNS and DHCP in the network.} Dynamic Host Configuration
Protocol (DHCP) provides network configuration information (including IP
addresses) to hosts, eliminating the need to perform the configurations
manually. Domain Name Service (DNS) resolves hostnames---both Internet
names such as \href{http://www.lammle.com}{www.lammle.com} and device
names such as Workstation 2---to IP addresses, eliminating the need to
know the IP address of a device for connection purposes.

\textbf{Identify what is contained in the TCP header of a
connection-oriented transmission.} The fields in the TCP header include
the source port, destination port, sequence number, acknowledgment
number, header length, a field reserved for future use, code bits,
window size, checksum, urgent pointer, options field, and finally, the
data field.

\textbf{Identify what is contained in the UDP header of a connectionless
transmission.} The fields in the UDP header include only the source
port, destination port, length, checksum, and data. The smaller number
of fields as compared to the TCP header comes at the expense of
providing none of the more advanced functions of the TCP frame.

\textbf{Identify what is contained in the IP header.} The fields of an
IP header include version, header length, priority or type of service,
total length, identification, flags, fragment offset, time to live,
protocol, header checksum, source IP address, destination IP address,
options, and finally, data.

\textbf{Compare and contrast UDP and TCP characteristics and features.}
TCP is connection-oriented, acknowledged, and sequenced and has flow and
error control, while UDP is connectionless, unacknowledged, and not
sequenced and provides no error or flow control.

\textbf{Understand the role of port numbers.} Port numbers are used to
identify the protocol or service that is to be used in the transmission.

\textbf{Identify the role of ICMP.} Internet Control Message Protocol
(ICMP) works at the Network layer and is used by IP for many different
services. ICMP is a management protocol and messaging service provider
for IP.

\textbf{Define the Class A IP address range.} The IP range for a Class A
network is 1--126. This provides 8 bits of network addressing and 24
bits of host addressing by default.

\textbf{Define the Class B IP address range.} The IP range for a Class B
network is 128--191. Class B addressing provides 16 bits of network
addressing and 16 bits of host addressing by default.

\textbf{Define the Class C IP address range.} The IP range for a Class C
network is 192 through 223. Class C addressing provides 24 bits of
network addressing and 8 bits of host addressing by default.

\textbf{Identify the private IP ranges.} The Class A private address
range is 10.0.0.0 through 10.255.255.255. The Class B private address
range is 172.16.0.0 through 172.31.255.255. The Class C private address
range is 192.168.0.0 through 192.168.255.255.

\textbf{Understand the difference between a broadcast, unicast, and
multicast address.} A broadcast is to all devices in a subnet, a unicast
is to one device, and a multicast is to some but not all devices.

\subsection[Written
Labs]{\texorpdfstring{\protect\hypertarget{c03.xhtmlux5cux23c03-sec-18}{}{}\protect\hypertarget{c03.xhtmlux5cux23Page_129}{}{}Written
Labs}{Written Labs}}

In this section, you'll complete the following labs to make sure you've
got the information and concepts contained within them fully dialed in:

\begin{enumerate}
\tightlist
\item
  Lab 3.1: TCP/IP
\item
  Lab 3.2: Mapping Applications to the DoD Model
\end{enumerate}

You can find the answers to these labs in Appendix A, ``Answers to
Written Labs.''

\subsubsection[Written Lab 3.1:
TCP/IP]{\texorpdfstring{\protect\hypertarget{c03.xhtmlux5cux23c03-sec-19}{}{}Written
Lab 3.1: TCP/IP}{Written Lab 3.1: TCP/IP}}

Answer the following questions about TCP/IP:

\begin{enumerate}
\tightlist
\item
  What is the Class C address range in decimal and in binary?
\item
  What layer of the DoD model is equivalent to the Transport layer of
  the OSI model?
\item
  What is the valid range of a Class A network address?
\item
  What is the 127.0.0.1 address used for?
\item
  How do you find the network address from a listed IP address?
\item
  How do you find the broadcast address from a listed IP address?
\item
  What is the Class A private IP address space?
\item
  What is the Class B private IP address space?
\item
  What is the Class C private IP address space?
\item
  What are all the available characters that you can use in hexadecimal
  addressing?
\end{enumerate}

\subsubsection[Written Lab 3.2: Mapping Applications to the DoD
Model]{\texorpdfstring{\protect\hypertarget{c03.xhtmlux5cux23c03-sec-20}{}{}Written
Lab 3.2: Mapping Applications to the DoD
Model}{Written Lab 3.2: Mapping Applications to the DoD Model}}

The four layers of the DoD model are Process/Application, Host-to-Host,
Internet, and Network Access. Identify the layer of the DoD model on
which each of these protocols operates.

\begin{enumerate}
\tightlist
\item
  Internet Protocol (IP)
\item
  Telnet
\item
  FTP
\item
  SNMP
\item
  DNS
\item
  Address Resolution Protocol (ARP)
\item
  DHCP/BootP
\item
  Transmission Control Protocol (TCP)
\item
  X Window
\item
  \protect\hypertarget{c03.xhtmlux5cux23Page_130}{}{}User Datagram
  Protocol (UDP)
\item
  NFS
\item
  Internet Control Message Protocol (ICMP)
\item
  Reverse Address Resolution Protocol (RARP)
\item
  Proxy ARP
\item
  TFTP
\item
  SMTP
\item
  LPD
\end{enumerate}

\subsection[Review
Questions]{\texorpdfstring{\protect\hypertarget{c03.xhtmlux5cux23c03-sec-21}{}{}\protect\hypertarget{c03.xhtmlux5cux23Page_131}{}{}Review
Questions}{Review Questions}}

\begin{center}\rule{0.5\linewidth}{0.5pt}\end{center}

\includegraphics{images/note.png} The following questions are designed
to test your understanding of this chapter's material. For more
information on how to get additional questions, please see
\href{http://www.lammle.com/ccna}{www.lammle.com/ccna}.

\begin{center}\rule{0.5\linewidth}{0.5pt}\end{center}

You can find the answers to these questions in Appendix B, ``Answers to
Review Questions.''

\begin{enumerate}
\item
  What must happen if a DHCP IP conflict occurs?

  \begin{enumerate}
  \def\labelenumii{\Alph{enumii}.}
  \tightlist
  \item
    Proxy ARP will fix the issue.
  \item
    The client uses a gratuitous ARP to fix the issue.
  \item
    The administrator must fix the conflict by hand at the DHCP server.
  \item
    The DHCP server will reassign new IP addresses to both computers.
  \end{enumerate}
\item
  Which of the following Application layer protocols sets up a secure
  session that's similar to Telnet?

  \begin{enumerate}
  \def\labelenumii{\Alph{enumii}.}
  \tightlist
  \item
    FTP
  \item
    SSH
  \item
    DNS
  \item
    DHCP
  \end{enumerate}
\item
  Which of the following mechanisms is used by the client to avoid a
  duplicate IP address during the DHCP process?

  \begin{enumerate}
  \def\labelenumii{\Alph{enumii}.}
  \tightlist
  \item
    Ping
  \item
    Traceroute
  \item
    Gratuitous ARP
  \item
    Pathping
  \end{enumerate}
\item
  What protocol is used to find the hardware address of a local device?

  \begin{enumerate}
  \def\labelenumii{\Alph{enumii}.}
  \tightlist
  \item
    RARP
  \item
    ARP
  \item
    IP
  \item
    ICMP
  \item
    BootP
  \end{enumerate}
\item
  \protect\hypertarget{c03.xhtmlux5cux23Page_132}{}{}Which of the
  following are layers in the TCP/IP model? (Choose three.)

  \begin{enumerate}
  \def\labelenumii{\Alph{enumii}.}
  \tightlist
  \item
    Application
  \item
    Session
  \item
    Transport
  \item
    Internet
  \item
    Data Link
  \item
    Physical
  \end{enumerate}
\item
  Which class of IP address provides a maximum of only 254 host
  addresses per network ID?

  \begin{enumerate}
  \def\labelenumii{\Alph{enumii}.}
  \tightlist
  \item
    Class A
  \item
    Class B
  \item
    Class C
  \item
    Class D
  \item
    Class E
  \end{enumerate}
\item
  Which of the following describe the DHCP Discover message? (Choose
  two.)

  \begin{enumerate}
  \def\labelenumii{\Alph{enumii}.}
  \tightlist
  \item
    It uses ff:ff:ff:ff:ff:ff as a layer 2 broadcast.
  \item
    It uses UDP as the Transport layer protocol.
  \item
    It uses TCP as the Transport layer protocol.
  \item
    It does not use a layer 2 destination address.
  \end{enumerate}
\item
  Which layer 4 protocol is used for a Telnet connection?

  \begin{enumerate}
  \def\labelenumii{\Alph{enumii}.}
  \tightlist
  \item
    IP
  \item
    TCP
  \item
    TCP/IP
  \item
    UDP
  \item
    ICMP
  \end{enumerate}
\item
  Private IP addressing was specified in RFC \_\_\_\_\_\_\_\_\_\_ .
\item
  Which of the following services use TCP? (Choose three.)

  \begin{enumerate}
  \def\labelenumii{\Alph{enumii}.}
  \tightlist
  \item
    DHCP
  \item
    SMTP
  \item
    SNMP
  \item
    FTP
  \item
    HTTP
  \item
    TFTP
  \end{enumerate}
\item
  \protect\hypertarget{c03.xhtmlux5cux23Page_133}{}{} Which Class of IP
  addresses uses the pattern shown here?

  \begin{figure}
  \centering
  \includegraphics{images/c03f025.jpg}
  \caption{}
  \end{figure}

  \begin{enumerate}
  \def\labelenumii{\Alph{enumii}.}
  \tightlist
  \item
    Class A
  \item
    Class B
  \item
    Class C
  \item
    Class D
  \end{enumerate}
\item
  Which of the following is an example of a multicast address?

  \begin{enumerate}
  \def\labelenumii{\Alph{enumii}.}
  \tightlist
  \item
    10.6.9.1
  \item
    192.168.10.6
  \item
    224.0.0.10
  \item
    172.16.9.5
  \end{enumerate}
\item
  The following illustration shows a data structure header. What
  protocol is this header from?

  \begin{figure}
  \centering
  \includegraphics{images/c03f026.jpg}
  \caption{}
  \end{figure}

  \begin{enumerate}
  \def\labelenumii{\Alph{enumii}.}
  \tightlist
  \item
    IP
  \item
    ICMP
  \item
    TCP
  \item
    UDP
  \item
    ARP
  \item
    RARP
  \end{enumerate}
\item
  If you use either Telnet or FTP, what layer are you using to generate
  the data?

  \begin{enumerate}
  \def\labelenumii{\Alph{enumii}.}
  \tightlist
  \item
    Application
  \item
    Presentation
  \item
    Session
  \item
    Transport
  \end{enumerate}
\item
  \protect\hypertarget{c03.xhtmlux5cux23Page_134}{}{}The DoD model (also
  called the TCP/IP stack) has four layers. Which layer of the DoD model
  is equivalent to the Network layer of the OSI model?

  \begin{enumerate}
  \def\labelenumii{\Alph{enumii}.}
  \tightlist
  \item
    Application
  \item
    Host-to-Host
  \item
    Internet
  \item
    Network Access
  \end{enumerate}
\item
  Which two of the following are private IP addresses?

  \begin{enumerate}
  \def\labelenumii{\Alph{enumii}.}
  \tightlist
  \item
    12.0.0.1
  \item
    168.172.19.39
  \item
    172.20.14.36
  \item
    172.33.194.30
  \item
    192.168.24.43
  \end{enumerate}
\item
  What layer in the TCP/IP stack is equivalent to the Transport layer of
  the OSI model?

  \begin{enumerate}
  \def\labelenumii{\Alph{enumii}.}
  \tightlist
  \item
    Application
  \item
    Host-to-Host
  \item
    Internet
  \item
    Network Access
  \end{enumerate}
\item
  Which statements are true regarding ICMP packets? (Choose two.)

  \begin{enumerate}
  \def\labelenumii{\Alph{enumii}.}
  \tightlist
  \item
    ICMP guarantees datagram delivery.
  \item
    ICMP can provide hosts with information about network problems.
  \item
    ICMP is encapsulated within IP datagrams.
  \item
    ICMP is encapsulated within UDP datagrams.
  \end{enumerate}
\item
  What is the address range of a Class B network address in binary?

  \begin{enumerate}
  \def\labelenumii{\Alph{enumii}.}
  \tightlist
  \item
    01\emph{xxxxxx}
  \item
    0\emph{xxxxxxx}
  \item
    10\emph{xxxxxx}
  \item
    110\emph{xxxxx}
  \end{enumerate}
\item
  Drag the steps in the DHCP process and place them in the correct order
  on the right.

  \begin{longtable}[]{@{}ll@{}}
  \toprule
  \endhead
  DHCPOffer & Drop Target A\tabularnewline
  DHCPDiscover & Drop Target B\tabularnewline
  DHCPAck & Drop Target C\tabularnewline
  DHCPRequest & Drop Target D\tabularnewline
  \bottomrule
  \end{longtable}
\end{enumerate}

\protect\hypertarget{c04.xhtml}{}{}

\section[{Chapter 4}\\
{Easy
Subnetting}]{\texorpdfstring{\protect\hypertarget{c04.xhtmlux5cux23c04}{}{}\protect\hypertarget{c04.xhtmlux5cux23Page_135}{}{}{Chapter
4}\\
{Easy Subnetting}}{Chapter 4 Easy Subnetting}}

\begin{center}\rule{0.5\linewidth}{0.5pt}\end{center}

\subsection{THE FOLLOWING ICND1 EXAM TOPICS ARE COVERED IN THIS
CHAPTER:}

\begin{enumerate}
\tightlist
\item
  \includegraphics{images/right.png} \textbf{Network Fundamentals}

  \begin{enumerate}
  \tightlist
  \item
    \includegraphics{images/squ.png} 1.8 Configure, verify, and
    troubleshoot IPv4 addressing and subnetting
  \end{enumerate}
\end{enumerate}

\protect\hypertarget{c04.xhtmlux5cux23Page_136}{}{}\includegraphics{images/intro.png}
We'll pick up right where we left off in the last chapter and continue
to explore the world of IP addressing. I'll open this chapter by telling
you how to subnet an IP network---an indispensably crucial skill that's
central to mastering networking in general! Forewarned is forearmed, so
prepare yourself because being able to subnet quickly and accurately is
pretty challenging and you'll need time to practice what you've learned
to really nail it. So be patient and don't give up on this key aspect of
networking until your skills are seriously sharp. I'm not kidding---this
chapter is so important you should really just graft it into your brain!

So be ready because we're going to hit the ground running and thoroughly
cover IP subnetting from the very start. And though I know this will
sound weird to you, you'll be much better off if you just try to forget
everything you've learned about subnetting before reading this
chapter---especially if you've been to an official Cisco or Microsoft
class! I think these forms of special torture often do more harm than
good and sometimes even scare people away from networking completely.
Those that survive and persevere usually at least question the sanity of
continuing to study in this field. If this is you, relax, breathe, and
know that you'll find that the way I tackle the issue of subnetting is
relatively painless because I'm going to show you a whole new, much
easier method to conquer this monster!

After working through this chapter, and I can't say this enough, after
working through the extra study material at the end as well, you'll be
able to tame the IP addressing/subnetting beast---just don't give up! I
promise that you'll be really glad you didn't. It's one of those things
that once you get it down, you'll wonder why you used to think it was so
hard!

\begin{center}\rule{0.5\linewidth}{0.5pt}\end{center}

\includegraphics{images/note.png} To find up-to-the minute updates for
this chapter, please see
\href{http://www.lammle.com/ccna}{www.lammle.com/ccna} or the book's web
page at \href{http://www.sybex.com/go/ccna}{www.sybex.com/go/ccna}.

\begin{center}\rule{0.5\linewidth}{0.5pt}\end{center}

\subsection[Subnetting
Basics]{\texorpdfstring{\protect\hypertarget{c04.xhtmlux5cux23c04-sec-1}{}{}Subnetting
Basics}{Subnetting Basics}}

In Chapter 3, ``Introduction to TCP/IP,'' you learned how to define and
find the valid host ranges used in a Class A, Class B, and Class C
network address by turning the host bits all off and then all on. This
is very good, but here's the catch: you were defining only one network,
as shown in \protect\hyperlink{c04.xhtmlux5cux23figure04-1}{Figure 4.1}.

\protect\hypertarget{c04.xhtmlux5cux23Page_137}{}{}

\begin{figure}
\centering
\includegraphics{images/c04f001.jpg}
\caption{{\protect\hyperlink{c04.xhtmlux5cux23figureanchor04-1}{\textbf{FIGURE
4.1}} One network}}
\end{figure}

By now you know that having one large network is not a good thing
because the first three chapters you just read were veritably peppered
with me incessantly telling you that! But how would you fix the
out-of-control problem that
\protect\hyperlink{c04.xhtmlux5cux23figure04-1}{Figure 4.1} illustrates?
Wouldn't it be nice to be able to break up that one, huge network
address and create four manageable networks from it? You betcha it
would, but to make that happen, you would need to apply the infamous
trick of \emph{subnetting} because it's the best way to break up a giant
network into a bunch of smaller ones. Take a look at
\protect\hyperlink{c04.xhtmlux5cux23figure04-2}{Figure 4.2} and see how
this might look.

\begin{figure}
\centering
\includegraphics{images/c04f002.jpg}
\caption{{\protect\hyperlink{c04.xhtmlux5cux23figureanchor04-2}{\textbf{FIGURE
4.2}} Multiple networks connected together}}
\end{figure}

What are those 192.168.10.\emph{x} addresses shown in the figure? Well
that is what this chapter will explain---how to make one network into
many networks!

Let's take off from where we left in Chapter 3 and start working in the
host section (host bits) of a network address, where we can borrow bits
to create subnets.

\subsubsection[How to Create
Subnets]{\texorpdfstring{\protect\hypertarget{c04.xhtmlux5cux23c04-sec-2}{}{}\protect\hypertarget{c04.xhtmlux5cux23Page_138}{}{}How
to Create Subnets}{How to Create Subnets}}

Creating subnetworks is essentially the act of taking bits from the host
portion of the address and reserving them to define the subnet address
instead. Clearly this will result in fewer bits being available for
defining your hosts, which is something you'll always want to keep in
mind.

Later in this chapter, I'll guide you through the entire process of
creating subnets starting with Class C addresses. As always in
networking, before you actually implement anything, including
subnetting, you must first determine your current requirements and make
sure to plan for future conditions as well.

\begin{center}\rule{0.5\linewidth}{0.5pt}\end{center}

\includegraphics{images/note.png} In this first section, we'll be
discussing classful routing, which refers to the fact that all hosts
(nodes) in the network are using the exact same subnet mask. Later, when
we move on to cover variable length subnet masks (VLSMs), I'll tell you
all about classless routing, which is an environment wherein each
network segment \emph{can} use a different subnet mask.

\begin{center}\rule{0.5\linewidth}{0.5pt}\end{center}

To create a subnet, we'll start by fulfilling these three steps:

\begin{enumerate}
\tightlist
\item
  Determine the number of required network IDs:

  \begin{enumerate}
  \tightlist
  \item
    One for each LAN subnet
  \item
    One for each wide area network connection
  \end{enumerate}
\item
  Determine the number of required host IDs per subnet:

  \begin{enumerate}
  \tightlist
  \item
    One for each TCP/IP host
  \item
    One for each router interface
  \end{enumerate}
\item
  Based on the previous requirements, create the following:

  \begin{enumerate}
  \tightlist
  \item
    A unique subnet mask for your entire network
  \item
    A unique subnet ID for each physical segment
  \item
    A range of host IDs for each subnet
  \end{enumerate}
\end{enumerate}

\subsubsection[Subnet
Masks]{\texorpdfstring{\protect\hypertarget{c04.xhtmlux5cux23c04-sec-3}{}{}Subnet
Masks}{Subnet Masks}}

For the subnet address scheme to work, every machine on the network must
know which part of the host address will be used as the subnet address.
This condition is met by assigning a \emph{subnet mask} to each machine.
A subnet mask is a 32-bit value that allows the device that's receiving
IP packets to distinguish the network ID portion of the IP address from
the host ID portion of the IP address. This 32-bit subnet mask is
composed of 1s and 0s, where the 1s represent the positions that refer
to the network subnet addresses.

Not all networks need subnets, and if not, it really means that they're
using the default subnet mask, which is basically the same as saying
that a network doesn't have a subnet address.
\protect\hyperlink{c04.xhtmlux5cux23table04-1}{Table 4.1} shows the
default subnet masks for Classes A, B, and C.

\protect\hypertarget{c04.xhtmlux5cux23Page_139}{}{}

{\protect\hyperlink{c04.xhtmlux5cux23tableanchor04-1}{Table 4.1} Default
subnet mask}

\begin{longtable}[]{@{}lll@{}}
\toprule
Class & Format & Default Subnet Mask\tabularnewline
\midrule
\endhead
A & \emph{network.node.node.node} & 255.0.0.0\tabularnewline
B & \emph{network.network.node.node} & 255.255.0.0\tabularnewline
C & \emph{network.network.network.node} & 255.255.255.0\tabularnewline
\bottomrule
\end{longtable}

Although you can use any mask in any way on an interface, typically it's
not usually good to mess with the default masks. In other words, you
don't want to make a Class B subnet mask read 255.0.0.0, and some hosts
won't even let you type it in. But these days, most devices will. For a
Class A network, you wouldn't change the first byte in a subnet mask
because it should read 255.0.0.0 at a minimum. Similarly, you wouldn't
assign 255.255.255.255 because this is all 1s, which is a broadcast
address. A Class B address starts with 255.255.0.0, and a Class C starts
with 255.255.255.0, and for the CCNA especially, there is no reason to
change the defaults!

\begin{center}\rule{0.5\linewidth}{0.5pt}\end{center}

\subsubsection{Understanding the Powers of 2}

Powers of 2 are important to understand and memorize for use with IP
subnetting. Reviewing powers of 2, remember that when you see a number
noted with an exponent, it means you should multiply the number by
itself as many times as the upper number specifies. For example,
2\textsuperscript{3} is 2 x 2 x 2, which equals 8. Here's a list of
powers of 2 to commit to memory:

2\textsuperscript{1} = 2

2\textsuperscript{2} = 4

2\textsuperscript{3} = 8

2\textsuperscript{4} = 16

2\textsuperscript{5} = 32

2\textsuperscript{6} = 64

2\textsuperscript{7} = 128

2\textsuperscript{8} = 256

2\textsuperscript{9} = 512

2\textsuperscript{10} = 1,024

\protect\hypertarget{c04.xhtmlux5cux23Page_140}{}{}2\textsuperscript{11}
= 2,048

2\textsuperscript{12} = 4,096

2\textsuperscript{13} = 8,192

2\textsuperscript{14} = 16,384

Memorizing these powers of 2 is a good idea, but it's not absolutely
necessary. Just remember that since you're working with powers of 2,
each successive power of 2 is double the previous one.

It works like this---all you have to do to remember the value of
2\textsuperscript{9} is to first know that 2\textsuperscript{8} = 256.
Why? Because when you double 2 to the eighth power (256), you get
2\textsuperscript{9} (or 512). To determine the value of
2\textsuperscript{10}, simply start at 2\textsuperscript{8} = 256, and
then double it twice.

You can go the other way as well. If you needed to know what
2\textsuperscript{6} is, for example, you just cut 256 in half two
times: once to reach 2\textsuperscript{7} and then one more time to
reach 2\textsuperscript{6}.

\begin{center}\rule{0.5\linewidth}{0.5pt}\end{center}

\subsubsection[Classless Inter-Domain Routing
(CIDR)]{\texorpdfstring{\protect\hypertarget{c04.xhtmlux5cux23c04-sec-4}{}{}Classless
Inter-Domain Routing (CIDR)}{Classless Inter-Domain Routing (CIDR)}}

Another term you need to familiarize yourself with is \emph{Classless
Inter-Domain Routing (CIDR)}. It's basically the method that Internet
service providers (ISPs) use to allocate a number of addresses to a
company, a home---their customers. They provide addresses in a certain
block size, something I'll talk about in greater detail soon.

When you receive a block of addresses from an ISP, what you get will
look something like this: 192.168.10.32/28. This is telling you what
your subnet mask is. The slash notation (/) means how many bits are
turned on (1s). Obviously, the maximum could only be /32 because a byte
is 8 bits and there are 4 bytes in an IP address: (4 × 8 = 32). But keep
in mind that regardless of the class of address, the largest subnet mask
available relevant to the Cisco exam objectives can only be a /30
because you've got to keep at least 2 bits for host bits.

Take, for example, a Class A default subnet mask, which is 255.0.0.0.
This tells us that the first byte of the subnet mask is all ones (1s),
or 11111111. When referring to a slash notation, you need to count all
the 1 bits to figure out your mask. The 255.0.0.0 is considered a /8
because it has 8 bits that are 1s---that is, 8 bits that are turned on.

A Class B default mask would be 255.255.0.0, which is a /16 because 16
bits are ones (1s): 11111111.11111111.00000000.00000000.

\protect\hyperlink{c04.xhtmlux5cux23table04-2}{Table 4.2} has a listing
of every available subnet mask and its equivalent CIDR slash notation.

{\protect\hyperlink{c04.xhtmlux5cux23tableanchor04-2}{Table 4.2} CIDR
values}

\begin{longtable}[]{@{}ll@{}}
\toprule
Subnet Mask & CIDR Value\tabularnewline
\midrule
\endhead
255.0.0.0 & /8\tabularnewline
255.128.0.0 & /9\tabularnewline
\protect\hypertarget{c04.xhtmlux5cux23Page_141}{}{}255.192.0.0 &
/10\tabularnewline
255.224.0.0 & /11\tabularnewline
255.240.0.0 & /12\tabularnewline
255.248.0.0 & /13\tabularnewline
255.252.0.0 & /14\tabularnewline
255.254.0.0 & /15\tabularnewline
255.255.0.0 & /16\tabularnewline
255.255.128.0 & /17\tabularnewline
255.255.192.0 & /18\tabularnewline
255.255.224.0 & /19\tabularnewline
255.255.240.0 & /20\tabularnewline
255.255.248.0 & /21\tabularnewline
255.255.252.0 & /22\tabularnewline
255.255.254.0 & /23\tabularnewline
255.255.255.0 & /24\tabularnewline
255.255.255.128 & /25\tabularnewline
255.255.255.192 & /26\tabularnewline
255.255.255.224 & /27\tabularnewline
255.255.255.240 & /28\tabularnewline
255.255.255.248 & /29\tabularnewline
255.255.255.252 & /30\tabularnewline
\bottomrule
\end{longtable}

\protect\hypertarget{c04.xhtmlux5cux23Page_142}{}{}The /8 through /15
can only be used with Class A network addresses. /16 through /23 can be
used by Class A and B network addresses. /24 through /30 can be used by
Class A, B, and C network addresses. This is a big reason why most
companies use Class A network addresses. Since they can use all subnet
masks, they get the maximum flexibility in network design.

\begin{center}\rule{0.5\linewidth}{0.5pt}\end{center}

\includegraphics{images/note.png} No, you cannot configure a Cisco
router using this slash format. But wouldn't that be nice? Nevertheless,
it's \emph{really} important for you to know subnet masks in the slash
notation (CIDR).

\begin{center}\rule{0.5\linewidth}{0.5pt}\end{center}

\subsubsection[\emph{IP
Subnet-Zero}]{\texorpdfstring{\protect\hypertarget{c04.xhtmlux5cux23c04-sec-5}{}{}\emph{IP
Subnet-Zero}}{IP Subnet-Zero}}

Even though \texttt{ip\ subnet-zero} is not a new command, Cisco
courseware and Cisco exam objectives didn't used to cover it. Know that
Cisco certainly covers it now! This command allows you to use the first
and last subnet in your network design. For instance, the Class C mask
of 255.255.255.192 provides subnets 64 and 128, another facet of
subnetting that we'll discuss more thoroughly later in this chapter. But
with the \texttt{ip\ subnet-zero} command, you now get to use subnets 0,
64, 128, and 192. It may not seem like a lot, but this provides two more
subnets for every subnet mask we use.

Even though we don't discuss the command-line interface (CLI) until
Chapter 6, ``Cisco's Internetworking Operating System (IOS),'' it's
important for you to be at least a little familiar with this command at
this point:

\begin{verbatim}
Router#sh running-config
Building configuration...
Current configuration : 827 bytes
!
hostname Pod1R1
!
ip subnet-zero
!
\end{verbatim}

This router output shows that the command \texttt{ip\ subnet-zero} is
enabled on the router. Cisco has turned this command on by default
starting with Cisco IOS version 12.\emph{x} and now we're running
15.\emph{x} code.

When taking your Cisco exams, make sure you read very carefully to see
if Cisco is asking you \emph{not} to use \texttt{ip\ subnet-zero}. There
are actually instances where this may happen.

\subsubsection[Subnetting Class C
Addresses]{\texorpdfstring{\protect\hypertarget{c04.xhtmlux5cux23c04-sec-6}{}{}Subnetting
Class C Addresses}{Subnetting Class C Addresses}}

There are many different ways to subnet a network. The right way is the
way that works best for you. In a Class C address, only 8 bits are
available for defining the hosts. Remember that subnet bits start at the
left and move to the right, without skipping bits. This means that the
only Class C subnet masks can be the following:

\begin{verbatim}
Binary     Decimal  CIDR
---------------------------------------------------------
00000000 = 255.255.255.0        /24
10000000 = 255.255.255.128      /25
11000000 = 255.255.255.192      /26
11100000 = 255.255.255.224      /27
11110000 = 255.255.255.240      /28
11111000 = 255.255.255.248      /29
11111100 = 255.255.255.252      /30
\end{verbatim}

We can't use a /31 or /32 because, as I've said, we must have at least 2
host bits for assigning IP addresses to hosts. But this is only mostly
true. Certainly we can never use a /32 because that would mean zero host
bits available, yet Cisco has various forms of the IOS, as well as the
new Cisco Nexus switches operating system, that support the /31 mask.
The /31 is above the scope of the CCENT and CCNA objectives, so we won't
be covering it in this book.

Coming up, I'm going to teach you that significantly less painful method
of subnetting I promised you at the beginning of this chapter, which
makes it ever so much easier to subnet larger numbers in a flash.
Excited? Good! Because I'm not kidding when I tell you that you
absolutely need to be able to subnet quickly and accurately to succeed
in the networking real world and on the exam too!

\paragraph{Subnetting a Class C Address---The Fast Way!}

When you've chosen a possible subnet mask for your network and need to
determine the number of subnets, valid hosts, and the broadcast
addresses of a subnet that mask will provide, all you need to do is
answer five simple questions:

\begin{enumerate}
\tightlist
\item
  How many subnets does the chosen subnet mask produce?
\item
  How many valid hosts per subnet are available?
\item
  What are the valid subnets?
\item
  What's the broadcast address of each subnet?
\item
  What are the valid hosts in each subnet?
\end{enumerate}

This is where you'll be really glad you followed my advice and took the
time to memorize your powers of 2. If you didn't, now would be a good
time\ldots{} Just refer back to the sidebar ``Understanding the Powers
of 2'' earlier if you need to brush up. Here's how you arrive at the
answers to those five big questions:

\begin{enumerate}
\tightlist
\item
  \emph{How many subnets?} 2\textsuperscript{\emph{x}} = number of
  subnets. \emph{x} is the number of masked bits, or the 1s. For
  example, in 11000000, the number of 1s gives us 2\textsuperscript{2}
  subnets. So in this example, there are 4 subnets.
\item
  \emph{How many hosts per subnet?} 2\textsuperscript{\emph{y}} -- 2 =
  number of hosts per subnet. \emph{y} is the number of unmasked bits,
  or the 0s. For example, in 11000000, the number of 0s gives us
  2\textsuperscript{6} -- 2 hosts, or 62 hosts per subnet. You need to
  subtract 2 for the subnet address and the broadcast address, which are
  not valid hosts.
\item
  \emph{What are the valid subnets?} 256 -- subnet mask = block size, or
  increment number. An example would be the 255.255.255.192 mask, where
  the interesting octet is the fourth octet (interesting because that is
  where our subnet numbers are). Just use this math: 256 -- 192 = 64.
  The block size of a 192 mask is always 64. Start counting at zero in
  blocks of 64 until you reach the subnet mask value and these are your
  subnets in the fourth octet: 0, 64, 128, 192. Easy, huh?
\item
  \protect\hypertarget{c04.xhtmlux5cux23Page_144}{}{}\emph{What's the
  broadcast address for each subnet?} Now here's the really easy part.
  Since we counted our subnets in the last section as 0, 64, 128, and
  192, the broadcast address is always the number right before the next
  subnet. For example, the 0 subnet has a broadcast address of 63
  because the next subnet is 64. The 64 subnet has a broadcast address
  of 127 because the next subnet is 128, and so on. Remember, the
  broadcast address of the last subnet is always 255.
\item
  \emph{What are the valid hosts?} Valid hosts are the numbers between
  the subnets, omitting the all-0s and all-1s. For example, if 64 is the
  subnet number and 127 is the broadcast address, then 65--126 is the
  valid host range. Your valid range is \emph{always} the group of
  numbers between the subnet address and the broadcast address.
\end{enumerate}

If you're still confused, don't worry because it really isn't as hard as
it seems to be at first---just hang in there! To help lift any mental
fog, try a few of the practice examples next.

\paragraph{Subnetting Practice Examples: Class C Addresses}

Here's your opportunity to practice subnetting Class C addresses using
the method I just described. This is so cool. We're going to start with
the first Class C subnet mask and work through every subnet that we can,
using a Class C address. When we're done, I'll show you how easy this is
with Class A and B networks too!

\subparagraph{Practice Example \#1C: 255.255.255.128 (/25)}

Since 128 is 10000000 in binary, there is only 1 bit for subnetting and
7 bits for hosts. We're going to subnet the Class C network address
192.168.10.0.

192.168.10.0 = Network address

255.255.255.128 = Subnet mask

Now, let's answer our big five:

\begin{enumerate}
\tightlist
\item
  \emph{How many subnets?} Since 128 is 1 bit on (\textbf{1}0000000),
  the answer would be 2\textsuperscript{1} = 2.
\item
  \emph{How many hosts per subnet?} We have 7 host bits off
  (1\textbf{0000000}), so the equation would be 2\textsuperscript{7} --
  2 = 126 hosts. Once you figure out the block size of a mask, the
  amount of hosts is always the block size minus 2. No need to do extra
  math if you don't need to!
\item
  \emph{What are the valid subnets?} 256 -- 128 = 128. Remember, we'll
  start at zero and count in our block size, so our subnets are 0, 128.
  By just counting your subnets when counting in your block size, you
  really don't need to do steps 1 and 2. We can see we have two subnets,
  and in the step before this one, just remember that the amount of
  hosts is always the block size minus 2, and in this example, that
  gives us 2 subnets, each with 126 hosts.
\item
  \protect\hypertarget{c04.xhtmlux5cux23Page_145}{}{}\emph{What's the
  broadcast address for each subnet?} The number right before the value
  of the next subnet is all host bits turned on and equals the broadcast
  address. For the zero subnet, the next subnet is 128, so the broadcast
  of the 0 subnet is 127.
\item
  \emph{What are the valid hosts?} These are the numbers between the
  subnet and broadcast address. The easiest way to find the hosts is to
  write out the subnet address and the broadcast address, which makes
  valid hosts completely obvious. The following table shows the 0 and
  128 subnets, the valid host ranges of each, and the broadcast address
  of both subnets:
\end{enumerate}

\begin{longtable}[]{@{}lll@{}}
\toprule
Subnet & 0 & 128\tabularnewline
\midrule
\endhead
First host & 1 & 129\tabularnewline
Last host & 126 & 254\tabularnewline
\textbf{Broadcast} & \textbf{127} & \textbf{255}\tabularnewline
\bottomrule
\end{longtable}

Looking at a Class C /25, it's pretty clear that there are two subnets.
But so what---why is this significant? Well actually, it's not because
that's not the right question. What you really want to know is what you
would do with this information!

I know this isn't exactly everyone's favorite pastime, but what we're
about to do is really important, so bear with me; we're going to talk
about subnetting---period. The key to understanding subnetting is to
understand the very reason you need to do it, and I'm going to
demonstrate this by going through the process of building a physical
network.

Okay---because we added that router shown in
\protect\hyperlink{c04.xhtmlux5cux23figure04-3}{Figure 4.3}, in order
for the hosts on our internetwork to communicate, they must now have a
logical network addressing scheme. We could use IPv6, but IPv4 is still
the most popular for now. It's also what we're studying at the moment,
so that's what we're going with.

\begin{figure}
\centering
\includegraphics{images/c04f003.jpg}
\caption{{\protect\hyperlink{c04.xhtmlux5cux23figureanchor04-3}{\textbf{FIGURE
4.3}} Implementing a Class C /25 logical network}}
\end{figure}

Looking at \protect\hyperlink{c04.xhtmlux5cux23figure04-3}{Figure 4.3},
you can see that there are two physical networks, so we're going to
implement a logical addressing scheme that allows for two logical
networks. As always,
\protect\hypertarget{c04.xhtmlux5cux23Page_146}{}{}it's a really good
idea to look ahead and consider likely short- and long-term growth
scenarios, but for this example in this book, a /25 gets it done.

\protect\hyperlink{c04.xhtmlux5cux23figure04-3}{Figure 4.3} shows us
that both subnets have been assigned to a router interface, which
creates our broadcast domains and assigns our subnets. Use the command
\texttt{show\ ip\ route} to see the routing table on a router. Notice
that instead of one large broadcast domain, there are now two smaller
broadcast domains, providing for up to 126 hosts in each. The \texttt{C}
in the router output translates to ``directly connected network,'' and
we can see we have two of those with two broadcast domains and that we
created and implemented them. So congratulations---you did it! You have
successfully subnetted a network and applied it to a network design.
Nice! Let's do it again.

\subparagraph{Practice Example \#2C: 255.255.255.192 (/26)}

This time, we're going to subnet the network address 192.168.10.0 using
the subnet mask 255.255.255.192.

192.168.10.0 = Network address

255.255.255.192 = Subnet mask

Now, let's answer the big five:

\begin{enumerate}
\tightlist
\item
  \emph{How many subnets?} Since 192 is 2 bits on (11000000), the answer
  would be 2\textsuperscript{2} = 4 subnets.
\item
  \emph{How many hosts per subnet?} We have 6 host bits off (11000000),
  giving us 2\textsuperscript{6} -- 2 = 62 hosts. The amount of hosts is
  always the block size minus 2.
\item
  \emph{What are the valid subnets?} 256 -- 192 = 64. Remember to start
  at zero and count in our block size. This means our subnets are 0, 64,
  128, and 192. We can see we have a block size of 64, so we have 4
  subnets, each with 62 hosts.
\item
  \emph{What's the broadcast address for each subnet?} The number right
  before the value of the next subnet is all host bits turned on and
  equals the broadcast address. For the zero subnet, the next subnet is
  64, so the broadcast address for the zero subnet is 63.
\item
  \emph{What are the valid hosts?} These are the numbers between the
  subnet and broadcast address. As I said, the easiest way to find the
  hosts is to write out the subnet address and the broadcast address,
  which clearly delimits our valid hosts. The following table shows the
  0, 64, 128, and 192 subnets, the valid host ranges of each, and the
  broadcast address of each subnet:
\end{enumerate}

\begin{longtable}[]{@{}lllll@{}}
\toprule
The subnets (do this first) & 0 & 64 & 128 & 192\tabularnewline
\midrule
\endhead
Our first host (perform host addressing last) & 1 & 65 & 129 &
193\tabularnewline
Our last host & 62 & 126 & 190 & 254\tabularnewline
The broadcast address (do this second) & 63 & 127 & 191 &
255\tabularnewline
\bottomrule
\end{longtable}

\protect\hypertarget{c04.xhtmlux5cux23Page_147}{}{}Again, before getting
into the next example, you can see that we can now subnet a /26 as long
as we can count in increments of 64. And what are you going to do with
this fascinating information? Implement it! We'll use
\protect\hyperlink{c04.xhtmlux5cux23figure04-4}{Figure 4.4} to practice
a /26 network implementation.

\begin{figure}
\centering
\includegraphics{images/c04f004.jpg}
\caption{{\protect\hyperlink{c04.xhtmlux5cux23figureanchor04-4}{\textbf{FIGURE
4.4}} Implementing a class C /26 (with three networks)}}
\end{figure}

The /26 mask provides four subnetworks, and we need a subnet for each
router interface. With this mask, in this example, we actually have room
with a spare subnet to add to another router interface in the future.
Always plan for growth if possible!

\subparagraph{Practice Example \#3C: 255.255.255.224 (/27)}

This time, we'll subnet the network address 192.168.10.0 and subnet mask
255.255.255.224.

192.168.10.0 = Network address

255.255.255.224 = Subnet mask

\begin{enumerate}
\tightlist
\item
  \emph{How many subnets?} 224 is 11100000, so our equation would be
  2\textsuperscript{3} = 8.
\item
  \emph{How many hosts?} 2\textsuperscript{5} -- 2 = 30.
\item
  \emph{What are the valid subnets?} 256 -- 224 = 32. We just start at
  zero and count to the subnet mask value in blocks (increments) of 32:
  0, 32, 64, 96, 128, 160, 192, and 224.
\item
  \emph{What's the broadcast address for each subnet (always the number
  right before the next subnet)?}
\item
  \emph{What are the valid hosts (the numbers between the subnet number
  and the broadcast address)?}
\end{enumerate}

\protect\hypertarget{c04.xhtmlux5cux23Page_148}{}{}To answer the last
two questions, first just write out the subnets, then write out the
broadcast addresses---the number right before the next subnet. Last,
fill in the host addresses. The following table gives you all the
subnets for the 255.255.255.224 Class C subnet mask:

\begin{longtable}[]{@{}lllllllll@{}}
\toprule
The subnet address & 0 & 32 & 64 & 96 & 128 & 160 & 192 &
224\tabularnewline
\midrule
\endhead
The first valid host & 1 & 33 & 65 & 97 & 129 & 161 & 193 &
225\tabularnewline
The last valid host & 30 & 62 & 94 & 126 & 158 & 190 & 222 &
254\tabularnewline
The broadcast address & 31 & 63 & 95 & 127 & 159 & 191 & 223 &
255\tabularnewline
\bottomrule
\end{longtable}

In practice example \#3C, we're using a 255.255.255.224 (/27) network,
which provides eight subnets as shown previously. We can take these
subnets and implement them as shown in
\protect\hyperlink{c04.xhtmlux5cux23figure04-5}{Figure 4.5} using any of
the subnets available.

\begin{figure}
\centering
\includegraphics{images/c04f005.jpg}
\caption{{\protect\hyperlink{c04.xhtmlux5cux23figureanchor04-5}{\textbf{FIGURE
4.5}} Implementing a Class C /27 logical network}}
\end{figure}

Notice that used six of the eight subnets available for my network
design. The lightning bolt symbol in the figure represents a wide area
network (WAN) such as a T1 or other serial connection through an ISP or
telco. In other words, something you don't own, but it's still a subnet
just like any LAN connection on a router. As usual, I used the first
valid host in each subnet as the router's interface address. This is
just a rule of thumb; you can use any address in the valid host range as
long as you remember what address you configured so you can set the
default gateways on your hosts to the router address.

\subparagraph{Practice Example \#4C: 255.255.255.240 (/28)}

Let's practice another one:

192.168.10.0 = Network address

255.255.255.240 = Subnet mask

\begin{enumerate}
\tightlist
\item
  \protect\hypertarget{c04.xhtmlux5cux23Page_149}{}{}\emph{Subnets?} 240
  is 11110000 in binary. 2\textsuperscript{4} = 16.
\item
  \emph{Hosts?} 4 host bits, or 2\textsuperscript{4} -- 2 = 14.
\item
  \emph{Valid subnets?} 256 -- 240 = 16. Start at 0: 0 + 16 = 16. 16 +
  16 = 32. 32 + 16 = 48. 48 + 16 = 64. 64 + 16 = 80. 80 + 16 = 96. 96 +
  16 = 112. 112 + 16 = 128. 128 + 16 = 144. 144 + 16 = 160. 160 + 16 =
  176. 176 + 16 = 192. 192 + 16 = 208. 208 + 16 = 224. 224 + 16 = 240.
\item
  \emph{Broadcast address for each subnet?}
\item
  \emph{Valid hosts?}
\end{enumerate}

To answer the last two questions, check out the following table. It
gives you the subnets, valid hosts, and broadcast addresses for each
subnet. First, find the address of each subnet using the block size
(increment). Second, find the broadcast address of each subnet
increment, which is always the number right before the next valid
subnet, and then just fill in the host addresses. The following table
shows the available subnets, hosts, and broadcast addresses provided
from a Class C 255.255.255.240 mask.

\begin{figure}
\centering
\includegraphics{images/c04f006.jpg}
\caption{}
\end{figure}

\begin{center}\rule{0.5\linewidth}{0.5pt}\end{center}

\includegraphics{images/tip.png} Cisco has figured out that most people
cannot count in 16s and therefore have a hard time finding valid
subnets, hosts, and broadcast addresses with the Class C 255.255.255.240
mask. You'd be wise to study this mask.

\begin{center}\rule{0.5\linewidth}{0.5pt}\end{center}

\subparagraph{Practice Example \#5C: 255.255.255.248 (/29)}

Let's keep practicing:

192.168.10.0 = Network address

255.255.255.248 = Subnet mask

\begin{enumerate}
\tightlist
\item
  \emph{Subnets?} 248 in binary = 11111000. 2\textsuperscript{5} = 32.
\item
  \emph{Hosts?} 2\textsuperscript{3} -- 2 = 6.
\item
  \emph{Valid subnets?} 256 -- 248 = 0, 8, 16, 24, 32, 40, 48, 56, 64,
  72, 80, 88, 96, 104, 112, 120, 128, 136, 144, 152, 160, 168, 176, 184,
  192, 200, 208, 216, 224, 232, 240, and 248.
\item
  \emph{Broadcast address for each subnet?}
\item
  \emph{Valid hosts?}
\end{enumerate}

\protect\hypertarget{c04.xhtmlux5cux23Page_150}{}{}Take a look at the
following table. It shows some of the subnets (first four and last four
only), valid hosts, and broadcast addresses for the Class C
255.255.255.248 mask:

\begin{longtable}[]{@{}llllllllll@{}}
\toprule
Subnet & 0 & 8 & 16 & 24 & \ldots{} & 224 & 232 & 240 &
248\tabularnewline
\midrule
\endhead
First host & 1 & 9 & 17 & 25 & \ldots{} & 225 & 233 & 241 &
249\tabularnewline
Last host & 6 & 14 & 22 & 30 & \ldots{} & 230 & 238 & 246 &
254\tabularnewline
Broadcast & 7 & 15 & 23 & 31 & \ldots{} & 231 & 239 & 247 &
255\tabularnewline
\bottomrule
\end{longtable}

\begin{center}\rule{0.5\linewidth}{0.5pt}\end{center}

\includegraphics{images/tip.png} If you try to configure a router
interface with the address 192.168.10.6 255.255.255.248 and receive the
following error, It means that \texttt{ip\ subnet-zero} is not enabled:

\begin{verbatim}
Bad mask /29 for address 192.168.10.6
\end{verbatim}

You must be able to subnet to see that the address used in this example
is in the zero subnet!

\begin{center}\rule{0.5\linewidth}{0.5pt}\end{center}

\subparagraph{Practice Example \#6C: 255.255.255.252 (/30)}

Okay---just one more:

192.168.10.0 = Network address

255.255.255.252 = Subnet mask

\begin{enumerate}
\tightlist
\item
  \emph{Subnets?} 64.
\item
  \emph{Hosts?} 2.
\item
  \emph{Valid subnets?} 0, 4, 8, 12, etc., all the way to 252.
\item
  \emph{Broadcast address for each subnet? (Always the number right
  before the next subnet.)}
\item
  \emph{Valid hosts?} (The numbers between the subnet number and the
  broadcast address.)
\end{enumerate}

The following table shows you the subnet, valid host, and broadcast
address of the first four and last four subnets in the 255.255.255.252
Class C subnet:

\begin{longtable}[]{@{}llllllllll@{}}
\toprule
Subnet & 0 & 4 & 8 & 12 & \ldots{} & 240 & 244 & 248 &
252\tabularnewline
\midrule
\endhead
First host & 1 & 5 & 9 & 13 & \ldots{} & 241 & 245 & 249 &
253\tabularnewline
Last host & 2 & 6 & 10 & 14 & \ldots{} & 242 & 246 & 250 &
254\tabularnewline
Broadcast & 3 & 7 & 11 & 15 & \ldots{} & 243 & 247 & 251 &
255\tabularnewline
\bottomrule
\end{longtable}

\protect\hypertarget{c04.xhtmlux5cux23Page_151}{}{}

\begin{center}\rule{0.5\linewidth}{0.5pt}\end{center}

\subsubsection[\hfill\break
Should We Really Use This Mask That Provides Only Two
Hosts?]{\texorpdfstring{\protect\includegraphics{images/earth.png}\\
Should We Really Use This Mask That Provides Only Two
Hosts?}{ Should We Really Use This Mask That Provides Only Two Hosts?}}

You are the network administrator for Acme Corporation in San Francisco,
with dozens of WAN links connecting to your corporate office. Right now
your network is a classful network, which means that the same subnet
mask is on each host and router interface. You've read about classless
routing, where you can have different sized masks, but don't know what
to use on your point-to-point WAN links. Is the 255.255.255.252 (/30) a
helpful mask in this situation?

Yes, this is a very helpful mask in wide area networks and of course
with any type of point-to-point link!

If you were to use the 255.255.255.0 mask in this situation, then each
network would have 254 hosts. But you use only 2 addresses with a WAN or
point-to-point link, which is a waste of 252 hosts per subnet! If you
use the 255.255.255.252 mask, then each subnet has only 2 hosts, and you
don't want to waste precious addresses. This is a really important
subject, one that we'll address in a lot more detail in the section on
VLSM network design in the next chapter!

\begin{center}\rule{0.5\linewidth}{0.5pt}\end{center}

\paragraph{Subnetting in Your Head: Class C Addresses}

It really is possible to subnet in your head? Yes, and it's not all that
hard either---take the following example:

192.168.10.50 = Node address

255.255.255.224 = Subnet mask

First, determine the subnet and broadcast address of the network in
which the previous IP address resides. You can do this by answering
question 3 of the big 5 questions: 256 -- 224 = 32. 0, 32, 64, and so
on. The address of 50 falls between the two subnets of 32 and 64 and
must be part of the 192.168.10.32 subnet. The next subnet is 64, so the
broadcast address of the 32 subnet is 63. Don't forget that the
broadcast address of a subnet is always the number right before the next
subnet. The valid host range equals the numbers between the subnet and
broadcast address, or 33--62. This is too easy!

Let's try another one. We'll subnet another Class C address:

192.168.10.50 = Node address

255.255.255.240 = Subnet mask

What is the subnet and broadcast address of the network of which the
previous IP address is a member? 256 -- 240 = 16. Now just count by our
increments of 16 until we
\protect\hypertarget{c04.xhtmlux5cux23Page_152}{}{}pass the host
address: 0, 16, 32, 48, 64. Bingo---the host address is between the 48
and 64 subnets. The subnet is 192.168.10.48, and the broadcast address
is 63 because the next subnet is 64. The valid host range equals the
numbers between the subnet number and the broadcast address, or 49--62.

Let's do a couple more to make sure you have this down.

You have a node address of 192.168.10.174 with a mask of
255.255.255.240. What is the valid host range?

The mask is 240, so we'd do a 256 -- 240 = 16. This is our block size.
Just keep adding 16 until we pass the host address of 174, starting at
zero, of course: 0, 16, 32, 48, 64, 80, 96, 112, 128, 144, 160, 176. The
host address of 174 is between 160 and 176, so the subnet is 160. The
broadcast address is 175; the valid host range is 161--174. That was a
tough one!

One more---just for fun. This one is the easiest of all Class C
subnetting:

192.168.10.17 = Node address

255.255.255.252 = Subnet mask

What is the subnet and broadcast address of the subnet in which the
previous IP address resides? 256 -- 252 = 0 (always start at zero unless
told otherwise). 0, 4, 8, 12, 16, 20, etc. You've got it! The host
address is between the 16 and 20 subnets. The subnet is 192.168.10.16,
and the broadcast address is 19. The valid host range is 17--18.

Now that you're all over Class C subnetting, let's move on to Class B
subnetting. But before we do, let's go through a quick review.

\paragraph{What Do We Know?}

Okay---here's where you can really apply what you've learned so far and
begin committing it all to memory. This is a very cool section that I've
been using in my classes for years. It will really help you nail down
subnetting for good!

When you see a subnet mask or slash notation (CIDR), you should know the
following:

\textbf{/25} What do we know about a /25?

\begin{enumerate}
\tightlist
\item
  128 mask
\item
  1 bit on and 7 bits off (10000000)
\item
  Block size of 128
\item
  Subnets 0 and 128
\item
  2 subnets, each with 126 hosts
\end{enumerate}

\textbf{/26} What do we know about a /26?

\begin{enumerate}
\tightlist
\item
  192 mask
\item
  2 bits on and 6 bits off (11000000)
\item
  \protect\hypertarget{c04.xhtmlux5cux23Page_153}{}{}Block size of 64
\item
  Subnets 0, 64, 128, 192
\item
  4 subnets, each with 62 hosts
\end{enumerate}

\textbf{/27} What do we know about a /27?

\begin{enumerate}
\tightlist
\item
  224 mask
\item
  3 bits on and 5 bits off (11100000)
\item
  Block size of 32
\item
  Subnets 0, 32, 64, 96, 128, 160, 192, 224
\item
  8 subnets, each with 30 hosts
\end{enumerate}

\textbf{/28} What do we know about a /28?

\begin{enumerate}
\tightlist
\item
  240 mask
\item
  4 bits on and 4 bits off
\item
  Block size of 16
\item
  Subnets 0, 16, 32, 48, 64, 80, 96, 112, 128, 144, 160, 176, 192, 208,
  224, 240
\item
  16 subnets, each with 14 hosts
\end{enumerate}

\textbf{/29} What do we know about a /29?

\begin{enumerate}
\tightlist
\item
  248 mask
\item
  5 bits on and 3 bits off
\item
  Block size of 8
\item
  Subnets 0, 8, 16, 24, 32, 40, 48, etc.
\item
  32 subnets, each with 6 hosts
\end{enumerate}

\textbf{/30} What do we know about a /30?

\begin{enumerate}
\tightlist
\item
  252 mask
\item
  6 bits on and 2 bits off
\item
  Block size of 4
\item
  Subnets 0, 4, 8, 12, 16, 20, 24, etc.
\item
  64 subnets, each with 2 hosts
\end{enumerate}

\protect\hyperlink{c04.xhtmlux5cux23table04-3}{Table 4.3} puts all of
the previous information into one compact little table. You should
practice writing this table out on scratch paper, and if you can do it,
write it down before you start your exam!

\protect\hypertarget{c04.xhtmlux5cux23Page_154}{}{}

{\protect\hyperlink{c04.xhtmlux5cux23tableanchor04-3}{Table 4.3} What do
you know?}

\begin{longtable}[]{@{}llllll@{}}
\toprule
CIDR Notation & Mask & Bits & Block Size & Subnets &
Hosts\tabularnewline
\midrule
\endhead
/25 & 128 & 1 bit on and 7 bits off & 128 & 0 and 128 & 2 subnets, each
with 126 hosts\tabularnewline
/26 & 192 & 2 bits on and 6 bits off & 64 & 0, 64, 128, 192 & 4 subnets,
each with 62 hosts\tabularnewline
/27 & 224 & 3 bits on and 5 bits off & 32 & 0, 32, 64, 96, 128, 160,
192, 224 & 8 subnets, each with 30 hosts\tabularnewline
/28 & 240 & 4 bits on and 4 bits off & 16 & 0, 16, 32, 48, 64, 80, 96,
112, 128, 144, 160, 176, 192, 208, 224, 240 & 16 subnets, each with 14
hosts\tabularnewline
/29 & 248 & 5 bits on and 3 bits off & 8 & 0, 8, 16, 24, 32, 40, 48,
etc. & 32 subnets, each with 6 hosts\tabularnewline
/30 & 252 & 6 bits on and 2 bits off & 4 & 0, 4, 8, 12, 16, 20, 24, etc.
& 64 subnets, each with 2 hosts\tabularnewline
\bottomrule
\end{longtable}

Regardless of whether you have a Class A, Class B, or Class C address,
the /30 mask will provide you with only two hosts, ever. As suggested by
Cisco, this mask is suited almost exclusively for use on point-to-point
links.

If you can memorize this ``What Do We Know?'' section, you'll be much
better off in your day-to-day job and in your studies. Try saying it out
loud, which helps you memorize things---yes, your significant other
and/or coworkers will think you've lost it, but they probably already do
if you're in the networking field anyway. And if you're not yet in the
networking field but are studying all this to break into it, get used to
it!

It's also helpful to write these on some type of flashcards and have
people test your skill. You'd be amazed at how fast you can get
subnetting down if you memorize block sizes as well as this ``What Do We
Know?'' section.

\subsubsection[Subnetting Class B
Addresses]{\texorpdfstring{\protect\hypertarget{c04.xhtmlux5cux23c04-sec-7}{}{}Subnetting
Class B Addresses}{Subnetting Class B Addresses}}

Before we dive into this, let's look at all the possible Class B subnet
masks first. Notice that we have a lot more possible subnet masks than
we do with a Class C network address:

\begin{verbatim}
255.255.0.0    (/16)
255.255.128.0  (/17)      255.255.255.0    (/24)
255.255.192.0  (/18)      255.255.255.128  (/25)
255.255.224.0  (/19)      255.255.255.192  (/26)
255.255.240.0  (/20)      255.255.255.224  (/27)
255.255.248.0  (/21)      255.255.255.240  (/28)
255.255.252.0  (/22)      255.255.255.248  (/29)
255.255.254.0  (/23)      255.255.255.252  (/30)
\end{verbatim}

We know the Class B network address has 16 bits available for host
addressing. This means we can use up to 14 bits for subnetting because
we need to leave at least 2 bits for host addressing. Using a /16 means
you are not subnetting with Class B, but it \emph{is} a mask you can
use!

\begin{center}\rule{0.5\linewidth}{0.5pt}\end{center}

\includegraphics{images/note.png} By the way, do you notice anything
interesting about that list of subnet values---a pattern, maybe? Ah ha!
That's exactly why I had you memorize the binary-to-decimal numbers
earlier in Chapter 2, ``Ethernet Networking and Data Encapsulation.''
Since subnet mask bits start on the left and move to the right and bits
can't be skipped, the numbers are always the same regardless of the
class of address. If you haven't already, memorize this pattern!

\begin{center}\rule{0.5\linewidth}{0.5pt}\end{center}

The process of subnetting a Class B network is pretty much the same as
it is for a Class C, except that you have more host bits and you start
in the third octet.

Use the same subnet numbers for the third octet with Class B that you
used for the fourth octet with Class C, but add a zero to the network
portion and a 255 to the broadcast section in the fourth octet. The
following table shows you an example host range of two subnets used in a
Class B 240 (/20) subnet mask:

\begin{longtable}[]{@{}lll@{}}
\toprule
Subnet address & 16.0 & 32.0\tabularnewline
\midrule
\endhead
Broadcast address & 31.255 & 47.255\tabularnewline
\bottomrule
\end{longtable}

Just add the valid hosts between the numbers and you're set!

\begin{center}\rule{0.5\linewidth}{0.5pt}\end{center}

\includegraphics{images/note.png} The preceding example is true only
until you get up to /24. After that, it's numerically exactly like Class
C.

\begin{center}\rule{0.5\linewidth}{0.5pt}\end{center}

\paragraph{Subnetting Practice Examples: Class B Addresses}

The following sections will give you an opportunity to practice
subnetting Class B addresses. Again, I have to mention that this is the
same as subnetting with Class C, except we start in the third
octet---with the exact same numbers!

\subparagraph[Practice Example \#1B: 255.255.128.0
(/17)]{\texorpdfstring{\protect\hypertarget{c04.xhtmlux5cux23Page_156}{}{}Practice
Example \#1B: 255.255.128.0
(/17)}{Practice Example \#1B: 255.255.128.0 (/17)}}

172.16.0.0 = Network address

255.255.128.0 = Subnet mask

\begin{enumerate}
\tightlist
\item
  \emph{Subnets?} 2\textsuperscript{1} = 2 (same amount as Class C).
\item
  \emph{Hosts?} 2\textsuperscript{15} -- 2 = 32,766 (7 bits in the third
  octet, and 8 in the fourth).
\item
  \emph{Valid subnets?} 256 -- 128 = 128. 0, 128. Remember that
  subnetting is performed in the third octet, so the subnet numbers are
  really 0.0 and 128.0, as shown in the next table. These are the exact
  numbers we used with Class C; we use them in the third octet and add a
  0 in the fourth octet for the network address.
\item
  \emph{Broadcast address for each subnet?}
\item
  \emph{Valid hosts?}
\end{enumerate}

The following table shows the two subnets available, the valid host
range, and the broadcast address of each:

\begin{longtable}[]{@{}lll@{}}
\toprule
Subnet & 0.0 & 128.0\tabularnewline
\midrule
\endhead
First host & 0.1 & 128.1\tabularnewline
Last host & 127.254 & 255.254\tabularnewline
Broadcast & 127.255 & 255.255\tabularnewline
\bottomrule
\end{longtable}

Okay, notice that we just added the fourth octet's lowest and highest
values and came up with the answers. And again, it's done exactly the
same way as for a Class C subnet. We just used the same numbers in the
third octet and added 0 and 255 in the fourth octet---pretty simple,
huh? I really can't say this enough: it's just not that hard. The
numbers never change; we just use them in different octets!

Question: Using the previous subnet mask, do you think 172.16.10.0 is a
valid host address? What about 172.16.10.255? Can 0 and 255 in the
fourth octet ever be a valid host address? The answer is absolutely,
yes, those are valid hosts! Any number between the subnet number and the
broadcast address is always a valid host.

\subparagraph{Practice Example \#2B: 255.255.192.0 (/18)}

172.16.0.0 = Network address

255.255.192.0 = Subnet mask

\begin{enumerate}
\tightlist
\item
  \emph{Subnets?} 2\textsuperscript{2} = 4.
\item
  \emph{Hosts?} 2\textsuperscript{14} -- 2 = 16,382 (6 bits in the third
  octet, and 8 in the fourth).
\item
  \protect\hypertarget{c04.xhtmlux5cux23Page_157}{}{}\emph{Valid
  subnets?} 256 -- 192 = 64. 0, 64, 128, 192. Remember that the
  subnetting is performed in the third octet, so the subnet numbers are
  really 0.0, 64.0, 128.0, and 192.0, as shown in the next table.
\item
  \emph{Broadcast address for each subnet?}
\item
  \emph{Valid hosts?}
\end{enumerate}

The following table shows the four subnets available, the valid host
range, and the broadcast address of each:

\begin{longtable}[]{@{}lllll@{}}
\toprule
Subnet & 0.0 & 64.0 & 128.0 & 192.0\tabularnewline
\midrule
\endhead
First host & 0.1 & 64.1 & 128.1 & 192.1\tabularnewline
Last host & 63.254 & 127.254 & 191.254 & 255.254\tabularnewline
Broadcast & 63.255 & 127.255 & 191.255 & 255.255\tabularnewline
\bottomrule
\end{longtable}

Again, it's pretty much the same as it is for a Class C subnet---we just
added 0 and 255 in the fourth octet for each subnet in the third octet.

\subparagraph{Practice Example \#3B: 255.255.240.0 (/20)}

172.16.0.0 = Network address

255.255.240.0 = Subnet mask

\begin{enumerate}
\tightlist
\item
  \emph{Subnets?} 2\textsuperscript{4} = 16.
\item
  \emph{Hosts?} 2\textsuperscript{12} -- 2 = 4094.
\item
  \emph{Valid subnets?} 256 -- 240 = 0, 16, 32, 48, etc., up to 240.
  Notice that these are the same numbers as a Class C 240 mask---we just
  put them in the third octet and add a 0 and 255 in the fourth octet.
\item
  \emph{Broadcast address for each subnet?}
\item
  \emph{Valid hosts?}
\end{enumerate}

The following table shows the first four subnets, valid hosts, and
broadcast addresses in a Class B 255.255.240.0 mask:

\begin{longtable}[]{@{}lllll@{}}
\toprule
Subnet & 0.0 & 16.0 & 32.0 & 48.0\tabularnewline
\midrule
\endhead
First host & 0.1 & 16.1 & 32.1 & 48.1\tabularnewline
Last host & 15.254 & 31.254 & 47.254 & 63.254\tabularnewline
Broadcast & 15.255 & 31.255 & 47.255 & 63.255\tabularnewline
\bottomrule
\end{longtable}

\subparagraph{Practice Example \#4B: 255.255.248.0 (/21)}

172.16.0.0 = Network address

255.255.248.0 = Subnet mask

\begin{enumerate}
\tightlist
\item
  \emph{Subnets?} 2\textsuperscript{5} = 32.
\item
  \emph{Hosts?} 2\textsuperscript{11} -- 2 = 2046.
\item
  \emph{Valid subnets?} 256 -- 248 = 0, 8, 16, 24, 32, etc., up to 248.
\item
  \emph{Broadcast address for each subnet?}
\item
  \emph{Valid hosts?}
\end{enumerate}

The following table shows the first five subnets, valid hosts, and
broadcast addresses in a Class B 255.255.248.0 mask:

\begin{longtable}[]{@{}llllll@{}}
\toprule
Subnet & 0.0 & 8.0 & 16.0 & 24.0 & 32.0\tabularnewline
\midrule
\endhead
First host & 0.1 & 8.1 & 16.1 & 24.1 & 32.1\tabularnewline
Last host & 7.254 & 15.254 & 23.254 & 31.254 & 39.254\tabularnewline
Broadcast & 7.255 & 15.255 & 23.255 & 31.255 & 39.255\tabularnewline
\bottomrule
\end{longtable}

\subparagraph{Practice Example \#5B: 255.255.252.0 (/22)}

172.16.0.0 = Network address

255.255.252.0 = Subnet mask

\begin{enumerate}
\tightlist
\item
  \protect\hypertarget{c04.xhtmlux5cux23Page_158}{}{}\emph{Subnets?}
  2\textsuperscript{6} = 64.
\item
  \emph{Hosts?} 2\textsuperscript{10} -- 2 = 1022.
\item
  \emph{Valid subnets?} 256 -- 252 = 0, 4, 8, 12, 16, etc., up to 252.
\item
  \emph{Broadcast address for each subnet?}
\item
  \emph{Valid hosts?}
\end{enumerate}

The following table shows the first five subnets, valid hosts, and
broadcast addresses in a Class B 255.255.252.0 mask:

\begin{longtable}[]{@{}llllll@{}}
\toprule
Subnet & 0.0 & 4.0 & 8.0 & 12.0 & 16.0\tabularnewline
\midrule
\endhead
First host & 0.1 & 4.1 & 8.1 & 12.1 & 16.1\tabularnewline
Last host & 3.254 & 7.254 & 11.254 & 15.254 & 19.254\tabularnewline
Broadcast & 3.255 & 7.255 & 11.255 & 15.255 & 19.255\tabularnewline
\bottomrule
\end{longtable}

\subparagraph{Practice Example \#6B: 255.255.254.0 (/23)}

172.16.0.0 = Network address

255.255.254.0 = Subnet mask

\begin{enumerate}
\tightlist
\item
  \emph{Subnets?} 2\textsuperscript{7} = 128.
\item
  \emph{Hosts?} 2\textsuperscript{9} -- 2 = 510.
\item
  \emph{Valid subnets?} 256 -- 254 = 0, 2, 4, 6, 8, etc., up to 254.
\item
  \emph{Broadcast address for each subnet?}
\item
  \emph{Valid hosts?}
\end{enumerate}

The following table shows the first five subnets, valid hosts, and
broadcast addresses in a Class B 255.255.254.0 mask:

\begin{longtable}[]{@{}llllll@{}}
\toprule
Subnet & 0.0 & 2.0 & 4.0 & 6.0 & 8.0\tabularnewline
\midrule
\endhead
First host & 0.1 & 2.1 & 4.1 & 6.1 & 8.1\tabularnewline
Last host & 1.254 & 3.254 & 5.254 & 7.254 & 9.254\tabularnewline
Broadcast & 1.255 & 3.255 & 5.255 & 7.255 & 9.255\tabularnewline
\bottomrule
\end{longtable}

\subparagraph[Practice Example \#7B: 255.255.255.0
(/24)]{\texorpdfstring{\protect\hypertarget{c04.xhtmlux5cux23Page_159}{}{}Practice
Example \#7B: 255.255.255.0
(/24)}{Practice Example \#7B: 255.255.255.0 (/24)}}

Contrary to popular belief, 255.255.255.0 used with a Class B network
address is not called a Class B network with a Class C subnet mask. It's
amazing how many people see this mask used in a Class B network and
think it's a Class C subnet mask. This is a Class B subnet mask with 8
bits of subnetting---it's logically different from a Class C mask.
Subnetting this address is fairly simple:

172.16.0.0 = Network address

255.255.255.0 = Subnet mask

\begin{enumerate}
\tightlist
\item
  \emph{Subnets?} 2\textsuperscript{8} = 256.
\item
  \emph{Hosts?} 2\textsuperscript{8} -- 2 = 254.
\item
  \emph{Valid subnets?} 256 -- 255 = 1. 0, 1, 2, 3, etc., all the way to
  255.
\item
  \emph{Broadcast address for each subnet?}
\item
  \emph{Valid hosts?}
\end{enumerate}

The following table shows the first four and last two subnets, the valid
hosts, and the broadcast addresses in a Class B 255.255.255.0 mask:

\begin{longtable}[]{@{}llllllll@{}}
\toprule
Subnet & 0.0 & 1.0 & 2.0 & 3.0 & \ldots{} & 254.0 & 255.0\tabularnewline
\midrule
\endhead
First host & 0.1 & 1.1 & 2.1 & 3.1 & \ldots{} & 254.1 &
255.1\tabularnewline
Last host & 0.254 & 1.254 & 2.254 & 3.254 & \ldots{} & 254.254 &
255.254\tabularnewline
Broadcast & 0.255 & 1.255 & 2.255 & 3.255 & \ldots{} & 254.255 &
255.255\tabularnewline
\bottomrule
\end{longtable}

\subparagraph{Practice Example \#8B: 255.255.255.128 (/25)}

This is actually one of the hardest subnet masks you can play with. And
worse, it actually is a really good subnet to use in production because
it creates over 500 subnets with 126 hosts for each subnet---a nice
mixture. So, don't skip over it!

172.16.0.0 = Network address

255.255.255.128 = Subnet mask

\begin{enumerate}
\tightlist
\item
  \protect\hypertarget{c04.xhtmlux5cux23Page_160}{}{}\emph{Subnets?}
  2\textsuperscript{9} = 512.
\item
  \emph{Hosts?} 2\textsuperscript{7} -- 2 = 126.
\item
  \emph{Valid subnets?} Now for the tricky part. 256 -- 255 = 1. 0, 1,
  2, 3, etc., for the third octet. But you can't forget the one subnet
  bit used in the fourth octet. Remember when I showed you how to figure
  one subnet bit with a Class C mask? You figure this the same way. You
  actually get two subnets for each third octet value, hence the 512
  subnets. For example, if the third octet is showing subnet 3, the two
  subnets would actually be 3.0 and 3.128.
\item
  \emph{Broadcast address for each subnet?} The numbers right before the
  next subnet.
\item
  \emph{Valid hosts?} The numbers between the subnet numbers and the
  broadcast address.
\end{enumerate}

The following graphic shows how you can create subnets, valid hosts, and
broadcast addresses using the Class B 255.255.255.128 subnet mask. The
first eight subnets are shown, followed by the last two subnets:

\subparagraph[Practice Example \#9B: 255.255.255.192
(/26)]{\texorpdfstring{\protect\hypertarget{c04.xhtmlux5cux23Page_161}{}{}Practice
Example \#9B: 255.255.255.192
(/26)}{Practice Example \#9B: 255.255.255.192 (/26)}}

Now, this is where Class B subnetting gets easy. Since the third octet
has a 255 in the mask section, whatever number is listed in the third
octet is a subnet number. And now that we have a subnet number in the
fourth octet, we can subnet this octet just as we did with Class C
subnetting. Let's try it out:

172.16.0.0 = Network address

255.255.255.192 = Subnet mask

\begin{enumerate}
\tightlist
\item
  \emph{Subnets?} 2\textsuperscript{10} = 1024.
\item
  \emph{Hosts?} 2\textsuperscript{6} -- 2 = 62.
\item
  \emph{Valid subnets?} 256 -- 192 = 64. The subnets are shown in the
  following table. Do these numbers look familiar?
\item
  \emph{Broadcast address for each subnet?}
\item
  \emph{Valid hosts?}
\end{enumerate}

The following table shows the first eight subnet ranges, valid hosts,
and broadcast addresses:

\begin{longtable}[]{@{}lllllllll@{}}
\toprule
Subnet & 0.0 & 0.64 & 0.128 & 0.192 & 1.0 & 1.64 & 1.128 &
1.192\tabularnewline
\midrule
\endhead
First host & 0.1 & 0.65 & 0.129 & 0.193 & 1.1 & 1.65 & 1.129 &
1.193\tabularnewline
Last host & 0.62 & 0.126 & 0.190 & 0.254 & 1.62 & 1.126 & 1.190 &
1.254\tabularnewline
Broadcast & 0.63 & 0.127 & 0.191 & 0.255 & 1.63 & 1.127 & 1.191 &
1.255\tabularnewline
\bottomrule
\end{longtable}

Notice that for each subnet value in the third octet, you get subnets 0,
64, 128, and 192 in the fourth octet.

\subparagraph{Practice Example \#10B: 255.255.255.224 (/27)}

This one is done the same way as the preceding subnet mask, except that
we just have more subnets and fewer hosts per subnet available.

172.16.0.0 = Network address

255.255.255.224 = Subnet mask

\begin{enumerate}
\tightlist
\item
  \protect\hypertarget{c04.xhtmlux5cux23Page_162}{}{}\emph{Subnets?}
  2\textsuperscript{11} = 2048.
\item
  \emph{Hosts?} 2\textsuperscript{5} -- 2 = 30.
\item
  \emph{Valid subnets?} 256 -- 224 = 32. 0, 32, 64, 96, 128, 160, 192,
  224.
\item
  \emph{Broadcast address for each subnet?}
\item
  \emph{Valid hosts?}
\end{enumerate}

The following table shows the first eight subnets:

\begin{longtable}[]{@{}lllllllll@{}}
\toprule
Subnet & 0.0 & 0.32 & 0.64 & 0.96 & 0.128 & 0.160 & 0.192 &
0.224\tabularnewline
\midrule
\endhead
First host & 0.1 & 0.33 & 0.65 & 0.97 & 0.129 & 0.161 & 0.193 &
0.225\tabularnewline
Last host & 0.30 & 0.62 & 0.94 & 0.126 & 0.158 & 0.190 & 0.222 &
0.254\tabularnewline
Broadcast & 0.31 & 0.63 & 0.95 & 0.127 & 0.159 & 0.191 & 0.223 &
0.255\tabularnewline
\bottomrule
\end{longtable}

\protect\hypertarget{c04.xhtmlux5cux23Page_163}{}{}This next table shows
the last eight subnets:

\begin{longtable}[]{@{}lllllllll@{}}
\toprule
Subnet & 255.0 & 255.32 & 255.64 & 255.96 & 255.128 & 255.160 & 255.192
& 255.224\tabularnewline
\midrule
\endhead
First host & 255.1 & 255.33 & 255.65 & 255.97 & 255.129 & 255.161 &
255.193 & 255.225\tabularnewline
Last host & 255.30 & 255.62 & 255.94 & 255.126 & 255.158 & 255.190 &
255.222 & 255.254\tabularnewline
Broadcast & 255.31 & 255.63 & 255.95 & 255.127 & 255.159 & 255.191 &
255.223 & 255.255\tabularnewline
\bottomrule
\end{longtable}

\paragraph{Subnetting in Your Head: Class B Addresses}

Are you nuts? Subnet Class B addresses in our heads? It's actually
easier than writing it out---I'm not kidding! Let me show you how:

\begin{quote}
\emph{Question:} What is the subnet and broadcast address of the subnet
in which 172.16.10.33 /27 resides?

\emph{Answer:} The interesting octet is the fourth one. 256 -- 224 = 32.
32 + 32 = 64. You've got it: 33 is between 32 and 64. But remember that
the third octet is considered part of the subnet, so the answer would be
the 10.32 subnet. The broadcast is 10.63, since 10.64 is the next
subnet. That was a pretty easy one.

\emph{Question:} What subnet and broadcast address is the IP address
172.16.66.10 255.255.192.0 (/18) a member of?

\emph{Answer:} The interesting octet here is the third octet instead of
the fourth one. 256 -- 192 = 64. 0, 64, 128. The subnet is 172.16.64.0.
The broadcast must be 172.16.127.255 since 128.0 is the next subnet.

\emph{Question:} What subnet and broadcast address is the IP address
172.16.50.10 255.255.224.0 (/19) a member of?

\emph{Answer:} 256 -- 224 = 0, 32, 64 (remember, we always start
counting at 0). The subnet is 172.16.32.0, and the broadcast must be
172.16.63.255 since 64.0 is the next subnet.

\emph{Question:} What subnet and broadcast address is the IP address
172.16.46.255 255.255.240.0 (/20) a member of?

\emph{Answer:} 256 -- 240 = 16. The third octet is important here: 0,
16, 32, 48. This subnet address must be in the 172.16.32.0 subnet, and
the broadcast must be 172.16.47.255 since 48.0 is the next subnet. So,
yes, 172.16.46.255 is a valid host.

\emph{Question:} What subnet and broadcast address is the IP address
172.16.45.14 255.255.255.252 (/30) a member of?

\emph{Answer:} Where is our interesting octet? 256 -- 252 = 0, 4, 8, 12,
16---the fourth. The subnet is 172.16.45.12, with a broadcast of
172.16.45.15 because the next subnet is 172.16.45.16.

\emph{Question:} What is the subnet and broadcast address of the host
172.16.88.255/20?

\protect\hypertarget{c04.xhtmlux5cux23Page_164}{}{}\emph{Answer:} What
is a /20 written out in dotted decimal? If you can't answer this, you
can't answer this question, can you? A /20 is 255.255.240.0, gives us a
block size of 16 in the third octet, and since no subnet bits are on in
the fourth octet, the answer is always 0 and 255 in the fourth octet: 0,
16, 32, 48, 64, 80, 96. Because 88 is between 80 and 96, the subnet is
80.0 and the broadcast address is 95.255.

\emph{Question:} A router receives a packet on an interface with a
destination address of 172.16.46.191/26. What will the router do with
this packet?

\emph{Answer:} Discard it. Do you know why? 172.16.46.191/26 is a
255.255.255.192 mask, which gives us a block size of 64. Our subnets are
then 0, 64, 128 and 192. 191 is the broadcast address of the 128 subnet,
and by default, a router will discard any broadcast packets.
\end{quote}

\subsubsection[Subnetting Class A
Addresses]{\texorpdfstring{\protect\hypertarget{c04.xhtmlux5cux23c04-sec-8}{}{}Subnetting
Class A Addresses}{Subnetting Class A Addresses}}

You don't go about Class A subnetting any differently than Classes B and
C, but there are 24 bits to play with instead of the 16 in a Class B
address and the 8 in a Class C address.

Let's start by listing all the Class A masks:

\begin{verbatim}
255.0.0.0    (/8)
255.128.0.0  (/9)           255.255.240.0  (/20)
255.192.0.0  (/10)          255.255.248.0  (/21)
255.224.0.0  (/11)          255.255.252.0  (/22)
255.240.0.0  (/12)          255.255.254.0  (/23)
255.248.0.0  (/13)          255.255.255.0  (/24)
255.252.0.0  (/14)          255.255.255.128  (/25)
255.254.0.0  (/15)          255.255.255.192  (/26)
255.255.0.0  (/16)          255.255.255.224  (/27)
255.255.128.0  (/17)        255.255.255.240  (/28)
255.255.192.0  (/18)        255.255.255.248  (/29)
255.255.224.0  (/19)        255.255.255.252  (/30)
\end{verbatim}

That's it. You must leave at least 2 bits for defining hosts. I hope you
can see the pattern by now. Remember, we're going to do this the same
way as a Class B or C subnet. It's just that, again, we simply have more
host bits and we just use the same subnet numbers we used with Class B
and C, but we start using these numbers in the second octet. However,
the reason Class A addresses are so popular to implement is because they
give the most flexibility. You can subnet in the second, third or fourth
octet. I'll show you this in the next examples.

\paragraph{Subnetting Practice Examples: Class A Addresses}

When you look at an IP address and a subnet mask, you must be able to
distinguish the bits used for subnets from the bits used for determining
hosts. This is imperative. If you're still struggling with this concept,
please reread the section ``IP Addressing'' in Chapter 3. It shows you
how to determine the difference between the subnet and host bits and
should help clear things up.

\subparagraph[Practice Example \#1A: 255.255.0.0
(/16)]{\texorpdfstring{\protect\hypertarget{c04.xhtmlux5cux23Page_165}{}{}Practice
Example \#1A: 255.255.0.0
(/16)}{Practice Example \#1A: 255.255.0.0 (/16)}}

Class A addresses use a default mask of 255.0.0.0, which leaves 22 bits
for subnetting because you must leave 2 bits for host addressing. The
255.255.0.0 mask with a Class A address is using 8 subnet bits:

\begin{enumerate}
\tightlist
\item
  \emph{Subnets?} 2\textsuperscript{8} = 256.
\item
  \emph{Hosts?} 2\textsuperscript{16} -- 2 = 65,534.
\item
  \emph{Valid subnets?} What is the interesting octet? 256 -- 255 = 1.
  0, 1, 2, 3, etc. (all in the second octet). The subnets would be
  10.0.0.0, 10.1.0.0, 10.2.0.0, 10.3.0.0, etc., up to 10.255.0.0.
\item
  \emph{Broadcast address for each subnet?}
\item
  \emph{Valid hosts?}
\end{enumerate}

The following table shows the first two and the last two subnets, the
valid host range and the broadcast addresses for the private Class A
10.0.0.0 network:

\begin{longtable}[]{@{}llllll@{}}
\toprule
Subnet & 10.0.0.0 & 10.1.0.0 & \ldots{} & 10.254.0.0 &
10.255.0.0\tabularnewline
\midrule
\endhead
First host & 10.0.0.1 & 10.1.0.1 & \ldots{} & 10.254.0.1 &
10.255.0.1\tabularnewline
Last host & 10.0.255.254 & 10.1.255.254 & \ldots{} & 10.254.255.254 &
10.255.255.254\tabularnewline
Broadcast & 10.0.255.255 & 10.1.255.255 & \ldots{} & 10.254.255.255 &
10.255.255.255\tabularnewline
\bottomrule
\end{longtable}

\subparagraph{Practice Example \#2A: 255.255.240.0 (/20)}

255.255.240.0 gives us 12 bits of subnetting and leaves us 12 bits for
host addressing.

\begin{enumerate}
\tightlist
\item
  \emph{Subnets?} 2\textsuperscript{12} = 4096.
\item
  \emph{Hosts?} 2\textsuperscript{12} -- 2 = 4094.
\item
  \emph{Valid subnets?} What is your interesting octet? 256 -- 240 = 16.
  The subnets in the second octet are a block size of 1 and the subnets
  in the third octet are 0, 16, 32, etc.
\item
  \emph{Broadcast address for each subnet?}
\item
  \emph{Valid hosts?}
\end{enumerate}

The following table shows some examples of the host ranges---the first
three subnets and the last subnet:

\begin{longtable}[]{@{}llllll@{}}
\toprule
Subnet & 10.0.0.0 & 10.0.16.0 & 10.0.32.0 & \ldots{} &
10.255.240.0\tabularnewline
\midrule
\endhead
First host & 10.0.0.1 & 10.0.16.1 & 10.0.32.1 & \ldots{} &
10.255.240.1\tabularnewline
Last host & 10.0.15.254 & 10.0.31.254 & 10.0.47.254 & \ldots{} &
10.255.255.254\tabularnewline
Broadcast & 10.0.15.255 & 10.0.31.255 & 10.0.47.255 & \ldots{} &
10.255.255.255\tabularnewline
\bottomrule
\end{longtable}

\subparagraph{Practice Example \#3A: 255.255.255.192 (/26)}

Let's do one more example using the second, third, and fourth octets for
subnetting:

\begin{enumerate}
\tightlist
\item
  \emph{Subnets?} 2\textsuperscript{18} = 262,144.
\item
  \emph{Hosts?} 2\textsuperscript{6} -- 2 = 62.
\item
  \emph{Valid subnets?} In the second and third octet, the block size is
  1, and in the fourth octet, the block size is 64.
\item
  \emph{Broadcast address for each subnet?}
\item
  \emph{Valid hosts?}
\end{enumerate}

The following table shows the first four subnets and their valid hosts
and broadcast addresses in the Class A 255.255.255.192 mask:

\begin{longtable}[]{@{}lllll@{}}
\toprule
Subnet & 10.0.0.0 & 10.0.0.64 & 10.0.0.128 & 10.0.0.192\tabularnewline
\midrule
\endhead
First host & 10.0.0.1 & 10.0.0.65 & 10.0.0.129 &
10.0.0.193\tabularnewline
Last host & 10.0.0.62 & 10.0.0.126 & 10.0.0.190 &
10.0.0.254\tabularnewline
Broadcast & 10.0.0.63 & 10.0.0.127 & 10.0.0.191 &
10.0.0.255\tabularnewline
\bottomrule
\end{longtable}

This table shows the last four subnets and their valid hosts and
broadcast addresses:

\begin{longtable}[]{@{}lllll@{}}
\toprule
\endhead
Subnet & 10.255.255.0 & 10.255.255.64 & 10.255.255.128 &
10.255.255.192\tabularnewline
First host & 10.255.255.1 & 10.255.255.65 & 10.255.255.129 &
10.255.255.193\tabularnewline
Last host & 10.255.255.62 & 10.255.255.126 & 10.255.255.190 &
10.255.255.254\tabularnewline
Broadcast & 10.255.255.63 & 10.255.255.127 & 10.255.255.191 &
10.255.255.255\tabularnewline
\bottomrule
\end{longtable}

\paragraph{Subnetting in Your Head: Class A Addresses}

Again, I know this sounds hard, but as with Class C and Class B, the
numbers are the same; we just start in the second octet. What makes this
easy? You only need to worry about the octet that has the largest block
size, which is typically called the interesting octet, and one that is
something other than 0 or 255, such as, for example, 255.255.240.0 (/20)
with a Class A network. The second octet has a block size of 1, so any
number listed in that octet is a subnet. The third octet is a 240 mask,
which means we have a block size of 16 in the third octet. If your host
ID is 10.20.80.30, what is your subnet, broadcast address, and valid
host range?

The subnet in the second octet is 20 with a block size of 1, but the
third octet is in block sizes of 16, so we'll just count them out: 0,
16, 32, 48, 64, 80, 96\ldots{} voilà! By the way, you
\protect\hypertarget{c04.xhtmlux5cux23Page_166}{}{}can count by 16s by
now, right? Good! This makes our subnet 10.20.80.0, with a broadcast
address of 10.20.95.255 because the next subnet is 10.20.96.0. The valid
host range is 10.20.80.1 through 10.20.95.254. And yes, no lie! You
really can do this in your head if you just get your block sizes nailed!

Let's practice on one more, just for fun!

Host IP: 10.1.3.65/23

First, you can't answer this question if you don't know what a /23 is.
It's 255.255.254.0. The interesting octet here is the third one: 256 --
254 = 2. Our subnets in the third octet are 0, 2, 4, 6, etc. The host in
this question is in subnet 2.0, and the next subnet is 4.0, so that
makes the broadcast address 3.255. And any address between 10.1.2.1 and
10.1.3.254 is considered a valid host.

\subsection[Summary]{\texorpdfstring{\protect\hypertarget{c04.xhtmlux5cux23c04-sec-9}{}{}Summary}{Summary}}

Did you read Chapters 3 and 4 and understand everything on the first
pass? If so, that is fantastic---congratulations! However, you probably
really did get lost a couple of times. No worries because as I told you,
that's what usually happens. Don't waste time feeling bad if you have to
read each chapter more than once, or even 10 times, before you're truly
good to go. If you do have to read the chapters more than once, you'll
be seriously better off in the long run even if you were pretty
comfortable the first time through!

This chapter provided you with an important understanding of IP
subnetting---the painless way! And when you've got the key material
presented in this chapter really nailed down, you should be able to
subnet IP addresses in your head.

This chapter is extremely essential to your Cisco certification process,
so if you just skimmed it, please go back, read it thoroughly, and don't
forget to do all the written labs too!

\subsection[Exam
Essentials]{\texorpdfstring{\protect\hypertarget{c04.xhtmlux5cux23c04-sec-10}{}{}Exam
Essentials}{Exam Essentials}}

\textbf{Identify the advantages of subnetting.} Benefits of subnetting a
physical network include reduced network traffic, optimized network
performance, simplified management, and facilitated spanning of large
geographical distances.

\textbf{Describe the effect of the} \texttt{ip\ subnet-zero} command.
This command allows you to use the first and last subnet in your network
design.

\textbf{Identify the steps to subnet a classful network.} Understand how
IP addressing and subnetting work. First, determine your block size by
using the 256-subnet mask math. Then count your subnets and determine
the broadcast address of each subnet---it is always the number right
before the next subnet. Your valid hosts are the numbers between the
subnet address and the broadcast address.

\textbf{\protect\hypertarget{c04.xhtmlux5cux23Page_167}{}{}Determine
possible block sizes.} This is an important part of understanding IP
addressing and subnetting. The valid block sizes are always 2, 4, 8, 16,
32, 64, 128, etc. You can determine your block size by using the
256-subnet mask math.

\textbf{Describe the role of a subnet mask in IP addressing.} A subnet
mask is a 32-bit value that allows the recipient of IP packets to
distinguish the network ID portion of the IP address from the host ID
portion of the IP address.

\textbf{Understand and apply the 2\textsuperscript{\emph{x}} -- 2
formula.} Use this formula to determine the proper subnet mask for a
particular size network given the application of that subnet mask to a
particular classful network.

\textbf{Explain the impact of Classless Inter-Domain Routing (CIDR).}
CIDR allows the creation of networks of a size other than those allowed
with the classful subnetting by allowing more than the three classful
subnet masks.

\subsection[Written
Labs]{\texorpdfstring{\protect\hypertarget{c04.xhtmlux5cux23c04-sec-11}{}{}Written
Labs}{Written Labs}}

In this section, you'll complete the following labs to make sure you've
got the information and concepts contained within them fully dialed in:

Lab 4.1: Written Subnet Practice \#1

Lab 4.2: Written Subnet Practice \#2

Lab 4.3: Written Subnet Practice \#3

You can find the answers to these labs in Appendix A, ``Answers to
Written Labs.''

\subsubsection[Written Lab 4.1: Written Subnet Practice
\#1]{\texorpdfstring{\protect\hypertarget{c04.xhtmlux5cux23c04-sec-12}{}{}Written
Lab 4.1: Written Subnet Practice
\#1}{Written Lab 4.1: Written Subnet Practice \#1}}

Write the subnet, broadcast address, and a valid host range for question
1 through question 6. Then answer the remaining questions.

\begin{enumerate}
\tightlist
\item
  192.168.100.25/30
\item
  192.168.100.37/28
\item
  192.168.100.66/27
\item
  192.168.100.17/29
\item
  192.168.100.99/26
\item
  192.168.100.99/25
\item
  You have a Class B network and need 29 subnets. What is your mask?
\item
  What is the broadcast address of 192.168.192.10/29?
\item
  How many hosts are available with a Class C /29 mask?
\item
  What is the subnet for host ID 10.16.3.65/23?
\end{enumerate}

\subsubsection[Written Lab 4.2: Written Subnet Practice
\#2]{\texorpdfstring{\protect\hypertarget{c04.xhtmlux5cux23c04-sec-13}{}{}\protect\hypertarget{c04.xhtmlux5cux23Page_168}{}{}Written
Lab 4.2: Written Subnet Practice
\#2}{Written Lab 4.2: Written Subnet Practice \#2}}

Given a Class B network and the net bits identified (CIDR), complete the
following table to identify the subnet mask and the number of host
addresses possible for each mask.

\begin{longtable}[]{@{}lll@{}}
\toprule
Classful Address & Subnet Mask & Number of Hosts per Subnet
(2\textsuperscript{\emph{x}} -- 2)\tabularnewline
\midrule
\endhead
/16 & &\tabularnewline
/17 & &\tabularnewline
/18 & &\tabularnewline
/19 & &\tabularnewline
/20 & &\tabularnewline
/21 & &\tabularnewline
/22 & &\tabularnewline
/23 & &\tabularnewline
/24 & &\tabularnewline
/25 & &\tabularnewline
/26 & &\tabularnewline
/27 & &\tabularnewline
/28 & &\tabularnewline
/29 & &\tabularnewline
/30 & &\tabularnewline
\bottomrule
\end{longtable}

\subsubsection[Written Lab 4.3: Written Subnet Practice
\#3]{\texorpdfstring{\protect\hypertarget{c04.xhtmlux5cux23c04-sec-14}{}{}\protect\hypertarget{c04.xhtmlux5cux23Page_169}{}{}Written
Lab 4.3: Written Subnet Practice
\#3}{Written Lab 4.3: Written Subnet Practice \#3}}

Complete the following based on the decimal IP address.

\begin{longtable}[]{@{}lllll@{}}
\toprule
Decimal IP Address & Address Class & Number of Subnet and Host Bits &
Number of Subnets (2\textsuperscript{\emph{x}}) & Number of Hosts
(2\textsuperscript{\emph{x}} -- 2)\tabularnewline
\midrule
\endhead
10.25.66.154/23 & & & &\tabularnewline
172.31.254.12/24 & & & &\tabularnewline
192.168.20.123/28 & & & &\tabularnewline
63.24.89.21/18 & & & &\tabularnewline
128.1.1.254/20 & & & &\tabularnewline
208.100.54.209/30 & & & &\tabularnewline
\bottomrule
\end{longtable}

\subsection[Review
Questions]{\texorpdfstring{\protect\hypertarget{c04.xhtmlux5cux23c04-sec-15}{}{}\protect\hypertarget{c04.xhtmlux5cux23Page_170}{}{}Review
Questions}{Review Questions}}

\begin{center}\rule{0.5\linewidth}{0.5pt}\end{center}

\includegraphics{images/note.png} The following questions are designed
to test your understanding of this chapter's material. For more
information on how to get additional questions, please see
\href{http://www.lammle.com/ccna}{www.lammle.com/ccna}.

\begin{center}\rule{0.5\linewidth}{0.5pt}\end{center}

You can find the answers to these questions in Appendix B, ``Answers to
Review Questions.''

\begin{enumerate}
\item
  What is the maximum number of IP addresses that can be assigned to
  hosts on a local subnet that uses the 255.255.255.224 subnet mask?

  \begin{enumerate}
  \def\labelenumii{\Alph{enumii}.}
  \tightlist
  \item
    14
  \item
    15
  \item
    16
  \item
    30
  \item
    31
  \item
    62
  \end{enumerate}
\item
  You have a network that needs 29 subnets while maximizing the number
  of host addresses available on each subnet. How many bits must you
  borrow from the host field to provide the correct subnet mask?

  \begin{enumerate}
  \def\labelenumii{\Alph{enumii}.}
  \tightlist
  \item
    2
  \item
    3
  \item
    4
  \item
    5
  \item
    6
  \item
    7
  \end{enumerate}
\item
  What is the subnetwork address for a host with the IP address
  200.10.5.68/28?

  \begin{enumerate}
  \def\labelenumii{\Alph{enumii}.}
  \tightlist
  \item
    200.10.5.56
  \item
    200.10.5.32
  \item
    200.10.5.64
  \item
    200.10.5.0
  \end{enumerate}
\item
  The network address of 172.16.0.0/19 provides how many subnets and
  hosts?

  \begin{enumerate}
  \def\labelenumii{\Alph{enumii}.}
  \tightlist
  \item
    7 subnets, 30 hosts each
  \item
    7 subnets, 2,046 hosts each
  \item
    7 subnets, 8,190 hosts each
  \item
    8 subnets, 30 hosts each
  \item
    \protect\hypertarget{c04.xhtmlux5cux23Page_171}{}{}8 subnets, 2,046
    hosts each
  \item
    8 subnets, 8,190 hosts each
  \end{enumerate}
\item
  Which two statements describe the IP address 10.16.3.65/23? (Choose
  two.)

  \begin{enumerate}
  \def\labelenumii{\Alph{enumii}.}
  \tightlist
  \item
    The subnet address is 10.16.3.0 255.255.254.0.
  \item
    The lowest host address in the subnet is 10.16.2.1 255.255.254.0.
  \item
    The last valid host address in the subnet is 10.16.2.254
    255.255.254.0.
  \item
    The broadcast address of the subnet is 10.16.3.255 255.255.254.0.
  \item
    The network is not subnetted.
  \end{enumerate}
\item
  If a host on a network has the address 172.16.45.14/30, what is the
  subnetwork this host belongs to?

  \begin{enumerate}
  \def\labelenumii{\Alph{enumii}.}
  \tightlist
  \item
    172.16.45.0
  \item
    172.16.45.4
  \item
    172.16.45.8
  \item
    172.16.45.12
  \item
    172.16.45.16
  \end{enumerate}
\item
  Which mask should you use on point-to-point links in order to reduce
  the waste of IP addresses?

  \begin{enumerate}
  \def\labelenumii{\Alph{enumii}.}
  \tightlist
  \item
    /27
  \item
    /28
  \item
    /29
  \item
    /30
  \item
    /31
  \end{enumerate}
\item
  What is the subnetwork number of a host with an IP address of
  172.16.66.0/21?

  \begin{enumerate}
  \def\labelenumii{\Alph{enumii}.}
  \tightlist
  \item
    172.16.36.0
  \item
    172.16.48.0
  \item
    172.16.64.0
  \item
    172.16.0.0
  \end{enumerate}
\item
  You have an interface on a router with the IP address of
  192.168.192.10/29. Including the router interface, how many hosts can
  have IP addresses on the LAN attached to the router interface?

  \begin{enumerate}
  \def\labelenumii{\Alph{enumii}.}
  \tightlist
  \item
    6
  \item
    8
  \item
    30
  \item
    62
  \item
    126
  \end{enumerate}
\item
  \protect\hypertarget{c04.xhtmlux5cux23Page_172}{}{}You need to
  configure a server that is on the subnet 192.168.19.24/29. The router
  has the first available host address. Which of the following should
  you assign to the server?

  \begin{enumerate}
  \def\labelenumii{\Alph{enumii}.}
  \tightlist
  \item
    192.168.19.0 255.255.255.0
  \item
    192.168.19.33 255.255.255.240
  \item
    192.168.19.26 255.255.255.248
  \item
    192.168.19.31 255.255.255.248
  \item
    192.168.19.34 255.255.255.240
  \end{enumerate}
\item
  You have an interface on a router with the IP address of
  192.168.192.10/29. What is the broadcast address the hosts will use on
  this LAN?

  \begin{enumerate}
  \def\labelenumii{\Alph{enumii}.}
  \tightlist
  \item
    192.168.192.15
  \item
    192.168.192.31
  \item
    192.168.192.63
  \item
    192.168.192.127
  \item
    192.168.192.255
  \end{enumerate}
\item
  You need to subnet a network that has 5 subnets, each with at least 16
  hosts. Which classful subnet mask would you use?

  \begin{enumerate}
  \def\labelenumii{\Alph{enumii}.}
  \tightlist
  \item
    255.255.255.192
  \item
    255.255.255.224
  \item
    255.255.255.240
  \item
    255.255.255.248
  \end{enumerate}
\item
  You configure a router interface with the IP address 192.168.10.62
  255.255.255.192 and receive the following error:

\begin{verbatim}
Bad mask /26 for address 192.168.10.62
\end{verbatim}

  \begin{enumerate}
  \def\labelenumii{\Alph{enumii}.}
  \tightlist
  \item
    Why did you receive this error?
  \item
    You typed this mask on a WAN link and that is not allowed.
  \item
    This is not a valid host and subnet mask combination.
  \item
    \texttt{ip\ subnet-zero} is not enabled on the router.
  \item
    The router does not support IP.
  \end{enumerate}
\item
  If an Ethernet port on a router were assigned an IP address of
  172.16.112.1/25, what would be the valid subnet address of this
  interface?

  \begin{enumerate}
  \def\labelenumii{\Alph{enumii}.}
  \tightlist
  \item
    172.16.112.0
  \item
    172.16.0.0
  \item
    172.16.96.0
  \item
    172.16.255.0
  \item
    172.16.128.0
  \end{enumerate}
\item
  \protect\hypertarget{c04.xhtmlux5cux23Page_173}{}{}Using the following
  illustration, what would be the IP address of E0 if you were using the
  eighth subnet? The network ID is 192.168.10.0/28 and you need to use
  the last available IP address in the range. The zero subnet should not
  be considered valid for this question.

  \begin{figure}
  \centering
  \includegraphics{images/c04f007.jpg}
  \caption{}
  \end{figure}

  \begin{enumerate}
  \def\labelenumii{\Alph{enumii}.}
  \tightlist
  \item
    192.168.10.142
  \item
    192.168.10.66
  \item
    192.168.100.254
  \item
    192.168.10.143
  \item
    192.168.10.126
  \end{enumerate}
\item
  Using the illustration from the previous question, what would be the
  IP address of S0 if you were using the first subnet? The network ID is
  192.168.10.0/28 and you need to use the last available IP address in
  the range. Again, the zero subnet should not be considered valid for
  this question.

  \begin{enumerate}
  \def\labelenumii{\Alph{enumii}.}
  \tightlist
  \item
    192.168.10.24
  \item
    192.168.10.62
  \item
    192.168.10.30
  \item
    192.168.10.127
  \end{enumerate}
\item
  You have a network in your data center that needs 310 hosts. Which
  mask should you use so you waste the least amount of addresses?

  \begin{enumerate}
  \def\labelenumii{\Alph{enumii}.}
  \tightlist
  \item
    255.255.255.0
  \item
    255.255.254.0
  \item
    255.255.252.0
  \item
    255.255.248.0
  \end{enumerate}
\item
  You have a network with a host address of 172.16.17.0/22. From the
  following options, which is another valid host address in the same
  subnet?

  \begin{enumerate}
  \def\labelenumii{\Alph{enumii}.}
  \tightlist
  \item
    172.16.17.1 255.255.255.252
  \item
    172.16.0.1 255.255.240.0
  \item
    172.16.20.1 255.255.254.0
  \item
    172.16.16.1 255.255.255.240
  \item
    \protect\hypertarget{c04.xhtmlux5cux23Page_174}{}{}172.16.18.255
    255.255.252.0
  \item
    172.16.0.1 255.255.255.0
  \end{enumerate}
\item
  Your router has the following IP address on Ethernet0: 172.16.2.1/23.
  Which of the following can be valid host IDs on the LAN interface
  attached to the router? (Choose two.)

  \begin{enumerate}
  \def\labelenumii{\Alph{enumii}.}
  \tightlist
  \item
    172.16.0.5
  \item
    172.16.1.100
  \item
    172.16.1.198
  \item
    172.16.2.255
  \item
    172.16.3.0
  \item
    172.16.3.255
  \end{enumerate}
\item
  Given an IP address 172.16.28.252 with a subnet mask of 255.255.240.0,
  what is the correct network address?

  \begin{enumerate}
  \def\labelenumii{\Alph{enumii}.}
  \tightlist
  \item
    172.16.16.0
  \item
    172.16.0.0
  \item
    172.16.24.0
  \item
    172.16.28.0
  \end{enumerate}
\end{enumerate}

\protect\hypertarget{c05.xhtml}{}{}

\section[{Chapter 5}\\
{VLSMs, Summarization, and Troubleshooting
TCP/IP}]{\texorpdfstring{\protect\hypertarget{c05.xhtmlux5cux23c05}{}{}\protect\hypertarget{c05.xhtmlux5cux23Page_175}{}{}{Chapter
5}\\
{VLSMs, Summarization, and Troubleshooting
TCP/IP}}{Chapter 5 VLSMs, Summarization, and Troubleshooting TCP/IP}}

\begin{center}\rule{0.5\linewidth}{0.5pt}\end{center}

\subsection{THE FOLLOWING ICND1 EXAM TOPICS ARE COVERED IN THIS
CHAPTER:}

\begin{enumerate}
\tightlist
\item
  \includegraphics{images/right.png} \textbf{Network Fundamentals}

  \begin{enumerate}
  \tightlist
  \item
    \includegraphics{images/squ.png} 1.7 Apply troubleshooting
    methodologies to resolve problems
  \item
    \includegraphics{images/squ.png} 1.7.a Perform fault isolation and
    document
  \item
    \includegraphics{images/squ.png} 1.7.b Resolve or escalate
  \item
    \includegraphics{images/squ.png} 1.7.c Verify and monitor resolution
  \item
    \includegraphics{images/squ.png} 1.8 Configure, verify, and
    troubleshoot IPv4 addressing and subnetting
  \end{enumerate}
\end{enumerate}

\protect\hypertarget{c05.xhtmlux5cux23Page_176}{}{}\includegraphics{images/intro.png}
Now that IP addressing and subnetting have been thoroughly covered in
the last two chapters, you're fully prepared and ready to learn all
about variable length subnet masks (VLSMs). I'll also show you how to
design and implement a network using VLSM in this chapter. After
ensuring you've mastered VLSM design and implementation, I'll
demonstrate how to summarize classful boundaries.

We'll wrap up the chapter by going over IP address troubleshooting,
focusing on the steps Cisco recommends to follow when troubleshooting an
IP network.

So get psyched because this chapter will give you powerful tools to hone
your knowledge of IP addressing and networking and seriously refine the
important skills you've gained so far. So stay with me---I guarantee
that your hard work will pay off! Ready? Let's go!

\begin{center}\rule{0.5\linewidth}{0.5pt}\end{center}

\includegraphics{images/note.png} To find up-to-the minute updates for
this chapter, please see
\href{http://www.lammle.com/ccna}{www.lammle.com/ccna} or the book's web
page at \href{http://www.sybex.com/go/ccna}{www.sybex.com/go/ccna}.

\begin{center}\rule{0.5\linewidth}{0.5pt}\end{center}

\subsection[Variable Length Subnet Masks
(VLSMs)]{\texorpdfstring{\protect\hypertarget{c05.xhtmlux5cux23c05-sec-1}{}{}Variable
Length Subnet Masks (VLSMs)}{Variable Length Subnet Masks (VLSMs)}}

Teaching you a simple way to create many networks from a large single
network using subnet masks of different lengths in various kinds of
network designs is what my primary focus will be in this chapter. Doing
this is called VLSM networking, and it brings up another important
subject I mentioned in Chapter 4, ``Easy Subnetting,'' classful and
classless networking.

Older routing protocols like Routing Information Protocol version 1
(RIPv1) do not have a field for subnet information, so the subnet
information gets dropped. This means that if a router running RIP has a
subnet mask of a certain value, it assumes that \emph{all} interfaces
within the classful address space have the same subnet mask. This is
called classful routing, and RIP is considered a classful routing
protocol. We'll cover RIP and the difference between classful and
classless networks later on in Chapter 9, ``IP Routing,'' but for now,
just remember that if you try to mix and match subnet mask lengths in a
network that's running an old routing protocol, such as RIP, it just
won't work!

However, classless routing protocols do support the advertisement of
subnet information, which means you can use VLSM with routing protocols
such as RIPv2, Enhanced
\protect\hypertarget{c05.xhtmlux5cux23Page_177}{}{}Interior Gateway
Protocol (EIGRP), and Open Shortest Path First (OSPF). The benefit of
this type of network is that it saves a bunch of IP address space.

As the name suggests, VLSMs can use subnet masks with different lengths
for different router interfaces. Check out
\protect\hyperlink{c05.xhtmlux5cux23figure05-1}{Figure 5.1} to see an
example of why classful network designs are inefficient.

\begin{figure}
\centering
\includegraphics{images/c05f001.jpg}
\caption{{\protect\hyperlink{c05.xhtmlux5cux23figureanchor05-1}{\textbf{FIGURE
5.1}} Typical classful network}}
\end{figure}

Looking at \protect\hyperlink{c05.xhtmlux5cux23figure05-1}{Figure 5.1},
you can see that there are two routers, each with two LANs and connected
together with a WAN serial link. In a typical classful network design
that's running RIP, you could subnet a network like this:

192.168.10.0 = Network

255.255.255.240 (/28) = Mask

Our subnets would be---you know this part, right?--- 0, 16, 32, 48, 64,
80, etc., which allows us to assign 16 subnets to our internetwork. But
how many hosts would be available on each network? Well, as you know by
now, each subnet provides only 14 hosts, so each LAN has only 14 valid
hosts available (don't forget that the router interface needs an address
too and is included in the amount of needed valid hosts). This means
that one LAN doesn't even have enough addresses needed for all the
hosts, and this network as it is shown would not work as addressed in
the figure! Since the point-to-point WAN link also has 14 valid hosts,
it would be great to be able to nick a few valid hosts from that WAN
link to give to our LANs!

All hosts and router interfaces have the same subnet mask---again, known
as classful routing---and if we want this network to be efficient, we
would definitely need to add different masks to each router interface.

\protect\hypertarget{c05.xhtmlux5cux23Page_178}{}{}But that's not our
only problem---the link between the two routers will never use more than
two valid hosts! This wastes valuable IP address space, and it's the big
reason you need to learn about VLSM network design.

\subsubsection[VLSM
Design]{\texorpdfstring{\protect\hypertarget{c05.xhtmlux5cux23c05-sec-2}{}{}VLSM
Design}{VLSM Design}}

Let's take \protect\hyperlink{c05.xhtmlux5cux23figure05-1}{Figure 5.1}
and use a classless design instead, which will become the new network
shown in \protect\hyperlink{c05.xhtmlux5cux23figure05-2}{Figure 5.2}. In
the previous example, we wasted address space---one LAN didn't have
enough addresses because every router interface and host used the same
subnet mask. Not so good. A better solution would be to provide for only
the needed number of hosts on each router interface, and we're going to
use VLSMs to achieve that goal.

\begin{figure}
\centering
\includegraphics{images/c05f002.jpg}
\caption{{\protect\hyperlink{c05.xhtmlux5cux23figureanchor05-2}{\textbf{FIGURE
5.2}} Classless network design}}
\end{figure}

Now remember that we can use different size masks on each router
interface. If we use a /30 on our WAN links and a /27, /28, and /29 on
our LANs, we'll get 2 hosts per WAN interface and 30, 14, and 6 hosts
per LAN interface---nice (remember to count your router interface as a
host)! This makes a huge difference---not only can we get just the right
amount of hosts on each LAN, we still have room to add more WANs and
LANs using this same network!

\begin{center}\rule{0.5\linewidth}{0.5pt}\end{center}

\includegraphics{images/note.png} To implement a VLSM design on your
network, you need to have a routing protocol that sends subnet mask
information with the route updates. The protocols that do that are
RIPv2, EIGRP, and OSPF. Remember, RIPv1 will not work in classless
networks, so it's considered a classful routing protocol.

\begin{center}\rule{0.5\linewidth}{0.5pt}\end{center}

\subsubsection[Implementing VLSM
Networks]{\texorpdfstring{\protect\hypertarget{c05.xhtmlux5cux23c05-sec-3}{}{}\protect\hypertarget{c05.xhtmlux5cux23Page_179}{}{}Implementing
VLSM Networks}{Implementing VLSM Networks}}

To create VLSMs quickly and efficiently, you need to understand how
block sizes and charts work together to create the VLSM masks.
\protect\hyperlink{c05.xhtmlux5cux23table05-1}{Table 5.1} shows you the
block sizes used when creating VLSMs with Class C networks. For example,
if you need 25 hosts, then you'll need a block size of 32. If you need
11 hosts, you'll use a block size of 16. Need 40 hosts? Then you'll need
a block of 64. You cannot just make up block sizes---they've got to be
the block sizes shown in
\protect\hyperlink{c05.xhtmlux5cux23table05-1}{Table 5.1}. So memorize
the block sizes in this table---it's easy. They're the same numbers we
used with subnetting!

{\protect\hyperlink{c05.xhtmlux5cux23tableanchor05-1}{Table 5.1} Block
sizes}

\begin{longtable}[]{@{}llll@{}}
\toprule
Prefix & Mask & Hosts & Block Size\tabularnewline
\midrule
\endhead
/25 & 128 & 126 & 128\tabularnewline
/26 & 192 & 62 & 64\tabularnewline
/27 & 224 & 30 & 32\tabularnewline
/28 & 240 & 14 & 16\tabularnewline
/29 & 248 & 6 & 8\tabularnewline
/30 & 252 & 2 & 4\tabularnewline
\bottomrule
\end{longtable}

The next step is to create a VLSM table.
\protect\hyperlink{c05.xhtmlux5cux23figure05-3}{Figure 5.3} shows you
the table used in creating a VLSM network. The reason we use this table
is so we don't accidentally overlap networks.

You'll find the sheet shown in
\protect\hyperlink{c05.xhtmlux5cux23figure05-3}{Figure 5.3} very
valuable because it lists every block size you can use for a network
address. Notice that the block sizes start at 4 and advance all the way
up to a block size of 128. If you have two networks with block sizes of
128, you can have only 2 networks. With a block size of 64, you can have
only 4, and so on, all the way to 64 networks using a block size of 4.
Of course, this is assuming you're using the \texttt{ip\ subnet-zero}
command in your network design.

So now all you need to do is fill in the chart in the lower-left corner,
then add the subnets to the worksheet and you're good to go!

Based on what you've learned so far about block sizes and the VLSM
table, let's create a VLSM network using a Class C network address
192.168.10.0 for the network in
\protect\hyperlink{c05.xhtmlux5cux23figure05-4}{Figure 5.4}, then fill
out the VLSM table, as shown in
\protect\hyperlink{c05.xhtmlux5cux23figure05-5}{Figure 5.5}.

In \protect\hyperlink{c05.xhtmlux5cux23figure05-4}{Figure 5.4}, we have
four WAN links and four LANs connected together, so we need to create a
VLSM network that will save address space. Looks like we have two block
sizes of 32, a block size of 16, and a block size of 8, and our WANs
each have a block size of 4. Take a look and see how I filled out our
VLSM chart in \protect\hyperlink{c05.xhtmlux5cux23figure05-5}{Figure
5.5}.

\protect\hypertarget{c05.xhtmlux5cux23Page_180}{}{}

\begin{figure}
\centering
\includegraphics{images/c05f003.jpg}
\caption{{\protect\hyperlink{c05.xhtmlux5cux23figureanchor05-3}{\textbf{FIGURE
5.3}} The VLSM table}}
\end{figure}

\protect\hypertarget{c05.xhtmlux5cux23Page_181}{}{}

\begin{figure}
\centering
\includegraphics{images/c05f004.jpg}
\caption{{\protect\hyperlink{c05.xhtmlux5cux23figureanchor05-4}{\textbf{FIGURE
5.4}} VLSM network example 1}}
\end{figure}

\begin{figure}
\centering
\includegraphics{images/c05f005.jpg}
\caption{{\protect\hyperlink{c05.xhtmlux5cux23figureanchor05-5}{\textbf{FIGURE
5.5}} VLSM table example 1}}
\end{figure}

\protect\hypertarget{c05.xhtmlux5cux23Page_182}{}{}There are two
important things to note here. The first is that we still have plenty of
room for growth with this VLSM network design. The second point is that
we could never achieve this goal with one subnet mask using classful
routing.

Let's do another one.
\protect\hyperlink{c05.xhtmlux5cux23figure05-6}{Figure 5.6} shows a
network with 11 networks, two block sizes of 64, one of 32, five of 16,
and three of 4.

\begin{figure}
\centering
\includegraphics{images/c05f006.jpg}
\caption{{\protect\hyperlink{c05.xhtmlux5cux23figureanchor05-6}{\textbf{FIGURE
5.6}} VLSM network example 2}}
\end{figure}

First, create your VLSM table and use your block size chart to fill in
the table with the subnets you need.
\protect\hyperlink{c05.xhtmlux5cux23figure05-7}{Figure 5.7} shows a
possible solution.

Notice that I filled in this entire chart and only have room for one
more block size of 4. You can only gain that amount of address space
savings with a VLSM network!

Keep in mind that it doesn't matter where you start your block sizes as
long as you always begin counting from zero. For example, if you had a
block size of 16, you must start at 0 and incrementally progress from
there---0, 16, 32, 48, and so on. You can't start with a block size of
16 or some value like 40, and you can't progress using anything but
increments of 16.

Here's another example. If you had block sizes of 32, start at zero like
this: 0, 32, 64, 96, etc. Again, you don't get to start wherever you
want; you must always start counting from zero. In the example in
\protect\hyperlink{c05.xhtmlux5cux23figure05-7}{Figure 5.7}, I started
at 64 and 128, with my two block sizes of 64. I didn't have much choice
because my options are 0, 64, 128, and 192. However, I added the block
size of 32, 16, 8, and 4 elsewhere, but they were always in the correct
increments required of the specific block size. Remember that if you
always start with the largest blocks first, then make your way to the
smaller blocks sizes, you will automatically fall on an increment
boundary. It also guarantees that you are using your address space in
the most effective way.

Okay---you have three locations you need to address, and the IP network
you have received is 192.168.55.0 to use as the addressing for the
entire network. You'll use \texttt{ip\ subnet-zero} and RIPv2 as the
routing protocol because RIPv2 supports VLSM networks but
\protect\hypertarget{c05.xhtmlux5cux23Page_183}{}{}\protect\hypertarget{c05.xhtmlux5cux23Page_184}{}{}RIPv1
does not. \protect\hyperlink{c05.xhtmlux5cux23figure05-8}{Figure 5.8}
shows the network diagram and the IP address of the RouterA S0/0
interface.

\begin{figure}
\centering
\includegraphics{images/c05f007.jpg}
\caption{{\protect\hyperlink{c05.xhtmlux5cux23figureanchor05-7}{\textbf{FIGURE
5.7}} VLSM table example 2}}
\end{figure}

\begin{figure}
\centering
\includegraphics{images/c05f008.jpg}
\caption{{\protect\hyperlink{c05.xhtmlux5cux23figureanchor05-8}{\textbf{FIGURE
5.8}} VLSM design example 1}}
\end{figure}

From the list of IP addresses on the right of the figure, which IP
address do you think will be placed in each router's FastEthernet 0/0
interface and serial 0/0 of RouterB?

To answer this, look for clues in
\protect\hyperlink{c05.xhtmlux5cux23figure05-8}{Figure 5.8}. The first
is that interface S0/0 on RouterA has IP address 192.168.55.2/30
assigned, which makes for an easy answer because A /30 is
255.255.255.252, which gives you a block size of 4. Your subnets are 0,
4, 8, etc. Since the known host has an IP address of 2, the only other
valid host in the zero subnet is 1, so the third answer down is the
right one for the S0/0 interface of RouterB.

The next clues are the listed number of hosts for each of the LANs.
RouterA needs 7 hosts---a block size of 16 (/28). RouterB needs 90
hosts---a block size of 128 (/25). And RouterC needs 23 hosts---a block
size of 32 (/27).

\protect\hyperlink{c05.xhtmlux5cux23figure05-9}{Figure 5.9} illustrates
this solution.

\begin{figure}
\centering
\includegraphics{images/c05f009.jpg}
\caption{{\protect\hyperlink{c05.xhtmlux5cux23figureanchor05-9}{\textbf{FIGURE
5.9}} Solution to VLSM design example 1}}
\end{figure}

\protect\hypertarget{c05.xhtmlux5cux23Page_185}{}{}This is actually
pretty simple because once you've figured out the block size needed for
each LAN, all you need to get to the right solution is to identify
proper clues and, of course, know your block sizes well!

One last example of VLSM design before we move on to summarization.
\protect\hyperlink{c05.xhtmlux5cux23figure05-10}{Figure 5.10} shows
three routers, all running RIPv2. Which Class C addressing scheme would
you use to maintain the needs of this network while saving as much
address space as possible?

\begin{figure}
\centering
\includegraphics{images/c05f010.jpg}
\caption{{\protect\hyperlink{c05.xhtmlux5cux23figureanchor05-10}{\textbf{FIGURE
5.10}} VLSM design example 2}}
\end{figure}

This is actually a pretty clean network design that's just waiting for
you to fill out the chart. There are block sizes of 64, 32, and 16 and
two block sizes of 4. Coming up with the right solution should be a slam
dunk! Take a look at my answer in
\protect\hyperlink{c05.xhtmlux5cux23figure05-11}{Figure 5.11}.

\begin{figure}
\centering
\includegraphics{images/c05f011.jpg}
\caption{{\protect\hyperlink{c05.xhtmlux5cux23figureanchor05-11}{\textbf{FIGURE
5.11}} Solution to VLSM design example 2}}
\end{figure}

My solution began at subnet 0, and I used the block size of 64. Clearly,
I didn't have to go with a block size of 64 because I could've chosen a
block size of 4 instead. But I didn't
\protect\hypertarget{c05.xhtmlux5cux23Page_186}{}{}because I usually
like to start with the largest block size and move to the smallest. With
that done, I added the block sizes of 32 and 16 as well as the two block
sizes of 4. This solution is optimal because it still leaves lots of
room to add subnets to this network!

\begin{center}\rule{0.5\linewidth}{0.5pt}\end{center}

\subsubsection{Why Bother with VLSM Design?}

You have just been hired by a new company and need to add on to their
existing network. There are no restrictions to prevent you from starting
over with a completely new IP address scheme. Should you use a VLSM
classless network or opt for a classful network?

Let's say you happen to have plenty of address space because you're
using the Class A 10.0.0.0 private network address, so you really can't
imagine that you'd ever run out of IP addresses. So why would you want
to bother with the VLSM design process in this environment?

Good question! Here's your answer\ldots{}

By creating contiguous blocks of addresses to specific areas of your
network, you can then easily summarize the network and keep route
updates with a routing protocol to a minimum. Why would anyone want to
advertise hundreds of networks between buildings when you can just send
one summary route between buildings and achieve the same result? This
approach will optimize the network's performance dramatically!

To make sure this is clear, let me take a second to explain summary
routes. Summarization, also called supernetting, provides route updates
in the most efficient way possible by advertising many routes in one
advertisement instead of individually. This saves a ton of bandwidth and
minimizes router processing. As always, you need to use blocks of
addresses to configure your summary routes and watch your network's
performance hum along efficiently! And remember, block sizes are used in
all sorts of networks anyway.

Still, it's important to understand that summarization works only if you
design your network properly. If you carelessly hand out IP subnets to
any location on the network, you'll quickly notice that you no longer
have any summary boundaries. And you won't get very far creating summary
routes without those, so watch your step!

\begin{center}\rule{0.5\linewidth}{0.5pt}\end{center}

\subsection[Summarization]{\texorpdfstring{\protect\hypertarget{c05.xhtmlux5cux23c05-sec-4}{}{}Summarization}{Summarization}}

Summarization, also called route aggregation, allows routing protocols
to advertise many networks as one address. The purpose of this is to
reduce the size of routing tables on routers to save memory, which also
shortens the amount of time IP requires to parse the routing table when
determining the best path to a remote network.

\protect\hypertarget{c05.xhtmlux5cux23Page_187}{}{}\protect\hyperlink{c05.xhtmlux5cux23figure05-12}{Figure
5.12} shows how a summary address would be used in an internetwork.

\begin{figure}
\centering
\includegraphics{images/c05f012.jpg}
\caption{{\protect\hyperlink{c05.xhtmlux5cux23figureanchor05-12}{\textbf{FIGURE
5.12}} Summary address used in an internetwork}}
\end{figure}

Summarization is pretty straightforward because all you really need to
have down is a solid understanding of the block sizes we've been using
for subnetting and VLSM design. For example, if you wanted to summarize
the following networks into one network advertisement, you just have to
find the block size first, which will make it easy to find your answer:

192.168.16.0 through network 192.168.31.0

Okay---so what's the block size? Well, there are exactly 16 Class C
networks, which fit neatly into a block size of 16.

Now that we've determined the block size, we just need to find the
network address and mask used to summarize these networks into one
advertisement. The network address used to advertise the summary address
is always the first network address in the block---in this example,
192.168.16.0. To figure out a summary mask, we just need to figure out
which mask will get us a block size of 16. If you came up with 240, you
got it right! 240 would be placed in the third octet, which is exactly
the octet where we're summarizing, so the mask would be 255.255.240.0.

Here's another example:

Networks 172.16.32.0 through 172.16.50.0

This isn't as clean as the previous example because there are two
possible answers. Here's why: Since you're starting at network 32, your
options for block sizes are 4, 8, 16, 32, 64, etc., and block sizes of
16 and 32 could work as this summary address. Let's explore your two
options:

\begin{enumerate}
\tightlist
\item
  If you went with a block size of 16, then the network address would be
  172.16.32.0 with a mask of 255.255.240.0 (240 provides a block of 16).
  The problem is that this only summarizes from 32 to 47, which means
  that networks 48 through 50 would be advertised as single networks.
  Even so, this could still be a good solution depending on your network
  design.
\item
  If you decided to go with a block size of 32 instead, then your
  summary address would still be 172.16.32.0, but the mask would be
  255.255.224.0 (224 provides a block of 32). The possible problem with
  this answer is that it will summarize networks 32 through 63 and we
  only have networks 32 to 50. No worries if you're planning on adding
  networks 51 to 63 later into the same network, but you could have
  serious problems in your internetwork if somehow networks 51 to 63
  were to show up and be advertised from somewhere else in your network!
  So even though this option does allow for growth, it's a lot safer to
  go with option \#1.
\end{enumerate}

\protect\hypertarget{c05.xhtmlux5cux23Page_188}{}{}Let's take a look at
another example: Your summary address is 192.168.144.0/20, so what's the
range of host addresses that would be forwarded according to this
summary? The /20 provides a summary address of 192.168.144.0 and mask of
255.255.240.0.

The third octet has a block size of 16, and starting at summary address
144, the next block of 16 is 160, so your network summary range is 144
to 159 in the third octet. This is why it comes in handy to be able to
count in 16s!

A router with this summary address in the routing table will forward any
packet having destination IP addresses of 192.168.144.1 through
192.168.159.254.

Only two more summarization examples, then we'll move on to
troubleshooting.

In summarization example 4,
\protect\hyperlink{c05.xhtmlux5cux23figure05-13}{Figure 5.13}, the
Ethernet networks connected to router R1 are being summarized to R2 as
192.168.144.0/20. Which range of IP addresses will R2 forward to R1
according to this summary?

\begin{figure}
\centering
\includegraphics{images/c05f013.jpg}
\caption{{\protect\hyperlink{c05.xhtmlux5cux23figureanchor05-13}{\textbf{FIGURE
5.13}} Summarization example 4}}
\end{figure}

No worries---solving this is easier than it looks initially. The
question actually has the summary address listed in it:
192.168.144.0/20. You already know that /20 is 255.255.240.0, which
means you've got a block size of 16 in the third octet. Starting at 144,
which is also right there in the question, makes the next block size of
16 equal 160. You can't go above 159 in the third octet, so the IP
addresses that will be forwarded are 192.168.144.1 through
192.168.159.254.

Okay, last one. In
\protect\hyperlink{c05.xhtmlux5cux23figure05-14}{Figure 5.14}, there are
five networks connected to router R1. What's the best summary address to
R2?

\begin{figure}
\centering
\includegraphics{images/c05f014.jpg}
\caption{{\protect\hyperlink{c05.xhtmlux5cux23figureanchor05-14}{\textbf{FIGURE
5.14}} Summarization example 5}}
\end{figure}

\protect\hypertarget{c05.xhtmlux5cux23Page_189}{}{}I'll be honest with
you---this is a much harder question than the one in
\protect\hyperlink{c05.xhtmlux5cux23figure05-13}{Figure 5.13}, so you're
going to have to look carefully to see the answer. A good approach here
would be to write down all the networks and see if you can find anything
in common with all of them:

\begin{enumerate}
\tightlist
\item
  172.1.4.128/25
\item
  172.1.7.0/24
\item
  172.1.6.0/24
\item
  172.1.5.0/24
\item
  172.1.4.0/25
\end{enumerate}

Do you see an octet that looks interesting to you? I do. It's the third
octet. 4, 5, 6, 7, and yes, it's a block size of 4. So you can summarize
172.1.4.0 using a mask of 255.255.252.0, meaning you would use a block
size of 4 in the third octet. The IP addresses forwarded with this
summary would be 172.1.4.1 through 172.1.7.254.

To summarize the summarization section, if you've nailed down your block
sizes, then finding and applying summary addresses and masks is a
relatively straightforward task. But you're going to get bogged down
pretty quickly if you don't know what a /20 is or if you can't count by
16s!

\subsection[Troubleshooting IP
Addressing]{\texorpdfstring{\protect\hypertarget{c05.xhtmlux5cux23c05-sec-5}{}{}Troubleshooting
IP Addressing}{Troubleshooting IP Addressing}}

Because running into trouble now and then in networking is a given,
being able to troubleshoot IP addressing is clearly a vital skill. I'm
not being negative here---just realistic. The positive side to this is
that if you're the one equipped with the tools to diagnose and clear up
the inevitable trouble, you get to be the hero when you save the day!
Even better? You can usually fix an IP network regardless of whether
you're on site or at home!

So this is where I'm going to show you the ``Cisco way'' of
troubleshooting IP addressing. Let's use
\protect\hyperlink{c05.xhtmlux5cux23figure05-15}{Figure 5.15} as an
example of your basic IP trouble---poor Sally can't log in to the
Windows server. Do you deal with this by calling the Microsoft team to
tell them their server is a pile of junk and causing all your problems?
Though tempting, a better approach is to first double-check and verify
your network instead.

\begin{figure}
\centering
\includegraphics{images/c05f015.jpg}
\caption{{\protect\hyperlink{c05.xhtmlux5cux23figureanchor05-15}{\textbf{FIGURE
5.15}} Basic IP troubleshooting}}
\end{figure}

\protect\hypertarget{c05.xhtmlux5cux23Page_190}{}{}Okay, let's get
started by going through the troubleshooting steps that Cisco
recommends. They're pretty simple, but important nonetheless. Pretend
you're at a customer host and they're complaining that they can't
communicate to a server that just happens to be on a remote network.
Here are the four troubleshooting steps Cisco recommends:

\begin{enumerate}
\item
  Open a Command window and ping 127.0.0.1. This is the diagnostic, or
  loopback, address, and if you get a successful ping, your IP stack is
  considered initialized. If it fails, then you have an IP stack failure
  and need to reinstall TCP/IP on the host.

\begin{verbatim}
C:\>ping 127.0.0.1
Pinging 127.0.0.1 with 32 bytes of data:
Reply from 127.0.0.1: bytes=32 time<1ms TTL=128
Reply from 127.0.0.1: bytes=32 time<1ms TTL=128
Reply from 127.0.0.1: bytes=32 time<1ms TTL=128
Reply from 127.0.0.1: bytes=32 time<1ms TTL=128
Ping statistics for 127.0.0.1:
    Packets: Sent &#x0003D; 4, Received = 4, Lost = 0 (0% loss),
Approximate round trip times in milli-seconds:
    Minimum = 0ms, Maximum = 0ms, Average = 0ms
\end{verbatim}
\item
  From the Command window, ping the IP address of the local host (we'll
  assume correct configuration here, but always check the IP
  configuration too!). If that's successful, your network interface card
  (NIC) is functioning. If it fails, there is a problem with the NIC.
  Success here doesn't just mean that a cable is plugged into the NIC,
  only that the IP protocol stack on the host can communicate to the NIC
  via the LAN driver.

\begin{verbatim}
C:\>ping 172.16.10.2
Pinging 172.16.10.2 with 32 bytes of data:
Reply from 172.16.10.2: bytes=32 time<1ms TTL=128
Reply from 172.16.10.2: bytes=32 time<1ms TTL=128
Reply from 172.16.10.2: bytes=32 time<1ms TTL=128
Reply from 172.16.10.2: bytes=32 time<1ms TTL=128
Ping statistics for 172.16.10.2:
    Packets: Sent = 4, Received = 4, Lost = 0 (0% loss),
Approximate round trip times in milli-seconds:
    Minimum = 0ms, Maximum = 0ms, Average = 0ms
\end{verbatim}
\item
  From the Command window, ping the default gateway (router). If the
  ping works, it means that the NIC is plugged into the network and can
  communicate on the local network. If it fails, you have a local
  physical network problem that could be anywhere from the NIC to the
  router.

\begin{verbatim}
C:\>ping 172.16.10.1
Pinging 172.16.10.1 with 32 bytes of data:
Reply from 172.16.10.1: bytes=32 time<1ms TTL=128
Reply from 172.16.10.1: bytes=32 time<1ms TTL=128
Reply from 172.16.10.1: bytes=32 time<1ms TTL=128
Reply from 172.16.10.1: bytes=32 time<1ms TTL=128
Ping statistics for 172.16.10.1:
    Packets: Sent = 4, Received = 4, Lost = 0 (0% loss),
Approximate round trip times in milli-seconds:
    Minimum = 0ms, Maximum = 0ms, Average = 0ms
\end{verbatim}
\item
  If steps 1 through 3 were successful, try to ping the remote server.
  If that works, then you know that you have IP communication between
  the local host and the remote server. You also know that the remote
  physical network is working.

\begin{verbatim}
C:\>ping 172.16.20.2
Pinging 172.16.20.2 with 32 bytes of data:
Reply from 172.16.20.2: bytes=32 time<1ms TTL=128
Reply from 172.16.20.2: bytes=32 time<1ms TTL=128
Reply from 172.16.20.2: bytes=32 time<1ms TTL=128
Reply from 172.16.20.2: bytes=32 time<1ms TTL=128
Ping statistics for 172.16.20.2:
    Packets: Sent = 4, Received = 4, Lost = 0 (0% loss),
Approximate round trip times in milli-seconds:
    Minimum = 0ms, Maximum = 0ms, Average = 0ms
\end{verbatim}
\end{enumerate}

If the user still can't communicate with the server after steps 1
through 4 have been completed successfully, you probably have some type
of name resolution problem and need to check your Domain Name System
(DNS) settings. But if the ping to the remote server fails, then you
know you have some type of remote physical network problem and need to
go to the server and work through steps 1 through 3 until you find the
snag.

Before we move on to determining IP address problems and how to fix
them, I just want to mention some basic commands that you can use to
help troubleshoot your network from both a PC and a Cisco router. Keep
in mind that though these commands may do the same thing, they're
implemented differently.

\texttt{ping} Uses ICMP echo request and replies to test if a node IP
stack is initialized and alive on the network.

\texttt{traceroute} Displays the list of routers on a path to a network
destination by using TTL time-outs and ICMP error messages. This command
will not work from a command prompt.

\texttt{tracert} Same function as \texttt{traceroute}, but it's a
Microsoft Windows command and will not work on a Cisco router.

\texttt{arp\ -a} Displays IP-to-MAC-address mappings on a Windows PC.

\texttt{show\ ip\ arp} Same function as \texttt{arp\ -a}, but displays
the ARP table on a Cisco router. Like the commands \texttt{traceroute}
and \texttt{tracert}, \texttt{arp\ -a} and \texttt{show\ ip\ arp} are
not interchangeable through DOS and Cisco.

\protect\hypertarget{c05.xhtmlux5cux23Page_192}{}{}\texttt{ipconfig\ /all}
Used only from a Windows command prompt; shows you the PC network
configuration.

Once you've gone through all these steps and, if necessary, used the
appropriate commands, what do you do when you find a problem? How do you
go about fixing an IP address configuration error? Time to cover the
next step---determining and fixing the issue at hand!

\subsubsection[Determining IP Address
Problems]{\texorpdfstring{\protect\hypertarget{c05.xhtmlux5cux23c05-sec-6}{}{}Determining
IP Address Problems}{Determining IP Address Problems}}

It's common for a host, router, or other network device to be configured
with the wrong IP address, subnet mask, or default gateway. Because this
happens way too often, you must know how to find and fix IP address
configuration errors.

A good way to start is to draw out the network and IP addressing scheme.
If that's already been done, consider yourself lucky because though
sensible, it's rarely done. Even if it is, it's usually outdated or
inaccurate anyway. So either way, it's a good idea to bite the bullet
and start from scratch.

\begin{center}\rule{0.5\linewidth}{0.5pt}\end{center}

\includegraphics{images/note.png} I'll show you how a great way to draw
out your network using the Cisco Discovery Protocol (CDP) soon, in
Chapter 7, ``Managing a Cisco Internetwork.''

\begin{center}\rule{0.5\linewidth}{0.5pt}\end{center}

Once you have your network accurately drawn out, including the IP
addressing scheme, you need to verify each host's IP address, mask, and
default gateway address to establish the problem. Of course, this is
assuming that you don't have a physical layer problem, or if you did,
that you've already fixed it.

Let's check out the example illustrated in
\protect\hyperlink{c05.xhtmlux5cux23figure05-16}{Figure 5.16}.

\begin{figure}
\centering
\includegraphics{images/c05f016.jpg}
\caption{{\protect\hyperlink{c05.xhtmlux5cux23figureanchor05-16}{\textbf{FIGURE
5.16}} IP address problem 1}}
\end{figure}

\protect\hypertarget{c05.xhtmlux5cux23Page_193}{}{}A user in the sales
department calls and tells you that she can't get to ServerA in the
marketing department. You ask her if she can get to ServerB in the
marketing department, but she doesn't know because she doesn't have
rights to log on to that server. What do you do?

First, guide your user through the four troubleshooting steps you
learned in the preceding section. Okay---let's say steps 1 through 3
work but step 4 fails. By looking at the figure, can you determine the
problem? Look for clues in the network drawing. First, the WAN link
between the Lab A router and the Lab B router shows the mask as a /27.
You should already know that this mask is 255.255.255.224 and determine
that all networks are using this mask. The network address is
192.168.1.0. What are our valid subnets and hosts? 256 -- 224 = 32, so
this makes our subnets 0, 32, 64, 96, 128, etc. So, by looking at the
figure, you can see that subnet 32 is being used by the sales
department. The WAN link is using subnet 96, and the marketing
department is using subnet 64.

Now you've got to establish what the valid host ranges are for each
subnet. From what you learned at the beginning of this chapter, you
should now be able to easily determine the subnet address, broadcast
addresses, and valid host ranges. The valid hosts for the Sales LAN are
33 through 62, and the broadcast address is 63 because the next subnet
is 64, right? For the Marketing LAN, the valid hosts are 65 through 94
(broadcast 95), and for the WAN link, 97 through 126 (broadcast 127). By
closely examining the figure, you can determine that the default gateway
on the Lab B router is incorrect. That address is the broadcast address
for subnet 64, so there's no way it could be a valid host!

\begin{center}\rule{0.5\linewidth}{0.5pt}\end{center}

\includegraphics{images/tip.png} If you tried to configure that address
on the Lab B router interface, you'd receive a \texttt{bad\ mask} error.
Cisco routers don't let you type in subnet and broadcast addresses as
valid hosts!

\begin{center}\rule{0.5\linewidth}{0.5pt}\end{center}

Did you get all that? Let's try another one to make sure.
\protect\hyperlink{c05.xhtmlux5cux23figure05-17}{Figure 5.17} shows a
network problem.

\begin{figure}
\centering
\includegraphics{images/c05f017.jpg}
\caption{{\protect\hyperlink{c05.xhtmlux5cux23figureanchor05-17}{\textbf{FIGURE
5.17}} IP address problem 2}}
\end{figure}

\protect\hypertarget{c05.xhtmlux5cux23Page_194}{}{}A user in the Sales
LAN can't get to ServerB. You have the user run through the four basic
troubleshooting steps and find that the host can communicate to the
local network but not to the remote network. Find and define the IP
addressing problem.

If you went through the same steps used to solve the last problem, you
can see that first, the WAN link again provides the subnet mask to
use--- /29, or 255.255.255.248. Assuming classful addressing, you need
to determine what the valid subnets, broadcast addresses, and valid host
ranges are to solve this problem.

The 248 mask is a block size of 8 (256 -- 248 = 8, as discussed in
Chapter 4), so the subnets both start and increment in multiples of 8.
By looking at the figure, you see that the Sales LAN is in the 24
subnet, the WAN is in the 40 subnet, and the Marketing LAN is in the 80
subnet. Can you see the problem yet? The valid host range for the Sales
LAN is 25--30, and the configuration appears correct. The valid host
range for the WAN link is 41--46, and this also appears correct. The
valid host range for the 80 subnet is 81--86, with a broadcast address
of 87 because the next subnet is 88. ServerB has been configured with
the broadcast address of the subnet.

Okay, now that you can figure out misconfigured IP addresses on hosts,
what do you do if a host doesn't have an IP address and you need to
assign one? What you need to do is scrutinize the other hosts on the LAN
and figure out the network, mask, and default gateway. Let's take a look
at a couple of examples of how to find and apply valid IP addresses to
hosts.

You need to assign a server and router IP addresses on a LAN. The subnet
assigned on that segment is 192.168.20.24/29. The router needs to be
assigned the first usable address and the server needs the last valid
host ID. What is the IP address, mask, and default gateway assigned to
the server?

To answer this, you must know that a /29 is a 255.255.255.248 mask,
which provides a block size of 8. The subnet is known as 24, the next
subnet in a block of 8 is 32, so the broadcast address of the 24 subnet
is 31 and the valid host range is 25--30.

Server IP address: 192.168.20.30

Server mask: 255.255.255.248

Default gateway: 192.168.20.25 (router's IP address)

Take a look at \protect\hyperlink{c05.xhtmlux5cux23figure05-18}{Figure
5.18} and solve this problem.

\begin{figure}
\centering
\includegraphics{images/c05f018.jpg}
\caption{{\protect\hyperlink{c05.xhtmlux5cux23figureanchor05-18}{\textbf{FIGURE
5.18}} Find the valid host \#1}}
\end{figure}

\protect\hypertarget{c05.xhtmlux5cux23Page_195}{}{}Look at the router's
IP address on Ethernet0. What IP address, subnet mask, and valid host
range could be assigned to the host?

The IP address of the router's Ethernet0 is 192.168.10.33/27. As you
already know, a /27 is a 224 mask with a block size of 32. The router's
interface is in the 32 subnet. The next subnet is 64, so that makes the
broadcast address of the 32 subnet 63 and the valid host range 33--62.

Host IP address: 192.168.10.34--62 (any address in the range except for
33, which is assigned to the router)

Mask: 255.255.255.224

Default gateway: 192.168.10.33

\protect\hyperlink{c05.xhtmlux5cux23figure05-19}{Figure 5.19} shows two
routers with Ethernet configurations already assigned. What are the host
addresses and subnet masks of HostA and HostB?

\begin{figure}
\centering
\includegraphics{images/c05f019.jpg}
\caption{{\protect\hyperlink{c05.xhtmlux5cux23figureanchor05-19}{\textbf{FIGURE
5.19}} Find the valid host \#2}}
\end{figure}

Router A has an IP address of 192.168.10.65/26 and Router B has an IP
address of 192.168.10.33/28. What are the host configurations? Router A
Ethernet0 is in the 192.168.10.64 subnet and Router B Ethernet0 is in
the 192.168.10.32 network.

\begin{enumerate}
\tightlist
\item
  Host A IP address: 192.168.10.66--126
\item
  Host A mask: 255.255.255.192
\item
  Host A default gateway: 192.168.10.65
\item
  Host B IP address: 192.168.10.34--46
\item
  Host B mask: 255.255.255.240
\item
  Host B default gateway: 192.168.10.33
\end{enumerate}

Just a couple more examples before you can put this chapter behind
you---hang in there!

\protect\hyperlink{c05.xhtmlux5cux23figure05-20}{Figure 5.20} shows two
routers. You need to configure the S0/0 interface on RouterA. The IP
address assigned to the serial link is 172.16.17.0/22. What IP address
can be assigned?

\protect\hypertarget{c05.xhtmlux5cux23Page_196}{}{}

\begin{figure}
\centering
\includegraphics{images/c05f020.jpg}
\caption{{\protect\hyperlink{c05.xhtmlux5cux23figureanchor05-20}{\textbf{FIGURE
5.20}} Find the valid host address \#3}}
\end{figure}

First, know that a /22 CIDR is 255.255.252.0, which makes a block size
of 4 in the third octet. Since 17 is listed, the available range is 16.1
through 19.254, so in this example, the IP address S0/0 could be
172.16.18.255 since that's within the range.

Okay, last one! You need to find a classful network address that has one
Class C network ID and you need to provide one usable subnet per city
while allowing enough usable host addresses for each city specified in
\protect\hyperlink{c05.xhtmlux5cux23figure05-21}{Figure 5.21}. What is
your mask?

\begin{figure}
\centering
\includegraphics{images/c05f021.jpg}
\caption{{\protect\hyperlink{c05.xhtmlux5cux23figureanchor05-21}{\textbf{FIGURE
5.21}} Find the valid subnet mask}}
\end{figure}

Actually, this is probably the easiest thing you've done all day! I
count 5 subnets needed, and the Wyoming office needs 16 users---always
look for the network that needs the most hosts! What block size is
needed for the Wyoming office? Your answer is 32. You can't use a block
size of 16 because you always have to subtract 2. What mask provides you
with a block size of 32? 224 is your answer because this provides 8
subnets, each with 30 hosts.

You're done---the diva has sung and the chicken has safely crossed the
road\ldots whew! Time to take a break, but skip the shot and the beer if
that's what you had in mind because you need to have your head straight
to go through the written lab and review questions next!

\subsection[Summary]{\texorpdfstring{\protect\hypertarget{c05.xhtmlux5cux23c05-sec-7}{}{}Summary}{Summary}}

Again, if you got to this point without getting lost along the way a few
times, you're awesome, but if you did get lost, don't stress because
most people do! Just be patient with yourself and go back over the
material that tripped you up until it's all crystal clear. You'll get
there!

This chapter provided you with keys to understanding the
oh-so-very-important topic of variable length subnet masks. You should
also know how to design and implement simple VLSM networks and be clear
on summarization as well.

\protect\hypertarget{c05.xhtmlux5cux23Page_197}{}{}And make sure you
understand and memorize Cisco's troubleshooting methods. You must
remember the four steps that Cisco recommends to take when trying to
narrow down exactly where a network and/or IP addressing problem is and
then know how to proceed systematically to fix it. In addition, you
should be able to find valid IP addresses and subnet masks by looking at
a network diagram.

\subsection[Exam
Essentials]{\texorpdfstring{\protect\hypertarget{c05.xhtmlux5cux23c05-sec-8}{}{}Exam
Essentials}{Exam Essentials}}

\textbf{Describe the benefits of variable length subnet masks (VLSMs).}
VLSMs enable the creation of subnets of specific sizes and allow the
division of a classless network into smaller networks that do not need
to be equal in size. This makes use of the address space more efficient
because many times IP addresses are wasted with classful subnetting.

\textbf{Understand the relationship between the subnet mask value and
the resulting block size and the allowable IP addresses in each
resulting subnet.} The relationship between the classful network being
subdivided and the subnet mask used determines the number of possible
hosts or the block size. It also determines where each subnet begins and
ends and which IP addresses cannot be assigned to a host within each
subnet.

\textbf{Describe the process of summarization or route aggregation and
its relationship to subnetting.} Summarization is the combining of
subnets derived from a classful network for the purpose of advertising a
single route to neighboring routers instead of multiple routes, reducing
the size of routing tables and speeding the route process.

\textbf{Calculate the summary mask that will advertise a single network
representing all subnets.} The network address used to advertise the
summary address is always the first network address in the block of
subnets. The mask is the subnet mask value that yields the same block
size.

\textbf{Remember the four diagnostic steps.} The four simple steps that
Cisco recommends for troubleshooting are ping the loopback address, ping
the NIC, ping the default gateway, and ping the remote device.

\textbf{Identify and mitigate an IP addressing problem.} Once you go
through the four troubleshooting steps that Cisco recommends, you must
be able to determine the IP addressing problem by drawing out the
network and finding the valid and invalid hosts addressed in your
network.

\textbf{Understand the troubleshooting tools that you can use from your
host and a Cisco router.} The \texttt{ping\ 127.0.0.1} command tests
your local IP stack, and \texttt{tracert} is a Windows command to track
the path a packet takes through an internetwork to a destination. Cisco
routers use the command \texttt{traceroute}, or just \texttt{trace} for
short. Don't confuse the Windows and Cisco commands. Although they
produce the same output, they don't work from the same prompts. The
command \texttt{ipconfig\ /all} will display your PC network
configuration from a DOS prompt, and \texttt{arp\ -a} (again from a DOS
prompt) will display IP-to-MAC-address mapping on a Windows PC.

\subsection[Written Lab
5]{\texorpdfstring{\protect\hypertarget{c05.xhtmlux5cux23c05-sec-9}{}{}\protect\hypertarget{c05.xhtmlux5cux23Page_198}{}{}Written
Lab 5}{Written Lab 5}}

In this section, you'll complete the following lab to make sure you've
got the information and concepts contained within them fully dialed in:

Lab 5.1: Summarization Practice

You can find the answers to this lab in Appendix A, ``Answers to Written
Labs.''

\subsubsection[Lab 5.1: Summarization
Practice]{\texorpdfstring{\protect\hypertarget{c05.xhtmlux5cux23c05-sec-10}{}{}Lab
5.1: Summarization Practice}{Lab 5.1: Summarization Practice}}

For each of the following sets of networks, determine the summary
address and the mask to be used that will summarize the subnets.

\begin{enumerate}
\tightlist
\item
  192.168.1.0/24 through 192.168.12.0/24
\item
  172.144.0.0 through 172.159.0.0
\item
  192.168.32.0 through 192.168.63.0
\item
  192.168.96.0 through 192.168.111.0
\item
  66.66.0.0 through 66.66.15.0
\item
  192.168.1.0 through 192.168.120.0
\item
  172.16.1.0 through 172.16.7.0
\item
  192.168.128.0 through 192.168.190.0
\item
  53.60.96.0 through 53.60.127.0
\item
  172.16.10.0 through 172.16.63.0
\end{enumerate}

\subsection[Review
Questions]{\texorpdfstring{\protect\hypertarget{c05.xhtmlux5cux23c05-sec-11}{}{}\protect\hypertarget{c05.xhtmlux5cux23Page_199}{}{}Review
Questions}{Review Questions}}

\begin{center}\rule{0.5\linewidth}{0.5pt}\end{center}

\includegraphics{images/note.png} The following questions are designed
to test your understanding of this chapter's material. For more
information on how to get additional questions, please see
\href{http://www.lammle.com/ccna}{www.lammle.com/ccna}.

\begin{center}\rule{0.5\linewidth}{0.5pt}\end{center}

You can find the answers to these questions in Appendix B, ``Answers to
Review Questions.''

\begin{enumerate}
\item
  On a VLSM network, which mask should you use on point-to-point WAN
  links in order to reduce the waste of IP addresses?

  \begin{enumerate}
  \def\labelenumii{\Alph{enumii}.}
  \tightlist
  \item
    /27
  \item
    /28
  \item
    /29
  \item
    /30
  \item
    /31
  \end{enumerate}
\item
  In the network shown in the diagram, how many computers could be in
  Network B?

  \begin{figure}
  \centering
  \includegraphics{images/c05f022.jpg}
  \caption{}
  \end{figure}

  \begin{enumerate}
  \def\labelenumii{\Alph{enumii}.}
  \tightlist
  \item
    6
  \item
    12
  \item
    14
  \item
    30
  \end{enumerate}
\item
  \protect\hypertarget{c05.xhtmlux5cux23Page_200}{}{}In the following
  diagram, in order to have IP addressing that's as efficient as
  possible, which network should use a /29 mask?

  \begin{figure}
  \centering
  \includegraphics{images/c05f023.jpg}
  \caption{}
  \end{figure}

  \begin{enumerate}
  \def\labelenumii{\Alph{enumii}.}
  \tightlist
  \item
    A
  \item
    B
  \item
    C
  \item
    D
  \end{enumerate}
\item
  To use VLSM, what capability must the routing protocols in use
  possess?

  \begin{enumerate}
  \def\labelenumii{\Alph{enumii}.}
  \tightlist
  \item
    Support for multicast
  \item
    Multiprotocol support
  \item
    Transmission of subnet mask information
  \item
    Support for unequal load balancing
  \end{enumerate}
\item
  What summary address would cover all the networks shown and advertise
  a single, efficient route to Router B that won't advertise more
  networks than needed?

  \begin{figure}
  \centering
  \includegraphics{images/c05f024.jpg}
  \caption{}
  \end{figure}

  \begin{enumerate}
  \def\labelenumii{\Alph{enumii}.}
  \tightlist
  \item
    172.16.0.0/24
  \item
    172.16.1.0/24
  \item
    172.16.0.0/24
  \item
    172.16.0.0/20
  \item
    \protect\hypertarget{c05.xhtmlux5cux23Page_201}{}{}172.16.16.0/28
  \item
    172.16.0.0/27
  \end{enumerate}
\item
  In the following diagram, what is the most likely reason the station
  cannot ping outside of its network?

  \begin{figure}
  \centering
  \includegraphics{images/c05f025.jpg}
  \caption{}
  \end{figure}

  \begin{enumerate}
  \def\labelenumii{\Alph{enumii}.}
  \tightlist
  \item
    The IP address is incorrect on interface E0 of the router.
  \item
    The default gateway address is incorrect on the station.
  \item
    The IP address on the station is incorrect.
  \item
    The router is malfunctioning.
  \end{enumerate}
\item
  If a host is configured with an incorrect default gateway and all the
  other computers and router are known to be configured correctly, which
  of the following statements is TRUE?

  \begin{enumerate}
  \def\labelenumii{\Alph{enumii}.}
  \tightlist
  \item
    Host A cannot communicate with the router.
  \item
    Host A can communicate with other hosts in the same subnet.
  \item
    Host A can communicate with hosts in other subnets.
  \item
    Host A can communicate with no other systems.
  \end{enumerate}
\item
  Which of the following troubleshooting steps, if completed
  successfully, also confirms that the other steps will succeed as well?

  \begin{enumerate}
  \def\labelenumii{\Alph{enumii}.}
  \tightlist
  \item
    Ping a remote computer.
  \item
    Ping the loopback address.
  \item
    Ping the NIC.
  \item
    Ping the default gateway.
  \end{enumerate}
\item
  When a ping to the local host IP address fails, what can you assume?

  \begin{enumerate}
  \def\labelenumii{\Alph{enumii}.}
  \tightlist
  \item
    The IP address of the local host is incorrect.
  \item
    The IP address of the remote host is incorrect.
  \item
    The NIC is not functional.
  \item
    The IP stack has failed to initialize.
  \end{enumerate}
\item
  \protect\hypertarget{c05.xhtmlux5cux23Page_202}{}{}When a ping to the
  local host IP address succeeds but a ping to the default gateway IP
  address fails, what can you rule out? (Choose all that apply.)

  \begin{enumerate}
  \def\labelenumii{\Alph{enumii}.}
  \tightlist
  \item
    The IP address of the local host is incorrect.
  \item
    The IP address of the gateway is incorrect.
  \item
    The NIC is not functional.
  \item
    The IP stack has failed to initialize.
  \end{enumerate}
\item
  Which of the networks in the diagram could use a /29 mask?

  \begin{figure}
  \centering
  \includegraphics{images/c05f026.jpg}
  \caption{}
  \end{figure}

  \begin{enumerate}
  \def\labelenumii{\Alph{enumii}.}
  \tightlist
  \item
    Corporate
  \item
    LA
  \item
    SF
  \item
    NY
  \item
    None
  \end{enumerate}
\item
  What network service is the most likely problem if you can ping a
  computer by IP address but not by name?

  \begin{enumerate}
  \def\labelenumii{\Alph{enumii}.}
  \tightlist
  \item
    DNS
  \item
    DHCP
  \item
    ARP
  \item
    ICMP
  \end{enumerate}
\item
  When you issue the \texttt{ping} command, what protocol are you using?

  \begin{enumerate}
  \def\labelenumii{\Alph{enumii}.}
  \tightlist
  \item
    DNS
  \item
    DHCP
  \item
    ARP
  \item
    ICMP
  \end{enumerate}
\item
  Which of the following commands displays the networks traversed on a
  path to a network destination?

  \begin{enumerate}
  \def\labelenumii{\Alph{enumii}.}
  \tightlist
  \item
    \texttt{ping}
  \item
    \texttt{traceroute}
  \item
    \texttt{pingroute}
  \item
    \texttt{pathroute}
  \end{enumerate}
\item
  What command generated the output shown below?

\begin{verbatim}
Reply from 172.16.10.2: bytes=32 time<1ms TTL=128
Reply from 172.16.10.2: bytes=32 time<1ms TTL=128
Reply from 172.16.10.2: bytes=32 time<1ms TTL=128
Reply from 172.16.10.2: bytes=32 time<1ms TTL=128
\end{verbatim}

  \begin{enumerate}
  \def\labelenumii{\Alph{enumii}.}
  \tightlist
  \item
    \texttt{traceroute}
  \item
    \texttt{show\ ip\ route}
  \item
    \texttt{ping}
  \item
    \texttt{pathping}
  \end{enumerate}
\item
  In the work area, match the command to its function on the right.

  \begin{figure}
  \centering
  \includegraphics{images/c05f027.jpg}
  \caption{}
  \end{figure}
\item
  Which of the following network addresses correctly summarizes the
  three networks shown below efficiently?

  10.0.0.0/16

  10.1.0.0/16

  10.2.0.0/16

  \begin{enumerate}
  \def\labelenumii{\Alph{enumii}.}
  \tightlist
  \item
    10.0.0.0/15
  \item
    10.1.0.0/8
  \item
    10.0.0.0/14
  \item
    10.0.0.8/16
  \end{enumerate}
\item
  What command displays the ARP table on a Cisco router?

  \begin{enumerate}
  \def\labelenumii{\Alph{enumii}.}
  \tightlist
  \item
    \texttt{show\ ip\ arp}
  \item
    \texttt{traceroute}
  \item
    \texttt{arp\ -a}
  \item
    \texttt{tracert}
  \end{enumerate}
\item
  \protect\hypertarget{c05.xhtmlux5cux23Page_204}{}{} What switch must
  be added to the \texttt{ipconfig} command on a PC to verify DNS
  configuration?

  \begin{enumerate}
  \def\labelenumii{\Alph{enumii}.}
  \tightlist
  \item
    \texttt{/dns}
  \item
    \texttt{-dns}
  \item
    \texttt{/all}
  \item
    \texttt{showall}
  \end{enumerate}
\item
  Which of the following is the best summarization of the following
  networks: 192.168.128.0 through 192.168.159.0?

  \begin{enumerate}
  \def\labelenumii{\Alph{enumii}.}
  \tightlist
  \item
    192.168.0.0/24
  \item
    192.168.128.0/16
  \item
    192.168.128.0/19
  \item
    192.168.128.0/20
  \end{enumerate}
\end{enumerate}

\protect\hypertarget{c06.xhtml}{}{}

\section[{Chapter 6}\\
{Cisco's Internetworking Operating System
(IOS)}]{\texorpdfstring{\protect\hypertarget{c06.xhtmlux5cux23c06}{}{}\protect\hypertarget{c06.xhtmlux5cux23Page_205}{}{}{Chapter
6}\\
{Cisco's Internetworking Operating System
(IOS)}}{Chapter 6 Cisco's Internetworking Operating System (IOS)}}

\begin{center}\rule{0.5\linewidth}{0.5pt}\end{center}

\subsection{The following ICND1 exam topics are covered in this
chapter:}

\begin{enumerate}
\tightlist
\item
  \includegraphics{images/tick.png} \textbf{2.0 LAN Switching
  Technologies}
\item
  \includegraphics{images/tick.png} \textbf{2.3 Troubleshoot interface
  and cable issues (collisions, errors, duplex, speed)}
\item
  \includegraphics{images/tick.png} \textbf{5.0 Infrastructure
  Management}
\item
  \includegraphics{images/tick.png} \textbf{5.3 Configure and verify
  initial device configuration}
\item
  \includegraphics{images/tick.png} \textbf{5.4 Configure, verify, and
  troubleshoot basic device hardening}
\item
  \includegraphics{images/tick.png} \textbf{5.4.a Local authentication}
\item
  \includegraphics{images/tick.png} \textbf{5.4.b Secure password}
\item
  \includegraphics{images/tick.png} \textbf{5.4.c Access to device}

  \begin{enumerate}
  \tightlist
  \item
    \includegraphics{images/squ.png} 5.4.c.(i) Voice
  \item
    \includegraphics{images/squ.png} 5.4.c.(ii) Video
  \end{enumerate}
\item
  \includegraphics{images/tick.png} \textbf{5.4.c. (iii) Data}
\item
  \includegraphics{images/tick.png} \textbf{5.4.d Source address
  Telnet/SSH}
\item
  \includegraphics{images/tick.png} \textbf{5.4.e Login banner}
\item
  \includegraphics{images/tick.png} \textbf{5.6 Use Cisco IOS tools to
  troubleshoot and resolve problems}

  \begin{enumerate}
  \tightlist
  \item
    \includegraphics{images/squ.png} 5.6.aPing and traceroute with
    extended option
  \item
    \includegraphics{images/squ.png} 5.6.bTerminal monitor
  \item
    \includegraphics{images/squ.png} 5.6.c Log events
  \end{enumerate}
\end{enumerate}

\protect\hypertarget{c06.xhtmlux5cux23Page_206}{}{}\includegraphics{images/intro.png}It's
time to introduce you to the Cisco Internetwork Operating System (IOS).
The IOS is what runs Cisco routers as well as Cisco's switches, and it's
also what we use to configure these devices.

So that's what you're going to learn about in this chapter. I'm going to
show you how to configure a Cisco IOS device using the Cisco IOS
command-line interface (CLI). Once proficient with this interface,
you'll be able to configure hostnames, banners, passwords, and more as
well as troubleshoot skillfully using the Cisco IOS.

We'll also begin the journey to mastering the basics of router and
switch configurations plus command verifications in this chapter.

I'll start with a basic IOS switch to begin building the network we'll
use throughout this book for configuration examples. Don't forget---I'll
be using both switches and routers throughout this chapter, and we
configure these devices pretty much the same way. Things diverge when we
get to the interfaces where the differences between the two become key,
so pay attention closely when we get to that point!

Just as it was with preceding chapters, the fundamentals presented in
this chapter are important building blocks to have solidly in place
before moving on to the more advanced material coming up in the next
ones.

\begin{center}\rule{0.5\linewidth}{0.5pt}\end{center}

\includegraphics{images/note.png} To find up-to-the minute updates for
this chapter, please see \texttt{www.lammle.com/ccna} or the book's web
page at \texttt{www.sybex.com/go/ccna}.

\begin{center}\rule{0.5\linewidth}{0.5pt}\end{center}

\subsection[The IOS User
Interface]{\texorpdfstring{\protect\hypertarget{c06.xhtmlux5cux23c06-sec-1}{}{}The
IOS User Interface}{The IOS User Interface}}

The \emph{Cisco Internetwork Operating System (IOS)} is the kernel of
Cisco routers as well as all current Catalyst switches. In case you
didn't know, a kernel is the elemental, indispensable part of an
operating system that allocates resources and manages tasks like
low-level hardware interfaces and security.

Coming up, I'll show you the Cisco IOS and how to configure a Cisco
switch using the \emph{command-line interface (CLI)}. By using the CLI,
we can provide access to a Cisco device and provide voice, video, and
data service. . . . The configurations you'll see in this chapter are
exactly the same as they are on a Cisco router.

\subsubsection[Cisco
IOS]{\texorpdfstring{\protect\hypertarget{c06.xhtmlux5cux23c06-sec-2}{}{}Cisco
IOS}{Cisco IOS}}

The Cisco IOS is a proprietary kernel that provides routing, switching,
internetworking, and telecommunications features. The first IOS was
written by William Yeager in 1986 and
\protect\hypertarget{c06.xhtmlux5cux23Page_207}{}{}enabled networked
applications. It runs on most Cisco routers as well as a growing number
of Cisco Catalyst switches, like the Catalyst 2960 and 3560 series
switches used in this book. And it's an essential for the Cisco exam
objectives!

Here's a short list of some important things that the Cisco router IOS
software is responsible for:

\begin{enumerate}
\tightlist
\item
  Carrying network protocols and functions
\item
  Connecting high-speed traffic between devices
\item
  Adding security to control access and stopping unauthorized network
  use
\item
  Providing scalability for ease of network growth and redundancy
\item
  Supplying network reliability for connecting to network resources
\end{enumerate}

You can access the Cisco IOS through the console port of a router or
switch, from a modem into the auxiliary (or aux) port on a router, or
even through Telnet and Secure Shell (SSH). Access to the IOS command
line is called an \emph{EXEC session}.

\subsubsection[Connecting to a Cisco IOS
Device]{\texorpdfstring{\protect\hypertarget{c06.xhtmlux5cux23c06-sec-3}{}{}Connecting
to a Cisco IOS Device}{Connecting to a Cisco IOS Device}}

We connect to a Cisco device to configure it, verify its configuration,
and check statistics, and although there are different approaches to
this, the first place you would usually connect to is the console port.
The \emph{console port} is usually an RJ45, 8-pin modular connection
located at the back of the device, and there may or may not be a
password set on it by default.

\begin{center}\rule{0.5\linewidth}{0.5pt}\end{center}

\includegraphics{images/note.png} Look back into Chapter 2, ``Ethernet
Networking and Data Encapsulation,'' to review how to configure a PC and
enable it to connect to a router console port.

\begin{center}\rule{0.5\linewidth}{0.5pt}\end{center}

You can also connect to a Cisco router through an \emph{auxiliary port},
which is really the same thing as a console port, so it follows that you
can use it as one. The main difference with an auxiliary port is that it
also allows you to configure modem commands so that a modem can be
connected to the router. This is a cool feature because it lets you dial
up a remote router and attach to the auxiliary port if the router is
down and you need to configure it remotely, \emph{out-of-band}. One of
the differences between Cisco routers and switches is that switches do
not have an auxiliary port.

The third way to connect to a Cisco device is \emph{in-band}, through
the program \emph{Telnet} or \emph{Secure Shell (SSH)}. In-band means
configuring the device via the network, the opposite of
\emph{out-of-band}. We covered Telnet and SSH in Chapter 3,
``Introduction to TCP/IP,'' and in this chapter, I'll show you how to
configure access to both of these protocols on a Cisco device.

\protect\hyperlink{c06.xhtmlux5cux23figure6-1}{Figure 6.1} shows an
illustration of a Cisco 2960 switch. Really focus in on all the
different kinds of interfaces and connections! On the right side is the
10/100/1000 uplink. You can use either the UTP port or the fiber port,
but not both at the same time.

\protect\hypertarget{c06.xhtmlux5cux23Page_208}{}{}

\begin{figure}
\centering
\includegraphics{images/c06f001.jpg}
\caption{{\protect\hyperlink{c06.xhtmlux5cux23figureanchor6-1}{\textbf{FIGURE
6.1}} A Cisco 2960 switch}}
\end{figure}

The 3560 switch I'll be using in this book looks a lot like the 2960,
but it can perform layer 3 switching, unlike the 2960, which is limited
to only layer 2 functions.

I also want to take a moment and tell you about the 2800 series router
because that's the router series I'll be using in this book. This router
is known as an Integrated Services Router (ISR) and Cisco has updated it
to the 2900 series, but I still have plenty of 2800 series routers in my
production networks.
\protect\hyperlink{c06.xhtmlux5cux23figure6-2}{Figure 6.2} shows a new
1900 series router. The new ISR series of routers are nice; they are so
named because many services, like security, are built into them. The ISR
series router is a modular device, much faster and a lot sleeker than
the older 2600 series routers, and it's elegantly designed to support a
broad new range of interface options. The new ISR series router can
offer multiple serial interfaces, which can be used for connecting a T1
using a serial V.35 WAN connection. And multiple Fast Ethernet or
Gigabit Ethernet ports can be used on the router, depending on the
model. This router also has one console via an RJ45 connector and
another through the USB port. There is also an auxiliary connection to
allow a console connection via a remote modem.

\begin{figure}
\centering
\includegraphics{images/c06f002.jpg}
\caption{{\protect\hyperlink{c06.xhtmlux5cux23figureanchor6-2}{\textbf{FIGURE
6.2}} A new Cisco 1900 router}}
\end{figure}

You need to keep in mind that for the most part, you get some serious
bang for your buck with the 2800/2900---unless you start adding a bunch
of interfaces to it. You've got to pony up for each one of those little
beauties, so this can really start to add up and fast!

A couple of other series of routers that will set you back a lot less
than the 2800 series are the 1800/1900s, so look into these routers if
you want a less-expensive alternative to the 2800/2900 but still want to
run the same IOS.

So even though I'm going to be using mostly 2800 series routers and
2960/3560 switches throughout this book to demonstrate examples of IOS
configurations, I want to point out that the particular \emph{router}
model you use to practice for the Cisco exam isn't really important. The
\emph{switch} types are, though---you definitely need a couple 2960
switches as well as a 3560 switch if you want to measure up to the exam
objectives!

\begin{center}\rule{0.5\linewidth}{0.5pt}\end{center}

\includegraphics{images/note.png} You can find more information about
all Cisco routers at
\texttt{www.cisco.com/en/US/products/hw/routers/index.html}.

\begin{center}\rule{0.5\linewidth}{0.5pt}\end{center}

\subsubsection[Bringing Up a
Switch]{\texorpdfstring{\protect\hypertarget{c06.xhtmlux5cux23c06-sec-4}{}{}\protect\hypertarget{c06.xhtmlux5cux23Page_209}{}{}Bringing
Up a Switch}{Bringing Up a Switch}}

When you first bring up a Cisco IOS device, it will run a power-on
self-test---a POST. Upon passing that, the machine will look for and
then load the Cisco IOS from flash memory if an IOS file is present,
then expand it into RAM. As you probably know, flash memory is
electronically erasable programmable read-only memory---an EEPROM. The
next step is for the IOS to locate and load a valid configuration known
as the startup-config that will be stored in \emph{nonvolatile RAM
(NVRAM)}.

Once the IOS is loaded and up and running, the startup-config will be
copied from NVRAM into RAM and from then on referred to as the
running-config.

But if a valid startup-config isn't found in NVRAM, your switch will
enter setup mode, giving you a step-by-step dialog to help configure
some basic parameters on it.

You can also enter setup mode at any time from the command line by
typing the command \texttt{setup} from privileged mode, which I'll get
to in a minute. Setup mode only covers some basic commands and generally
isn't really all that helpful. Here's an example:

\begin{verbatim}
Would you like to enter the initial configuration dialog? [yes/no]: y
\end{verbatim}

\begin{verbatim}
At any point you may enter a question mark '?' for help.
Use ctrl-c to abort configuration dialog at any prompt.
Default settings are in square brackets '[]'.
\end{verbatim}

\begin{verbatim}
 
Basic management setup configures only enough connectivity
for management of the system, extended setup will ask you
to configure each interface on the system
\end{verbatim}

\begin{verbatim}
Would you like to enter basic management setup? [yes/no]: y
Configuring global parameters:
\end{verbatim}

\begin{verbatim}
 
  Enter host name [Switch]: Ctrl+C
Configuration aborted, no changes made.
\end{verbatim}

\begin{center}\rule{0.5\linewidth}{0.5pt}\end{center}

\includegraphics{images/note.png} You can exit setup mode at any time by
pressing Ctrl+C.

\begin{center}\rule{0.5\linewidth}{0.5pt}\end{center}

I highly recommend going through setup mode once, then never again
because you should always use the CLI instead!

\subsection[Command-Line Interface
(CLI)]{\texorpdfstring{\protect\hypertarget{c06.xhtmlux5cux23c06-sec-5}{}{}Command-Line
Interface (CLI)}{Command-Line Interface (CLI)}}

I sometimes refer to the CLI as ``cash line interface'' because the
ability to create advanced configurations on Cisco routers and switches
using the CLI will earn you some decent cash!

\subsubsection[Entering the
CLI]{\texorpdfstring{\protect\hypertarget{c06.xhtmlux5cux23c06-sec-6}{}{}\protect\hypertarget{c06.xhtmlux5cux23Page_210}{}{}Entering
the CLI}{Entering the CLI}}

After the interface status messages appear and you press Enter, the
\texttt{Switch\textgreater{}} prompt will pop up. This is called
\emph{user exec mode}, or user mode for short, and although it's mostly
used to view statistics, it is also a stepping stone along the way to
logging in to \emph{privileged exec mode}, called privileged mode for
short.

You can view and change the configuration of a Cisco router only while
in privileged mode, and you enter it via the \texttt{enable} command
like this:

\begin{verbatim}
Switch>enable
Switch#
\end{verbatim}

The \texttt{Switch\#} prompt signals you're in privileged mode where you
can both view and change the switch configuration. You can go back from
privileged mode into user mode by using the \texttt{disable} command:

\begin{verbatim}
Switch#disable
Switch>
\end{verbatim}

You can type \texttt{logout} from either mode to exit the console:

\begin{verbatim}
Switch>logout
Switch con0 is now available
Press RETURN to get started.
\end{verbatim}

Next, I'll show how to perform some basic administrative configurations.

\subsubsection[Overview of Router
Modes]{\texorpdfstring{\protect\hypertarget{c06.xhtmlux5cux23c06-sec-7}{}{}Overview
of Router Modes}{Overview of Router Modes}}

To configure from a CLI, you can make global changes to the router by
typing \texttt{configure\ terminal} or just \texttt{config\ t}. This
will get you into global configuration mode where you can make changes
to the running-config. Commands run from global configuration mode are
predictably referred to as global commands, and they are typically set
only once and affect the entire router.

Type \texttt{config} from the privileged-mode prompt and then press
Enter to opt for the default of \texttt{terminal} like this:

\begin{verbatim}
Switch#config
Configuring from terminal, memory, or network [terminal]? [press enter]
Enter configuration commands, one per line.  End with CNTL/Z.
Switch(config)#
\end{verbatim}

At this point, you make changes that affect the router as a whole
(globally), hence the term \emph{global configuration mode}. For
instance, to change the running-config---the current configuration
running in dynamic RAM (DRAM)---use the \texttt{configure\ terminal}
command, as I just demonstrated.

\subsubsection[CLI
Prompts]{\texorpdfstring{\protect\hypertarget{c06.xhtmlux5cux23c06-sec-8}{}{}\protect\hypertarget{c06.xhtmlux5cux23Page_211}{}{}CLI
Prompts}{CLI Prompts}}

Let's explore the different prompts you'll encounter when configuring a
switch or router now, because knowing them well will really help you
orient yourself and recognize exactly where you are at any given time
while in configuration mode. I'm going to demonstrate some of the
prompts used on a Cisco switch and cover the various terms used along
the way. Make sure you're very familiar with them, and always check your
prompts before making any changes to a router's configuration!

We're not going to venture into every last obscure command prompt you
could potentially come across in the configuration mode world because
that would get us deep into territory that's beyond the scope of this
book. Instead, I'm going to focus on the prompts you absolutely must
know to pass the exam plus the very handy and seriously vital ones
you'll need and use the most in real-life networking---the cream of the
crop.

\begin{center}\rule{0.5\linewidth}{0.5pt}\end{center}

\includegraphics{images/note.png} Don't freak! It's not important that
you understand exactly what each of these command prompts accomplishes
just yet because I'm going to completely fill you in on all of them
really soon. For now, relax and focus on just becoming familiar with the
different prompts available and all will be well!

\begin{center}\rule{0.5\linewidth}{0.5pt}\end{center}

\paragraph{Interfaces}

To make changes to an interface, you use the \texttt{interface} command
from global configuration mode:

\begin{verbatim}
Switch(config)#interface ?
  Async              Async interface
  BVI                Bridge-Group Virtual Interface
  CTunnel            CTunnel interface
  Dialer             Dialer interface
  FastEthernet       FastEthernet IEEE 802.3
  Filter             Filter interface
  Filtergroup        Filter Group interface
  GigabitEthernet    GigabitEthernet IEEE 802.3z
  Group-Async        Async Group interface
  Lex                Lex interface
  Loopback           Loopback interface
  Null               Null interface
  Port-channel       Ethernet Channel of interfaces
  Portgroup          Portgroup interface
  Pos-channel        POS Channel of interfaces
  Tunnel             Tunnel interface
  Vif                PGM Multicast Host interface
  Virtual-Template   Virtual Template interface
  Virtual-TokenRing  Virtual TokenRing
  Vlan               Catalyst Vlans
  fcpa               Fiber Channel
  range              interface range command
Switch(config)#interface fastEthernet 0/1
Switch(config-if)#)
\end{verbatim}

Did you notice that the prompt changed to \texttt{Switch(config-if)\#}?
This tells you that you're in \emph{interface configuration mode}. And
wouldn't it be nice if the prompt also gave you an indication of what
interface you were configuring? Well, at least for now we'll have to
live without the prompt information, because it doesn't. But it should
already be clear to you that you really need to pay attention when
configuring an IOS device!

\paragraph{Line Commands}

To configure user-mode passwords, use the \texttt{line} command. The
prompt then becomes \texttt{Switch(config-line)\#}:

\begin{verbatim}
Switch(config)#line ?
  <0-16>   First Line number
  console  Primary terminal line
  vty      Virtual terminal
Switch(config)#line console 0
Switch(config-line)#
\end{verbatim}

The \texttt{line\ console\ 0} command is a global command, and sometimes
you'll also hear people refer to global commands as major commands. In
this example, any command typed from the \texttt{(config-line)} prompt
is known as a subcommand.

\paragraph{Access List Configurations}

To configure a standard named access list, you'll need to get to the
prompt \texttt{Switch(config-std-nacl)\#}:

\begin{verbatim}
Switch#config t
Switch(config)#ip access-list standard Todd
Switch(config-std-nacl)#
\end{verbatim}

What you see here is a typical basic standard ACL prompt. There are
various ways to configure access lists, and the prompts are only
slightly different from this particular example.

\paragraph{Routing Protocol Configurations}

I need to point out that we don't use routing or router protocols on
2960 switches, but we can and will use them on my 3560 switches. Here is
an example of configuring routing on a layer 3 switch:

\begin{verbatim}
Switch(config)#router rip
IP routing not enabled
Switch(config)#ip routing
Switch(config)#router rip
Switch(config-router)#
\end{verbatim}

Did you notice that the prompt changed to
\texttt{Switch(config-router)\#}? To make sure you achieve the
objectives specific to the Cisco exam and this book, I'll configure
static routing, RIPv2, and RIPng. And don't worry---I'll explain all of
these in detail soon, in Chapter 9, ``IP Routing,'' and Chapter 14,
``Internet Protocol Version 6 (IPv6)''!

\paragraph{Defining Router Terms}

\protect\hyperlink{c06.xhtmlux5cux23table6-1}{Table 6.1} defines some of
the terms I've used so far.

{\protect\hyperlink{c06.xhtmlux5cux23tableanchor6-1}{\textbf{TABLE 6.1}}
Router terms}

\begin{longtable}[]{@{}ll@{}}
\toprule
Mode & Definition\tabularnewline
\midrule
\endhead
User exec mode & Limited to basic monitoring commands\tabularnewline
Privileged exec mode & Provides access to all other router
commands\tabularnewline
Global configuration mode & Commands that affect the entire
system\tabularnewline
Specific configuration modes & Commands that affect interfaces/processes
only\tabularnewline
Setup mode & Interactive configuration dialog\tabularnewline
\bottomrule
\end{longtable}

\subsubsection[Editing and Help
Features]{\texorpdfstring{\protect\hypertarget{c06.xhtmlux5cux23c06-sec-9}{}{}Editing
and Help Features}{Editing and Help Features}}

The Cisco advanced editing features can also help you configure your
router. If you type in a question mark (\texttt{?}) at any prompt,
you'll be given a list of all the commands available from that prompt:

\begin{verbatim}
Switch#?
Exec commands:
  access-enable    Create a temporary Access-List entry
  access-template  Create a temporary Access-List entry
  archive          manage archive files
  cd               Change current directory
  clear            Reset functions
  clock            Manage the system clock
  cns              CNS agents
  configure        Enter configuration mode
  connect          Open a terminal connection
  copy             Copy from one file to another
  debug            Debugging functions (see also 'undebug')
  delete           Delete a file
  diagnostic       Diagnostic commands
  dir              List files on a filesystem
  disable          Turn off privileged commands
  disconnect       Disconnect an existing network connection
  dot1x            IEEE 802.1X Exec Commands
  enable           Turn on privileged commands
  eou              EAPoUDP
  erase            Erase a filesystem
  exit             Exit from the EXEC
 ––More–– ?
Press RETURN for another line, SPACE for another page, anything else to quit
\end{verbatim}

And if this is not enough information for you, you can press the
spacebar to get another whole page of information, or you can press
Enter to go one command at a time. You can also press Q, or any other
key for that matter, to quit and return to the prompt. Notice that I
typed a question mark (?) at the \texttt{more} prompt and it told me
what my options were from that prompt.

Here's a shortcut: To find commands that start with a certain letter,
use the letter and the question mark with no space between them, like
this:

\begin{verbatim}
Switch#c?
cd       clear  clock  cns  configure
connect  copy
Switch#c
\end{verbatim}

Okay, see that? By typing \texttt{c?}, I got a response listing all the
commands that start with \emph{c}. Also notice that the
\texttt{Switch\#}\texttt{c} prompt reappears after the list of commands
is displayed. This can be really helpful when you happen to be working
with long commands but you're short on patience and still need the next
possible one. It would get old fast if you actually had to retype the
entire command every time you used a question mark!

So with that, let's find the next command in a string by typing the
first command and then a question mark:

\begin{verbatim}
Switch#clock ?
  set  Set the time and date
\end{verbatim}

\begin{verbatim}
Switch#clock set ?
  hh:mm:ss  Current Time
\end{verbatim}

\begin{verbatim}
Switch#clock set 2:34 ?
% Unrecognized command
Switch#clock set 2:34:01 ?
  <1-31>  Day of the month
  MONTH   Month of the year
\end{verbatim}

\begin{verbatim}
Switch#clock set 2:34:01 21 july ?
  <1993-2035>  Year
\end{verbatim}

\begin{verbatim}
Switch#clock set 2:34:01 21 august 2013
Switch#
00:19:45: %SYS-6-CLOCKUPDATE: System clock has been updated from 00:19:45
UTC Mon Mar 1 1993 to 02:34:01 UTC Wed Aug 21 2013, configured from console
by console.
\end{verbatim}

I entered the \texttt{clock\ ?} command and got a list of the next
possible parameters plus what they do. Make note of the fact that you
can just keep typing a command, a space, and then a question mark until
\texttt{\textless{}cr\textgreater{}} (carriage return) is your only
option left.

And if you're typing commands and receive

\begin{verbatim}
Switch#clock set 11:15:11
% Incomplete command.
\end{verbatim}

no worries---that's only telling you that the command string simply
isn't complete quite yet. All you need to do is to press the up arrow
key to redisplay the last command entered and then continue with the
command by using your question mark.

But if you get the error

\begin{verbatim}
Switch(config)#access-list 100 permit host 1.1.1.1 host 2.2.2.2
                                      ^
% Invalid input detected at '^' marker.
\end{verbatim}

all is not well because it means you actually have entered a command
incorrectly. See that little caret---the \texttt{\^{}}? It's a very
helpful tool that marks the exact point where you blew it and made a
mess.

Here's another example of when you'll see that caret:

\begin{verbatim}
Switch#sh fastethernet 0/0
            ^
% Invalid input detected at '^' marker.
\end{verbatim}

This command looks right, but be careful! The problem is that the full
command is \texttt{show\ interface\ fastethernet\ 0/0}.

Now if you receive the error

\begin{verbatim}
Switch#sh cl
% Ambiguous command:  "sh cl"
\end{verbatim}

\protect\hypertarget{c06.xhtmlux5cux23Page_216}{}{}you're being told
that there are multiple commands that begin with the string you entered
and it's not unique. Use the question mark to find the exact command you
need:

\begin{verbatim}
Switch#sh cl?
class-map  clock  cluster
\end{verbatim}

Case in point: There are three commands that start with
\texttt{show\ cl}.

\protect\hyperlink{c06.xhtmlux5cux23table6-2}{Table 6.2} lists the
enhanced editing commands available on a Cisco router.

{\protect\hyperlink{c06.xhtmlux5cux23tableanchor6-2}{\textbf{TABLE 6.2}}
Enhanced editing commands}

\begin{longtable}[]{@{}ll@{}}
\toprule
Command & Meaning\tabularnewline
\midrule
\endhead
Ctrl+A & Moves your cursor to the beginning of the line\tabularnewline
Ctrl+E & Moves your cursor to the end of the line\tabularnewline
Esc+B & Moves back one word\tabularnewline
Ctrl+B & Moves back one character\tabularnewline
Ctrl+F & Moves forward one character\tabularnewline
Esc+F & Moves forward one word\tabularnewline
Ctrl+D & Deletes a single character\tabularnewline
Backspace & Deletes a single character\tabularnewline
Ctrl+R & Redisplays a line\tabularnewline
Ctrl+U & Erases a line\tabularnewline
Ctrl+W & Erases a word\tabularnewline
Ctrl+Z & Ends configuration mode and returns to EXEC\tabularnewline
Tab & Finishes typing a command for you\tabularnewline
\bottomrule
\end{longtable}

Another really cool editing feature you need to know about is the
automatic scrolling of long lines. In the following example, the command
I typed reached the right margin and automatically moved 11 spaces to
the left. How do I know this? Because the dollar sign {[}\texttt{\$}{]}
is telling me that the line has been scrolled to the left:

\begin{verbatim}
Switch#config t
Switch(config)#$ 100 permit ip host 192.168.10.1 192.168.10.0 0.0.0.255
\end{verbatim}

\protect\hypertarget{c06.xhtmlux5cux23Page_217}{}{}You can review the
router-command history with the commands shown in
\protect\hyperlink{c06.xhtmlux5cux23table6-3}{Table 6.3}.

{\protect\hyperlink{c06.xhtmlux5cux23tableanchor6-3}{\textbf{TABLE 6.3}}
IOS-command history}

\begin{longtable}[]{@{}ll@{}}
\toprule
Command & Meaning\tabularnewline
\midrule
\endhead
Ctrl+P or up arrow & Shows last command entered\tabularnewline
Ctrl+N or down arrow & Shows previous commands entered\tabularnewline
\texttt{show\ history} & Shows last 20 commands entered by
default\tabularnewline
\texttt{show\ terminal} & Shows terminal configurations and history
buffer size\tabularnewline
\texttt{terminal\ history\ size} & Changes buffer size (max
256)\tabularnewline
\bottomrule
\end{longtable}

The following example demonstrates the \texttt{show\ history} command as
well as how to change the history's size. It also shows how to verify
the history with the \texttt{show\ terminal} command. First, use the
\texttt{show\ history} command, which will allow you to see the last 20
commands that were entered on the router (even though my particular
router reveals only 10 commands because that's all I've entered since
rebooting it). Check it out:

\begin{verbatim}
Switch#sh history
  sh fastethernet 0/0
  sh ru
  sh cl
  config t
  sh history
  sh flash
  sh running-config
  sh startup-config
  sh ver
  sh history
\end{verbatim}

Okay---now, we'll use the \texttt{show\ terminal} command to verify the
terminal history size:

\begin{verbatim}
Switch#sh terminal
Line 0, Location: "", Type: ""
Length: 24 lines, Width: 80 columns
Baud rate (TX/RX) is 9600/9600, no parity, 2 stopbits, 8 databits
Status: PSI Enabled, Ready, Active, Ctrl-c Enabled, Automore On
  0x40000
Capabilities: none
Modem state: Ready
[output cut]
Modem type is unknown.
Session limit is not set.
Time since activation: 00:17:22
Editing is enabled.
History is enabled, history size is 10.
DNS resolution in show commands is enabled
Full user help is disabled
Allowed input transports are none.
Allowed output transports are telnet.
Preferred transport is telnet.
No output characters are padded
No special data dispatching characters
\end{verbatim}

\begin{center}\rule{0.5\linewidth}{0.5pt}\end{center}

\includegraphics{images/globe1.png}\\
\textbf{When Should I Use the Cisco Editing Features?}

You'll find yourself using a couple of editing features quite often and
some not so much, if at all. Understand that Cisco didn't make these up;
these are just old Unix commands! Even so, Ctrl+A is still a really
helpful way to negate a command.

For example, if you were to put in a long command and then decide you
didn't want to use that command in your configuration after all, or if
it didn't work, then you could just press your up arrow key to show the
last command entered, press Ctrl+A, type \texttt{no} and then a space,
press Enter---and poof! The command is negated. This doesn't work on
every command, but it works on a lot of them and saves some serious
time!

\begin{center}\rule{0.5\linewidth}{0.5pt}\end{center}

\subsection[Administrative
Configurations]{\texorpdfstring{\protect\hypertarget{c06.xhtmlux5cux23c06-sec-10}{}{}Administrative
Configurations}{Administrative Configurations}}

Even though the following sections aren't critical to making a router or
switch \emph{work} on a network, they're still really important. I'm
going to guide you through configuring specific commands that are
particularly helpful when administering your network.

You can configure the following administrative functions on a router and
switch:

\begin{enumerate}
\tightlist
\item
  Hostnames
\item
  Banners
\item
  Passwords
\item
  Interface descriptions
\end{enumerate}

Remember, none of these will make your routers or switches work better
or faster, but trust me, your life will be a whole lot better if you
just take the time to set these
\protect\hypertarget{c06.xhtmlux5cux23Page_219}{}{}configurations on
each of your network devices. This is because doing so makes
troubleshooting and maintaining your network a great deal
easier---seriously! In this next section, I'll be demonstrating commands
on a Cisco switch, but understand that these commands are used in the
exact same way on a Cisco router.

\subsubsection[Hostnames]{\texorpdfstring{\protect\hypertarget{c06.xhtmlux5cux23c06-sec-11}{}{}Hostnames}{Hostnames}}

We use the \texttt{hostname} command to set the identity of the router
and switch. This is only locally significant, meaning it doesn't affect
how the router or switch performs name lookups or how the device
actually works on the internetwork. But the hostname is still important
in routes because it's often used for authentication in many wide area
networks (WANs). Here's an example:

\begin{verbatim}
Switch#config t
Switch(config)#hostname Todd
Todd(config)#hostname Chicago
Chicago(config)#hostname Todd
Todd(config)#
\end{verbatim}

I know it's pretty tempting to configure the hostname after your own
name, but it's usually a much better idea to name the device something
that relates to its physical location. A name that maps to where the
device lives will make finding it a whole lot easier, which among other
things, confirms that you're actually configuring the correct device.
Even though it seems like I'm completely ditching my own advice by
naming mine \emph{Todd}, I'm not, because this particular device really
does live in ``Todd's'' office. Its name perfectly maps to where it is,
so it won't be confused with those in the other networks I work with!

\subsubsection[Banners]{\texorpdfstring{\protect\hypertarget{c06.xhtmlux5cux23c06-sec-12}{}{}Banners}{Banners}}

A very good reason for having a \emph{banner} is to give any and all who
dare attempt to telnet or sneak into your internetwork a little security
notice. And they're very cool because you can create and customize them
so that they'll greet anyone who shows up on the router with exactly the
information you want them to have!

Here are the three types of banners you need to be sure you're familiar
with:

\begin{enumerate}
\tightlist
\item
  Exec process creation banner
\item
  Login banner
\item
  Message of the day banner
\end{enumerate}

And you can see them all illustrated in the following code:

\begin{verbatim}
Todd(config)#banner ?
  LINE            c banner-text c, where 'c' is a delimiting character
  exec            Set EXEC process creation banner
  incoming        Set incoming terminal line banner
  login           Set login banner
  motd            Set Message of the Day banner
  prompt-timeout  Set Message for login authentication timeout
  slip-ppp        Set Message for SLIP/PPP
\end{verbatim}

Message of the day (MOTD) banners are the most widely used banners
because they give a message to anyone connecting to the router via
Telnet or an auxiliary port or even through a console port as seen here:

\begin{verbatim}
Todd(config)#banner motd ?
LINE c banner-text c, where 'c' is a delimiting character
Todd(config)#banner motd #
Enter TEXT message. End with the character '#'.
$ Acme.com network, then you must disconnect immediately.
#
\end{verbatim}

\begin{verbatim}
Todd(config)#^Z (Press the control key + z keys to return to privileged mode)
Todd#exit
con0 is now available
Press RETURN to get started.
If you are not authorized to be in Acme.com network, then you
must disconnect immediately.
Todd#
\end{verbatim}

This MOTD banner essentially tells anyone connecting to the device to
get lost if they're not on the guest list. The part to focus upon here
is the delimiting character, which is what informs the router the
message is done. Clearly, you can use any character you want for it
except for the delimiting character in the message itself. Once the
message is complete, press Enter, then the delimiting character, and
then press Enter again. Everything will still work if you don't follow
this routine unless you have more than one banner. If that's the case,
make sure you do follow it or your banners will all be combined into one
message and put on a single line!

You can set a banner on one line like this:

\begin{verbatim}
Todd(config)#banner motd x Unauthorized access prohibited! x
\end{verbatim}

Let's take a minute to go into more detail about the other two types of
banners I mentioned:

\textbf{Exec banner} You can configure a line-activation (exec) banner
to be displayed when EXEC processes such as a line activation or an
incoming connection to a VTY line have been created. Simply initiating a
user exec session through a console port will activate the exec banner.

\textbf{Login banner} You can configure a login banner for display on
all connected terminals. It will show up after the MOTD banner but
before the login prompts. This login banner can't be disabled on a
per-line basis, so to globally disable it you've got to delete it with
the \texttt{no\ banner\ login} command.

\protect\hypertarget{c06.xhtmlux5cux23Page_221}{}{}Here's what a login
banner output looks like:

\begin{verbatim}
!
banner login ^C
———————————————————————————————————————————————————————————————————————————
Cisco Router and Security Device Manager (SDM) is installed on this device.
This feature requires the one-time use of the username "cisco"
with the password "cisco". The default username and password
have a privilege level of 15.
Please change these publicly known initial credentials using
SDM or the IOS CLI.
Here are the Cisco IOS commands.
username <myuser>  privilege 15 secret 0 <mypassword>
no username cisco
Replace <myuser> and <mypassword> with the username and
password you want to use.
For more information about SDM please follow the instructions
in the QUICK START GUIDE for your router or go to www.cisco.com/go/sdm
————————————————————————————————————————————————————————————————————————————–
^C
!
\end{verbatim}

The previous login banner should look pretty familiar to anyone who's
ever logged into an ISR router because it's the banner Cisco has in the
default configuration for its ISR routers.

\begin{center}\rule{0.5\linewidth}{0.5pt}\end{center}

\includegraphics{images/note.png} Remember that the login banner is
displayed before the login prompts and after the MOTD banner.

\begin{center}\rule{0.5\linewidth}{0.5pt}\end{center}

\subsubsection[Setting
Passwords]{\texorpdfstring{\protect\hypertarget{c06.xhtmlux5cux23c06-sec-13}{}{}Setting
Passwords}{Setting Passwords}}

There are five passwords you'll need to secure your Cisco routers:
console, auxiliary, telnet/SSH (VTY), enable password, and enable
secret. The enable secret and enable password are the ones used to set
the password for securing privileged mode. Once the \texttt{enable}
commands are set, users will be prompted for a password. The other three
are used to configure a password when user mode is accessed through the
console port, through the auxiliary port, or via Telnet.

Let's take a look at each of these now.

\paragraph{Enable Passwords}

You set the enable passwords from global configuration mode like this:

\begin{verbatim}
Todd(config)#enable ?
 last-resort Define enable action if no TACACS servers
             respond
 password    Assign the privileged level password
 secret      Assign the privileged level secret
 use-tacacs  Use TACACS to check enable passwords
\end{verbatim}

The following list describes the enable password parameters:

\texttt{last-resort} This allows you to still enter the device if you
set up authentication through a TACACS server and it's not available. It
won't be used if the TACACS server is working.

\texttt{password} This sets the enable password on older, pre-10.3
systems and isn't ever used if an enable secret is set.

\texttt{secret} The newer, encrypted password that overrides the enable
password if it has been set.

\texttt{use-tacacs} This tells the router or switch to authenticate
through a TACACS server. It comes in really handy when you have lots of
routers because changing the password on a multitude of them can be
insanely tedious. It's much easier to simply go through the TACACS
server and change the password only once!

Here's an example that shows how to set the enable passwords:

\begin{verbatim}
Todd(config)#enable secret todd
Todd(config)#enable password todd
The enable password you have chosen is the same as your
  enable secret. This is not recommended. Re-enter the
  enable password.
\end{verbatim}

If you try to set the enable secret and enable passwords the same, the
device will give you a polite warning to change the second password.
Make a note to yourself that if there aren't any old legacy routers
involved, you don't even bother to use the enable password!

User-mode passwords are assigned via the \texttt{line} command like
this:

\begin{verbatim}
Todd(config)#line ?
  <0-16>   First Line number
  console  Primary terminal line
  vty      Virtual terminal
\end{verbatim}

And these two lines are especially important for the exam objectives:

\texttt{console} Sets a console user-mode password.

\texttt{vty} Sets a Telnet password on the device. If this password
isn't set, then by default, Telnet can't be used.

To configure user-mode passwords, choose the line you want and configure
it using the \texttt{login} command to make the switch prompt for
authentication. Let's focus in on the configuration of individual lines
now.

\paragraph[Console
Password]{\texorpdfstring{\protect\hypertarget{c06.xhtmlux5cux23Page_223}{}{}Console
Password}{Console Password}}

We set the console password with the \texttt{line\ console\ 0} command,
but look at what happened when I tried to type \texttt{line\ console\ ?}
from the \texttt{(config-line)\#} prompt---I received an error! Here's
the example:

\begin{verbatim}
Todd(config-line)#line console ?
% Unrecognized command
Todd(config-line)#exit
Todd(config)#line console ?
  <0-0>  First Line number
Todd(config)#line console 0
Todd(config-line)#password console
Todd(config-line)#login
\end{verbatim}

You can still type \texttt{line\ console\ 0} and that will be accepted,
but the help screens just don't work from that prompt. Type
\texttt{exit} to go back one level, and you'll find that your help
screens now work. This is a ``feature.'' Really.

Because there's only one console port, I can only choose line console 0.
You can set all your line passwords to the same password, but doing this
isn't exactly a brilliant security move!

And it's also important to remember to apply the \texttt{login} command
or the console port won't prompt for authentication. The way Cisco has
this process set up means you can't set the \texttt{login} command
before a password is set on a line because if you set it but don't then
set a password, that line won't be usable. You'll actually get prompted
for a password that doesn't exist, so Cisco's method isn't just a
hassle; it makes sense and is a feature after all!

\begin{center}\rule{0.5\linewidth}{0.5pt}\end{center}

\includegraphics{images/note.png} Definitely remember that although
Cisco has this ``password feature'' on its routers starting with IOS
12.2 and above, it's not included in older IOSs.

\begin{center}\rule{0.5\linewidth}{0.5pt}\end{center}

Okay, there are a few other important commands you need to know
regarding the console port.

For one, the \texttt{exec-timeout\ 0\ 0} command sets the time-out for
the console EXEC session to zero, ensuring that it never times out. The
default time-out is 10 minutes.

\begin{center}\rule{0.5\linewidth}{0.5pt}\end{center}

\includegraphics{images/tip.png} If you're feeling mischievous, try this
on people at work: Set the \texttt{exec-}\texttt{timeout} command to 0
1. This will make the console time out in 1 second, and to fix it, you
have to continually press the down arrow key while changing the time-out
time with your free hand!

\begin{center}\rule{0.5\linewidth}{0.5pt}\end{center}

\texttt{Logging\ synchronous} is such a cool command that it should be a
default, but it's not. It's great because it's the antidote for those
annoying console messages that disrupt the input you're trying to type.
The messages will still pop up, but at least you get returned to your
device prompt without your input being interrupted! This makes your
input messages oh-so-much easier to read!

\protect\hypertarget{c06.xhtmlux5cux23Page_224}{}{}Here's an example of
how to configure both commands:

\begin{verbatim}
Todd(config-line)#line con 0
Todd(config-line)#exec-timeout ?
  <0-35791>  Timeout in minutes
Todd(config-line)#exec-timeout 0 ?
  <0-2147483>  Timeout in seconds
  <cr>
Todd(config-line)#exec-timeout 0 0
Todd(config-line)#logging synchronous
\end{verbatim}

\begin{center}\rule{0.5\linewidth}{0.5pt}\end{center}

\includegraphics{images/note.png} You can set the console to go from
never timing out (0 0) to timing out in 35,791 minutes and 2,147,483
seconds. Remember that the default is 10 minutes.

\begin{center}\rule{0.5\linewidth}{0.5pt}\end{center}

\paragraph{Telnet Password}

To set the user-mode password for Telnet access into the router or
switch, use the \texttt{line\ vty} command. IOS switches typically have
16 lines, but routers running the Enterprise edition have considerably
more. The best way to find out how many lines you have is to use that
handy question mark like this:

\begin{verbatim}
Todd(config-line)#line vty 0 ?
% Unrecognized command
Todd(config-line)#exit
Todd(config)#line vty 0 ?
  <1-15>  Last Line number
  <cr>
Todd(config)#line vty 0 15
Todd(config-line)#password telnet
Todd(config-line)#login
\end{verbatim}

This output clearly shows that you cannot get help from your
\texttt{(config-line)\#} prompt. You must go back to global config mode
in order to use the question mark (\texttt{?}).

So what will happen if you try to telnet into a device that doesn't have
a VTY password set? You'll receive an error saying the connection has
been refused because the password isn't set. So, if you telnet into a
switch and receive a message like this one that I got from Switch B

\begin{verbatim}
Todd#telnet SwitchB
Trying SwitchB (10.0.0.1)…Open
\end{verbatim}

\begin{verbatim}
Password required, but none set
[Connection to SwitchB closed by foreign host]
Todd#
\end{verbatim}

\protect\hypertarget{c06.xhtmlux5cux23Page_225}{}{}it means the switch
doesn't have the VTY password set. But you can still get around this and
tell the switch to allow Telnet connections without a password by using
the \texttt{no\ login} command:

\begin{verbatim}
SwitchB(config-line)#line vty 0 15
SwitchB(config-line)#no login
\end{verbatim}

\begin{center}\rule{0.5\linewidth}{0.5pt}\end{center}

\includegraphics{images/warning.png} I definitely do not recommend using
the \texttt{no\ login} command to allow Telnet connections without a
password, unless you're in a testing or classroom environment. In a
production network, always set your VTY password!

\begin{center}\rule{0.5\linewidth}{0.5pt}\end{center}

After your IOS devices are configured with an IP address, you can use
the Telnet program to configure and check your routers instead of having
to use a console cable. You can use the Telnet program by typing
\texttt{telnet} from any command prompt (DOS or Cisco). I'll cover all
things Telnet more thoroughly in Chapter 7, ``Managing a Cisco
Internetwork.''

\paragraph{Auxiliary Password}

To configure the auxiliary password on a router, go into global
configuration mode and type \texttt{line\ aux\ ?}. And by the way, you
won't find these ports on a switch. This output shows that you only get
a choice of 0--0, which is because there's only one port:

\begin{verbatim}
Todd#config t
Todd(config)#line aux ?
  <0-0>  First Line number
Todd(config)#line aux 0
Todd(config-line)#login
% Login disabled on line 1, until 'password' is set
Todd(config-line)#password aux
Todd(config-line)#login
\end{verbatim}

\paragraph{Setting Up Secure Shell (SSH)}

I strongly recommend using Secure Shell (SSH) instead of Telnet because
it creates a more secure session. The Telnet application uses an
unencrypted data stream, but SSH uses encryption keys to send data so
your username and password aren't sent in the clear, vulnerable to
anyone lurking around!

Here are the steps for setting up SSH:

\begin{enumerate}
\item
  Set your hostname:

\begin{verbatim}
Router(config)#hostname Todd
\end{verbatim}
\item
  \protect\hypertarget{c06.xhtmlux5cux23Page_226}{}{}Set the domain
  name---both the hostname and domain name are required for the
  encryption keys to be generated:

\begin{verbatim}
Todd(config)#ip domain-name Lammle.com
\end{verbatim}
\item
  Set the username to allow SSH client access:

\begin{verbatim}
Todd(config)#username Todd password Lammle
\end{verbatim}
\item
  Generate the encryption keys for securing the session:

\begin{verbatim}
Todd(config)#crypto key generate rsa
The name for the keys will be: Todd.Lammle.com
Choose the size of the key modulus in the range of 360 to
4096 for your General Purpose Keys. Choosing a key modulus
Greater than 512 may take a few minutes.
\end{verbatim}

\begin{verbatim}
How many bits in the modulus [512]: 1024
% Generating 1024 bit RSA keys, keys will be non-exportable...
[OK] (elapsed time was 6 seconds)
 
Todd(config)#
1d14h: %SSH-5-ENABLED: SSH 1.99 has been enabled*June 24
19:25:30.035: %SSH-5-ENABLED: SSH 1.99 has been enabled
\end{verbatim}
\item
  Enable SSH version 2 on the device---not mandatory, but strongly
  suggested:

\begin{verbatim}
Todd(config)#ip ssh version 2
\end{verbatim}
\item
  Connect to the VTY lines of the switch or router:

\begin{verbatim}
Todd(config)#line vty 0 15
\end{verbatim}
\item
  Tell the lines to use the local database for password:

\begin{verbatim}
Todd(config-line)#login local
\end{verbatim}
\item
  Configure your access protocols:

\begin{verbatim}
Todd(config-line)#transport input ?
  all     All protocols
  none    No protocols
  ssh     TCP/IP SSH protocol
  telnet  TCP/IP Telnet protocol
\end{verbatim}

  Beware of this next line, and make sure you never use it in production
  because it's a horrendous security risk:

\begin{verbatim}
Todd(config-line)#transport input all
\end{verbatim}

  \protect\hypertarget{c06.xhtmlux5cux23Page_227}{}{}I recommend using
  the next line to secure your VTY lines with SSH:

\begin{verbatim}
Todd(config-line)#transport input ssh ?
  telnet  TCP/IP Telnet protocol
  <cr>
\end{verbatim}
\end{enumerate}

I actually do use Telnet once in a while when a situation arises that
specifically calls for it. It just doesn't happen very often. But if you
want to go with Telnet, here's how you do that:

\begin{verbatim}
Todd(config-line)#transport input ssh telnet
\end{verbatim}

Know that if you don't use the keyword \texttt{telnet} at the end of the
command string, then only SSH will work on the device. You can go with
either, just so long as you understand that SSH is way more secure than
Telnet.

\subsubsection[Encrypting Your
Passwords]{\texorpdfstring{\protect\hypertarget{c06.xhtmlux5cux23c06-sec-14}{}{}Encrypting
Your Passwords}{Encrypting Your Passwords}}

Because only the enable secret password is encrypted by default, you'll
need to manually configure the user-mode and enable passwords for
encryption.

Notice that you can see all the passwords except the enable secret when
performing a \texttt{show\ running-config} on a switch:

\begin{verbatim}
Todd#sh running-config
Building configuration...
\end{verbatim}

\begin{verbatim}
Current configuration : 1020 bytes
!
! Last configuration change at 00:03:11 UTC Mon Mar 1 1993
!
version 15.0
no service pad
service timestamps debug datetime msec
service timestamps log datetime msec
no service password-encryption
!
hostname Todd
!
enable secret 4 ykw.3/tgsOuy9.6qmgG/EeYOYgBvfX4v.S8UNA9Rddg
enable password todd
!
[output cut]
!
line con 0
 password console
 login
line vty 0 4
 password telnet
 login
line vty 5 15
 password telnet
 login
!
end
\end{verbatim}

To manually encrypt your passwords, use the
\texttt{service\ password-encryption} command. Here's how:

\begin{verbatim}
Todd#config t
Todd(config)#service password-encryption
Todd(config)#exit
Todd#show run
Building configuration...
!
!
enable secret 4 ykw.3/tgsOuy9.6qmgG/EeYOYgBvfX4v.S8UNA9Rddg
enable password 7 1506040800
!
[output cut]
!
!
line con 0
 password 7 050809013243420C
 login
line vty 0 4
 password 7 06120A2D424B1D
 login
line vty 5 15
 password 7 06120A2D424B1D
 login
!
end
Todd#config t
Todd(config)#no service password-encryption
Todd(config)#^Z
Todd#
\end{verbatim}

\protect\hypertarget{c06.xhtmlux5cux23Page_229}{}{}Nicely done---the
passwords will now be encrypted. All you need to do is encrypt the
passwords, perform a \texttt{show\ run}, then turn off the command if
you want. This output clearly shows us that the enable password and the
line passwords are all encrypted.

Before we move on to find out how to set descriptions on your
interfaces, I want to stress some points about password encryption. As I
said, if you set your passwords and then turn on the
\texttt{service\ password-encryption} command, you have to perform a
\texttt{show\ running-config} before you turn off the encryption service
or your passwords won't be encrypted. You don't have to turn off the
encryption service at all---you'd only do that if your switch is running
low on processes. And if you turn on the service before you set your
passwords, then you don't even have to view them to have them encrypted.

\subsubsection[Descriptions]{\texorpdfstring{\protect\hypertarget{c06.xhtmlux5cux23c06-sec-15}{}{}Descriptions}{Descriptions}}

Setting descriptions on an interface is another administratively helpful
thing, and like the hostname, it's also only locally significant. One
case where the \texttt{description} command comes in really handy is
when you want to keep track of circuit numbers on a switch or a router's
serial WAN port.

Here's an example on my switch:

\begin{verbatim}
Todd#config t
Todd(config)#int fa0/1
Todd(config-if)#description Sales VLAN Trunk Link
Todd(config-if)#^Z
Todd#
\end{verbatim}

And on a router serial WAN:

\begin{verbatim}
Router#config t
Router(config)#int s0/0/0
Router(config-if)#description WAN to Miami
Router(config-if)#^Z
\end{verbatim}

You can view an interface's description with either the
\texttt{show\ running-config} command or the
\texttt{show\ interface}---even with the
\texttt{show\ interface\ description} command:

\begin{verbatim}
Todd#sh run
Building configuration...
\end{verbatim}

\begin{verbatim}
Current configuration : 855 bytes
!
interface FastEthernet0/1
 description Sales VLAN Trunk Link
!
 [output cut]
Todd#sh int f0/1
FastEthernet0/1 is up, line protocol is up (connected)
  Hardware is Fast Ethernet, address is ecc8.8202.8282 (bia ecc8.8202.8282)
  Description: Sales VLAN Trunk Link
  MTU 1500 bytes, BW 100000 Kbit/sec, DLY 100 usec,
 [output cut]
\end{verbatim}

\begin{verbatim}
Todd#sh int description
Interface                      Status         Protocol Description
Vl1                            up             up
Fa0/1                          up             up     Sales VLAN Trunk Link
Fa0/2                          up             up
\end{verbatim}

\begin{center}\rule{0.5\linewidth}{0.5pt}\end{center}

\includegraphics{images/globe1.png}\\
\textbf{description: A Helpful Command}

Bob, a senior network admin at Acme Corporation in San Francisco, has
over 50 WAN links to branches throughout the United States and Canada.
Whenever an interface goes down, Bob wastes lots of time trying to
figure out the circuit number and the phone number of the provider of
his ailing WAN link.

This kind of scenario shows just how helpful the interface
\texttt{description} command can be. It would save Bob a lot of work
because he could use it on his most important switch LAN links to find
out exactly where every interface is connected. Bob's life would also be
made a lot easier by adding circuit numbers to each and every WAN
interface on his routers, along with the phone number of the responsible
provider.

So if Bob had just taken time in advance to preventively add this
information to his interfaces, he would have saved himself an ocean of
stress and a ton of precious time when his WAN links inevitably go down!

\begin{center}\rule{0.5\linewidth}{0.5pt}\end{center}

\paragraph{\texorpdfstring{Doing the \emph{\textbf{do}}
Command}{Doing the do Command}}

In every previous example so far, we've had to run all
\texttt{show\ commands} from privileged mode. But I've got great
news---beginning with IOS version 12.3, Cisco has finally added a
command to the IOS that allows you to view the configuration and
statistics from within configuration mode!

In fact, with any IOS, you'd get the following error if you tried to
view the configuration from global config:

\begin{verbatim}
Todd(config)#sh run
                  ^
% Invalid input detected at '^' marker.
\end{verbatim}

\protect\hypertarget{c06.xhtmlux5cux23Page_231}{}{}Compare that to the
output I get from entering that same command on my router that's running
the 15.0 IOS using the ``do'' syntax:

\begin{verbatim}
Todd(config)#do show run
Building configuration...
\end{verbatim}

\begin{verbatim}
Current configuration : 759 bytes
!
version 15.0
no service pad
service timestamps debug datetime msec
service timestamps log datetime msec
no service password-encryption
!
hostname Todd
!
boot-start-marker
boot-end-marker
!
[output cut]
\end{verbatim}

So now you can pretty much run any command from any configuration
prompt---nice, huh? Looking back through all those examples for
encrypting our passwords, you can see that the \texttt{do} command would
definitely have gotten the party started sooner, making this innovation
one to celebrate for sure!

\subsection[Router and Switch
Interfaces]{\texorpdfstring{\protect\hypertarget{c06.xhtmlux5cux23c06-sec-16}{}{}Router
and Switch Interfaces}{Router and Switch Interfaces}}

Interface configuration is arguably the most important router
configuration because without interfaces, a router is a pretty useless
object. Furthermore, interface configurations must be totally precise to
enable communication with other devices. Network layer addresses, media
type, bandwidth, and other administrator commands are all used to
configure an interface.

On a layer 2 switch, interface configurations typically involve a lot
less work than router interface configuration. Check out the output from
the powerful verification command \texttt{show\ ip\ interface\ brief},
which reveals all the interfaces on my 3560 switch:

\begin{verbatim}
Todd#sh ip interface brief
Interface              IP-Address      OK? Method Status          Protocol
Vlan1                  192.168.255.8   YES DHCP   up                    up
FastEthernet0/1        unassigned      YES unset  up                    up
FastEthernet0/2        unassigned      YES unset  up                    up
FastEthernet0/3        unassigned      YES unset  down                 down
FastEthernet0/4        unassigned      YES unset  down                 down
FastEthernet0/5        unassigned      YES unset  up                    up
FastEthernet0/6        unassigned      YES unset  up                    up
FastEthernet0/7        unassigned      YES unset  down                 down
FastEthernet0/8        unassigned      YES unset  down                 down
GigabitEthernet0/1     unassigned      YES unset  down                 down
\end{verbatim}

The previous output shows the default routed port found on all Cisco
switches (VLAN 1), plus nine switch FastEthernet interface ports, with
one port being a Gigabit Ethernet port used for uplinks to other
switches.

Different routers use different methods to choose the interfaces used on
them. For instance, the following command shows one of my 2800 ISR Cisco
routers with two FastEthernet interfaces along with two serial WAN
interfaces:

\begin{verbatim}
Router>sh ip int brief
Interface       IP-Address       OK? Method Status            Protocol
FastEthernet0/0  192.168.255.11  YES DHCP   up                    up
FastEthernet0/1  unassigned      YES unset  administratively down down
Serial0/0/0      unassigned      YES unset  administratively down down
Serial0/1/0      unassigned      YES unset  administratively down down
Router>
\end{verbatim}

Previously, we always used the \texttt{interface\ type} \texttt{number}
sequence to configure an interface, but the newer routers come with an
actual physical slot and include a port number on the module plugged
into it. So on a modular router, the configuration would be
\texttt{interface} \texttt{type\ slot/port}, as demonstrated here:

\begin{verbatim}
Todd#config t
Todd(config)#interface GigabitEthernet 0/1
Todd(config-if)#
\end{verbatim}

You can see that we are now at the Gigabit Ethernet slot 0, port 1
prompt, and from here we can make configuration changes to the
interface. Make note of the fact that you can't just type
\texttt{int\ gigabitethernet\ 0}. No shortcuts on the slot/port---you've
got to type the slot/port variables in the command:
\texttt{type\ slot/port} or, for example,
\texttt{int}\texttt{gigabitethernet}\texttt{\ 0/1} (or just
\texttt{int\ g0/1}).

Once in interface configuration mode, we can configure various options.
Keep in mind that speed and duplex are the two factors to be concerned
with for the LAN:

\begin{verbatim}
Todd#config t
Todd(config)#interface GigabitEthernet 0/1
Todd(config-if)#speed 1000
Todd(config-if)#duplex full
\end{verbatim}

\protect\hypertarget{c06.xhtmlux5cux23Page_233}{}{}So what's happened
here? Well basically, this has shut off the auto-detect mechanism on the
port, forcing it to only run gigabit speeds at full duplex. For the ISR
series router, it's basically the same, but you get even more options!
The LAN interfaces are the same, but the rest of the modules are
different---they use three numbers instead of two. The three numbers
used here can represent \texttt{slot/subslot/port}, but this depends on
the card used in the ISR router. For the objectives, you just need to
remember this: The first 0 is the router itself. You then choose the
slot and then the port. Here's an example of a serial interface on my
2811:

\begin{verbatim}
Todd(config)#interface serial ?
  <0-2>  Serial interface number
Todd(config)#interface serial 0/0/?
  <0-1>  Serial interface number
Todd(config)#interface serial 0/0/0
Todd(config-if)#
\end{verbatim}

This might look a little dicey to you, but I promise it's really not
that hard! It helps to remember that you should always view the output
of the \texttt{show\ ip\ interface\ brief} command or a
\texttt{show\ running-config} output first so you know the exact
interfaces you have to deal with. Here's one of my 2811's output that
has even more serial interfaces installed:

\begin{verbatim}
Todd(config-if)#do show run
Building configuration...
[output cut]
!
interface FastEthernet0/0
 no ip address
 shutdown
 duplex auto
 speed auto
!
interface FastEthernet0/1
 no ip address
 shutdown
 duplex auto
 speed auto
!
interface Serial0/0/0
 no ip address
 shutdown
 no fair-queue
!
interface Serial0/0/1
 no ip address
 shutdown
!
interface Serial0/1/0
 no ip address
 shutdown
!
interface Serial0/2/0
 no ip address
 shutdown
 clock rate 2000000
!
 [output cut]
\end{verbatim}

For the sake of brevity, I didn't include my complete running-config,
but I've displayed all you really need. You can see the two built-in
FastEthernet interfaces, the two serial interfaces in slot 0 (0/0/0 and
0/0/1), the serial interface in slot 1 (0/1/0), and the serial interface
in slot 2 (0/2/0). And once you see the interfaces like this, it makes
it a lot easier to understand how the modules are inserted into the
router.

Just understand that if you type \texttt{interface\ e0} on an old 2500
series router, \texttt{interface}\texttt{\ fastethernet\ 0/0} on a
modular router (such as the 2800 series router), or
\texttt{interface\ serial\ 0/1/0} on an ISR router, all you're actually
doing is choosing an interface to configure. Essentially, they're all
configured the same way after that.

Let's delve deeper into our router interface discussion by exploring how
to bring up the interface and set an IP address on it next.

\subsubsection[Bringing Up an
Interface]{\texorpdfstring{\protect\hypertarget{c06.xhtmlux5cux23c06-sec-17}{}{}Bringing
Up an Interface}{Bringing Up an Interface}}

You can disable an interface with the interface command
\texttt{shutdown} and enable it with the \texttt{no\ shutdown} command.
Just to remind you, all switch ports are enabled by default and all
router ports are disabled by default, so we're going to talk more about
router ports than switch ports in the next few sections.

If an interface is shut down, it'll display as administratively down
when you use the \texttt{show\ interfaces} command (\texttt{sh\ int} for
short):

\begin{verbatim}
Router#sh int f0/0
FastEthernet0/1 is administratively down, line protocol is down
[output cut]
\end{verbatim}

Another way to check an interface's status is via the
\texttt{show\ running-config} command. You can bring up the router
interface with the \texttt{no\ shutdown} command (\texttt{no\ shut} for
short):

\begin{verbatim}
Router(config)#int f0/0
Router(config-if)#no shutdown
*August 21 13:45:08.455: %LINK-3-UPDOWN: Interface FastEthernet0/0,
     changed state to up
Router(config-if)#do show int f0/0
FastEthernet0/0 is up, line protocol is up
[output cut]
\end{verbatim}

\paragraph{Configuring an IP Address on an Interface}

Even though you don't have to use IP on your routers, it's usually what
everyone uses. To configure IP addresses on an interface, use the
\texttt{ip\ address} command from interface configuration mode and
remember that you do not set an IP address on a layer 2 switch port!

\begin{verbatim}
Todd(config)#int f0/1
Todd(config-if)#ip address 172.16.10.2 255.255.255.0
\end{verbatim}

Also, don't forget to enable the interface with the
\texttt{no\ shutdown} command. Remember to look at the command
\texttt{show\ interface} \texttt{int} output to see if the interface is
administratively shut down or not. \texttt{Show\ ip\ int\ brief} and
\texttt{show\ running-config} will also give you this information.

\begin{center}\rule{0.5\linewidth}{0.5pt}\end{center}

\includegraphics{images/note.png} The \texttt{ip\ address}
\texttt{address\ mask} command starts the IP processing on the router
interface. Again, you do not configure an IP address on a layer 2 switch
interface!

\begin{center}\rule{0.5\linewidth}{0.5pt}\end{center}

Okay---now if you want to add a second subnet address to an interface,
you have to use the \texttt{secondary} parameter. If you type another IP
address and press Enter, it will replace the existing primary IP address
and mask. This is definitely one of the Cisco IOS's coolest features!

So let's try it. To add a secondary IP address, just use the
\texttt{secondary} parameter:

\begin{verbatim}
Todd(config-if)#ip address 172.16.20.2 255.255.255.0 ?
  secondary  Make this IP address a secondary address
  <cr>
Todd(config-if)#ip address 172.16.20.2 255.255.255.0 secondary
Todd(config-if)#do sh run
Building configuration...
[output cut]
\end{verbatim}

\begin{verbatim}
interface FastEthernet0/1
 ip address 172.16.20.2 255.255.255.0 secondary
 ip address 172.16.10.2 255.255.255.0
 duplex auto
 speed auto
!
\end{verbatim}

\protect\hypertarget{c06.xhtmlux5cux23Page_236}{}{}But I've got to stop
here to tell you that I really wouldn't recommend having multiple IP
addresses on an interface because it's really inefficient. I showed you
how anyway just in case you someday find yourself dealing with an MIS
manager who's in love with really bad network design and makes you
administer it! And who knows? Maybe someone will ask you about it
someday and you'll get to seem really smart because you know this.

\paragraph{Using the Pipe}

No, not that pipe. I mean the output modifier. Although, I've got to say
that some of the router configurations I've seen in my career make me
wonder! Anyway, this pipe ( \textbar{} ) allows us to wade through all
the configurations or other long outputs and get straight to our goods
fast. Here's an example:

\begin{verbatim}
Router#sh run | ?
  append    Append redirected output to URL (URLs supporting append
            operation only)
  begin     Begin with the line that matches
  exclude   Exclude lines that match
  include   Include lines that match
  redirect  Redirect output to URL
  section   Filter a section of output
  tee       Copy output to URL
\end{verbatim}

\begin{verbatim}
Router#sh run | begin interface
interface FastEthernet0/0
 description Sales VLAN
 ip address 10.10.10.1 255.255.255.248
 duplex auto
 speed auto
!
interface FastEthernet0/1
 ip address 172.16.20.2 255.255.255.0 secondary
 ip address 172.16.10.2 255.255.255.0
 duplex auto
 speed auto
!
interface Serial0/0/0
 description Wan to SF circuit number 6fdda 12345678
 no ip address
!
\end{verbatim}

So basically, the pipe symbol---the output modifier---is what you need
to help you get where you want to go light years faster than mucking
around in a router's entire
\protect\hypertarget{c06.xhtmlux5cux23Page_237}{}{}configuration. I use
it a lot when scrutinizing a large routing table to find out whether a
certain route is in the routing table. Here's an example:

\begin{verbatim}
Todd#sh ip route | include 192.168.3.32
R       192.168.3.32 [120/2] via 10.10.10.8, 00:00:25, FastEthernet0/0
Todd#
\end{verbatim}

First, you need to know that this routing table had over 100 entries, so
without my trusty pipe, I'd probably still be looking through that
output! It's a powerfully efficient tool that saves you major time and
effort by quickly finding a line in a configuration---or as the
preceding example shows, a single route within a huge routing table.

Give yourself a little time to play around with the pipe command to get
the hang of it and you'll be naturally high on your newfound ability to
quickly parse through router output!

\paragraph{Serial Interface Commands}

But wait! Before you just jump in and configure a serial interface, you
need some key information, like knowing the interface will usually be
attached to a CSU/DSU type of device that provides clocking for the line
to the router. Check out
\protect\hyperlink{c06.xhtmlux5cux23figure6-3}{Figure 6.3} for an
example.

\begin{figure}
\centering
\includegraphics{images/c06f003.jpg}
\caption{{\protect\hyperlink{c06.xhtmlux5cux23figureanchor6-3}{\textbf{FIGURE
6.3}} A typical WAN connection. Clocking is typically provided by a DCE
network to routers. In nonproduction environments, a DCE network is not
always present.}}
\end{figure}

Here you can see that the serial interface is used to connect to a DCE
network via a CSU/DSU that provides the clocking to the router
interface. But if you have a back-to-back configuration, such as one
that's used in a lab environment like the one in
\protect\hyperlink{c06.xhtmlux5cux23figure6-4}{Figure 6.4}, one
end---the data communication equipment (DCE) end of the cable---must
provide clocking!

\protect\hypertarget{c06.xhtmlux5cux23Page_238}{}{}

\begin{figure}
\centering
\includegraphics{images/c06f004.jpg}
\caption{{\protect\hyperlink{c06.xhtmlux5cux23figureanchor6-4}{\textbf{FIGURE
6.4}} Providing clocking on a nonproduction network}}
\end{figure}

By default, Cisco router serial interfaces are all data terminal
equipment (DTE) interfaces, which means that you must configure an
interface to provide clocking if you need it to act like a DCE device.
Again, you would not provide clocking on a production WAN serial
connection because you would have a CSU/DSU connected to your serial
interface, as shown in
\protect\hyperlink{c06.xhtmlux5cux23figure6-3}{Figure 6.3}.

You configure a DCE serial interface with the \texttt{clock\ rate}
command:

\begin{verbatim}
Router#config t
Enter configuration commands, one per line.  End with CNTL/Z.
Router(config)#int s0/0/0
Router(config-if)#clock rate ?
        Speed (bits per second)
  1200
  2400
  4800
  9600
  14400
  19200
  28800
  32000
  38400
  48000
  56000
  57600
  64000
  72000
  115200
  125000
  128000
  148000
  192000
  250000
  256000
  384000
  500000
  512000
  768000
  800000
  1000000
  2000000
  4000000
  5300000
  8000000
\end{verbatim}

\begin{verbatim}
  <300-8000000>    Choose clockrate from list above
Router(config-if)#clock rate 1000000
\end{verbatim}

The \texttt{clock\ rate} command is set in bits per second. Besides
looking at the cable end to check for a label of DCE or DTE, you can see
if a router's serial interface has a DCE cable connected with the
\texttt{show\ controllers} \texttt{int} command:

\begin{verbatim}
Router#sh controllers s0/0/0
Interface Serial0/0/0
Hardware is GT96K
DTE V.35idb at 0x4342FCB0, driver data structure at 0x434373D4
\end{verbatim}

Here is an example of an output depicting a DCE connection:

\begin{verbatim}
Router#sh controllers s0/2/0
Interface Serial0/2/0
Hardware is GT96K
DCE V.35, clock rate 1000000
\end{verbatim}

The next command you need to get acquainted with is the
\texttt{bandwidth} command. Every Cisco router ships with a default
serial link bandwidth of T1 (1.544 Mbps). But this has nothing to do
with how data is transferred over a link. The bandwidth of a serial link
is used by routing protocols such as EIGRP and OSPF to calculate the
best cost path to a remote network. So if you're using RIP routing, the
bandwidth setting of a serial link is irrelevant since RIP uses only hop
count to determine this.

\begin{center}\rule{0.5\linewidth}{0.5pt}\end{center}

\includegraphics{images/tip.png} You may be rereading this part and
thinking, ``Huh? What? Routing protocols? Metrics?'' But don't freak!
I'm going over all of that soon in Chapter 9.

\begin{center}\rule{0.5\linewidth}{0.5pt}\end{center}

\protect\hypertarget{c06.xhtmlux5cux23Page_240}{}{}Here's an example of
using the \texttt{bandwidth} command:

\begin{verbatim}
Router#config t
Router(config)#int s0/0/0
Router(config-if)#bandwidth ?
  <1-10000000>  Bandwidth in kilobits
  inherit       Specify that bandwidth is inherited
  receive       Specify receive-side bandwidth
Router(config-if)#bandwidth 1000
\end{verbatim}

Did you notice that, unlike the \texttt{clock\ rate} command, the
\texttt{bandwidth} command is configured in kilobits per second?

\begin{center}\rule{0.5\linewidth}{0.5pt}\end{center}

\includegraphics{images/note.png} After going through all these
configuration examples regarding the \texttt{clock\ rate} command,
understand that the new ISR routers automatically detect DCE connections
and set \texttt{clock\ rate} to 2000000. But know that you still need to
understand the \texttt{clock\ rate} command for the Cisco objectives,
even though the new routers set it for you automatically!

\begin{center}\rule{0.5\linewidth}{0.5pt}\end{center}

\subsection[Viewing, Saving, and Erasing
Configurations]{\texorpdfstring{\protect\hypertarget{c06.xhtmlux5cux23c06-sec-18}{}{}Viewing,
Saving, and Erasing
Configurations}{Viewing, Saving, and Erasing Configurations}}

If you run through setup mode, you'll be asked if you want to use the
configuration you just created. If you say yes, the configuration
running in DRAM that's known as the \texttt{running-config} will be
copied into NVRAM, and the file will be named \texttt{startup-config}.
Hopefully, you'll be smart and always use the CLI, not setup mode!

You can manually save the file from DRAM, which is usually just called
RAM, to NVRAM by using the \texttt{copy\ running-config\ startup-config}
command. You can use the shortcut \texttt{copy\ run\ start} as well:

\begin{verbatim}
Todd#copy running-config startup-config
Destination filename [startup-config]? [press enter]
Building configuration...
[OK]
Todd#
Building configuration...
\end{verbatim}

When you see a question with an answer in \texttt{{[}{]}}, it means that
if you just press Enter, you're choosing the default answer.

\protect\hypertarget{c06.xhtmlux5cux23Page_241}{}{}Also, when the
command asks for the destination filename, the default answer is
startup-config. The reason it asks is because you can copy the
configuration to pretty much anywhere you want. Take a look at the
output from my switch:

\begin{verbatim}
Todd#copy running-config ?
\end{verbatim}

\begin{verbatim}
  flash:          Copy to flash: file system
  ftp:            Copy to ftp: file system
  http:           Copy to http: file system
  https:          Copy to https: file system
  null:           Copy to null: file system
  nvram:          Copy to nvram: file system
  rcp:            Copy to rcp: file system
  running-config  Update (merge with) current system configuration
  scp:            Copy to scp: file system
  startup-config  Copy to startup configuration
  syslog:         Copy to syslog: file system
  system:         Copy to system: file system
  tftp:           Copy to tftp: file system
  tmpsys:         Copy to tmpsys: file system
  vb:             Copy to vb: file system
\end{verbatim}

To reassure you, we'll get deeper into how and where to copy files in
Chapter 7.

For now, you can view the files by typing \texttt{show\ running-config}
or \texttt{show\ startup-config} from privileged mode. The
\texttt{sh\ run} command, which is a shortcut for
\texttt{show\ ­running-config}, tells us that we're viewing the current
configuration:

\begin{verbatim}
Todd#sh run
Building configuration...
\end{verbatim}

\begin{verbatim}
Current configuration : 855 bytes
!
! Last configuration change at 23:20:06 UTC Mon Mar 1 1993
!
version 15.0
[output cut]
\end{verbatim}

The \texttt{sh\ start} command---one of the shortcuts for the
\texttt{show\ startup-config} ­command---shows us the configuration that
will be used the next time the router is reloaded. It also tells us how
much NVRAM is being used to store the startup-config file. Here's an
example:

\begin{verbatim}
Todd#sh start
Using 855 out of 524288 bytes
!
! Last configuration change at 23:20:06 UTC Mon Mar 1 1993
!
version 15.0
[output cut]
\end{verbatim}

But beware---if you try and view the configuration and see

\begin{verbatim}
Todd#sh start
startup-config is not present
\end{verbatim}

you have not saved your running-config to NVRAM, or you've deleted the
backup configuration! Let me talk about just how you would do that now.

\subsubsection[Deleting the Configuration and Reloading the
Device]{\texorpdfstring{\protect\hypertarget{c06.xhtmlux5cux23c06-sec-19}{}{}Deleting
the Configuration and Reloading the
Device}{Deleting the Configuration and Reloading the Device}}

You can delete the startup-config file by using the
\texttt{erase\ startup-config} command:

\begin{verbatim}
Todd#erase start
% Incomplete command.
\end{verbatim}

First, notice that you can no longer use the shortcut commands for
erasing the backup configuration. This started in IOS 12.4 with the ISR
routers.

\begin{verbatim}
Todd#erase startup-config
Erasing the nvram filesystem will remove all configuration files! Continue? [confirm]
[OK]
Erase of nvram: complete
Todd#
*Mar  5 01:59:45.206: %SYS-7-NV_BLOCK_INIT: Initialized the geometry of nvram
Todd#reload
Proceed with reload? [confirm]
\end{verbatim}

Now if you reload or power the router down after using the
\texttt{erase\ startup-­config} command, you'll be offered setup mode
because there's no configuration saved in NVRAM. You can press Ctrl+C to
exit setup mode at any time, but the \texttt{reload} command can only be
used from privileged mode.

At this point, you shouldn't use setup mode to configure your router. So
just say \texttt{no} to setup mode, because it's there to help people
who don't know how to use the command line interface (CLI), and this no
longer applies to you. Be strong---you can do it!

\subsubsection[Verifying Your
Configuration]{\texorpdfstring{\protect\hypertarget{c06.xhtmlux5cux23c06-sec-20}{}{}Verifying
Your Configuration}{Verifying Your Configuration}}

Obviously, \texttt{show\ running-config} would be the best way to verify
your configuration and \texttt{show\ startup-config} would be the best
way to verify the configuration that'll be used the next time the router
is reloaded---right?

Well, once you take a look at the running-config, if all appears well,
you can verify your configuration with utilities like Ping and Telnet.
Ping is a program that uses ICMP echo
\protect\hypertarget{c06.xhtmlux5cux23Page_243}{}{}requests and replies,
which we covered in Chapter 3. For review, Ping sends a packet to a
remote host, and if that host responds, you know that it's alive. But
you don't know if it's alive and also \emph{well}; just because you can
ping a Microsoft server does not mean you can log in! Even so, Ping is
an awesome starting point for troubleshooting an internetwork.

Did you know that you can ping with different protocols? You can, and
you can test this by typing \texttt{ping\ ?} at either the router
user-mode or privileged-mode prompt:

\begin{verbatim}
Todd#ping ?
  WORD  Ping destination address or hostname
  clns  CLNS echo
  ip    IP echo
  ipv6  IPv6 echo
  tag   Tag encapsulated IP echo
  <cr>
\end{verbatim}

If you want to find a neighbor's Network layer address, either you go
straight to the router or switch itself or you can type
\texttt{show\ cdp\ entry\ *\ protocol} to get the Network layer
addresses you need for pinging.

You can also use an extended ping to change the default variables, as
shown here:

\begin{verbatim}
Todd#ping
Protocol [ip]:
Target IP address: 10.1.1.1
Repeat count [5]:
% A decimal number between 1 and 2147483647.
Repeat count [5]: 5000
Datagram size [100]:
% A decimal number between 36 and 18024.
Datagram size [100]: 1500
Timeout in seconds [2]:
Extended commands [n]: y
Source address or interface: FastEthernet 0/1
Source address or interface: Vlan 1
Type of service [0]:
Set DF bit in IP header? [no]:
Validate reply data? [no]:
Data pattern [0xABCD]:
Loose, Strict, Record, Timestamp, Verbose[none]:
Sweep range of sizes [n]:
Type escape sequence to abort.
Sending 5000, 1500-byte ICMP Echos to 10.1.1.1, timeout is 2 seconds:
Packet sent with a source address of 10.10.10.1
\end{verbatim}

Notice that by using the question mark, I was able to determine that
extended ping allows you to set the repeat count higher than the default
of 5 and the datagram size larger.
\protect\hypertarget{c06.xhtmlux5cux23Page_244}{}{}This raises the MTU
and allows for a more accurate testing of throughput. The source
interface is one last important piece of information I'll pull out of
the output. You can choose which interface the ping is sourced from,
which is really helpful in certain diagnostic situations. Using my
switch to display the extended ping capabilities, I had to use my only
routed port, which is named VLAN 1, by default.

However, if you want to use a different diagnostic port, you can create
a logical interface called a loopback interface as so:

\begin{verbatim}
Todd(config)#interface loopback ?
  <0-2147483647>  Loopback interface number
\end{verbatim}

\begin{verbatim}
Todd(config)#interface loopback 0
*May 19 03:06:42.697: %LINEPROTO-5-UPDOWN: Line prot
 changed state to ups
Todd(config-if)#ip address 20.20.20.1 255.255.255.0
\end{verbatim}

Now I can use this port for diagnostics, and even as my source port of
my ping or traceroute, as so:

\begin{verbatim}
Todd#ping
Protocol [ip]:
Target IP address: 10.1.1.1
Repeat count [5]:
Datagram size [100]:
Timeout in seconds [2]:
Extended commands [n]: y
Source address or interface: 20.20.20.1
Type of service [0]:
Set DF bit in IP header? [no]:
Validate reply data? [no]:
Data pattern [0xABCD]:
Loose, Strict, Record, Timestamp, Verbose[none]:
Sweep range of sizes [n]:
Type escape sequence to abort.
Sending 5, 100-byte ICMP Echos to 10.1.1.1, timeout is 2 seconds:
Packet sent with a source address of 20.20.20.1
\end{verbatim}

The logical interface are great for diagnostics and for using them in
our home labs where we don't have any real interfaces to play with, but
we'll also use them in our OSPF configurations in ICND2.

\begin{center}\rule{0.5\linewidth}{0.5pt}\end{center}

\includegraphics{images/note.png} Cisco Discovery Protocol (CDP) is
covered in Chapter 7.

\begin{center}\rule{0.5\linewidth}{0.5pt}\end{center}

\protect\hypertarget{c06.xhtmlux5cux23Page_245}{}{}Traceroute uses ICMP
with IP time to live (TTL) time-outs to track the path a given packet
takes through an internetwork. This is in contrast to Ping, which just
finds the host and responds. Traceroute can also be used with multiple
protocols. Check out this output:

\begin{verbatim}
Todd#traceroute ?
  WORD       Trace route to destination address or hostname
  aaa        Define trace options for AAA events/actions/errors
  appletalk  AppleTalk Trace
  clns       ISO CLNS Trace
  ip         IP Trace
  ipv6       IPv6 Trace
  ipx        IPX Trace
  mac        Trace Layer2 path between 2 endpoints
  oldvines   Vines Trace (Cisco)
  vines      Vines Trace (Banyan)
  <cr>
\end{verbatim}

And as with ping, we can perform an extended traceroute using additional
parameters, typically used to change the source interface:

\begin{verbatim}
Todd#traceroute
Protocol [ip]: 
Target IP address: 10.1.1.1
Source address: 172.16.10.1
Numeric display [n]: 
Timeout in seconds [3]: 
Probe count [3]: 
Minimum Time to Live [1]: 255
Maximum Time to Live [30]: 
Type escape sequence to abort.
Tracing the route to 10.1.1.1
\end{verbatim}

Telnet, FTP, and HTTP are really the best tools because they use IP at
the Network layer and TCP at the Transport layer to create a session
with a remote host. If you can telnet, ftp, or http into a device, you
know that your IP connectivity just has to be solid!

\begin{verbatim}
Todd#telnet ?
 WORD IP address or hostname of a remote system
 <cr>
Todd#telnet 10.1.1.1
\end{verbatim}

When you telnet into a remote device, you won't see console messages by
default. For example, you will not see debugging output. To allow
console messages to be sent to your Telnet session, use the terminal
monitor command, as shown on the SF router.

\begin{verbatim}
SF#terminal monitor
\end{verbatim}

\protect\hypertarget{c06.xhtmlux5cux23Page_246}{}{}From the switch or
router prompt, you just type a hostname or IP address and it will assume
you want to telnet---you don't need to type the actual command,
\texttt{telnet}.

Coming up, I'll show you how to verify the interface statistics.

\paragraph{\texorpdfstring{Verifying with the \emph{\textbf{show
interface}} Command}{Verifying with the show interface Command}}

Another way to verify your configuration is by typing
\texttt{show\ interface} commands, the first of which is the
\texttt{show\ interface\ ?} command. Doing this will reveal all the
available interfaces to verify and configure.

\begin{center}\rule{0.5\linewidth}{0.5pt}\end{center}

\includegraphics{images/note.png} The \texttt{show\ interfaces} command,
plural, displays the configurable parameters and statistics of all
interfaces on a router.

\begin{center}\rule{0.5\linewidth}{0.5pt}\end{center}

This command comes in really handy when you're verifying and
troubleshooting router and network issues.

The following output is from my freshly erased and rebooted 2811 router:

\begin{verbatim}
Router#sh int ?
  Async              Async interface
  BVI                Bridge-Group Virtual Interface
  CDMA-Ix            CDMA Ix interface
  CTunnel            CTunnel interface
  Dialer             Dialer interface
  FastEthernet       FastEthernet IEEE 802.3
  Loopback           Loopback interface
  MFR                Multilink Frame Relay bundle interface
  Multilink          Multilink-group interface
  Null               Null interface
  Port-channel       Ethernet Channel of interfaces
  Serial             Serial
  Tunnel             Tunnel interface
  Vif                PGM Multicast Host interface
  Virtual-PPP        Virtual PPP interface
  Virtual-Template   Virtual Template interface
  Virtual-TokenRing  Virtual TokenRing
  accounting         Show interface accounting
  counters           Show interface counters
  crb                Show interface routing/bridging info
  dampening          Show interface dampening info
  description        Show interface description
  etherchannel       Show interface etherchannel information
  irb                Show interface routing/bridging info
  mac-accounting     Show interface MAC accounting info
  mpls-exp           Show interface MPLS experimental accounting info
  precedence         Show interface precedence accounting info
  pruning            Show interface trunk VTP pruning information
  rate-limit         Show interface rate-limit info
  status             Show interface line status
  summary            Show interface summary
  switching          Show interface switching
  switchport         Show interface switchport information
  trunk              Show interface trunk information
  |                  Output modifiers
  <cr>
\end{verbatim}

The only ``real'' physical interfaces are FastEthernet, Serial, and
Async---the rest are all logical interfaces or commands you can use to
verify with.

The next command is \texttt{show\ interface\ fastethernet\ 0/0}. It
reveals the hardware address, logical address, and encapsulation method
as well as statistics on collisions, as seen here:

\begin{verbatim}
Router#sh int f0/0
FastEthernet0/0 is up, line protocol is up
  Hardware is MV96340 Ethernet, address is 001a.2f55.c9e8 (bia 001a.2f55.c9e8)
  Internet address is 192.168.1.33/27
MTU 1500 bytes, BW 100000 Kbit, DLY 100 usec,
     reliability 255/255, txload 1/255, rxload 1/255
  Encapsulation ARPA, loopback not set
  Keepalive set (10 sec)
  Auto-duplex, Auto Speed, 100BaseTX/FX
  ARP type: ARPA, ARP Timeout 04:00:00
  Last input never, output 00:02:07, output hang never
  Last clearing of "show interface" counters never
  Input queue: 0/75/0/0 (size/max/drops/flushes); Total output drops: 0
  Queueing strategy: fifo
  Output queue: 0/40 (size/max)
  5 minute input rate 0 bits/sec, 0 packets/sec
  5 minute output rate 0 bits/sec, 0 packets/sec
     0 packets input, 0 bytes
     Received 0 broadcasts, 0 runts, 0 giants, 0 throttles
     0 input errors, 0 CRC, 0 frame, 0 overrun, 0 ignored
     0 watchdog
     0 input packets with dribble condition detected
     16 packets output, 960 bytes, 0 underruns
     0 output errors, 0 collisions, 0 interface resets
     0 babbles, 0 late collision, 0 deferred
     0 lost carrier, 0 no carrier
     0 output buffer failures, 0 output buffers swapped out
Router#
\end{verbatim}

You probably guessed that we're going to go over the important
statistics from this output, but first, just for fun, I've got to ask
you, which subnet is FastEthernet 0/0 a member of and what's the
broadcast address and valid host range?

I'm serious---you really have to be able to nail these things
NASCAR-fast! Just in case you didn't, the address is 192.168.1.33/27.
And I've gotta be honest---if you don't know what a /27 is at this
point, you'll need a miracle to pass the exam! That or you need to
actually read this book. (As a quick reminder, a /27 is
255.255.255.224.) The fourth octet is a block size of 32. The subnets
are 0, 32, 64, etc.; the FastEthernet interface is in the 32 subnet; the
broadcast address is 63; and the valid hosts are 33--62. All good now?

\begin{center}\rule{0.5\linewidth}{0.5pt}\end{center}

\includegraphics{images/note.png} If you struggled with any of this,
please save yourself from certain doom and get yourself back into
Chapter 4, ``Easy Subnetting,'' now! Read and reread it until you've got
it dialed in!

\begin{center}\rule{0.5\linewidth}{0.5pt}\end{center}

Okay---back to the output. The preceding interface is working and looks
to be in good shape. The \texttt{show\ interfaces} command will show you
if you're receiving errors on the interface, and it will also show you
the maximum transmission unit (MTU). MTU is the maximum packet size
allowed to transmit on that interface, bandwidth (BW) is for use with
routing protocols, and 255/255 means that reliability is perfect! The
load is 1/255, meaning no load.

Continuing through the output, can you figure out the bandwidth of the
interface? Well, other than the easy giveaway of the interface being
called a ``FastEthernet'' interface, we can see that the bandwidth is
100000 Kbit, which is 100,000,000. Kbit means to add three zeros, which
is 100 Mbits per second, or FastEthernet. Gigabit would be 1000000 Kbits
per second.

Be sure you don't miss the output errors and collisions, which show 0 in
my output. If these numbers are increasing, then you have some sort of
Physical or Data Link layer issue. Check your duplex! If you have one
side as half-duplex and one at full-duplex, your interface will work,
albeit really slow and those numbers will be increasing fast!

The most important statistic of the \texttt{show\ interface} command is
the output of the line and Data Link protocol status. If the output
reveals that FastEthernet 0/0 is up and the line protocol is up, then
the interface is up and running:

\begin{verbatim}
Router#sh int fa0/0
FastEthernet0/0 is up, line protocol is up
\end{verbatim}

The first parameter refers to the Physical layer, and it's up when it
receives carrier detect. The second parameter refers to the Data Link
layer, and it looks for keepalives from the connecting end. Keepalives
are important because they're used between devices to make sure
connectivity hasn't been dropped.

\protect\hypertarget{c06.xhtmlux5cux23Page_249}{}{}Here's an example of
where your problem will often be found---on serial interfaces:

\begin{verbatim}
Router#sh int s0/0/0
Serial0/0 is up, line protocol is down
\end{verbatim}

If you see that the line is up but the protocol is down, as displayed
here, you're experiencing a clocking (keepalive) or framing
problem---possibly an encapsulation mismatch. Check the keepalives on
both ends to make sure they match. Make sure that the clock rate is set,
if needed, and that the encapsulation type is equal on both ends. The
preceding output tells us that there's a Data Link layer problem.

If you discover that both the line interface and the protocol are down,
it's a cable or interface problem. The following output would indicate a
Physical layer problem:

\begin{verbatim}
Router#sh int s0/0/0
Serial0/0 is down, line protocol is down
\end{verbatim}

As you'll see next, if one end is administratively shut down, the remote
end would present as down and down:

\begin{verbatim}
Router#sh int s0/0/0
Serial0/0 is administratively down, line protocol is down
\end{verbatim}

To enable the interface, use the command \texttt{no\ shutdown} from
interface configuration mode.

The next \texttt{show\ interface\ serial\ 0/0/0} command demonstrates
the serial line and the maximum transmission unit (MTU)---1,500 bytes by
default. It also shows the default bandwidth (BW) on all Cisco serial
links, which is 1.544 Kbps. This is used to determine the bandwidth of
the line for routing protocols like EIGRP and OSPF. Another important
configuration to notice is the keepalive, which is 10 seconds by
default. Each router sends a keepalive message to its neighbor every 10
seconds, and if both routers aren't configured for the same keepalive
time, it won't work! Check out this output:

\begin{verbatim}
Router#sh int s0/0/0
Serial0/0 is up, line protocol is up
 Hardware is HD64570
 MTU 1500 bytes, BW 1544 Kbit, DLY 20000 usec,
   reliability 255/255, txload 1/255, rxload 1/255
 Encapsulation HDLC, loopback not set, keepalive set
  (10 sec)
 Last input never, output never, output hang never
 Last clearing of "show interface" counters never
 Queueing strategy: fifo
 Output queue 0/40, 0 drops; input queue 0/75, 0 drops
 5 minute input rate 0 bits/sec, 0 packets/sec
 5 minute output rate 0 bits/sec, 0 packets/sec
   0 packets input, 0 bytes, 0 no buffer
   Received 0 broadcasts, 0 runts, 0 giants, 0 throttles
   0 input errors, 0 CRC, 0 frame, 0 overrun, 0 ignored,
   0 abort
   0 packets output, 0 bytes, 0 underruns
   0 output errors, 0 collisions, 16 interface resets
   0 output buffer failures, 0 output buffers swapped out
   0 carrier transitions
   DCD=down DSR=down DTR=down RTS=down CTS=down
\end{verbatim}

You can clear the counters on the interface by typing the command
\texttt{clear\ counters}:

\begin{verbatim}
Router#clear counters ?
  Async              Async interface
  BVI                Bridge-Group Virtual Interface
  CTunnel            CTunnel interface
  Dialer             Dialer interface
  FastEthernet       FastEthernet IEEE 802.3
  Group-Async        Async Group interface
  Line               Terminal line
  Loopback           Loopback interface
  MFR                Multilink Frame Relay bundle interface
  Multilink          Multilink-group interface
  Null               Null interface
  Serial             Serial
  Tunnel             Tunnel interface
  Vif                PGM Multicast Host interface
  Virtual-Template   Virtual Template interface
  Virtual-TokenRing  Virtual TokenRing
  <cr>
\end{verbatim}

\begin{verbatim}
Router#clear counters s0/0/0
Clear "show interface" counters on this interface
  [confirm][enter]
Router#
00:17:35: %CLEAR-5-COUNTERS: Clear counter on interface
  Serial0/0/0 by console
Router#
\end{verbatim}

\paragraph{\texorpdfstring{Troubleshooting with the \emph{\textbf{show
interfaces}} Command}{Troubleshooting with the show interfaces Command}}

Let's take a look at the output of the \texttt{show\ interfaces} command
one more time before I move on. There are some statistics in this output
that are important for the Cisco objectives.

\begin{verbatim}
  275496 packets input, 35226811 bytes, 0 no buffer
     Received 69748 broadcasts (58822 multicasts)
     0 runts, 0 giants, 0 throttles
     0 input errors, 0 CRC, 0 frame, 0 overrun, 0 ignored
     0 watchdog, 58822 multicast, 0 pause input
     0 input packets with dribble condition detected
     2392529 packets output, 337933522 bytes, 0 underruns
     0 output errors, 0 collisions, 1 interface resets
     0 babbles, 0 late collision, 0 deferred
     0 lost carrier, 0 no carrier, 0 PAUSE output
     0 output buffer failures, 0 output buffers swapped out
\end{verbatim}

Finding where to start when troubleshooting an interface can be the
difficult part, but certainly we'll look for the number of input errors
and CRCs right away. Typically we'd see those statistics increase with a
duplex error, but it could be another Physical layer issue such as the
cable might be receiving excessive interference or the network interface
cards might have a failure. Typically you can tell if it is interference
when the CRC and input errors output grow but the collision counters do
not.

Let's take a look at some of the output:

\textbf{No buffer} This isn't a number you want to see incrementing.
This means you don't have any buffer room left for incoming packets. Any
packets received once the buffers are full are discarded. You can see
how many packets are dropped with the ignored output.

\textbf{Ignored} If the packet buffers are full, packets will be
dropped. You see this increment along with the no buffer output.
Typically if the no buffer and ignored outputs are incrementing, you
have some sort of broadcast storm on your LAN. This can be caused by a
bad NIC or even a bad network design.

\begin{center}\rule{0.5\linewidth}{0.5pt}\end{center}

\includegraphics{images/tip.png} I'll repeat this because it is so
important for the exam objectives: Typically if the no buffer and
ignored outputs are incrementing, you have some sort of broadcast storm
on your LAN. This can be caused by a bad NIC or even a bad network
design.

\begin{center}\rule{0.5\linewidth}{0.5pt}\end{center}

\textbf{Runts} Frames that did not meet the minimum frame size
requirement of 64 bytes. Typically caused by collisions.

\textbf{Giants} Frames received that are larger than 1518 bytes

\textbf{Input Errors} This is the total of many counters: runts, giants,
no buffer, CRC, frame, overrun, and ignored counts.

\textbf{CRC} At the end of each frame is a Frame Check Sequence (FCS)
field that holds the answer to a cyclic redundancy check (CRC). If the
receiving host's answer to the CRC does not match the sending host's
answer, then a CRC error will occur.

\textbf{Frame} This output increments when frames received are of an
illegal format, or not complete, which is typically incremented when a
collision occurs.

\protect\hypertarget{c06.xhtmlux5cux23Page_252}{}{}\textbf{Packets
Output} Total number of packets (frames) forwarded out to the interface.

\textbf{Output Errors} Total number of packets (frames) that the switch
port tried to transmit but for which some problem occurred.

\textbf{Collisions} When transmitting a frame in half-duplex, the NIC
listens on the receiving pair of the cable for another signal. If a
signal is transmitted from another host, a collision has occurred. This
output should not increment if you are running full-duplex.

\textbf{Late Collisions} If all Ethernet specifications are followed
during the cable install, all collisions should occur by the 64th byte
of the frame. If a collision occurs after 64 bytes, the late collisions
counter increments. This counter will increment on a duplex mismatched
interface, or if cable length exceeds specifications.

\begin{center}\rule{0.5\linewidth}{0.5pt}\end{center}

\includegraphics{images/tip.png} A duplex mismatch causes late collision
errors at the end of the connection. To avoid this situation, manually
set the duplex parameters of the switch to match the attached device.

\begin{center}\rule{0.5\linewidth}{0.5pt}\end{center}

A duplex mismatch is a situation in which the switch operates at
full-duplex and the connected device operates at half-duplex, or vice
versa. The result of a duplex mismatch is extremely slow performance,
intermittent connectivity, and loss of connection. Other possible causes
of data-link errors at full-duplex are bad cables, a faulty switch port,
or NIC software or hardware issues. Use the \texttt{show\ interface}
command to verify the duplex settings.

If the mismatch occurs between two Cisco devices with Cisco Discovery
Protocol enabled, you will see Cisco Discovery Protocol error messages
on the console or in the logging buffer of both devices.

\begin{verbatim}
%CDP-4-DUPLEX_MISMATCH: duplex mismatch discovered on FastEthernet0/2 (not
half duplex)
\end{verbatim}

Cisco Discovery Protocol is useful for detecting errors and for
gathering port and system statistics on nearby Cisco devices. CDP is
covered in Chapter 7.

\paragraph{\texorpdfstring{Verifying with the \emph{\textbf{show ip
interface}} Command}{Verifying with the show ip interface Command}}

The \texttt{show\ ip\ interface} command will provide you with
information regarding the layer 3 configurations of a router's
interface, such as the IP address and subnet mask, MTU, and if an access
list is set on the interface:

\begin{verbatim}
Router#sh ip interface
FastEthernet0/0 is up, line protocol is up
  Internet address is 1.1.1.1/24
  Broadcast address is 255.255.255.255
  Address determined by setup command
  MTU is 1500 bytes
  Helper address is not set
  Directed broadcast forwarding is disabled
  Outgoing access list is not set
  Inbound  access list is not set
  Proxy ARP is enabled
  Security level is default
  Split horizon is enabled
[output cut]
\end{verbatim}

The status of the interface, the IP address and mask, information on
whether an access list is set on the interface, and basic IP information
are all included in this output.

\paragraph{\texorpdfstring{Using the \emph{\textbf{show ip interface
brief}} Command}{Using the show ip interface brief Command}}

The \texttt{show\ ip\ interface\ brief} command is probably one of the
best commands that you can ever use on a Cisco router or switch. This
command provides a quick overview of the devices interfaces, including
the logical address and status:

\begin{verbatim}
Router#sh ip int brief
Interface         IP-Address      OK? Method Status  Protocol
FastEthernet0/0    unassigned      YES unset  up       up
FastEthernet0/1    unassigned      YES unset  up       up
Serial0/0/0        unassigned      YES unset  up       down
Serial0/0/1        unassigned      YES unset  administratively down down
Serial0/1/0        unassigned      YES unset  administratively down down
Serial0/2/0        unassigned      YES unset  administratively down down
\end{verbatim}

Remember, administratively down means that you need to type
\texttt{no\ shutdown} in order to enable the interface. Notice that
Serial0/0/0 is up/down, which means that the Physical layer is good and
carrier detect is sensed but no keepalives are being received from the
remote end. In a nonproduction network, like the one I am working with,
this tells us the clock rate hasn't been set.

\paragraph{\texorpdfstring{Verifying with the \emph{\textbf{show
protocols}} Command}{Verifying with the show protocols Command}}

The \texttt{show\ protocols} command is also a really helpful command
that you'd use in order to quickly see the status of layers 1 and 2 of
each interface as well as the IP addresses used.

Here's a look at one of my production routers:

\begin{verbatim}
Router#sh protocols
Global values:
  Internet Protocol routing is enabled
Ethernet0/0 is administratively down, line protocol is down
Serial0/0 is up, line protocol is up
  Internet address is 100.30.31.5/24
Serial0/1 is administratively down, line protocol is down
Serial0/2 is up, line protocol is up
  Internet address is 100.50.31.2/24
Loopback0 is up, line protocol is up
  Internet address is 100.20.31.1/24
\end{verbatim}

The \texttt{show\ ip\ interface\ brief} and \texttt{show\ protocols}
commands provide the layer 1 and layer 2 statistics of an interface as
well as the IP addresses. The next command, \texttt{show\ controllers},
only provides layer 1 information. Let's take a look.

\paragraph{\texorpdfstring{Using the \emph{\textbf{show controllers}}
Command}{Using the show controllers Command}}

The \texttt{show\ controllers} command displays information about the
physical interface itself. It'll also give you the type of serial cable
plugged into a serial port. Usually, this will only be a DTE cable that
plugs into a type of data service unit (DSU).

\begin{verbatim}
Router#sh controllers serial 0/0
HD unit 0, idb = 0x1229E4, driver structure at 0x127E70
buffer size 1524 HD unit 0, V.35 DTE cable
\end{verbatim}

\begin{verbatim}
Router#sh controllers serial 0/1
HD unit 1, idb = 0x12C174, driver structure at 0x131600
buffer size 1524 HD unit 1, V.35 DCE cable
\end{verbatim}

Notice that serial 0/0 has a DTE cable, whereas the serial 0/1
connection has a DCE cable. Serial 0/1 would have to provide clocking
with the \texttt{clock\ rate} command. Serial 0/0 would get its clocking
from the DSU.

Let's look at this command again. In
\protect\hyperlink{c06.xhtmlux5cux23figure6-5}{Figure 6.5}, see the
DTE/DCE cable between the two routers? Know that you will not see this
in production networks!

\begin{figure}
\centering
\includegraphics{images/c06f005.jpg}
\caption{{\protect\hyperlink{c06.xhtmlux5cux23figureanchor6-5}{\textbf{FIGURE
6.5}} Where do you configure clocking? Use the
\texttt{show\ controllers} command on each router's serial interface to
find out.}}
\end{figure}

Router R1 has a DTE connection, which is typically the default for all
Cisco routers. Routers R1 and R2 can't communicate. Check out the output
of the \texttt{show\ controllers\ s0/0} command here:

\begin{verbatim}
R1#sh controllers serial 0/0
HD unit 0, idb = 0x1229E4, driver structure at 0x127E70
buffer size 1524 HD unit 0, V.35 DCE cable
\end{verbatim}

\protect\hypertarget{c06.xhtmlux5cux23Page_255}{}{}The
\texttt{show\ controllers\ s0/0} command reveals that the interface is a
V.35 DCE cable. This means that R1 needs to provide clocking of the line
to router R2. Basically, the interface has the wrong label on the cable
on the R1 router's serial interface. But if you add clocking on the R1
router's serial interface, the network should come right up.

Let's check out another issue in
\protect\hyperlink{c06.xhtmlux5cux23figure6-6}{Figure 6.6} that you can
solve by using the \texttt{show\ ­controllers} command. Again, routers
R1 and R2 can't communicate.

\begin{figure}
\centering
\includegraphics{images/c06f006.jpg}
\caption{{\protect\hyperlink{c06.xhtmlux5cux23figureanchor6-6}{\textbf{FIGURE
6.6}} By looking at R1, the \texttt{show\ controllers} command reveals
that R1 and R2 can't communicate.}}
\end{figure}

Here's the output of R1's \texttt{show\ controllers\ s0/0} command and
\texttt{show\ ip\ interface\ s0/0}:

\begin{verbatim}
R1#sh controllers s0/0
HD unit 0, idb = 0x1229E4, driver structure at 0x127E70
buffer size 1524 HD unit 0,
DTE V.35 clocks stopped
cpb = 0xE2, eda = 0x4140, cda = 0x4000
\end{verbatim}

\begin{verbatim}
R1#sh ip interface s0/0
Serial0/0 is up, line protocol is down
  Internet address is 192.168.10.2/24
  Broadcast address is 255.255.255.255
\end{verbatim}

If you use the \texttt{show\ controllers} command and the
\texttt{show\ ip\ interface} command, you'll see that router R1 isn't
receiving the clocking of the line. This network is a nonproduction
network, so no CSU/DSU is connected to provide clocking for it. This
means the DCE end of the cable will be providing the clock rate---in
this case, the R2 router. The \texttt{show\ ip\ interface} indicates
that the interface is up but the protocol is down, which means that no
keepalives are being received from the far end. In this example, the
likely culprit is the result of bad cable, or simply the lack of
clocking.

\subsection[Summary]{\texorpdfstring{\protect\hypertarget{c06.xhtmlux5cux23c06-sec-21}{}{}Summary}{Summary}}

This was a fun chapter! I showed you a lot about the Cisco IOS, and I
really hope you gained a lot of insight into the Cisco router world. I
started off by explaining the Cisco Internetwork Operating System (IOS)
and how you can use the IOS to run and configure Cisco routers. You
learned how to bring a router up and what setup mode does. Oh, and by
the way, since you can now basically configure Cisco routers, you should
never use setup mode, right?

\protect\hypertarget{c06.xhtmlux5cux23Page_256}{}{}After I discussed how
to connect to a router with a console and LAN connection, I covered the
Cisco help features and how to use the CLI to find commands and command
parameters. In addition, I discussed some basic \texttt{show} commands
to help you verify your configurations.

Administrative functions on a router help you administer your network
and verify that you are configuring the correct device. Setting router
passwords is one of the most important configurations you can perform on
your routers. I showed you the five passwords you must set, plus I
introduced you to the hostname, interface description, and banners as
tools to help you administer your router.

Well, that concludes your introduction to the Cisco IOS. And, as usual,
it's super-important for you to have the basics that we went over in
this chapter down rock-solid before you move on to the following
chapters!

\subsection[Exam
Essentials]{\texorpdfstring{\protect\hypertarget{c06.xhtmlux5cux23c06-sec-22}{}{}Exam
Essentials}{Exam Essentials}}

\textbf{Describe the responsibilities of the IOS.} The Cisco router IOS
software is responsible for network protocols and providing supporting
functions, connecting high-speed traffic between devices, adding
security to control access and prevent unauthorized network use,
providing scalability for ease of network growth and redundancy, and
supplying network reliability for connecting to network resources.

\textbf{List the options available to connect to a Cisco device for
management purposes.} The three options available are the console port,
auxiliary port, and in-band communication, such as Telnet, SSH, and
HTTP. Don't forget, a Telnet connection is not possible until an IP
address has been configured and a Telnet password has been configured.

\textbf{Understand the boot sequence of a router.} When you first bring
up a Cisco router, it will run a power-on self-test (POST), and if that
passes, it will look for and load the Cisco IOS from flash memory, if a
file is present. The IOS then proceeds to load and looks for a valid
configuration in NVRAM called the startup-config. If no file is present
in NVRAM, the router will go into setup mode.

\textbf{Describe the use of setup mode.} Setup mode is automatically
started if a router boots and no startup-config is in NVRAM. You can
also bring up setup mode by typing \texttt{setup} from privileged mode.
Setup provides a minimum amount of configuration in an easy format for
someone who does not understand how to configure a Cisco router from the
command line.

\textbf{Differentiate user, privileged, and global configuration modes,
both visually and from a command capabilities perspective.} User mode,
indicated by the \texttt{routername\textgreater{}} prompt, provides a
command-line interface with very few available commands by default. User
mode does not allow the configuration to be viewed or changed.
Privileged mode, indicated by the \texttt{routername\#} prompt, allows a
user to both view and change the configuration of a router. You can
enter privileged mode by typing the command \texttt{enable} and entering
the \protect\hypertarget{c06.xhtmlux5cux23Page_257}{}{}enable password
or enable secret password, if set. Global configuration mode, indicated
by the \texttt{routername(config)\#} prompt, allows configuration
changes to be made that apply to the entire router (as opposed to a
configuration change that might affect only one interface, for example).

\textbf{Recognize additional prompts available in other modes and
describe their use.} Additional modes are reached via the global
configuration prompt, \texttt{routername(config)\#}, and their prompts
include interface, \texttt{router(config-if)\#}, for making interface
settings; line configuration mode, \texttt{router(config-line)\#}, used
to set passwords and make other settings to various connection methods;
and routing protocol modes for various routing protocols;
\texttt{router(config-router)\#}, used to enable and configure routing
protocols.

\textbf{Access and utilize editing and help features.} Make use of
typing a question mark at the end of commands for help in using the
commands. Additionally, understand how to filter command help with the
same question mark and letters. Use the command history to retrieve
commands previously utilized without retyping. Understand the meaning of
the caret when an incorrect command is rejected. Finally, identify
useful hot key combinations.

\textbf{Identify the information provided by the}\texttt{show\ version}
\textbf{command.} The \texttt{show\ version} command will provide basic
configuration for the system hardware as well as the software version,
the names and sources of configuration files, the configuration register
setting, and the boot images.

\textbf{Set the hostname of a router.} The command sequence to set the
hostname of a router is as follows:

\begin{verbatim}
enable
config t
hostname Todd
\end{verbatim}

\textbf{Differentiate the enable password and enable secret password.}
Both of these passwords are used to gain access into privileged mode.
However, the enable secret password is newer and is always encrypted by
default. Also, if you set the enable password and then set the enable
secret, only the enable secret will be used.

\textbf{Describe the configuration and use of banners.} Banners provide
information to users accessing the device and can be displayed at
various login prompts. They are configured with the \texttt{banner}
command and a keyword describing the specific type of banner.

\textbf{Set the enable secret on a router.} To set the enable secret,
you use the global config command \texttt{enable\ secret}. Do not use
\texttt{enable\ secret\ password} \texttt{password} or you will set your
password to \texttt{password}\texttt{} \texttt{password}. Here is an
example:

\begin{verbatim}
enable
config t
enable secret todd
\end{verbatim}

\protect\hypertarget{c06.xhtmlux5cux23Page_258}{}{}\textbf{Set the
console password on a router.} To set the console password, use the
following sequence:

\begin{verbatim}
enable
config t
line console 0
password todd
login
\end{verbatim}

\textbf{Set the Telnet password on a router.} To set the Telnet
password, the sequence is as follows:

\begin{verbatim}
enable
config t
line vty 0 4
password todd
login
\end{verbatim}

\textbf{Describe the advantages of using Secure Shell and list its
requirements.} Secure Shell (SSH) uses encrypted keys to send data so
that usernames and passwords are not sent in the clear. It requires that
a hostname and domain name be configured and that encryption keys be
generated.

\textbf{Describe the process of preparing an interface for use.} To use
an interface, you must configure it with an IP address and subnet mask
in the same subnet of the hosts that will be connecting to the switch
that is connected to that interface. It also must be enabled with the
\texttt{no\ shutdown} command. A serial interface that is connected back
to back with another router serial interface must also be configured
with a clock rate on the DCE end of the serial cable.

\textbf{Understand how to troubleshoot a serial link problem.} If you
type \texttt{show\ interface\ serial\ 0/0} and see
\texttt{down,\ line\ protocol\ is\ down}, this will be considered a
Physical layer problem. If you see it as
\texttt{up,\ line\ protocol\ is\ down}, then you have a Data Link layer
problem.

\textbf{Understand how to verify your router with the
\texttt{show\ interfaces\ command}} . If you type
\texttt{show\ interfaces}, you can view the statistics for the
interfaces on the router, verify whether the interfaces are shut down,
and see the IP address of each interface.

\textbf{Describe how to view, edit, delete, and save a configuration.}
The \texttt{show\ running-config} command is used to view the current
configuration being used by the router. The
\texttt{show\ startup-config} command displays the last configuration
that was saved and is the one that will be used at next startup. The
\texttt{copy\ running-config\ startup-config} command is used to save
changes made to the running configuration in NVRAM. The
\texttt{erase\ startup-config} command deletes the saved configuration
and will result in the invocation of the setup menu when the router is
rebooted because there will be no configuration present.

\subsection[Written Lab 6: IOS
Understanding]{\texorpdfstring{\protect\hypertarget{c06.xhtmlux5cux23c06-sec-23}{}{}\protect\hypertarget{c06.xhtmlux5cux23Page_259}{}{}Written
Lab 6: IOS Understanding}{Written Lab 6: IOS Understanding}}

In this section, you'll complete the following lab to make sure you've
got the information and concepts contained within them fully dialed in:

\begin{quote}
Lab 6.1: IOS Understanding

You can find the answers to this lab in Appendix A, ``Answers to Written
Labs.''

Write out the command or commands for the following questions:
\end{quote}

\begin{enumerate}
\tightlist
\item
  What command is used to set a serial interface to provide clocking to
  another router at 1000 Kb?
\item
  If you telnet into a switch and get the response
  \texttt{connection\ refused,\ password\ not\ set}, what commands would
  you execute on the destination device to stop receiving this message
  and not be prompted for a password?
\item
  If you type \texttt{show\ int\ fastethernet\ 0/1} and notice the port
  is administratively down, what commands would you execute to enable
  the interface?
\item
  If you wanted to delete the configuration stored in NVRAM, what
  command(s) would you type?
\item
  If you wanted to set the user-mode password to \emph{todd} for the
  console port, what command(s) would you type?
\item
  If you wanted to set the enable secret password to \emph{cisco}, what
  command(s) would you type?
\item
  If you wanted to determine if serial interface 0/2 on your router
  should provide clocking, what command would you use?
\item
  What command would you use to see the terminal history size?
\item
  You want to reinitialize the switch and totally replace the
  running-config with the current startup-config. What command will you
  use?
\item
  How would you set the name of a switch to \emph{Sales}?
\end{enumerate}

\subsection[Hands-on
Labs]{\texorpdfstring{\protect\hypertarget{c06.xhtmlux5cux23c06-sec-24}{}{}Hands-on
Labs}{Hands-on Labs}}

In this section, you will perform commands on a Cisco switch (or you can
use a router) that will help you understand what you learned in this
chapter.

You'll need at least one Cisco device---two would be better, three would
be outstanding. The hands-on labs in this section are included for use
with real Cisco routers, but all of these labs work with the LammleSim
IOS version (see \texttt{www.lammle.com/ccna}) or use the Cisco Packet
Tracer router simulator. Last, for the Cisco exam it doesn't matter what
model of switch or router you use with these labs, as long as you're
running IOS 12.2 or newer. Yes, I know the objectives are 15 code, but
that is not important for any of these labs.

\protect\hypertarget{c06.xhtmlux5cux23Page_260}{}{}It is assumed that
the device you're going to use has no current configuration present. If
necessary, erase any existing configuration with Hands-on Lab 6.1;
otherwise, proceed to Hands-on Lab 6.2:

\begin{enumerate}
\tightlist
\item
  Lab 6.1: Erasing an Existing Configuration
\item
  Lab 6.2: Exploring User, Privileged, and Configuration Modes
\item
  Lab 6.3: Using the Help and Editing Features
\item
  Lab 6.4: Saving a Configuration
\item
  Lab 6.5: Setting Passwords
\item
  Lab 6.6: Setting the Hostname, Descriptions, IP Address, and Clock
  Rate
\end{enumerate}

\subsubsection[Hands-on Lab 6.1: Erasing an Existing
Configuration]{\texorpdfstring{\protect\hypertarget{c06.xhtmlux5cux23c06-sec-25}{}{}Hands-on
Lab 6.1: Erasing an Existing
Configuration}{Hands-on Lab 6.1: Erasing an Existing Configuration}}

The following lab may require the knowledge of a username and password
to enter privileged mode. If the router has a configuration with an
unknown username and password for privileged mode, this procedure will
not be possible. It is possible to erase a configuration without a
privileged mode password, but the exact steps depend on the model and
will not be covered until Chapter 7.

\begin{enumerate}
\tightlist
\item
  Start the switch up and when prompted, press Enter.
\item
  At the \texttt{Switch\textgreater{}} prompt, type \texttt{enable}.
\item
  If prompted, enter the username and press Enter. Then enter the
  correct password and press Enter.
\item
  At the \texttt{privileged\ mode} prompt, type
  \texttt{erase\ startup-config}.
\item
  At the \texttt{privileged\ mode} prompt, type \texttt{reload}, and
  when prompted to save the configuration, type \texttt{n} for no.
\end{enumerate}

\subsubsection[Hands-on Lab 6.2: Exploring User, Privileged, and
Configuration
Modes]{\texorpdfstring{\protect\hypertarget{c06.xhtmlux5cux23c06-sec-26}{}{}Hands-on
Lab 6.2: Exploring User, Privileged, and Configuration
Modes}{Hands-on Lab 6.2: Exploring User, Privileged, and Configuration Modes}}

In the following lab, you'll explore user, privileged, and configuration
modes:

\begin{enumerate}
\tightlist
\item
  Plug the switch in, or turn the router on. If you just erased the
  configuration as in Hands-on Lab 6.1, when prompted to continue with
  the configuration dialog, enter \texttt{n} for no and press Enter.
  When prompted, press Enter to connect to your router. This will put
  you into user mode.
\item
  At the \texttt{Switch\textgreater{}} prompt, type a question mark
  (\texttt{?}).
\item
  Notice the \texttt{–more–} at the bottom of the screen.
\item
  Press the Enter key to view the commands line by line. Press the
  spacebar to view the commands a full screen at a time. You can type
  \texttt{q} at any time to quit.
\item
  Type \texttt{enable} or \texttt{en} and press Enter. This will put you
  into privileged mode where you can change and view the router
  configuration.
\item
  \protect\hypertarget{c06.xhtmlux5cux23Page_261}{}{}At the
  \texttt{Switch\#} prompt, type a question mark (\texttt{?}). Notice
  how many options are available to you in privileged mode.
\item
  Type \texttt{q} to quit.
\item
  Type \texttt{config} and press Enter.
\item
  When prompted for a method, press Enter to configure your router using
  your terminal (which is the default).
\item
  At the \texttt{Switch(config)\#} prompt, type a question mark
  (\texttt{?}), then \texttt{q} to quit, or press the spacebar to view
  the commands.
\item
  Type \texttt{interface\ f0/1} or \texttt{int\ f0/1} (or even
  \texttt{int\ gig0/1}) and press Enter. This will allow you to
  configure interface FastEthernet 0/1 or Gigabit 0/1.
\item
  At the \texttt{Switch(config-if)\#} prompt, type a question mark
  (\texttt{?}).
\item
  If using a router, type \texttt{int\ s0/0}, \texttt{interface\ s0/0}
  or even \texttt{interface\ s0/0/0} and press Enter. This will allow
  you to configure interface serial 0/0. Notice that you can go from
  interface to interface easily.
\item
  Type \texttt{encapsulation\ ?}.
\item
  Type \texttt{exit}. Notice how this brings you back one level.
\item
  Press Ctrl+Z. Notice how this brings you out of configuration mode and
  places you back into privileged mode.
\item
  Type \texttt{disable}. This will put you into user mode.
\item
  Type \texttt{exit}, which will log you out of the router or switch.
\end{enumerate}

\subsubsection[Hands-on Lab 6.3: Using the Help and Editing
Features]{\texorpdfstring{\protect\hypertarget{c06.xhtmlux5cux23c06-sec-27}{}{}Hands-on
Lab 6.3: Using the Help and Editing
Features}{Hands-on Lab 6.3: Using the Help and Editing Features}}

This lab will provide hands-on experience with Cisco's help and editing
features.

\begin{enumerate}
\tightlist
\item
  Log into your device and go to privileged mode by typing \texttt{en}
  or \texttt{enable}.
\item
  Type a question mark (\texttt{?}).
\item
  Type \texttt{cl?} and then press Enter. Notice that you can see all
  the commands that start with \emph{cl}.
\item
  Type \texttt{clock\ ?} and press Enter.

  \begin{center}\rule{0.5\linewidth}{0.5pt}\end{center}

  \includegraphics{images/note.png} Notice the difference between steps
  3 and 4. Step 3 has you type letters with no space and a question
  mark, which will give you all the commands that start with \emph{cl}.
  Step 4 has you type a command, space, and question mark. By doing
  this, you will see the next available parameter.

  \begin{center}\rule{0.5\linewidth}{0.5pt}\end{center}
\item
  Set the clock by typing \texttt{clock\ ?} and, following the help
  screens, setting the time and date. The following steps walk you
  through setting the date and time.
\item
  Type \texttt{clock\ ?}.
\item
  \protect\hypertarget{c06.xhtmlux5cux23Page_262}{}{}Type
  \texttt{clock\ set\ ?}.
\item
  Type \texttt{clock\ set\ 10:30:30\ ?}.
\item
  Type \texttt{clock\ set\ 10:30:30\ 14\ May\ ?}.
\item
  Type \texttt{clock\ set\ 10:30:30\ 14\ May\ 2011}.
\item
  Press Enter.
\item
  Type \texttt{show\ clock} to see the time and date.
\item
  From privileged mode, type \texttt{show\ access-list\ 10}. Don't press
  Enter.
\item
  Press Ctrl+A. This takes you to the beginning of the line.
\item
  Press Ctrl+E. This should take you back to the end of the line.
\item
  Ctrl+A takes your cursor back to the beginning of the line, and then
  Ctrl+F moves your cursor forward one character.
\item
  Press Ctrl+B, which will move you back one character.
\item
  Press Enter, then press Ctrl+P. This will repeat the last command.
\item
  Press the up arrow key on your keyboard. This will also repeat the
  last command.
\item
  Type \texttt{sh\ history}. This shows you the last 10 commands
  entered.
\item
  Type \texttt{terminal\ history\ size\ ?}. This changes the history
  entry size. The \texttt{?} is the number of allowed lines.
\item
  Type \texttt{show\ terminal} to gather terminal statistics and history
  size.
\item
  Type \texttt{terminal\ no\ editing}. This turns off advanced editing.
  Repeat steps 14 through 18 to see that the shortcut editing keys have
  no effect until you type \texttt{terminal\ editing}.
\item
  Type \texttt{terminal\ editing} and press Enter to re-enable advanced
  editing.
\item
  Type \texttt{sh\ run}, then press your Tab key. This will finish
  typing the command for you.
\item
  Type \texttt{sh\ start}, then press your Tab key. This will finish
  typing the command for you.
\end{enumerate}

\subsubsection[Hands-on Lab 6.4: Saving a
Configuration]{\texorpdfstring{\protect\hypertarget{c06.xhtmlux5cux23c06-sec-28}{}{}Hands-on
Lab 6.4: Saving a
Configuration}{Hands-on Lab 6.4: Saving a Configuration}}

In this lab, you will get hands-on experience saving a configuration:

\begin{enumerate}
\tightlist
\item
  Log into your device and go into privileged mode by typing \texttt{en}
  or \texttt{enable}, then press Enter.
\item
  To see the configuration stored in NVRAM, type \texttt{sh\ start} and
  press Tab and Enter, or type \texttt{show\ startup-config} and press
  Enter. However, if no configuration has been saved, you will get an
  error message.
\item
  To save a configuration to NVRAM, which is known as startup-config,
  you can do one of the following:

  \begin{enumerate}
  \tightlist
  \item
    Type \texttt{copy\ run\ start} and press Enter.
  \item
    Type \texttt{copy\ running}, press Tab, type \texttt{start}, press
    Tab, and press Enter.
  \item
    Type \texttt{copy\ running-config\ startup-config} and press Enter.
  \end{enumerate}
\item
  \protect\hypertarget{c06.xhtmlux5cux23Page_263}{}{}Type
  \texttt{sh\ start}, press Tab, then press Enter.
\item
  Type \texttt{sh\ run}, press Tab, then press Enter.
\item
  Type \texttt{erase\ startup-config}, press Tab, then press Enter.
\item
  Type \texttt{sh\ start}, press Tab, then press Enter. The router will
  either tell you that NVRAM is not present or display some other type
  of message, depending on the IOS and hardware.
\item
  Type \texttt{reload}, then press Enter. Acknowledge the reload by
  pressing Enter. Wait for the device to reload.
\item
  Say no to entering setup mode, or just press Ctrl+C.
\end{enumerate}

\subsubsection[Hands-on Lab 6.5: Setting
Passwords]{\texorpdfstring{\protect\hypertarget{c06.xhtmlux5cux23c06-sec-29}{}{}Hands-on
Lab 6.5: Setting Passwords}{Hands-on Lab 6.5: Setting Passwords}}

This hands-on lab will have you set your passwords.

\begin{enumerate}
\item
  Log into the router and go into privileged mode by typing \texttt{en}
  or \texttt{enable}.
\item
  Type \texttt{config\ t} and press Enter.
\item
  Type \texttt{enable\ ?}.
\item
  Set your enable secret password by typing
  \texttt{enable\ secret}\texttt{password} (the third word should be
  your own personalized password) and pressing Enter. Do not add the
  parameter \texttt{password} after the parameter \texttt{secret} (this
  would make your password the word \emph{password}). An example would
  be \texttt{enable\ secret\ todd}.
\item
  Now let's see what happens when you log all the way out of the router
  and then log in. Log out by pressing Ctrl+Z, and then type
  \texttt{exit} and press Enter. Go to privileged mode. Before you are
  allowed to enter privileged mode, you will be asked for a password. If
  you successfully enter the secret password, you can proceed.
\item
  Remove the secret password. Go to privileged mode, type
  \texttt{config\ t}, and press Enter. Type \texttt{no\ enable\ secret}
  and press Enter. Log out and then log back in again; now you should
  not be asked for a password.
\item
  One more password used to enter privileged mode is called the enable
  password. It is an older, less secure password and is not used if an
  enable secret password is set. Here is an example of how to set it:

\begin{verbatim}
config t
enable password todd1
\end{verbatim}
\item
  Notice that the enable secret and enable passwords are different. They
  should never be set the same. Actually, you should never use the
  enable password, only enable secret.
\item
  Type \texttt{config\ t} to be at the right level to set your console
  and auxiliary passwords, then type \texttt{line\ ?}.
\item
  Notice that the parameters for the line commands are
  \texttt{auxiliary}, \texttt{vty}, and \texttt{console}. You will set
  all three if you're on a router; if you're on a switch, only the
  console and VTY lines are available.
\item
  \protect\hypertarget{c06.xhtmlux5cux23Page_264}{}{}To set the Telnet
  or VTY password, type \texttt{line\ vty\ 0\ 4} and then press Enter.
  The \texttt{0\ 4} is the range of the five available virtual lines
  used to connect with Telnet. If you have an enterprise IOS, the number
  of lines may vary. Use the question mark to determine the last line
  number available on your router.
\item
  The next command is used to set the authentication on or off. Type
  \texttt{login} and press Enter to prompt for a user-mode password when
  telnetting into the device. You will not be able to telnet into a
  Cisco device if the password is not set.

  \begin{center}\rule{0.5\linewidth}{0.5pt}\end{center}

  \includegraphics{images/note.png} You can use the \texttt{no\ login}
  command to disable the user-mode password prompt when using Telnet. Do
  not do this in production!

  \begin{center}\rule{0.5\linewidth}{0.5pt}\end{center}
\item
  One more command you need to set for your VTY password is
  \texttt{password}. Type \texttt{password}\texttt{password} to set the
  password. (\texttt{password} is your password.)
\item
  Here is an example of how to set the VTY password:

\begin{verbatim}
config t
line vty 0 4
password todd
login
\end{verbatim}
\item
  Set your auxiliary password by first typing
  \texttt{line\ auxiliary\ 0} or \texttt{line\ aux\ 0} (if you are using
  a router).
\item
  Type \texttt{login}.
\item
  Type \texttt{password}\texttt{password}.
\item
  Set your console password by first typing \texttt{line\ console\ 0} or
  \texttt{line\ con\ 0}.
\item
  Type \texttt{login}.
\item
  Type \texttt{password}\texttt{password}. Here is an example of the
  last two command sequences:

\begin{verbatim}
config t
line con 0
password todd1
login
line aux 0
password todd
login
\end{verbatim}
\item
  You can add the \texttt{Exec-timeout\ 0\ 0} command to the
  \texttt{console\ 0} line. This will stop the console from timing out
  and logging you out. The command sequence will now look like this:

\begin{verbatim}
config t
line con 0
password todd2
login
exec-timeout 0 0
\end{verbatim}
\item
  Set the console prompt to not overwrite the command you're typing with
  console messages by using the command \texttt{logging\ synchronous}.

\begin{verbatim}
config t
line con 0
logging synchronous
\end{verbatim}
\end{enumerate}

\subsubsection[Hands-on Lab 6.6: Setting the Hostname, Descriptions, IP
Address, and Clock
Rate]{\texorpdfstring{\protect\hypertarget{c06.xhtmlux5cux23c06-sec-30}{}{}Hands-on
Lab 6.6: Setting the Hostname, Descriptions, IP Address, and Clock
Rate}{Hands-on Lab 6.6: Setting the Hostname, Descriptions, IP Address, and Clock Rate}}

This lab will have you set your administrative functions on each device.

\begin{enumerate}
\item
  Log into the switch or router and go into privileged mode by typing
  \texttt{en} or \texttt{enable}. If required, enter a username and
  password.
\item
  Set your hostname by using the \texttt{hostname} command. Notice that
  it is one word. Here is an example of setting your hostname on your
  router, but the switch uses the exact same command:

\begin{verbatim}
Router#config t
Router(config)#hostname RouterA
RouterA(config)#
\end{verbatim}

  Notice that the hostname of the router changed in the prompt as soon
  as you pressed Enter.
\item
  Set a banner that the network administrators will see by using the
  \texttt{banner} command, as shown in the following steps.
\item
  Type \texttt{config\ t}, then \texttt{banner\ ?}.
\item
  Notice that you can set at least four different banners. For this lab
  we are only interested in the login and message of the day (MOTD)
  banners.
\item
  Set your MOTD banner, which will be displayed when a console,
  auxiliary, or Telnet connection is made to the router, by typing this:

\begin{verbatim}
config t
banner motd #
This is an motd banner
#
\end{verbatim}
\item
  The preceding example used a \texttt{\#} sign as a delimiting
  character. This tells the router when the message is done. You cannot
  use the delimiting character in the message itself.
\item
  You can remove the MOTD banner by typing the following command:

\begin{verbatim}
config t
no banner motd
\end{verbatim}
\item
  \protect\hypertarget{c06.xhtmlux5cux23Page_266}{}{}Set the login
  banner by typing this:

\begin{verbatim}
config t
banner login #
This is a login banner
#
\end{verbatim}
\item
  The login banner will display immediately after the MOTD but before
  the user-mode password prompt. Remember that you set your user-mode
  passwords by setting the console, auxiliary, and VTY line passwords.
\item
  You can remove the login banner by typing this:

\begin{verbatim}
config t
no banner login
\end{verbatim}
\item
  You can add an IP address to an interface with the
  \texttt{ip\ address} command if you are using a router. You need to
  get into interface configuration mode first; here is an example of how
  you do that:

\begin{verbatim}
config t
int f0/1
ip address 1.1.1.1 255.255.0.0
no shutdown
\end{verbatim}

  Notice that the IP address (\texttt{1.1.1.1}) and subnet mask
  (\texttt{255.255.0.0}) are configured on one line. The
  \texttt{no\ shutdown} (or \texttt{no\ shut} for short) command is used
  to enable the interface. All interfaces are shut down by default on a
  router. If you are on a layer 2 switch, you can set an IP address only
  on the VLAN 1 interface.
\item
  You can add identification to an interface by using the
  \texttt{description} command. This is useful for adding information
  about the connection. Here is an example:

\begin{verbatim}
config t
int f0/1
ip address 2.2.2.1 255.255.0.0
no shut
description LAN link to Finance
\end{verbatim}
\item
  You can add the bandwidth of a serial link as well as the clock rate
  when simulating a DCE WAN link on a router. Here is an example:

\begin{verbatim}
config t
int s0/0
bandwidth 1000
clock rate 1000000
\end{verbatim}
\end{enumerate}

\subsection[Review
Questions]{\texorpdfstring{\protect\hypertarget{c06.xhtmlux5cux23c06-sec-31}{}{}\protect\hypertarget{c06.xhtmlux5cux23Page_267}{}{}Review
Questions}{Review Questions}}

\begin{center}\rule{0.5\linewidth}{0.5pt}\end{center}

\includegraphics{images/note.png} The following questions are designed
to test your understanding of this chapter's material. For more
information on how to get additional questions, please see
\texttt{www.lammle.com/ccna}.

\begin{center}\rule{0.5\linewidth}{0.5pt}\end{center}

You can find the answers to these questions in Appendix B, ``Answers to
Review Questions.''

\begin{enumerate}
\item
  You type \texttt{show\ interfaces\ fa0/1} and get this output:

\begin{verbatim}
275496 packets input, 35226811 bytes, 0 no buffer
   Received 69748 broadcasts (58822 multicasts)
   0 runts, 0 giants, 0 throttles
   111395 input errors, 511987 CRC, 0 frame, 0 overrun, 0 ignored
   0 watchdog, 58822 multicast, 0 pause input
   0 input packets with dribble condition detected
   2392529 packets output, 337933522 bytes, 0 underruns
   0 output errors, 0 collisions, 1 interface resets
   0 babbles, 0 late collision, 0 deferred
   0 lost carrier, 0 no carrier, 0 PAUSE output
   0 output buffer failures, 0 output buffers swapped out
\end{verbatim}

  What could the problem possibly be with this interface?

  \begin{enumerate}
  \tightlist
  \item
    Speed mismatch on directly connected interfaces
  \item
    Collisions causing CRC errors
  \item
    Frames received are too large
  \item
    Interference on the Ethernet cable
  \end{enumerate}
\item
  The output of the \texttt{show\ running-config} command comes from
  \_\_\_\_\_\_\_\_\_.

  \begin{enumerate}
  \tightlist
  \item
    NVRAM
  \item
    Flash
  \item
    RAM
  \item
    Firmware
  \end{enumerate}
\item
  Which two of the following commands are required when configuring SSH
  on your router? (Choose two.)

  \begin{enumerate}
  \tightlist
  \item
    \texttt{enable\ secret\ \ password}
  \item
    \texttt{exec-timeout\ 0\ 0}
  \item
    \protect\hypertarget{c06.xhtmlux5cux23Page_268}{}{}\texttt{ip\ domain-name\ \ name}
  \item
    \texttt{username\ \ name\ \ password\ \ password}
  \item
    \texttt{ip\ ssh\ version\ 2}
  \end{enumerate}
\item
  Which command will show you whether a DTE or a DCE cable is plugged
  into serial 0/0 on your router's WAN port?

  \begin{enumerate}
  \tightlist
  \item
    \texttt{sh\ int\ s0/0}
  \item
    \texttt{sh\ int\ serial0/0}
  \item
    \texttt{show\ controllers\ s0/0}
  \item
    \texttt{show\ serial0/0\ controllers}
  \end{enumerate}
\item
  In the work area, drag the router term to its definition on the right.

  \begin{longtable}[]{@{}ll@{}}
  \toprule
  Mode & Definition\tabularnewline
  \midrule
  \endhead
  user exec mode & Commands that affect the entire system\tabularnewline
  privileged exec mode & Commands that affect interfaces/processes
  only\tabularnewline
  Global configuration mode & Interactive configuration
  dialog\tabularnewline
  Specific configuration modes & Provides access to all other router
  commands\tabularnewline
  Setup mode & Limited to basic monitoring commands\tabularnewline
  \bottomrule
  \end{longtable}
\item
  Using the given output, what type of interface is shown?

\begin{verbatim}
[output cut]
Hardware is MV96340 Ethernet, address is 001a.2f55.c9e8 (bia 001a.2f55.c9e8)
Internet address is 192.168.1.33/27
MTU 1500 bytes, BW 100000 Kbit, DLY 100 usec,
   reliability 255/255, txload 1/255, rxload 1/255
\end{verbatim}

  \begin{enumerate}
  \tightlist
  \item
    10 Mb
  \item
    100 Mb
  \item
    1000 Mb
  \item
    1000 MB
  \end{enumerate}
\item
  Which of the following commands will configure all the default VTY
  ports on a switch?

  \begin{enumerate}
  \tightlist
  \item
    \texttt{Switch\#\ line\ vty\ 0\ 4}
  \item
    \texttt{Switch(config)\#\ line\ vty\ 0\ 4}
  \item
    \texttt{Switch(config-if)\#\ line\ console\ 0}
  \item
    \texttt{Switch(config)\#\ line\ vty\ all}
  \end{enumerate}
\item
  \protect\hypertarget{c06.xhtmlux5cux23Page_269}{}{}Which of the
  following commands sets the privileged mode password to Cisco and
  encrypts the password?

  \begin{enumerate}
  \tightlist
  \item
    \texttt{enable\ secret\ password\ Cisco}
  \item
    \texttt{enable\ secret\ cisco}
  \item
    \texttt{enable\ secret\ Cisco}
  \item
    \texttt{enable\ password\ Cisco}
  \end{enumerate}
\item
  If you wanted administrators to see a message when logging into the
  switch, which command would you use?

  \begin{enumerate}
  \tightlist
  \item
    \texttt{message\ banner\ motd}
  \item
    \texttt{banner\ message\ motd}
  \item
    \texttt{banner\ motd}
  \item
    \texttt{message\ motd}
  \end{enumerate}
\item
  Which of the following prompts indicates that the switch is currently
  in privileged mode?

  \begin{enumerate}
  \tightlist
  \item
    \texttt{Switch(config)\#}
  \item
    \texttt{Switch\textgreater{}}
  \item
    \texttt{Switch\#}
  \item
    \texttt{Switch(config-if)}
  \end{enumerate}
\item
  What command do you type to save the configuration stored in RAM to
  NVRAM?

  \begin{enumerate}
  \tightlist
  \item
    \texttt{Switch(config)\#\ copy\ current\ to\ starting}
  \item
    \texttt{Switch\#\ copy\ starting\ to\ running}
  \item
    \texttt{Switch(config)\#\ copy\ running-config\ startup-config}
  \item
    \texttt{Switch\#\ copy\ run\ start}
  \end{enumerate}
\item
  You try to telnet into SF from router Corp and receive this message:

\begin{verbatim}
Corp#telnet SF
Trying SF (10.0.0.1)...Open
\end{verbatim}

\begin{verbatim}
Password required, but none set
[Connection to SF closed by foreign host]
Corp#
\end{verbatim}

  Which of the following sequences will address this problem correctly?

  \begin{enumerate}
  \item
\begin{verbatim}
Corp(config)#line console 0
Corp(config-line)#password password
Corp(config-line)#login
\end{verbatim}
  \item
\begin{verbatim}
SF config)#line console 0
SF(config-line)#enable secret password
SF(config-line)#login
\end{verbatim}
  \item
\begin{verbatim}
Corp(config)#line vty 0 4
Corp(config-line)#password password
Corp(config-line)#login
\end{verbatim}
  \item
\begin{verbatim}
SF(config)#line vty 0 4
SF(config-line)#password password
SF(config-line)#login
\end{verbatim}
  \end{enumerate}
\item
  Which command will delete the contents of NVRAM on a switch?

  \begin{enumerate}
  \tightlist
  \item
    \texttt{delete\ NVRAM}
  \item
    \texttt{delete\ startup-config}
  \item
    \texttt{erase\ flash}
  \item
    \texttt{erase\ startup-config}
  \item
    \texttt{erase\ start}
  \end{enumerate}
\item
  What is the problem with an interface if you type
  \texttt{show\ interface\ g0/1} and receive the following message?

\begin{verbatim}
Gigabit 0/1 is administratively down, line protocol is down
\end{verbatim}

  \begin{enumerate}
  \tightlist
  \item
    The keepalives are different times.
  \item
    The administrator has the interface shut down.
  \item
    The administrator is pinging from the interface.
  \item
    No cable is attached.
  \end{enumerate}
\item
  Which of the following commands displays the configurable parameters
  and statistics of all interfaces on a switch?

  \begin{enumerate}
  \tightlist
  \item
    \texttt{show\ running-config}
  \item
    \texttt{show\ startup-config}
  \item
    \texttt{show\ interfaces}
  \item
    \texttt{show\ versions}
  \end{enumerate}
\item
  If you delete the contents of NVRAM and reboot the switch, what mode
  will you be in?

  \begin{enumerate}
  \tightlist
  \item
    Privileged mode
  \item
    Global mode
  \item
    Setup mode
  \item
    NVRAM loaded mode
  \end{enumerate}
\item
  You type the following command into the switch and receive the
  following output:

\begin{verbatim}
Switch#show fastethernet 0/1
            ^
% Invalid input detected at '^' marker.
\end{verbatim}

  \protect\hypertarget{c06.xhtmlux5cux23Page_271}{}{}Why was this error
  message displayed?

  \begin{enumerate}
  \tightlist
  \item
    You need to be in privileged mode.
  \item
    You cannot have a space between \texttt{fastethernet} and
    \texttt{0/1}.
  \item
    The switch does not have a FastEthernet 0/1 interface.
  \item
    Part of the command is missing.
  \end{enumerate}
\item
  You type \texttt{Switch\#sh\ r} and receive a
  \texttt{\%\ ambiguous\ command} error. Why did you receive this
  message?

  \begin{enumerate}
  \tightlist
  \item
    The command requires additional options or parameters.
  \item
    There is more than one \texttt{show} command that starts with the
    letter \emph{r}.
  \item
    There is no \texttt{show} command that starts with \emph{r.}
  \item
    The command is being executed from the wrong mode.
  \end{enumerate}
\item
  Which of the following commands will display the current IP addressing
  and the layer 1 and 2 status of an interface? (Choose two.)

  \begin{enumerate}
  \tightlist
  \item
    \texttt{show\ version}
  \item
    \texttt{show\ interfaces}
  \item
    \texttt{show\ controllers}
  \item
    \texttt{show\ ip\ interface}
  \item
    \texttt{show\ running-config}
  \end{enumerate}
\item
  At which layer of the OSI model would you assume the problem is if you
  type \texttt{show\ interface\ serial\ 1} and receive the following
  message?

\begin{verbatim}
Serial1 is down, line protocol is down
\end{verbatim}

  \begin{enumerate}
  \tightlist
  \item
    Physical layer
  \item
    Data Link layer
  \item
    Network layer
  \item
    None; it is a router problem.
  \end{enumerate}
\end{enumerate}

\protect\hypertarget{c07.xhtml}{}{}

\section[{Chapter 7}\\
{Managing a Cisco
Internetwork}]{\texorpdfstring{\protect\hypertarget{c07.xhtmlux5cux23c07}{}{}\protect\hypertarget{c07.xhtmlux5cux23Page_273}{}{}{Chapter
7}\\
{Managing a Cisco
Internetwork}}{Chapter 7 Managing a Cisco Internetwork}}

\begin{center}\rule{0.5\linewidth}{0.5pt}\end{center}

\subsection{The following ICND1 exam topics are covered in this
chapter:}

\begin{enumerate}
\tightlist
\item
  \includegraphics{images/tick.png} \textbf{2.0 LAN Switching
  Technologies}

  \begin{enumerate}
  \tightlist
  \item
    \includegraphics{images/squ.png} 2.6 Configure and verify Layer 2
    protocols

    \begin{enumerate}
    \tightlist
    \item
      \includegraphics{images/squ.png} 2.6.a Cisco Discovery Protocol
    \item
      \includegraphics{images/squ.png} 2.6.bLLDP
    \end{enumerate}
  \end{enumerate}
\item
  \includegraphics{images/tick.png} \textbf{4.0 Infrastructure Services}

  \begin{enumerate}
  \tightlist
  \item
    \includegraphics{images/squ.png} 4.1 Describe DNS lookup operation
  \item
    \includegraphics{images/squ.png} 4.2 Troubleshoot client
    connectivity issues involving DNS
  \item
    \includegraphics{images/squ.png} 4.3 Configure and verify DHCP on a
    router (excluding static reservations)

    \begin{enumerate}
    \tightlist
    \item
      \includegraphics{images/squ.png} 4.3.a Server
    \item
      \includegraphics{images/squ.png} 4.3.b Relay
    \item
      \includegraphics{images/squ.png} 4.3.c Client
    \item
      \includegraphics{images/squ.png} 4.3.d TFTP, DNS, and gateway
      options
    \end{enumerate}
  \item
    \includegraphics{images/squ.png} 4.4 Troubleshoot client- and
    router-based DHCP connectivity issues
  \item
    \includegraphics{images/squ.png} 4.5 Configure and verify NTP
    operating in client/server mode
  \end{enumerate}
\item
  \includegraphics{images/tick.png} \textbf{5.0 Infrastructure
  Management}

  \begin{enumerate}
  \tightlist
  \item
    \includegraphics{images/squ.png} 5.1 Configure and verify
    device-monitoring using syslog
  \item
    \includegraphics{images/squ.png} 5.2 Configure and verify device
    management

    \begin{enumerate}
    \tightlist
    \item
      \includegraphics{images/squ.png} 5.2.a Backup and restore device
      configuration
    \item
      \includegraphics{images/squ.png} 5.2.b Using Cisco Discovery
      Protocol and LLDP for device discovery
    \item
      \includegraphics{images/squ.png} 5.2.d Logging
    \item
      \includegraphics{images/squ.png} 5.2.e Timezone
    \item
      \includegraphics{images/squ.png} 5.2.f Loopback
    \end{enumerate}
  \end{enumerate}
\end{enumerate}

\protect\hypertarget{c07.xhtmlux5cux23Page_274}{}{}\includegraphics{images/intro.png}Here
in Chapter 7, I'm going to show you how to manage Cisco routers and
switches on an internetwork. You'll be learning about the main
components of a router, as well as the router boot sequence. You'll also
find out how to manage Cisco devices by using the \texttt{copy} command
with a TFTP host and how to configure DHCP and NTP, plus you'll get a
survey of the Cisco Discovery Protocol (CDP). I'll also show you how to
resolve hostnames.

I'll wrap up the chapter by guiding you through some important Cisco IOS
troubleshooting techniques to ensure that you're well equipped with
these key skills.

\begin{center}\rule{0.5\linewidth}{0.5pt}\end{center}

\includegraphics{images/note.png} To find up-to-the minute updates for
this chapter, please see \texttt{www.lammle.com/ccna} or the book's web
page at \texttt{www.sybex.com/go/ccna}.

\begin{center}\rule{0.5\linewidth}{0.5pt}\end{center}

\subsection[The Internal Components of a Cisco Router and
Switch]{\texorpdfstring{\protect\hypertarget{c07.xhtmlux5cux23c07-sec-1}{}{}The
Internal Components of a Cisco Router and
Switch}{The Internal Components of a Cisco Router and Switch}}

Unless you happen to be really savvy about the inner and outer workings
of all your car's systems and its machinery and how all of that
technology works together, you'll take it to someone who \emph{does}
know how to keep it maintained, figure out what's wrong when it stops
running, and get it up and running again. It's the same deal with Cisco
networking devices---you need to know all about their major components,
pieces, and parts as well as what they all do and why and how they all
work together to make a network work. The more solid your knowledge, the
more expert you are about these things and the better equipped you'll be
to configure and troubleshoot a Cisco internetwork. Toward that goal,
study \protect\hyperlink{c07.xhtmlux5cux23table7-1}{Table 7.1} for an
introductory description of a Cisco router's major components.

{\protect\hyperlink{c07.xhtmlux5cux23tableanchor7-1}{\textbf{TABLE 7.1}}
Cisco router components}

\begin{longtable}[]{@{}ll@{}}
\toprule
Component & Description\tabularnewline
\midrule
\endhead
Bootstrap & Stored in the microcode of the ROM, the bootstrap is used to
bring a router up during initialization. It boots the router up and then
loads the IOS.\tabularnewline
POST (power-on self-test) & Also stored in the microcode of the ROM, the
POST is used to check the basic functionality of the router hardware and
determines which interfaces are present.\tabularnewline
\protect\hypertarget{c07.xhtmlux5cux23Page_275}{}{}ROM monitor & Again,
stored in the microcode of the ROM, the ROM monitor is used for
manufacturing, testing, and troubleshooting, as well as running a
mini-IOS when the IOS in flash fails to load.\tabularnewline
Mini-IOS & Called the RXBOOT or bootloader by Cisco, the mini-IOS is a
small IOS in ROM that can be used to bring up an interface and load a
Cisco IOS into flash memory. The mini-IOS can also perform a few other
maintenance operations.\tabularnewline
RAM (random access memory) & Used to hold packet buffers, ARP cache,
routing tables, and also the software and data structures that allow the
router to function. Running-config is stored in RAM, and most routers
expand the IOS from flash into RAM upon boot.\tabularnewline
\emph{ROM (read-only memory)} & Used to start and maintain the router.
Holds the POST and the bootstrap program as well as the
mini-IOS.\tabularnewline
\emph{Flash memory} & Stores the Cisco IOS by default. Flash memory is
not erased when the router is reloaded. It is EEPROM (electronically
erasable programmable read-only memory) created by Intel.\tabularnewline
NVRAM (nonvolatile RAM) & Used to hold the router and switch
configuration. NVRAM is not erased when the router or switch is
reloaded. Does not store an IOS. The configuration register is stored in
NVRAM.\tabularnewline
\emph{Configuration register} & Used to control how the router boots up.
This value can be found as the last line of the \texttt{show\ version}
command output and by default is set to 0x2102, which tells the router
to load the IOS from flash memory as well as to load the configuration
from NVRAM.\tabularnewline
\bottomrule
\end{longtable}

\subsubsection[The Router and Switch Boot
Sequence]{\texorpdfstring{\protect\hypertarget{c07.xhtmlux5cux23c07-sec-2}{}{}The
Router and Switch Boot Sequence}{The Router and Switch Boot Sequence}}

When a Cisco device boots up, it performs a series of steps, called the
\emph{boot sequence}, to test the hardware and load the necessary
software. The boot sequence comprises the following steps, as shown in
\protect\hyperlink{c07.xhtmlux5cux23figure7-1}{Figure 7.1}:

\begin{enumerate}
\item
  The IOS device performs a POST, which tests the hardware to verify
  that all components of the device are present and operational. The
  post takes stock of the different interfaces on the switch or router,
  and it's stored in and runs from read-only memory (ROM).
\item
  The bootstrap in ROM then locates and loads the Cisco IOS software by
  executing programs responsible for finding where each IOS program is
  located. Once they are found, it then loads the proper files. By
  default, the IOS software is loaded from flash memory in all Cisco
  devices.

  \protect\hypertarget{c07.xhtmlux5cux23Page_276}{}{}

  \begin{figure}
  \centering
  \includegraphics{images/c07f001.jpg}
  \caption{{\protect\hyperlink{c07.xhtmlux5cux23figureanchor7-1}{\textbf{FIGURE
  7.1}} Router bootup process}}
  \end{figure}
\item
  The IOS software then looks for a valid configuration file stored in
  NVRAM. This file is called startup-config and will be present only if
  an administrator has copied the running-config file into NVRAM.
\item
  If a startup-config file is found in NVRAM, the router or switch will
  copy it, place it in RAM, and name the file the running-config. The
  device will use this file to run, and the router/switch should now be
  operational. If a startup-config file is not in NVRAM, the router will
  broadcast out any interface that detects carrier detect (CD) for a
  TFTP host looking for a configuration, and when that fails (typically
  it will fail---most people won't even realize the router has attempted
  this process), it will start the setup mode configuration process.
\end{enumerate}

\begin{center}\rule{0.5\linewidth}{0.5pt}\end{center}

\includegraphics{images/tip.png} The default order of an IOS loading
from a Cisco device begins with flash, then TFTP server, and finally,
ROM.

\begin{center}\rule{0.5\linewidth}{0.5pt}\end{center}

\subsection[Backing Up and Restoring the Cisco
Configuration]{\texorpdfstring{\protect\hypertarget{c07.xhtmlux5cux23c07-sec-3}{}{}Backing
Up and Restoring the Cisco
Configuration}{Backing Up and Restoring the Cisco Configuration}}

Any changes that you make to the configuration are stored in the
running-config file. And if you don't enter a \texttt{copy\ run\ start}
command after you make a change to running-config, that change will
totally disappear if the device reboots or gets powered down. As always,
\protect\hypertarget{c07.xhtmlux5cux23Page_277}{}{}backups are good, so
you'll want to make another backup of the configuration information just
in case the router or switch completely dies on you. Even if your
machine is healthy and happy, it's good to have a backup for reference
and documentation reasons!

Next, I'll cover how to copy the configuration of a router to a TFTP
server as well as how to restore that configuration.

\subsubsection[Backing Up the Cisco
Configuration]{\texorpdfstring{\protect\hypertarget{c07.xhtmlux5cux23c07-sec-4}{}{}Backing
Up the Cisco Configuration}{Backing Up the Cisco Configuration}}

To copy the configuration from an IOS device to a TFTP server, you can
use either the \texttt{copy\ running-config\ tftp} or the
\texttt{copy\ startup-config\ tftp} command. Either one will back up the
router configuration that's currently running in DRAM or one that's
stored in NVRAM.

\paragraph{Verifying the Current Configuration}

To verify the configuration in DRAM, use the
\texttt{show\ running-config} command (\texttt{sh\ run} for short) like
this:

\begin{verbatim}
Router#show running-config
Building configuration...
 
Current configuration : 855 bytes
!
version 15.0
\end{verbatim}

The current configuration information indicates that the router is
running version 15.0 of the IOS.

\paragraph{Verifying the Stored Configuration}

Next, you should check the configuration stored in NVRAM. To see this,
use the \texttt{show\ startup-config} command (\texttt{sh\ start} for
short) like this:

\begin{verbatim}
Router#sh start
Using 855 out of 524288 bytes
!
! Last configuration change at 04:49:14 UTC Fri Mar 5 1993
!
version 15.0
\end{verbatim}

The first line shows you how much room your backup configuration is
taking up. Here, we can see that NVRAM is about 524 KB and that only 855
bytes of it are being used. But memory is easier to reveal via the
\texttt{show\ version} command when you're using an ISR router.

If you're not sure that the files are the same and the running-config
file is what you want to go with, then use the
\texttt{copy\ running-config\ startup-config} command. This will help
you ensure that both files are in fact the same. I'll guide you through
this in the next section.

\paragraph[Copying the Current Configuration to
NVRAM]{\texorpdfstring{\protect\hypertarget{c07.xhtmlux5cux23Page_278}{}{}Copying
the Current Configuration to
NVRAM}{Copying the Current Configuration to NVRAM}}

By copying running-config to NVRAM as a backup, as shown in the
following output, you ensure that your running-config will always be
reloaded if the router gets rebooted. Starting in the 12.0 IOS, you'll
be prompted for the filename you want to use:

\begin{verbatim}
Router#copy running-config startup-config
Destination filename [startup-config]?[enter]
Building configuration...
[OK]
\end{verbatim}

The reason the filename prompt appears is that there are now so many
options you can use when using the \texttt{copy} command---check it out:

\begin{verbatim}
Router#copy running-config ?
  flash:          Copy to flash: file system
  ftp:            Copy to ftp: file system
  http:           Copy to http: file system
  https:          Copy to https: file system
  null:           Copy to null: file system
  nvram:          Copy to nvram: file system
  rcp:            Copy to rcp: file system
  running-config  Update (merge with) current system configuration
  scp:            Copy to scp: file system
  startup-config  Copy to startup configuration
  syslog:         Copy to syslog: file system
  system:         Copy to system: file system
  tftp:           Copy to tftp: file system
  tmpsys:         Copy to tmpsys: file system
\end{verbatim}

\paragraph{Copying the Configuration to a TFTP Server}

Once the file is copied to NVRAM, you can make a second backup to a TFTP
server by using the \texttt{copy\ running-config\ tftp} command, or
\texttt{copy\ run\ tftp} for short. I'm going to set the hostname to
\texttt{Todd} before I run this command:

\begin{verbatim}
Todd#copy running-config tftp
Address or name of remote host []? 10.10.10.254
Destination filename [todd-confg]?
!!
776 bytes copied in 0.800 secs (970 bytes/sec)
\end{verbatim}

If you have a hostname already configured, the command will
automatically use the hostname plus the extension \texttt{-confg} as the
name of the file.

\subsubsection[Restoring the Cisco
Configuration]{\texorpdfstring{\protect\hypertarget{c07.xhtmlux5cux23c07-sec-5}{}{}\protect\hypertarget{c07.xhtmlux5cux23Page_279}{}{}Restoring
the Cisco Configuration}{Restoring the Cisco Configuration}}

What do you do if you've changed your running-config file and want to
restore the configuration to the version in the startup-config file? The
easiest way to get this done is to use the
\texttt{copy\ startup-config\ running-config} command, or
\texttt{copy\ start\ run} for short, but this will work only if you
copied running-config into NVRAM before you made any changes! Of course,
a reload of the device will work too!

If you did copy the configuration to a TFTP server as a second backup,
you can restore the configuration using the
\texttt{copy\ tftp\ running-config} command (\texttt{copy\ tftp\ run}
for short), or the \texttt{copy\ tftp\ startup-config} command
(\texttt{copy\ tftp\ start} for short), as shown in the following
output. Just so you know, the old command we used to use for this is
\texttt{config\ net}:

\begin{verbatim}
Todd#copy tftp running-config
Address or name of remote host []?10.10.10.254
Source filename []?todd-confg
Destination filename[running-config]?[enter]
Accessing tftp://10.10.10.254/todd-confg...
Loading todd-confg from 10.10.10.254 (via FastEthernet0/0):
!!
[OK - 776 bytes]
776 bytes copied in 9.212 secs (84 bytes/sec)
Todd#
*Mar  7 17:53:34.071: %SYS-5-CONFIG_I: Configured from
    tftp://10.10.10.254/todd-confg by console
\end{verbatim}

Okay that the configuration file is an ASCII text file . . . meaning
that before you copy the configuration stored on a TFTP server back to a
router, you can make changes to the file with any text editor.

\begin{center}\rule{0.5\linewidth}{0.5pt}\end{center}

\includegraphics{images/note.png} Remember that when you copy or merge a
configuration from a TFTP server to a freshly erased and rebooted
router's RAM, the interfaces are shut down by default and you must
manually enable each interface with the \texttt{no\ shutdown} command.

\begin{center}\rule{0.5\linewidth}{0.5pt}\end{center}

\subsubsection[Erasing the
Configuration]{\texorpdfstring{\protect\hypertarget{c07.xhtmlux5cux23c07-sec-6}{}{}Erasing
the Configuration}{Erasing the Configuration}}

To delete the startup-config file on a Cisco router or switch, use the
command \texttt{erase\ startup-config}, like this:

\begin{verbatim}
Todd#erase startup-config
Erasing the nvram filesystem will remove all configuration files!
    Continue? [confirm][enter]
[OK]
Erase of nvram: complete
*Mar  7 17:56:20.407: %SYS-7-NV_BLOCK_INIT: Initialized the geometry of nvram
Todd#reload
System configuration has been modified. Save? [yes/no]:n
Proceed with reload? [confirm][enter]
 *Mar  7 17:56:31.059: %SYS-5-RELOAD: Reload requested by console.
    Reload Reason: Reload Command.
\end{verbatim}

This command deletes the contents of NVRAM on the switch and router. If
you type \texttt{reload} while in privileged mode and say no to saving
changes, the switch or router will reload and come up into setup mode.

\subsection[Configuring
DHCP]{\texorpdfstring{\protect\hypertarget{c07.xhtmlux5cux23c07-sec-7}{}{}Configuring
DHCP}{Configuring DHCP}}

We went over DHCP in Chapter 3, ``Introduction to TCP/IP,'' where I
described how it works and what happens when there's a conflict. At this
point, you're ready to learn how to configure DHCP on Cisco's IOS as
well as how to configure a DHCP forwarder for when your hosts don't live
on the same LAN as the DHCP server. Do you remember the four-step
process hosts used to get an address from a server? If not, now would be
a really great time to head back to Chapter 3 and thoroughly review that
before moving on with this!

To configure a DHCP server for your hosts, you need the following
information at minimum:

\begin{quote}
\textbf{Network and mask for each LAN} Network ID, also called a scope.
All addresses in a subnet can be leased to hosts by default.

\textbf{Reserved/excluded addresses} Reserved addresses for printers,
servers, routers, etc. These addresses will not be handed out to hosts.
I usually reserve the first address of each subnet for the router, but
you don't have to do this.

\textbf{Default router} This is the router's address for each LAN.

\textbf{DNS address} A list of DNS server addresses provided to hosts so
they can resolve names.
\end{quote}

Here are your configuration steps:

\begin{enumerate}
\tightlist
\item
  Exclude the addresses you want to reserve. The reason you do this step
  first is because as soon as you set a network ID, the DHCP service
  will start responding to client requests.
\item
  Create your pool for each LAN using a unique name.
\item
  Choose the network ID and subnet mask for the DHCP pool that the
  server will use to provide addresses to hosts.
\item
  Add the address used for the default gateway of the subnet.
\item
  \protect\hypertarget{c07.xhtmlux5cux23Page_281}{}{}Provide the DNS
  server address(es).
\item
  If you don't want to use the default lease time of 24 hours, you need
  to set the lease time in days, hours, and minutes.
\end{enumerate}

I'll configure the switch in
\protect\hyperlink{c07.xhtmlux5cux23figure7-2}{Figure 7.2} to be the
DHCP server for the Sales wireless LAN.

\begin{figure}
\centering
\includegraphics{images/c07f002.jpg}
\caption{{\protect\hyperlink{c07.xhtmlux5cux23figureanchor7-2}{\textbf{FIGURE
7.2}} DHCP configuration example on a switch}}
\end{figure}

Understand that this configuration could just have easily been placed on
the router in \protect\hyperlink{c07.xhtmlux5cux23figure7-2}{Figure
7.2}. Here's how we'll configure DHCP using the 192.168.10.0/24 network
ID:

\begin{verbatim}
Switch(config)#ip dhcp excluded-address 192.168.10.1 192.168.10.10
Switch(config)#ip dhcp pool Sales_Wireless
Switch(dhcp-config)#network 192.168.10.0 255.255.255.0
Switch(dhcp-config)#default-router 192.168.10.1
Switch(dhcp-config)#dns-server 4.4.4.4
Switch(dhcp-config)#lease 3 12 15
Switch(dhcp-config)#option 66 ascii tftp.lammle.com
\end{verbatim}

First, you can see that I reserved 10 addresses in the range for the
router, servers, and printers, etc. I then created the pool named
Sales\_Wireless, added the default gateway and DNS server, and set the
lease to 3 days, 12 hours, and 15 minutes (which isn't really
significant because I just set it that way for demonstration purposes).
Lastly, I provided an example on you how you would set option 66, which
is sending a TFTP server address to a DHCP client. Typically used for
VoIP phones, or auto installs, and needs to be listed as a FQDN. Pretty
straightforward, right? The switch will now respond to DHCP client
requests. But what happens if we need to provide an IP address from a
DHCP server to a host that's not in our broadcast domain, or if we want
to receive a DHCP address for a client from a remote server?

\subsubsection[DHCP
Relay]{\texorpdfstring{\protect\hypertarget{c07.xhtmlux5cux23c07-sec-8}{}{}DHCP
Relay}{DHCP Relay}}

If you need to provide addresses from a DHCP server to hosts that aren't
on the same LAN as the DHCP server, you can configure your router
interface to relay or forward the DHCP client requests, as shown in
\protect\hyperlink{c07.xhtmlux5cux23figure7-3}{Figure 7.3}. If we don't
provide this service, our router would receive the DHCP client
broadcast, promptly discard it, and the remote host would never
\protect\hypertarget{c07.xhtmlux5cux23Page_282}{}{}receive an
address---unless we added a DHCP server on every broadcast domain! Let's
take a look at how we would typically configure DHCP service in today's
networks.

\begin{figure}
\centering
\includegraphics{images/c07f003.jpg}
\caption{{\protect\hyperlink{c07.xhtmlux5cux23figureanchor7-3}{\textbf{FIGURE
7.3}} Configuring a DHCP relay}}
\end{figure}

So we know that because the hosts off the router don't have access to a
DHCP server, the router will simply drop their client request broadcast
messages by default. To solve this problem, we can configure the Fa0/0
interface of the router to accept the DHCP client requests and forward
them to the DHCP server like this:

\begin{verbatim}
Router#config t
Router(config)#interface fa0/0
Router(config-if)#ip helper-address 10.10.10.254
\end{verbatim}

Now I know that was a pretty simple example, and there are definitely
other ways to configure the relay, but rest assured that I've covered
the objectives for you. Also, I want you to know that
\texttt{ip\ helper-address} forwards more than just DHCP client
requests, so be sure to research this command before you implement it!
Now that I've demonstrated how to create the DHCP service, let's take a
minute to verify DHCP before moving on to NTP.

\subsubsection[Verifying DHCP on Cisco
IOS]{\texorpdfstring{\protect\hypertarget{c07.xhtmlux5cux23c07-sec-9}{}{}Verifying
DHCP on Cisco IOS}{Verifying DHCP on Cisco IOS}}

There are some really useful verification commands to use on a Cisco IOS
device for monitoring and verifying a DHCP service. You'll get to see
the output for these commands when I build the network in Chapter 9,
``IP Routing,'' and add DHCP to the two remote LANs. I just want you to
begin getting familiar with them, so here's a list of four very
important ones and what they do:

\begin{enumerate}
\tightlist
\item
  \texttt{show\ ip\ dhcp\ binding} Lists state information about each IP
  address currently leased to a client.
\item
  \protect\hypertarget{c07.xhtmlux5cux23Page_283}{}{}\texttt{show\ ip\ dhcp\ pool\ {[}poolname{]}}
  Lists the configured range of IP addresses, plus statistics for the
  number of currently leased addresses and the high watermark for leases
  from each pool.
\item
  \texttt{show\ ip\ dhcp\ server\ statistics} Lists DHCP server
  statistics---a lot of them!
\item
  \texttt{show\ ip\ dhcp\ conflict} If someone statically configures an
  IP address on a LAN and the DHCP server hands out that same address,
  you'll end up with a duplicate address. This isn't good, which is why
  this command is so helpful!
\end{enumerate}

Again, no worries because we'll cover these vital commands thoroughly in
Chapter 9.

\subsection[Syslog]{\texorpdfstring{\protect\hypertarget{c07.xhtmlux5cux23c07-sec-10}{}{}Syslog}{Syslog}}

Reading system messages from a switch's or router's internal buffer is
the most popular and efficient method of seeing what's going on with
your network at a particular time. But the best way is to log messages
to a \emph{syslog} server, which stores messages from you and can even
time-stamp and sequence them for you, and it's easy to set up and
configure!

Syslog allows you to display, sort, and even search messages, all of
which makes it a really great troubleshooting tool. The search feature
is especially powerful because you can use keywords and even severity
levels. Plus, the server can email admins based on the severity level of
the message.

Network devices can be configured to generate a syslog message and
forward it to various destinations. These four examples are popular ways
to gather messages from Cisco devices:

\begin{enumerate}
\tightlist
\item
  Logging buffer (on by default)
\item
  Console line (on by default)
\item
  Terminal lines (using the \texttt{terminal\ monitor} command)
\item
  Syslog server
\end{enumerate}

As you already know, all system messages and debug output generated by
the IOS go out only the console port by default and are also logged in
buffers in RAM. And you also know that Cisco routers aren't exactly shy
about sending messages! To send a message to the VTY lines, use the
\texttt{terminal\ monitor} command. We'll also add a small configuration
needed for syslog, which I'll show you soon in the configuration
section.

So by default, we'd see something like this on our console line:

\begin{verbatim}
*Oct 21 17:33:50.565:%LINK-5-CHANGED:Interface FastEthernet0/0, changed
state to administratively down
*Oct 21 17:33:51.565:%LINEPROTO-5-UPDOWN:Line protocol on Interface FastEthernet0/0, changed state to down
\end{verbatim}

And the Cisco router would send a general version of the message to the
syslog server that would be formatted into something like this:

\begin{verbatim}
Seq no:timestamp: %facility-severity-MNEMONIC:description
\end{verbatim}

\protect\hypertarget{c07.xhtmlux5cux23Page_284}{}{}The system message
format can be broken down in this way:

\textbf{seq no} This stamp logs messages with a sequence number, but not
by default. If you want this output, you've got to configure it.

\textbf{Timestamp} Data and time of the message or event, which again
will show up only if configured.

\textbf{Facility} The facility to which the message refers.

\textbf{Severity} A single-digit code from 0 to 7 that indicates the
severity of the message.

\textbf{MNEMONIC} Text string that uniquely describes the message.

\textbf{Description} Text string containing detailed information about
the event being reported.

The severity levels, from the most severe level to the least severe, are
explained in \protect\hyperlink{c07.xhtmlux5cux23table7-2}{Table 7.2}.
Informational is the default and will result in all messages being sent
to the buffers and console.

{\protect\hyperlink{c07.xhtmlux5cux23tableanchor7-2}{\textbf{TABLE 7.2}}
Severity levels}

\begin{longtable}[]{@{}ll@{}}
\toprule
Severity Level & Explanation\tabularnewline
\midrule
\endhead
Emergency (severity 0) & System is unusable.\tabularnewline
Alert (severity 1) & Immediate action is needed.\tabularnewline
Critical (severity 2) & Critical condition.\tabularnewline
Error (severity 3) & Error condition.\tabularnewline
Warning (severity 4) & Warning condition.\tabularnewline
Notification (severity 5) & Normal but significant
condition.\tabularnewline
Informational (severity 6) & Normal information message.\tabularnewline
Debugging (severity 7) & Debugging message.\tabularnewline
\bottomrule
\end{longtable}

\begin{center}\rule{0.5\linewidth}{0.5pt}\end{center}

\includegraphics{images/tip.png} If you are studying for your Cisco
exam, you need to memorize
\protect\hyperlink{c07.xhtmlux5cux23table7-2}{Table 7.2} using this
acronym: Every Awesome Cisco Engineer Will Need Icecream Daily.

\begin{center}\rule{0.5\linewidth}{0.5pt}\end{center}

Understand that only emergency-level messages will be displayed if
you've configured severity level 0. But if, for example, you opt for
level 4 instead, level 0 through 4 will be displayed, giving you
emergency, alert, critical, error, and warning messages too. Level 7
\protect\hypertarget{c07.xhtmlux5cux23Page_285}{}{}is the highest-level
security option and displays everything, but be warned that going with
it could have a serious impact on the performance of your device. So
always use debugging commands carefully, with an eye on the messages you
really need to meet your specific business requirements!

\subsubsection[Configuring and Verifying
Syslog]{\texorpdfstring{\protect\hypertarget{c07.xhtmlux5cux23c07-sec-11}{}{}Configuring
and Verifying Syslog}{Configuring and Verifying Syslog}}

As I said, Cisco devices send all log messages of the severity level
you've chosen to the console. They'll also go to the buffer, and both
happen by default. Because of this, it's good to know that you can
disable and enable these features with the following commands:

\begin{verbatim}
Router(config)#logging ?
  Hostname or A.B.C.D  IP address of the logging host
  buffered             Set buffered logging parameters
  buginf               Enable buginf logging for debugging
  cns-events           Set CNS Event logging level
  console              Set console logging parameters
  count                Count every log message and timestamp last occurrence
  esm                  Set ESM filter restrictions
  exception            Limit size of exception flush output
  facility             Facility parameter for syslog messages
  filter               Specify logging filter
  history              Configure syslog history table
  host                 Set syslog server IP address and parameters
  monitor              Set terminal line (monitor) logging parameters
  on                   Enable logging to all enabled destinations
  origin-id            Add origin ID to syslog messages
  queue-limit          Set logger message queue size
  rate-limit           Set messages per second limit
  reload               Set reload logging level
  server-arp           Enable sending ARP requests for syslog servers when
                       first configured
  source-interface     Specify interface for source address in logging
                       transactions
  trap                 Set syslog server logging level
  userinfo             Enable logging of user info on privileged mode enabling
 
Router(config)#logging console
Router(config)#logging buffered
\end{verbatim}

Wow---as you can see in this output, there are plenty of options you can
use with the \texttt{logging} command! The preceding configuration
enabled the console and buffer to receive
\protect\hypertarget{c07.xhtmlux5cux23Page_286}{}{}all log messages of
all severities, and don't forget that this is the default setting for
all Cisco IOS devices. If you want to disable the defaults, use the
following commands:

\begin{verbatim}
Router(config)#no logging console
Router(config)#no logging buffered
\end{verbatim}

I like leaving the console and buffer commands on in order to receive
the logging info, but that's up to you. You can see the buffers with the
\texttt{show\ logging} command here:

\begin{verbatim}
Router#sh logging
Syslog logging: enabled (11 messages dropped, 1 messages rate-limited,
                0 flushes, 0 overruns, xml disabled, filtering disabled)
    Console logging: level debugging, 29 messages logged, xml disabled,
                     filtering disabled
    Monitor logging: level debugging, 0 messages logged, xml disabled,
                     filtering disabled
    Buffer logging: level debugging, 1 messages logged, xml disabled,
                    filtering disabled
    Logging Exception size (4096 bytes)
    Count and timestamp logging messages: disabled
No active filter modules.
\end{verbatim}

\begin{verbatim}
    Trap logging: level informational, 33 message lines logged
\end{verbatim}

\begin{verbatim}
Log Buffer (4096 bytes):
*Jun 21 23:09:37.822: %SYS-5-CONFIG_I: Configured from console by console
Router#
\end{verbatim}

The default trap (message from device to NMS) level is
\texttt{debugging}, but you can change this too. And now that you've
seen the system message format on a Cisco device, I want to show you how
you can also control the format of your messages via sequence numbers
and time stamps, which aren't enabled by default. We'll begin with a
basic, simple example of how to configure a device to send messages to a
syslog server, demonstrated in
\protect\hyperlink{c07.xhtmlux5cux23figure7-4}{Figure 7.4}.

\begin{figure}
\centering
\includegraphics{images/c07f004.jpg}
\caption{{\protect\hyperlink{c07.xhtmlux5cux23figureanchor7-4}{\textbf{FIGURE
7.4}} Messages sent to a syslog server}}
\end{figure}

\protect\hypertarget{c07.xhtmlux5cux23Page_287}{}{}A syslog server saves
copies of console messages and can time-stamp them for viewing at a
later time. This is actually pretty easy to configure, and here's how
doing that would look on the SF router:

\begin{verbatim}
SF(config)#logging 172.16.10.1
SF(config)#logging informational
\end{verbatim}

This is awesome---now all the console messages will be stored in one
location to be viewed at your convenience! I typically use the
\texttt{logging\ host} \texttt{ip\_address} command, but
\texttt{logging} \texttt{IP\_address} without the \texttt{host} keyword
gets the same result.

We can limit the amount of messages sent to the syslog server, based on
severity, with the following command:

\begin{verbatim}
SF(config)#logging trap ?
  <0-7>          Logging severity level
  alerts         Immediate action needed           (severity=1)
  critical       Critical conditions               (severity=2)
  debugging      Debugging messages                (severity=7)
  emergencies    System is unusable                (severity=0)
  errors         Error conditions                  (severity=3)
  informational  Informational messages            (severity=6)
  notifications  Normal but significant conditions (severity=5)
  warnings       Warning conditions                (severity=4)
  <cr>
SF(config)#logging trap informational
\end{verbatim}

Notice that we can use either the number or the actual severity level
name---and they are in alphabetical order, not severity order, which
makes it even harder to memorize the order! (Thanks, Cisco!) Since I
went with severity level 6 (Informational), I'll receive messages for
levels 0 through 6. These are referred to as local levels as well, such
as, for example, local6---no difference.

Now let's configure the router to use sequence numbers:

\begin{verbatim}
SF(config)#no service timestamps
SF(config)#service sequence-numbers
SF(config)#^Z
000038: %SYS-5-CONFIG_I: Configured from console by console
\end{verbatim}

When you exit configuration mode, the router will send a message like
the one shown in the preceding code lines. Without the time stamps
enabled, we'll no longer see a time and date, but we will see a sequence
number.

So we now have the following:

\begin{enumerate}
\tightlist
\item
  Sequence number: 000038
\item
  Facility: \%SYS
\item
  \protect\hypertarget{c07.xhtmlux5cux23Page_288}{}{}Severity level: 5
\item
  MNEMONIC: CONFIG\_I
\item
  Description: Configured from console by console
\end{enumerate}

I want to stress that of all of these, the severity level is what you
need to pay attention to the most for the Cisco exams as well as for a
means to control the amount of messages sent to the syslog server.

\subsection[Network Time Protocol
(NTP)]{\texorpdfstring{\protect\hypertarget{c07.xhtmlux5cux23c07-sec-12}{}{}Network
Time Protocol (NTP)}{Network Time Protocol (NTP)}}

Network Time Protocol provides pretty much what it describes: time to
all your network devices. To be more precise, NTP synchronizes clocks of
computer systems over packet-switched, variable-latency data networks.

Typically you'll have an NTP server that connects through the Internet
to an atomic clock. This time can then be synchronized through the
network to keep all routers, switches, servers, etc. receiving the same
time information.

Correct network time within the network is important:

\begin{enumerate}
\tightlist
\item
  Correct time allows the tracking of events in the network in the
  correct order.
\item
  Clock synchronization is critical for the correct interpretation of
  events within the syslog data.
\item
  Clock synchronization is critical for digital certificates.
\end{enumerate}

Making sure all your devices have the correct time is especially helpful
for your routers and switches for looking at logs regarding security
issues or other maintenance issues. Routers and switches issue log
messages when different events take place---for example, when an
interface goes down and then back up. As you already know, all messages
generated by the IOS go only to the console port by default. However, as
shown in \protect\hyperlink{c07.xhtmlux5cux23figure7-4}{Figure 7.4},
those console messages can be directed to a syslog server.

A syslog server saves copies of console messages and can time-stamp them
so you can view them at a later time. This is actually rather easy to
do. Here would be your configuration on the SF router:

\begin{verbatim}
SF(config)#service timestamps log datetime msec
\end{verbatim}

Even though I had the messages time-stamped with the command
\texttt{service\ timestamps\ log\ datetime\ msec}, this doesn't mean
that we'll know the exact time if using default clock sources.

To make sure all devices are synchronized with the same time
information, we'll configure our devices to receive the accurate time
information from a centralized server, as shown here in the following
command and in \protect\hyperlink{c07.xhtmlux5cux23figure7-5}{Figure
7.5}:

\begin{verbatim}
SF(config)#ntp server 172.16.10.1 version 4
\end{verbatim}

\protect\hypertarget{c07.xhtmlux5cux23Page_289}{}{}

\begin{figure}
\centering
\includegraphics{images/c07f005.jpg}
\caption{{\protect\hyperlink{c07.xhtmlux5cux23figureanchor7-5}{\textbf{FIGURE
7.5}} Synchronizing time information}}
\end{figure}

Just use that one simple command on all your devices and each network
device on your network will then have the same exact time and date
information. You can then rest assured that your time stamps are
accurate. You can also make your router or switch be an NTP server with
the \texttt{ntp\ master} command.

To verify that our NTP client is receiving clocking information, we use
the following commands:

\begin{verbatim}
SF#sh ntp ?
  associations  NTP associations
  status        NTP status  status     VTP domain status
 
SF#sh ntp status
Clock is unsynchronized, stratum 16, no reference clock
nominal freq is 119.2092 Hz, actual freq is 119.2092 Hz, precision is 2**18
reference time is 00000000.00000000 (00:00:00.000 UTC Mon Jan 1 1900)
clock offset is 0.0000 msec, root delay is 0.00 msec
S1#sh ntp associations
 
address    ref clock     st  when  poll reach  delay  offset    disp
 ~172.16.10.1   0.0.0.0          16     -    64    0     0.0    0.00  16000.
 * master (synced), # master (unsynced), + selected, - candidate, ~ configured
\end{verbatim}

You can see in the example that the NTP client in SF is not synchronized
with the server by using the \texttt{show\ ntp\ status} command. The
stratum value is a number from 1 to 15, and a lower stratum value
indicates a higher NTP priority; 16 means there is no clocking received.

There are many other configurations of an NTP client that are available,
such as authentication of NTP so a router or switch isn't fooled into
changing the time of an attack, for example.

\subsection[Exploring Connected Devices Using CDP and
LLDP]{\texorpdfstring{\protect\hypertarget{c07.xhtmlux5cux23c07-sec-13}{}{}Exploring
Connected Devices Using CDP and
LLDP}{Exploring Connected Devices Using CDP and LLDP}}

Cisco Discovery Protocol (CDP) is a proprietary Layer 2 protocol
designed by Cisco to help administrators collect information about
locally attached Cisco devices. Armed with CDP, you can gather hardware
and protocol information about neighbor devices, which is
\protect\hypertarget{c07.xhtmlux5cux23Page_290}{}{}crucial information
to have when troubleshooting and documenting the network. Another
dynamic discovery protocol is Link Layer Discovery Protocol (LLDP), but
instead of being proprietary like CDP, it is vendor independent.

Let's start by exploring the CDP timer and CDP commands we'll need to
verify our network.

\subsubsection[Getting CDP Timers and Holdtime
Information]{\texorpdfstring{\protect\hypertarget{c07.xhtmlux5cux23c07-sec-14}{}{}Getting
CDP Timers and Holdtime
Information}{Getting CDP Timers and Holdtime Information}}

The \texttt{show\ cdp} command (\texttt{sh\ cdp} for short) gives you
information about two CDP global parameters that can be configured on
Cisco devices:

\begin{enumerate}
\tightlist
\item
  \emph{CDP timer} delimits how often CDP packets are transmitted out
  all active interfaces.
\item
  \emph{CDP holdtime} delimits the amount of time that the device will
  hold packets received from neighbor devices.
\end{enumerate}

Both Cisco routers and switches use the same parameters. Check out
\protect\hyperlink{c07.xhtmlux5cux23figure7-6}{Figure 7.6} to see how
CDP works within a switched network that I set up for my switching labs
in this book.

\begin{figure}
\centering
\includegraphics{images/c07f006.jpg}
\caption{{\protect\hyperlink{c07.xhtmlux5cux23figureanchor7-6}{\textbf{FIGURE
7.6}} Cisco Discovery Protocol}}
\end{figure}

The output on my 3560 SW-3 looks like this:

\begin{verbatim}
SW-3#sh cdp
Global CDP information:
        Sending CDP packets every 60 seconds
        Sending a holdtime value of 180 seconds
        Sending CDPv2 advertisements is enabled
\end{verbatim}

This output tells us that the default transmits every 60 seconds and
will hold packets from a neighbor in the CDP table for 180 seconds. I
can use the global commands \texttt{cdp\ holdtime} and
\texttt{cdp\ timer} to configure the CDP holdtime and timer on a router
if necessary like this:

\begin{verbatim}
SW-3(config)#cdp ?
  advertise-v2  CDP sends version-2 advertisements
  holdtime      Specify the holdtime (in sec) to be sent in packets
  run           Enable CDP
  timer         Specify the rate at which CDP packets are sent (in sec)
  tlv           Enable exchange of specific tlv information
 
SW-3(config)#cdp holdtime ?
  <10-255>  Length  of time  (in sec) that receiver must keep this packet
 
SW-3(config)#cdp timer ?
  <5-254>  Rate at which CDP packets are sent (in  sec)
\end{verbatim}

You can turn off CDP completely with the \texttt{no\ cdp\ run} command
from global configuration mode of a router and enable it with the
\texttt{cdp\ run} command:

\begin{verbatim}
SW-3(config)#no cdp run
SW-3(config)#cdp run
\end{verbatim}

To turn CDP off or on for an interface, use the \texttt{no\ cdp\ enable}
and \texttt{cdp\ enable} commands.

\subsubsection[Gathering Neighbor
Information]{\texorpdfstring{\protect\hypertarget{c07.xhtmlux5cux23c07-sec-15}{}{}Gathering
Neighbor Information}{Gathering Neighbor Information}}

The \texttt{show\ cdp\ neighbor} command (\texttt{sh\ cdp\ nei} for
short) delivers information about directly connected devices. It's
important to remember that CDP packets aren't passed through a Cisco
switch and that you only see what's directly attached. So this means
that if your router is connected to a switch, you won't see any of the
Cisco devices connected beyond that switch!

The following output shows the \texttt{show\ cdp\ neighbor} command I
used on my SW-3:

\begin{verbatim}
SW-3#sh cdp neighbors
Capability Codes: R - Router, T - Trans Bridge, B - Source Route Bridge
                  S - Switch, H - Host, I - IGMP, r - Repeater, P - Phone,
                  D - Remote, C - CVTA, M - Two-port Mac Relay Device ID     Local Intrfce     Holdtme    Capability  Platform  Port ID
SW-1   Fas 0/1      170          S I     WS-C3560- Fas 0/15
SW-1   Fas 0/2      170          S I     WS-C3560- Fas 0/16
SW-2   Fas 0/5      162          S I     WS-C3560- Fas 0/5
SW-2   Fas 0/6      162          S I     WS-C3560- Fas 0/6
\end{verbatim}

Okay---we can see that I'm directly connected with a console cable to
the SW-3 switch and also that SW-3 is directly connected to two other
switches. However, do we really need the figure to draw out our network?
We don't! CDP allows me to see who my directly connected neighbors are
and gather information about them. From the SW-3 switch, we can see that
there are two connections to SW-1 and two connections to SW-2. SW-3
connects to SW-1 with ports Fas 0/1 and Fas 0/2, and we have connections
to SW-2 with local
\protect\hypertarget{c07.xhtmlux5cux23Page_292}{}{}interfaces Fas 0/5
and Fas 0/6. Both the SW-1 and SW-2 switches are 3650 switches, and SW-1
is using ports Fas 0/15 and Fas 0/16 to connect to SW-3. SW-2 is using
ports Fas 0/5 and Fas 0/6.

To sum this up, the device ID shows the configured hostname of the
connected device, that the local interface is our interface, and the
port ID is the remote devices' directly connected interface. Remember
that all you get to view are directly connected devices!

\protect\hyperlink{c07.xhtmlux5cux23table7-3}{Table 7.3} summarizes the
information displayed by the \texttt{show\ cdp\ neighbor} command for
each device.

{\protect\hyperlink{c07.xhtmlux5cux23tableanchor7-3}{\textbf{TABLE 7.3}}
Output of the \texttt{show\ cdp\ neighbors} command}

\begin{longtable}[]{@{}ll@{}}
\toprule
Field & Description\tabularnewline
\midrule
\endhead
\texttt{Device\ ID} & The hostname of the device directly
connected.\tabularnewline
\texttt{Local\ Interface} & The port or interface on which you are
receiving the CDP packet.\tabularnewline
\texttt{Holdtime} & The remaining amount of time the router will hold
the information before discarding it if no more CDP packets are
received.\tabularnewline
\texttt{Capability} & The capability of the neighbor---the router,
switch, or repeater. The capability codes are listed at the top of the
command output.\tabularnewline
\texttt{Platform} & The type of Cisco device directly connected. In the
previous output, the SW-3 shows it's directly connected to two 3560
switches.\tabularnewline
\texttt{Port\ ID} & The neighbor device's port or interface on which the
CDP packets are multicast.\tabularnewline
\bottomrule
\end{longtable}

\begin{center}\rule{0.5\linewidth}{0.5pt}\end{center}

\includegraphics{images/note.png} It's imperative that you can look at
the output of a \texttt{show\ cdp\ neighbors} command and decipher the
information gained about the neighbor device's capability, whether it's
a router or switch, the model number (platform), your port connecting to
that device (local interface), and the port of the neighbor connecting
to you (port ID).

\begin{center}\rule{0.5\linewidth}{0.5pt}\end{center}

Another command that will deliver the goods on neighbor information is
the \texttt{show\ cdp\ neighbors\ detail} command
(\texttt{show\ cdp\ nei\ de} for short). This command can be run on both
routers and switches, and it displays detailed information about each
device connected to the device you're running the command on. Check out
the router output in Listing 7.1.

\textbf{\texttt{Listing\ 7.1:}} Showing CDP neighbors

\begin{verbatim}
SW-3#sh cdp neighbors detail
-------------------------
Device ID: SW-1
Entry address(es):
  IP address: 10.100.128.10
Platform: cisco WS-C3560-24TS,  Capabilities: Switch IGMP
Interface: FastEthernet0/1,  Port ID (outgoing port): FastEthernet0/15
Holdtime : 137 sec
\end{verbatim}

\begin{verbatim}
Version :
Cisco IOS Software, C3560 Software (C3560-IPSERVICESK9-M), Version 12.2(55)SE7, RELEASE SOFTWARE (fc1)
Technical Support: http://www.cisco.com/techsupport
Copyright (c) 1986-2013 by Cisco Systems, Inc.
Compiled Mon 28-Jan-13 10:10 by prod_rel_team
\end{verbatim}

\begin{verbatim}
advertisement version: 2
Protocol Hello:  OUI=0x00000C, Protocol ID=0x0112; payload len=27, value=00000000FFFFFFFF010221FF000000000000001C575EC880Fc00f000
VTP Management Domain: 'NULL'
Native VLAN: 1
Duplex: full
Power Available TLV:
\end{verbatim}

\begin{verbatim}
    Power request id: 0, Power management id: 1, Power available: 0, Power management level: -1
Management address(es):
  IP address: 10.100.128.10
-------------------------
\end{verbatim}

\begin{verbatim}
[ouput cut]
\end{verbatim}

\begin{verbatim}
-------------------------
Device ID: SW-2
Entry address(es):
  IP address: 10.100.128.9
Platform: cisco WS-C3560-8PC,  Capabilities: Switch IGMP
Interface: FastEthernet0/5,  Port ID (outgoing port): FastEthernet0/5
Holdtime : 129 sec
\end{verbatim}

\begin{verbatim}
Version :
Cisco IOS Software, C3560 Software (C3560-IPBASE-M), Version 12.2(35)SE5, RELEASE SOFTWARE (fc1)
Copyright (c) 1986-2007 by Cisco Systems, Inc.
Compiled Thu 19-Jul-07 18:15 by nachen
\end{verbatim}

\begin{verbatim}
advertisement version: 2
Protocol Hello:  OUI=0x00000C, Protocol ID=0x0112; payload len=27, value=00000000FFFFFFFF010221FF000000000000B41489D91880Fc00f000
VTP Management Domain: 'NULL'
Native VLAN: 1
Duplex: full
Power Available TLV:
\end{verbatim}

\begin{verbatim}
    Power request id: 0, Power management id: 1, Power available: 0, Power management level: -1
Management address(es):
  IP address: 10.100.128.9
[output cut]
\end{verbatim}

So what's revealed here? First, we've been given the hostname and IP
address of all directly connected devices. And in addition to the same
information displayed by the \texttt{show\ cdp\ neighbors} command (see
\protect\hyperlink{c07.xhtmlux5cux23table7-3}{Table 7.3}), the
\texttt{show\ cdp\ neighbors\ detail} command tells us about the IOS
version and IP address of the neighbor device---that's quite a bit!

The \texttt{show\ cdp\ entry\ *} command displays the same information
as the \texttt{show\ cdp\ neighbors\ detail} command. There isn't any
difference between these commands.

\begin{center}\rule{0.5\linewidth}{0.5pt}\end{center}

\includegraphics{images/globe1.png}\\
\textbf{CDP Can Save Lives!}

Karen has just been hired as a senior network consultant at a large
hospital in Dallas, Texas, so she's expected to be able to take care of
any problem that rears its ugly head. As if that weren't enough
pressure, she also has to worry about the horrid possibility that people
won't receive correct health care solutions---even the correct
medications---if the network goes down. Talk about a potential
life-or-death situation!

But Karen is confident and begins her job optimistically. Of course,
it's not long before the network reveals that it has a few problems.
Unfazed, she asks one of the junior administrators for a network map so
she can troubleshoot the network. This person tells her that the old
senior administrator, who she replaced, had them with him and now no one
can find them. The sky begins to darken!

Doctors are calling every couple of minutes because they can't get the
necessary information they need to take care of their patients. What
should she do?

It's CDP to the rescue! And it's a gift that this hospital happens to be
running Cisco ­routers and switches exclusively, because CDP is enabled
by default on all Cisco devices. Karen is also in luck because the
disgruntled former administrator didn't turn off CDP on any devices
before he left!

So all Karen has to do now is to use the
\texttt{show\ cdp\ neighbor\ detail} command to find all the information
she needs about each device to help draw out the hospital network,
bringing it back up to speed so the personnel who rely upon it can get
on to the important business of saving lives!

\protect\hypertarget{c07.xhtmlux5cux23Page_295}{}{}The only snag for you
nailing this in your own network is if you don't know the passwords of
all those devices. Your only hope then is to somehow find out the access
passwords or to perform password recovery on them.

So, use CDP---you never know when you may end up saving someone's life.

By the way, this is a true story!

\begin{center}\rule{0.5\linewidth}{0.5pt}\end{center}

\subsubsection[Documenting a Network Topology Using
CDP]{\texorpdfstring{\protect\hypertarget{c07.xhtmlux5cux23c07-sec-16}{}{}Documenting
a Network Topology Using CDP}{Documenting a Network Topology Using CDP}}

With that moving real-life scenario in mind, I'm now going to show you
how to document a sample network by using CDP. You'll learn to determine
the appropriate router types, interface types, and IP addresses of
various interfaces using only CDP commands and the
\texttt{show\ running-config} command. And you can only console into the
Lab\_A router to ­document the network. You'll have to assign any remote
routers the next IP address in each range. We'll use a different figure
for this example---\protect\hyperlink{c07.xhtmlux5cux23figure7-7}{Figure
7.7}--- to help us to complete the necessary documentation.

\begin{figure}
\centering
\includegraphics{images/c07f007.jpg}
\caption{{\protect\hyperlink{c07.xhtmlux5cux23figureanchor7-7}{\textbf{FIGURE
7.7}} Documenting a network topology using CDP}}
\end{figure}

In this output, you can see that you have a router with four interfaces:
two Fast Ethernet and two serial. First, determine the IP addresses of
each interface by using the \texttt{show\ ­running-config} command like
this:

\begin{verbatim}
Lab_A#sh running-config
Building configuration...
\end{verbatim}

\begin{verbatim}
Current configuration : 960 bytes
!
version 12.2
service timestamps debug uptime
service timestamps log uptime
no service password-encryption
!
hostname Lab_A
!
ip subnet-zero
!
!
interface FastEthernet0/0
 ip address 192.168.21.1 255.255.255.0
 duplex auto
!
interface FastEthernet0/1
 ip address 192.168.18.1 255.255.255.0
 duplex auto
!
interface Serial0/0
ip address 192.168.23.1 255.255.255.0
!
interface Serial0/1
ip address 192.168.28.1 255.255.255.0
!
ip classless
!
line con 0
line aux 0
line vty 0 4
!
end
\end{verbatim}

With this step completed, you can now write down the IP addresses of the
Lab\_A router's four interfaces. Next, you must determine the type of
device on the other end of each of these interfaces. It's easy---just
use the \texttt{show\ cdp\ neighbors} command:

\begin{verbatim}
Lab_A#sh cdp neighbors
Capability Codes: R - Router, T - Trans Bridge, B - Source Route Bridge
S - Switch, H - Host, I - IGMP, r - Repeater
Device ID   Local Intrfce     Holdtme    Capability Platform  Port ID
Lab_B        Fas 0/0            178          R        2501     E0
Lab_C        Fas 0/1            137          R        2621     Fa0/0
Lab_D        Ser 0/0            178          R        2514     S1
Lab_E        Ser 0/1            137          R        2620     S0/1
\end{verbatim}

\protect\hypertarget{c07.xhtmlux5cux23Page_297}{}{}Wow---looks like
we're connected to some old routers! But it's not our job to judge. Our
mission is to draw out our network, so it's good that we've got some
nice information to meet the challenge with now. By using both the
\texttt{show\ running-config} and \texttt{show\ cdp\ neighbors}
commands, we know about all the IP addresses of the Lab\_A router, the
types of routers connected to each of the Lab\_A router's links, and all
the interfaces of the remote routers.

Now that we're equipped with all the information gathered via
\texttt{show\ running-config} and \texttt{show\ cdp\ neighbors}, we can
accurately create the topology in
\protect\hyperlink{c07.xhtmlux5cux23figure7-8}{Figure 7.8}.

\begin{figure}
\centering
\includegraphics{images/c07f008.jpg}
\caption{{\protect\hyperlink{c07.xhtmlux5cux23figureanchor7-8}{\textbf{FIGURE
7.8}} Network topology documented}}
\end{figure}

If we needed to, we could've also used the
\texttt{show\ cdp\ neighbors\ detail} command to view the neighbor's IP
addresses. But since we know the IP addresses of each link on the Lab\_A
router, we already know what the next available IP address is going to
be.

\paragraph{Link Layer Discovery Protocol (LLDP)}

Before moving on from CDP, I want to tell you about a nonproprietary
discovery protocol that provides pretty much the same information as CDP
but works in multi-vendor networks.

The IEEE created a new standardized discovery protocol called 802.1AB
for Station and Media Access Control Connectivity Discovery. We'll just
call it \emph{Link Layer Discovery Protocol (LLDP)}.

LLDP defines basic discovery capabilities, but it was also enhanced to
specifically address the voice application, and this version is called
LLDP-MED (Media Endpoint Discovery). It's good to remember that LLDP and
LLDP-MED are not compatible.

LLDP has the following configuration guidelines and limitations:

\begin{enumerate}
\tightlist
\item
  LLDP must be enabled on the device before you can enable or disable it
  on any ­interface.
\item
  LLDP is supported only on physical interfaces.
\item
  LLDP can discover up to one device per port.
\item
  LLDP can discover Linux servers.
\end{enumerate}

\protect\hypertarget{c07.xhtmlux5cux23Page_298}{}{}You can turn off LLDP
completely with the \texttt{no\ lldp\ run} command from global
configuration mode of a device and enable it with the \texttt{lldp\ run}
command, which enables it on all interfaces as well:

\begin{verbatim}
SW-3(config)#no lldp run
SW-3(config)#lldp run
\end{verbatim}

To turn LLDP off or on for an interface, use
the\texttt{\ lldp\ transmit} and \texttt{lldp\ receive} commands.

\begin{verbatim}
SW-3(config-if)#no lldp transmit
SW-3(config-if)#no lldp receive
 
SW-3(config-if)#lldp transmit
SW-3(config-if)#lldp receive
\end{verbatim}

\subsection[Using
Telnet]{\texorpdfstring{\protect\hypertarget{c07.xhtmlux5cux23c07-sec-17}{}{}Using
Telnet}{Using Telnet}}

As part of the TCP/IP protocol suite, \emph{Telnet} is a virtual
terminal protocol that allows you to make connections to remote devices,
gather information, and run programs.

After your routers and switches are configured, you can use the Telnet
program to reconfigure and/or check up on them without using a console
cable. You run the Telnet program by typing \texttt{telnet} from any
command prompt (Windows or Cisco), but you need to have VTY passwords
set on the IOS devices for this to work.

Remember, you can't use CDP to gather information about routers and
switches that aren't directly connected to your device. But you can use
the Telnet application to connect to your neighbor devices and then run
CDP on those remote devices to get information on them.

You can issue the \texttt{telnet} command from any router or switch
prompt. In the following code, I'm trying to telnet from switch 1 to
switch 3:

\begin{verbatim}
SW-1#telnet 10.100.128.8
Trying 10.100.128.8 ... Open
\end{verbatim}

\begin{verbatim}
Password required, but none set
\end{verbatim}

\begin{verbatim}
[Connection to 10.100.128.8 closed by foreign host]
\end{verbatim}

Oops---clearly, I didn't set my passwords---how embarrassing! Remember
that the VTY ports are default configured as \texttt{login}, meaning
that we have to either set the VTY passwords or use the
\texttt{no\ login} command. If you need to review the process of setting
­passwords, take a quick look back in Chapter 6, ``Cisco's
Internetworking Operating System (IOS).''

\protect\hypertarget{c07.xhtmlux5cux23Page_299}{}{}

\begin{center}\rule{0.5\linewidth}{0.5pt}\end{center}

\includegraphics{images/note.png} If you can't telnet into a device, it
could be that the password on the remote device hasn't been set. It's
also quite possible that an access control list is filtering the Telnet
session.

\begin{center}\rule{0.5\linewidth}{0.5pt}\end{center}

On a Cisco device, you don't need to use the \texttt{telnet} command;
you can just type in an IP address from a command prompt and the router
will assume that you want to telnet to the device. Here's how that looks
using just the IP address:

\begin{verbatim}
SW-1#10.100.128.8
Trying 10.100.128.8... Open
\end{verbatim}

\begin{verbatim}
Password required, but none set
\end{verbatim}

\begin{verbatim}
[Connection to 10.100.128.8 closed by foreign host]
SW-1#
\end{verbatim}

Now would be a great time to set those VTY passwords on the SW-3 that I
want to telnet into. Here's what I did on the switch named SW-3:

\begin{verbatim}
SW-3(config)#line vty 0 15
SW-3(config-line)#login
SW-3(config-line)#password telnet
SW-3(config-line)#login
SW-3(config-line)#^Z
\end{verbatim}

Now let's try this again. This time, I'm connecting to SW-3 from the
SW-1 console:

\begin{verbatim}
SW-1#10.100.128.8
Trying 10.100.128.8 ... Open


User Access Verification

Password:
SW-3>
\end{verbatim}

Remember that the VTY password is the user-mode password, not the
enable-mode password. Watch what happens when I try to go into
privileged mode after telnetting into the switch:

\begin{verbatim}
SW-3>en
% No password set
SW-3>
\end{verbatim}

It's totally slamming the door in my face, which happens to be a really
nice security feature! After all, you don't want just anyone telnetting
into your device and typing the \texttt{enable} command to get into
privileged mode now, do you? You've got to set your enable-mode password
or enable secret password to use Telnet to configure remote devices.

\protect\hypertarget{c07.xhtmlux5cux23Page_300}{}{}

\begin{center}\rule{0.5\linewidth}{0.5pt}\end{center}

\includegraphics{images/note.png} When you telnet into a remote device,
you won't see console messages by default. For example, you will not see
debugging output. To allow console messages to be sent to your Telnet
session, use the \texttt{terminal\ monitor} command.

\begin{center}\rule{0.5\linewidth}{0.5pt}\end{center}

Using the next group of examples, I'll show you how to telnet into
multiple devices simultaneously as well as how to use hostnames instead
of IP addresses.

\subsubsection[Telnetting into Multiple Devices
Simultaneously]{\texorpdfstring{\protect\hypertarget{c07.xhtmlux5cux23c07-sec-18}{}{}Telnetting
into Multiple Devices
Simultaneously}{Telnetting into Multiple Devices Simultaneously}}

If you telnet to a router or switch, you can end the connection by
typing \texttt{exit} at any time. But what if you want to keep your
connection to a remote device going while still coming back to your
original router console? To do that, you can press the Ctrl+Shift+6 key
combination, release it, and then press X.

Here's an example of connecting to multiple devices from my SW-1
console:

\begin{verbatim}
SW-1#10.100.128.8
Trying 10.100.128.8... Open
\end{verbatim}

\begin{verbatim}
User Access Verification
\end{verbatim}

\begin{verbatim}
Password:
SW-3>Ctrl+Shift+6
SW-1#
\end{verbatim}

Here you can see that I telnetted to SW-1 and then typed the password to
enter user mode. Next, I pressed Ctrl+Shift+6, then X, but you won't see
any of that because it doesn't show on the screen output. Notice that my
command prompt now has me back at the SW-1 switch.

Now let's run through some verification commands.

\subsubsection[Checking Telnet
Connections]{\texorpdfstring{\protect\hypertarget{c07.xhtmlux5cux23c07-sec-19}{}{}Checking
Telnet Connections}{Checking Telnet Connections}}

If you want to view the connections from your router or switch to a
remote device, just use the \texttt{show\ sessions} command. In this
case, I've telnetted into both the SW-3 and SW-2 switches from SW1:

\begin{verbatim}
SW-1#sh sessions
Conn Host            Address          Byte  Idle Conn Name
   1 10.100.128.9    10.100.128.9     0          10.100.128.9
*  2 10.100.128.8    10.100.128.8     0          10.100.128.8
SW-1#
\end{verbatim}

\protect\hypertarget{c07.xhtmlux5cux23Page_301}{}{}See that asterisk
(\texttt{*}) next to connection 2? It means that session 2 was the last
session I connected to. You can return to your last session by pressing
Enter twice. You can also return to any session by typing the number of
the connection and then Enter.

\subsubsection[Checking Telnet
Users]{\texorpdfstring{\protect\hypertarget{c07.xhtmlux5cux23c07-sec-20}{}{}Checking
Telnet Users}{Checking Telnet Users}}

You can reveal all active consoles and VTY ports in use on your router
with the \texttt{show\ users} command:

\begin{verbatim}
SW-1#sh users
    Line       User       Host(s)              Idle       Location
*  0 con 0                10.100.128.9         00:00:01
                          10.100.128.8         00:01:06
\end{verbatim}

In the command's output, \texttt{con} represents the local console, and
we can see that the console session is connected to two remote IP
addresses---in other words, two devices.

\subsubsection[Closing Telnet
Sessions]{\texorpdfstring{\protect\hypertarget{c07.xhtmlux5cux23c07-sec-21}{}{}Closing
Telnet Sessions}{Closing Telnet Sessions}}

You can end Telnet sessions a few different ways. Typing \texttt{exit}
or \texttt{disconnect} are probably the two quickest and easiest.

To end a session from a remote device, use the \texttt{exit} command:

\begin{verbatim}
SW-3>exit
[Connection to 10.100.128.8 closed by foreign host]
SW-1#
\end{verbatim}

To end a session from a local device, use the \texttt{disconnect}
command:

\begin{verbatim}
SW-1#sh session
Conn Host             Address           Byte  Idle Conn Name
   *2 10.100.128.9    10.100.128.9      0          10.100.128.9
SW-1#disconnect ?
  <2-2>  The number of an active network connection
  qdm    Disconnect QDM web-based clients
  ssh    Disconnect an active SSH connection
SW-1#disconnect 2
Closing connection to 10.100.128.9 [confirm][enter]
\end{verbatim}

In this example, I used session number 2 because that was the connection
I wanted to conclude. As demonstrated, you can use the
\texttt{show\ sessions} command to see the connection number.

\subsection[Resolving
Hostnames]{\texorpdfstring{\protect\hypertarget{c07.xhtmlux5cux23c07-sec-22}{}{}\protect\hypertarget{c07.xhtmlux5cux23Page_302}{}{}Resolving
Hostnames}{Resolving Hostnames}}

If you want to use a hostname instead of an IP address to connect to a
remote device, the device that you're using to make the connection must
be able to translate the hostname to an IP address.

There are two ways to resolve hostnames to IP addresses. The first is by
building a host table on each router, and the second is to build a
Domain Name System (DNS) server. The latter method is similar to
creating a dynamic host table, assuming that you're dealing with dynamic
DNS.

\subsubsection[Building a Host
Table]{\texorpdfstring{\protect\hypertarget{c07.xhtmlux5cux23c07-sec-23}{}{}Building
a Host Table}{Building a Host Table}}

An important factor to remember is that although a host table provides
name resolution, it does that only on the specific router that it was
built upon. The command you use to build a host table on a router looks
this:

\begin{verbatim}
ip host host_name [tcp_port_number] ip_address
\end{verbatim}

The default is TCP port number 23, but you can create a session using
Telnet with a different TCP port number if you want. You can also assign
up to eight IP addresses to a hostname.

Here's how I configured a host table on the SW-1 switch with two entries
to resolve the names for the SW-2 and SW-3:

\begin{verbatim}
SW-1#config t
SW-1(config)#ip host SW-2 ?
  <0-65535>   Default telnet port number
  A.B.C.D     Host IP address
  additional  Append addresses
\end{verbatim}

\begin{verbatim}
SW-1(config)#ip host SW-2 10.100.128.9
SW-1(config)#ip host SW-3 10.100.128.8
\end{verbatim}

Notice that I can just keep adding IP addresses to reference a unique
host, one after another. To view our newly built host table, I'll just
use the \texttt{show\ hosts} command:

\begin{verbatim}
SW-1(config)#do sho hosts
Default domain is not set
Name/address lookup uses domain service
Name servers are 255.255.255.255
\end{verbatim}

\begin{verbatim}
Codes: u - unknown, e - expired, * - OK, ? - revalidate
       t - temporary, p - permanent
\end{verbatim}

\begin{verbatim}
Host                   Port  Flags      Age Type   Address(es)
SW-3                   None  (perm, OK)  0   IP    10.100.128.8
SW-2                   None  (perm, OK)  0   IP    10.100.128.9
\end{verbatim}

In this output, you can see the two hostnames plus their associated IP
addresses. The \texttt{perm} in the \texttt{Flags} column means that the
entry has been manually configured. If it read \texttt{temp}, it would
be an entry that was resolved by DNS.

\begin{center}\rule{0.5\linewidth}{0.5pt}\end{center}

\includegraphics{images/note.png} The \texttt{show\ hosts} command
provides information on temporary DNS entries and permanent
name-to-address mappings created using the \texttt{ip\ host} command.

\begin{center}\rule{0.5\linewidth}{0.5pt}\end{center}

To verify that the host table resolves names, try typing the hostnames
at a router prompt. Remember that if you don't specify the command, the
router will assume you want to telnet.

In the following example, I'll use the hostnames to telnet into the
remote devices and press Ctrl+Shift+6 and then X to return to the main
console of the SW-1 router:

\begin{verbatim}
SW-1#sw-3
Trying SW-3 (10.100.128.8)... Open
 
User Access Verification
 
Password:
SW-3> Ctrl+Shift+6
SW-1#
\end{verbatim}

It worked---I successfully used entries in the host table to create a
session to the SW-3 device by using the name to telnet into it. And just
so you know, names in the host table are not case sensitive.

Notice that the entries in the following \texttt{show\ sessions} output
now display the hostnames and IP addresses instead of just the IP
addresses:

\begin{verbatim}
SW-1#sh sessions
Conn Host                Address             Byte  Idle Conn Name
   1 SW-3                10.100.128.8        0     1    SW-3
*  2 SW-2                10.100.128.9        0     1    SW-2
SW-1#
\end{verbatim}

If you want to remove a hostname from the table, all you need to do is
use the \texttt{no\ ip\ host} command like this:

\begin{verbatim}
SW-1(config)#no ip host SW-3
\end{verbatim}

\protect\hypertarget{c07.xhtmlux5cux23Page_304}{}{}The drawback to going
with this host table method is that you must create a host table on each
router in order to be able to resolve names. So clearly, if you have a
whole bunch of routers and want to resolve names, using DNS is a much
better option!

\subsubsection[Using DNS to Resolve
Names]{\texorpdfstring{\protect\hypertarget{c07.xhtmlux5cux23c07-sec-24}{}{}Using
DNS to Resolve Names}{Using DNS to Resolve Names}}

If you have a lot of devices, you don't want to create a host table in
each one of them unless you've also got a lot of time to waste. Since
most of us don't, I highly recommend using a DNS server to resolve
hostnames instead!

Anytime a Cisco device receives a command it doesn't understand, it will
try to resolve it through DNS by default. Watch what happens when I type
the special command \texttt{todd} at a Cisco router prompt:

\begin{verbatim}
SW-1#todd
Translating "todd"...domain server (255.255.255.255)
% Unknown command or computer name, or unable to find
  computer address
SW-1#
\end{verbatim}

Because it doesn't know my name or the command I'm trying to type, it
tries to resolve this through DNS. This is really annoying for two
reasons: first, because it doesn't know my name
\textless grin\textgreater, and second, because I need to hang out and
wait for the name lookup to time out. You can get around this and
prevent a time-consuming DNS lookup by using the
\texttt{no\ ip\ domain-lookup} command on your router from global
configuration mode.

So if you have a DNS server on your network, you'll need to add a few
commands to make DNS name resolution work well for you:

\begin{enumerate}
\tightlist
\item
  The first command is \texttt{ip\ domain-lookup}, which is turned on by
  default. It needs to be entered only if you previously turned it off
  with the \texttt{no\ ip\ domain-lookup} command. The command can be
  used without the hyphen as well with the syntax
  \texttt{ip\ domain\ lookup}.
\item
  The second command is \texttt{ip\ name-server}. This sets the IP
  address of the DNS server. You can enter the IP addresses of up to six
  servers.
\item
  The last command is \texttt{ip\ domain-name}. Although this command is
  optional, you really need to set it because it appends the domain name
  to the hostname you type in. Since DNS uses a fully qualified domain
  name (FQDN) system, you must have a second-level DNS name, in the form
  \texttt{domain.com}.
\end{enumerate}

Here's an example of using these three commands:

\begin{verbatim}
SW-1#config t
SW-1(config)#ip domain-lookup
SW-1(config)#ip name-server ?
  A.B.C.D  Domain server IP address (maximum of 6)
SW-1(config)#ip name-server 4.4.4.4
SW-1(config)#ip domain-name lammle.com
SW-1(config)#^Z
\end{verbatim}

After the DNS configurations have been set, you can test the DNS server
by using a hostname to ping or telnet into a device like this:

\begin{verbatim}
SW-1#ping SW-3
Translating "SW-3"...domain server (4.4.4.4) [OK]
Type escape sequence to abort.
Sending 5, 100-byte ICMP Echos to 10.100.128.8, timeout is
  2 seconds:
!!!!!
Success rate is 100 percent (5/5), round-trip min/avg/max
  = 28/31/32 ms
\end{verbatim}

Notice that the router uses the DNS server to resolve the name.

After a name is resolved using DNS, use the \texttt{show\ hosts} command
to verify that the device cached this information in the host table. If
I hadn't used the \texttt{ip\ domain-name\ ­lammle.com} command, I would
have needed to type in \texttt{ping\ sw-3.lammle.com}, which is kind of
a hassle.

\begin{center}\rule{0.5\linewidth}{0.5pt}\end{center}

\includegraphics{images/globe1.png}\\
\textbf{Should You Use a Host Table or a DNS Server?}

Karen has finally finished mapping her network via CDP and the
hospital's staff is now much happier. But Karen is still having a
difficult time administering the network because she has to look at the
network drawing to find an IP address every time she needs to telnet to
a remote router.

Karen was thinking about putting host tables on each router, but with
literally hundreds of routers, this is a daunting task and not the best
solution. What should she do?

Most networks have a DNS server now anyway, so adding a hundred or so
hostnames into it would be much easier---certainly better than adding
these hostnames to each and every router! She can just add the three
commands on each router and voilà---she's resolving names!

Using a DNS server makes it easy to update any old entries too.
Remember, for even one little change, her alternative would be to go to
each and every router to manually update its table if she's using static
host tables.

Keep in mind that this has nothing to do with name resolution on the
network and nothing to do with what a host on the network is trying to
accomplish. You only use this method when you're trying to resolve names
from the router console.

\begin{center}\rule{0.5\linewidth}{0.5pt}\end{center}

\subsection[Checking Network Connectivity and
Troubleshooting]{\texorpdfstring{\protect\hypertarget{c07.xhtmlux5cux23c07-sec-25}{}{}\protect\hypertarget{c07.xhtmlux5cux23Page_306}{}{}Checking
Network Connectivity and
Troubleshooting}{Checking Network Connectivity and Troubleshooting}}

You can use the \texttt{ping} and \texttt{traceroute} commands to test
connectivity to remote devices, and both of them can be used with many
protocols, not just IP. But don't forget that the
\texttt{show\ ip\ route} command is a great troubleshooting command for
verifying your routing table and the \texttt{show\ interfaces} command
will reveal the status of each interface to you.

I'm not going to get into the \texttt{show\ interfaces} commands here
because we've already been over that in Chapter 6. But I am going to go
over both the \texttt{debug} command and the \texttt{show\ processes}
command, both of which come in very handy when you need to troubleshoot
a router.

\subsubsection[Using the \emph{ping}
Command]{\texorpdfstring{\protect\hypertarget{c07.xhtmlux5cux23c07-sec-26}{}{}Using
the \emph{ping} Command}{Using the ping Command}}

So far, you've seen lots of examples of pinging devices to test IP
connectivity and name resolution using the DNS server. To see all the
different protocols that you can use with the \emph{Ping} program, type
\texttt{ping\ ?}:

\begin{verbatim}
SW-1#ping ?
  WORD  Ping destination address or hostname
  clns  CLNS echo
  ip    IP echo
  ipv6  IPv6 echo
  tag   Tag encapsulated IP echo
  <cr>
\end{verbatim}

The \texttt{ping} output displays the minimum, average, and maximum
times it takes for a \texttt{ping} packet to find a specified system and
return. Here's an example:

\begin{verbatim}
SW-1#ping SW-3
Translating "SW-3"...domain server (4.4.4.4) [OK]
Type escape sequence to abort.
Sending 5, 100-byte ICMP Echos to 10.100.128.8, timeout is
  2 seconds:
!!!!!
Success rate is 100 percent (5/5), round-trip min/avg/max
  = 28/31/32 ms
\end{verbatim}

This output tells us that the DNS server was used to resolve the name,
and the device was pinged in a minimum of 28 ms (milliseconds), an
average of 31 ms, and up to 32 ms. This network has some latency!

\protect\hypertarget{c07.xhtmlux5cux23Page_307}{}{}

\begin{center}\rule{0.5\linewidth}{0.5pt}\end{center}

\includegraphics{images/note.png} The \texttt{ping} command can be used
in user and privileged mode but not configuration mode!

\begin{center}\rule{0.5\linewidth}{0.5pt}\end{center}

\subsubsection[Using the \emph{traceroute}
Command]{\texorpdfstring{\protect\hypertarget{c07.xhtmlux5cux23c07-sec-27}{}{}Using
the \emph{traceroute} Command}{Using the traceroute Command}}

\emph{Traceroute}---the \texttt{traceroute} command, or \texttt{trace}
for short---shows the path a packet takes to get to a remote device. It
uses time to live (TTL), time-outs, and ICMP error messages to outline
the path a packet takes through an internetwork to arrive at a remote
host.

The \texttt{trace} command, which you can deploy from either user mode
or privileged mode, allows you to figure out which router in the path to
an unreachable network host should be examined more closely as the
probable cause of your network's failure.

To see the protocols that you can use with the \texttt{traceroute}
command, type \texttt{traceroute\ ?}:

\begin{verbatim}
SW-1#traceroute ?
  WORD       Trace route to destination address or hostname
  appletalk  AppleTalk Trace
  clns       ISO CLNS Trace
  ip         IP Trace
  ipv6       IPv6 Trace
  ipx        IPX Trace
  mac        Trace Layer2 path between 2 endpoints
  oldvines   Vines Trace (Cisco)
  vines      Vines Trace (Banyan)
  <cr>
\end{verbatim}

The \texttt{traceroute} command shows the hop or hops that a packet
traverses on its way to a remote device.

\begin{center}\rule{0.5\linewidth}{0.5pt}\end{center}

\includegraphics{images/tip.png} Do not get confused! You can't use the
\texttt{tracert} command; that's a ­Windows command. For a router, use
the \texttt{traceroute} command!

\begin{center}\rule{0.5\linewidth}{0.5pt}\end{center}

Here's an example of using \texttt{tracert} on a Windows prompt---notice
that the command is \texttt{tracert}, not \texttt{traceroute}:

\begin{verbatim}
C:\>tracert www.whitehouse.gov

Tracing route to a1289.g.akamai.net [69.8.201.107]
over a maximum of 30 hops:

  1     *        *        *     Request timed out.
  2    53 ms    61 ms    53 ms  hlrn-dsl-gw15-207.hlrn.qwest.net [207.225.112.207]
  3    53 ms    55 ms    54 ms  hlrn-agw1.inet.qwest.net [71.217.188.113]
  4    54 ms    53 ms    54 ms  hlr-core-01.inet.qwest.net [205.171.253.97]
  5    54 ms    53 ms    54 ms  apa-cntr-01.inet.qwest.net [205.171.253.26]
  6    54 ms    53 ms    53 ms  63.150.160.34
  7    54 ms    54 ms    53 ms  www.whitehouse.gov [69.8.201.107]

Trace complete.
\end{verbatim}

Okay, let's move on now and talk about how to troubleshoot your network
using the \texttt{debug} command.

\subsubsection[Debugging]{\texorpdfstring{\protect\hypertarget{c07.xhtmlux5cux23c07-sec-28}{}{}Debugging}{Debugging}}

Debug is a useful troubleshooting command that's available from the
privileged exec mode of Cisco IOS. It's used to display information
about various router operations and the related traffic generated or
received by the router, plus any error messages.

Even though it's a helpful, informative tool, there are a few important
facts that you need to know about it. Debug is regarded as a very
high-overhead task because it can consume a huge amount of resources and
the router is forced to process-switch the packets being debugged. So
you don't just use debug as a monitoring tool---it's meant to be used
for a short period of time and only as a troubleshooting tool. It's
highly useful for discovering some truly significant facts about both
working and faulty software and/or hardware components, but remember to
limit its use as the beneficial troubleshooting tool it's designed to
be.

Because debugging output takes priority over other network traffic, and
because the \texttt{debug\ all} command generates more output than any
other \texttt{debug} command, it can severely diminish the router's
performance---even render it unusable! Because of this, it's nearly
always best to use more specific \texttt{debug} commands.

As you can see from the following output, you can't enable debugging
from user mode, only privileged mode:

\begin{verbatim}
SW-1>debug ?
% Unrecognized command
SW-1>en
SW-1#debug ?
  aaa                   AAA Authentication, Authorization and Accounting
  access-expression     Boolean access expression
  adjacency             adjacency
  aim                   Attachment Information Manager
  all                   Enable all debugging
  archive               debug archive commands
  arp                   IP ARP and HP Probe transactions
  authentication        Auth Manager debugging
  auto                  Debug Automation
  beep                  BEEP debugging
  bgp                   BGP information
  bing                  Bing(d) debugging
  call-admission        Call admission control
  cca                   CCA activity
  cdp                   CDP information
  cef                   CEF address family independent operations
  cfgdiff               debug cfgdiff commands
  cisp                  CISP debugging
  clns                  CLNS information
  cluster               Cluster information
  cmdhd                 Command Handler
  cns                   CNS agents
  condition             Condition
  configuration         Debug Configuration behavior
[output cut]
\end{verbatim}

If you've got the freedom to pretty much take out a router or switch and
you really want to have some fun with debugging, use the
\texttt{debug\ all} command:

\begin{verbatim}
Sw-1#debug all
\end{verbatim}

\begin{verbatim}
This may severely impact network performance. Continue? (yes/[no]):yes
All possible debugging has been turned on
\end{verbatim}

At this point my switch overloaded and crashed and I had to reboot it.
Try this on your switch at work and see if you get the same results.
Just kidding!

To disable debugging on a router, just use the command \texttt{no} in
front of the \texttt{debug} command:

\begin{verbatim}
SW-1#no debug all
\end{verbatim}

I typically just use the \texttt{undebug\ all} command since it is so
easy when using the shortcut:

\begin{verbatim}
SW-1#un all
\end{verbatim}

Remember that instead of using the \texttt{debug\ all} command, it's
usually a much better idea to use specific commands---and only for short
periods of time. Here's an example:

\begin{verbatim}
S1#debug ip icmp
ICMP packet debugging is on
S1#ping 192.168.10.17
\end{verbatim}

\begin{verbatim}
Type escape sequence to abort.
Sending 5, 100-byte ICMP Echos to 192.168.10.17, timeout is 2 seconds:
!!!!!
Success rate is 100 percent (5/5), round-trip min/avg/max = 1/1/1 ms
S1#
1w4d: ICMP: echo reply sent, src 192.168.10.17, dst 192.168.10.17
1w4d: ICMP: echo reply rcvd, src 192.168.10.17, dst 192.168.10.17
1w4d: ICMP: echo reply sent, src 192.168.10.17, dst 192.168.10.17
1w4d: ICMP: echo reply rcvd, src 192.168.10.17, dst 192.168.10.17
1w4d: ICMP: echo reply sent, src 192.168.10.17, dst 192.168.10.17
1w4d: ICMP: echo reply rcvd, src 192.168.10.17, dst 192.168.10.17
1w4d: ICMP: echo reply sent, src 192.168.10.17, dst 192.168.10.17
1w4d: ICMP: echo reply rcvd, src 192.168.10.17, dst 192.168.10.17
1w4d: ICMP: echo reply sent, src 192.168.10.17, dst 192.168.10.17
1w4d: ICMP: echo reply rcvd, src 192.168.10.17, dst 192.168.10.17
SW-1#un all
\end{verbatim}

I'm sure you can see that the \texttt{debug} command is one powerful
command. And because of this, I'm also sure you realize that before you
use any of the debugging commands, you should make sure you check the
CPU utilization capacity of your router. This is important because in
most cases, you don't want to negatively impact the device's ability to
process the packets on your internetwork. You can determine a specific
router's CPU utilization information by using the
\texttt{show\ processes} command.

\begin{center}\rule{0.5\linewidth}{0.5pt}\end{center}

\includegraphics{images/note.png} Remember, when you telnet into a
remote device, you will not see console messages by default! For
example, you will not see debugging output. To allow console messages to
be sent to your Telnet session, use the \texttt{terminal\ monitor}
command.

\begin{center}\rule{0.5\linewidth}{0.5pt}\end{center}

\subsubsection[Using the \emph{show processes}
Command]{\texorpdfstring{\protect\hypertarget{c07.xhtmlux5cux23c07-sec-29}{}{}Using
the \emph{show processes} Command}{Using the show processes Command}}

As I've said, you've really got to be careful when using the
\texttt{debug} command on your devices. If your router's CPU utilization
is consistently at 50 percent or more, it's probably not a good idea to
type in the \texttt{debug\ all} command unless you want to see what a
router looks like when it crashes!

So what other approaches can you use? Well, the \texttt{show\ processes}
(or \texttt{show\ processes\ cpu}) is a good tool for determining a
given router's CPU utilization. Plus, it'll give you a list of active
processes along with their corresponding process ID, priority, scheduler
test (status), CPU time used, number of times invoked, and so on. Lots
of great stuff! Plus, this command is super handy when you want to
evaluate your router's performance and CPU utilization and are otherwise
tempted to reach for the \texttt{debug} command!

Okay---what do you see in the following output? The first line shows the
CPU utilization output for the last 5 seconds, 1 minute, and 5 minutes.
The output provides 5\%/0\% in front of the CPU utilization for the last
5 seconds: The first number equals the total
\protect\hypertarget{c07.xhtmlux5cux23Page_311}{}{}utilization, and the
second one indicates the utilization due to interrupt routines. Take a
look:

\begin{verbatim}
SW-1#sh processes
CPU utilization for five seconds: 5%/0%; one minute: 7%; five minutes: 8%
 PID QTy       PC Runtime(ms)   Invoked   uSecs    Stacks   TTY Process
   1 Cwe  29EBC58      0        22       0  5236/6000    0 Chunk Manager
   2 Csp  1B9CF10    241    206881       1  2516/3000    0 Load Meter
   3 Hwe  1F108D0      0         1       0  8768/9000    0 Connection Mgr
   4 Lst  29FA5C4 9437909    454026   20787  5540/6000   0 Check heaps
   5 Cwe  2A02468      0         2       0  5476/6000    0 Pool Manager
   6 Mst  1E98F04      0         2       0  5488/6000    0 Timers
   7 Hwe  13EB1B4   3686    101399      36  5740/6000    0 Net Input
   8 Mwe  13BCD84      0         1       0 23668/24000   0 Crash writer
   9 Mwe  1C591B4   4346     53691      80  4896/6000    0 ARP Input
  10 Lwe  1DA1504      0         1       0  5760/6000    0 CEF MIB API
  11 Lwe  1E76ACC      0         1       0  5764/6000    0 AAA_SERVER_DEADT
  12 Mwe  1E6F980      0         2       0  5476/6000    0 AAA high-capacit
  13 Mwe  1F56F24      0         1       0 11732/12000   0 Policy Manager [output cut]
\end{verbatim}

So basically, the output from the \texttt{show\ processes} command
reveals that our router is happily able to process debugging commands
without being overloaded---nice!

\subsection[Summary]{\texorpdfstring{\protect\hypertarget{c07.xhtmlux5cux23c07-sec-30}{}{}Summary}{Summary}}

In this chapter, you learned how Cisco routers are configured and how to
manage those configurations.

We covered the internal components of a router, including ROM, RAM,
NVRAM, and flash.

Next, you found out how to back up and restore the configuration of a
Cisco router and switch.

You also learned how to use CDP and Telnet to gather information about
remote devices. Finally, you discovered how to resolve hostnames and use
the \texttt{ping} and \texttt{trace} commands to test network
connectivity as well as how to use the \texttt{debug} and
\texttt{show\ processes} commands---well done!

\subsection[Exam
Essentials]{\texorpdfstring{\protect\hypertarget{c07.xhtmlux5cux23c07-sec-31}{}{}Exam
Essentials}{Exam Essentials}}

\textbf{Define the Cisco router components.} Describe the functions of
the bootstrap, POST, ROM monitor, mini-IOS, RAM, ROM, flash memory,
NVRAM, and the configuration register.

\protect\hypertarget{c07.xhtmlux5cux23Page_312}{}{}\textbf{Identify the
steps in the router boot sequence.} The steps in the boot sequence are
POST, loading the IOS, and copying the startup configuration from NVRAM
to RAM.

\textbf{Save the configuration of a router or switch.} There are a
couple of ways to do this, but the most common method, as well as the
most tested, is \texttt{copy\ running-config\ startup-config}.

\textbf{Erase the configuration of a router or switch.} Type the
privileged-mode command \texttt{erase\ startup-config} and reload the
router.

\textbf{Understand the various levels of syslog.} It's rather simple to
configure syslog; however, there are a bunch of options you have to
remember for the exam. To configure basic syslog with \texttt{debugging}
as the default level, it's just this one command:

\begin{verbatim}
SF(config)#logging 172.16.10.1
\end{verbatim}

However, you must remember all eight options:

\begin{verbatim}
SF(config)#logging trap ?
  <0-7>          Logging severity level
  alerts         Immediate action needed           (severity=1)
  critical       Critical conditions               (severity=2)
  debugging      Debugging messages                (severity=7)
  emergencies    System is unusable                (severity=0)
  errors         Error conditions                  (severity=3)
  informational  Informational messages            (severity=6)
  notifications  Normal but significant conditions (severity=5)
  warnings       Warning conditions                (severity=4)
  <cr>
\end{verbatim}

\textbf{Understand how to configure NTP.} It's pretty simple to
configure NTP, just like it was syslog, but we don't have to remember a
bunch of options! It's just telling the syslog to mark the time and date
and enabling NTP:

\begin{verbatim}
SF(config)#service timestamps log datetime msec
SF(config)#ntp server 172.16.10.1 version 4
\end{verbatim}

\textbf{Describe the value of CDP and LLDP.} Cisco Discovery Protocol
can be used to help you document as well as troubleshoot your network;
also, LLDP is a nonproprietary protocol that can provide the same
information as CDP.

\textbf{List the information provided by the output of
the}\texttt{show\ cdp\ neighbors}\textbf{command.} The
\texttt{show\ cdp\ neighbors} command provides the following
information: device ID, local interface, holdtime, capability, platform,
and port ID (remote interface).

\textbf{Understand how to establish a Telnet session with multiple
routers simultaneously.} If you telnet to a router or switch, you can
end the connection by typing \texttt{exit} at any time.
\protect\hypertarget{c07.xhtmlux5cux23Page_313}{}{}However, if you want
to keep your connection to a remote device but still come back to your
original router console, you can press the Ctrl+Shift+6 key combination,
release it, and then press X.

\textbf{Identify current Telnet sessions.} The command
\texttt{show\ sessions} will provide you with information about all the
currently active sessions your router has with other routers.

\textbf{Build a static host table on a router.} By using the global
configuration command \texttt{ip\ host}
\texttt{host\_name\ ip\_address}, you can build a static host table on
your router. You can apply multiple IP addresses against the same host
entry.

\textbf{Verify the host table on a router.} You can verify the host
table with the \texttt{show\ hosts} command.

\textbf{Describe the function of the}\texttt{ping}\textbf{command.}
Packet Internet Groper (\texttt{ping}) uses ICMP echo requests and ICMP
echo replies to verify an active IP address on a network.

\textbf{Ping a valid host ID from the correct prompt.} You can ping an
IP address from a router's user mode or privileged mode but not from
configuration mode, unless you use the \texttt{do} ­command. You must
ping a valid address, such as 1.1.1.1.

\subsection[Written Labs
7]{\texorpdfstring{\protect\hypertarget{c07.xhtmlux5cux23c07-sec-32}{}{}Written
Labs 7}{Written Labs 7}}

In this section, you'll complete the following labs to make sure you've
got the information and concepts contained within them fully dialed in:

\begin{enumerate}
\tightlist
\item
  Lab 7.1: IOS Management
\item
  Lab 7.2: Router Memory
\end{enumerate}

You can find the answers to these labs in Appendix A, ``Answers to
Written Labs.''

\subsubsection[Written Lab 7.1: IOS
Management]{\texorpdfstring{\protect\hypertarget{c07.xhtmlux5cux23c07-sec-33}{}{}Written
Lab 7.1: IOS Management}{Written Lab 7.1: IOS Management}}

Write the answers to the following questions:

\begin{enumerate}
\tightlist
\item
  What is the command to copy the startup-config file to DRAM?
\item
  What command can you use to see the neighbor router's IP address from
  your router prompt?
\item
  What command can you use to see the hostname, local interface,
  platform, and remote port of a neighbor router?
\item
  What keystrokes can you use to telnet into multiple devices
  simultaneously?
\item
  What command will show you your active Telnet connections to neighbor
  and remote devices?
\item
  What command can you use to merge a backup configuration with the
  configuration in RAM?
\item
  \protect\hypertarget{c07.xhtmlux5cux23Page_314}{}{}What protocol can
  be used on a network to synchronize clock and date information?
\item
  What command is used by a router to forward a DHCP client request to a
  remote DHCP server?
\item
  What command enables your switch or router to receive clock and date
  information and synchronize with the NTP server?
\item
  Which NTP verification command will show the reference master for the
  client?
\end{enumerate}

\subsubsection[Written Lab 7.2: Router
Memory]{\texorpdfstring{\protect\hypertarget{c07.xhtmlux5cux23c07-sec-34}{}{}Written
Lab 7.2: Router Memory}{Written Lab 7.2: Router Memory}}

Identify the location in a router where each of the following files is
stored by default.

\begin{enumerate}
\tightlist
\item
  Cisco IOS
\item
  Bootstrap
\item
  Startup configuration
\item
  POST routine
\item
  Running configuration
\item
  ARP cache
\item
  Mini-IOS
\item
  ROM Monitor
\item
  Routing tables
\item
  Packet buffers
\end{enumerate}

\subsection[Hands-on
Labs]{\texorpdfstring{\protect\hypertarget{c07.xhtmlux5cux23c07-sec-35}{}{}Hands-on
Labs}{Hands-on Labs}}

To complete the labs in this section, you need at least one router or
switch (three would be best) and at least one PC running as a TFTP
server. TFTP server software must be installed and running on the PC.
For this lab, it is also assumed that your PC and the Cisco devices are
connected together with a switch and that all interfaces (PC NIC and
router interfaces) are in the same subnet. You can alternately connect
the PC directly to the router or connect the routers directly to one
another (use a crossover cable in that case). Remember that the labs
listed here were created for use with real routers but can easily be
used with the LammleSim IOS Version (see \texttt{www.lammle.com/ccna})
or you can use the Cisco Packet Tracer router simulator. Last, although
it doesn't matter if you are using a switch or router in these labs, I'm
just going to use my routers, but feel free to use your switch to go
through these labs!

Here is a list of the labs in this chapter:

\begin{enumerate}
\tightlist
\item
  Lab 7.1: Backing Up the Router Configuration
\item
  Lab 7.2: Using the Cisco Discovery Protocol (CDP)
\item
  \protect\hypertarget{c07.xhtmlux5cux23Page_315}{}{}Lab 7.3: Using
  Telnet
\item
  Lab 7.4: Resolving Hostnames
\end{enumerate}

\subsubsection[Hands-on Lab 7.1: Backing Up the Router
Configuration]{\texorpdfstring{\protect\hypertarget{c07.xhtmlux5cux23c07-sec-36}{}{}Hands-on
Lab 7.1: Backing Up the Router
Configuration}{Hands-on Lab 7.1: Backing Up the Router Configuration}}

In this lab, you'll back up the router configuration:

\begin{enumerate}
\item
  Log into your router and go into privileged mode by typing \texttt{en}
  or \texttt{enable}.
\item
  Ping the TFTP server to make sure you have IP connectivity.
\item
  From RouterB, type \texttt{copy\ run\ tftp}.
\item
  When prompted, type the IP address of the TFTP server (for example,
  172.16.30.2) and press Enter.
\item
  By default, the router will prompt you for a filename. The hostname of
  the router is followed by the suffix \texttt{-confg} (yes, I spelled
  that correctly). You can use any name you want.

\begin{verbatim}
Name of configuration file to write [RouterB-confg]?
\end{verbatim}

  Press Enter to accept the default name.

\begin{verbatim}
Write file RouterB-confg on host 172.16.30.2? [confirm]
\end{verbatim}

  Press Enter to confirm.
\end{enumerate}

\subsubsection[Hands-on Lab 7.2: Using the Cisco Discovery Protocol
(CDP)]{\texorpdfstring{\protect\hypertarget{c07.xhtmlux5cux23c07-sec-37}{}{}Hands-on
Lab 7.2: Using the Cisco Discovery Protocol
(CDP)}{Hands-on Lab 7.2: Using the Cisco Discovery Protocol (CDP)}}

CDP is an important objective for the Cisco exams. Please go through
this lab and use CDP as much as possible during your studies.

\begin{enumerate}
\item
  Log into your router and go into privileged mode by typing \texttt{en}
  or \texttt{enable}.
\item
  From the router, type \texttt{sh\ cdp} and press Enter. You should see
  that CDP packets are being sent out to all active interfaces every 60
  seconds and the holdtime is 180 seconds (these are the defaults).
\item
  To change the CDP update frequency to 90 seconds, type
  \texttt{cdp\ timer\ 90} in global configuration mode.

\begin{verbatim}
Router#config t
Enter configuration commands, one per line.  End with
  CNTL/Z.
Router(config)#cdp timer ?
  <5-900>  Rate at which CDP packets are sent (in sec)
Router(config)#cdp timer 90
\end{verbatim}
\item
  \protect\hypertarget{c07.xhtmlux5cux23Page_316}{}{}Verify that your
  CDP timer frequency has changed by using the command
  \texttt{show\ cdp} in privileged mode.

\begin{verbatim}
Router#sh cdp
Global CDP information:
Sending CDP packets every 90 seconds
Sending a holdtime value of 180 seconds
\end{verbatim}
\item
  Now use CDP to gather information about neighbor routers. You can get
  the list of available commands by typing \texttt{sh\ cdp\ ?}.

\begin{verbatim}
Router#sh cdp ?
  entry     Information for specific neighbor entry
  interface CDP interface status and configuration
  neighbors CDP neighbor entries
  traffic   CDP statistics
  <cr>
\end{verbatim}
\item
  Type \texttt{sh\ cdp\ int} to see the interface information plus the
  default encapsulation used by the interface. It also shows the CDP
  timer information.
\item
  Type \texttt{sh\ cdp\ entry\ *} to see complete CDP information
  received from all devices.
\item
  Type \texttt{show\ cdp\ neighbors} to gather information about all
  connected neighbors. (You should know the specific information output
  by this command.)
\item
  Type \texttt{show\ cdp\ neighbors\ detail}. Notice that it produces
  the same output as \texttt{show\ cdp\ entry\ *}.
\end{enumerate}

\subsubsection[Hands-on Lab 7.3: Using
Telnet]{\texorpdfstring{\protect\hypertarget{c07.xhtmlux5cux23c07-sec-38}{}{}Hands-on
Lab 7.3: Using Telnet}{Hands-on Lab 7.3: Using Telnet}}

Secure Shell was covered in Chapter 6, and it is what you should use for
remote access into a Cisco device. However, the Cisco objectives cover
Telnet configuration, so let's do a lab on Telnet!

\begin{enumerate}
\tightlist
\item
  Log into your router and go into privileged mode by typing \texttt{en}
  or \texttt{enable}.
\item
  From RouterA, telnet into your remote router (RouterB) by typing
  \texttt{telnet}\texttt{ip\_address} from the command prompt. Type
  \texttt{exit} to disconnect.
\item
  Now type in RouterB's IP address from RouterA's command prompt. Notice
  that the router automatically tries to telnet to the IP address you
  specified. You can use the \texttt{telnet} command or just type in the
  IP address.
\item
  From RouterB, press Ctrl+Shift+6 and then X to return to RouterA's
  command prompt. Now telnet into your third router, RouterC. Press
  Ctrl+Shift+6 and then X to return to RouterA.
\item
  From RouterA, type \texttt{show\ sessions}. Notice your two sessions.
  You can press the number displayed to the left of the session and
  press Enter twice to return to that session. The asterisk shows the
  default session. You can press Enter twice to return to that session.
\item
  \protect\hypertarget{c07.xhtmlux5cux23Page_317}{}{}Go to the session
  for your RouterB. Type \texttt{show\ users}. This shows the console
  connection and the remote connection. You can use the
  \texttt{disconnect} command to clear the session or just type
  \texttt{exit} from the prompt to close your session with RouterB.
\item
  Go to RouterC's console port by typing \texttt{show\ sessions} on the
  first router and using the connection number to return to RouterC.
  Type \texttt{show\ user} and notice the connection to your first
  router, RouterA.
\item
  Type \texttt{clear\ line}\texttt{} \texttt{line\_number} to disconnect
  the Telnet session.
\end{enumerate}

\subsubsection[Hands-on Lab 7.4: Resolving
Hostnames]{\texorpdfstring{\protect\hypertarget{c07.xhtmlux5cux23c07-sec-39}{}{}Hands-on
Lab 7.4: Resolving Hostnames}{Hands-on Lab 7.4: Resolving Hostnames}}

It's best to use a DNS server for name resolution, but you can also
create a local hosts table to resolve names. Let's take a look.

\begin{enumerate}
\item
  Log into your router and go into privileged mode by typing \texttt{en}
  or \texttt{enable}.
\item
  From RouterA, type \texttt{todd} and press Enter at the command
  prompt. Notice the error you receive and the delay. The router is
  trying to resolve the hostname to an IP address by looking for a DNS
  server. You can turn this feature off by using the
  \texttt{no\ ip\ domain-lookup} command from global configuration mode.
\item
  To build a host table, you use the \texttt{ip\ host} command. From
  RouterA, add a host table entry for RouterB and RouterC by entering
  the following commands:

\begin{verbatim}
ip host routerbip_address
ip host routerc ip_address
\end{verbatim}

  Here is an example:

\begin{verbatim}
ip host routerb 172.16.20.2
ip host routerc 172.16.40.2
\end{verbatim}
\item
  Test your host table by typing \texttt{ping\ routerb} from the
  privileged mode prompt (not the \texttt{config} prompt).

\begin{verbatim}
RouterA#ping routerb
Type escape sequence to abort.
Sending 5, 100-byte ICMP Echos to 172.16.20.2, timeout
  is 2 seconds:
!!!!!
Success rate is 100 percent (5/5), round-trip
  min/avg/max = 4/4/4 ms
\end{verbatim}
\item
  Test your host table by typing \texttt{ping\ routerc}.

\begin{verbatim}
RouterA#ping routerc
Type escape sequence to abort.
Sending 5, 100-byte ICMP Echos to 172.16.40.2, timeout
  is 2 seconds:
!!!!!
Success rate is 100 percent (5/5), round-trip
  min/avg/max = 4/6/8 ms
\end{verbatim}
\item
  Telnet to RouterB and keep your session to RouterB open to RouterA by
  pressing Ctrl+Shift+6, then X.
\item
  Telnet to RouterC by typing \texttt{routerc} at the command prompt.
\item
  Return to RouterA and keep the session to RouterC open by pressing
  Ctrl+Shift+6, then X.
\item
  View the host table by typing \texttt{show\ hosts} and pressing Enter.

\begin{verbatim}
Default domain is not set
Name/address lookup uses domain service
Name servers are 255.255.255.255
Host                 Flags      Age Type   Address(es)
routerb             (perm, OK)  0   IP    172.16.20.2
routerc             (perm, OK)  0   IP    172.16.40.2
\end{verbatim}
\end{enumerate}

\subsection[Review
Questions]{\texorpdfstring{\protect\hypertarget{c07.xhtmlux5cux23c07-sec-40}{}{}\protect\hypertarget{c07.xhtmlux5cux23Page_319}{}{}Review
Questions}{Review Questions}}

\begin{center}\rule{0.5\linewidth}{0.5pt}\end{center}

\includegraphics{images/note.png} The following questions are designed
to test your understanding of this chapter's material. For more
information on how to get additional questions, please see
\texttt{www.lammle.com/ccna}.

\begin{center}\rule{0.5\linewidth}{0.5pt}\end{center}

You can find the answers to these questions in Appendix B, ``Answers to
Review Questions.''

\begin{enumerate}
\item
  Which of the following is a standards-based protocol that provides
  dynamic network discovery?

  \begin{enumerate}
  \tightlist
  \item
    DHCP
  \item
    LLDP
  \item
    DDNS
  \item
    SSTP
  \item
    CDP
  \end{enumerate}
\item
  Which command can be used to determine a router's CPU utilization?

  \begin{enumerate}
  \tightlist
  \item
    \texttt{show\ version}
  \item
    \texttt{show\ controllers}
  \item
    \texttt{show\ processes\ cpu}
  \item
    \texttt{show\ memory}
  \end{enumerate}
\item
  You are troubleshooting a connectivity problem in your corporate
  network and want to isolate the problem. You suspect that a router on
  the route to an unreachable network is at fault. What IOS
  \texttt{user\ exec} command should you issue?

  \begin{enumerate}
  \tightlist
  \item
    \texttt{Router\textgreater{}ping}
  \item
    \texttt{Router\textgreater{}trace}
  \item
    \texttt{Router\textgreater{}show\ ip\ route}
  \item
    \texttt{Router\textgreater{}show\ interface}
  \item
    \texttt{Router\textgreater{}show\ cdp\ neighbors}
  \end{enumerate}
\item
  You copy a configuration from a network host to a router's RAM. The
  configuration looks correct, yet it is not working at all. What could
  the problem be?

  \begin{enumerate}
  \tightlist
  \item
    You copied the wrong configuration into RAM.
  \item
    You copied the configuration into flash memory instead.
  \item
    The copy did not override the \texttt{shutdown} command in
    running-config.
  \item
    The IOS became corrupted after the \texttt{copy} command was
    initiated.
  \end{enumerate}
\item
  In the following command, what does the IP address 10.10.10.254 refer
  to?

\begin{verbatim}
Router#config t
Router(config)#interface fa0/0
Router(config-if)#ip helper-address 10.10.10.254
\end{verbatim}

  \begin{enumerate}
  \tightlist
  \item
    \protect\hypertarget{c07.xhtmlux5cux23Page_320}{}{}IP address of the
    ingress interface on the router
  \item
    IP address of the egress interface on the router
  \item
    IP address of the next hop on the path to the DHCP server
  \item
    IP address of the DHCP server
  \end{enumerate}
\item
  The corporate office sends you a new router to connect, but upon
  connecting the console cable, you see that there is already a
  configuration on the router. What should be done before a new
  configuration is entered in the router?

  \begin{enumerate}
  \tightlist
  \item
    RAM should be erased and the router restarted.
  \item
    Flash should be erased and the router restarted.
  \item
    NVRAM should be erased and the router restarted.
  \item
    The new configuration should be entered and saved.
  \end{enumerate}
\item
  What command can you use to determine the IP address of a directly
  connected neighbor?

  \begin{enumerate}
  \tightlist
  \item
    \texttt{show\ cdp}
  \item
    \texttt{show\ cdp\ neighbors}
  \item
    \texttt{show\ cdp\ neighbors\ detail}
  \item
    \texttt{show\ neighbor\ detail}
  \end{enumerate}
\item
  According to the output, what interface does SW-2 use to connect to
  SW-3?

\begin{verbatim}
SW-3#sh cdp neighbors
Capability Codes: R - Router, T - Trans Bridge, B - Source Route BridgeS - Switch, H - Host, I - IGMP, r - Repeater, P - Phone, D - Remote, C - CVTA, M - Two-port Mac Relay Device ID
Local Intrfce     Holdtme    Capability  Platform  Port ID
SW-1   Fas 0/1      170          S I     WS-C3560- Fas 0/15
SW-1   Fas 0/2      170          S I     WS-C3560- Fas 0/16
SW-2   Fas 0/5      162          S I     WS-C3560- Fas 0/2
\end{verbatim}

  \begin{enumerate}
  \tightlist
  \item
    Fas 0/1
  \item
    Fas 0/16
  \item
    Fas 0/2
  \item
    Fas 0/5
  \end{enumerate}
\item
  Which of the following commands enables syslog on a Cisco device with
  \texttt{debugging} as the level?

  \begin{enumerate}
  \tightlist
  \item
    \texttt{syslog\ 172.16.10.1}
  \item
    \texttt{logging\ 172.16.10.1}
  \item
    \texttt{remote\ console\ 172.16.10.1\ syslog\ debugging}
  \item
    \texttt{transmit\ console\ messages\ level\ 7\ 172.16.10.1}
  \end{enumerate}
\item
  \protect\hypertarget{c07.xhtmlux5cux23Page_321}{}{}You save the
  configuration on a router with the
  \texttt{copy\ running-config\ startup-config} command and reboot the
  router. The router, however, comes up with a blank configuration. What
  can the problem be?

  \begin{enumerate}
  \tightlist
  \item
    You didn't boot the router with the correct command.
  \item
    NVRAM is corrupted.
  \item
    The configuration register setting is incorrect.
  \item
    The newly upgraded IOS is not compatible with the hardware of the
    router.
  \item
    The configuration you saved is not compatible with the hardware.
  \end{enumerate}
\item
  If you want to have more than one Telnet session open at the same
  time, what keystroke combination would you use?

  \begin{enumerate}
  \tightlist
  \item
    Tab+spacebar
  \item
    Ctrl+X, then 6
  \item
    Ctrl+Shift+X, then 6
  \item
    Ctrl+Shift+6, then X
  \end{enumerate}
\item
  You are unsuccessful in telnetting into a remote device from your
  switch, but you could telnet to the router earlier. However, you can
  still ping the remote device. What could the problem be? (Choose two.)

  \begin{enumerate}
  \tightlist
  \item
    IP addresses are incorrect.
  \item
    Access control list is filtering Telnet.
  \item
    There is a defective serial cable.
  \item
    The VTY password is missing.
  \end{enumerate}
\item
  What information is displayed by the \texttt{show\ hosts} command?
  (Choose two.)

  \begin{enumerate}
  \tightlist
  \item
    Temporary DNS entries
  \item
    The names of the routers created using the \texttt{hostname} command
  \item
    The IP addresses of workstations allowed to access the router
  \item
    Permanent name-to-address mappings created using the
    \texttt{ip\ host} command
  \item
    The length of time a host has been connected to the router via
    Telnet
  \end{enumerate}
\item
  Which three commands can be used to check LAN connectivity problems on
  an enterprise switch? (Choose three.)

  \begin{enumerate}
  \tightlist
  \item
    \texttt{show\ interfaces}
  \item
    \texttt{show\ ip\ route}
  \item
    \texttt{tracert}
  \item
    \texttt{ping}
  \item
    \texttt{dns\ lookups}
  \end{enumerate}
\item
  What is the default syslog facility level?

  \begin{enumerate}
  \tightlist
  \item
    local4
  \item
    local5
  \item
    \protect\hypertarget{c07.xhtmlux5cux23Page_322}{}{}local6
  \item
    local7
  \end{enumerate}
\item
  You telnet into a remote device and type \texttt{debug\ ip\ icmp}, but
  no output from the \texttt{debug} command is seen. What could the
  problem be?

  \begin{enumerate}
  \tightlist
  \item
    You must type the \texttt{show\ ip\ icmp} command first.
  \item
    IP addressing on the network is incorrect.
  \item
    You must use the \texttt{terminal\ monitor} command.
  \item
    Debug output is sent only to the console.
  \end{enumerate}
\item
  Which three statements about syslog utilization are true? (Choose
  three.)

  \begin{enumerate}
  \tightlist
  \item
    Utilizing syslog improves network performance.
  \item
    The syslog server automatically notifies the network administrator
    of network problems.
  \item
    A syslog server provides the storage space necessary to store log
    files without using router disk space.
  \item
    There are more syslog messages available within Cisco IOS than there
    are comparable SNMP trap messages.
  \item
    Enabling syslog on a router automatically enables NTP for accurate
    time stamping.
  \item
    A syslog server helps in aggregation of logs and alerts.
  \end{enumerate}
\item
  You need to gather the IP address of a remote switch that is located
  in Hawaii. What can you do to find the address?

  \begin{enumerate}
  \tightlist
  \item
    Fly to Hawaii, console into the switch, then relax and have a drink
    with an umbrella in it.
  \item
    Issue the \texttt{show\ ip\ route} command on the router connected
    to the switch.
  \item
    Issue the \texttt{show\ cdp\ neighbor} command on the router
    connected to the switch.
  \item
    Issue the \texttt{show\ ip\ arp} command on the router connected to
    the switch.
  \item
    Issue the \texttt{show\ cdp\ neighbors\ detail} command on the
    router connected to the switch.
  \end{enumerate}
\item
  You need to configure all your routers and switches so they
  synchronize their clocks from one time source. What command will you
  type for each device?

  \begin{enumerate}
  \tightlist
  \item
    \texttt{clock\ synchronization} \texttt{ip\_address}
  \item
    \texttt{ntp\ master\ ip\_address}
  \item
    \texttt{sync\ ntp\ ip\_address}
  \item
    \texttt{ntp\ server} \texttt{ip\_address}\texttt{\ version}
    \texttt{number}
  \end{enumerate}
\item
  A network administrator enters the following command on a router:
  \texttt{logging\ trap\ 3}. What are three message types that will be
  sent to the syslog server? (Choose three.)

  \begin{enumerate}
  \tightlist
  \item
    Informational
  \item
    Emergency
  \item
    Warning
  \item
    Critical
  \item
    Debug
  \item
    Error
  \end{enumerate}
\end{enumerate}

\protect\hypertarget{c08.xhtml}{}{}

\section[{Chapter 8}\\
{Managing Cisco
Devices}]{\texorpdfstring{\protect\hypertarget{c08.xhtmlux5cux23c08}{}{}\protect\hypertarget{c08.xhtmlux5cux23Page_323}{}{}{Chapter
8}\\
{Managing Cisco Devices}}{Chapter 8 Managing Cisco Devices}}

\begin{center}\rule{0.5\linewidth}{0.5pt}\end{center}

\subsection{The following ICND1 exam topics are covered in this
chapter:}

\begin{enumerate}
\tightlist
\item
  \includegraphics{images/tick.png} x\textbf{5.0 Infrastructure
  Management}

  \begin{enumerate}
  \tightlist
  \item
    \includegraphics{images/squ.png} 5.2 Configure and verify device
    management

    \begin{enumerate}
    \tightlist
    \item
      \includegraphics{images/squ.png} 5.2.c Licensing
    \end{enumerate}
  \item
    \includegraphics{images/squ.png} 5.5 Perform device maintenance

    \begin{enumerate}
    \tightlist
    \item
      \includegraphics{images/squ.png} 5.5.a Cisco IOS upgrades and
      recovery (SCP, FTP, TFTP, and MD5 verify)
    \item
      \includegraphics{images/squ.png} 5.5.b Password recovery and
      configuration register
    \item
      \includegraphics{images/squ.png} 5.5.c File system management
    \end{enumerate}
  \end{enumerate}
\end{enumerate}

\protect\hypertarget{c08.xhtmlux5cux23Page_324}{}{}\includegraphics{images/intro.png}Here
in Chapter 8, I'm going to show you how to ­manage Cisco routers on an
internetwork. The Internetwork Operating System (IOS) and configuration
files reside in ­different locations in a Cisco device, so it's really
important to understand both where these files are located and how they
work.

You'll be learning about the configuration register, including how to
use the ­configuration register for password recovery.

Finally, I'll cover how to verify licenses on the ISRG2 routers as well
as how to install a permanent license and configure evaluation features
in the latest universal images.

\begin{center}\rule{0.5\linewidth}{0.5pt}\end{center}

\includegraphics{images/note.png} To find up-to-the-minute updates for
this chapter, please see \texttt{www.lammle.com/ccna} or the book's web
page at \texttt{www.sybex.com/go/ccna}.

\begin{center}\rule{0.5\linewidth}{0.5pt}\end{center}

\subsection[Managing the Configuration
Register]{\texorpdfstring{\protect\hypertarget{c08.xhtmlux5cux23c08-sec-1}{}{}Managing
the Configuration Register}{Managing the Configuration Register}}

All Cisco routers have a 16-bit software register that's written into
NVRAM. By default, the \emph{configuration register} is set to load the
Cisco IOS from \emph{flash memory} and to look for and load the
startup-config file from NVRAM. In the following sections, I am going to
­discuss the configuration register settings and how to use these
settings to provide ­password recovery on your routers.

\subsubsection[Understanding the Configuration Register
Bits]{\texorpdfstring{\protect\hypertarget{c08.xhtmlux5cux23c08-sec-2}{}{}Understanding
the Configuration Register
Bits}{Understanding the Configuration Register Bits}}

The 16 bits (2 bytes) of the configuration register are read from 15 to
0, from left to right. The default configuration setting on Cisco
routers is 0x2102. This means that bits 13, 8, and 1 are on, as shown in
\protect\hyperlink{c08.xhtmlux5cux23table8-1}{Table 8.1}. Notice that
each set of 4 bits (called a nibble) is read in binary with a value of
8, 4, 2, 1.

{\protect\hyperlink{c08.xhtmlux5cux23tableanchor8-1}{\textbf{TABLE 8.1}}
The configuration register bit numbers}

Configuration Register

2

1

0

2

Bit number

15

14

13

12

11

10

9

8

7

6

5

4

3

2

1

0

Binary

0

0

1

0

0

0

0

1

0

0

0

0

0

0

1

0

\protect\hypertarget{c08.xhtmlux5cux23Page_325}{}{}

\begin{center}\rule{0.5\linewidth}{0.5pt}\end{center}

\includegraphics{images/note.png} Add the prefix \emph{0x} to the
configuration register address. The \emph{0x} means that the digits that
follow are in hexadecimal.

\begin{center}\rule{0.5\linewidth}{0.5pt}\end{center}

\protect\hyperlink{c08.xhtmlux5cux23table8-2}{Table 8.2} lists the
software configuration bit meanings. Notice that bit 6 can be used to
ignore the NVRAM contents. This bit is used for password
recovery---something I'll go over with you soon in the section
``Recovering Passwords,'' later in this chapter.

\begin{center}\rule{0.5\linewidth}{0.5pt}\end{center}

\includegraphics{images/note.png} Remember that in hex, the scheme is
0--9 and A--F (A = 10, B = 11, C = 12, D = 13, E = 14, and F = 15). This
means that a 210F setting for the configuration register is actually
210(15), or 1111 in binary.

\begin{center}\rule{0.5\linewidth}{0.5pt}\end{center}

{\protect\hyperlink{c08.xhtmlux5cux23tableanchor8-2}{\textbf{TABLE 8.2}}
Software configuration meanings}

\begin{longtable}[]{@{}lll@{}}
\toprule
Bit & Hex & Description\tabularnewline
\midrule
\endhead
0--3 & 0x0000--0x000F & Boot field (see
\protect\hyperlink{c08.xhtmlux5cux23table8-3}{Table
8.3}).\tabularnewline
6 & 0x0040 & Ignore NVRAM contents.\tabularnewline
7 & 0x0080 & OEM bit enabled.\tabularnewline
8 & 0x101 & Break disabled.\tabularnewline
10 & 0x0400 & IP broadcast with all zeros.\tabularnewline
5, 11--12 & 0x0800--0x1000 & Console line speed.\tabularnewline
13 & 0x2000 & Boot default ROM software if network boot
fails.\tabularnewline
14 & 0x4000 & IP broadcasts do not have net numbers.\tabularnewline
15 & 0x8000 & Enable diagnostic messages and ignore NVRAM
contents.\tabularnewline
\bottomrule
\end{longtable}

The boot field, which consists of bits 0--3 in the configuration
register (the last 4 bits), controls the router boot sequence and
locates the Cisco IOS.
\protect\hyperlink{c08.xhtmlux5cux23table8-3}{Table 8.3} describes the
boot field bits.

\protect\hypertarget{c08.xhtmlux5cux23Page_326}{}{}

{\protect\hyperlink{c08.xhtmlux5cux23tableanchor8-3}{\textbf{TABLE 8.3}}
The boot field (configuration register bits 00--03)}

\begin{longtable}[]{@{}lll@{}}
\toprule
Boot Field & Meaning & Use\tabularnewline
\midrule
\endhead
00 & ROM monitor mode & To boot to ROM monitor mode, set the
configuration register to 2100. You must manually boot the router with
the \texttt{b} command. The router will show the
\texttt{rommon\textgreater{}} prompt.\tabularnewline
01 & Boot image from ROM & To boot the mini-IOS image stored in ROM, set
the configuration register to 2101. The router will show the
\texttt{Router(boot)\textgreater{}} prompt. The mini-IOS is not
available in all routers and is also referred to as
RXBOOT.\tabularnewline
02--F & Specifies a default boot filename & Any value from 2102 through
210F tells the router to use the \texttt{boot} commands specified in
NVRAM.\tabularnewline
\bottomrule
\end{longtable}

\subsubsection[Checking the Current Configuration Register
Value]{\texorpdfstring{\protect\hypertarget{c08.xhtmlux5cux23c08-sec-3}{}{}Checking
the Current Configuration Register
Value}{Checking the Current Configuration Register Value}}

You can see the current value of the configuration register by using the
\texttt{show\ version} command (\texttt{sh\ version} or
\texttt{show\ ver} for short), as demonstrated here:

\begin{verbatim}
Router>sh version
Cisco IOS Software, 2800 Software (C2800NM-ADVSECURITYK9-M),
Version 15.1(4)M6, RELEASE SOFTWARE (fc2)
[output cut]
Configuration register is 0x2102
\end{verbatim}

The last information given from this command is the value of the
configuration register. In this example, the value is 0x2102---the
default setting. The configuration register setting of 0x2102 tells the
router to look in NVRAM for the boot sequence.

Notice that the \texttt{show\ version} command also provides the IOS
version, and in the preceding example, it shows the IOS version as
15.1(4)M6.

\begin{center}\rule{0.5\linewidth}{0.5pt}\end{center}

\includegraphics{images/note.png} The \texttt{show\ version} command
will display system hardware configuration information, system serial
number, the software verision, and the names of the boot images on a
router.

\begin{center}\rule{0.5\linewidth}{0.5pt}\end{center}

To change the configuration register, use the \texttt{config-register}
command from global configuration mode:

\begin{verbatim}
Router(config)#config-register 0x2142
Router(config)#do sh ver
[output cut]
Configuration register is 0x2102 (will be 0x2142 at next reload)
\end{verbatim}

It's important that you are careful when you set the configuration
register!

\begin{center}\rule{0.5\linewidth}{0.5pt}\end{center}

\includegraphics{images/note.png} If you save your configuration and
reload the router and it comes up in setup mode, the configuration
register setting is probably incorrect.

\begin{center}\rule{0.5\linewidth}{0.5pt}\end{center}

\subsubsection[Boot System
Commands]{\texorpdfstring{\protect\hypertarget{c08.xhtmlux5cux23c08-sec-4}{}{}Boot
System Commands}{Boot System Commands}}

Did you know that you can configure your router to boot another IOS if
the flash is corrupted? Well, you can. You can boot all of your routers
from a TFTP server, but it's old school, and people just don't do it
anymore; it's just for backup in case of failure.

There are some \texttt{boot} commands you can play with that will help
you manage the way your router boots the Cisco IOS---but please
remember, we're talking about the router's IOS here, \emph{not} the
router's configuration!

\begin{verbatim}
Router>en
Router#config t
Enter configuration commands, one per line.  End with CNTL/Z.
Router(config)#boot ?
  bootstrap  Bootstrap image file
  config     Configuration file
  host       Router-specific config file
  network    Network-wide config file
  system     System image file
\end{verbatim}

The \texttt{boot} command truly gives you a wealth of options, but
first, I'll show you the typical settings that Cisco recommends. So
let's get started---the \texttt{boot\ system} command will allow you to
tell the router which system IOS file to boot from flash memory.
Remember that the router, by default, boots the first system IOS file
found in flash. You can change that with the following commands, as
shown in the output:

\begin{verbatim}
Router(config)#boot system ?
  WORD   TFTP filename or URL
  flash  Boot from flash memory
  ftp    Boot from a server via ftp
  mop    Boot from a Decnet MOP server
  rcp    Boot from a server via rcp
  rom    Boot from rom
  tftp   Boot from a tftp server
Router(config)#boot system flash c2800nm-advsecurityk9-mz.151-4.M6.bin
\end{verbatim}

\protect\hypertarget{c08.xhtmlux5cux23Page_328}{}{}Notice I could boot
from FLASH, FTP, ROM, TFTP, or another useless options. The command I
used configures the router to boot the IOS listed in it. This is a
helpful command for when you load a new IOS into flash and want to test
it, or even when you want to totally change which IOS is loading by
default.

The next command is considered a fallback routine, but as I said, you
can make it a permanent way to have your routers boot from a TFTP host.
Personally, I wouldn't necessarily recommend doing this (single point of
failure); I'm just showing you that it's possible:

\begin{verbatim}
Router(config)#boot system tftp ?
  WORD  System image filename
Router(config)#boot system tftp c2800nm-advsecurityk9-mz.151-4.M6.bin?
  Hostname or A.B.C.D  Address from which to download the file
  <cr>
Router(config)#boot system tftp c2800nm-advsecurityk9-mz.151-4.M6.bin 1.1.1.2
Router(config)#
\end{verbatim}

As your last recommended fallback option---the one to go to if the IOS
in flash doesn't load and the TFTP host does not produce the IOS---load
the mini-IOS from ROM like this:

\begin{verbatim}
Router(config)#boot system rom
Router(config)#do show run | include boot system
boot system flash c2800nm-advsecurityk9-mz.151-4.M6.bin
boot system tftp c2800nm-advsecurityk9-mz.151-4.M6.bin 1.1.1.2
boot system rom
Router(config)#
\end{verbatim}

If the preceding configuration is set, the router will try to boot from
the TFTP server if flash fails, and if the TFTP boot fails, the mini-IOS
will load after six unsuccessful attempts of trying to locate the TFTP
server.

In the next section, I'll show you how to load the router into ROM
monitor mode so you can perform password recovery.

\subsubsection[Recovering
Passwords]{\texorpdfstring{\protect\hypertarget{c08.xhtmlux5cux23c08-sec-5}{}{}Recovering
Passwords}{Recovering Passwords}}

If you're locked out of a router because you forgot the password, you
can change the ­configuration register to help you get back on your
feet. As I said earlier, bit 6 in the ­configuration register is used to
tell the router whether to use the contents of NVRAM to load a router
configuration.

The default configuration register value is 0x2102, meaning that bit 6
is off. With the default setting, the router will look for and load a
router configuration stored in NVRAM (startup-config). To recover a
password, you need to turn on bit 6. Doing this will tell the router to
ignore the NVRAM contents. The configuration register value to turn on
bit 6 is 0x2142.

\protect\hypertarget{c08.xhtmlux5cux23Page_329}{}{}Here are the main
steps to password recovery:

\begin{enumerate}
\tightlist
\item
  Boot the router and interrupt the boot sequence by performing a break,
  which will take the router into ROM monitor mode.
\item
  Change the configuration register to turn on bit 6 (with the value
  0x2142).
\item
  Reload the router.
\item
  Say ``no'' to entering setup mode, then enter privileged mode.
\item
  Copy the startup-config file to running-config, and don't forget to
  verify that your interfaces are re-enabled.
\item
  Change the password.
\item
  Reset the configuration register to the default value.
\item
  Save the router configuration.
\item
  Reload the router (optional).
\end{enumerate}

I'm going to cover these steps in more detail in the following sections.
I'll also show you the commands to restore access to ISR series routers.

You can enter ROM monitor mode by pressing Ctrl+Break or Ctrl+Shift+6,
then b, ­during router bootup. But if the IOS is corrupt or missing, if
there's no network ­connectivity available to find a TFTP host, or if
the mini-IOS from ROM doesn't load (meaning the default router fallback
failed), the router will enter ROM monitor mode by default.

\paragraph{Interrupting the Router Boot Sequence}

Your first step is to boot the router and perform a break. This is
usually done by pressing the Ctrl+Break key combination when using
HyperTerminal (personally, I use SecureCRT or PuTTY) while the router
first reboots.

\begin{verbatim}
System Bootstrap, Version 15.1(4)M6, RELEASE SOFTWARE (fc2)
Copyright (c) 1999 by cisco Systems, Inc.
TAC:Home:SW:IOS:Specials for info
PC = 0xfff0a530, Vector = 0x500, SP = 0x680127b0
C2800 platform with 32768 Kbytes of main memory
PC = 0xfff0a530, Vector = 0x500, SP = 0x80004374
monitor: command “boot” aborted due to user interrupt
rommon 1 >
\end{verbatim}

Notice the line
\texttt{monitor:\ command\ “boot”\ aborted\ due\ to\ user\ interrupt}.
At this point, you will be at the \texttt{rommon\ 1\textgreater{}}
prompt, which is called the ROM monitor mode.

\paragraph{Changing the Configuration Register}

As I explained earlier, you can change the configuration register from
within the IOS by using the \texttt{config-register} command. To turn on
bit 6, use the configuration register value 0x2142.

\protect\hypertarget{c08.xhtmlux5cux23Page_330}{}{}

\begin{center}\rule{0.5\linewidth}{0.5pt}\end{center}

\includegraphics{images/note.png} Remember that if you change the
configuration register to 0x2142, the startup-config will be bypassed
and the router will load into setup mode.

\begin{center}\rule{0.5\linewidth}{0.5pt}\end{center}

To change the bit value on a Cisco ISR series router, you just enter the
following command at the \texttt{rommon\ 1\textgreater{}} prompt:

\begin{verbatim}
rommon 1 >confreg 0x2142
You must reset or power cycle for new config to take effect
rommon 2 >reset
\end{verbatim}

\paragraph{Reloading the Router and Entering Privileged Mode}

At this point, you need to reset the router like this:

\begin{enumerate}
\tightlist
\item
  From the ISR series router, type \texttt{I} (for initialize) or
  \texttt{reset}.
\item
  From an older series router, type \texttt{I}.
\end{enumerate}

The router will reload and ask if you want to use setup mode (because no
startup-config is used). Answer no to entering setup mode, press Enter
to go into user mode, and then type \texttt{enable} to go into
privileged mode.

\paragraph{Viewing and Changing the Configuration}

Now you're past the point where you would need to enter the user-mode
and privileged-mode passwords in a router. Copy the startup-config file
to the running-config file:

\begin{verbatim}
copy startup-config running-config
\end{verbatim}

Or use the shortcut:

\begin{verbatim}
copy start run
\end{verbatim}

The configuration is now running in \emph{random access memory (RAM)},
and you're in privileged mode, meaning that you can now view and change
the configuration. But you can't view the enable secret setting for the
password since it is encrypted. To change the password, do this:

\begin{verbatim}
config t
enable secret todd
\end{verbatim}

\paragraph{Resetting the Configuration Register and Reloading the
Router}

After you're finished changing passwords, set the configuration register
back to the default value with the \texttt{config-register} command:

\begin{verbatim}
config t
config-register 0x2102
\end{verbatim}

It's important to remember to enable your interfaces after copying the
configuration from NVRAM to RAM.

\protect\hypertarget{c08.xhtmlux5cux23Page_331}{}{}Finally, save the new
configuration with a \texttt{copy\ running-config\ startup-config} and
use \texttt{reload} to reload the router.

\begin{center}\rule{0.5\linewidth}{0.5pt}\end{center}

\includegraphics{images/note.png} If you save your configuration and
reload the router and it comes up in setup mode, the configuration
register setting is probably incorrect.

\begin{center}\rule{0.5\linewidth}{0.5pt}\end{center}

To sum this up, we now have Cisco's suggested IOS backup routine
configured on our router: flash, TFTP host, ROM.

\subsection[Backing Up and Restoring the Cisco
IOS]{\texorpdfstring{\protect\hypertarget{c08.xhtmlux5cux23c08-sec-6}{}{}Backing
Up and Restoring the Cisco IOS}{Backing Up and Restoring the Cisco IOS}}

Before you upgrade or restore a Cisco IOS, you really should copy the
existing file to a \emph{TFTP host} as a backup just in case the new
image crashes and burns.

And you can use any TFTP host to accomplish this. By default, the flash
memory in a router is used to store the Cisco IOS. In the following
sections, I'll describe how to check the amount of flash memory, how to
copy the Cisco IOS from flash memory to a TFTP host, and how to copy the
IOS from a TFTP host to flash memory.

But before you back up an IOS image to a network server on your
intranet, you've got to do these three things:

\begin{enumerate}
\tightlist
\item
  Make sure you can access the network server.
\item
  Ensure that the network server has adequate space for the code image.
\item
  Verify the file naming and path requirements.
\end{enumerate}

You can connect your laptop or workstation's Ethernet port directly to a
router's Ethernet interface, as shown in
\protect\hyperlink{c08.xhtmlux5cux23figure8-1}{Figure 8.1}.

\begin{figure}
\centering
\includegraphics{images/c08f001.jpg}
\caption{{\protect\hyperlink{c08.xhtmlux5cux23figureanchor8-1}{\textbf{FIGURE
8.1}} Copying an IOS from a router to a TFTP host}}
\end{figure}

\protect\hypertarget{c08.xhtmlux5cux23Page_332}{}{}You need to verify
the following before attempting to copy the image to or from the router:

\begin{enumerate}
\tightlist
\item
  TFTP server software must be running on the laptop or workstation.
\item
  The Ethernet connection between the router and the workstation must be
  made with a crossover cable.
\item
  The workstation must be on the same subnet as the router's Ethernet
  interface.
\item
  The \texttt{copy\ flash\ tftp} command must be supplied the IP address
  of the workstation if you are copying from the router flash.
\item
  And if you're copying ``into'' flash, you need to verify that there's
  enough room in flash memory to accommodate the file to be copied.
\end{enumerate}

\subsubsection[Verifying Flash
Memory]{\texorpdfstring{\protect\hypertarget{c08.xhtmlux5cux23c08-sec-7}{}{}Verifying
Flash Memory}{Verifying Flash Memory}}

Before you attempt to upgrade the Cisco IOS on your router with a new
IOS file, it's a good idea to verify that your flash memory has enough
room to hold the new image. You verify the amount of flash memory and
the file or files being stored in flash memory by using the
\texttt{show\ flash} command (\texttt{sh\ flash} for short):

\begin{verbatim}
Router#sh flash
-#- --length-- -----date/time------ path
1    45392400 Apr 14 2013 05:31:44 +00:00 c2800nm-advsecurityk9-mz.151-4.M6.bin
 
18620416 bytes available (45395968 bytes used)
\end{verbatim}

There are about 45 MB of flash used, but there are still about 18 MB
available. If you want to copy a file into flash that is more than 18 MB
in size, the router will ask you if you want to erase flash. Be careful
here!

\begin{center}\rule{0.5\linewidth}{0.5pt}\end{center}

\includegraphics{images/note.png} The \texttt{show\ flash} command will
display the amount of memory consumed by the current IOS image as well
as tell you if there's enough room available to hold both current and
new images. You should know that if there's not enough room for both the
old and new image you want to load, the old image will be erased!

\begin{center}\rule{0.5\linewidth}{0.5pt}\end{center}

The amount of RAM and flash is actually easy to tally using the
\texttt{show\ version} command on routers:

\begin{verbatim}
Router#show version
[output cut]
System returned to ROM by power-on
System image file is "flash:c2800nm-advsecurityk9-mz.151-4.M6.bin"
[output cut]
Cisco 2811 (revision 1.0) with 249856K/12288K bytes of memory.
Processor board ID FTX1049A1AB
2 FastEthernet interfaces
2 Serial(sync/async) interfaces
1 Virtual Private Network (VPN) Module
DRAM configuration is 64 bits wide with parity enabled.
239K bytes of non-volatile configuration memory.
62720K bytes of ATA CompactFlash (Read/Write)
\end{verbatim}

The second highlighted line shows us that this router has about 256 MB
of RAM, and you can see that the amount of flash shows up on the last
line. By estimating up, we get the amount of flash to 64 MB.

Notice in the first highlighted line that the filename in this example
is \texttt{c2800nm-advsecurity\ k9-mz.151-4.M6.bin}. The main difference
in the output of the \texttt{show\ flash} and \texttt{show\ version}
commands is that the \texttt{show\ flash} command displays all files in
flash memory and the \texttt{show\ version} command shows the actual
name of the file used to run the router and the location from which it
was loaded, which is flash memory.

\subsubsection[Backing Up the Cisco
IOS]{\texorpdfstring{\protect\hypertarget{c08.xhtmlux5cux23c08-sec-8}{}{}Backing
Up the Cisco IOS}{Backing Up the Cisco IOS}}

To back up the Cisco IOS to a TFTP server, you use the
\texttt{copy\ flash\ tftp} command. It's a straightforward command that
requires only the source filename and the IP address of the TFTP server.

The key to success in this backup routine is to make sure you've got
good, solid connectivity to the TFTP server. Check this by pinging the
TFTP device from the router console prompt like this:

\begin{verbatim}
Router#ping 1.1.1.2
Type escape sequence to abort.
Sending 5, 100-byte ICMP Echos to 1.1.1.2, timeout
  is 2 seconds:
!!!!!
Success rate is 100 percent (5/5), round-trip min/avg/max
  = 4/4/8 ms
\end{verbatim}

After you ping the TFTP server to make sure that IP is working, you can
use the \texttt{copy\ flash\ tftp} command to copy the IOS to the TFTP
server as shown next:

\begin{verbatim}
Router#copy flash tftp
Source filename []?c2800nm-advsecurityk9-mz.151-4.M6.bin
Address or name of remote host []?1.1.1.2
Destination filename [c2800nm-advsecurityk9-mz.151-4.M6.bin]?[enter]
!!!!!!!!!!!!!!!!!!!!!!!!!!!!!!!!!!!!!!!!!!!!!!!!!!!!!!!!!!!!!!!!!!!!!!!!!!!!!!!
45395968 bytes copied in 123.724 secs (357532 bytes/sec)
Router#
\end{verbatim}

\protect\hypertarget{c08.xhtmlux5cux23Page_334}{}{}Just copy the IOS
filename from either the \texttt{show\ flash} or \texttt{show\ version}
command and then paste it when prompted for the source filename.

In the preceding example, the contents of flash memory were copied
successfully to the TFTP server. The address of the remote host is the
IP address of the TFTP host, and the source filename is the file in
flash memory.

\begin{center}\rule{0.5\linewidth}{0.5pt}\end{center}

\includegraphics{images/warning.png} Many newer Cisco routers have
removable memory. You may see names for this memory such as
\texttt{flash0:}, in which case the command in the ­preceding example
would be \texttt{copy\ flash0:\ tftp:}. Alternately, you may see it as
\texttt{usbflash0:}.

\begin{center}\rule{0.5\linewidth}{0.5pt}\end{center}

\subsubsection[Restoring or Upgrading the Cisco Router
IOS]{\texorpdfstring{\protect\hypertarget{c08.xhtmlux5cux23c08-sec-9}{}{}Restoring
or Upgrading the Cisco Router
IOS}{Restoring or Upgrading the Cisco Router IOS}}

What happens if you need to restore the Cisco IOS to flash memory to
replace an original file that has been damaged or if you want to upgrade
the IOS? You can download the file from a TFTP server to flash memory by
using the \texttt{copy\ tftp\ flash} command. This command requires the
IP address of the TFTP host and the name of the file you want to
download.

However, since IOS's can be very large today, we may want to use
something other than tftp, which is unreliable and can only transfer
smaller files. Check this out:

\begin{verbatim}
Corp#copy ?
  /erase          Erase destination file system.
  /error          Allow to copy error file.
  /noverify       Don’t verify image signature before reload.
  /verify         Verify image signature before reload.
  archive:        Copy from archive: file system
  cns:            Copy from cns: file system
  flash:          Copy from flash: file system
  ftp:            Copy from ftp: file system
  http:           Copy from http: file system
  https:          Copy from https: file system
  null:           Copy from null: file system
  nvram:          Copy from nvram: file system
  rcp:            Copy from rcp: file system
  running-config  Copy from current system configuration
  scp:            Copy from scp: file system
  startup-config  Copy from startup configuration
  system:         Copy from system: file system
  tar:            Copy from tar: file system
  tftp:           Copy from tftp: file system
  tmpsys:         Copy from tmpsys: file system
  xmodem:         Copy from xmodem: file system
  ymodem:         Copy from ymodem: file system
\end{verbatim}

\protect\hypertarget{c08.xhtmlux5cux23Page_335}{}{}You can see from the
output above that we have many options, and for the larger files we'll
use \texttt{ftp}: or \texttt{scp}: to copy our IOS into or from routers
and switches, and you can even perform an MD5 verification with the
/\texttt{verify} at the end of a command.

Let's just use tftp for our examples in the chapter because it's
easiest. But before you begin, make sure the file you want to place in
flash memory is in the default TFTP directory on your host. When you
issue the command, TFTP won't ask you where the file is, so if the file
you want to use isn't in the default directory of the TFTP host, this
just won't work.

\begin{verbatim}
Router#copy tftp flash
Address or name of remote host []?1.1.1.2
Source filename []?c2800nm-advsecurityk9-mz.151-4.M6.bin
Destination filename [c2800nm-advsecurityk9-mz.151-4.M6.bin]?[enter]
%Warning: There is a file already existing with this name
Do you want to over write? [confirm][enter]
Accessing tftp://1.1.1.2/ c2800nm-advsecurityk9-mz.151-4.M6.bin...
Loading c2800nm-advsecurityk9-mz.151-4.M6.bin from 1.1.1.2 (via
   FastEthernet0/0): !!!!!!!!!!!!!!!!!!!!!!!!!!!!!!!!!!!!!!!!!!!!!!!!!!!!!!!!!!!!
[OK - 21710744 bytes]
 
45395968 bytes copied in 82.880 secs (261954 bytes/sec)
Router#
\end{verbatim}

In the preceding example, I copied the same file into flash memory, so
it asked me if I wanted to overwrite it. Remember that we are
``playing'' with files in flash memory. If I had just corrupted my file
by overwriting it, I won't know for sure until I reboot the router. Be
careful with this command! If the file is corrupted, you'll need to do
an IOS-restore from ROM monitor mode.

If you are loading a new file and you don't have enough room in flash
memory to store both the new and existing copies, the router will ask to
erase the contents of flash memory before writing the new file into
flash memory, and if you are able to copy the IOS without erasing the
old version, then make sure you remember to use the
\texttt{boot\ system\ flash:}\texttt{ios-file} command.

\begin{center}\rule{0.5\linewidth}{0.5pt}\end{center}

\includegraphics{images/note.png} A Cisco router can become a TFTP
server host for a router system image that's run in flash memory. The
global configuration command is \texttt{tftp-server\ flash:\ ios-file}.

\begin{center}\rule{0.5\linewidth}{0.5pt}\end{center}

\begin{center}\rule{0.5\linewidth}{0.5pt}\end{center}

\includegraphics{images/globe1.png}\\
\textbf{It's Monday Morning and You Just Upgraded Your IOS}

You came in early to work to upgrade the IOS on your router. After the
upgrade, you reload the router and the router now shows the
\texttt{rommon\textgreater{}} prompt.

\protect\hypertarget{c08.xhtmlux5cux23Page_336}{}{}It seems that you're
about to have a bad day! This is what I call an RGE: a resume-generating
event! So, now what do you do? Just keep calm and chive on! Follow these
steps to save your job:

\begin{verbatim}
rommon 1 > tftpdnld

Missing or illegal ip address for variable IP_ADDRESS
Illegal IP address.

usage: tftpdnld [-hr]
  Use this command for disaster recovery only to recover an image via TFTP.
  Monitor variables are used to set up parameters for the transfer.
  (Syntax: "VARIABLE_NAME=value" and use "set" to show current variables.)
  "ctrl-c" or "break" stops the transfer before flash erase begins.

  The following variables are REQUIRED to be set for tftpdnld:
            IP_ADDRESS: The IP address for this unit
        IP_SUBNET_MASK: The subnet mask for this unit
       DEFAULT_GATEWAY: The default gateway for this unit
           TFTP_SERVER: The IP address of the server to fetch from
             TFTP_FILE: The filename to fetch

  The following variables are OPTIONAL:
[unneeded output cut]
rommon 2 >set IP_Address:1.1.1.1
rommon 3 >set IP_SUBNET_MASK:255.0.0.0
rommon 4 >set DEFAULT_GATEWAY:1.1.1.2
rommon 5 >set TFTP_SERVER:1.1.1.2
rommon 6 >set TFTP_FILE: flash:c2800nm-advipservicesk9-mz.124-12.bin
rommon 7 >tftpdnld
\end{verbatim}

From here you can see the variables you need to configure using the
\texttt{set} command; be sure you use ALL\_CAPS with these commands as
well as underscore (\_). From here, you need to set the IP address,
mask, and default gateway of your router, then the IP address of the
TFTP host, which in this example is a directly connected router that I
made a TFTP server with this command:

\texttt{Router(config)\#}\textbf{tftp-server
flash:c2800nm-advipservicesk9-mz.124-12.bin}

And finally, you set the IOS filename of the file on your TFTP server.
Whew! Job saved.

\begin{center}\rule{0.5\linewidth}{0.5pt}\end{center}

There is one other way you can restore the IOS on a router, but it takes
a while. You can use what is called the \texttt{Xmodem} protocol to
actually upload an IOS file into flash memory
\protect\hypertarget{c08.xhtmlux5cux23Page_337}{}{}through the console
port. You'd use the \texttt{Xmodem} through the console port procedure
if you had no network connectivity to the router or switch.

\subsubsection[Using the Cisco IOS File System (Cisco
IFS)]{\texorpdfstring{\protect\hypertarget{c08.xhtmlux5cux23c08-sec-10}{}{}Using
the Cisco IOS File System (Cisco
IFS)}{Using the Cisco IOS File System (Cisco IFS)}}

Cisco has created a file system called Cisco IFS that allows you to work
with files and directories just as you would from a Windows DOS prompt.
The commands you use are \texttt{dir}, \texttt{copy}, \texttt{more},
\texttt{delete}, \texttt{erase} or \texttt{format}, \texttt{cd} and
\texttt{pwd}, and \texttt{mkdir} and \texttt{rmdir}.

Working with IFS gives you the ability to view all files, even those on
remote servers. And you definitely want to find out if an image on one
of your remote servers is valid before you copy it, right? You also need
to know how big it is---size matters here! It's also a really good idea
to take a look at the remote server's configuration and make sure it's
all good before loading that file on your router.

It's very cool that IFS makes the file system user interface
universal---it's not platform specific anymore. You now get to use the
same syntax for all your commands on all of your routers, no matter the
platform!

Sound too good to be true? Well, it kind of is because you'll find out
that support for all commands on each file system and platform just
isn't there. But it's really no big deal since various file systems
differ in the actions they perform; the commands that aren't relevant to
a particular file system are the very ones that aren't supported on that
file system. Be assured that any file system or platform will fully
support all the commands you need to manage it.

Another cool IFS feature is that it cuts down on all those obligatory
prompts for a lot of the commands. If you want to enter a command, all
you have to do is type all the necessary info straight into the command
line---no more jumping through hoops of prompts! So, if you want to copy
a file to an FTP server, all you'd do is first indicate where the
desired source file is on your router, pinpoint where the destination
file is to be on the FTP server, determine the username and password
you're going to use when you want to connect to that server, and type it
all in on one line---sleek! And for those of you resistant to change,
you can still have the router prompt you for all the information it
needs and enjoy entering a more elegantly minimized version of the
command than you did before.

But even in spite of all this, your router might still prompt you---even
if you did everything right in your command line. It comes down to how
you've got the \texttt{file\ prompt} command configured and which
command you're trying to use. But no worries---if that happens, the
default value will be entered right there in the command, and all you
have to do is hit Enter to verify the correct values.

IFS also lets you explore various directories and inventory files in any
directory you want. Plus, you can make subdirectories in flash memory or
on a card, but you only get to do that if you're working on one of the
more recent platforms.

And get this---the new file system interface uses URLs to determine the
whereabouts of a file. So just as they pinpoint places on the Web, URLs
now indicate where files are on your Cisco router, or even on a remote
file server! You just type URLs right into your commands to identify
where the file or directory is. It's really that easy---to copy a file
from one place to another, you simply enter the
\texttt{copy\ source-url\ destination-url} command---sweet! IFS URLs are
a tad different than what you're used to though, and there's an array of
formats to use that vary depending on where, exactly, the file is that
you're after.

\protect\hypertarget{c08.xhtmlux5cux23Page_338}{}{}We're going to use
Cisco IFS commands pretty much the same way that we used the
\texttt{copy} command in the IOS section earlier:

\begin{enumerate}
\tightlist
\item
  For backing up the IOS
\item
  For upgrading the IOS
\item
  For viewing text files
\end{enumerate}

Okay---with all that down, let's take a look at the common IFS commands
available to us for managing the IOS. I'll get into configuration files
soon, but for now I'm going to get you started with going over the
basics used to manage the new Cisco IOS.

\texttt{dir} Same as with Windows, this command lets you view files in a
directory. Type \texttt{dir}, hit Enter, and by default you get the
contents of the \texttt{flash:/} directory output.

\texttt{copy} This is one popular command, often used to upgrade,
restore, or back up an IOS. But as I said, when you use it, it's really
important to focus on the details---what you're copying, where it's
coming from, and where it's going to land.

\texttt{more} Same as with Unix, this will take a text file and let you
look at it on a card. You can use it to check out your configuration
file or your backup configuration file. I'll go over it more when we get
into actual configuration.

\texttt{show\ file} This command will give you the skinny on a specified
file or file system, but it's kind of obscure because people don't use
it a lot.

\texttt{delete} Three guesses---yep, it deletes stuff. But with some
types of routers, not as well as you'd think. That's because even though
it whacks the file, it doesn't always free up the space it was using. To
actually get the space back, you have to use something called the
\texttt{squeeze} command too.

\texttt{erase/format} Use these with care---make sure that when you're
copying files, you say no to the dialog that asks you if you want to
erase the file system! The type of memory you're using determines if you
can nix the flash drive or not.

\texttt{cd/pwd} Same as with Unix and DOS, \texttt{cd} is the command
you use to change directories. Use the \texttt{pwd} command to print
(show) the working directory.

\texttt{mkdir/rmdir} Use these commands on certain routers and switches
to create and delete directories---the \texttt{mkdir} command for
creation and the \texttt{rmdir} command for deletion. Use the
\texttt{cd} and \texttt{pwd} commands to change into these directories.

\begin{center}\rule{0.5\linewidth}{0.5pt}\end{center}

\includegraphics{images/note.png} The Cisco IFS uses the alternate term
\texttt{system:running-config} as well as \texttt{nvram:startup-config}
when copying the configurations on a router, although it is not
mandatory that you use this naming convention.

\begin{center}\rule{0.5\linewidth}{0.5pt}\end{center}

\paragraph{Using the Cisco IFS to Upgrade an IOS}

Let's take a look at some of these Cisco IFS commands on my ISR router
(1841 series) with a hostname of R1.

\protect\hypertarget{c08.xhtmlux5cux23Page_339}{}{}We'll start with the
\texttt{pwd} command to verify our default directory and then use the
\texttt{dir} command to verify its contents (\texttt{flash:/}):

\begin{verbatim}
R1#pwd
flash:
R1#dir
Directory of flash:/
    1  -rw-    13937472  Dec 20 2006 19:58:18 +00:00  c1841-ipbase-
   mz.124-1c.bin
    2  -rw-        1821  Dec 20 2006 20:11:24 +00:00  sdmconfig-18xx.cfg
    3  -rw-     4734464  Dec 20 2006 20:12:00 +00:00  sdm.tar
    4  -rw-      833024  Dec 20 2006 20:12:24 +00:00  es.tar
    5  -rw-     1052160  Dec 20 2006 20:12:50 +00:00  common.tar
    6  -rw-        1038  Dec 20 2006 20:13:10 +00:00  home.shtml
    7  -rw-      102400  Dec 20 2006 20:13:30 +00:00  home.tar
    8  -rw-      491213  Dec 20 2006 20:13:56 +00:00  128MB.sdf
    9  -rw-     1684577  Dec 20 2006 20:14:34 +00:00  securedesktop-
   ios-3.1.1.27-k9.pkg
   10  -rw-      398305  Dec 20 2006 20:15:04 +00:00  sslclient-win-1.1.0.154.pkg
 
32071680 bytes total (8818688 bytes free)
\end{verbatim}

What we can see here is that we have the basic IP IOS
(\texttt{c1841-ipbase-mz.124-1c.bin}). Looks like we need to upgrade our
1841. You've just got to love how Cisco puts the IOS type in the
filename now! First, let's check the size of the file that's in flash
with the \texttt{show\ file} command (\texttt{show\ flash} would also
work):

\begin{verbatim}
R1#show file info flash:c1841-ipbase-mz.124-1c.bin
flash:c1841-ipbase-mz.124-1c.bin:
  type is image (elf) []
  file size is 13937472 bytes, run size is 14103140 bytes
  Runnable image, entry point 0x8000F000, run from ram
\end{verbatim}

With a file that size, the existing IOS will have to be erased before we
can add our new IOS file (\texttt{c1841-advipservicesk9-mz.124-12.bin}),
which is over 21 MB. We'll use the \texttt{delete} command, but
remember, we can play with any file in flash memory and nothing serious
will happen until we reboot---that is, if we made a mistake. So
obviously, and as I pointed out earlier, we need to be very careful
here!

\begin{verbatim}
R1#delete flash:c1841-ipbase-mz.124-1c.bin
Delete filename [c1841-ipbase-mz.124-1c.bin]?[enter]
Delete flash:c1841-ipbase-mz.124-1c.bin? [confirm][enter]
R1#sh flash
-#- --length-- -----date/time------ path
1         1821 Dec 20 2006 20:11:24 +00:00 sdmconfig-18xx.cfg
2      4734464 Dec 20 2006 20:12:00 +00:00 sdm.tar
3       833024 Dec 20 2006 20:12:24 +00:00 es.tar
4      1052160 Dec 20 2006 20:12:50 +00:00 common.tar
5         1038 Dec 20 2006 20:13:10 +00:00 home.shtml
6       102400 Dec 20 2006 20:13:30 +00:00 home.tar
7       491213 Dec 20 2006 20:13:56 +00:00 128MB.sdf
8      1684577 Dec 20 2006 20:14:34 +00:00 securedesktop-ios-3.1.1.27-k9.pkg
9       398305 Dec 20 2006 20:15:04 +00:00 sslclient-win-1.1.0.154.pkg
22757376 bytes available (9314304 bytes used)
R1#sh file info flash:c1841-ipbase-mz.124-1c.bin
%Error opening flash:c1841-ipbase-mz.124-1c.bin (File not found)
R1#
\end{verbatim}

So with the preceding commands, we deleted the existing file and then
verified the deletion by using both the \texttt{show\ flash} and
\texttt{show\ file} commands. We'll add the new file with the
\texttt{copy} command, but again, we need to make sure to be careful
because this way isn't any safer than the first method I showed you
earlier:

\begin{verbatim}
R1#copy tftp://1.1.1.2/c1841-advipservicesk9-mz.124-12.bin/ flash:/
    c1841-advipservicesk9-mz.124-12.bin
Source filename [/c1841-advipservicesk9-mz.124-12.bin/]?[enter]
Destination filename [c1841-advipservicesk9-mz.124-12.bin]?[enter]
Loading /c1841-advipservicesk9-mz.124-12.bin/ from 1.1.1.2 (via
    FastEthernet0/0): !!!!!!!!!!!!!!!!!!!!!!!!!!!!!!!!!!!!!!!!
[output cut]
!!!!!!!!!!!!!!!!!!!!!!!!!!!!!!!!!!!!!!!!!!!!!!!!!!!!!!!
[OK - 22103052 bytes]
22103052 bytes copied in 72.008 secs (306953 bytes/sec)
R1#sh flash
-#- --length-- -----date/time------ path
1         1821 Dec 20 2006 20:11:24 +00:00 sdmconfig-18xx.cfg
2      4734464 Dec 20 2006 20:12:00 +00:00 sdm.tar
3       833024 Dec 20 2006 20:12:24 +00:00 es.tar
4      1052160 Dec 20 2006 20:12:50 +00:00 common.tar
5         1038 Dec 20 2006 20:13:10 +00:00 home.shtml
6       102400 Dec 20 2006 20:13:30 +00:00 home.tar
7       491213 Dec 20 2006 20:13:56 +00:00 128MB.sdf
8      1684577 Dec 20 2006 20:14:34 +00:00 securedesktop-ios-3.1.1.27-k9.pkg
9       398305 Dec 20 2006 20:15:04 +00:00 sslclient-win-1.1.0.154.pkg
10    22103052 Mar 10 2007 19:40:50 +00:00 c1841-advipservicesk9-mz.124-12.bin
651264 bytes available (31420416 bytes used)
R1#
\end{verbatim}

We can also check the file information with the \texttt{show\ file}
command:

\begin{verbatim}
R1#sh file information flash:c1841-advipservicesk9-mz.124-12.bin
flash:c1841-advipservicesk9-mz.124-12.bin:
  type is image (elf) []
  file size is 22103052 bytes, run size is 22268736 bytes
  Runnable image, entry point 0x8000F000, run from ram
\end{verbatim}

Remember that the IOS is expanded into RAM when the router boots, so the
new IOS will not run until you reload the router.

I really recommend experimenting with the Cisco IFS commands on a router
just to get a good feel for them because, as I've said, they can
definitely give you some grief if not executed properly!

\begin{center}\rule{0.5\linewidth}{0.5pt}\end{center}

\includegraphics{images/tip.png} I mention ``safer methods'' a lot in
this chapter. Clearly, I've caused myself some serious pain by not being
careful enough when working in flash memory! I cannot stress this
enough---pay attention when messing around with flash memory!

\begin{center}\rule{0.5\linewidth}{0.5pt}\end{center}

One of the brilliant features of the ISR routers is that they use the
physical flash cards that are accessible from the front or back of any
router. These typically have a name like \texttt{usbflash0}:, so to view
the contents, you'd type \texttt{dir\ usbflash0:}, for example. You can
pull these flash cards out, put them in an appropriate slot in your PC,
and the card will show up as a drive. You can then add, change, and
delete files. Just put the flash card back in your router and power
up---instant upgrade. Nice!

\subsubsection[Licensing]{\texorpdfstring{\protect\hypertarget{c08.xhtmlux5cux23c08-sec-11}{}{}Licensing}{Licensing}}

IOS licensing is now done quite differently than it was with previous
versions of the IOS. Actually, there was no licensing before the new
15.0 IOS code, just your word and honor, and we can only guess based on
how all products are downloaded on the Internet daily how well that has
worked out for Cisco!

Starting with the IOS 15.0 code, things are much different---almost too
different. I can imagine that Cisco will come back toward the middle on
its licensing issues, so that the administration and management won't be
as detailed as it is with the new 15.0 code license is now; but you can
be the judge of that after reading this section.

A new ISR router is pre-installed with the software images and licenses
that you ordered, so as long as you ordered and paid for everything you
need, you're set! If not, you can just install another license, which
can be a tad tedious at first---enough so that installing
\protect\hypertarget{c08.xhtmlux5cux23Page_342}{}{}a license was made an
objective on the Cisco exam! Of course, it can be done, but it
definitely requires some effort. As is typical with Cisco, if you spend
enough money on their products, they tend to make it easier on you and
your administration, and the licensing for the newest IOS is no
exception, as you'll soon see.

On a positive note, Cisco provides evaluation licenses for most software
packages and features that are supported on the hardware you purchased,
and it's always nice to be able to try it out before you buy. Once the
temporary license expires after 60 days, you need to acquire a permanent
license in order to continue to use the extended features that aren't
available in your current version. This method of licensing allows you
to enable a router to use different parts of the IOS. So, what happens
after 60 days? Well, nothing---back to the honor system for now. This is
now called \emph{Right-To-Use (RTU) licensing}, and it probably won't
always be available via your honor, but for now it is.

But that's not the best part of the new licensing features. Prior to the
15.0 code release, there were eight different software feature sets for
each hardware router type. With the IOS 15.0 code, the packaging is now
called a \emph{universal image}, meaning all feature sets are available
in one file with all features packed neatly inside. So instead of the
pre-15.0 IOS file packages of one image per feature set, Cisco now just
builds one universal image that includes all of them in the file. Even
so, we still need a different universal image per router model or
series, just not a different image for each feature set as we did with
previous IOS versions.

To use the features in the IOS software, you must unlock them using the
software activation process. Since all features available are inside the
universal image already, you can just unlock the features you need as
you need them, and of course pay for these features when you determine
that they meet your business requirements. All routers come with
something called the IP Base licensing, which is the prerequisite for
installing all other features.

There are three different technology packages available for purchase
that can be installed as additional feature packs on top of the
prerequisite IP Base (default), which provides entry-level IOS
functionality. These are as follows:

\begin{enumerate}
\tightlist
\item
  \textbf{Data:} MPLS, ATM, and multiprotocol support
\item
  \textbf{Unified Communications:} VoIP and IP telephony
\item
  \textbf{Security:} Cisco IOS Firewall, IPS, IPsec, 3DES, and VPN
\end{enumerate}

For example, if you need MPLS and IPsec, you'll need the default IP
Base, Data, and Security premium packages unlocked on your router.

To obtain the license, you'll need the unique device identifier (UDI),
which has two components: the product ID (PID) and the serial number of
the router. The \texttt{show\ license\ UDI} command provides this
information in an output as shown:

\begin{verbatim}
Router#sh license udi
Device#   PID                   SN              UDI
-------------------------------------------------------------------------
*0       CISCO2901/K9          FTX1641Y07J     CISCO2901/K9:FTX1641Y07J
\end{verbatim}

After the time has expired for your 60-day evaluation period, you can
either obtain the license file from the Cisco License Manager (CLM),
which is an automated process, or use
\protect\hypertarget{c08.xhtmlux5cux23Page_343}{}{}the manual process
through the Cisco Product License Registration portal. Typically only
larger companies will use the CLM because you'd need to install software
on a server, which then keeps track of all your licenses for you. If you
have just a few licenses that you use, you can opt for the manual web
browser process found on the Cisco Product License Registration portal
and then just add in a few CLI commands. After that, you just basically
keep track of putting all the different license features together for
each device you manage. Although this sounds like a lot of work, you
don't need to perform these steps often. But clearly, going with the CLM
makes a lot of sense if you have bunches of licenses to manage because
it will put together all the little pieces of licensing for each router
in one easy process.

When you purchase the software package with the features that you want
to install, you need to permanently activate the software package using
your UDI and the \emph{product authorization key (PAK)} that you
received with your purchase. This is essentially your receipt
acknowledging that you purchased the license. You then need to connect
the license with a particular router by combining the PAK and the UDI,
which you do online at the Cisco Product License Registration portal
(\texttt{www.cisco.com/go/license}). If you haven't already registered
the license on a different router, and it is valid, Cisco will then
email you your permanent license, or you can download it from your
account.

But wait! You're still not done. You now need to activate the license on
the router. Whew... maybe it's worthwhile to install the CLM on a server
after all! Staying with the manual method, you need to make the new
license file available to the router either via a USB port on the router
or through a TFTP server. Once it's available to the router, you'll use
the \texttt{license\ install} command from privileged mode.

Assuming that you copied the file into flash memory, the command would
look like something like this:

\begin{verbatim}
Router#license install ?
  archive:  Install from archive: file system
  flash:    Install from flash: file system
  ftp:      Install from ftp: file system
  http:     Install from http: file system
  https:    Install from https: file system
  null:     Install from null: file system
  nvram:    Install from nvram: file system
  rcp:      Install from rcp: file system
  scp:      Install from scp: file system
  syslog:   Install from syslog: file system
  system:   Install from system: file system
  tftp:     Install from tftp: file system
  tmpsys:   Install from tmpsys: file system
  xmodem:   Install from xmodem: file system
  ymodem:   Install from ymodem: file system
Router#license install flash:FTX1628838P_201302111432454180.lic
Installing licenses from "flash::FTX1628838P_201302111432454180.lic"
Installing...Feature:datak9...Successful:Supported
1/1 licenses were successfully installed
0/1 licenses were existing licenses
0/1 licenses were failed to install
April 12 2:31:19.786: %LICENSE-6-INSTALL: Feature datak9 1.0 was
installed in this device. UDI=CISCO2901/K9:FTX1628838P; StoreIndex=1:Primary License Storage
 
April 12 2:31:20.078: %IOS_LICENSE_IMAGE_APPLICATION-6-LICENSE_LEVEL: Module name =c2800 Next reboot level = datak9 and License = datak9
\end{verbatim}

You need to reboot to have the new license take effect. Now that you
have your license installed and running, how do you use Right-To-Use
licensing to check out new features on your router? Let's look into that
now.

\subsubsection[Right-To-Use Licenses (Evaluation
Licenses)]{\texorpdfstring{\protect\hypertarget{c08.xhtmlux5cux23c08-sec-12}{}{}Right-To-Use
Licenses (Evaluation
Licenses)}{Right-To-Use Licenses (Evaluation Licenses)}}

Originally called evaluation licenses, Right-To-Use (RTU) licenses are
what you need when you want to update your IOS to load a new feature but
either don't want to wait to get the license or just want to test if
this feature will truly meet your business requirements. This makes
sense because if Cisco made it complicated to load and check out a
feature, they could potentially miss out on a sale! Of course if the
feature does work for you, they'll want you to buy a permanent license,
but again, this is on the honor system at the time of this writing.

Cisco's license model allows you to install the feature you want without
a PAK. The Right-To-Use license works for 60 days before you would need
to install your permanent license. To enable the Right-To-Use license
you would use the \texttt{license\ boot\ module} command. The following
demonstrates starting the Right-To-Use license on my 2900 series router,
enabling the security module named \texttt{securityk9}:

\begin{verbatim}
Router(config)#license boot module c2900 technology-package securityk9
PLEASE READ THE FOLLOWING TERMS CAREFULLY. INSTALLING THE LICENSE OR LICENSE KEY PROVIDED FOR ANY CISCO PRODUCT FEATURE OR USING
SUCHPRODUCT FEATURE CONSTITUTES YOUR FULL ACCEPTANCE OF THE
FOLLOWING TERMS. YOU MUST NOT PROCEED FURTHER IF YOU ARE NOT WILLING
TO BE BOUND BY ALL THE TERMS SET FORTH HEREIN.
[output cut]
Activation of the software command line interface will be evidence of
your acceptance of this agreement.
 
ACCEPT? [yes/no]: yes
 
% use 'write' command to make license boot config take effect on next boot
Feb 12 01:35:45.060: %IOS_LICENSE_IMAGE_APPLICATION-6-LICENSE_LEVEL:
Module name =c2900 Next reboot level = securityk9 and License = securityk9
 
Feb 12 01:35:45.524: %LICENSE-6-EULA_ACCEPTED: EULA for feature
securityk9 1.0 has been accepted. UDI=CISCO2901/K9:FTX1628838P; StoreIndex=0:Built-In License Storage
\end{verbatim}

Once the router is reloaded, you can use the security feature set. And
it is really nice that you don't need to reload the router again if you
choose to install a permanent license for this feature. The
\texttt{show\ license} command shows the licenses installed on the
router:

\begin{verbatim}
Router#show license
Index 1 Feature: ipbasek9
     Period left: Life time
     License Type: Permanent
     License State: Active, In Use
     License Count: Non-Counted
     License Priority: Medium
Index 2 Feature: securityk9
     Period left: 8 weeks  2 days
     Period Used: 0  minute  0  second
     License Type: EvalRightToUse
     License State: Active, In Use
     License Count: Non-Counted
     License Priority: None
Index 3 Feature: uck9
     Period left: Life time
     License Type: Permanent
     License State: Active, In Use
     License Count: Non-Counted
     License Priority: Medium
Index 4 Feature: datak9
     Period left: Not Activated
     Period Used: 0  minute  0  second
     License Type: EvalRightToUse
     License State: Not in Use, EULA not accepted
     License Count: Non-Counted
     License Priority: None
Index 5 Feature: gatekeeper
 [output cut]
\end{verbatim}

\protect\hypertarget{c08.xhtmlux5cux23Page_346}{}{}You can see in the
preceding output that the \texttt{ipbasek9} is permanent and the
\texttt{securityk9} has a license type of \texttt{EvalRightToUse}. The
\texttt{show\ license\ feature} command provides the same information as
\texttt{show\ license}, but it's summarized into one line as shown in
the next output:

\begin{verbatim}
Router#sh license feature
Feature name    Enforcement  Evaluation  Subscription   Enabled  RightToUse
ipbasek9             no           no          no             yes      no
securityk9           yes          yes         no             no       yes
uck9                 yes          yes         no             yes      yes
datak9               yes          yes         no             no       yes
gatekeeper           yes          yes         no             no       yes
SSL_VPN              yes          yes         no             no       yes
ios-ips-update       yes          yes         yes            no       yes
SNASw                yes          yes         no             no       yes
hseck9               yes          no          no             no       no
cme-srst             yes          yes         no             yes      yes
WAAS_Express         yes          yes         no             no       yes
UCVideo              yes          yes         no             no       yes
\end{verbatim}

The \texttt{show\ version} command also shows the license information at
the end of the command output:

\begin{verbatim}
Router#show version
[output cut]
License Info:
 
License UDI:
 
-------------------------------------------------
Device#   PID                   SN
-------------------------------------------------
*0        CISCO2901/K9          FTX1641Y07J
 
Technology Package License Information for Module:'c2900'
 
-----------------------------------------------------------------
Technology    Technology-package           Technology-package
              Current       Type           Next reboot
------------------------------------------------------------------
ipbase        ipbasek9      Permanent      ipbasek9
security      None          None           None
uc            uck9          Permanent      uck9
data          None          None           None
 
Configuration register is 0x2102
\end{verbatim}

The \texttt{show\ version} command shows if the license was activated.
Don't forget, you'll need to reload the router to have the license
features take effect if the license evaluation is not already active.

\subsubsection[Backing Up and Uninstalling the
License]{\texorpdfstring{\protect\hypertarget{c08.xhtmlux5cux23c08-sec-13}{}{}Backing
Up and Uninstalling the
License}{Backing Up and Uninstalling the License}}

It would be a shame to lose your license if it has been stored in flash
and your flash files become corrupted. So always back up your IOS
license!

If your license has been saved in a location other than flash, you can
easily back it up to flash memory via the \texttt{license\ save}
command:

\begin{verbatim}
Router#license save flash:Todd_License.lic
\end{verbatim}

The previous command will save your current license to flash. You can
restore your license with the \texttt{license\ install} command I
demonstrated earlier.

There are two steps to uninstalling the license on a router. First, to
uninstall the license you need to disable the technology package, using
the \texttt{no\ license\ boot\ module} command with the keyword
\texttt{disable} at the end of the command line:

\begin{verbatim}
Router#license boot module c2900 technology-package securityk9 disable
\end{verbatim}

The second step is to clear the license. To achieve this from the
router, use the \texttt{license\ clear} command and then remove the
license with the \texttt{no\ license\ boot\ module} command:

\begin{verbatim}
Router#license clear securityk9
Router#config t
Router(config)#no license boot module c2900 technology-package securityk9 disable
Router(config)#exit
Router#reload
\end{verbatim}

After you run through the preceding commands, the license will be
removed from your router.

Here's a summary of the license commands I used in this chapter. These
are important commands to have down and you really need to understand
these to meet the Cisco objectives:

\begin{enumerate}
\tightlist
\item
  \texttt{show\ license} determines the licenses that are active on your
  system. It also displays a group of lines for each feature in the
  currently running IOS image along with several status variables
  related to software activation and licensing, both licensed and
  unlicensed features.
\item
  \texttt{show\ license\ feature} allows you to view the technology
  package licenses and feature licenses that are supported on your
  router along with several status variables related to software
  activation and licensing. This includes both licensed and unlicensed
  features.
\item
  \texttt{show\ license\ udi} displays the unique device identifier
  (UDI) of the router, which comprises the product ID (PID) and serial
  number of the router.
\item
  \texttt{show\ version} displays various pieces of information about
  the current IOS version, including the licensing details at the end of
  the command's output.
\item
  \texttt{license\ install\ url} installs a license key file into a
  router.
\item
  \texttt{license\ boot\ module} installs a Right-To-Use license feature
  on a router.
\end{enumerate}

\begin{center}\rule{0.5\linewidth}{0.5pt}\end{center}

\includegraphics{images/tip.png} To help you organize a large amount of
licenses, search on \texttt{Cisco.com} for the Cisco Smart Software
Manager. This web page enables you to manage all your licenses from one
centralized website. With Cisco Smart Software Manager, you organize and
view your licenses in groups that are called \emph{virtual accounts},
which are collections of licenses and product instances.

\begin{center}\rule{0.5\linewidth}{0.5pt}\end{center}

\subsection[Summary]{\texorpdfstring{\protect\hypertarget{c08.xhtmlux5cux23c08-sec-14}{}{}Summary}{Summary}}

You now know how Cisco routers are configured and how to manage those
configurations.

This chapter covered the internal components of a router, which included
ROM, RAM, NVRAM, and flash.

In addition, I covered what happens when a router boots and which files
are loaded at that time. The configuration register tells the router how
to boot and where to find files. You learned how to change and verify
the configuration register settings for ­password recovery purposes. I
also showed you how to manage these files using the CLI and IFS.

Finally, the chapter covered licensing with the new 15.0 code, including
how to install a permanent license and a Right-To-Use license to install
features for 60 days. I also showed you the verification commands used
to see what licenses are installed and to verify their status.

\subsection[Exam
Essentials]{\texorpdfstring{\protect\hypertarget{c08.xhtmlux5cux23c08-sec-15}{}{}Exam
Essentials}{Exam Essentials}}

\textbf{Define the Cisco router components.} Describe the functions of
the bootstrap, POST, ROM monitor, mini-IOS, RAM, ROM, flash memory,
NVRAM, and the configuration register.

\protect\hypertarget{c08.xhtmlux5cux23Page_349}{}{}\textbf{Identify the
steps in the router boot sequence.} The steps in the boot sequence are
POST, loading the IOS, and copying the startup configuration from NVRAM
to RAM.

\textbf{Understand configuration register commands and settings.} The
0x2102 setting is the default on all Cisco routers and tells the router
to look in NVRAM for the boot sequence. 0x2101 tells the router to boot
from ROM, and 0x2142 tells the router not to load the startup-config in
NVRAM to provide password recovery.

\textbf{Perform password recovery.} The steps in the password recovery
process are interrupt the router boot sequence, change the configuration
register, reload the router and enter privileged mode, copy the
startup-config file to running-config and verify that your interfaces
are re-enabled, change/set the password, save the new configuration,
reset the configuration register, and reload the router.

\textbf{Back up an IOS image.} By using the privileged-mode command
\texttt{copy\ flash\ tftp}, you can back up a file from flash memory to
a TFTP (network) server.

\textbf{Restore or upgrade an IOS image.} By using the privileged-mode
command \texttt{copy\ tftp\ flash}, you can restore or upgrade a file
from a TFTP (network) server to flash memory.

\textbf{Describe best practices to prepare to back up an IOS image to a
network server.} Make sure that you can access the network server,
ensure that the network server has adequate space for the code image,
and verify the file naming and path requirement.

\textbf{Understand and use Cisco IFS file system management commands.}
The commands to use are \texttt{dir}, \texttt{copy}, \texttt{more},
\texttt{delete}, \texttt{erase} or \texttt{format}, \texttt{cd} and
\texttt{pwd}, and \texttt{mkdir} and \texttt{rmdir}, as well
as\texttt{\ system:running-config} and \texttt{nvram:startup-config}.

\textbf{Remember how to install a permanent and Right-To-Use license.}
To install a permanent license on a router, use the
\texttt{install\ license} \texttt{url} command. To install an evaluation
feature, use the \texttt{license\ boot\ module} command.

\textbf{Remember the verification commands used for licensing in the new
ISR G2 ­routers.} The \texttt{show\ license} command determines the
licenses that are active on your ­system. The \texttt{show}
\texttt{license\ feature} command allows you to view the technology
package licenses and feature licenses that are supported on your router.
The \texttt{show\ license\ udi} command displays the unique device
identifier (UDI) of the router, which comprises the product ID (PID) and
serial number of the router, and the \texttt{show\ version} command
displays information about the current IOS version, including the
licensing details at the end of the command's output.

\subsection[Written Lab
8]{\texorpdfstring{\protect\hypertarget{c08.xhtmlux5cux23c08-sec-16}{}{}Written
Lab 8}{Written Lab 8}}

You can find the answers to this labs in Appendix A, ``Answers to
Written Labs.''

In this section, you'll complete the following lab to make sure you've
got the information and concepts contained within them fully dialed in:

\begin{enumerate}
\tightlist
\item
  Lab 8.1: IOS Management
\end{enumerate}

\subsubsection[Written Lab 8.1: IOS
Management]{\texorpdfstring{\protect\hypertarget{c08.xhtmlux5cux23c08-sec-17}{}{}\protect\hypertarget{c08.xhtmlux5cux23Page_350}{}{}Written
Lab 8.1: IOS Management}{Written Lab 8.1: IOS Management}}

Write the answers to the following questions:

\begin{enumerate}
\tightlist
\item
  What is the command to copy a Cisco IOS to a TFTP server?
\item
  What do you set the configuration register setting to in order to boot
  the mini-IOS in ROM?
\item
  What is the configuration register setting to tell the router to look
  in NVRAM for the boot sequence?
\item
  What do you set the configuration register setting to in order to boot
  to ROM monitor mode?
\item
  What is used with a PAK to generate a license file?
\item
  What is the configuration register setting for password recovery?
\item
  Which command can change the location from which the system loads the
  IOS?
\item
  What is the first step of the router boot sequence?
\item
  What command can you use to upgrade a Cisco IOS?
\item
  Which command determines the licenses that are active on your system?
\end{enumerate}

\subsection[Hands-on
Labs]{\texorpdfstring{\protect\hypertarget{c08.xhtmlux5cux23c08-sec-18}{}{}Hands-on
Labs}{Hands-on Labs}}

To complete the labs in this section, you need at least one router
(three would be best) and at least one PC running as a TFTP server. TFTP
server software must be installed and running on the PC. For these labs,
it is also assumed that your PC and the router(s) are connected together
with a switch or hub and that all interfaces (PC NIC and router
interfaces) are in the same subnet. You can alternately connect the PC
directly to the router or connect the routers directly to one another
(use a crossover cable in that case). Remember that the labs listed here
were created for use with real routers but can easily be used with the
LammleSim IOS version (found at \texttt{www.lammle.com/ccna}) or Cisco's
Packet Tracer program.

Here is a list of the labs in this chapter:

\begin{enumerate}
\tightlist
\item
  Lab 8.1: Backing Up Your Router IOS
\item
  Lab 8.2: Upgrading or Restoring Your Router IOS
\end{enumerate}

\subsubsection[Hands-on Lab 8.1: Backing Up Your Router
IOS]{\texorpdfstring{\protect\hypertarget{c08.xhtmlux5cux23c08-sec-19}{}{}Hands-on
Lab 8.1: Backing Up Your Router
IOS}{Hands-on Lab 8.1: Backing Up Your Router IOS}}

In this lab, we'll be backing up the IOS from flash to a TFTP host.

\begin{enumerate}
\tightlist
\item
  Log into your router and go into privileged mode by typing \texttt{en}
  or \texttt{enable}.
\item
  Make sure you can connect to the TFTP server that is on your network
  by pinging the IP address from the router console.
\item
  \protect\hypertarget{c08.xhtmlux5cux23Page_351}{}{}Type
  \texttt{show\ flash} to see the contents of flash memory.
\item
  Type \texttt{show\ version} at the router privileged-mode prompt to
  get the name of the IOS currently running on the router. If there is
  only one file in flash memory, the \texttt{show\ flash} and
  \texttt{show\ version} commands show the same file. Remember that the
  \texttt{show\ version} command shows you the file that is currently
  running and the \texttt{show\ flash} command shows you all of the
  files in flash memory.
\item
  Once you know you have good Ethernet connectivity to the TFTP server
  and you also know the IOS filename, back up your IOS by typing
  \texttt{copy\ flash\ tftp}. This command tells the router to copy a
  specified file from flash memory (this is where the IOS is stored by
  default) to a TFTP server.
\item
  Enter the IP address of the TFTP server and the source IOS filename.
  The file is now copied and stored in the TFTP server's default
  directory.
\end{enumerate}

\subsubsection[Hands-on Lab 8.2: Upgrading or Restoring Your Router
IOS]{\texorpdfstring{\protect\hypertarget{c08.xhtmlux5cux23c08-sec-20}{}{}Hands-on
Lab 8.2: Upgrading or Restoring Your Router
IOS}{Hands-on Lab 8.2: Upgrading or Restoring Your Router IOS}}

In this lab, we'll be copying an IOS from a TFTP host to flash memory.

\begin{enumerate}
\tightlist
\item
  Log into your router and go into privileged mode by typing \texttt{en}
  or \texttt{enable}.
\item
  Make sure you can connect to the TFTP server by pinging the IP address
  of the server from the router console.
\item
  Once you know you have good Ethernet connectivity to the TFTP server,
  type the \texttt{copy\ tftp\ flash} command.
\item
  Confirm that the router will not function during the restore or
  upgrade by following the prompts provided on the router console. It is
  possible this prompt may not occur.
\item
  Enter the IP address of the TFTP server.
\item
  Enter the name of the IOS file you want to restore or upgrade.
\item
  Confirm that you understand that the contents of flash memory will be
  erased if there is not enough room in flash to store the new image.
\item
  Watch in amazement as your IOS is deleted out of flash memory and your
  new IOS is copied to flash memory.
\end{enumerate}

If the file that was in flash memory is deleted but the new version
wasn't copied to flash memory, the router will boot from ROM monitor
mode. You'll need to figure out why the copy operation did not take
place.

\subsection[Review
Questions]{\texorpdfstring{\protect\hypertarget{c08.xhtmlux5cux23c08-sec-21}{}{}\protect\hypertarget{c08.xhtmlux5cux23Page_352}{}{}Review
Questions}{Review Questions}}

\begin{center}\rule{0.5\linewidth}{0.5pt}\end{center}

\includegraphics{images/note.png} The following questions are designed
to test your understanding of this chapter's material. For more
information on how to get additional questions, please see
\texttt{www.lammle.com/ccna}.

\begin{center}\rule{0.5\linewidth}{0.5pt}\end{center}

You can find the answers to these questions in Appendix B, ``Answers to
Review Questions.''

\begin{enumerate}
\item
  What does the command \texttt{confreg\ 0x2142} provide?

  \begin{enumerate}
  \tightlist
  \item
    It is used to restart the router.
  \item
    It is used to bypass the configuration in NVRAM.
  \item
    It is used to enter ROM monitor mode.
  \item
    It is used to view the lost password.
  \end{enumerate}
\item
  Which command will copy the IOS to a backup host on your network?

  \begin{enumerate}
  \tightlist
  \item
    \texttt{transfer\ IOS\ to\ 172.16.10.1}
  \item
    \texttt{copy\ run\ start}
  \item
    \texttt{copy\ tftp\ flash}
  \item
    \texttt{copy\ start\ tftp}
  \item
    \texttt{copy\ flash\ tftp}
  \end{enumerate}
\item
  What command is used to permanently install a license on an ISR2
  router?

  \begin{enumerate}
  \tightlist
  \item
    \texttt{install\ license}
  \item
    \texttt{license\ install}
  \item
    \texttt{boot\ system\ license}
  \item
    \texttt{boot\ license\ module}
  \end{enumerate}
\item
  You type the following into the router and reload. What will the
  router do?

\begin{verbatim}
Router(config)#boot system flash c2800nm-advsecurityk9-mz.151-4.M6.bin
Router(config)#config-register 0x2101
Router(config)#do sh ver
[output cut]
Configuration register is 0x2102 (will be 0x2101 at next reload)
\end{verbatim}

  \begin{enumerate}
  \tightlist
  \item
    The router will expand and run the
    \texttt{c2800nm-advsecurityk9-mz.151-4.M6.bin} IOS from flash
    memory.
  \item
    The router will go into setup mode.
  \item
    The router will load the mini-IOS from ROM.
  \item
    The router will enter ROM monitor mode.
  \end{enumerate}
\item
  \protect\hypertarget{c08.xhtmlux5cux23Page_353}{}{}A network
  administrator wants to upgrade the IOS of a router without removing
  the image currently installed. What command will display the amount of
  memory consumed by the current IOS image and indicate whether there is
  enough room available to hold both the current and new images?

  \begin{enumerate}
  \tightlist
  \item
    \texttt{show\ version}
  \item
    \texttt{show\ flash}
  \item
    \texttt{show\ memory}
  \item
    \texttt{show\ buffers}
  \item
    \texttt{show\ running-config}
  \end{enumerate}
\item
  The corporate office sends you a new router to connect, but upon
  connecting the console cable, you see that there is already a
  configuration on the router. What should be done before a new
  configuration is entered in the router?

  \begin{enumerate}
  \tightlist
  \item
    RAM should be erased and the router restarted.
  \item
    Flash should be erased and the router restarted.
  \item
    NVRAM should be erased and the router restarted.
  \item
    The new configuration should be entered and saved.
  \end{enumerate}
\item
  Which command loads a new version of the Cisco IOS into a router?

  \begin{enumerate}
  \tightlist
  \item
    \texttt{copy\ flash\ ftp}
  \item
    \texttt{copy\ nvram\ flash}
  \item
    \texttt{copy\ flash\ tftp}
  \item
    \texttt{copy\ tftp\ flash}
  \end{enumerate}
\item
  Which command will show you the IOS version running on your router?

  \begin{enumerate}
  \tightlist
  \item
    \texttt{sh\ IOS}
  \item
    \texttt{sh\ flash}
  \item
    \texttt{sh\ version}
  \item
    \texttt{sh\ protocols}
  \end{enumerate}
\item
  What should the configuration register value be after you successfully
  complete the password recovery procedure and return the router to
  normal operation?

  \begin{enumerate}
  \tightlist
  \item
    0x2100
  \item
    0x2101
  \item
    0x2102
  \item
    0x2142
  \end{enumerate}
\item
  You save the configuration on a router with the
  \texttt{copy\ running-config\ startup-config} command and reboot the
  router. The router, however, comes up with a blank configuration. What
  can the problem be?

  \begin{enumerate}
  \tightlist
  \item
    You didn't boot the router with the correct command.
  \item
    NVRAM is corrupted.
  \item
    \protect\hypertarget{c08.xhtmlux5cux23Page_354}{}{}The configuration
    register setting is incorrect.
  \item
    The newly upgraded IOS is not compatible with the hardware of the
    router.
  \item
    The configuration you saved is not compatible with the hardware.
  \end{enumerate}
\item
  Which command will install a Right-To-Use license so you can use an
  evaluation version of a feature?

  \begin{enumerate}
  \tightlist
  \item
    \texttt{install\ Right-To-Use\ license\ feature\ feature}
  \item
    \texttt{install\ temporary\ feature\ feature}
  \item
    \texttt{license\ install\ feature}
  \item
    \texttt{license\ boot\ module}
  \end{enumerate}
\item
  Which command determines the licenses that are active on your system
  along with several status variables?

  \begin{enumerate}
  \tightlist
  \item
    \texttt{show\ license}
  \item
    \texttt{show\ license\ feature}
  \item
    \texttt{show\ license\ udi}
  \item
    \texttt{show\ version}
  \end{enumerate}
\item
  Which command allows you to view the technology package licenses and
  feature licenses that are supported on your router along with several
  status variables?

  \begin{enumerate}
  \tightlist
  \item
    \texttt{show\ license}
  \item
    \texttt{show\ license\ feature}
  \item
    \texttt{show\ license\ udi}
  \item
    \texttt{show\ version}
  \end{enumerate}
\item
  Which command displays the unique device identifier that comprises the
  product ID and serial number of the router?

  \begin{enumerate}
  \tightlist
  \item
    \texttt{show\ license}
  \item
    \texttt{show\ license\ feature}
  \item
    \texttt{show\ license\ udi}
  \item
    \texttt{show\ version}
  \end{enumerate}
\item
  Which command displays various pieces of information about the current
  IOS version, including the licensing details at the end of the
  command's output?

  \begin{enumerate}
  \tightlist
  \item
    \texttt{show\ license}
  \item
    \texttt{show\ license\ feature}
  \item
    \texttt{show\ license\ udi}
  \item
    \texttt{show\ version}
  \end{enumerate}
\item
  Which command backs up your license to flash memory?

  \begin{enumerate}
  \tightlist
  \item
    \texttt{copy\ tftp\ flash}
  \item
    \texttt{save\ license\ flash}
  \item
    \texttt{license\ save\ flash}
  \item
    \texttt{copy\ license\ flash}
  \end{enumerate}
\item
  Which command displays the configuration register setting?

  \begin{enumerate}
  \tightlist
  \item
    \texttt{show\ ip\ route}
  \item
    \texttt{show\ boot\ version}
  \item
    \texttt{show\ version}
  \item
    \texttt{show\ flash}
  \end{enumerate}
\item
  What two steps are needed to remove a license from a router? (Choose
  two.)

  \begin{enumerate}
  \tightlist
  \item
    Use the \texttt{erase\ flash:license} command.
  \item
    Reload the system.
  \item
    Use the \texttt{license\ boot} command with the \texttt{disable}
    variable at the end of the command line.
  \item
    Clear the license with the \texttt{license\ clear} command.
  \end{enumerate}
\item
  You have your laptop directly connected into a router's Ethernet port.
  Which of the following are among the requirements for the
  \texttt{copy\ flash\ tftp} command to be successful? (Choose three.)

  \begin{enumerate}
  \tightlist
  \item
    TFTP server software must be running on the router.
  \item
    TFTP server software must be running on your laptop.
  \item
    The Ethernet cable connecting the laptop directly into the router's
    Ethernet port must be a straight-through cable.
  \item
    The laptop must be on the same subnet as the router's Ethernet
    interface.
  \item
    The \texttt{copy\ flash\ tftp} command must be supplied the IP
    address of the laptop.
  \item
    There must be enough room in the flash memory of the router to
    accommodate the file to be copied.
  \end{enumerate}
\item
  The configuration register setting of 0x2102 provides what function to
  a router?

  \begin{enumerate}
  \tightlist
  \item
    Tells the router to boot into ROM monitor mode
  \item
    Provides password recovery
  \item
    Tells the router to look in NVRAM for the boot sequence
  \item
    Boots the IOS from a TFTP server
  \item
    Boots an IOS image stored in ROM
  \end{enumerate}
\end{enumerate}

\protect\hypertarget{c09.xhtml}{}{}

\section[{Chapter 9}\\
{IP
Routing}]{\texorpdfstring{\protect\hypertarget{c09.xhtmlux5cux23c09}{}{}\protect\hypertarget{c09.xhtmlux5cux23Page_357}{}{}{Chapter
9}\\
{IP Routing}}{Chapter 9 IP Routing}}

\begin{center}\rule{0.5\linewidth}{0.5pt}\end{center}

\subsection{The following ICND1 exam topics are covered in this
chapter:}

\begin{enumerate}
\tightlist
\item
  \includegraphics{images/tick.png} \textbf{3.0 Routing Technologies}

  \begin{enumerate}
  \tightlist
  \item
    \includegraphics{images/squ.png} 3.1 Describe the routing concepts

    \begin{enumerate}
    \tightlist
    \item
      \includegraphics{images/squ.png} 3.1.a Packet handling along the
      path through a network
    \item
      \includegraphics{images/squ.png} 3.1.b Forwarding decision based
      on route lookup
    \item
      \includegraphics{images/squ.png} 3.1.c Frame rewrite
    \end{enumerate}
  \item
    \includegraphics{images/squ.png} 3.2 Interpret the components of
    routing table

    \begin{enumerate}
    \tightlist
    \item
      \includegraphics{images/squ.png} 3.2.a Prefix
    \item
      \includegraphics{images/squ.png} 3.2.b Network mask
    \item
      \includegraphics{images/squ.png} 3.2.c Next hop
    \item
      \includegraphics{images/squ.png} 3.2.d Routing protocol code
    \item
      \includegraphics{images/squ.png} 3.2.e Administrative distance
    \item
      \includegraphics{images/squ.png} 3.2.f Metric
    \item
      \includegraphics{images/squ.png} 3.2.g Gateway of last resort
    \end{enumerate}
  \item
    \includegraphics{images/squ.png} 3.3 Describe how a routing table is
    populated by different routing information sources

    \begin{enumerate}
    \tightlist
    \item
      \includegraphics{images/squ.png} 3.3.a Admin distance
    \end{enumerate}
  \item
    \includegraphics{images/squ.png} 3.5 Compare and contrast static
    routing and dynamic routing

    \begin{enumerate}
    \tightlist
    \item
      \includegraphics{images/squ.png} 3.6 Configure, verify, and
      troubleshoot IPv4 and IPv6 static routing
    \item
      \includegraphics{images/squ.png} 3.6.a Default route
    \item
      \includegraphics{images/squ.png} 3.6.b Network route
    \item
      \includegraphics{images/squ.png} 3.6.c Host route
    \item
      \includegraphics{images/squ.png} 3.6.d Floating static
    \end{enumerate}
  \item
    \includegraphics{images/squ.png} 3.7 Configure, verify, and
    troubleshoot RIPv2 for IPv4 (excluding authentication, filtering,
    manual summarization, redistribution)
  \end{enumerate}
\end{enumerate}

\protect\hypertarget{c09.xhtmlux5cux23Page_358}{}{}\includegraphics{images/intro.png}It's
time now to turn our focus toward the core topic of the ubiquitous IP
routing process. It's integral to networking because it pertains to all
routers and configurations that use it, which is easily the lion's
share. IP routing is basically the process of moving packets from one
network to another network using routers. And by routers, I mean Cisco
routers, of course! However, the terms \emph{router} and \emph{layer 3
device} are interchangeable, and throughout this chapter when I use the
term \emph{router}, I am referring to any layer 3 device.

Before jumping into this chapter, I want to make sure you understand the
difference between a \emph{routing protocol} and a \emph{routed
protocol}. Routers use routing protocols to dynamically find all
networks within the greater internetwork and to ensure that all routers
have the same routing table. Routing protocols are also employed to
determine the best path a packet should take through an internetwork to
get to its destination most efficiently. RIP, RIPv2, EIGRP, and OSPF are
great examples of the most common routing protocols.

Once all routers know about all networks, a routed protocol can be used
to send user data (packets) through the established enterprise. Routed
protocols are assigned to an interface and determine the method of
packet delivery. Examples of routed protocols are IP and IPv6.

I'm pretty confident I don't have to underscore how crucial it is for
you to have this chapter's material down to a near instinctive level. IP
routing is innately what Cisco routers do, and they do it very well, so
having a firm grasp of the fundamentals and basics of this topic is
vital if you want to excel during the exam and in a real-world
networking environment as well!

In this chapter, I'm going to show you how to configure and verify IP
routing with Cisco routers and guide you through these five key
subjects:

\begin{enumerate}
\tightlist
\item
  Routing basics
\item
  The IP routing process
\item
  Static routing
\item
  Default routing
\item
  Dynamic routing
\end{enumerate}

I want to start by nailing down the basics of how packets actually move
through an internetwork, so let's get started!

\begin{center}\rule{0.5\linewidth}{0.5pt}\end{center}

\includegraphics{images/note.png} To find up-to-the-minute updates for
this chapter, please see \texttt{www.lammle.com/ccna} or the book's web
page at \texttt{www.sybex.com/go/ccna}.

\begin{center}\rule{0.5\linewidth}{0.5pt}\end{center}

\subsection[Routing
Basics]{\texorpdfstring{\protect\hypertarget{c09.xhtmlux5cux23c09-sec-1}{}{}\protect\hypertarget{c09.xhtmlux5cux23Page_359}{}{}Routing
Basics}{Routing Basics}}

Once you create an internetwork by connecting your WANs and LANs to a
router, you'll need to configure logical network addresses, like IP
addresses, to all hosts on that internetwork for them to communicate
successfully throughout it.

The term \emph{routing} refers to taking a packet from one device and
sending it through the network to another device on a different network.
Routers don't really care about hosts---they only care about networks
and the best path to each one of them. The logical network address of
the destination host is key to getting packets through a routed network.
It's the hardware address of the host that's used to deliver the packet
from a router and ensure it arrives at the correct destination host.

Routing is irrelevant if your network has no routers because their job
is to route traffic to all the networks in your internetwork, but this
is rarely the case! So here's an important list of the minimum factors a
router must know to be able to effectively route packets:

\begin{enumerate}
\tightlist
\item
  Destination address
\item
  Neighbor routers from which it can learn about remote networks
\item
  Possible routes to all remote networks
\item
  The best route to each remote network
\item
  How to maintain and verify routing information
\end{enumerate}

The router learns about remote networks from neighboring routers or from
an administrator. The router then builds a routing table, which is
basically a map of the internetwork, and it describes how to find remote
networks. If a network is directly connected, then the router already
knows how to get to it.

But if a network isn't directly connected to the router, the router must
use one of two ways to learn how to get to the remote network. The
\emph{static routing} method requires someone to hand-type all network
locations into the routing table, which can be a pretty daunting task
when used on all but the smallest of networks!

Conversely, when \emph{dynamic routing} is used, a protocol on one
router communicates with the same protocol running on neighboring
routers. The routers then update each other about all the networks they
know about and place this information into the routing table. If a
change occurs in the network, the dynamic routing protocols
automatically inform all routers about the event. If static routing is
used, the administrator is responsible for updating all changes by hand
onto all routers. Most people usually use a combination of dynamic and
static routing to administer a large network.

Before we jump into the IP routing process, let's take a look at a very
simple example that demonstrates how a router uses the routing table to
route packets out of an interface. We'll be going into a more detailed
study of the process soon, but I want to show you something called the
``longest match rule'' first. With it, IP will scan a routing table to
find the longest match as compared to the destination address of a
packet. Let's take a look at
\protect\hyperlink{c09.xhtmlux5cux23figure9-1}{Figure 9.1} to get a
picture of this process.

\protect\hypertarget{c09.xhtmlux5cux23Page_360}{}{}

\begin{figure}
\centering
\includegraphics{images/c09f001.jpg}
\caption{{\protect\hyperlink{c09.xhtmlux5cux23figureanchor9-1}{\textbf{FIGURE
9.1}} A simple routing example}}
\end{figure}

\protect\hyperlink{c09.xhtmlux5cux23figure9-1}{Figure 9.1} shows a
simple network. Lab\_A has four interfaces. Can you see which interface
will be used to forward an IP datagram to a host with a destination IP
address of 10.10.10.30?

By using the command \texttt{show\ ip\ route} on a router, we can see
the routing table (map of the internetwork) that Lab\_A has used to make
its forwarding decisions:

\begin{verbatim}
Lab_A#sh ip route
Codes: L - local, C - connected, S - static,
[output cut]
        10.0.0.0/8 is variably subnetted, 6 subnets, 4 masks
C       10.0.0.0/8 is directly connected, FastEthernet0/3
L       10.0.0.1/32 is directly connected, FastEthernet0/3
C       10.10.0.0/16 is directly connected, FastEthernet0/2
L       10.10.0.1/32 is directly connected, FastEthernet0/2
C       10.10.10.0/24 is directly connected, FastEthernet0/1
L       10.10.10.1/32 is directly connected, FastEthernet0/1
S*      0.0.0.0/0 is directly connected, FastEthernet0/0
\end{verbatim}

The \texttt{C} in the routing table output means that the networks
listed are ``directly connected,'' and until we add a routing protocol
like RIPv2, OSPF, etc. to the routers in our internetwork, or enter
static routes, only directly connected networks will show up in our
routing table. But wait---what about that \texttt{L} in the routing
table---that's new, isn't it? Yes it is, because in the new Cisco IOS 15
code, Cisco defines a different route, called a local host route. Each
local route has a /32 prefix, defining a route just for the one address.
So in this example, the router has relied upon these routes that list
their own local IP addresses to more efficiently forward packets to the
router itself.

\protect\hypertarget{c09.xhtmlux5cux23Page_361}{}{}So let's get back to
the original question: By looking at the figure and the output of the
routing table, can you determine what IP will do with a received packet
that has a destination IP address of 10.10.10.30? The answer is that the
router will packet-switch the packet to interface FastEthernet 0/1,
which will frame the packet and then send it out on the network segment.
This is referred to as frame rewrite. Based upon the longest match rule,
IP would look for 10.10.10.30, and if that isn't found in the table,
then IP would search for 10.10.10.0, then 10.10.0.0, and so on until a
route is discovered.

Here's another example: Based on the output of the next routing table,
which interface will a packet with a destination address of 10.10.10.14
be forwarded from?

\begin{verbatim}
Lab_A#sh ip route
[output cut]
Gateway of last resort is not set
C      10.10.10.16/28 is directly connected, FastEthernet0/0
L      10.10.10.17/32 is directly connected, FastEthernet0/0
C      10.10.10.8/29 is directly connected, FastEthernet0/1
L      10.10.10.9/32 is directly connected, FastEthernet0/1
C      10.10.10.4/30 is directly connected, FastEthernet0/2
L      10.10.10.5/32 is directly connected, FastEthernet0/2
C      10.10.10.0/30 is directly connected, Serial 0/0
L      10.10.10.1/32 is directly connected, Serial0/0
\end{verbatim}

To figure this out, look closely at the output until you see that the
network is subnetted and each interface has a different mask. And I have
to tell you---you just can't answer this question if you can't subnet!
10.10.10.14 would be a host in the 10.10.10.8/29 subnet that's connected
to the FastEthernet0/1 interface. Don't freak if you're struggling and
don't get this! Instead, just go back and reread Chapter 4, ``Easy
Subnetting,'' until it becomes clear to you.

\subsection[The IP Routing
Process]{\texorpdfstring{\protect\hypertarget{c09.xhtmlux5cux23c09-sec-2}{}{}The
IP Routing Process}{The IP Routing Process}}

The IP routing process is fairly simple and doesn't change, regardless
of the size of your network. For a good example of this fact, I'll use
\protect\hyperlink{c09.xhtmlux5cux23figure9-2}{Figure 9.2} to describe
step-by-step what happens when Host A wants to communicate with Host B
on a different network.

\begin{figure}
\centering
\includegraphics{images/c09f002.jpg}
\caption{{\protect\hyperlink{c09.xhtmlux5cux23figureanchor9-2}{\textbf{FIGURE
9.2}} IP routing example using two hosts and one router}}
\end{figure}

\protect\hypertarget{c09.xhtmlux5cux23Page_362}{}{}In
\protect\hyperlink{c09.xhtmlux5cux23figure9-2}{Figure 9.2} a user on
Host\_A pinged Host\_B's IP address. Routing doesn't get any simpler
than this, but it still involves a lot of steps, so let's work through
them now:

\begin{enumerate}
\item
  Internet Control Message Protocol (ICMP) creates an echo request
  payload, which is simply the alphabet in the data field.
\item
  ICMP hands that payload to Internet Protocol (IP), which then creates
  a packet. At a minimum, this packet contains an IP source address, an
  IP destination address, and a Protocol field with 01h. Don't forget
  that Cisco likes to use \emph{0x} in front of hex characters, so this
  could also look like 0x01. This tells the receiving host to whom it
  should hand the payload when the destination is reached---in this
  example, ICMP.
\item
  Once the packet is created, IP determines whether the destination IP
  address is on the local network or a remote one.
\item
  Since IP has determined that this is a remote request, the packet must
  be sent to the default gateway so it can be routed to the remote
  network. The Registry in Windows is parsed to find the configured
  default gateway.
\item
  The default gateway of Host\_A is configured to 172.16.10.1. For this
  packet to be sent to the default gateway, the hardware address of the
  router's interface Ethernet 0, which is configured with the IP address
  of 172.16.10.1, must be known. Why? So the packet can be handed down
  to the Data Link layer, framed, and sent to the router's interface
  that's connected to the 172.16.10.0 network. Because hosts communicate
  only via hardware addresses on the local LAN, it's important to
  recognize that for Host\_A to communicate to Host\_B, it has to send
  packets to the Media Access Control (MAC) address of the default
  gateway on the local network.

  \begin{center}\rule{0.5\linewidth}{0.5pt}\end{center}

  \includegraphics{images/note.png} MAC addresses are always local on
  the LAN and never go through and past a router.

  \begin{center}\rule{0.5\linewidth}{0.5pt}\end{center}
\item
  Next, the Address Resolution Protocol (ARP) cache of the host is
  checked to see if the IP address of the default gateway has already
  been resolved to a hardware address.

  If it has, the packet is then free to be handed to the Data Link layer
  for framing. Remember that the hardware destination address is also
  handed down with that packet. To view the ARP cache on your host, use
  the following command:

\begin{verbatim}
C:\>arp -a
Interface: 172.16.10.2 --- 0x3
  Internet Address      Physical Address      Type
  172.16.10.1          00-15-05-06-31-b0     dynamic
\end{verbatim}

  If the hardware address isn't already in the ARP cache of the host, an
  ARP broadcast will be sent out onto the local network to search for
  the 172.16.10.1 hardware address. The router then responds to the
  request and provides the hardware address of Ethernet 0, and the host
  caches this address.
\item
  Once the packet and destination hardware address are handed to the
  Data Link layer, the LAN driver is used to provide media access via
  the type of LAN being used, which
  \protect\hypertarget{c09.xhtmlux5cux23Page_363}{}{}is Ethernet in this
  case. A frame is then generated, encapsulating the packet with control
  information. Within that frame are the hardware destination and source
  addresses plus, in this case, an Ether-Type field, which identifies
  the specific Network layer protocol that handed the packet to the Data
  Link layer. In this instance, it's IP. At the end of the frame is
  something called a Frame Check Sequence (FCS) field that houses the
  result of the cyclic redundancy check (CRC). The frame would look
  something like what I've detailed in
  \protect\hyperlink{c09.xhtmlux5cux23figure9-3}{Figure 9.3}. It
  contains Host A's hardware (MAC) address and the destination hardware
  address of the default gateway. It does not include the remote host's
  MAC address---remember that!

  \begin{figure}
  \centering
  \includegraphics{images/c09f003.jpg}
  \caption{{\protect\hyperlink{c09.xhtmlux5cux23figureanchor9-3}{\textbf{FIGURE
  9.3}} Frame used from Host A to the Lab\_A router when Host B is
  pinged}}
  \end{figure}
\item
  Once the frame is completed, it's handed down to the Physical layer to
  be put on the physical medium (in this example, twisted-pair wire) one
  bit at a time.
\item
  Every device in the collision domain receives these bits and builds
  the frame. They each run a CRC and check the answer in the FCS field.
  If the answers don't match, the frame is discarded.

  \begin{enumerate}
  \tightlist
  \item
    If the CRC matches, then the hardware destination address is checked
    to see if it matches (which, in this example, is the router's
    interface Ethernet 0).
  \item
    If it's a match, then the Ether-Type field is checked to find the
    protocol used at the Network layer.
  \end{enumerate}
\item
  The packet is pulled from the frame, and what is left of the frame is
  discarded. The packet is handed to the protocol listed in the
  Ether-Type field---it's given to IP.
\item
  IP receives the packet and checks the IP destination address. Since
  the packet's destination address doesn't match any of the addresses
  configured on the receiving router itself, the router will look up the
  destination IP network address in its routing table.
\item
  The routing table must have an entry for the network 172.16.20.0 or
  the packet will be discarded immediately and an ICMP message will be
  sent back to the originating device with a destination network
  unreachable message.
\item
  If the router does find an entry for the destination network in its
  table, the packet is switched to the exit interface---in this example,
  interface Ethernet 1. The following output displays the Lab\_A
  router's routing table. The \texttt{C} means ``directly connected.''
  No routing protocols are needed in this network since all networks
  (all two of them) are directly connected.

\begin{verbatim}
Lab_A>sh ip route
C       172.16.10.0 is directly connected,    Ethernet0
L       172.16.10.1/32 is directly connected, Ethernet0
C       172.16.20.0 is directly connected,    Ethernet1
L       172.16.20.1/32 is directly connected, Ethernet1
\end{verbatim}
\item
  \protect\hypertarget{c09.xhtmlux5cux23Page_364}{}{}The router
  packet-switches the packet to the Ethernet 1 buffer.
\item
  The Ethernet 1 buffer needs to know the hardware address of the
  destination host and first checks the ARP cache.

  \begin{enumerate}
  \item
    If the hardware address of Host\_B has already been resolved and is
    in the router's ARP cache, then the packet and the hardware address
    will be handed down to the Data Link layer to be framed. Let's take
    a look at the ARP cache on the Lab\_A router by using the
    \texttt{show\ ip\ arp} command:

\begin{verbatim}
Lab_A#sh ip arp
Protocol  Address     Age(min) Hardware Addr  Type   Interface
Internet  172.16.20.1   -     00d0.58ad.05f4  ARPA   Ethernet1
Internet  172.16.20.2   3     0030.9492.a5dd  ARPA   Ethernet1
Internet  172.16.10.1   -     00d0.58ad.06aa  ARPA   Ethernet0
Internet  172.16.10.2  12     0030.9492.a4ac  ARPA   Ethernet0
\end{verbatim}

    The dash (-) signifies that this is the physical interface on the
    router. This output shows us that the router knows the 172.16.10.2
    (Host\_A) and 172.16.20.2 (Host\_B) hardware addresses. Cisco
    routers will keep an entry in the ARP table for 4 hours.
  \item
    Now if the hardware address hasn't already been resolved, the router
    will send an ARP request out E1 looking for the 172.16.20.2 hardware
    address. Host\_B responds with its hardware address, and the packet
    and destination hardware addresses are then both sent to the Data
    Link layer for framing.
  \end{enumerate}
\item
  The Data Link layer creates a frame with the destination and source
  hardware addresses, Ether-Type field, and FCS field at the end. The
  frame is then handed to the Physical layer to be sent out on the
  physical medium one bit at a time.
\item
  Host\_B receives the frame and immediately runs a CRC. If the result
  matches the information in the FCS field, the hardware destination
  address will then be checked next. If the host finds a match, the
  Ether-Type field is then checked to determine the protocol that the
  packet should be handed to at the Network layer---IP in this example.
\item
  At the Network layer, IP receives the packet and runs a CRC on the IP
  header. If that passes, IP then checks the destination address. Since
  a match has finally been made, the Protocol field is checked to find
  out to whom the payload should be given.
\item
  The payload is handed to ICMP, which understands that this is an echo
  request. ICMP responds to this by immediately discarding the packet
  and generating a new payload as an echo reply.
\item
  A packet is then created including the source and destination
  addresses, Protocol field, and payload. The destination device is now
  Host\_A.
\item
  IP then checks to see whether the destination IP address is a device
  on the local LAN or on a remote network. Since the destination device
  is on a remote network, the packet needs to be sent to the default
  gateway.
\item
  \protect\hypertarget{c09.xhtmlux5cux23Page_365}{}{}The default gateway
  IP address is found in the Registry of the Windows device, and the ARP
  cache is checked to see if the hardware address has already been
  resolved from an IP address.
\item
  Once the hardware address of the default gateway is found, the packet
  and destination hardware addresses are handed down to the Data Link
  layer for framing.
\item
  The Data Link layer frames the packet of information and includes the
  following in the header:

  \begin{enumerate}
  \tightlist
  \item
    The destination and source hardware addresses
  \item
    The Ether-Type field with 0x0800 (IP) in it
  \item
    The FCS field with the CRC result in tow
  \end{enumerate}
\item
  The frame is now handed down to the Physical layer to be sent out over
  the network medium one bit at a time.
\item
  The router's Ethernet 1 interface receives the bits and builds a
  frame. The CRC is run, and the FCS field is checked to make sure the
  answers match.
\item
  Once the CRC is found to be okay, the hardware destination address is
  checked. Since the router's interface is a match, the packet is pulled
  from the frame and the Ether-Type field is checked to determine which
  protocol the packet should be delivered to at the Network layer.
\item
  The protocol is determined to be IP, so it gets the packet. IP runs a
  CRC check on the IP header first and then checks the destination IP
  address.

  \begin{center}\rule{0.5\linewidth}{0.5pt}\end{center}

  \includegraphics{images/note.png} IP does not run a complete CRC as
  the Data Link layer does---it only checks the header for errors.

  \begin{center}\rule{0.5\linewidth}{0.5pt}\end{center}

  Since the IP destination address doesn't match any of the router's
  interfaces, the routing table is checked to see whether it has a route
  to 172.16.10.0. If it doesn't have a route over to the destination
  network, the packet will be discarded immediately. I want to take a
  minute to point out that this is exactly where the source of confusion
  begins for a lot of administrators because when a ping fails, most
  people think the packet never reached the destination host. But as we
  see here, that's not \emph{always} the case. All it takes for this to
  happen is for even just one of the remote routers to lack a route back
  to the originating host's network and--- \emph{poof!}---the packet is
  dropped on the \emph{return trip}, not on its way to the host!

  \begin{center}\rule{0.5\linewidth}{0.5pt}\end{center}

  \includegraphics{images/tip.png} Just a quick note to mention that
  when (and if) the packet is lost on the way back to the originating
  host, you will typically see a request timed-out message because it is
  an unknown error. If the error occurs because of a known issue, such
  as if a route is not in the routing table on the way to the
  destination device, you will see a destination unreachable message.
  This should help you determine if the problem occurred on the way to
  the destination or on the way back.

  \begin{center}\rule{0.5\linewidth}{0.5pt}\end{center}
\item
  \protect\hypertarget{c09.xhtmlux5cux23Page_366}{}{}In this case, the
  router happens to know how to get to network 172.16.10.0---the exit
  interface is Ethernet 0---so the packet is switched to interface
  Ethernet 0.
\item
  The router then checks the ARP cache to determine whether the hardware
  address for 172.16.10.2 has already been resolved.
\item
  Since the hardware address to 172.16.10.2 is already cached from the
  originating trip to Host\_B, the hardware address and packet are then
  handed to the Data Link layer.
\item
  The Data Link layer builds a frame with the destination hardware
  address and source hardware address and then puts IP in the Ether-Type
  field. A CRC is run on the frame and the result is placed in the FCS
  field.
\item
  The frame is then handed to the Physical layer to be sent out onto the
  local network one bit at a time.
\item
  The destination host receives the frame, runs a CRC, checks the
  destination hardware address, then looks into the Ether-Type field to
  find out to whom to hand the packet.
\item
  IP is the designated receiver, and after the packet is handed to IP at
  the Network layer, it checks the Protocol field for further direction.
  IP finds instructions to give the payload to ICMP, and ICMP determines
  the packet to be an ICMP echo reply.
\item
  ICMP acknowledges that it has received the reply by sending an
  exclamation point (!) to the user interface. ICMP then attempts to
  send four more echo requests to the destination host.
\end{enumerate}

You've just experienced Todd's 36 easy steps to understanding IP
routing. The key point here is that if you had a much larger network,
the process would be the \emph{same}. It's just that the larger the
internetwork, the more hops the packet goes through before it finds the
destination host.

It's super-important to remember that when Host\_A sends a packet to
Host\_B, the destination hardware address used is the default gateway's
Ethernet interface. Why? Because frames can't be placed on remote
networks---only local networks. So packets destined for remote networks
must go through the default gateway.

Let's take a look at Host\_A's ARP cache now:

\begin{verbatim}
C:\ >arp -a
Interface: 172.16.10.2 --- 0x3
  Internet Address      Physical Address      Type
  172.16.10.1           00-15-05-06-31-b0     dynamic
  172.16.20.1           00-15-05-06-31-b0     dynamic
\end{verbatim}

Did you notice that the hardware (MAC) address that Host\_A uses to get
to Host\_B is the Lab\_A E0 interface? Hardware addresses are
\emph{always} local, and they never pass through a router's interface.
Understanding this process is as important as air to you, so carve this
into your memory!

\subsubsection[The Cisco Router Internal
Process]{\texorpdfstring{\protect\hypertarget{c09.xhtmlux5cux23c09-sec-3}{}{}The
Cisco Router Internal Process}{The Cisco Router Internal Process}}

One more thing before we get to testing your understanding of my 36
steps of IP routing. I think it's important to explain how a router
forwards packets internally. For IP to look up a
\protect\hypertarget{c09.xhtmlux5cux23Page_367}{}{}destination address
in a routing table on a router, processing in the router must take
place, and if there are tens of thousands of routes in that table, the
amount of CPU time would be enormous. It results in a potentially
overwhelming amount of overhead---think about a router at your ISP that
has to calculate millions of packets per second and even subnet to find
the correct exit interface! Even with the little network I'm using in
this book, lots of processing would need to be done if there were actual
hosts connected and sending data.

Cisco uses three types of packet-forwarding techniques.

\textbf{Process switching} This is actually how many people see routers
to this day, because it's true that routers actually did perform this
type of bare-bones packet switching back in 1990 when Cisco released
their very first router. But those days when traffic demands were
unimaginably light are long gone---not in today's networks! This process
is now extremely complex and involves looking up every destination in
the routing table and finding the exit interface for every packet. This
is pretty much how I just explained the process in my 36 steps. But even
though what I wrote was absolutely true in concept, the internal process
requires much more than packet-switching technology today because of the
millions of packets per second that must now be processed. So Cisco came
up with some other technologies to help with the ``big process
problem.''

\textbf{Fast switching} This solution was created to make the slow
performance of process switching faster and more efficient. Fast
switching uses a cache to store the most recently used destinations so
that lookups are not required for every packet. By caching the exit
interface of the destination device, as well as the layer 2 header,
performance was dramatically improved, but as our networks evolved with
the need for even more speed, Cisco created yet another technology!

\textbf{Cisco Express Forwarding (CEF)} This is Cisco's newer creation,
and it's the default packet-forwarding method used on all the latest
Cisco routers. CEF makes many different cache tables to help improve
performance and is change triggered, not packet triggered. Translated,
this means that when the network topology changes, the cache changes
along with it.

\begin{center}\rule{0.5\linewidth}{0.5pt}\end{center}

\includegraphics{images/tip.png} To see which packet switching method
your router interface is using, use the command
\texttt{show\ ip\ interface}.

\begin{center}\rule{0.5\linewidth}{0.5pt}\end{center}

\subsubsection[Testing Your IP Routing
Understanding]{\texorpdfstring{\protect\hypertarget{c09.xhtmlux5cux23c09-sec-4}{}{}Testing
Your IP Routing Understanding}{Testing Your IP Routing Understanding}}

Since understanding IP routing is super-important, it's time for that
little test I talked about earlier on how well you've got the IP routing
process down so far. I'm going to do that by having you look at a couple
of figures and answer some very basic IP routing questions based upon
them.

\protect\hyperlink{c09.xhtmlux5cux23figure9-4}{Figure 9.4} shows a LAN
connected to RouterA that's connected via a WAN link to RouterB. RouterB
has a LAN connected with an HTTP server attached.

\protect\hypertarget{c09.xhtmlux5cux23Page_368}{}{}

\begin{figure}
\centering
\includegraphics{images/c09f004.jpg}
\caption{{\protect\hyperlink{c09.xhtmlux5cux23figureanchor9-4}{\textbf{FIGURE
9.4}} IP routing example 1}}
\end{figure}

The critical information you want to obtain by looking at this figure is
exactly how IP routing will occur in this example. Let's determine the
characteristics of a frame as it leaves HostA. Okay---we'll cheat a bit.
I'll give you the answer, but then you should go back over the figure
and see if you can answer example 2 without looking at my three-step
answer!

\begin{enumerate}
\tightlist
\item
  The destination address of a frame from HostA would be the MAC address
  of Router A's Fa0/0 interface.
\item
  The destination address of a packet would be the IP address of the
  HTTP server's network interface card (NIC).
\item
  The destination port number in the segment header would be 80.
\end{enumerate}

That was a pretty simple, straightforward scenario. One thing to
remember is that when multiple hosts are communicating to a server using
HTTP, they must all use a different source port number. The source and
destination IP addresses and port numbers are how the server keeps the
data separated at the Transport layer.

Let's complicate matters by adding another device into the network and
then see if you can find the answers.
\protect\hyperlink{c09.xhtmlux5cux23figure9-5}{Figure 9.5} shows a
network with only one router but two switches.

\begin{figure}
\centering
\includegraphics{images/c09f005.jpg}
\caption{{\protect\hyperlink{c09.xhtmlux5cux23figureanchor9-5}{\textbf{FIGURE
9.5}} IP routing example 2}}
\end{figure}

\protect\hypertarget{c09.xhtmlux5cux23Page_369}{}{}The key thing to
understand about the IP routing process in this scenario is what happens
when HostA sends data to the HTTPS server? Here's your answer:

\begin{enumerate}
\tightlist
\item
  The destination address of a frame from HostA would be the MAC address
  of RouterA's Fa0/0 interface.
\item
  The destination address of a packet is the IP address of the HTTPS
  server's network interface card (NIC).
\item
  The destination port number in the segment header will have a value of
  443.
\end{enumerate}

Did you notice that the switches weren't used as either a default
gateway or any other destination? That's because switches have nothing
to do with routing. I wonder how many of you chose the switch as the
default gateway (destination) MAC address for HostA? If you did, don't
feel bad---just take another look to see where you went wrong and why.
It's very important to remember that the destination MAC address will
always be the router's interface---if your packets are destined for
outside the LAN, as they were in these last two examples!

Before moving on into some of the more advanced aspects of IP routing,
let's look at another issue. Take a look at the output of this router's
routing table:

\begin{verbatim}
Corp#sh ip route
[output cut]
R    192.168.215.0 [120/2] via 192.168.20.2, 00:00:23, Serial0/0
R    192.168.115.0 [120/1] via 192.168.20.2, 00:00:23, Serial0/0
R    192.168.30.0 [120/1] via 192.168.20.2, 00:00:23, Serial0/0
C    192.168.20.0 is directly connected, Serial0/0
L    192.168.20.1/32 is directly connected, Serial0/0
C    192.168.214.0 is directly connected, FastEthernet0/0
L    192.168.214.1/32 is directly connected, FastEthernet0/0
\end{verbatim}

What do we see here? If I were to tell you that the corporate router
received an IP packet with a source IP address of 192.168.214.20 and a
destination address of 192.168.22.3, what do you think the Corp router
will do with this packet?

If you said, ``The packet came in on the FastEthernet 0/0 interface, but
because the routing table doesn't show a route to network 192.168.22.0
(or a default route), the router will discard the packet and send an
ICMP destination unreachable message back out to interface FastEthernet
0/0,'' you're a genius! The reason that's the correct answer is because
that's the source LAN where the packet originated from.

Now, let's check out the next figure and talk about the frames and
packets in detail. We're not really going over anything new here; I'm
just making sure you totally, completely, thoroughly, fully understand
basic IP routing! It is the crux of this book, and the topic the exam
objectives are geared toward. It's all about IP routing, which means you
need to be all over this stuff! We'll use
\protect\hyperlink{c09.xhtmlux5cux23figure9-6}{Figure 9.6} for the next
few scenarios.

\protect\hypertarget{c09.xhtmlux5cux23Page_370}{}{}

\begin{figure}
\centering
\includegraphics{images/c09f006.jpg}
\caption{{\protect\hyperlink{c09.xhtmlux5cux23figureanchor9-6}{\textbf{FIGURE
9.6}} Basic IP routing using MAC and IP addresses}}
\end{figure}

Referring to \protect\hyperlink{c09.xhtmlux5cux23figure9-6}{Figure 9.6},
here's a list of all the answers to questions you need inscribed in your
brain:

\begin{enumerate}
\tightlist
\item
  In order to begin communicating with the Sales server, Host 4 sends
  out an ARP request. How will the devices exhibited in the topology
  respond to this request?
\item
  Host 4 has received an ARP reply. Host 4 will now build a packet, then
  place this packet in the frame. What information will be placed in the
  header of the packet that leaves Host 4 if Host 4 is going to
  communicate to the Sales server?
\item
  The Lab\_A router has received the packet and will send it out Fa0/0
  onto the LAN toward the server. What will the frame have in the header
  as the source and destination addresses?
\item
  Host 4 is displaying two web documents from the Sales server in two
  browser windows at the same time. How did the data find its way to the
  correct browser windows?
\end{enumerate}

The following should probably be written in a teensy font and put upside
down in another part of the book so it would be really hard for you to
cheat and peek, but since I'm not that mean and you really need to have
this down, here are your answers in the same order that the scenarios
were just presented:

\begin{enumerate}
\tightlist
\item
  In order to begin communicating with the server, Host 4 sends out an
  ARP request. How will the devices exhibited in the topology respond to
  this request? Since MAC addresses must stay on the local network, the
  Lab\_B router will respond with the MAC address of the Fa0/0 interface
  and Host 4 will send all frames to the MAC address of the Lab\_B Fa0/0
  interface when sending packets to the Sales server.
\item
  Host 4 has received an ARP reply. Host 4 will now build a packet, then
  place this packet in the frame. What information will be placed in the
  header of the packet that leaves Host 4 if Host 4 is going to
  communicate to the Sales server? Since we're now talking about
  packets, not frames, the source address will be the IP address of Host
  4 and the destination address will be the IP address of the Sales
  server.
\item
  Finally, the Lab\_A router has received the packet and will send it
  out Fa0/0 onto the LAN toward the server. What will the frame have in
  the header as the source and ­destination addresses? The source MAC
  address will be the Lab\_A router's Fa0/0 interface, and the
  \protect\hypertarget{c09.xhtmlux5cux23Page_371}{}{}destination MAC
  address will be the Sales server's MAC address because all MAC
  addresses must be local on the LAN.
\item
  Host 4 is displaying two web documents from the Sales server in two
  different browser windows at the same time. How did the data find its
  way to the correct browser windows? TCP port numbers are used to
  direct the data to the correct application window.
\end{enumerate}

Great! But we're not quite done yet. I've got a few more questions for
you before you actually get to configure routing in a real network.
Ready? \protect\hyperlink{c09.xhtmlux5cux23figure9-7}{Figure 9.7} shows
a basic network, and Host 4 needs to get email. Which address will be
placed in the destination address field of the frame when it leaves Host
4?

\begin{figure}
\centering
\includegraphics{images/c09f007.jpg}
\caption{{\protect\hyperlink{c09.xhtmlux5cux23figureanchor9-7}{\textbf{FIGURE
9.7}} Testing basic routing knowledge}}
\end{figure}

The answer is that Host 4 will use the destination MAC address of the
Fa0/0 interface on the Lab\_B router---you knew that, right? Look at
\protect\hyperlink{c09.xhtmlux5cux23figure9-7}{Figure 9.7} again: What
if Host 4 needs to communicate with Host 1---not the server, but with
Host 1. Which OSI layer 3 source address will be found in the packet
header when it reaches Host 1?

Hopefully you've got this: At layer 3, the source IP address will be
Host 4 and the destination address in the packet will be the IP address
of Host 1. Of course, the destination MAC address from Host 4 will
always be the Fa0/0 address of the Lab\_B router, right? And since we
have more than one router, we'll need a routing protocol that
communicates between both of them so that traffic can be forwarded in
the right direction to reach the network that Host 1 is connected to.

Okay---one more scenario and you're on your way to being an IP routing
machine! Again, using
\protect\hyperlink{c09.xhtmlux5cux23figure9-7}{Figure 9.7}, Host 4 is
transferring a file to the email server connected to the Lab\_A router.
What would be the layer 2 destination address leaving Host 4? Yes, I've
asked this question more than once. But not this one: What will be the
source MAC address when the frame is received at the email server?

\protect\hypertarget{c09.xhtmlux5cux23Page_372}{}{}Hopefully, you
answered that the layer 2 destination address leaving Host 4 is the MAC
address of the Fa0/0 interface on the Lab\_B router and that the source
layer 2 address that the email server will receive is the Fa0/0
interface of the Lab\_A router.

If you did, you're ready to discover how IP routing is handled in a
larger network environment!

\subsection[Configuring IP
Routing]{\texorpdfstring{\protect\hypertarget{c09.xhtmlux5cux23c09-sec-5}{}{}Configuring
IP Routing}{Configuring IP Routing}}

It's time to get serious and configure a real network.
\protect\hyperlink{c09.xhtmlux5cux23figure9-8}{Figure 9.8} shows three
routers: Corp, SF, and LA. Remember that, by default, these routers only
know about networks that are directly connected to them. I'll continue
to use this figure and network throughout the rest of the chapters in
this book. As I progress through this book, I'll add more routers and
switches as needed.

\begin{figure}
\centering
\includegraphics{images/c09f008.jpg}
\caption{{\protect\hyperlink{c09.xhtmlux5cux23figureanchor9-8}{\textbf{FIGURE
9.8}} Configuring IP routing}}
\end{figure}

As you might guess, I've got quite a nice collection of routers for us
to play with. But you don't need a closet full of devices to perform
most, if not all, of the commands we'll use in this book. You can get by
nicely with pretty much any router or even with a good router simulator.

Getting back to business, the Corp router has two serial interfaces,
which will provide a WAN connection to the SF and LA router and two Fast
Ethernet interfaces as well. The two remote routers have two serial
interfaces and two Fast Ethernet interfaces.

The first step for this project is to correctly configure each router
with an IP address on each interface. The following list shows the IP
address scheme I'm going to use to configure the network. After we go
over how the network is configured, I'll cover how to configure
\protect\hypertarget{c09.xhtmlux5cux23Page_373}{}{}IP routing. Pay
attention to the subnet masks---they're important! The LANs all use a
/24 mask, but the WANs are using a /30.

Corp

\begin{enumerate}
\tightlist
\item
  Serial 0/0: 172.16.10.1/30
\item
  Serial 0/1: 172.16.10.5/30
\item
  Fa0/0: 10.10.10.1/24
\end{enumerate}

SF

\begin{enumerate}
\tightlist
\item
  S0/0/0: 172.16.10.2/30
\item
  Fa0/0: 192.168.10.1/24
\end{enumerate}

LA

\begin{enumerate}
\tightlist
\item
  S0/0/0: 172.16.10.6/30
\item
  Fa0/0: 192.168.20.1/24
\end{enumerate}

The router configuration is really a pretty straightforward process
since you just need to add IP addresses to your interfaces and then
perform a \texttt{no\ shutdown} on those same interfaces. It gets a tad
more complex later on, but for right now, let's configure the IP
addresses in the network.

\subsubsection[Corp
Configuration]{\texorpdfstring{\protect\hypertarget{c09.xhtmlux5cux23c09-sec-6}{}{}Corp
Configuration}{Corp Configuration}}

We need to configure three interfaces to configure the Corp router. And
configuring the hostnames of each router will make identification much
easier. While we're at it, let's set the interface descriptions, banner,
and router passwords too because it's a really good idea to make a habit
of configuring these commands on every router!

To get started, I performed an \texttt{erase\ startup-config} on the
router and reloaded, so we'll start in setup mode. I chose \texttt{no}
when prompted to enter setup mode, which will get us straight to the
username prompt of the console. I'm going to configure all my routers
this same way.

Here's how what I just did looks:

\begin{verbatim}
         --- System Configuration Dialog ---
Would you like to enter the initial configuration dialog? [yes/no]: n
 
Press RETURN to get started!
Router>en
Router#config t
Router(config)#hostname Corp
Corp(config)#enable secret GlobalNet
Corp(config)#no ip domain-lookup
Corp(config)#int f0/0
Corp(config-if)#desc Connection to LAN BackBone
Corp(config-if)#ip address 10.10.10.1 255.255.255.0
Corp(config-if)#no shut
Corp(config-if)#int s0/0
Corp(config-if)#desc WAN connection to SF
Corp(config-if)#ip address 172.16.10.1 255.255.255.252
Corp(config-if)#no shut
Corp(config-if)#int s0/1
Corp(config-if)#desc WAN connection to LA
Corp(config-if)#ip address 172.16.10.5 255.255.255.252
Corp(config-if)#no shut
Corp(config-if)#line con 0
Corp(config-line)#password console
Corp(config-line)#logging
Corp(config-line)#logging sync
Corp(config-line)#exit
Corp(config)#line vty 0 ?
  <1-181>  Last Line number
  <cr>
Corp(config)#line vty 0 181
Corp(config-line)#password telnet
Corp(config-line)#login
Corp(config-line)#exit
Corp(config)#banner motd # This is my Corp Router #
Corp(config)#^Z
Corp#copy run start
Destination filename [startup-config]?
Building configuration...
[OK]
Corp# [OK]
\end{verbatim}

Let's talk about the configuration of the Corp router. First, I set the
hostname and enable secret, but what is that
\texttt{no\ ip\ domain-lookup} command? That command stops the router
from trying to resolve hostnames, which is an annoying feature unless
you've configured a host table or DNS. Next, I configured the three
interfaces with descriptions and IP addresses and enabled them with the
\texttt{no\ shutdown} command. The console and VTY passwords came next,
but what is that \texttt{logging\ sync} command under the console line?
The logging synchronous command stops console messages from writing over
what you are typing in, meaning it's a sanity-saving command that you'll
come to love! Last, I set my banner and then saved my configs.

\begin{center}\rule{0.5\linewidth}{0.5pt}\end{center}

\includegraphics{images/note.png} If you're having a hard time
understanding this configuration process, refer back to Chapter 6,
``Cisco's Internetworking Operating System (IOS).''

\begin{center}\rule{0.5\linewidth}{0.5pt}\end{center}

\protect\hypertarget{c09.xhtmlux5cux23Page_375}{}{}To view the IP
routing tables created on a Cisco router, use the command
\texttt{show\ ip\ route}. Here's the command's output:

\begin{verbatim}
Corp#sh ip route
Codes: L - local, C - connected, S - static, R - RIP, M - mobile, B - BGP
   D - EIGRP, EX - EIGRP external, O - OSPF, IA - OSPF inter area
   N1 - OSPF NSSA external type 1, N2 - OSPF NSSA external type 2
   E1 - OSPF external type 1, E2 - OSPF external type 2
   i - IS-IS, su - IS-IS summary, L1 - IS-IS level-1, L2 - IS-IS level-2
   ia - IS-IS inter area, * - candidate default, U - per-user static route
   o - ODR, P - periodic downloaded static route, H - NHRP, l - LISP
   + - replicated route, % - next hop override
Gateway of last resort is not set
 
     10.0.0.0/24 is subnetted, 1 subnets
C       10.10.10.0 is directly connected, FastEthernet0/0
L       10.10.10.1/32 is directly connected, FastEthernet0/0
Corp#
\end{verbatim}

It's important to remember that only configured, directly connected
networks are going to show up in the routing table. So why is it that
only the FastEthernet 0/0 interface shows up in the table? No
worries---that's just because you won't see the serial interfaces come
up until the other side of the links are operational. As soon as we
configure our SF and LA routers, those interfaces should pop right up!

But did you notice the \texttt{C} on the left side of the output of the
routing table? When you see that there, it means that the network is
directly connected. The codes for each type of connection are listed at
the top of the \texttt{show\ ip\ route} command, along with their
descriptions.

\begin{center}\rule{0.5\linewidth}{0.5pt}\end{center}

\includegraphics{images/note.png} For brevity, the codes at the top of
the output will be cut in the rest of this chapter.

\begin{center}\rule{0.5\linewidth}{0.5pt}\end{center}

\subsubsection[SF
Configuration]{\texorpdfstring{\protect\hypertarget{c09.xhtmlux5cux23c09-sec-7}{}{}SF
Configuration}{SF Configuration}}

Now we're ready to configure the next router---SF. To make that happen
correctly, keep in mind that we have two interfaces to deal with: Serial
0/0/0 and FastEthernet 0/0. So let's make sure we don't forget to add
the hostname, passwords, interface descriptions, and banners to the
router configuration. As I did with the Corp router, I erased the
configuration and reloaded since this router had already been configured
before.

Here's the configuration I used:

\begin{verbatim}
R1#erase start
% Incomplete command.
R1#erase startup-config
Erasing the nvram filesystem will remove all configuration files!
   Continue? [confirm][enter]
[OK]
Erase of nvram: complete
R1#reload
Proceed with reload? [confirm][enter]
[output cut]
%Error opening tftp://255.255.255.255/network-confg (Timed out)
%Error opening tftp://255.255.255.255/cisconet.cfg (Timed out)
 
         --- System Configuration Dialog ---
 
Would you like to enter the initial configuration dialog? [yes/no]: n
\end{verbatim}

Before we move on, let's talk about this output for a second. First,
notice that beginning with IOS 12.4, ISR routers will no longer take the
command \texttt{erase\ start}. The router has only one command after
\texttt{erase} that starts with \emph{s}, as shown here:

\begin{verbatim}
Router#erase s?
startup-config
\end{verbatim}

I know, you'd think that the IOS would continue to accept the command,
but nope---sorry! The second thing I want to point out is that the
output tells us the router is looking for a TFTP host to see if it can
download a configuration. When that fails, it goes straight into setup
mode. This gives you a great picture of the Cisco router default boot
sequence we talked about in Chapter 7, ``Managing a Cisco
Internetwork.''

Let's get back to configuring our router:

\begin{verbatim}
Press RETURN to get started!
Router#config t
Router(config)#hostname SF
SF(config)#enable secret GlobalNet
SF(config)#no ip domain-lookup
SF(config)#int s0/0/0
SF(config-if)#desc WAN Connection to Corp
SF(config-if)#ip address 172.16.10.2 255.255.255.252
SF(config-if)#no shut
SF(config-if)#clock rate 1000000
SF(config-if)#int f0/0
SF(config-if)#desc SF LAN
SF(config-if)#ip address 192.168.10.1 255.255.255.0
SF(config-if)#no shut
SF(config-if)#line con 0
SF(config-line)#password console
SF(config-line)#login
SF(config-line)#logging sync
SF(config-line)#exit
SF(config)#line vty 0 ?
  <1-1180>  Last Line number
  <cr>
SF(config)#line vty 0 1180
SF(config-line)#password telnet
SF(config-line)#login
SF(config-line)#banner motd #This is the SF Branch router#
SF(config)#exit
SF#copy run start
Destination filename [startup-config]?
Building configuration...
 [OK]
\end{verbatim}

Let's take a look at our configuration of the interfaces with the
following two commands:

\begin{verbatim}
SF#sh run | begin int
interface FastEthernet0/0
 description SF LAN
 ip address 192.168.10.1 255.255.255.0
 duplex auto
 speed auto
!
interface FastEthernet0/1
 no ip address
 shutdown
 duplex auto
 speed auto
!
interface Serial0/0/0
 description WAN Connection to Corp
 ip address 172.16.10.2 255.255.255.252
 clock rate 1000000
!
SF#sh ip int brief
Interface             IP-Address      OK? Method Status                Protocol
FastEthernet0/0       192.168.10.1    YES manual up                    up
FastEthernet0/1       unassigned      YES unset  administratively down down
Serial0/0/0           172.16.10.2     YES manual up                    up
Serial0/0/1           unassigned      YES unset  administratively down down
SF#
\end{verbatim}

Now that both ends of the serial link are configured, the link comes up.
Remember, the up/up status for the interfaces are Physical/Data Link
layer status indicators that don't reflect the layer 3 status! I ask
students in my classes, ``If the link shows up/up, can you ping the
directly connected network?'' And they say, ``Yes!'' The correct answer
is, ``I don't know,'' because we can't see the layer 3 status with this
command. We only see layers 1 and 2 and verify that the IP addresses
don't have a typo. This is really important to understand!

The \texttt{show\ ip\ route} command for the SF router reveals the
following:

\begin{verbatim}
SF#sh ip route
C    192.168.10.0/24 is directly connected, FastEthernet0/0
L    192.168.10.1/32 is directly connected, FastEthernet0/0
     172.16.0.0/30 is subnetted, 1 subnets
C       172.16.10.0 is directly connected, Serial0/0/0
L       172.16.10.2/32 is directly connected, Serial0/0/0
\end{verbatim}

Notice that router SF knows how to get to networks 172.16.10.0/30 and
192.168.10.0/24; we can now ping to the Corp router from SF:

\begin{verbatim}
SF#ping 172.16.10.1
 
Type escape sequence to abort.
Sending 5, 100-byte ICMP Echos to 172.16.10.1, timeout is 2 seconds:
!!!!!
Success rate is 100 percent (5/5), round-trip min/avg/max = 1/3/4 ms
\end{verbatim}

Now let's head back to the Corp router and check out the routing table:

\begin{verbatim}
Corp>sh ip route
     172.16.0.0/30 is subnetted, 1 subnets
C       172.16.10.0 is directly connected, Serial0/0
L       172.16.10.1/32 is directly connected, Serial0/0
     10.0.0.0/24 is subnetted, 1 subnets
C       10.10.10.0 is directly connected, FastEthernet0/0
L       10.10.10.1/32 is directly connected, FastEthernet0/0
\end{verbatim}

On the SF router's serial interface 0/0/0 is a DCE connection, which
means a clock rate needs to be set on the interface. Remember that you
don't need to use the \texttt{clock\ rate} command in production. While
true, it's still imperative that you know how/when you can use it and
that you understand it really well when studying for your CCNA exam!

\protect\hypertarget{c09.xhtmlux5cux23Page_379}{}{}We can see our
clocking with the \texttt{show\ controllers} command:

\begin{verbatim}
SF#sh controllers s0/0/0
Interface Serial0/0/0
Hardware is GT96K
DCE V.35, clock rate 1000000

Corp>sh controllers s0/0
Interface Serial0/0
Hardware is PowerQUICC MPC860
DTE V.35 TX and RX clocks detected.
\end{verbatim}

Since the SF router has a DCE cable connection, I needed to add clock
rate to this interface because DTE receives clock. Keep in mind that the
new ISR routers will autodetect this and set the clock rate to 2000000.
And you still need to make sure you're able to find an interface that is
DCE and set clocking to meet the objectives.

Since the serial links are showing up, we can now see both networks in
the Corp routing table. And once we configure LA, we'll see one more
network in the routing table of the Corp router. The Corp router can't
see the 192.168.10.0 network because we don't have any routing
configured yet---routers see only directly connected networks by
default.

\subsubsection[LA
Configuration]{\texorpdfstring{\protect\hypertarget{c09.xhtmlux5cux23c09-sec-8}{}{}LA
Configuration}{LA Configuration}}

To configure LA, we're going to do pretty much the same thing we did
with the other two routers. There are two interfaces to deal with,
Serial 0/0/1 and FastEthernet 0/0, and again, we'll be sure to add the
hostname, passwords, interface descriptions, and a banner to the router
configuration:

\begin{verbatim}
Router(config)#hostname LA
LA(config)#enable secret GlobalNet
LA(config)#no ip domain-lookup
LA(config)#int s0/0/1
LA(config-if)#ip address 172.16.10.6 255.255.255.252
LA(config-if)#no shut
LA(config-if)#clock rate 1000000
LA(config-if)#description WAN To Corporate
LA(config-if)#int f0/0
LA(config-if)#ip address 192.168.20.1 255.255.255.0
LA(config-if)#no shut
LA(config-if)#description LA LAN
LA(config-if)#line con 0
LA(config-line)#password console
LA(config-line)#login
LA(config-line)#logging sync
LA(config-line)#exit
LA(config)#line vty 0 ?
  <1-1180>  Last Line number
  <cr>
LA(config)#line vty 0 1180
LA(config-line)#password telnet
LA(config-line)#login
LA(config-line)#exit
LA(config)#banner motd #This is my LA Router#
LA(config)#exit
LA#copy run start
Destination filename [startup-config]?
Building configuration...
[OK]
\end{verbatim}

Nice---everything was pretty straightforward. The following output,
which I gained via the \texttt{show\ ip\ route} command, displays the
directly connected networks of 192.168.20.0 and 172.16.10.0:

\begin{verbatim}
LA#sh ip route
     172.16.0.0/30 is subnetted, 1 subnets
C       172.16.10.4 is directly connected, Serial0/0/1
L       172.16.10.6/32 is directly connected, Serial0/0/1
C    192.168.20.0/24 is directly connected, FastEthernet0/0
L    192.168.20.1/32 is directly connected, FastEthernet0/0
\end{verbatim}

So now that we've configured all three routers with IP addresses and
administrative functions, we can move on to deal with routing. But I
want to do one more thing on the SF and LA routers---since this is a
very small network, let's build a DHCP server on the Corp router for
each LAN.

\paragraph{Configuring DHCP on Our Corp Router}

While it's true that I could approach this task by going to each remote
router and creating a pool, why bother with all that when I can easily
create two pools on the Corp router and have the remote routers forward
requests to the Corp router? Of course, you remember how to do this from
Chapter 7!

Let's give it a shot:

\begin{verbatim}
Corp#config t
Corp(config)#ip dhcp excluded-address 192.168.10.1
Corp(config)#ip dhcp excluded-address 192.168.20.1
Corp(config)#ip dhcp pool SF_LAN
Corp(dhcp-config)#network 192.168.10.0 255.255.255.0
Corp(dhcp-config)#default-router 192.168.10.1
Corp(dhcp-config)#dns-server 4.4.4.4
Corp(dhcp-config)#exit
Corp(config)#ip dhcp pool LA_LAN
Corp(dhcp-config)#network 192.168.20.0 255.255.255.0
Corp(dhcp-config)#default-router 192.168.20.1
Corp(dhcp-config)#dns-server 4.4.4.4
Corp(dhcp-config)#exit
Corp(config)#exit
Corp#copy run start
Destination filename [startup-config]?
Building configuration...
\end{verbatim}

Creating DHCP pools on a router is actually a simple process, and you
would go about the configuration the same way on any router you wish to
add a DHCP pool to. To designate a router as a DHCP server, you just
create the pool name, add the network/subnet and the default gateway,
and then exclude any addresses that you don't want handed out. You
definitely want to make sure you've excluded the default gateway
address, and you'd usually add a DNS server as well. I always add any
exclusions first, and remember that you can conveniently exclude a range
of addresses on a single line. Soon, I'll demonstrate those verification
commands I promised I'd show you back in Chapter 7, but first, we need
to figure out why the Corp router still can't get to the remote networks
by default!

Now I'm pretty sure I configured DHCP correctly, but I just have this
nagging feeling I forgot something important. What could that be? Well,
the hosts are remote across a router, so what would I need to do that
would allow them to get an address from a DHCP server? If you concluded
that I've got to configure the SF and LA F0/0 interfaces to forward the
DHCP client requests to the server, you got it!

Here's how we'd go about doing that:

\begin{verbatim}
LA#config t
LA(config)#int f0/0
LA(config-if)#ip helper-address 172.16.10.5

SF#config t
SF(config)#int f0/0
SF(config-if)#ip helper-address 172.16.10.1
\end{verbatim}

I'm pretty sure I did this correctly, but we won't know until I have
some type of routing configured and working. So let's get to that next!

\subsection[Configuring IP Routing in Our
Network]{\texorpdfstring{\protect\hypertarget{c09.xhtmlux5cux23c09-sec-9}{}{}Configuring
IP Routing in Our Network}{Configuring IP Routing in Our Network}}

So is our network really good to go? After all, I've configured it with
IP addressing, administrative functions, and even clocking that will
automatically occur with the ISR routers. But how will our routers send
packets to remote networks when they get their destination
\protect\hypertarget{c09.xhtmlux5cux23Page_382}{}{}information by
looking into their tables that only include directions about directly
connected networks? And you know routers promptly discard packets they
receive with addresses for networks that aren't listed in their routing
table!

So we're not exactly ready to rock after all. But we will be soon
because there are several ways to configure the routing tables to
include all the networks in our little internetwork so that packets will
be properly forwarded. As usual, one size fits all rarely fits at all,
and what's best for one network isn't necessarily what's best for
another. That's why understanding the different types of routing will be
really helpful when choosing the best solution for your specific
environment and business requirements.

These are the three routing methods I'm going to cover with you:

\begin{enumerate}
\tightlist
\item
  Static routing
\item
  Default routing
\item
  Dynamic routing
\end{enumerate}

We're going to start with the first way and implement static routing on
our network, because if you can implement static routing \emph{and} make
it work, you've demonstrated that you definitely have a solid
understanding of the internetwork. So let's get started.

\subsubsection[Static
Routing]{\texorpdfstring{\protect\hypertarget{c09.xhtmlux5cux23c09-sec-10}{}{}Static
Routing}{Static Routing}}

Static routing is the process that ensues when you manually add routes
in each router's routing table. Predictably, there are pros and cons to
static routing, but that's true for all routing approaches.

Here are the pros:

\begin{enumerate}
\tightlist
\item
  There is no overhead on the router CPU, which means you could probably
  make do with a cheaper router than you would need for dynamic routing.
\item
  There is no bandwidth usage between routers, saving you money on WAN
  links as well as minimizing overhead on the router since you're not
  using a routing protocol.
\item
  It adds security because you, the administrator, can be very exclusive
  and choose to allow routing access to certain networks only.
\end{enumerate}

And here are the cons:

\begin{enumerate}
\tightlist
\item
  Whoever the administrator is must have a vault-tight knowledge of the
  internetwork and how each router is connected in order to configure
  routes correctly. If you don't have a good, accurate map of your
  internetwork, things will get very messy quickly!
\item
  If you add a network to the internetwork, you have to tediously add a
  route to it on all routers by hand, which only gets increasingly
  insane as the network grows.
\item
  Due to the last point, it's just not feasible to use it in most large
  networks because maintaining it would be a full-time job in itself.
\end{enumerate}

But that list of cons doesn't mean you get to skip learning all about it
mainly because of that first disadvantage I listed---the fact that you
must have such a solid understanding of a network to configure it
properly and that your administrative knowledge has to practically
\protect\hypertarget{c09.xhtmlux5cux23Page_383}{}{}verge on the
supernatural! So let's dive in and develop those skills. Starting at the
beginning, here's the command syntax you use to add a static route to a
routing table from global config:

\begin{verbatim}
ip route [destination_network] [mask] [next-hop_address or
  exitinterface] [administrative_distance] [permanent]
\end{verbatim}

This list describes each command in the string:

\texttt{ip\ route} The command used to create the static route.

\texttt{destination\_network} The network you're placing in the routing
table.

\texttt{mask} The subnet mask being used on the network.

\texttt{next-hop\_address} This is the IP address of the next-hop router
that will receive packets and forward them to the remote network, which
must signify a router interface that's on a directly connected network.
You must be able to successfully ping the router interface before you
can add the route. Important note to self is that if you type in the
wrong next-hop address or the interface to the correct router is down,
the static route will show up in the router's configuration but not in
the routing table.

\texttt{exitinterface} Used in place of the next-hop address if you
want, and shows up as a directly connected route.

\texttt{administrative\_distance} By default, static routes have an
administrative distance of 1 or 0 if you use an exit interface instead
of a next-hop address. You can change the default value by adding an
administrative weight at the end of the command. I'll talk a lot more
about this later in the chapter when we get to the section on dynamic
routing.

\texttt{permanent} If the interface is shut down or the router can't
communicate to the next-hop router, the route will automatically be
discarded from the routing table by default. Choosing the
\texttt{permanent} option keeps the entry in the routing table no matter
what happens.

Before I guide you through configuring static routes, let's take a look
at a sample static route to see what we can find out about it:

\begin{verbatim}
Router(config)#ip route 172.16.3.0 255.255.255.0 192.168.2.4
\end{verbatim}

\begin{enumerate}
\tightlist
\item
  The \texttt{ip\ route} command tells us simply that it's a static
  route.
\item
  172.16.3.0 is the remote network we want to send packets to.
\item
  255.255.255.0 is the mask of the remote network.
\item
  192.168.2.4 is the next hop, or router, that packets will be sent to.
\end{enumerate}

But what if the static route looked like this instead?

\begin{verbatim}
Router(config)#ip route 172.16.3.0 255.255.255.0 192.168.2.4 150
\end{verbatim}

That 150 at the end changes the default administrative distance (AD) of
1 to 150. As I said, I'll talk much more about AD when we get into
dynamic routing, but for now, just remember that the AD is the
trustworthiness of a route, where 0 is best and 255 is worst.

\protect\hypertarget{c09.xhtmlux5cux23Page_384}{}{}One more example,
then we'll start configuring:

\begin{verbatim}
Router(config)#ip route 172.16.3.0 255.255.255.0 s0/0/0
\end{verbatim}

Instead of using a next-hop address, we can use an exit interface that
will make the route show up as a directly connected network.
Functionally, the next hop and exit interface work exactly the same.

To help you understand how static routes work, I'll demonstrate the
configuration on the internetwork shown previously in
\protect\hyperlink{c09.xhtmlux5cux23figure9-8}{Figure 9.8}. Here it is
again in \protect\hyperlink{c09.xhtmlux5cux23figure9-9}{Figure 9.9} to
save you the trouble of having to go back and forth to view the same
figure.

\begin{figure}
\centering
\includegraphics{images/c09f009.jpg}
\caption{{\protect\hyperlink{c09.xhtmlux5cux23figureanchor9-9}{\textbf{FIGURE
9.9}} Our internetwork}}
\end{figure}

\paragraph{Corp}

Each routing table automatically includes directly connected networks.
To be able to route to all indirectly connected networks within the
internetwork, the routing table must include information that describes
where these other networks are located and how to get to them.

The Corp router is connected to three networks. For the Corp router to
be able to route to all networks, the following networks have to be
configured into its routing table:

\begin{enumerate}
\tightlist
\item
  192.168.10.0
\item
  192.168.20.0
\end{enumerate}

The following router output shows the static routes on the Corp router
and the routing table after the configuration. For the Corp router to
find the remote networks, I had to place an entry into the routing table
describing the remote network, the remote mask, and where to send the
packets. I am going to add a 150 at the end of each line to raise the
administrative distance. You'll see why soon when we get to dynamic
routing. Many times this is also referred to as a floating static route
because the static route has a higher
\protect\hypertarget{c09.xhtmlux5cux23Page_385}{}{}administrative
distance than any routing protocol and will only be used if the routes
found with the routing protocols go down. Here's the output:

\begin{verbatim}
Corp#config t
Corp(config)#ip route 192.168.10.0 255.255.255.0 172.16.10.2 150
Corp(config)#ip route 192.168.20.0 255.255.255.0 s0/1 150
Corp(config)#do show run | begin ip route
ip route 192.168.10.0 255.255.255.0 172.16.10.2 150
ip route 192.168.20.0 255.255.255.0 Serial0/1 150
\end{verbatim}

I needed to use different paths for networks 192.168.10.0 and
192.168.20.0, so I used a next-hop address for the SF router and an exit
interface for the LA router. After the router has been configured, you
can just type \texttt{show\ ip\ route} to see the static routes:

\begin{verbatim}
Corp(config)#do show ip route
S    192.168.10.0/24 [150/0] via 172.16.10.2
     172.16.0.0/30 is subnetted, 2 subnets
C       172.16.10.4 is directly connected, Serial0/1
L       172.16.10.5/32 is directly connected, Serial0/1
C       172.16.10.0 is directly connected, Serial0/0
L       172.16.10.1/32 is directly connected, Serial0/0
S    192.168.20.0/24 is directly connected, Serial0/1
     10.0.0.0/24 is subnetted, 1 subnets
C       10.10.10.0 is directly connected, FastEthernet0/0
L       10.10.10.1/32 is directly connected, FastEthernet0/0
\end{verbatim}

The Corp router is configured to route and know all routes to all
networks. But can you see a difference in the routing table for the
routes to SF and LA? That's right! The next-hop configuration showed up
as via, and the route configured with an exit interface configuration
shows up as static but also as directly connected! This demonstrates how
they are functionally the same but will display differently in the
routing table.

Understand that if the routes don't appear in the routing table, it's
because the router can't communicate with the next-hop address you've
configured. But you can still use the \texttt{permanent} parameter to
keep the route in the routing table even if the next-hop device can't be
contacted.

The \texttt{S} in the first routing table entry means that the route is
a static entry. The \texttt{{[}150/0{]}} stands for the administrative
distance and metric to the remote network, respectively.

Okay---we're good. The Corp router now has all the information it needs
to communicate with the other remote networks. Still, keep in mind that
if the SF and LA routers aren't configured with all the same
information, the packets will be discarded. We can fix this by
configuring static routes.

\begin{center}\rule{0.5\linewidth}{0.5pt}\end{center}

\includegraphics{images/note.png} Don't stress about the 150 at the end
of the static route configuration at all, because I promise to get to it
really soon in \emph{this} chapter, not a later one! You really don't
need to worry about it at this point.

\begin{center}\rule{0.5\linewidth}{0.5pt}\end{center}

\paragraph[SF]{\texorpdfstring{\protect\hypertarget{c09.xhtmlux5cux23Page_386}{}{}SF}{SF}}

The SF router is directly connected to networks 172.16.10.0/30 and
192.168.10.0/24, which means I've got to configure the following static
routes on the SF router:

\begin{enumerate}
\tightlist
\item
  10.10.10.0/24
\item
  192.168.20.0/24
\item
  172.16.10.4/30
\end{enumerate}

The configuration for the SF router is revealed in the following output.
Remember that we'll never create a static route to any network we're
directly connected to as well as the fact that we must use the next hop
of 172.16.10.1 since that's our only router connection. Let's check out
the commands:

\begin{verbatim}
SF(config)#ip route 10.10.10.0 255.255.255.0 172.16.10.1 150
SF(config)#ip route 172.16.10.4 255.255.255.252 172.16.10.1 150
SF(config)#ip route 192.168.20.0 255.255.255.0 172.16.10.1 150
SF(config)#do show run | begin ip route
ip route 10.10.10.0 255.255.255.0 172.16.10.1 150
ip route 172.16.10.4 255.255.255.252 172.16.10.1 150
ip route 192.168.20.0 255.255.255.0 172.16.10.1 150
\end{verbatim}

By looking at the routing table, you can see that the SF router now
understands how to find each network:

\begin{verbatim}
SF(config)#do show ip route
C    192.168.10.0/24 is directly connected, FastEthernet0/0
L    192.168.10.1/32 is directly connected, FastEthernet0/0
     172.16.0.0/30 is subnetted, 3 subnets
S       172.16.10.4 [150/0] via 172.16.10.1
C       172.16.10.0 is directly connected, Serial0/0/0
L       172.16.10.2/32 is directly connected, Serial0/0
S    192.168.20.0/24 [150/0] via 172.16.10.1
     10.0.0.0/24 is subnetted, 1 subnets
S       10.10.10.0 [150/0] via 172.16.10.1
\end{verbatim}

And we now can rest assured that the SF router has a complete routing
table as well. As soon as the LA router has all the networks in its
routing table, SF will be able to communicate with all remote networks!

\paragraph{LA}

The LA router is directly connected to 192.168.20.0/24 and
172.16.10.4/30, so these are the routes that must be added:

\begin{enumerate}
\tightlist
\item
  10.10.10.0/24
\item
  172.16.10.0/30
\item
  192.168.10.0/24
\end{enumerate}

\protect\hypertarget{c09.xhtmlux5cux23Page_387}{}{}And here's the LA
router's configuration:

\begin{verbatim}
LA#config t
LA(config)#ip route 10.10.10.0 255.255.255.0 172.16.10.5 150
LA(config)#ip route 172.16.10.0 255.255.255.252 172.16.10.5 150
LA(config)#ip route 192.168.10.0 255.255.255.0 172.16.10.5 150
LA(config)#do show run | begin ip route
ip route 10.10.10.0 255.255.255.0 172.16.10.5 150
ip route 172.16.10.0 255.255.255.252 172.16.10.5 150
ip route 192.168.10.0 255.255.255.0 172.16.10.5 150
\end{verbatim}

This output displays the routing table on the LA router:

\begin{verbatim}
LA(config)#do sho ip route
S    192.168.10.0/24 [150/0] via 172.16.10.5
     172.16.0.0/30 is subnetted, 3 subnets
C       172.16.10.4 is directly connected, Serial0/0/1
L       172.16.10.6/32 is directly connected, Serial0/0/1
S       172.16.10.0 [150/0] via 172.16.10.5
C    192.168.20.0/24 is directly connected, FastEthernet0/0
L    192.168.20.1/32 is directly connected, FastEthernet0/0
     10.0.0.0/24 is subnetted, 1 subnets
S       10.10.10.0 [150/0] via 172.16.10.5
\end{verbatim}

LA now shows all five networks in the internetwork, so it too can now
communicate with all routers and networks. But before we test our little
network, as well as our DHCP server, let's cover one more topic.

\subsubsection[Default
Routing]{\texorpdfstring{\protect\hypertarget{c09.xhtmlux5cux23c09-sec-11}{}{}Default
Routing}{Default Routing}}

The SF and LA routers that I've connected to the Corp router are
considered stub routers. A \emph{stub} indicates that the networks in
this design have only one way out to reach all other networks, which
means that instead of creating multiple static routes, we can just use a
single default route. This default route is used by IP to forward any
packet with a destination not found in the routing table, which is why
it is also called a gateway of last resort. Here's the configuration I
could have done on the LA router instead of typing in the static routes
due to its stub status:

\begin{verbatim}
LA#config t
LA(config)#no ip route 10.10.10.0 255.255.255.0 172.16.10.5 150
LA(config)#no ip route 172.16.10.0 255.255.255.252 172.16.10.5 150
LA(config)#no ip route 192.168.10.0 255.255.255.0 172.16.10.5 150
LA(config)#ip route 0.0.0.0 0.0.0.0 172.16.10.5
LA(config)#do sho ip route
[output cut]
Gateway of last resort is 172.16.10.5 to network 0.0.0.0
172.16.0.0/30 is subnetted, 1 subnets
C       172.16.10.4 is directly connected, Serial0/0/1
L       172.16.10.6/32 is directly connected, Serial0/0/1
C    192.168.20.0/24 is directly connected, FastEthernet0/0
L    192.168.20.0/32 is directly connected, FastEthernet0/0
S*   0.0.0.0/0 [1/0] via 172.16.10.5
\end{verbatim}

Okay---I've removed all the initial static routes I had configured, and
adding a default route is a lot easier than typing a bunch of static
routes! Can you see the default route listed last in the routing table?
The \texttt{S*} shows that as a candidate for the default route. And I
really want you to notice that the gateway of last resort is now set
too. Everything the router receives with a destination not found in the
routing table will be forwarded to 172.16.10.5. You need to be careful
where you place default routes because you can easily create a network
loop!

So we're there---we've configured all our routing tables! All the
routers have the correct routing table, so all routers and hosts should
be able to communicate without a hitch---for now. But if you add even
one more network or another router to the internetwork, you'll have to
update each and every router's routing tables by hand---ugh! Not really
a problem at all if you've got a small network like we do, but it would
be a time-consuming monster if you're dealing with a large internetwork!

\paragraph{Verifying Your Configuration}

But we're not done yet---once all the routers' routing tables are
configured, they must be verified. The best way to do this, besides
using the \texttt{show\ ip\ route} command, is via Ping. I'll start by
pinging from the Corp router to the SF router.

Here's the output I got:

\begin{verbatim}
Corp#ping 192.168.10.1
Type escape sequence to abort.
Sending 5, 100-byte ICMP Echos to 192.168.10.1, timeout is 2 seconds:
!!!!!
Success rate is 100 percent (5/5), round-trip min/avg/max = 4/4/4 ms
Corp#
\end{verbatim}

Here you can see that I pinged from the Corp router to the remote
interface of the SF router. Now let's ping the remote network on the LA
router, and after that, we'll test our DHCP server and see if that is
working too!

\begin{verbatim}
Corp#ping 192.168.20.1
Type escape sequence to abort.
Sending 5, 100-byte ICMP Echos to 192.168.20.1, timeout is 2 seconds:
!!!!!
Success rate is 100 percent (5/5), round-trip min/avg/max = 1/2/4 ms
Corp#
\end{verbatim}

\protect\hypertarget{c09.xhtmlux5cux23Page_389}{}{}And why not test my
configuration of the DHCP server on the Corp router while we're at it?
I'm going to go to each host on the SF and LA routers and make them DHCP
clients. By the way, I'm using an old router to represent ``hosts,''
which just happens to work great for studying purposes. Here's how I did
that:

\begin{verbatim}
SF_PC(config)#int e0
SF_PC(config-if)#ip address dhcp
SF_PC(config-if)#no shut
Interface Ethernet0 assigned DHCP address 192.168.10.8, mask 255.255.255.0
LA_PC(config)#int e0
LA_PC(config-if)#ip addr dhcp
LA_PC(config-if)#no shut
Interface Ethernet0 assigned DHCP address 192.168.20.4, mask 255.255.255.0
\end{verbatim}

Nice! Don't you love it when things just work the first time? Sadly,
this just isn't exactly a realistic expectation in the networking world,
so we must be able to troubleshoot and verify our networks. Let's verify
our DHCP server with a few of the commands you learned back in Chapter
7:

\begin{verbatim}
Corp#sh ip dhcp binding
Bindings from all pools not associated with VRF:
IP address          Client-ID/              Lease expiration        Type
                    Hardware address/
                    User name
192.168.10.8        0063.6973.636f.2d30.    Sept 16 2013 10:34 AM    Automatic
                    3035.302e.3062.6330.
                    2e30.3063.632d.4574.
                    30
192.168.20.4        0063.6973.636f.2d30.    Sept 16 2013 10:46 AM    Automatic
                    3030.322e.3137.3632.
                    2e64.3032.372d.4574.
                    30
\end{verbatim}

We can see from earlier that our little DHCP server is working! Let's
try another couple of commands:

\begin{verbatim}
Corp#sh ip dhcp pool SF_LAN
Pool SF_LAN :
 Utilization mark (high/low)    : 100 / 0
 Subnet size (first/next)       : 0 / 0
 Total addresses                : 254
 Leased addresses               : 3
 Pending event                  : none
 1 subnet is currently in the pool :
 Current index        IP address range                    Leased addresses
 192.168.10.9         192.168.10.1     - 192.168.10.254    3
 
Corp#sh ip dhcp conflict
IP address        Detection method   Detection time          VRF
\end{verbatim}

The last command would tell us if we had two hosts with the same IP
address, so it's good news because there are no conflicts reported! Two
detection methods are used to confirm this:

\begin{enumerate}
\tightlist
\item
  A ping from the DHCP server to make sure no other host responds before
  handing out an address
\item
  A gratuitous ARP from a host that receives a DHCP address from the
  server
\end{enumerate}

The DHCP client will send an ARP request with its new IP address looking
to see if anyone responds, and if so, it will report the conflict to the
server.

Okay, since we can communicate from end to end and to each host without
a problem while receiving DHCP addresses from our server, I'd say our
static and default route configurations have been a success---cheers!

\subsection[Dynamic
Routing]{\texorpdfstring{\protect\hypertarget{c09.xhtmlux5cux23c09-sec-12}{}{}Dynamic
Routing}{Dynamic Routing}}

Dynamic routing is when protocols are used to find networks and update
routing tables on routers. This is whole lot easier than using static or
default routing, but it will cost you in terms of router CPU processing
and bandwidth on network links. A routing protocol defines the set of
rules used by a router when it communicates routing information between
neighboring routers.

The routing protocol I'm going to talk about in this chapter is Routing
Information Protocol (RIP) versions 1 and 2.

Two types of routing protocols are used in internetworks: \emph{interior
gateway protocols (IGPs)} and \emph{exterior gateway protocols (EGPs)}.
IGPs are used to exchange routing information with routers in the same
\emph{autonomous system (AS)}. An AS is either a single network or a
collection of networks under a common administrative domain, which
basically means that all routers sharing the same routing-table
information are in the same AS. EGPs are used to communicate between
ASs. An example of an EGP is Border Gateway Protocol (BGP), which we're
not going to bother with because it's beyond the scope of this book.

Since routing protocols are so essential to dynamic routing, I'm going
to give you the basic information you need to know about them next.
Later on in this chapter, we'll focus on configuration.

\subsubsection[Routing Protocol
Basics]{\texorpdfstring{\protect\hypertarget{c09.xhtmlux5cux23c09-sec-13}{}{}Routing
Protocol Basics}{Routing Protocol Basics}}

There are some important things you should know about routing protocols
before we get deeper into RIP routing. Being familiar with
administrative distances and the three different kinds of routing
protocols, for example. Let's take a look.

\paragraph[Administrative
Distances]{\texorpdfstring{\protect\hypertarget{c09.xhtmlux5cux23Page_391}{}{}Administrative
Distances}{Administrative Distances}}

The \emph{administrative distance (AD)} is used to rate the
trustworthiness of routing information received on a router from a
neighbor router. An administrative distance is an integer from 0 to 255,
where 0 is the most trusted and 255 means no traffic will be passed via
this route.

If a router receives two updates listing the same remote network, the
first thing the router checks is the AD. If one of the advertised routes
has a lower AD than the other, then the route with the lowest AD will be
chosen and placed in the routing table.

If both advertised routes to the same network have the same AD, then
routing protocol metrics like \emph{hop count} and/or the bandwidth of
the lines will be used to find the best path to the remote network. The
advertised route with the lowest metric will be placed in the routing
table, but if both advertised routes have the same AD as well as the
same metrics, then the routing protocol will load-balance to the remote
network, meaning the protocol will send data down each link.

\protect\hyperlink{c09.xhtmlux5cux23table9-1}{Table 9.1} shows the
default administrative distances that a Cisco router uses to decide
which route to take to a remote network.

{\protect\hyperlink{c09.xhtmlux5cux23tableanchor9-1}{\textbf{TABLE 9.1}}
Default administrative distances}

\begin{longtable}[]{@{}ll@{}}
\toprule
Route Source & Default AD\tabularnewline
\midrule
\endhead
Connected interface & 0\tabularnewline
Static route & 1\tabularnewline
External BGP & 20\tabularnewline
EIGRP & 90\tabularnewline
OSPF & 110\tabularnewline
RIP & 120\tabularnewline
External EIGRP & 170\tabularnewline
Internal BGP & 200\tabularnewline
Unknown & 255 (This route will never be used.)\tabularnewline
\bottomrule
\end{longtable}

If a network is directly connected, the router will always use the
interface connected to the network. If you configure a static route, the
router will then believe that route over any other ones it learns about.
You can change the administrative distance of static routes, but by
default, they have an AD of 1. In our previous static route
configuration, the AD of each route is set at 150. This AD allows us to
configure routing protocols without having to remove the static routes
because it's nice to have them there for backup in case the routing
protocol experiences some kind of failure.

\protect\hypertarget{c09.xhtmlux5cux23Page_392}{}{}If you have a static
route, an RIP-advertised route, and an EIGRP-advertised route listing
the same network, which route will the router go with? That's right---by
default, the router will always use the static route unless you change
its AD---which we did!

\paragraph{Routing Protocols}

There are three classes of routing protocols:

\textbf{Distance vector} The distance-vector protocols in use today find
the best path to a remote network by judging distance. In RIP routing,
each instance where a packet goes through a router is called a hop, and
the route with the least number of hops to the network will be chosen as
the best one. The vector indicates the direction to the remote network.
RIP is a distance-vector routing protocol and periodically sends out the
entire routing table to directly connected neighbors.

\textbf{Link state} In link-state protocols, also called
shortest-path-first (SPF) protocols, the routers each create three
separate tables. One of these tables keeps track of directly attached
neighbors, one determines the topology of the entire internetwork, and
one is used as the routing table. Link-state routers know more about the
internetwork than any distance-vector routing protocol ever could. OSPF
is an IP routing protocol that's completely link-state. Link-state
routing tables are not exchanged periodically. Instead, triggered
updates containing only specific link-state information are sent.
Periodic keepalives that are small and efficient, in the form of hello
messages, are exchanged between directly connected neighbors to
establish and maintain neighbor relationships.

\textbf{Advanced distance vector} Advanced distance-vector protocols use
aspects of both distance-vector and link-state protocols, and EIGRP is a
great example. EIGRP may act like a link-state routing protocol because
it uses a Hello protocol to discover neighbors and form neighbor
relationships and because only partial updates are sent when a change
occurs. However, EIGRP is still based on the key distance-vector routing
protocol principle that information about the rest of the network is
learned from directly connected neighbors.

There's no set of rules to follow that dictate exactly how to broadly
configure routing protocols for every situation. It's a task that really
must be undertaken on a case-by-case basis, with an eye on specific
requirements of each one. If you understand how the different routing
protocols work, you can make good, solid decisions that will solidly
meet the individual needs of any business!

\subsection[Routing Information Protocol
(RIP)]{\texorpdfstring{\protect\hypertarget{c09.xhtmlux5cux23c09-sec-14}{}{}Routing
Information Protocol (RIP)}{Routing Information Protocol (RIP)}}

Routing Information Protocol (RIP) is a true distance-vector routing
protocol. RIP sends the complete routing table out of all active
interfaces every 30 seconds. It relies on hop count to determine the
best way to a remote network, but it has a maximum allowable hop count
of 15 by default, so a destination of 16 would be considered
unreachable. RIP works okay in very small networks, but it's super
inefficient on large networks with slow WAN
\protect\hypertarget{c09.xhtmlux5cux23Page_393}{}{}links or on networks
with a large number of routers installed and completely useless on
networks that have links with variable bandwidths!

RIP version 1 uses only \emph{classful routing}, which means that all
devices in the network must use the same subnet mask. This is because
RIP version 1 doesn't send updates with subnet mask information in tow.
RIP version 2 provides something called \emph{prefix routing} and does
send subnet mask information with its route updates. This is called
\emph{classless routing}.

So, with that let's configure our current network with RIPv2, before we
move onto the next chapter.

\subsubsection[Configuring RIP
Routing]{\texorpdfstring{\protect\hypertarget{c09.xhtmlux5cux23c09-sec-15}{}{}Configuring
RIP Routing}{Configuring RIP Routing}}

To configure RIP routing, just turn on the protocol with the
\texttt{router\ rip} command and tell the RIP routing protocol the
networks to advertise. Remember that with static routing, we always
configured remote networks and never typed a route to our directly
connected networks? Well, dynamic routing is carried out the complete
opposite way. You would never type a \emph{remote} network under your
routing protocol---only enter your directly connected networks! Let's
configure our three-router internetwork, revisited in
\protect\hyperlink{c09.xhtmlux5cux23figure9-9}{Figure 9.9}, with RIP
routing.

\paragraph{Corp}

RIP has an administrative distance of 120. Static routes have an
administrative distance of 1 by default, and since we currently have
static routes configured, the routing tables won't be populated with RIP
information by default. We're still good though because I added the 150
to the end of each static route!

You can add the RIP routing protocol by using the \texttt{router\ rip}
command and the \texttt{net}\texttt{work} command. The \texttt{network}
command tells the routing protocol which classful network to advertise.
By doing this, you're activating the RIP routing process on the
interfaces whose addressing falls within the specified classful networks
configured with the \texttt{network} command under the RIP routing
process.

Look at the Corp router configuration to see how easy this is. Oh
wait---first, I want to verify my directly connected networks so I know
what to configure RIP with:

\begin{verbatim}
Corp#sh ip int brief
Interface        IP-Address      OK? Method Status                Protocol
FastEthernet0/0  10.10.10.1      YES manual up                    up
Serial0/0        172.16.10.1     YES manual up                    up
FastEthernet0/1  unassigned      YES unset  administratively down down
Serial0/1        172.16.10.5     YES manual up                    up
Corp#config t
Corp(config)#router rip
Corp(config-router)#network 10.0.0.0
Corp(config-router)#network 172.16.0.0
Corp(config-router)#version 2
Corp(config-router)#no auto-summary
\end{verbatim}

\protect\hypertarget{c09.xhtmlux5cux23Page_394}{}{}That's it---really!
Typically just two or three commands and you're done, which sure makes
your job a lot easier than dealing with static routes, doesn't it? Be
sure to keep in mind the extra router CPU process and bandwidth that
you're consuming.

Anyway, so what exactly did I do here? I enabled the RIP routing
protocol, added my directly connected networks, made sure I was only
running RIPv2, which is a classless routing protocol, and then I
disabled auto-summary. We typically don't want our routing protocols
summarizing for us because it's better to do that manually and both RIP
and EIGRP (before 15.x code) auto-summarize by default. So a general
rule of thumb is to disable auto-summary, which allows them to advertise
subnets.

Notice I didn't type in subnets, only the classful network address,
which is betrayed by the fact that all subnet bits and host bits are
off! That's because with dynamic routing, it's not my job and it's up to
the routing protocol to find the subnets and populate the routing
tables. And since we have no router buddies running RIP, we won't see
any RIP routes in the routing table yet.

\begin{center}\rule{0.5\linewidth}{0.5pt}\end{center}

\includegraphics{images/note.png} Remember that RIP uses the classful
address when configuring the network address. To clarify this, refer to
the example in our network with an address of 172.16.0.0/24 using
subnets 172.16.10.0 and 172.16.20.0. You would only type in the classful
network address of 172.16.0.0 and let RIP find the subnets and place
them in the routing table. This doesn't mean you are running a classful
routing protocol; this is just the way that both RIP and EIGRP are
configured.

\begin{center}\rule{0.5\linewidth}{0.5pt}\end{center}

\paragraph{SF}

Let's configure our SF router now, which is connected to two networks.
We need to configure both directly connected classful networks, not
subnets:

\begin{verbatim}
SF#sh ip int brief
Interface         IP-Address      OK? Method Status             Protocol
FastEthernet0/0   192.168.10.1    YES manual up                    up
FastEthernet0/1   unassigned      YES unset  administratively down down
Serial0/0/0       172.16.10.2     YES manual up                    up
Serial0/0/1       unassigned      YES unset  administratively down down
SF#config
SF(config)#router rip
SF(config-router)#network 192.168.10.0
SF(config-router)#network 172.16.0.0
SF(config-router)#version 2
SF(config-router)#no auto-summary
SF(config-router)#do show ip route
C    192.168.10.0/24 is directly connected, FastEthernet0/0
L    192.168.10.1/32 is directly connected, FastEthernet0/0
     172.16.0.0/30 is subnetted, 3 subnets
R       172.16.10.4 [120/1] via 172.16.10.1, 00:00:08, Serial0/0/0
C       172.16.10.0 is directly connected, Serial0/0/0
L       172.16.10.2/32 is directly connected, Serial0/0
S    192.168.20.0/24 [150/0] via 172.16.10.1
     10.0.0.0/24 is subnetted, 1 subnets
R       10.10.10.0 [120/1] via 172.16.10.1, 00:00:08, Serial0/0/0
\end{verbatim}

That was pretty straightforward. Let's talk about this routing table.
Since we have one RIP buddy out there with whom we are exchanging
routing tables, we can see the RIP networks coming from the Corp router.
All the other routes still show up as static and local. RIP also found
both connections through the Corp router to networks 10.10.10.0 and
172.16.10.4. But we're not done yet!

\paragraph{LA}

Let's configure our LA router with RIP, only I'm going to remove the
default route first, even though I don't have to. You'll see why soon:

\begin{verbatim}
LA#config t
LA(config)#no ip route 0.0.0.0 0.0.0.0
LA(config)#router rip
LA(config-router)#network 192.168.20.0
LA(config-router)#network 172.16.0.0
LA(config-router)#no auto
LA(config-router)#vers 2
LA(config-router)#do show ip route
R    192.168.10.0/24 [120/2] via 172.16.10.5, 00:00:10, Serial0/0/1
     172.16.0.0/30 is subnetted, 3 subnets
C       172.16.10.4 is directly connected, Serial0/0/1
L       172.16.10.6/32 is directly connected, Serial0/0/1
R       172.16.10.0 [120/1] via 172.16.10.5, 00:00:10, Serial0/0/1
C    192.168.20.0/24 is directly connected, FastEthernet0/0
L    192.168.20.1/32 is directly connected, FastEthernet0/0
     10.0.0.0/24 is subnetted, 1 subnets
R       10.10.10.0 [120/1] via 172.16.10.5, 00:00:10, Serial0/0/1
\end{verbatim}

The routing table is sprouting new \texttt{R}'s as we add RIP buddies!
We can still see that all routes are in the routing table.

This output shows us basically the same routing table and the same
entries that it had when we were using static routes---except for those
\texttt{R}'s. An \texttt{R} indicates that the networks were added
dynamically using the RIP routing protocol. The \texttt{{[}120/1{]}} is
the administrative distance of the route (120) along with the metric,
which for RIP is the number of hops to that remote network (1). From the
Corp router, all networks are one hop away.

\protect\hypertarget{c09.xhtmlux5cux23Page_396}{}{}So, while yes, it's
true that RIP has worked in our little internetwork, it's just not a
great solution for most enterprises. Its maximum hop count of only 15 is
a highly limiting factor. And it performs full routing-table updates
every 30 seconds, which would bring a larger internetwork to a painful
crawl in no time!

There's still one more thing I want to show you about RIP routing tables
and the parameters used to advertise remote networks. Using a different
router on a different network as an example for a second, look into the
following output. Can you spot where the following routing table shows
\texttt{{[}120/15{]}} in the 10.1.3.0 network metric? This means that
the administrative distance is 120, the default for RIP, but the hop
count is 15. Remember that each time a router sends out an update to a
neighbor router, the hop count goes up by one incrementally for each
route! Here's that output now:

\begin{verbatim}
Router#sh ip route
     10.0.0.0/24 is subnetted, 12 subnets
C       10.1.11.0 is directly connected, FastEthernet0/1
L       10.1.11.1/32 is directly connected, FastEthernet0/1
C       10.1.10.0 is directly connected, FastEthernet0/0
L       10.1.10.1/32 is directly connected, FastEthernet/0/0
R       10.1.9.0 [120/2] via 10.1.5.1, 00:00:15, Serial0/0/1
R       10.1.8.0 [120/2] via 10.1.5.1, 00:00:15, Serial0/0/1
R       10.1.12.0 [120/1] via 10.1.11.2, 00:00:00, FastEthernet0/1
R      10.1.3.0 [120/15] via 10.1.5.1, 00:00:15, Serial0/0/1
R       10.1.2.0 [120/1] via 10.1.5.1, 00:00:15, Serial0/0/1
R       10.1.1.0 [120/1] via 10.1.5.1, 00:00:15, Serial0/0/1
R       10.1.7.0 [120/2] via 10.1.5.1, 00:00:15, Serial0/0/1
R       10.1.6.0 [120/2] via 10.1.5.1, 00:00:15, Serial0/0/1
C       10.1.5.0 is directly connected, Serial0/0/1
L       10.1.5.1/32 is directly connected, Serial0/0/1
R       10.1.4.0 [120/1] via 10.1.5.1, 00:00:15, Serial0/0/1
\end{verbatim}

So this \texttt{{[}120/15{]}} is really bad. We're basically doomed
because the next router that receives the table from this router will
just discard the route to network 10.1.3.0 since the hop count would
rise to 16, which is invalid!

\begin{center}\rule{0.5\linewidth}{0.5pt}\end{center}

\includegraphics{images/note.png} If a router receives a routing update
that contains a higher-cost path to a network that's already in its
routing table, the update will be ignored.

\begin{center}\rule{0.5\linewidth}{0.5pt}\end{center}

\subsubsection[Holding Down RIP
Propagations]{\texorpdfstring{\protect\hypertarget{c09.xhtmlux5cux23c09-sec-16}{}{}Holding
Down RIP Propagations}{Holding Down RIP Propagations}}

You probably don't want your RIP network advertised everywhere on your
LAN and WAN. There's enough stress in networking already and not a whole
lot to be gained by advertising your RIP network to the Internet!

\protect\hypertarget{c09.xhtmlux5cux23Page_397}{}{}There are a few
different ways to stop unwanted RIP updates from propagating across your
LANs and WANs, and the easiest one is through the
\texttt{passive-interface} command. This command prevents RIP update
broadcasts from being sent out of a specified interface but still allows
that same interface to receive RIP updates.

Here's an example of how to configure a \texttt{passive-interface} on
the Corp router's Fa0/1 interface, which we will pretend is connected to
a LAN that we don't want RIP on (and the interface isn't shown in the
figure):

\begin{verbatim}
Corp#config t
Corp(config)#router rip
Corp(config-router)#passive-interface FastEthernet 0/1
\end{verbatim}

This command will stop RIP updates from being propagated out of
FastEthernet interface 0/1, but it can still receive RIP updates.

\begin{center}\rule{0.5\linewidth}{0.5pt}\end{center}

\includegraphics{images/globe1.png}\\
\textbf{Should We Really Use RIP in an Internetwork?}

You have been hired as a consultant to install a couple of Cisco routers
into a growing network. They have a couple of old Unix routers that they
want to keep in the network. These routers do not support any routing
protocol except RIP. I guess this means you just have to run RIP on the
entire network. If you were balding before, your head now shines like
chrome.

No need for hairs abandoning ship though---you can run RIP on a router
connecting that old network, but you certainly don't need to run RIP
throughout the whole internetwork!

You can do what is called \emph{redistribution}, which is basically
translating from one type of routing protocol to another. This means
that you can support those old routers using RIP but use something much
better like Enhanced IGRP on the rest of your network.

This will prevent RIP routes from being sent all over the internetwork
gobbling up all that precious bandwidth!

\begin{center}\rule{0.5\linewidth}{0.5pt}\end{center}

\paragraph{Advertising a Default Route Using RIP}

Now I'm going to guide you through how to advertise a way out of your
autonomous system to other routers, and you'll see this is completed the
same way with OSPF. Imagine that our Corp router's Fa0/0 interface is
connected to some type of Metro-Ethernet as a connection to the
Internet. This is a pretty common configuration today that uses a LAN
interface to connect to the ISP instead of a serial interface.

If we do add an Internet connection to Corp, all routers in our AS (SF
and LA) must know where to send packets destined for networks on the
Internet or they'll just drop the
\protect\hypertarget{c09.xhtmlux5cux23Page_398}{}{}packets when they get
a remote request. One solution to this little hitch would be to place a
default route on every router and funnel the information to Corp, which
in turn would have a default route to the ISP. Most people do this type
of configuration in small- to medium-size networks because it actually
works pretty well!

But since I'm running RIPv2 on all routers, I'll just add a default
route on the Corp router to our ISP, as I would normally. I'll then add
another command to advertise my network to the other routers in the AS
as the default route to show them where to send packets destined for the
Internet.

Here's my new Corp configuration:

\begin{verbatim}
Corp(config)#ip route 0.0.0.0 0.0.0.0 fa0/0
Corp(config)#router rip
Corp(config-router)#default-information originate
\end{verbatim}

Now, let's take a look at the last entry found in the Corp routing
table:

\begin{verbatim}
S*   0.0.0.0/0 is directly connected, FastEthernet0/0
\end{verbatim}

Let's see if the LA router can see this same entry:

\begin{verbatim}
LA#sh ip route
Gateway of last resort is 172.16.10.5 to network 0.0.0.0

R    192.168.10.0/24 [120/2] via 172.16.10.5, 00:00:04, Serial0/0/1
     172.16.0.0/30 is subnetted, 2 subnets
C       172.16.10.4 is directly connected, Serial0/0/1
L       172.16.10.5/32 is directly connected, Serial0/0/1
R       172.16.10.0 [120/1] via 172.16.10.5, 00:00:04, Serial0/0/1
C    192.168.20.0/24 is directly connected, FastEthernet0/0
L    192.168.20.1/32 is directly connected, FastEthernet0/0
     10.0.0.0/24 is subnetted, 1 subnets
R       10.10.10.0 [120/1] via 172.16.10.5, 00:00:04, Serial0/0/1
R    192.168.218.0/24 [120/3] via 172.16.10.5, 00:00:04, Serial0/0/1
R    192.168.118.0/24 [120/2] via 172.16.10.5, 00:00:05, Serial0/0/1
R*   0.0.0.0/0 [120/1] via 172.16.10.5, 00:00:05, Serial0/0/1
\end{verbatim}

Can you see that last entry? It screams that it's an RIP injected route,
but it's also a default route, so our
\texttt{default-information\ originate} command is working! Last, notice
that the gateway of last resort is now set as well.

If all of what you've learned is clear and understood,
congratulations---you're ready to move on to the next chapter right
after you go through the written and hands-on labs, and while you're at
it, don't forget the review questions!

\subsection[Summary]{\texorpdfstring{\protect\hypertarget{c09.xhtmlux5cux23c09-sec-17}{}{}\protect\hypertarget{c09.xhtmlux5cux23Page_399}{}{}Summary}{Summary}}

This chapter covered IP routing in detail. Again, it's extremely
important to fully understand the basics we covered in this chapter
because everything that's done on a Cisco router will typically have
some kind of IP routing configured and running.

You learned how IP routing uses frames to transport packets between
routers and to the destination host. From there, we configured static
routing on our routers and discussed the administrative distance used by
IP to determine the best route to a destination network. You found out
that if you have a stub network, you can configure default routing,
which sets the gateway of last resort on a router.

We then discussed dynamic routing, specifically RIPv2 and how it works
on an internetwork, which is not very well!

\subsection[Exam
Essentials]{\texorpdfstring{\protect\hypertarget{c09.xhtmlux5cux23c09-sec-18}{}{}Exam
Essentials}{Exam Essentials}}

\textbf{Describe the basic IP routing process.} You need to remember
that the frame changes at each hop but that the packet is never changed
or manipulated in any way until it reaches the destination device (the
TTL field in the IP header is decremented for each hop, but that's it!).

\textbf{List the information required by a router to successfully route
packets.} To be able to route packets, a router must know, at a minimum,
the destination address, the location of neighboring routers through
which it can reach remote networks, possible routes to all remote
networks, the best route to each remote network, and how to maintain and
verify routing information.

\textbf{Describe how MAC addresses are used during the routing process.}
A MAC (hardware) address will only be used on a local LAN. It will never
pass a router's interface. A frame uses MAC (hardware) addresses to send
a packet on a LAN. The frame will take the packet to either a host on
the LAN or a router's interface (if the packet is destined for a remote
network). As packets move from one router to another, the MAC addresses
used will change, but normally the original source and destination IP
addresses within the packet will not.

\textbf{View and interpret the routing table of a router.} Use the
\texttt{show\ ip\ route} command to view the routing table. Each route
will be listed along with the source of the routing information. A
\texttt{C} to the left of the route will indicate directly connected
routes, and other letters next to the route can also indicate a
particular routing protocol that provided the information, such as, for
example, \texttt{R} for RIP.

\textbf{Differentiate the three types of routing.} The three types of
routing are static (in which routes are manually configured at the CLI),
dynamic (in which the routers share routing information via a routing
protocol), and default routing (in which a special route is configured
for all traffic without a more specific destination network found in the
table).

\protect\hypertarget{c09.xhtmlux5cux23Page_400}{}{}\textbf{Compare and
contrast static and dynamic routing.} Static routing creates no routing
update traffic and creates less overhead on the router and network
links, but it must be configured manually and does not have the ability
to react to link outages. Dynamic routing creates routing update traffic
and uses more overhead on the router and network links.

\textbf{Configure static routes at the CLI.} The command syntax to add a
route is
\texttt{ip\ route\ {[}destination\_network{]}\ {[}mask{]}\ {[}next-hop\_address\ or\ exitinterface{]}\ {[}administrative\_distance{]}\ {[}permanent{]}}.

\textbf{Create a default route.} To add a default route, use the command
syntax \texttt{ip\ route\ 0.0.0.0\ 0.0.0.0} \texttt{ip-address} or
\texttt{exit\ interface\ type\ and\ number}.

\textbf{Understand administrative distance and its role in the selection
of the best route.} Adminis­trative distance (AD) is used to rate the
trustworthiness of routing information received on a router from a
neighbor router. Administrative distance is an integer from 0 to 255,
where 0 is the most trusted and 255 means no traffic will be passed via
this route. All routing protocols are assigned a default AD, but it can
be changed at the CLI.

\textbf{Differentiate distance-vector, link-state, and hybrid routing
protocols.} Distance-vector routing protocols make routing decisions
based on hop count (think RIP), while link-state routing protocols are
able to consider multiple factors such as bandwidth available and
building a topology table. Hybrid routing protocols exhibit
characteristics of both types.

\textbf{Configure RIPv2 routing.} To configure RIP routing, first you
must be in global configuration mode and then you type the command
\texttt{router\ rip}. Then you add all directly connected networks,
making sure to use the classful address and the \texttt{version\ 2}
command and to disable auto-summarization with the
\texttt{no\ auto-summary} command.

\subsection[Written Lab
9]{\texorpdfstring{\protect\hypertarget{c09.xhtmlux5cux23c09-sec-19}{}{}Written
Lab 9}{Written Lab 9}}

In this section, you'll complete the following lab to make sure you've
got the information and concepts contained within them fully dialed in:

\begin{quote}
Lab 9.1: IP Routing

You can find the answers to this lab in Appendix A, ``Answers to Written
Labs.''
\end{quote}

Write the answers to the following questions:

\begin{enumerate}
\tightlist
\item
  At the appropriate command prompt, create a static route to network
  172.16.10.0/24 with a next-hop gateway of 172.16.20.1 and an
  administrative distance of 150.
\item
  When a PC sends a packet to another PC in a remote network, what
  destination addresses will be in the frame that it sends to its
  default gateway?
\item
  At the appropriate command prompt, create a default route to
  172.16.40.1.
\item
  On which type of network is a default route most beneficial?
\item
  At the appropriate command prompt, display the routing table on your
  router.
\item
  When creating a static or default route, you don't have to use the
  next-hop IP address; you can use the\_\_\_\_\_\_\_\_\_\_\_\_\_\_.
\item
  True/False: To reach a remote host, you must know the MAC address of
  the remote host.
\item
  \protect\hypertarget{c09.xhtmlux5cux23Page_401}{}{}True/False: To
  reach a remote host, you must know the IP address of the remote host.
\item
  At the appropriate command prompt(s), prevent a router from
  propagating RIP information out serial 1.
\item
  True/False: RIPv2 is considered classless.
\end{enumerate}

\subsection[Hands-on
Labs]{\texorpdfstring{\protect\hypertarget{c09.xhtmlux5cux23c09-sec-20}{}{}Hands-on
Labs}{Hands-on Labs}}

In the following hands-on labs, you will configure a network with three
routers. These exercises assume all the same setup requirements as the
labs found in earlier chapters. You can use real routers, the LammleSim
IOS version found at \texttt{www.lammle.com/ccna}, or the Cisco Packet
Tracer program to run these labs.

This chapter includes the following labs:

\begin{enumerate}
\tightlist
\item
  Lab 9.1: Creating Static Routes
\item
  Lab 9.2: Configuring RIP Routing
\end{enumerate}

The internetwork shown in the following graphic will be used to
configure all routers.

\begin{figure}
\centering
\includegraphics{images/c09f010.jpg}
\caption{}
\end{figure}

\protect\hyperlink{c09.xhtmlux5cux23table9-2}{Table 9.2} shows our IP
addresses for each router (each interface uses a /24 mask).

{\protect\hyperlink{c09.xhtmlux5cux23tableanchor9-2}{\textbf{TABLE 9.2}}
Our IP addresses}

\begin{longtable}[]{@{}lll@{}}
\toprule
Router & Interface & IP Address\tabularnewline
\midrule
\endhead
Lab\_A & Fa0/0 & 172.16.10.1\tabularnewline
Lab\_A & S0/0 & 172.16.20.1\tabularnewline
Lab\_B & S0/0 & 172.16.20.2\tabularnewline
Lab\_B & S0/1 & 172.16.30.1\tabularnewline
Lab\_C & S0/0 & 172.16.30.2\tabularnewline
Lab\_C & Fa0/0 & 172.16.40.1\tabularnewline
\bottomrule
\end{longtable}

\protect\hypertarget{c09.xhtmlux5cux23Page_402}{}{}These labs were
written without using the LAN interface on the Lab\_B router. You can
choose to add that LAN into the labs if necessary. Also, if you have
enough LAN interfaces, then you don't need to add the serial interfaces
into this lab. Using all LAN interfaces is fine.

\subsubsection[Hands-on Lab 9.1: Creating Static
Routes]{\texorpdfstring{\protect\hypertarget{c09.xhtmlux5cux23c09-sec-21}{}{}Hands-on
Lab 9.1: Creating Static
Routes}{Hands-on Lab 9.1: Creating Static Routes}}

In this lab, you will create a static route in all three routers so that
the routers see all networks. Verify with the Ping program when
complete.

\begin{enumerate}
\item
  The Lab\_A router is connected to two networks, 172.16.10.0 and
  172.16.20.0. You need to add routes to networks 172.16.30.0 and
  172.16.40.0. Use the following commands to add the static routes:

\begin{verbatim}
Lab_A#config t
Lab_A(config)#ip route 172.16.30.0 255.255.255.0
  172.16.20.2
Lab_A(config)#ip route 172.16.40.0 255.255.255.0
  172.16.20.2
\end{verbatim}
\item
  Save the current configuration for the Lab\_A router by going to
  privileged mode, typing \texttt{copy\ run\ start}, and pressing Enter.
\item
  On the Lab\_B router, you have direct connections to networks
  172.16.20.0 and 172.16.30.0. You need to add routes to networks
  172.16.10.0 and 172.16.40.0. Use the following commands to add the
  static routes:

\begin{verbatim}
Lab_B#config t
Lab_B(config)#ip route 172.16.10.0 255.255.255.0
  172.16.20.1
Lab_B(config)#ip route 172.16.40.0 255.255.255.0
  172.16.30.2
\end{verbatim}
\item
  Save the current configuration for router Lab\_B by going to the
  enabled mode, typing \texttt{copy\ run\ start}, and pressing Enter.
\item
  On router Lab\_C, create a static route to networks 172.16.10.0 and
  172.16.20.0, which are not directly connected. Create static routes so
  that router Lab\_C can see all networks, using the commands shown
  here:

\begin{verbatim}
Lab_C#config t
Lab_C(config)#ip route 172.16.10.0 255.255.255.0
  172.16.30.1
Lab_C(config)#ip route 172.16.20.0 255.255.255.0
  172.16.30.1
\end{verbatim}
\item
  Save the current configuration for router Lab\_C by going to the
  enable mode, typing \texttt{copy\ run\ start}, and pressing Enter.
\item
  \protect\hypertarget{c09.xhtmlux5cux23Page_403}{}{}Check your routing
  tables to make sure all four networks show up by executing the
  \texttt{show\ ip\ route} command.
\item
  Now ping from each router to your hosts and from each router to each
  router. If it is set up correctly, it will work.
\end{enumerate}

\subsubsection[Hands-on Lab 9.2: Configuring RIP
Routing]{\texorpdfstring{\protect\hypertarget{c09.xhtmlux5cux23c09-sec-22}{}{}Hands-on
Lab 9.2: Configuring RIP
Routing}{Hands-on Lab 9.2: Configuring RIP Routing}}

In this lab, we will use the dynamic routing protocol RIP instead of
static routing.

\begin{enumerate}
\item
  Remove any static routes or default routes configured on your routers
  by using the \texttt{no\ ip\ route} command. For example, here is how
  you would remove the static routes on the Lab\_A router:

\begin{verbatim}
Lab_A#config t
Lab_A(config)#no ip route 172.16.30.0 255.255.255.0
  172.16.20.2
Lab_A(config)#no ip route 172.16.40.0 255.255.255.0
  172.16.20.2
\end{verbatim}

  Do the same thing for routers Lab\_B and Lab\_C. Verify that only your
  directly connected networks are in the routing tables.
\item
  After your static and default routes are clear, go into configuration
  mode on router Lab\_A by typing \texttt{config\ t}.
\item
  Tell your router to use RIP routing by typing \texttt{router\ rip} and
  pressing Enter, as shown here:

\begin{verbatim}
config t
router rip
\end{verbatim}
\item
  Add the network number for the networks you want to advertise. Since
  router Lab\_A has two interfaces that are in two different networks,
  you must enter a network statement using the network ID of the network
  in which each interface resides. Alternately, you could use a
  summarization of these networks and use a single statement, minimizing
  the size of the routing table. Since the two networks are
  172.16.10.0/24 and 172.16.20.0/24, the network summarization
  172.16.0.0 would include both subnets. Do this by typing
  \texttt{network\ 172.16.0.0} and pressing Enter.
\item
  Press Ctrl+Z to get out of configuration mode.
\item
  The interfaces on Lab\_B and Lab\_C are in the 172.16.20.0/24 and
  172.16.30.0/24 networks; therefore, the same summarized network
  statement will work there as well. Type the same commands, as shown
  here:

\begin{verbatim}
Config t
Router rip
network 172.16.0.0
\end{verbatim}
\item
  \protect\hypertarget{c09.xhtmlux5cux23Page_404}{}{}Verify that RIP is
  running at each router by typing the following commands at each
  router:

\begin{verbatim}
show ip protocols
\end{verbatim}

  (Should indicate to you that RIP is present on the router.)

\begin{verbatim}
show ip route
\end{verbatim}

  (Should have routes present with an \texttt{R} to the left of them.)

\begin{verbatim}
show running-config or show run
\end{verbatim}

  (Should indicate that RIP is present and the networks are being
  advertised.)
\item
  Save your configurations by typing \texttt{copy\ run\ start} or
  \texttt{copy\ running-config\ startup-config} and pressing Enter at
  each router.
\item
  Verify the network by pinging all remote networks and hosts.
\end{enumerate}

\subsection[Review
Questions]{\texorpdfstring{\protect\hypertarget{c09.xhtmlux5cux23c09-sec-23}{}{}\protect\hypertarget{c09.xhtmlux5cux23Page_405}{}{}Review
Questions}{Review Questions}}

\begin{center}\rule{0.5\linewidth}{0.5pt}\end{center}

\includegraphics{images/note.png} The following questions are designed
to test your understanding of this chapter's material. For more
information on how to get additional questions, please see
\texttt{www.lammle.com/ccna}.

\begin{center}\rule{0.5\linewidth}{0.5pt}\end{center}

You can find the answers to these questions in Appendix B, ``Answers to
Review Questions.''

\begin{enumerate}
\item
  What command was used to generate the following output?

\begin{verbatim}
Codes: L - local, C - connected, S - static,
[output cut]
        10.0.0.0/8 is variably subnetted, 6 subnets, 4 masks
C       10.0.0.0/8 is directly connected, FastEthernet0/3
L       10.0.0.1/32 is directly connected, FastEthernet0/3
C       10.10.0.0/16 is directly connected, FastEthernet0/2
L       10.10.0.1/32 is directly connected, FastEthernet0/2
C       10.10.10.0/24 is directly connected, FastEthernet0/1
L       10.10.10.1/32 is directly connected, FastEthernet0/1
S*      0.0.0.0/0 is directly connected, FastEthernet0/0
\end{verbatim}
\item
  You are viewing the routing table and you see an entry 10.1.1.1/32.
  What legend code would you expect to see next to this route?

  \begin{enumerate}
  \tightlist
  \item
    C
  \item
    L
  \item
    S
  \item
    D
  \end{enumerate}
\item
  Which of the following statements are true regarding the command
  \texttt{ip\ route\ 172.16.4.0\ 255.255.255.0\ 192.168.4.2}? (Choose
  two.)

  \begin{enumerate}
  \tightlist
  \item
    The command is used to establish a static route.
  \item
    The default administrative distance is used.
  \item
    The command is used to configure the default route.
  \item
    The subnet mask for the source address is 255.255.255.0.
  \item
    The command is used to establish a stub network.
  \end{enumerate}
\item
  What destination addresses will be used by HostA to send data to the
  HTTPS server as shown in the following network? (Choose two.)

  \begin{enumerate}
  \item
    The IP address of the switch
  \item
    The MAC address of the remote switch
  \item
    The IP address of the HTTPS server
  \item
    \protect\hypertarget{c09.xhtmlux5cux23Page_406}{}{}The MAC address
    of the HTTPS server
  \item
    The IP address of RouterA's Fa0/0 interface
  \item
    The MAC address of RouterA's Fa0/0 interface

    \begin{figure}
    \centering
    \includegraphics{images/c09f011.jpg}
    \caption{}
    \end{figure}
  \end{enumerate}
\item
  Using the output shown, what protocol was used to learn the MAC
  address for 172.16.10.1?

\begin{verbatim}
Interface: 172.16.10.2 --- 0x3
  Internet Address      Physical Address      Type
  172.16.10.1          00-15-05-06-31-b0     dynamic
\end{verbatim}

  \begin{enumerate}
  \tightlist
  \item
    ICMP
  \item
    ARP
  \item
    TCP
  \item
    UDP
  \end{enumerate}
\item
  Which of the following is called an advanced distance-vector routing
  protocol?

  \begin{enumerate}
  \tightlist
  \item
    OSPF
  \item
    EIGRP
  \item
    BGP
  \item
    RIP
  \end{enumerate}
\item
  When a packet is routed across a network,
  the\_\_\_\_\_\_\_\_\_\_\_\_\_\_\_\_\_ in the packet changes at every
  hop while the\_\_\_\_\_\_\_\_\_\_ does not.

  \begin{enumerate}
  \tightlist
  \item
    MAC address, IP address
  \item
    IP address, MAC address
  \item
    Port number, IP address
  \item
    IP address, port number
  \end{enumerate}
\item
  \protect\hypertarget{c09.xhtmlux5cux23Page_407}{}{}Which statements
  are true regarding classless routing protocols? (Choose two.)

  \begin{enumerate}
  \tightlist
  \item
    The use of discontiguous networks is not allowed.
  \item
    The use of variable length subnet masks is permitted.
  \item
    RIPv1 is a classless routing protocol.
  \item
    IGRP supports classless routing within the same autonomous system.
  \item
    RIPv2 supports classless routing.
  \end{enumerate}
\item
  Which two of the following are true regarding the distance-vector and
  link-state routing protocols? (Choose two.)

  \begin{enumerate}
  \tightlist
  \item
    Link state sends its complete routing table out of all active
    interfaces at periodic time intervals.
  \item
    Distance vector sends its complete routing table out of all active
    interfaces at periodic time intervals.
  \item
    Link state sends updates containing the state of its own links to
    all routers in the internetwork.
  \item
    Distance vector sends updates containing the state of its own links
    to all routers in the internetwork.
  \end{enumerate}
\item
  When a router looks up the destination in the routing table for every
  single packet, it is called\_\_\_\_\_\_\_\_\_\_\_\_\_.

  \begin{enumerate}
  \tightlist
  \item
    dynamic switching
  \item
    fast switching
  \item
    process switching
  \item
    Cisco Express Forwarding
  \end{enumerate}
\item
  What type(s) of route is the following? (Choose all that apply.)

\begin{verbatim}
S*   0.0.0.0/0 [1/0] via 172.16.10.5
\end{verbatim}

  \begin{enumerate}
  \tightlist
  \item
    Default
  \item
    Subnetted
  \item
    Static
  \item
    Local
  \end{enumerate}
\item
  A network administrator views the output from the
  \texttt{show\ ip\ route} command. A network that is advertised by both
  RIP and EIGRP appears in the routing table flagged as an EIGRP route.
  Why is the RIP route to this network not used in the routing table?

  \begin{enumerate}
  \tightlist
  \item
    EIGRP has a faster update timer.
  \item
    EIGRP has a lower administrative distance.
  \item
    RIP has a higher metric value for that route.
  \item
    The EIGRP route has fewer hops.
  \item
    The RIP path has a routing loop.
  \end{enumerate}
\item
  \protect\hypertarget{c09.xhtmlux5cux23Page_408}{}{}Which of the
  following is \emph{not} an advantage of static routing?

  \begin{enumerate}
  \tightlist
  \item
    Less overhead on the router CPU
  \item
    No bandwidth usage between routers
  \item
    Adds security
  \item
    Recovers automatically from lost routes
  \end{enumerate}
\item
  What metric does RIPv2 use to find the best path to a remote network?

  \begin{enumerate}
  \tightlist
  \item
    Hop count
  \item
    MTU
  \item
    Cumulative interface delay
  \item
    Load
  \item
    Path bandwidth value
  \end{enumerate}
\item
  The Corporate router receives an IP packet with a source IP address of
  192.168.214.20 and a destination address of 192.168.22.3. Looking at
  the output from the Corp router, what will the router do with this
  packet?

\begin{verbatim}
Corp#sh ip route
[output cut]
R    192.168.215.0 [120/2] via 192.168.20.2, 00:00:23, Serial0/0
R    192.168.115.0 [120/1] via 192.168.20.2, 00:00:23, Serial0/0
R    192.168.30.0 [120/1] via 192.168.20.2, 00:00:23, Serial0/0
C    192.168.20.0 is directly connected, Serial0/0
C    192.168.214.0 is directly connected, FastEthernet0/0
\end{verbatim}

  \begin{enumerate}
  \tightlist
  \item
    The packet will be discarded.
  \item
    The packet will be routed out of the S0/0 interface.
  \item
    The router will broadcast looking for the destination.
  \item
    The packet will be routed out of the Fa0/0 interface.
  \end{enumerate}
\item
  If your routing table has a static, an RIP, and an EIGRP route to the
  same network, which route will be used to route packets by default?

  \begin{enumerate}
  \tightlist
  \item
    Any available route
  \item
    RIP route
  \item
    Static route
  \item
    EIGRP route
  \item
    They will all load-balance.
  \end{enumerate}
\item
  Which of the following is an EGP?

  \begin{enumerate}
  \tightlist
  \item
    RIPv2
  \item
    EIGRP
  \item
    BGP
  \item
    RIP
  \end{enumerate}
\item
  \protect\hypertarget{c09.xhtmlux5cux23Page_409}{}{}Which of the
  following is an advantage of static routing?

  \begin{enumerate}
  \tightlist
  \item
    Less overhead on the router CPU
  \item
    No bandwidth usage between routers
  \item
    Adds security
  \item
    Recovers automatically from lost routes
  \end{enumerate}
\item
  What command produced the following output?

\begin{verbatim}
Interface         IP-Address      OK? Method Status             Protocol
FastEthernet0/0   192.168.10.1    YES manual up                    up
FastEthernet0/1   unassigned      YES unset  administratively down down
Serial0/0/0       172.16.10.2     YES manual up                    up
Serial0/0/1       unassigned      YES unset  administratively down down
\end{verbatim}

  \begin{enumerate}
  \tightlist
  \item
    \texttt{show\ ip\ route}
  \item
    \texttt{show\ interfaces}
  \item
    \texttt{show\ ip\ interface\ brief}
  \item
    \texttt{show\ ip\ arp}
  \end{enumerate}
\item
  What does the 150 at the end of the following command mean?

\begin{verbatim}
Router(config)#ip route 172.16.3.0 255.255.255.0 192.168.2.4 150
\end{verbatim}

  \begin{enumerate}
  \tightlist
  \item
    Metric
  \item
    Administrative distance
  \item
    Hop count
  \item
    Cost
  \end{enumerate}
\end{enumerate}

\protect\hypertarget{c10.xhtml}{}{}

\section[{Chapter 10}\\
{Layer 2
Switching}]{\texorpdfstring{\protect\hypertarget{c10.xhtmlux5cux23c10}{}{}\protect\hypertarget{c10.xhtmlux5cux23Page_411}{}{}{Chapter
10}\\
{Layer 2 Switching}}{Chapter 10 Layer 2 Switching}}

\begin{center}\rule{0.5\linewidth}{0.5pt}\end{center}

\subsection{THE FOLLOWING ICND1 EXAM TOPICS ARE COVERED IN THIS
CHAPTER:}

\begin{enumerate}
\tightlist
\item
  \includegraphics{images/tick.png} 2.0 LAN Switching Technologies

  \begin{enumerate}
  \tightlist
  \item
    \includegraphics{images/squ.png} 2.1 Describe and verify switching
    concepts

    \begin{enumerate}
    \tightlist
    \item
      \includegraphics{images/squ.png} 2.1.a MAC learning and aging
    \item
      \includegraphics{images/squ.png} 2.1.b Frame switching
    \item
      \includegraphics{images/squ.png} 2.1.c Frame flooding
    \item
      \includegraphics{images/squ.png} 2.1.d MAC address table
    \end{enumerate}
  \item
    \includegraphics{images/squ.png} 2.7 Configure, verify, and
    troubleshoot port security

    \begin{enumerate}
    \tightlist
    \item
      \includegraphics{images/squ.png} 2.7.a Static
    \item
      \includegraphics{images/squ.png} 2.7.b Dynamic
    \item
      \includegraphics{images/squ.png} 2.7.c Sticky
    \item
      \includegraphics{images/squ.png} 2.7.d Max MAC addresses
    \item
      \includegraphics{images/squ.png} 2.7.e Violation actions
    \item
      \includegraphics{images/squ.png} 2.7.f Err-disable recovery
    \end{enumerate}
  \end{enumerate}
\end{enumerate}

\protect\hypertarget{c10.xhtmlux5cux23Page_412}{}{}\includegraphics{images/intro.png}When
people at Cisco discuss switching in regards to the Cisco exam
objectives, they're talking about layer 2 switching unless they say
otherwise. Layer 2 switching is the process of using the hardware
address of devices on a LAN to segment a network. Since you've got the
basic idea of how that works nailed down by now, we're going to dive
deeper into the particulars of layer 2 switching to ensure that your
concept of how it works is solid and complete.

You already know that we rely on switching to break up large collision
domains into smaller ones and that a collision domain is a network
segment with two or more devices sharing the same bandwidth. A hub
network is a typical example of this type of technology. But since each
port on a switch is actually its own collision domain, we were able to
create a much better Ethernet LAN network by simply replacing our hubs
with switches!

Switches truly have changed the way networks are designed and
implemented. If a pure switched design is properly implemented, it
absolutely will result in a clean, cost-effective, and resilient
internetwork. In this chapter, we'll survey and compare how networks
were designed before and after switching technologies were introduced.

I'll be using three switches to begin our configuration of a switched
network, and we'll actually continue with their configurations in
Chapter 11, ``VLANs and Inter-VLAN Routing.''

\begin{center}\rule{0.5\linewidth}{0.5pt}\end{center}

\includegraphics{images/note.png} To find up-to-the-minute updates for
this chapter, please see \texttt{www.lammle.com/ccna} or the book's web
page at \texttt{www.sybex.com/go/ccna}.

\begin{center}\rule{0.5\linewidth}{0.5pt}\end{center}

\subsection[Switching
Services]{\texorpdfstring{\protect\hypertarget{c10.xhtmlux5cux23c10-sec-1}{}{}Switching
Services}{Switching Services}}

Unlike old bridges, which used software to create and manage a Content
Addressable Memory (CAM) filter table, our new, fast switches use
application-specific integrated ­circuits (ASICs) to build and maintain
their MAC filter tables. But it's still okay to think of a layer 2
switch as a multiport bridge because their basic reason for being is the
same: to break up collision domains.

Layer 2 switches and bridges are faster than routers because they don't
take up time looking at the Network layer header information. Instead,
they look at the frame's ­hardware addresses before deciding to either
forward, flood, or drop the frame.

\protect\hypertarget{c10.xhtmlux5cux23Page_413}{}{}Unlike hubs, switches
create private, dedicated collision domains and provide ­independent
bandwidth exclusive on each port.

Here's a list of four important advantages we gain when using layer 2
switching:

\begin{enumerate}
\tightlist
\item
  Hardware-based bridging (ASICs)
\item
  Wire speed
\item
  Low latency
\item
  Low cost
\end{enumerate}

A big reason layer 2 switching is so efficient is that no modification
to the data packet takes place. The device only reads the frame
encapsulating the packet, which makes the switching process considerably
faster and less error-prone than routing ­processes are.

And if you use layer 2 switching for both workgroup connectivity and
network ­segmentation (breaking up collision domains), you can create
more network segments than you can with traditional routed networks.
Plus, layer 2 switching increases bandwidth for each user because,
again, each connection, or interface into the switch, is its own,
self-contained collision domain.

\subsubsection[Three Switch Functions at Layer
2]{\texorpdfstring{\protect\hypertarget{c10.xhtmlux5cux23c10-sec-2}{}{}Three
Switch Functions at Layer 2}{Three Switch Functions at Layer 2}}

There are three distinct functions of layer 2 switching that are vital
for you to remember: \emph{address learning}, \emph{forward/filter
decisions}, and \emph{loop avoidance}.

\textbf{Address learning} Layer 2 switches remember the source hardware
address of each frame received on an interface and enter this
information into a MAC database called a forward/filter table.

\textbf{Forward/filter decisions} When a frame is received on an
interface, the switch looks at the destination hardware address, then
chooses the appropriate exit interface for it in the MAC database. This
way, the frame is only forwarded out of the correct destination port.

\textbf{Loop avoidance} If multiple connections between switches are
created for redundancy ­purposes, network loops can occur. Spanning Tree
Protocol (STP) is used to prevent ­network loops while still permitting
redundancy.

Next, I'm going to talk about address learning and forward/filtering
decisions. Loop avoidance is beyond the scope of the objectives being
covered in this chapter.

\paragraph{Address Learning}

When a switch is first powered on, the MAC forward/filter table (CAM) is
empty, as shown in
\protect\hyperlink{c10.xhtmlux5cux23figure10-1}{Figure 10.1}.

\protect\hypertarget{c10.xhtmlux5cux23Page_414}{}{}

\begin{figure}
\centering
\includegraphics{images/c10f001.jpg}
\caption{{\protect\hyperlink{c10.xhtmlux5cux23figureanchor10-1}{\textbf{FIGURE
10.1}} Empty forward/filter table on a switch}}
\end{figure}

When a device transmits and an interface receives a frame, the switch
places the frame's source address in the MAC forward/filter table,
allowing it to refer to the precise interface the sending device is
located on. The switch then has no choice but to flood the network with
this frame out of every port except the source port because it has no
idea where the destination device is actually located.

If a device answers this flooded frame and sends a frame back, then the
switch will take the source address from that frame and place that MAC
address in its database as well, associating this address with the
interface that received the frame. Because the switch now has both of
the relevant MAC addresses in its filtering table, the two devices can
now make a point-to-point connection. The switch doesn't need to flood
the frame as it did the first time because now the frames can and will
only be forwarded between these two devices. This is exactly why layer 2
switches are so superior to hubs. In a hub network, all frames are
forwarded out all ports every time---no matter what.
\protect\hyperlink{c10.xhtmlux5cux23figure10-2}{Figure 10.2} shows the
processes involved in building a MAC database.

\begin{figure}
\centering
\includegraphics{images/c10f002.jpg}
\caption{{\protect\hyperlink{c10.xhtmlux5cux23figureanchor10-2}{\textbf{FIGURE
10.2}} How switches learn hosts' locations}}
\end{figure}

In this figure, you can see four hosts attached to a switch. When the
switch is powered on, it has nothing in its MAC address forward/filter
table, just as in \protect\hyperlink{c10.xhtmlux5cux23figure10-1}{Figure
10.1}. But when the hosts start communicating, the switch places the
source hardware address of each frame into the table along with the port
that the frame's source address corresponds to.

\protect\hypertarget{c10.xhtmlux5cux23Page_415}{}{}Let me give you an
example of how a forward/filter table is populated using
\protect\hyperlink{c10.xhtmlux5cux23figure10-2}{Figure 10.2}:

\begin{enumerate}
\tightlist
\item
  Host A sends a frame to Host B. Host A's MAC address is
  0000.8c01.000A; Host B's MAC address is 0000.8c01.000B.
\item
  The switch receives the frame on the Fa0/0 interface and places the
  source address in the MAC address table.
\item
  Since the destination address isn't in the MAC database, the frame is
  forwarded out all interfaces except the source port.
\item
  Host B receives the frame and responds to Host A. The switch receives
  this frame on interface Fa0/1 and places the source hardware address
  in the MAC database.
\item
  Host A and Host B can now make a point-to-point connection and only
  these specific devices will receive the frames. Hosts C and D won't
  see the frames, nor will their MAC addresses be found in the database
  because they haven't sent a frame to the switch yet.
\end{enumerate}

If Host A and Host B don't communicate to the switch again within a
certain time period, the switch will flush their entries from the
database to keep it as current as possible.

\paragraph{Forward/Filter Decisions}

When a frame arrives at a switch interface, the destination hardware
address is compared to the forward/filter MAC database. If the
destination hardware address is known and listed in the database, the
frame is only sent out of the appropriate exit interface. The switch
won't transmit the frame out any interface except for the destination
interface, which preserves bandwidth on the other network segments. This
process is called \emph{frame filtering}.

But if the destination hardware address isn't listed in the MAC
database, then the frame will be flooded out all active interfaces
except the interface it was received on. If a device answers the flooded
frame, the MAC database is then updated with the device's location---its
correct interface.

If a host or server sends a broadcast on the LAN, by default, the switch
will flood the frame out all active ports except the source port.
Remember, the switch creates smaller collision domains, but it's always
still one large broadcast domain by default.

In \protect\hyperlink{c10.xhtmlux5cux23figure10-3}{Figure 10.3}, Host A
sends a data frame to Host D. What do you think the switch will do when
it receives the frame from Host A?

\begin{figure}
\centering
\includegraphics{images/c10f003.jpg}
\caption{{\protect\hyperlink{c10.xhtmlux5cux23figureanchor10-3}{\textbf{FIGURE
10.3}} Forward/filter table}}
\end{figure}

\protect\hypertarget{c10.xhtmlux5cux23Page_416}{}{}Let's examine
\protect\hyperlink{c10.xhtmlux5cux23figure10-4}{Figure 10.4} to find the
answer.

\begin{figure}
\centering
\includegraphics{images/c10f004.jpg}
\caption{{\protect\hyperlink{c10.xhtmlux5cux23figureanchor10-4}{\textbf{FIGURE
10.4}} Forward/filter table answer}}
\end{figure}

Since Host A's MAC address is not in the forward/filter table, the
switch will add the source address and port to the MAC address table,
then forward the frame to Host D. It's really important to remember that
the source MAC is always checked first to make sure it's in the CAM
table. After that, if Host D's MAC address wasn't found in the
forward/filter table, the switch would've flooded the frame out all
ports except for port Fa0/3 because that's the specific port the frame
was received on.

Now let's take a look at the output that results from using a
\texttt{show\ mac\ address-table} command:

\begin{verbatim}
Switch#sh mac address-table
Vlan    Mac Address       Type        Ports
----    -----------       --------    -----
   1    0005.dccb.d74b    DYNAMIC     Fa0/1
   1    000a.f467.9e80    DYNAMIC     Fa0/3
   1    000a.f467.9e8b    DYNAMIC     Fa0/4
   1    000a.f467.9e8c    DYNAMIC     Fa0/3
   1    0010.7b7f.c2b0    DYNAMIC     Fa0/3
   1    0030.80dc.460b    DYNAMIC     Fa0/3
   1    0030.9492.a5dd    DYNAMIC     Fa0/1
   1    00d0.58ad.05f4    DYNAMIC     Fa0/1
\end{verbatim}

But let's say the preceding switch received a frame with the following
MAC addresses:

\begin{enumerate}
\tightlist
\item
  Source MAC: \textbf{0005.dccb.d74b}
\item
  Destination MAC: \textbf{000a.f467.9e8c}
\end{enumerate}

How will the switch handle this frame? The right answer is that the
destination MAC address will be found in the MAC address table and the
frame will only be forwarded out Fa0/3. Never forget that if the
destination MAC address isn't found in the forward/filter
\protect\hypertarget{c10.xhtmlux5cux23Page_417}{}{}table, the frame will
be forwarded out all of the switch's ports except for the one on which
it was originally received in an attempt to locate the destination
device. Now that you can see the MAC address table and how switches add
host addresses to the forward filter table, how do think we can secure
it from unauthorized users?

\subsubsection[Port
Security]{\texorpdfstring{\protect\hypertarget{c10.xhtmlux5cux23c10-sec-3}{}{}Port
Security}{Port Security}}

It's usually not a good thing to have your switches available for anyone
to just plug into and play around with. I mean, we worry about wireless
security, so why wouldn't we demand switch security just as much, if not
more?

But just how do we actually prevent someone from simply plugging a host
into one of our switch ports---or worse, adding a hub, switch, or access
point into the Ethernet jack in their office? By default, MAC addresses
will just dynamically appear in your MAC ­forward/filter database and
you can stop them in their tracks by using port security!

\protect\hyperlink{c10.xhtmlux5cux23figure10-5}{Figure 10.5} shows two
hosts connected to the single switch port Fa0/3 via either a hub or
access point (AP).

\begin{figure}
\centering
\includegraphics{images/c10f005.jpg}
\caption{{\protect\hyperlink{c10.xhtmlux5cux23figureanchor10-5}{\textbf{FIGURE
10.5}} ``Port security'' on a switch port restricts port access by MAC
address.}}
\end{figure}

Port Fa0/3 is configured to observe and allow only certain MAC addresses
to associate with the specific port, so in this example, Host A is
denied access, but Host B is allowed to associate with the port.

By using port security, you can limit the number of MAC addresses that
can be assigned dynamically to a port, set static MAC addresses,
and---here's my favorite part---set penalties for users who abuse your
policy! Personally, I like to have the port shut down when the security
policy is violated. Making abusers bring me a memo from their boss
explaining why they violated the security policy brings with it a
certain poetic justice, which is nice. And I'll also require something
like that before I'll enable their port again. Things like this really
seem to help people remember to behave!

This is all good, but you still need to balance your particular security
needs with the time that implementing and managing them will
realistically require. If you have tons of time on your hands, then go
ahead and seriously lock your network down vault-tight! If you're busy
like the rest of us, I'm here to reassure you that there are ways to
secure things nicely without being totally overwhelmed with a massive
amount of administrative
\protect\hypertarget{c10.xhtmlux5cux23Page_418}{}{}overhead. First, and
painlessly, always remember to shut down unused ports or assign them to
an unused VLAN. All ports are enabled by default, so you need to make
sure there's no access to unused switch ports!

Here are your options for configuring port security:

\begin{verbatim}
Switch#config t
Switch(config)#int f0/1
Switch(config-if)#switchport mode access
Switch(config-if)#switchport port-security
Switch(config-if)#switchport port-security ?
   aging           Port-security aging commands
   mac-address     Secure mac address
   maximum         Max secure addresses
   violation       Security violation mode
   <cr>
\end{verbatim}

Most Cisco switches ship with their ports in desirable mode, which means
that those ports will desire to trunk when sensing that another switch
has just been connected. So first, we need to change the port out from
desirable mode and make it an access port instead. If we don't do that,
we won't be able to configure port security on it at all! Once that's
out of the way, we can move on using our \texttt{port-security}
commands, never ­forgetting that we must enable port security on the
interface with the basic command \texttt{switchport\ port-security}.
Notice that I did this after I made the port an access port!

The preceding output clearly illustrates that the
\texttt{switchport\ port-security} command can be used with four
options. You can use the \texttt{switchport\ port-security\ mac-address}
\texttt{mac-address} command to assign individual MAC addresses to each
switch port, but be warned because if you go with that option, you had
better have boatloads of time on your hands!

You can configure the device to take one of the following actions when a
security violation occurs by using the
\texttt{switchport\ port-security} command:

\begin{enumerate}
\tightlist
\item
  \texttt{Protect}: The protect violation mode drops packets with
  unknown source addresses until you remove enough secure MAC addresses
  to drop below the maximum value.
\item
  \texttt{Restrict}: The restrict violation mode also drops packets with
  unknown source addresses until you remove enough secure MAC addresses
  to drop below the maximum value. However, it also generates a log
  message, causes the security violation counter to increment, and sends
  an SNMP trap.
\item
  \texttt{Shutdown}: Shutdown is the default violation mode. The
  shutdown violation mode puts the interface into an error-disabled
  state immediately. The entire port is shut down. Also, in this mode,
  the system generates a log message, sends an SNMP trap, and increments
  the violation counter. To make the interface usable, you must perform
  a \texttt{shut/no\ shut} on the interface.
\end{enumerate}

\protect\hypertarget{c10.xhtmlux5cux23Page_419}{}{}If you want to set up
a switch port to allow only one host per port and make sure the port
will shut down if this rule is violated, use the following commands like
this:

\begin{verbatim}
Switch(config-if)#switchport port-security maximum 1
Switch(config-if)#switchport port-security violation shutdown
\end{verbatim}

These commands really are probably the most popular because they prevent
random users from connecting to a specific switch or access point that's
in their office. The port security default that's immediately set on a
port when it's enabled is \texttt{maximum} \texttt{1} and
­\texttt{violation\ shutdown}. This sounds okay, but the drawback to
this is that it only allows a single MAC address to be used on the port,
so if anyone, including you, tries to add another host on that segment,
the switch port will immediately enter error-disabled state and the port
will turn amber. And when that happens, you have to manually go into the
switch and re-enable the port by cycling it with a \texttt{shutdown} and
then a \texttt{no\ shutdown} command.

Probably one of my favorite commands is the \texttt{sticky} command, and
not just because it's got a cool name. It also makes very cool things
happen! You can find this command under the \texttt{mac-address}
command:

\begin{verbatim}
Switch(config-if)#switchport port-security mac-address sticky
Switch(config-if)#switchport port-security maximum 2
Switch(config-if)#switchport port-security violation shutdown
\end{verbatim}

Basically, with the \texttt{sticky} command you can provide static MAC
address security without having to type in absolutely everyone's MAC
address on the network. I like things that save me time like that!

In the preceding example, the first two MAC addresses coming into the
port ``stick'' to it as static addresses and will be placed in the
running-config, but when a third address tried to connect, the port
would shut down immediately.

\begin{center}\rule{0.5\linewidth}{0.5pt}\end{center}

\includegraphics{images/note.png} I'll be going over port security CCENT
objectives again in the configuration examples later in this chapter.
They're important!

\begin{center}\rule{0.5\linewidth}{0.5pt}\end{center}

Let me show you one more example.
\protect\hyperlink{c10.xhtmlux5cux23figure10-6}{Figure 10.6} displays a
host in a company lobby that needs to be secured against the Ethernet
cable used by anyone other than a single authorized individual.

\begin{figure}
\centering
\includegraphics{images/c10f006.jpg}
\caption{{\protect\hyperlink{c10.xhtmlux5cux23figureanchor10-6}{\textbf{FIGURE
10.6}} Protecting a PC in a lobby}}
\end{figure}

\protect\hypertarget{c10.xhtmlux5cux23Page_420}{}{}What can you do to
ensure that only the MAC address of the lobby PC is allowed by switch
port Fa0/1?

The solution is pretty straightforward because in this case, the
defaults for port security will work well. All I have left to do is add
a static MAC entry:

\begin{verbatim}
Switch(config-if)#switchport port-security
Switch(config-if)#switchport port-security violation restrict
Switch(config-if)#switchport port-security mac-address aa.bb.cc.dd.ee.ff
\end{verbatim}

To protect the lobby PC, we would set the maximum allowed MAC addresses
to 1 and the violation to \texttt{restrict} so the port didn't get shut
down every time someone tried to use the Ethernet cable (which would be
constantly). By using \texttt{violation\ restrict}, the unauthorized
frames would just be dropped. But did you notice that I enabled
\texttt{port-security} and then set a static MAC address? Remember that
as soon as you enable \texttt{port-security} on a port, it defaults to
\texttt{violation\ shutdown} and a maximum of 1. So all I needed to do
was change the violation mode and add the static MAC address and our
business requirement is solidly met!

\begin{center}\rule{0.5\linewidth}{0.5pt}\end{center}

\includegraphics{images/globe1.png}\\
\textbf{Lobby PC Always Being Disconnected Becomes a Security Risk}

At a large Fortune 50 company in San Jose, California, there was a PC in
the lobby that held the company directory. With no security guard
present in the lobby, the Ethernet cable connecting the PC was free game
to all vendors, contractors, and visitors waiting in the lobby.

Port security to the rescue! When port security was enabled on the port
with the ­\texttt{switchport\ port-security} command, the switch port
connecting to the PC was ­automatically secured with the defaults of
allowing only one MAC address to associate to the port and violation
shutdown. However, the port was always going into err-shutdown mode
whenever anyone tried to use the Ethernet port. When the violation mode
was changed to \texttt{restrict} and a static MAC address was set for
the port with the \texttt{switchport\ port-security\ mac-address}
command, only the Lobby PC was able to connect and communicate on the
network! Problem solved!

\begin{center}\rule{0.5\linewidth}{0.5pt}\end{center}

\paragraph{Loop Avoidance}

Redundant links between switches are important to have in place because
they help prevent nasty network failures in the event that one link
stops working.

But while it's true that redundant links can be extremely helpful, they
can also cause more problems than they solve! This is because frames can
be flooded down all redundant
\protect\hypertarget{c10.xhtmlux5cux23Page_421}{}{}links simultaneously,
creating network loops as well as other evils. Here's a list of some of
the ugliest problems that can occur:

\begin{enumerate}
\item
  If no loop avoidance schemes are put in place, the switches will flood
  broadcasts ­endlessly throughout the internetwork. This is sometimes
  referred to as a \emph{broadcast storm}. Most of the time, they're
  referred to in very unprintable ways!
  \protect\hyperlink{c10.xhtmlux5cux23figure10-7}{Figure 10.7}
  ­illustrates how a broadcast can be propagated throughout the network.
  Observe how a frame is continually being flooded through the
  internetwork's physical network media.

  \begin{figure}
  \centering
  \includegraphics{images/c10f007.jpg}
  \caption{{\protect\hyperlink{c10.xhtmlux5cux23figureanchor10-7}{\textbf{FIGURE
  10.7}} Broadcast storm}}
  \end{figure}
\item
  A device can receive multiple copies of the same frame because that
  frame can arrive from different segments at the same time.
  \protect\hyperlink{c10.xhtmlux5cux23figure10-8}{Figure 10.8}
  demonstrates how a whole bunch of frames can arrive from multiple
  segments simultaneously. The server in the figure sends a unicast
  frame to Router C. Because it's a unicast frame, Switch A forwards the
  frame and Switch B provides the same service---it forwards the
  unicast. This is bad because it means that Router C receives that
  unicast frame twice, causing additional overhead on the network.

  \begin{figure}
  \centering
  \includegraphics{images/c10f008.jpg}
  \caption{{\protect\hyperlink{c10.xhtmlux5cux23figureanchor10-8}{\textbf{FIGURE
  10.8}} Multiple frame copies}}
  \end{figure}
\item
  \protect\hypertarget{c10.xhtmlux5cux23Page_422}{}{}You may have
  thought of this one: The MAC address filter table could be totally
  ­confused about the source device's location because the switch can
  receive the frame from more than one link. Worse, the bewildered
  switch could get so caught up in ­constantly updating the MAC filter
  table with source hardware address locations that it will fail to
  forward a frame! This is called thrashing the MAC table.
\item
  One of the most vile events is when multiple loops propagate
  throughout a network. Loops can occur within other loops, and if a
  broadcast storm were to occur simultaneously, the network wouldn't be
  able to perform frame switching---period!
\end{enumerate}

All of these problems spell disaster or close, and they're all evil
situations that must be avoided or fixed somehow. That's where the
Spanning Tree Protocol comes into play. It was actually developed to
solve each and every one of the problems I just told you about!

Now that I explained the issues that can occur when you have redundant
links, or when you have links that are improperly implemented, I'm sure
you understand how vital it is to prevent them. However, the best
solutions are beyond the scope of this chapter and among the territory
covered in the more advanced Cisco exam objectives. For now, let's focus
on configuring some switching!

\subsection[Configuring Catalyst
Switches]{\texorpdfstring{\protect\hypertarget{c10.xhtmlux5cux23c10-sec-4}{}{}Configuring
Catalyst Switches}{Configuring Catalyst Switches}}

Cisco Catalyst switches come in many flavors; some run 10 Mbps, while
others can speed all the way up to 10 Gbps or higher switched ports with
a combination of twisted-pair and fiber. These newer switches, like the
3850, also have more intelligence, so they can give you data
fast---mixed media services, too!

With that in mind, it's time to show you how to start up and configure a
Cisco Catalyst switch using the command-line interface (CLI). After you
get the basic commands down in this chapter, I'll show you how to
configure virtual LANs (VLANs) plus Inter-Switch Link (ISL) and 802.1q
trunking in the next one.

Here's a list of the basic tasks we'll be covering next:

\begin{enumerate}
\tightlist
\item
  Administrative functions
\item
  Configuring the IP address and subnet mask
\item
  Setting the IP default gateway
\item
  Setting port security
\item
  Testing and verifying the network
\end{enumerate}

\begin{center}\rule{0.5\linewidth}{0.5pt}\end{center}

\includegraphics{images/note.png} You can learn all about the Cisco
family of Catalyst switches at
\texttt{www.cisco.com/en/US/products/hw/switches/index.html}.

\begin{center}\rule{0.5\linewidth}{0.5pt}\end{center}

\subsubsection[Catalyst Switch
Configuration]{\texorpdfstring{\protect\hypertarget{c10.xhtmlux5cux23c10-sec-5}{}{}\protect\hypertarget{c10.xhtmlux5cux23Page_423}{}{}Catalyst
Switch Configuration}{Catalyst Switch Configuration}}

But before we actually get into configuring one of the Catalyst
switches, I've got to fill you in regarding the boot process of these
switches, just as I did with the routers in Chapter 7, ``Managing a
Cisco Internetwork.''
\protect\hyperlink{c10.xhtmlux5cux23figure10-9}{Figure 10.9} shows a
typical Cisco Catalyst switch, and I need to tell you about the
different interfaces and features of this device.

\begin{figure}
\centering
\includegraphics{images/c10f009.jpg}
\caption{{\protect\hyperlink{c10.xhtmlux5cux23figureanchor10-9}{\textbf{FIGURE
10.9}} A Cisco Catalyst switch}}
\end{figure}

The first thing I want to point out is that the console port for the
Catalyst switches are typically located on the back of the switch. Yet,
on a smaller switch like the 3560 shown in the figure, the console is
right in the front to make it easier to use. (The eight-port 2960 looks
exactly the same.) If the POST completes successfully, the system LED
turns green, but if the POST fails, it will turn amber. And seeing that
amber glow is an ominous thing---typically fatal. So you may just want
to keep a spare switch around---especially in case it's a production
switch that's croaked! The bottom button is used to show you which
lights are providing Power over Ethernet (PoE). You can see this by
pressing the Mode button. The PoE is a very nice feature of these
switches. It allows me to power my access point and phone by just
connecting them into the switch with an Ethernet cable---sweet.

Just as we did with the routers we configured in Chapter 9, ``IP
Routing,'' we'll use a diagram and switch setup in this chapter as well
as in Chapter 11.
\protect\hyperlink{c10.xhtmlux5cux23figure10-10}{Figure 10.10} shows the
switched network we'll be working on.

\begin{figure}
\centering
\includegraphics{images/c10f010.jpg}
\caption{{\protect\hyperlink{c10.xhtmlux5cux23figureanchor10-10}{\textbf{FIGURE
10.10}} Our switched network}}
\end{figure}

\protect\hypertarget{c10.xhtmlux5cux23Page_424}{}{}I'm going to use
three 3560 switches, which I also used for demonstration in Chapter 6,
``Cisco's Internetworking Operating System (IOS),'' and Chapter 7. You
can use any layer 2 switches for this chapter to follow the
configuration, but when we get to Chapter 11, you'll need at least one
router as well as a layer 3 switch, like my 3560.

Now if we connect our switches to each other, as shown in
\protect\hyperlink{c10.xhtmlux5cux23figure10-10}{Figure 10.10}, remember
that first we'll need a crossover cable between the switches. My 3560
switches autodetect the connection type, so I was able to use
straight-through cables. But not all switches autodetect the cable type.
Different switches have different needs and abilities, so just keep this
in mind when connecting your various switches together. Make a note that
in the Cisco exam objectives, switches never autodetect!

When you first connect the switch ports to each other, the link lights
are amber and then turn green, indicating normal operation. What you're
actually watching is spanning-tree converging, and this process takes
around 50 seconds with no extensions enabled. But if you connect into a
switch port and the switch port LED is alternating green and amber, it
means the port is experiencing errors. If this happens, check the host
NIC or the cabling, possibly even the duplex settings on the port to
make sure they match the host setting.

\paragraph{Do We Need to Put an IP Address on a Switch?}

Absolutely not! Switches have all ports enabled and ready to rock. Take
the switch out of the box, plug it in, and the switch starts learning
MAC addresses in the CAM. So why would I need an IP address since
switches are providing layer 2 services? Because you still need it for
in-band management purposes! Telnet, SSH, SNMP, etc. all need an IP
address in order to communicate with the switch through the network
(in-band). Remember, since all ports are enabled by default, you need to
shut down unused ports or assign them to an unused VLAN for security
reasons.

So where do we put this management IP address the switch needs for
management purposes? On what is predictably called the management VLAN
interface---a routed interface on every Cisco switch and called
interface VLAN 1. This management interface can be changed, and Cisco
recommends that you do change this to a different management interface
for security purposes. No worries---I'll demonstrate how to do this in
Chapter 11.

Let's configure our switches now so you can watch how I configure the
management interfaces on each switch.

\paragraph{S1}

We're going to begin our configuration by connecting into each switch
and setting the administrative functions. We'll also assign an IP
address to each switch, but as I said, doing that isn't really necessary
to make our network function. The only reason we're going to do that is
so we can manage/administer it remotely, via Telnet for example. Let's
use a simple IP scheme like 192.168.10.16/28. This mask should be
familiar to you! Check out the ­following output:

\begin{verbatim}
Switch>en
Switch#config t
Switch(config)#hostname S1
S1(config)#enable secret todd
S1(config)#int f0/15
S1(config-if)#description 1st connection to S3
S1(config-if)#int f0/16
S1(config-if)#description 2nd connection to S3
S1(config-if)#int f0/17
S1(config-if)#description 1st connection to S2
S1(config-if)#int f0/18
S1(config-if)#description 2nd connection to S2
S1(config-if)#int f0/8
S1(config-if)#desc Connection to IVR
S1(config-if)#line con 0
S1(config-line)#password console
S1(config-line)#login
S1(config-line)#line vty 0 15
S1(config-line)#password telnet
S1(config-line)#login
S1(config-line)#int vlan 1
S1(config-if)#ip address 192.168.10.17 255.255.255.240
S1(config-if)#no shut
S1(config-if)#exit
S1(config)#banner motd #this is my S1 switch#
S1(config)#exit
S1#copy run start
Destination filename [startup-config]? [enter]
Building configuration...
[OK]
S1#
\end{verbatim}

The first thing to notice about this is that there's no IP address
configured on the switch's physical interfaces. Since all ports on a
switch are enabled by default, there's not really a whole lot to
configure! The IP address is configured under a logical interface,
called a management domain or VLAN. You can use the default VLAN 1 to
manage a switched network just as we're doing here, or you can opt to
use a different VLAN for management.

The rest of the configuration is basically the same as the process you
go through for router configuration. So remember\ldots{} no IP addresses
on physical switch interfaces, no routing protocols, and so on. We're
performing layer 2 switching at this point, not routing! Also, make a
note to self that there is no AUX port on Cisco switches.

\paragraph[S2]{\texorpdfstring{\protect\hypertarget{c10.xhtmlux5cux23Page_426}{}{}S2}{S2}}

Here is the S2 configuration:

\begin{verbatim}
Switch#config t
Switch(config)#hostname S2
S2(config)#enable secret todd
S2(config)#int f0/1
S2(config-if)#desc 1st connection to S1
S2(config-if)#int f0/2
S2(config-if)#desc 2nd connection to s2
S2(config-if)#int f0/5
S2(config-if)#desc 1st connection to S3
S2(config-if)#int f0/6
S2(config-if)#desc 2nd connection to s3
S2(config-if)#line con 0
S2(config-line)#password console
S2(config-line)#login
S2(config-line)#line vty 0 15
S2(config-line)#password telnet
S2(config-line)#login
S2(config-line)#int vlan 1
S2(config-if)#ip address 192.168.10.18 255.255.255.240
S2(config)#exit
S2#copy run start
Destination filename [startup-config]?[enter]
Building configuration...
[OK]
S2#
\end{verbatim}

We should now be able to ping from S2 to S1. Let's try it:

\begin{verbatim}
S2#ping 192.168.10.17

Type escape sequence to abort.
Sending 5, 100-byte ICMP Echos to 192.168.10.17, timeout is 2 seconds:
.!!!!
Success rate is 80 percent (4/5), round-trip min/avg/max = 1/1/1 ms
S2#
\end{verbatim}

Okay---now why did I get only four pings to work instead of five? The
first period {[}.{]} is a time-out, but the exclamation point {[}!{]} is
a success.

\protect\hypertarget{c10.xhtmlux5cux23Page_427}{}{}It's a good question,
and here's your answer: the first ping didn't work because of the time
that ARP takes to resolve the IP address to its corresponding hardware
MAC address.

\paragraph{S3}

Check out the S3 switch configuration:

\begin{verbatim}
Switch>en
Switch#config t
SW-3(config)#hostname S3
S3(config)#enable secret todd
S3(config)#int f0/1
S3(config-if)#desc 1st connection to S1
S3(config-if)#int f0/2
S3(config-if)#desc 2nd connection to S1
S3(config-if)#int f0/5
S3(config-if)#desc 1st connection to S2
S3(config-if)#int f0/6
S3(config-if)#desc 2nd connection to S2
S3(config-if)#line con 0
S3(config-line)#password console
S3(config-line)#login
S3(config-line)#line vty 0 15
S3(config-line)#password telnet
S3(config-line)#login
S3(config-line)#int vlan 1
S3(config-if)#ip address 192.168.10.19 255.255.255.240
S3(config-if)#no shut
S3(config-if)#banner motd #This is the S3 switch#
S3(config)#exit
S3#copy run start
Destination filename [startup-config]?[enter]
Building configuration...
[OK]
S3#
\end{verbatim}

Now let's ping to S1 and S2 from the S3 switch and see what happens:

\begin{verbatim}
S3#ping 192.168.10.17
Type escape sequence to abort.
Sending 5, 100-byte ICMP Echos to 192.168.10.17, timeout is 2 seconds:
.!!!!
Success rate is 80 percent (4/5), round-trip min/avg/max = 1/3/9 ms
S3#ping 192.168.10.18
Type escape sequence to abort.
Sending 5, 100-byte ICMP Echos to 192.168.10.18, timeout is 2 seconds:
.!!!!
Success rate is 80 percent (4/5), round-trip min/avg/max = 1/3/9 ms
S3#sh ip arp
Protocol  Address          Age (min)  Hardware Addr   Type   Interface
Internet  192.168.10.17           0   001c.575e.c8c0  ARPA   Vlan1
Internet  192.168.10.18           0   b414.89d9.18c0  ARPA   Vlan1
Internet  192.168.10.19           -   ecc8.8202.82c0  ARPA   Vlan1
S3#
\end{verbatim}

In the output of the \texttt{show\ ip\ arp} command, the dash
(\texttt{-}) in the minutes column means that it is the physical
interface of the device.

Now, before we move on to verifying the switch configurations, there's
one more command you need to know about, even though we don't really
need it in our current network because we don't have a router involved.
It's the \texttt{ip\ default-gateway} command. If you want to manage
your switches from outside your LAN, you must set a default gateway on
the switches just as you would with a host, and you do this from global
config. Here's an example where we introduce our router with an IP
address using the last IP address in our subnet range:

\begin{verbatim}
S3#config t
S3(config)#ip default-gateway 192.168.10.30
\end{verbatim}

Now that we have all three switches basically configured, let's have
some fun with them!

\paragraph{Port Security}

A secured switch port can associate anywhere from 1 to 8,192 MAC
addresses, but the 3560s I am using can support only 6,144, which seems
like way more than enough to me. You can choose to allow the switch to
learn these values dynamically, or you can set static addresses for each
port using the \texttt{switchport\ port-security\ mac-address}
\texttt{mac-address} command.

So let's set port security on our S3 switch now. Ports Fa0/3 and Fa0/4
will have only one device connected in our lab. By using port security,
we're assured that no other device can connect once our hosts in ports
Fa0/3 and in Fa0/4 are connected. Here's how to easily do that with just
a couple commands:

\begin{verbatim}
S3#config t
S3(config)#int range f0/3-4
S3(config-if-range)#switchport mode access
S3(config-if-range)#switchport port-security
S3(config-if-range)#do show port-security int f0/3
Port Security              : Enabled
Port Status                : Secure-down
Violation Mode             : Shutdown
Aging Time                 : 0 mins
Aging Type                 : Absolute
SecureStatic Address Aging : Disabled
Maximum MAC Addresses      : 1
Total MAC Addresses        : 0
Configured MAC Addresses   : 0
Sticky MAC Addresses       : 0
Last Source Address:Vlan   : 0000.0000.0000:0
Security Violation Count   : 0
\end{verbatim}

The first command sets the mode of the ports to ``access'' ports. These
ports must be access or trunk ports to enable port security. By using
the command \texttt{switchport\ ­port-security} on the interface, I've
enabled port security with a maximum MAC address of 1 and violation of
shutdown. These are the defaults, and you can see them in the
highlighted output of the \texttt{show\ port-security\ int\ f0/3}
command in the preceding code.

Port security is enabled, as displayed on the first line, but the second
line shows \texttt{Secure-down} because I haven't connected my hosts
into the ports yet. Once I do, the status will show \texttt{Secure-up}
and would become \texttt{Secure-shutdown} if a violation occurs.

I've just got to point out this all-so-important fact one more time:
It's very important to remember that you can set parameters for port
security but it won't work until you enable port security at the
interface level. Notice the output for port F0/6:

\begin{verbatim}
S3#config t
S3(config)#int range f0/6
S3(config-if-range)#switchport mode access
S3(config-if-range)#switchport port-security violation restrict
S3(config-if-range)#do show port-security int f0/6
Port Security              : Disabled
Port Status                : Secure-up
Violation Mode             : restrict
[output cut]
\end{verbatim}

Port Fa0/6 has been configured with a violation of restrict, but the
first line shows that port security has not been enabled on the port
yet. Remember, you must use this command at interface level to enable
port security on a port:

\begin{verbatim}
S3(config-if-range)#switchport port-security
\end{verbatim}

There are two other modes you can use instead of just shutting down the
port. The restrict and protect modes mean that another host can connect
up to the maximum MAC addresses allowed, but after the maximum has been
met, all frames will just be dropped
\protect\hypertarget{c10.xhtmlux5cux23Page_430}{}{}and the port won't be
shut down. Additionally, both the restrict and shutdown violation modes
alert you via SNMP that a violation has occurred on a port. You can then
call the abuser and tell them they're so busted---you can see them, you
know what they did, and they're in serious trouble!

If you've configured ports with the \texttt{violation\ shutdown}
command, then the ports will look like this when a violation occurs:

\begin{verbatim}
S3#sh port-security int f0/3
Port Security              : Enabled
Port Status                : Secure-shutdown
Violation Mode             : Shutdown
Aging Time                 : 0 mins
Aging Type                 : Absolute
SecureStatic Address Aging : Disabled
Maximum MAC Addresses      : 1
Total MAC Addresses        : 2
Configured MAC Addresses   : 0
Sticky MAC Addresses       : 0
Last Source Address:Vlan   : 0013:0ca69:00bb3:00ba8:1
Security Violation Count   : 1
\end{verbatim}

Here you can see that the port is in \texttt{Secure-shutdown} mode and
the light for the port would be amber. To enable the port again, you'd
need to do the following:

\begin{verbatim}
S3(config-if)#shutdown
S3(config-if)#no shutdown
\end{verbatim}

Let's verify our switch configurations before we move onto VLANs in the
next chapter. Beware that even though some switches will show
\texttt{err-disabled} instead of \texttt{Secure-shutdown} as my switch
shows, there is no difference between the two.

\subsubsection[Verifying Cisco Catalyst
Switches]{\texorpdfstring{\protect\hypertarget{c10.xhtmlux5cux23c10-sec-6}{}{}Verifying
Cisco Catalyst Switches}{Verifying Cisco Catalyst Switches}}

The first thing I like to do with any router or switch is to run through
the configurations with a \texttt{show\ running-config} command. Why?
Because doing this gives me a really great overview of each device. But
it is time consuming, and showing you all the configs would take up way
too many pages in this book. Besides, we can instead run other commands
that will still stock us up with really good information.

For example, to verify the IP address set on a switch, we can use the
\texttt{show\ interface} command. Here's the output:

\begin{verbatim}
S3#sh int vlan 1
Vlan1 is up, line protocol is up
  Hardware is EtherSVI, address is ecc8.8202.82c0 (bia ecc8.8202.82c0)
  Internet address is 192.168.10.19/28
  MTU 1500 bytes, BW 1000000 Kbit/sec, DLY 10 usec,
     reliability 255/255, txload 1/255, rxload 1/255
  Encapsulation ARPA, loopback not set
  [output cut]
\end{verbatim}

The previous output shows the interface is in up/up status. Remember to
always check this interface, either with this command or the
\texttt{show\ ip\ interface\ brief} command. Lots of people tend to
forget that this interface is \texttt{shutdown} by default.

\begin{center}\rule{0.5\linewidth}{0.5pt}\end{center}

\includegraphics{images/note.png} Never forget that IP addresses aren't
needed on a switch for it to operate. The only reason we would set an IP
address, mask, and default gateway is for management purposes.

\begin{center}\rule{0.5\linewidth}{0.5pt}\end{center}

\paragraph{\texorpdfstring{\emph{show mac
address-table}}{show mac address-table}}

I'm sure you remember being shown this command earlier in the chapter.
Using it displays the forward filter table, also called a content
addressable memory (CAM) table. Here's the output from the S1 switch:

\begin{verbatim}
S3#sh mac address-table
          Mac Address Table
-------------------------------------------
Vlan    Mac Address       Type        Ports
----    -----------       --------    -----
 All    0100.0ccc.cccc    STATIC      CPU
[output cut]
   1    000e.83b2.e34b    DYNAMIC     Fa0/1
   1    0011.1191.556f    DYNAMIC     Fa0/1
   1    0011.3206.25cb    DYNAMIC     Fa0/1
   1    001a.2f55.c9e8    DYNAMIC     Fa0/1
   1    001a.4d55.2f7e    DYNAMIC     Fa0/1
   1    001c.575e.c891    DYNAMIC     Fa0/1
   1    b414.89d9.1886    DYNAMIC     Fa0/5
   1    b414.89d9.1887    DYNAMIC     Fa0/6
\end{verbatim}

The switches use things called base MAC addresses, which are assigned to
the CPU. The first one listed is the base mac address of the switch.
From the preceding output, you can see that we have six MAC addresses
dynamically assigned to Fa0/1, meaning that port Fa0/1 is connected to
another switch. Ports Fa0/5 and Fa0/6 only have one MAC address
assigned, and all ports are assigned to VLAN 1.

Let's take a look at the S2 switch CAM and see what we can find out.

\begin{verbatim}
S2#sh mac address-table
          Mac Address Table
-------------------------------------------
Vlan    Mac Address       Type        Ports
----    -----------       --------    -----
 All    0100.0ccc.cccc    STATIC      CPU
[output cut
   1    000e.83b2.e34b    DYNAMIC     Fa0/5
   1    0011.1191.556f    DYNAMIC     Fa0/5
   1    0011.3206.25cb    DYNAMIC     Fa0/5
   1    001a.4d55.2f7e    DYNAMIC     Fa0/5
   1    581f.aaff.86b8    DYNAMIC     Fa0/5
   1    ecc8.8202.8286    DYNAMIC     Fa0/5
   1    ecc8.8202.82c0    DYNAMIC     Fa0/5
Total Mac Addresses for this criterion: 27
S2#
\end{verbatim}

This output tells us that we have seven MAC addresses assigned to Fa0/5,
which is our connection to S3. But where's port 6? Since port 6 is a
redundant link to S3, STP placed Fa0/6 into blocking mode.

\subparagraph{Assigning Static MAC Addresses}

You can set a static MAC address in the MAC address table, but like
setting static MAC port security without the \texttt{sticky} command,
it's a ton of work. Just in case you want to do it, here's how it's
done:

\begin{verbatim}
S3(config)#mac address-table ?
  aging-time    Set MAC address table entry maximum age
  learning      Enable MAC table learning feature
  move          Move keyword
  notification  Enable/Disable MAC Notification on the switch
  static        static keyword

S3(config)#mac address-table static aaaa.bbbb.cccc vlan 1 int fa0/7
S3(config)#do show mac address-table
          Mac Address Table
-------------------------------------------
Vlan    Mac Address       Type        Ports
----    -----------       --------    -----
 All    0100.0ccc.cccc    STATIC      CPU
[output cut]
   1    000e.83b2.e34b    DYNAMIC     Fa0/1
   1    0011.1191.556f    DYNAMIC     Fa0/1
   1    0011.3206.25cb    DYNAMIC     Fa0/1
   1    001a.4d55.2f7e    DYNAMIC     Fa0/1
   1    001b.d40a.0538    DYNAMIC     Fa0/1
   1    001c.575e.c891    DYNAMIC     Fa0/1
   1    aaaa.bbbb.0ccc    STATIC      Fa0/7
[output cut]
Total Mac Addresses for this criterion: 59
\end{verbatim}

As shown on the left side of the output, you can see that a static MAC
address has now been assigned permanently to interface Fa0/7 and that
it's also been assigned to VLAN 1 only.

Now admit it---this chapter had a lot of great information, and you
really did learn a lot and, well, maybe even had a little fun along the
way too! You've now configured and verified all switches and set port
security. That means you're now ready to learn all about virtual LANs!
I'm going to save all our switch configurations so we'll be able to
start right from here in Chapter 11.

\subsection[Summary]{\texorpdfstring{\protect\hypertarget{c10.xhtmlux5cux23c10-sec-7}{}{}Summary}{Summary}}

In this chapter, I talked about the differences between switches and
bridges and how they both work at layer 2. They create MAC address
forward/filter tables in order to make decisions on whether to forward
or flood a frame.

Although everything in this chapter is important, I wrote two
port-security sections---one to provide a foundation and one with a
configuration example. You must know both these sections in detail.

I also covered some problems that can occur if you have multiple links
between bridges (switches).

Finally, I covered detailed configuration of Cisco's Catalyst switches,
including verifying the configuration.

\subsection[Exam
Essentials]{\texorpdfstring{\protect\hypertarget{c10.xhtmlux5cux23c10-sec-8}{}{}Exam
Essentials}{Exam Essentials}}

\textbf{Remember the three switch functions.} Address learning,
forward/filter decisions, and loop avoidance are the functions of a
switch.

\textbf{Remember the command}\texttt{show\ mac\ address-table.} The
command \texttt{show\ mac\ address-table} will show you the
forward/filter table used on the LAN switch.

\textbf{Understand the reason for port security.} Port security
restricts access to a switch based on MAC addresses.

\protect\hypertarget{c10.xhtmlux5cux23Page_434}{}{}\textbf{Know the
command to enable port security.} To enable port security on a port, you
must first make sure the port is an access port with
\texttt{switchport\ mode\ access} and then use the
\texttt{switchport\ port-security} command at the interface level. You
can set the port security parameters before or after enabling port
security.

\textbf{Know the commands to verify port security.} To verify port
security, use the
\texttt{show\ port-security,\ show\ port-security\ interface\ interface},
and \texttt{show\ running-config} commands.

\subsection[Written Lab
10]{\texorpdfstring{\protect\hypertarget{c10.xhtmlux5cux23c10-sec-9}{}{}Written
Lab 10}{Written Lab 10}}

In this section, you'll complete the following lab to make sure you've
got the information and concepts contained within them fully dialed in:

\begin{quote}
Lab 10.1: Layer 2 Switching

You can find the answers to this lab in Appendix A, ``Answers to Written
Labs.''

Write the answers to the following questions:
\end{quote}

\begin{enumerate}
\tightlist
\item
  What command will show you the forward/filter table?
\item
  If a destination MAC address is not in the forward/filter table, what
  will the switch do with the frame?
\item
  What are the three switch functions at layer 2?
\item
  If a frame is received on a switch port and the source MAC address is
  not in the forward/filter table, what will the switch do?
\item
  What are the default modes for a switch port configured with port
  security?
\item
  Which two violation modes send out an SNMP trap?
\item
  Which violation mode drops packets with unknown source addresses until
  you remove enough secure MAC addresses to drop below the maximum but
  also generates a log message, causes the security violation counter to
  increment, and sends an SNMP trap but does not disable the port?
\item
  What does the \texttt{sticky} keyword in the \texttt{port-security}
  command provide?
\item
  What two commands can you use to verify that port security has been
  configured on a port FastEthernet 0/12 on a switch?
\item
  True/False: The layer 2 switch must have an IP address set and the PCs
  connecting to the switch must use that address as their default
  gateway.
\end{enumerate}

\subsection[Hands-on
Labs]{\texorpdfstring{\protect\hypertarget{c10.xhtmlux5cux23c10-sec-10}{}{}Hands-on
Labs}{Hands-on Labs}}

In this section, you will use the following switched network to
configure your switching labs. You can use any Cisco switches to do this
lab, as well as LammleSim IOS version simulator found at
\texttt{www.lammle.com/ccna}. They do not need to be multilayer
switches, just layer 2 switches.

\protect\hypertarget{c10.xhtmlux5cux23Page_435}{}{}

\begin{figure}
\centering
\includegraphics{images/c10f011.jpg}
\caption{}
\end{figure}

The first lab (Lab 10.1) requires you to configure three switches, and
then you will verify them in Lab 10.2.

The labs in this chapter are as follows:

\begin{enumerate}
\tightlist
\item
  Hands-on Lab 10.1: Configuring Layer 2 Switches
\item
  Hands-on Lab 10.2: Verifying Layer 2 Switches
\item
  Hands-on Lab 10.3: Configuring Port Security
\end{enumerate}

\subsubsection[Lab 10.1: Configuring Layer 2
Switches]{\texorpdfstring{\protect\hypertarget{c10.xhtmlux5cux23c10-sec-11}{}{}Lab
10.1: Configuring Layer 2
Switches}{Lab 10.1: Configuring Layer 2 Switches}}

In this lab, you will configure the three switches in the graphic:

\begin{enumerate}
\tightlist
\item
  Connect to the S1 switch and configure the following, not in any
  particular order:

  \begin{enumerate}
  \item
    Hostname
  \item
    Banner
  \item
    Interface description
  \item
    Passwords
  \item
    IP address, subnet mask, default gateway

\begin{verbatim}
Switch>en
Switch#config t
Switch(config)#hostname S1
S1(config)#enable secret todd
S1(config)#int f0/15
S1(config-if)#description 1st connection to S3
S1(config-if)#int f0/16
S1(config-if)#description 2nd connection to S3
S1(config-if)#int f0/17
S1(config-if)#description 1st connection to S2
S1(config-if)#int f0/18
S1(config-if)#description 2nd connection to S2
S1(config-if)#int f0/8
S1(config-if)#desc Connection to IVR
S1(config-if)#line con 0
S1(config-line)#password console
S1(config-line)#login
S1(config-line)#line vty 0 15
S1(config-line)#password telnet
S1(config-line)#login
S1(config-line)#int vlan 1
S1(config-if)#ip address 192.168.10.17 255.255.255.240
S1(config-if)#no shut
S1(config-if)#exit
S1(config)#banner motd #this is my S1 switch#
S1(config)#exit
S1#copy run start
Destination filename [startup-config]? [enter]
Building configuration...
\end{verbatim}
  \end{enumerate}
\item
  Connect to the S2 switch and configure all the settings you used in
  step 1. Do not forget to use a different IP address on the switch.
\item
  Connect to the S3 switch and configure all the settings you used in
  steps 1 and 2. Do not forget to use a different IP address on the
  switch.
\end{enumerate}

\subsubsection[Lab 10.2: Verifying Layer 2
Switches]{\texorpdfstring{\protect\hypertarget{c10.xhtmlux5cux23c10-sec-12}{}{}Lab
10.2: Verifying Layer 2 Switches}{Lab 10.2: Verifying Layer 2 Switches}}

Once you configure a device, you must be able to verify it.

\begin{enumerate}
\item
  Connect to each switch and verify the management interface.

\begin{verbatim}
S1#sh interface vlan 1
\end{verbatim}
\item
  Connect to each switch and verify the CAM.

\begin{verbatim}
S1#sh mac address-table
\end{verbatim}
\item
  Verify your configurations with the following commands:

\begin{verbatim}
S1#sh running-config
S1#sh ip int brief
\end{verbatim}
\end{enumerate}

\subsubsection[Lab 10.3: Configuring Port
Security]{\texorpdfstring{\protect\hypertarget{c10.xhtmlux5cux23c10-sec-13}{}{}\protect\hypertarget{c10.xhtmlux5cux23Page_437}{}{}Lab
10.3: Configuring Port Security}{Lab 10.3: Configuring Port Security}}

Port security is a big Cisco objective. Do not skip this lab!

\begin{enumerate}
\item
  Connect to your S3 switch.
\item
  Configure port Fa0/3 with port security.

\begin{verbatim}
S3#config t
S(config)#int fa0/3
S3(config-if#Switchport mode access
S3(config-if#switchport port-security
\end{verbatim}
\item
  Check your default setting for port security.

\begin{verbatim}
S3#show port-security int f0/3
\end{verbatim}
\item
  Change the settings to have a maximum of two MAC addresses that can
  associate to interface Fa0/3.

\begin{verbatim}
S3#config t
S(config)#int fa0/3
S3(config-if#switchport port-security maximum 2
\end{verbatim}
\item
  Change the violation mode to \texttt{restrict}.

\begin{verbatim}
S3#config t
S(config)#int fa0/3
S3(config-if#switchport port-security violation restrict
\end{verbatim}
\item
  Verify your configuration with the following commands:

\begin{verbatim}
S3#show port-security
S3#show port-security int fa0/3
S3#show running-config
\end{verbatim}
\end{enumerate}

\subsection[Review
Questions]{\texorpdfstring{\protect\hypertarget{c10.xhtmlux5cux23c10-sec-14}{}{}\protect\hypertarget{c10.xhtmlux5cux23Page_438}{}{}Review
Questions}{Review Questions}}

\begin{center}\rule{0.5\linewidth}{0.5pt}\end{center}

\includegraphics{images/note.png} The following questions are designed
to test your understanding of this chapter's material. For more
information on how to get additional questions, please see
\texttt{www.lammle.com/ccna}.

\begin{center}\rule{0.5\linewidth}{0.5pt}\end{center}

You can find the answers to these questions in Appendix B, ``Answers to
Review Questions.''

\begin{enumerate}
\item
  Which of the following statements is \emph{not} true with regard to
  layer 2 switching?

  \begin{enumerate}
  \tightlist
  \item
    Layer 2 switches and bridges are faster than routers because they
    don't take up time looking at the Data Link layer header
    information.
  \item
    Layer 2 switches and bridges look at the frame's hardware addresses
    before deciding to either forward, flood, or drop the frame.
  \item
    Switches create private, dedicated collision domains and provide
    independent bandwidth on each port.
  \item
    Switches use application-specific integrated circuits (ASICs) to
    build and maintain their MAC filter tables.
  \end{enumerate}
\item
  List the two commands that generated the last entry in the MAC address
  table shown.

\begin{verbatim}
Mac Address Table
-------------------------------------------

Vlan    Mac Address       Type        Ports
----    -----------       --------    -----
 All    0100.0ccc.cccc    STATIC      CPU
[output cut]
   1    000e.83b2.e34b    DYNAMIC     Fa0/1
   1    0011.1191.556f    DYNAMIC     Fa0/1
   1    0011.3206.25cb    DYNAMIC     Fa0/1
   1    001a.4d55.2f7e    DYNAMIC     Fa0/1
   1    001b.d40a.0538    DYNAMIC     Fa0/1
   1    001c.575e.c891    DYNAMIC     Fa0/1
   1    aaaa.bbbb.0ccc    STATIC      Fa0/7
\end{verbatim}
\item
  In the diagram shown, what will the switch do if a frame with a
  destination MAC address of 000a.f467.63b1 is received on Fa0/4?
  (Choose all that apply.)

  \protect\hypertarget{c10.xhtmlux5cux23Page_439}{}{}

  \begin{figure}
  \centering
  \includegraphics{images/c10f012.jpg}
  \caption{}
  \end{figure}

  \begin{enumerate}
  \tightlist
  \item
    Drop the frame.
  \item
    Send the frame out of Fa0/3.
  \item
    Send the frame out of Fa0/4.
  \item
    Send the frame out of Fa0/5.
  \item
    Send the frame out of Fa0/6.
  \end{enumerate}
\item
  Write the command that generated the following output.

\begin{verbatim}
          Mac Address Table
-------------------------------------------
Vlan    Mac Address       Type        Ports
----    -----------       --------    -----
 All    0100.0ccc.cccc    STATIC      CPU
[output cut]
   1    000e.83b2.e34b    DYNAMIC     Fa0/1
   1    0011.1191.556f    DYNAMIC     Fa0/1
   1    0011.3206.25cb    DYNAMIC     Fa0/1
   1    001a.2f55.c9e8    DYNAMIC     Fa0/1
   1    001a.4d55.2f7e    DYNAMIC     Fa0/1
   1    001c.575e.c891    DYNAMIC     Fa0/1
   1    b414.89d9.1886    DYNAMIC     Fa0/5
   1    b414.89d9.1887    DYNAMIC     Fa0/6
\end{verbatim}
\item
  In the work area in the following graphic, draw the functions of a
  switch from the list on the left to the right.

  \begin{figure}
  \centering
  \includegraphics{images/c10f013.jpg}
  \caption{}
  \end{figure}
\item
  \protect\hypertarget{c10.xhtmlux5cux23Page_440}{}{}What statement(s)
  is/are true about the output shown here? (Choose all that apply.)

\begin{verbatim}
S3#sh port-security int f0/3
Port Security              : Enabled
Port Status                : Secure-shutdown
Violation Mode             : Shutdown
Aging Time                 : 0 mins
Aging Type                 : Absolute
SecureStatic Address Aging : Disabled
Maximum MAC Addresses      : 1
Total MAC Addresses        : 2
Configured MAC Addresses   : 0
Sticky MAC Addresses       : 0
Last Source Address:Vlan   : 0013:0ca69:00bb3:00ba8:1
Security Violation Count   : 1
\end{verbatim}

  \begin{enumerate}
  \tightlist
  \item
    The port light for F0/3 will be amber in color.
  \item
    The F0/3 port is forwarding frames.
  \item
    This problem will resolve itself in a few minutes.
  \item
    This port requires the \texttt{shutdown} command to function.
  \end{enumerate}
\item
  Write the command that would limit the number of MAC addresses allowed
  on a port to 2. Write only the command and not the prompt.
\item
  Which of the following commands in this configuration is a
  prerequisite for the other commands to function?

\begin{verbatim}
S3#config t
S(config)#int fa0/3
S3(config-if#switchport port-security
S3(config-if#switchport port-security maximum 3
S3(config-if#switchport port-security violation restrict
S3(config-if#Switchport mode-security aging time 10
\end{verbatim}

  \begin{enumerate}
  \tightlist
  \item
    \texttt{switchport\ mode-security\ aging\ time\ 10}
  \item
    \texttt{switchport\ port-security}
  \item
    \texttt{switchport\ port-security\ maximum\ 3}
  \item
    \texttt{switchport\ port-security\ violation\ restrict}
  \end{enumerate}
\item
  Which if the following is \emph{not} an issue addressed by STP?

  \begin{enumerate}
  \tightlist
  \item
    Broadcast storms
  \item
    Gateway redundancy
  \item
    \protect\hypertarget{c10.xhtmlux5cux23Page_441}{}{}A device
    receiving multiple copies of the same frame
  \item
    Constant updating of the MAC filter table
  \end{enumerate}
\item
  What issue that arises when redundancy exists between switches is
  shown in the figure?

  \begin{figure}
  \centering
  \includegraphics{images/c10f014.jpg}
  \caption{}
  \end{figure}

  \begin{enumerate}
  \tightlist
  \item
    Broadcast storm
  \item
    Routing loop
  \item
    Port violation
  \item
    Loss of gateway
  \end{enumerate}
\item
  Which two of the following switch port violation modes will alert you
  via SNMP that a violation has occurred on a port?

  \begin{enumerate}
  \tightlist
  \item
    \texttt{restrict}
  \item
    \texttt{protect}
  \item
    \texttt{shutdown}
  \item
    \texttt{err-disable}
  \end{enumerate}
\item
  \_\_\_\_\_\_\_\_\_\_\_is the loop avoidance mechanism used by
  switches.
\item
  Write the command that must be present on any switch that you need to
  manage from a different subnet.
\item
  On which default interface have you configured an IP address for a
  switch?

  \begin{enumerate}
  \tightlist
  \item
    \texttt{int\ fa0/0}
  \item
    \texttt{int\ vty\ 0\ 15}
  \item
    \texttt{int\ vlan\ 1}
  \item
    \texttt{int\ s/0/0}
  \end{enumerate}
\item
  Which Cisco IOS command is used to verify the port security
  configuration of a switch port?

  \begin{enumerate}
  \tightlist
  \item
    \texttt{show\ interfaces\ port-security}
  \item
    \texttt{show\ port-security\ interface}
  \item
    \texttt{show\ ip\ interface}
  \item
    \texttt{show\ interfaces\ switchport}
  \end{enumerate}
\item
  Write the command that will save a dynamically learned MAC address in
  the running-configuration of a Cisco switch?
\item
  Which of the following methods will ensure that only one specific host
  can connect to port F0/3 on a switch? (Choose two. Each correct answer
  is a separate solution.)

  \begin{enumerate}
  \tightlist
  \item
    Configure port security on F0/3 to accept traffic other than that of
    the MAC address of the host.
  \item
    Configure the MAC address of the host as a static entry associated
    with port F0/3.
  \item
    Configure an inbound access control list on port F0/3 limiting
    traffic to the IP address of the host.
  \item
    Configure port security on F0/3 to accept traffic only from the MAC
    address of the host.
  \end{enumerate}
\item
  What will be the effect of executing the following command on port
  F0/1?

\begin{verbatim}
switch(config-if)# switchport port-security mac-address 00C0.35F0.8301
\end{verbatim}

  \begin{enumerate}
  \tightlist
  \item
    The command configures an inbound access control list on port F0/1,
    limiting traffic to the IP address of the host.
  \item
    The command expressly prohibits the MAC address of 00c0.35F0.8301 as
    an allowed host on the switch port.
  \item
    The command encrypts all traffic on the port from the MAC address of
    00c0.35F0.8301.
  \item
    The command statically defines the MAC address of 00c0.35F0.8301 as
    an allowed host on the switch port.
  \end{enumerate}
\item
  The conference room has a switch port available for use by the
  presenter during classes, and each presenter uses the same PC attached
  to the port. You would like to prevent other PCs from using that port.
  You have completely removed the former configuration in order to start
  anew. Which of the following steps is \emph{not} required to prevent
  any other PCs from using that port?

  \begin{enumerate}
  \tightlist
  \item
    Enable port security.
  \item
    Assign the MAC address of the PC to the port.
  \item
    Make the port an access port.
  \item
    Make the port a trunk port.
  \end{enumerate}
\item
  Write the command required to disable the port if a security violation
  occurs. Write only the command and not the prompt.
\end{enumerate}

\protect\hypertarget{c11.xhtml}{}{}

\section[{Chapter 11}\\
{VLANs and Inter-VLAN
Routing}]{\texorpdfstring{\protect\hypertarget{c11.xhtmlux5cux23c11}{}{}\protect\hypertarget{c11.xhtmlux5cux23Page_443}{}{}{Chapter
11}\\
{VLANs and Inter-VLAN
Routing}}{Chapter 11 VLANs and Inter-VLAN Routing}}

\begin{center}\rule{0.5\linewidth}{0.5pt}\end{center}

\subsection{THE FOLLOWING ICND1 EXAM TOPICS ARE COVERED IN THIS
CHAPTER:}

\begin{enumerate}
\tightlist
\item
  \includegraphics{images/tick.png} \textbf{2.0 LAN Switching
  Technologies}

  \begin{enumerate}
  \tightlist
  \item
    \includegraphics{images/square.png} 2.4 Configure, verify, and
    troubleshoot VLANs (normal range) spanning multiple switches

    \begin{enumerate}
    \tightlist
    \item
      \includegraphics{images/square.png} 2.4.a Access ports (data and
      voice)
    \item
      \includegraphics{images/square.png} 2.4.b Default VLAN
    \end{enumerate}
  \item
    \includegraphics{images/square.png} 2.5 Configure, verify, and
    troubleshoot interswitch connectivity

    \begin{enumerate}
    \tightlist
    \item
      \includegraphics{images/square.png} 2.5.a Trunk ports
    \item
      \includegraphics{images/square.png} 2.5.b 802.1Q
    \item
      \includegraphics{images/square.png} 2.5.c Native VLAN
    \end{enumerate}
  \end{enumerate}
\item
  \includegraphics{images/tick.png} 3.0 Routing Technologies

  \begin{enumerate}
  \tightlist
  \item
    \includegraphics{images/square.png} 3.4 Configure, verify, and
    troubleshoot inter-VLAN routing

    \begin{enumerate}
    \tightlist
    \item
      \includegraphics{images/square.png} 3.4.a Router on a stick
    \end{enumerate}
  \end{enumerate}
\end{enumerate}

\protect\hypertarget{c11.xhtmlux5cux23Page_444}{}{}\includegraphics{images/intro.png}I
know I keep telling you this, but so you never forget it, here I go, one
last time: By default, switches break up collision domains and routers
break up broadcast domains. Okay, I feel better! Now we can move on.

In contrast to the networks of yesterday that were based on collapsed
backbones, today's network design is characterized by a flatter
architecture---thanks to switches. So now what? How do we break up
broadcast domains in a pure switched internetwork? By creating virtual
local area networks (VLANs). A VLAN is a logical grouping of network
users and resources connected to administratively defined ports on a
switch. When you create VLANs, you'regiven the ability to create smaller
broadcast domains within a layer 2 switched internetwork by assigning
different ports on the switch to service different subnetworks. A VLAN
is treated like its own subnet or broadcast domain, meaning that frames
broadcast onto the network are only switched between the ports logically
grouped within the same VLAN.

So, does this mean we no longer need routers? Maybe yes; maybe no. It
really depends on what your particular networking needs and goals are.
By default, hosts in a specific VLAN can't communicate with hosts that
are members of another VLAN, so if you want inter-VLAN communication,
the answer is that you still need a router or Inter-VLAN Routing (IVR).

In this chapter, you're going to comprehensively learn exactly what a
VLAN is and how VLAN memberships are used in a switched network. You'll
also become well-versed in what a trunk link is and how to configure and
verify them.

I'll finish this chapter by demonstrating how you can make inter-VLAN
communication happen by introducing a router into a switched network. Of
course, we'll configure our familiar switched network layout we used in
the last chapter for creating VLANs and for implementing trunking and
Inter-VLAN routing on a layer 3 switch by creating switched virtual
interfaces (SVIs).

\begin{center}\rule{0.5\linewidth}{0.5pt}\end{center}

\includegraphics{images/note.png}To find up-to-the-minute updates for
this chapter, please see \texttt{www.lammle.com/ccna} or the book's web
page at \texttt{www.sybex.com/go/ccna}.

\begin{center}\rule{0.5\linewidth}{0.5pt}\end{center}

\subsection[VLAN
Basics]{\texorpdfstring{\protect\hypertarget{c11.xhtmlux5cux23c11-sec-1}{}{}VLAN
Basics}{VLAN Basics}}

\protect\hyperlink{c11.xhtmlux5cux23figure11-1}{Figure 11.1} illustrates
the flat network architecture that used to be so typical for layer 2
switched networks. With this configuration, every broadcast packet
transmitted is seen by every device on the network regardless of whether
the device needs to receive that data or not.

\protect\hypertarget{c11.xhtmlux5cux23Page_445}{}{}

\begin{figure}
\centering
\includegraphics{images/c11f001.jpg}
\caption{{\protect\hyperlink{c11.xhtmlux5cux23figureanchor11-1}{\textbf{FIGURE
11.1}} Flat network structure}}
\end{figure}

By default, routers allow broadcasts to occur only within the
originating network, while switches forward broadcasts to all segments.
Oh, and by the way, the reason it's called a\emph{flat network} is
because it's one \emph{broadcast domain}, not because the actual design
is physically flat. In
\protect\hyperlink{c11.xhtmlux5cux23figure11-1}{Figure 11.1} we see Host
A sending out a broadcast and all ports on all switches forwarding
it---all except the port that originally received it.

Now check out \protect\hyperlink{c11.xhtmlux5cux23figure11-2}{Figure
11.2}. It pictures a switched network and shows Host A sending a frame
with Host D as its destination. Clearly, the important factor here is
that the frame is only forwarded out the port where Host D is located.

\begin{figure}
\centering
\includegraphics{images/c11f002.jpg}
\caption{{\protect\hyperlink{c11.xhtmlux5cux23figureanchor11-2}{\textbf{FIGURE
11.2}} The benefit of a switched network}}
\end{figure}

This is a huge improvement over the old hub networks, unless having one
\emph{collision domain} by default is what you really want for some
reason!

Okay---you already know that the biggest benefit gained by having a
layer 2 switched network is that it creates individual collision domain
segments for each device plugged into each port on the switch. This
scenario frees us from the old Ethernet density constraints and makes us
able to build larger networks. But too often, each new advance comes
with new issues. For instance, the more users and devices that populate
and use a network, the more broadcasts and packets each switch must
handle.

\protect\hypertarget{c11.xhtmlux5cux23Page_446}{}{}And there's another
big issue---security! This one is real trouble because within the
typical layer 2 switched internetwork, all users can see all devices by
default. And you can't stop devices from broadcasting, plus you can't
stop users from trying to respond to broadcasts. This means your
security options are dismally limited to placing passwords on your
servers and other devices.

But wait---there's hope if you create a \emph{virtual LAN (VLAN)}! You
can solve many of the problems associated with layer 2 switching with
VLANs, as you'll soon see.

VLANs work like this:
\protect\hyperlink{c11.xhtmlux5cux23figure11-3}{Figure 11.3} shows all
hosts in this very small company connected to one switch, meaning all
hosts will receive all frames, which is the default behavior of all
switches.

\begin{figure}
\centering
\includegraphics{images/c11f003.jpg}
\caption{{\protect\hyperlink{c11.xhtmlux5cux23figureanchor11-3}{\textbf{FIGURE
11.3}} One switch, one LAN: Before VLANs, there were no separations
between hosts.}}
\end{figure}

If we want to separate the host's data, we could either buy another
switch or create virtual LANs, as shown in
\protect\hyperlink{c11.xhtmlux5cux23figure11-4}{Figure 11.4}.

\begin{figure}
\centering
\includegraphics{images/c11f004.jpg}
\caption{{\protect\hyperlink{c11.xhtmlux5cux23figureanchor11-4}{\textbf{FIGURE
11.4}} One switch, two virtual LANs (\emph{logical} separation between
hosts): Still physically one switch, but this switch acts as many
separate devices.}}
\end{figure}

In \protect\hyperlink{c11.xhtmlux5cux23figure11-4}{Figure 11.4}, I
configured the switch to be two separate LANs, two subnets, two
broadcast domains, two VLANs---they all mean the same thing---without
buying another switch. We can do this 1,000 times on most Cisco
switches, which saves thousands of dollars and more!

\protect\hypertarget{c11.xhtmlux5cux23Page_447}{}{}Notice that even
though the separation is virtual and the hosts are all still connected
to the same switch, the LANs can't send data to each other by default.
This is because they are still separate networks, but no worries---we'll
get into inter-VLAN communication later in this chapter.

Here's a short list of ways VLANs simplify network management:

\begin{enumerate}
\tightlist
\item
  Network adds, moves, and changes are achieved with ease by just
  configuring a port into the appropriate VLAN.
\item
  A group of users that need an unusually high level of security can be
  put into its own VLAN so that users outside of that VLAN can't
  communicate with the group's users.
\item
  As a logical grouping of users by function, VLANs can be considered
  independent from their physical or geographic locations.
\item
  VLANs greatly enhance network security if implemented correctly.
\item
  VLANs increase the number of broadcast domains while decreasing their
  size.
\end{enumerate}

Coming up, we'll thoroughly explore the world of switching, and you
learn exactly how and why switches provide us with much better network
services than hubs can in our networks today.

\subsubsection[Broadcast
Control]{\texorpdfstring{\protect\hypertarget{c11.xhtmlux5cux23c11-sec-2}{}{}Broadcast
Control}{Broadcast Control}}

Broadcasts occur in every protocol, but how often they occur depends
upon three things:

\begin{enumerate}
\tightlist
\item
  The type of protocol
\item
  The application(s) running on the internetwork
\item
  How these services are used
\end{enumerate}

Some older applications have been rewritten to reduce their bandwidth
consumption, but there's a new generation of applications that are so
bandwidth greedy they'll consume any and all they can find. These
gluttons are the legion of multimedia applications that use both
broadcasts and multicasts extensively. As if they weren't enough
trouble, factors like faulty equipment, inadequate segmentation, and
poorly designed firewalls can seriously compound the problems already
caused by these broadcast-intensive applications. All of this has added
a major new dimension to network design and presents a bunch of new
challenges for an administrator. Positively making sure your network is
properly segmented so you can quickly isolate a single segment's
problems to prevent them from propagating throughout your entire
internetwork is now imperative. And the most effective way to do that is
through strategic switching and routing!

Since switches have become more affordable, most everyone has replaced
their flat hub networks with pure switched network and VLAN
environments. All devices within a VLAN are members of the same
broadcast domain and receive all broadcasts relevant to it. By default,
these broadcasts are filtered from all ports on a switch that aren't
members of \protect\hypertarget{c11.xhtmlux5cux23Page_448}{}{}the same
VLAN. This is great because you get all the benefits you would with a
switched design without getting hit with all the problems you'd have if
all your users were in the same broadcast domain---sweet!

\subsubsection[Security]{\texorpdfstring{\protect\hypertarget{c11.xhtmlux5cux23c11-sec-3}{}{}Security}{Security}}

But there's always a catch, right? Time to get back to those security
issues. A flat internetwork's security used to be tackled by connecting
hubs and switches together with routers. So it was basically the
router's job to maintain security. This arrangement was pretty
ineffective for several reasons. First, anyone connecting to the
physical network could access the network resources located on that
particular physical LAN. Second, all anyone had to do to observe any and
all traffic traversing that network was to simply plug a network
analyzer into the hub. And similar to that last, scary, fact, users
could easily join a workgroup by just plugging their workstations into
the existing hub. That's about as secure as a barrel of honey in a bear
enclosure!

But that's exactly what makes VLANs so cool. If you build them and
create multiple broadcast groups, you can still have total control over
each port and user! So the days when anyone could just plug their
workstations into any switch port and gain access to network resources
are history because now you get to control each port and any resources
it can access.

And that's not even all---VLANs can be created in harmony with a
specific user's need for the network resources. Plus, switches can be
configured to inform a network management station about unauthorized
access to those vital network resources. And if you need inter-VLAN
communication, you can implement restrictions on a router to make sure
this all happens securely. You can also place restrictions on hardware
addresses, protocols, and applications. \emph{Now} we're talking
security---our honey barrel is now sealed tightly, made of solid
titanium and wrapped in razor wire!

\subsubsection[Flexibility and
Scalability]{\texorpdfstring{\protect\hypertarget{c11.xhtmlux5cux23c11-sec-4}{}{}Flexibility
and Scalability}{Flexibility and Scalability}}

If you've been paying attention so far, you know that layer 2 switches
only read frames for filtering because they don't look at the Network
layer protocol. You also know that by default, switches forward
broadcasts to all ports. But if you create and implement VLANs, you're
essentially creating smaller broadcast domains at layer 2.

As a result, broadcasts sent out from a node in one VLAN won't be
forwarded to ports configured to belong to a different VLAN. But if we
assign switch ports or users to VLAN groups on a switch or on a group of
connected switches, we gain the flexibility to exclusively add only the
users we want to let into that broadcast domain regardless of their
physical location. This setup can also work to block broadcast storms
caused by a faulty network interface card (NIC) as well as prevent an
intermediate device from propagating broadcast storms throughout the
entire internetwork. Those evils can still happen on the VLAN where the
problem originated, but the disease will be fully contained in that one
ailing VLAN!

\protect\hypertarget{c11.xhtmlux5cux23Page_449}{}{}Another advantage is
that when a VLAN gets too big, you can simply create more VLANs to keep
the broadcasts from consuming too much bandwidth. The fewer users in a
VLAN, the fewer users affected by broadcasts. This is all good, but you
seriously need to keep network services in mind and understand how the
users connect to these services when creating a VLAN. A good strategy is
to try to keep all services, except for the email and Internet access
that everyone needs, local to all users whenever possible.

\subsection[Identifying
VLANs]{\texorpdfstring{\protect\hypertarget{c11.xhtmlux5cux23c11-sec-5}{}{}Identifying
VLANs}{Identifying VLANs}}

Switch ports are layer 2--only interfaces that are associated with a
physical port that can belong to only one VLAN if it's an access port or
all VLANs if it's a trunk port.

Switches are definitely pretty busy devices. As myriad frames are
switched throughout the network, switches have to be able to keep track
of all of them, plus understand what to do with them depending on their
associated hardware addresses. And remember---frames are handled
differently according to the type of link they're traversing.

There are two different types of ports in a switched environment. Let's
take a look at the first type in
\protect\hyperlink{c11.xhtmlux5cux23figure11-5}{Figure 11.5}.

\begin{figure}
\centering
\includegraphics{images/c11f005.jpg}
\caption{{\protect\hyperlink{c11.xhtmlux5cux23figureanchor11-5}{\textbf{FIGURE
11.5}} Access ports}}
\end{figure}

Notice there are access ports for each host and an access port between
switches---one for each VLAN.

\protect\hypertarget{c11.xhtmlux5cux23Page_450}{}{}\textbf{Access ports}
An \emph{access port} belongs to and carries the traffic of only one
VLAN. Traffic is both received and sent in native formats with no VLAN
information (tagging) whatsoever. Anything arriving on an access port is
simply assumed to belong to the VLAN assigned to the port. Because an
access port doesn't look at the source address, tagged traffic---a frame
with added VLAN information---can be correctly forwarded and received
only on trunk ports.

With an access link, this can be referred to as the \emph{configured
VLAN} of the port. Any device attached to an \emph{access link} is
unaware of a VLAN membership---the device just assumes it's part of some
broadcast domain. But it doesn't have the big picture, so it doesn't
understand the physical network topology at all.

Another good bit of information to know is that switches remove any VLAN
information from the frame before it's forwarded out to an access-link
device. Remember that access-link devices can't communicate with devices
outside their VLAN unless the packet is routed. Also, you can only
create a switch port to be either an access port or a trunk port---not
both. So you've got to choose one or the other and know that if you make
it an access port, that port can be assigned to one VLAN only. In
\protect\hyperlink{c11.xhtmlux5cux23figure11-5}{Figure 11.5}, only the
hosts in the Sales VLAN can talk to other hosts in the same VLAN. This
is the same with the Admin VLAN, and they can both communicate to hosts
on the other switch because of an access link for each VLAN configured
between switches.

\begin{quote}
\textbf{Voice access ports} Not to confuse you, but all that I just said
about the fact that an access port can be assigned to only one VLAN is
really only sort of true. Nowadays, most switches will allow you to add
a second VLAN to an access port on a switch port for your voice traffic,
called the voice VLAN. The voice VLAN used to be called the auxiliary
VLAN, which allowed it to be overlaid on top of the data VLAN, enabling
both types of traffic to travel through the same port. Even though this
is technically considered to be a different type of link, it's still
just an access port that can be configured for both data and voice
VLANs. This allows you to connect both a phone and a PC device to one
switch port but still have each device in a separate VLAN.
\end{quote}

\textbf{Trunk ports} Believe it or not, the term \emph{trunk port} was
inspired by the telephone system trunks, which carry multiple telephone
conversations at a time. So it follows that trunk ports can similarly
carry multiple VLANs at a time as well.

A \emph{trunk link} is a 100, 1,000, or 10,000 Mbps point-to-point link
between two switches, between a switch and router, or even between a
switch and server, and it carries the traffic of multiple VLANs---from 1
to 4,094 VLANs at a time. But the amount is really only up to 1,001
unless you're going with something called extended VLANs.

Instead of an access link for each VLAN between switches, we'll create a
trunk link, demonstrated in
\protect\hyperlink{c11.xhtmlux5cux23figure11-6}{Figure 11.6}.

\protect\hypertarget{c11.xhtmlux5cux23Page_451}{}{}

\begin{figure}
\centering
\includegraphics{images/c11f006.jpg}
\caption{{\protect\hyperlink{c11.xhtmlux5cux23figureanchor11-6}{\textbf{FIGURE
11.6}} VLANs can span across multiple switches by using trunk links,
which carry traffic for multiple VLANs.}}
\end{figure}

Trunking can be a real advantage because with it, you get to make a
single port part of a whole bunch of different VLANs at the same time.
This is a great feature because you can actually set ports up to have a
server in two separate broadcast domains simultaneously so your users
won't have to cross a layer 3 device (router) to log in and access it.
Another benefit to trunking comes into play when you're connecting
switches. Trunk links can carry the frames of various VLANs across them,
but by default, if the links between your switches aren't trunked, only
information from the configured access VLAN will be switched across that
link.

It's also good to know that all VLANs send information on a trunked link
unless you clear each VLAN by hand, and no worries, I'll show you how to
clear individual VLANs from a trunk in a bit.

Okay---it's finally time to tell you about frame tagging and the VLAN
identification methods used in it across our trunk links.

\subsubsection[Frame
Tagging]{\texorpdfstring{\protect\hypertarget{c11.xhtmlux5cux23c11-sec-6}{}{}Frame
Tagging}{Frame Tagging}}

As you now know, you can set up your VLANs to span more than one
connected switch. You can see that going on in
\protect\hyperlink{c11.xhtmlux5cux23figure11-6}{Figure 11.6}, which
depicts hosts from two VLANs spread across two switches. This flexible,
power-packed capability is probably the main advantage to implementing
VLANs, and we can do this with up to a thousand VLANs and thousands upon
thousands of hosts!

\protect\hypertarget{c11.xhtmlux5cux23Page_452}{}{}All this can get kind
of complicated---even for a switch---so there needs to be a way for each
one to keep track of all the users and frames as they travel the switch
fabric and VLANs. When I say, ``switch fabric,'' I'm just referring to a
group of switches that share the same VLAN information. And this just
happens to be where\emph{frame tagging} enters the scene. This frame
identification method uniquely assigns a user-defined VLAN ID to each
frame.

Here's how it works: Once within the switch fabric, each switch that the
frame reaches must first identify the VLAN ID from the frame tag. It
then finds out what to do with the frame by looking at the information
in what's known as the filter table. If the frame reaches a switch that
has another trunked link, the frame will be forwarded out of the
trunk-link port.

Once the frame reaches an exit that's determined by the forward/filter
table to be an access link matching the frame's VLAN ID, the switch will
remove the VLAN identifier. This is so the destination device can
receive the frames without being required to understand their VLAN
identification information.

Another great thing about trunk ports is that they'll support tagged and
untagged traffic simultaneously if you're using 802.1q trunking, which
we will talk about next. The trunk port is assigned a default port VLAN
ID (PVID) for a VLAN upon which all untagged traffic will travel. This
VLAN is also called the native VLAN and is always VLAN 1 by default, but
it can be changed to any VLAN number.

Similarly, any untagged or tagged traffic with a NULL (unassigned) VLAN
ID is assumed to belong to the VLAN with the port default PVID. Again,
this would be VLAN 1 by default. A packet with a VLAN ID equal to the
outgoing port native VLAN is sent untagged and can communicate to only
hosts or devices in that same VLAN. All other VLAN traffic has to be
sent with a VLAN tag to communicate within a particular VLAN that
corresponds with that tag.

\subsubsection[VLAN Identification
Methods]{\texorpdfstring{\protect\hypertarget{c11.xhtmlux5cux23c11-sec-7}{}{}VLAN
Identification Methods}{VLAN Identification Methods}}

VLAN identification is what switches use to keep track of all those
frames as they're traversing a switch fabric. It's how switches identify
which frames belong to which VLANs, and there's more than one trunking
method.

\paragraph{Inter-Switch Link (ISL)}

\emph{Inter-Switch Link (ISL)} is a way of explicitly tagging VLAN
information onto an Ethernet frame. This tagging information allows
VLANs to be multiplexed over a trunk link through an external
encapsulation method. This allows the switch to identify the VLAN
membership of a frame received over the trunked link.

By running ISL, you can interconnect multiple switches and still
maintain VLAN information as traffic travels between switches on trunk
links. ISL functions at layer 2 by encapsulating a data frame with a new
header and by performing a new cyclic redundancy check (CRC).

Of note is that ISL is proprietary to Cisco switches and is pretty
versatile as well. ISL can be used on a switch port, router interfaces,
and server interface cards to trunk a server.

\protect\hypertarget{c11.xhtmlux5cux23Page_453}{}{}Although some Cisco
switches still support ISL frame tagging, Cisco is moving toward using
only 802.1q.

\paragraph{IEEE 802.1q}

Created by the IEEE as a standard method of frame tagging, IEEE 802.1q
actually inserts a field into the frame to identify the VLAN. If you're
trunking between a Cisco switched link and a different brand of switch,
you've got to use 802.1q for the trunk to work.

Unlike ISL, which encapsulates the frame with control information,
802.1q inserts an 802.1q field along with tag control information, as
shown in \protect\hyperlink{c11.xhtmlux5cux23figure11-7}{Figure 11.7}.

\begin{figure}
\centering
\includegraphics{images/c11f007.jpg}
\caption{{\protect\hyperlink{c11.xhtmlux5cux23figureanchor11-7}{\textbf{FIGURE
11.7}} IEEE 802.1q encapsulation with and without the 802.1q tag}}
\end{figure}

For the Cisco exam objectives, it's only the 12-bit VLAN ID that
matters. This field identifies the VLAN and can be 2 to the 12th, minus
2 for the 0 and 4,095 reserved VLANs, which means an 802.1q tagged frame
can carry information for 4,094 VLANs.

It works like this: You first designate each port that's going to be a
trunk with 802.1q encapsulation. The other ports must be assigned a
specific VLAN ID in order for them to communicate. VLAN 1 is the default
native VLAN, and when using 802.1q, all traffic for a native VLAN is
untagged. The ports that populate the same trunk create a group with
this native VLAN and each port gets tagged with an identification number
reflecting that. Again the default is VLAN 1. The native VLAN allows the
trunks to accept information that was received without any VLAN
identification or frame tag.

Most 2960 model switches only support the IEEE 802.1q trunking protocol,
but the 3560 will support both the ISL and IEEE methods, which you'll
see later in this chapter.

\begin{center}\rule{0.5\linewidth}{0.5pt}\end{center}

\includegraphics{images/note.png}The basic purpose of ISL and 802.1q
frame-tagging methods is to provide inter-switch VLAN communication.
Remember that any ISL or 802.1q frame tagging is removed if a frame is
forwarded out an access link---tagging is used internally and across
trunk links only!

\begin{center}\rule{0.5\linewidth}{0.5pt}\end{center}

\subsection[Routing between
VLANs]{\texorpdfstring{\protect\hypertarget{c11.xhtmlux5cux23c11-sec-8}{}{}\protect\hypertarget{c11.xhtmlux5cux23Page_454}{}{}Routing
between VLANs}{Routing between VLANs}}

Hosts in a VLAN live in their own broadcast domain and can communicate
freely. VLANs create network partitioning and traffic separation at
layer 2 of the OSI, and as I said when I told you why we still need
routers, if you want hosts or any other IP-addressable device to
communicate between VLANs, you must have a layer 3 device to provide
routing.

For this, you can use a router that has an interface for each VLAN or a
router that supports ISL or 802.1q routing. The least expensive router
that supports ISL or 802.1q routing is the 2600 series router. You'd
have to buy that from a used-equipment reseller because they are
end-of-life, or EOL. I'd recommend at least a 2800 as a bare minimum,
but even that only supports 802.1q; Cisco is really moving away from
ISL, so you probably should only be using 802.1q anyway. Some 2800s may
support both ISL and 802.1q; I've just never seen it supported.

Anyway, as shown in
\protect\hyperlink{c11.xhtmlux5cux23figure11-8}{Figure 11.8}, if you had
two or three VLANs, you could get by with a router equipped with two or
three FastEthernet connections. And 10Base-T is okay for home study
purposes, and I mean only for your studies, but for anything else I'd
highly recommend Gigabit interfaces for real power under the hood!

What we see in \protect\hyperlink{c11.xhtmlux5cux23figure11-8}{Figure
11.8} is that each router interface is plugged into an access link. This
means that each of the routers' interface IP addresses would then become
the default gateway address for each host in each respective VLAN.

\begin{figure}
\centering
\includegraphics{images/c11f008.jpg}
\caption{{\protect\hyperlink{c11.xhtmlux5cux23figureanchor11-8}{\textbf{FIGURE
11.8}} Router connecting three VLANs together for inter-VLAN
communication, one router interface for each VLAN}}
\end{figure}

If you have more VLANs available than router interfaces, you can
configure trunking on one FastEthernet interface or buy a layer 3
switch, like the old and now cheap 3560 or a higher-end switch like a
3850. You could even opt for a 6800 if you've got money to burn!

Instead of using a router interface for each VLAN, you can use one
FastEthernet interface and run ISL or 802.1q trunking.
\protect\hyperlink{c11.xhtmlux5cux23figure11-9}{Figure 11.9} shows how a
FastEthernet interface on a
\protect\hypertarget{c11.xhtmlux5cux23Page_455}{}{}router will look when
configured with ISL or 802.1q trunking. This allows all VLANs to
communicate through one interface. Cisco calls this a router on a stick
(ROAS).

\begin{figure}
\centering
\includegraphics{images/c11f009.jpg}
\caption{{\protect\hyperlink{c11.xhtmlux5cux23figureanchor11-9}{\textbf{FIGURE
11.9}} Router on a stick: single router interface connecting all three
VLANs together for inter-VLAN communication}}
\end{figure}

I really want to point out that this creates a potential bottleneck, as
well as a single point of failure, so your host/VLAN count is limited.
To how many? Well, that depends on your traffic level. To really make
things right, you'd be better off using a higher-end switch and routing
on the backplane. But if you just happen to have a router sitting
around, configuring this method is free, right?

\protect\hyperlink{c11.xhtmlux5cux23figure11-10}{Figure 11.10} shows how
we would create a router on a stick using a router's physical interface
by creating logical interfaces---one for each VLAN.

\begin{figure}
\centering
\includegraphics{images/c11f010.jpg}
\caption{{\protect\hyperlink{c11.xhtmlux5cux23figureanchor11-10}{\textbf{FIGURE
11.10}} A router creates logical interfaces.}}
\end{figure}

Here we see one physical interface divided into multiple subinterfaces,
with one subnet assigned per VLAN, each subinterface being the default
gateway address for each VLAN/subnet. An encapsulation identifier must
be assigned to each subinterface to define the VLAN ID of that
subinterface. In the next section where I'll configure VLANs and
inter-VLAN routing, I'll configure our switched network with a router on
a stick and demonstrate this configuration for you.

But wait, there's still one more way to go about routing! Instead of
using an external router interface for each VLAN, or an external router
on a stick, we can configure logical interfaces on the backplane of the
layer 3 switch; this is called inter-VLAN routing (IVR), and it's
configured with a switched virtual interface (SVI).
\protect\hyperlink{c11.xhtmlux5cux23figure11-11}{Figure 11.11} shows how
hosts see these virtual interfaces.

\protect\hypertarget{c11.xhtmlux5cux23Page_456}{}{}

\begin{figure}
\centering
\includegraphics{images/c11f011.jpg}
\caption{{\protect\hyperlink{c11.xhtmlux5cux23figureanchor11-11}{\textbf{FIGURE
11.11}} With IVR, routing runs on the backplane of the switch, and it
appears to the hosts that a router is present.}}
\end{figure}

In \protect\hyperlink{c11.xhtmlux5cux23figure11-11}{Figure 11.11}, it
appears there's a router present, but there is no physical router
present as there was when we used router on a stick. The IVR process
takes little effort and is easy to implement, which makes it very cool!
Plus, it's a lot more efficient for inter-VLAN routing than an external
router is. To implement IVR on a multilayer switch, we just need to
create logical interfaces in the switch configuration for each VLAN.
We'll configure this method in a minute, but first let's take our
existing switched network from Chapter 10, ``Layer 2 Switching,'' and
add some VLANs, then configure VLAN memberships and trunk links between
our switches.

\subsection[Configuring
VLANs]{\texorpdfstring{\protect\hypertarget{c11.xhtmlux5cux23c11-sec-9}{}{}Configuring
VLANs}{Configuring VLANs}}

Now this may come as a surprise to you, but configuring VLANs is
actually pretty easy. It's just that figuring out which users you want
in each VLAN is not, and doing that can eat up a lot of your time! But
once you've decided on the number of VLANs you want to create and
established which users you want belonging to each one, it's time to
bring your first VLAN into the world.

To configure VLANs on a Cisco Catalyst switch, use the global config
\texttt{vlan} command. In the following example, I'm going to
demonstrate how to configure VLANs on the S1 switch by creating three
VLANs for three different departments---again, remember that VLAN 1 is
the native and management VLAN by default:

\begin{verbatim}
S1(config)#vlan ?

  WORD        ISL VLAN IDs 1-4094
  access-map  Create vlan access-map or enter vlan access-map command mode
  dot1q       dot1q parameters
  filter      Apply a VLAN Map
  group       Create a vlan group
  internal    internal VLAN
S1(config)#vlan 2
S1(config-vlan)#name Sales
S1(config-vlan)#vlan 3
S1(config-vlan)#name Marketing
S1(config-vlan)#vlan 4
S1(config-vlan)#name Accounting
S1(config-vlan)#vlan 5
S1(config-vlan)#name Voice
S1(config-vlan)#^Z
S1#
\end{verbatim}

In this output, you can see that you can create VLANs from 1 to 4094.
But this is only mostly true. As I said, VLANs can really only be
created up to 1001, and you can't use, change, rename, or delete VLANs 1
or 1002 through 1005 because they're reserved. The VLAN numbers above
1005 are called extended VLANs and won't be saved in the database unless
your switch is set to what is called VLAN Trunking Protocol (VTP)
transparent mode. You won't see these VLAN numbers used too often in
production. Here's an example of me attempting to set my S1 switch to
VLAN 4000 when my switch is set to VTP server mode (the default VTP
mode):

\begin{verbatim}
S1#config t
S1(config)#vlan 4000
S1(config-vlan)#^Z
% Failed to create VLANs 4000
Extended VLAN(s) not allowed in current VTP mode.
%Failed to commit extended VLAN(s) changes.
\end{verbatim}

After you create the VLANs that you want, you can use the
\texttt{show\ vlan} command to check them out. But notice that, by
default, all ports on the switch are in VLAN 1. To change the VLAN
associated with a port, you need to go to each interface and
specifically tell it which VLAN to be a part of.

\begin{center}\rule{0.5\linewidth}{0.5pt}\end{center}

\includegraphics{images/note.png}Remember that a created VLAN is unused
until it is assigned to a switch port or ports and that all ports are
always assigned in VLAN 1 unless set otherwise.

\begin{center}\rule{0.5\linewidth}{0.5pt}\end{center}

Once the VLANs are created, verify your configuration with the
\texttt{show\ vlan} command (\texttt{sh\ vlan} for short):

\begin{verbatim}
S1#sh vlan
VLAN Name                       Status    Ports
---- ------------------------- --------- -------------------------------
1    default                    active    Fa0/1, Fa0/2, Fa0/3, Fa0/4
                                          Fa0/5, Fa0/6, Fa0/7, Fa0/8
                                          Fa0/9, Fa0/10, Fa0/11, Fa0/12
                                          Fa0/13, Fa0/14, Fa0/19, Fa0/20
                                          Fa0/21, Fa0/22, Fa0/23, Gi0/1
                                          Gi0/2
2    Sales                            active
3    Marketing                        active
4    Accounting                       active
5    Voice                            active
[output cut]
\end{verbatim}

This may seem repetitive, but it's important, and I want you to remember
it: You can't change, delete, or rename VLAN 1 because it's the default
VLAN and you just can't change that---period. It's also the native VLAN
of all switches by default, and Cisco recommends that you use it as your
management VLAN. If you're worried about security issues, then change
it! Basically, any ports that aren't specifically assigned to a
different VLAN will be sent down to the native VLAN---VLAN 1.

In the preceding S1 output, you can see that ports Fa0/1 through Fa0/14,
Fa0/19 through 23, and Gi0/1 and Gi0/2 uplinks are all in VLAN 1. But
where are ports 15 through 18? First, understand that the command
\texttt{show\ vlan} only displays access ports, so now that you know
what you're looking at with the \texttt{show\ vlan} command, where do
you think ports Fa15--18 are? That's right! They are trunked ports.
Cisco switches run a proprietary protocol called \emph{Dynamic Trunk
Protocol (DTP)}, and if there is a compatible switch connected, they
will start trunking automatically, which is precisely where my four
ports are. You have to use the \texttt{show\ interfaces\ trunk} command
to see your trunked ports like this:

\begin{verbatim}
S1# show interfaces trunk
Port        Mode             Encapsulation  Status        Native vlan
Fa0/15      desirable        n-isl          trunking      1
Fa0/16      desirable        n-isl          trunking      1
Fa0/17      desirable        n-isl          trunking      1
Fa0/18      desirable        n-isl          trunking      1
\end{verbatim}

\begin{verbatim}
Port        Vlans allowed on trunk
Fa0/15      1-4094
Fa0/16      1-4094
Fa0/17      1-4094
Fa0/18      1-4094
\end{verbatim}

\begin{verbatim}
 [output cut]
\end{verbatim}

This output reveals that the VLANs from 1 to 4094 are allowed across the
trunk by default. Another helpful command, which is also part of the
Cisco exam objectives, is
the\texttt{show\ interfaces\ interface\ switchport} command:

\begin{verbatim}
S1#sh interfaces fastEthernet 0/15 switchport
Name: Fa0/15
Switchport: Enabled
Administrative Mode: dynamic desirable
Operational Mode: trunk
Administrative Trunking Encapsulation: negotiate
Operational Trunking Encapsulation: isl
Negotiation of Trunking: On
Access Mode VLAN: 1 (default)
Trunking Native Mode VLAN: 1 (default)
Administrative Native VLAN tagging: enabled
Voice VLAN: none
[output cut]
\end{verbatim}

The highlighted output shows us the administrative mode of
\texttt{dynamic\ desirable}, that the port is a trunk port, and that DTP
was used to negotiate the frame-tagging method of ISL. It also
predictably shows that the native VLAN is the default of 1.

Now that we can see the VLANs created, we can assign switch ports to
specific ones. Each port can be part of only one VLAN, with the
exception of voice access ports. Using trunking, you can make a port
available to traffic from all VLANs. I'll cover that next.

\subsubsection[Assigning Switch Ports to
VLANs]{\texorpdfstring{\protect\hypertarget{c11.xhtmlux5cux23c11-sec-10}{}{}Assigning
Switch Ports to VLANs}{Assigning Switch Ports to VLANs}}

You configure a port to belong to a VLAN by assigning a membership mode
that specifies the kind of traffic the port carries plus the number of
VLANs it can belong to. You can also configure each port on a switch to
be in a specific VLAN (access port) by using the interface
\texttt{switchport} command. You can even configure multiple ports at
the same time with the \texttt{interface\ range} command.

In the next example, I'll configure interface Fa0/3 to VLAN 3. This is
the connection from the S3 switch to the host device:

\begin{verbatim}
S3#config t
S3(config)#int fa0/3
S3(config-if)#switchport ?
  access         Set access mode characteristics of the interface
  autostate      Include or exclude this port from vlan link up calculation
  backup         Set backup for the interface
  block          Disable forwarding of unknown uni/multi cast addresses
  host           Set port host
  mode           Set trunking mode of the interface
  nonegotiate    Device will not engage in negotiation protocol on this
                 interface
  port-security  Security related command
  priority       Set appliance 802.1p priority
  private-vlan   Set the private VLAN configuration
  protected      Configure an interface to be a protected port
  trunk          Set trunking characteristics of the interface
  voice          Voice appliance attributes voice
\end{verbatim}

Well now, what do we have here? There's some new stuff showing up in our
output now. We can see various commands---some that I've already
covered, but no worries because I'm going to cover the \texttt{access},
\texttt{mode}, \texttt{nonegotiate}, and \texttt{trunk} commands very
soon. Let's start with setting an access port on S1, which is probably
the most widely used type of port you'll find on production switches
that have VLANs configured:

\begin{verbatim}
S3(config-if)#switchport mode ?
    access        Set trunking mode to ACCESS unconditionally
  dot1q-tunnel  set trunking mode to TUNNEL unconditionally
  dynamic       Set trunking mode to dynamically negotiate access or trunk mode
  private-vlan  Set private-vlan mode
  trunk         Set trunking mode to TRUNK unconditionally
\end{verbatim}

\begin{verbatim}
S3(config-if)#switchport mode access
S3(config-if)#switchport access vlan 3
S3(config-if)#switchport voice vlan 5
\end{verbatim}

By starting with the \texttt{switchport\ mode\ access} command, you're
telling the switch that this is a nontrunking layer 2 port. You can then
assign a VLAN to the port with the \texttt{switchport\ access} command,
as well as configure the same port to be a member of a different type of
VLAN, called the \texttt{voice} VLAN. This allows you to connect a
laptop into a phone, and the phone into a single switch port. Remember,
you can choose many ports to configure simultaneously with the
\texttt{interface\ range} command.

Let's take a look at our VLANs now:

\begin{verbatim}
S3#show vlan
VLAN Name                       Status    Ports
---- ------------------------ --------- -------------------------------
1    default                   active     Fa0/4, Fa0/5, Fa0/6, Fa0/7
                                          Fa0/8, Fa0/9, Fa0/10, Fa0/11,
                                          Fa0/12, Fa0/13, Fa0/14, Fa0/19,
                                          Fa0/20, Fa0/21, Fa0/22, Fa0/23,
                                          Gi0/1, Gi0/2
2    Sales                     active
3    Marketing                 active    Fa0/3
5    Voice                     active    Fa0/3
\end{verbatim}

Notice that port Fa0/3 is now a member of VLAN 3 and VLAN 5---two
different types of VLANs. But, can you tell me where ports 1 and 2 are?
And why aren't they showing up in the output of \texttt{show\ vlan}?
That's right, because they are trunk ports!

We can also see this with the
\texttt{show\ interfaces\ interface\ switchport} command:

\begin{verbatim}
S3#sh int fa0/3 switchport
Name: Fa0/3
Switchport: Enabled
Administrative Mode: static access
Operational Mode: static access
Administrative Trunking Encapsulation: negotiate
Negotiation of Trunking: Off
Access Mode VLAN: 3 (Marketing)
Trunking Native Mode VLAN: 1 (default)
Administrative Native VLAN tagging: enabled
Voice VLAN: 5 (Voice)
\end{verbatim}

The highlighted output shows that Fa0/3 is an access port and a member
of VLAN 3 (Marketing), as well as a member of the Voice VLAN 5.

That's it. Well, sort of. If you plugged devices into each VLAN port,
they can only talk to other devices in the same VLAN. But as soon as you
learn a bit more about trunking, we're going to enable inter-VLAN
communication!

\subsubsection[Configuring Trunk
Ports]{\texorpdfstring{\protect\hypertarget{c11.xhtmlux5cux23c11-sec-11}{}{}Configuring
Trunk Ports}{Configuring Trunk Ports}}

The 2960 switch only runs the IEEE 802.1q encapsulation method. To
configure trunking on a FastEthernet port, use the interface command
\texttt{switchport\ mode\ trunk}. It's a tad diff­erent on the 3560
switch.

The following switch output shows the trunk configuration on interfaces
Fa0/15--18 as set to \texttt{trunk}:

\begin{verbatim}
S1(config)#int range f0/15-18
S1(config-if-range)#switchport trunk encapsulation dot1q
S1(config-if-range)#switchport mode trunk
\end{verbatim}

If you have a switch that only runs the 802.1q encapsulation method,
then you wouldn't use the \texttt{encapsulation} command as I did in the
preceding output. Let's check out our trunk ports now:

\begin{verbatim}
S1(config-if-range)#do sh int f0/15 swi
Name: Fa0/15
Switchport: Enabled
Administrative Mode: trunk
Operational Mode: trunk
Administrative Trunking Encapsulation: dot1q
Operational Trunking Encapsulation: dot1q
Negotiation of Trunking: On
Access Mode VLAN: 1 (default)
Trunking Native Mode VLAN: 1 (default)
Administrative Native VLAN tagging: enabled
Voice VLAN: none
\end{verbatim}

\protect\hypertarget{c11.xhtmlux5cux23Page_462}{}{}Notice that port
Fa0/15 is a trunk and running 802.1q. Let's take another look:

\begin{verbatim}
S1(config-if-range)#do sh int trunk
Port        Mode             Encapsulation  Status        Native vlan
Fa0/15      on               802.1q         trunking      1
Fa0/16      on               802.1q         trunking      1
Fa0/17      on               802.1q         trunking      1
Fa0/18      on               802.1q         trunking      1
Port        Vlans allowed on trunk
Fa0/15      1-4094
Fa0/16      1-4094
Fa0/17      1-4094
Fa0/18      1-4094
\end{verbatim}

Take note of the fact that ports 15--18 are now in the trunk mode of on
and the encapsulation is now 802.1q instead of the negotiated ISL.
Here's a description of the different options available when configuring
a switch interface:

\texttt{switchport\ mode\ access} I discussed this in the previous
section, but this puts the interface (access port) into permanent
nontrunking mode and negotiates to convert the link into a nontrunk
link. The interface becomes a nontrunk interface regardless of whether
the neighboring interface is a trunk interface. The port would be a
dedicated layer 2 access port.

\texttt{switchport\ mode\ dynamic\ auto} This mode makes the interface
able to convert the link to a trunk link. The interface becomes a trunk
interface if the neighboring interface is set to trunk or desirable
mode. The default is \texttt{dynamic\ auto} on a lot of Cisco switches,
but that default trunk method is changing to \texttt{dynamic\ desirable}
on most new models.

\texttt{switchport\ mode\ dynamic\ desirable} This one makes the
interface actively attempt to convert the link to a trunk link. The
interface becomes a trunk interface if the neighboring interface is set
to \texttt{trunk}, \texttt{desirable}, or \texttt{auto} mode. I used to
see this mode as the default on some switches, but not any longer. This
is now the default switch port mode for all Ethernet interfaces on all
new Cisco switches.

\texttt{switchport\ mode\ trunk} Puts the interface into permanent
trunking mode and negotiates to convert the neighboring link into a
trunk link. The interface becomes a trunk interface even if the
neighboring interface isn't a trunk interface.

\texttt{switchport\ nonegotiate} Prevents the interface from generating
DTP frames. You can use this command only when the interface switchport
mode is access or trunk. You must manually configure the neighboring
interface as a trunk interface to establish a trunk link.

\begin{center}\rule{0.5\linewidth}{0.5pt}\end{center}

\includegraphics{images/note.png}Dynamic Trunking Protocol (DTP) is used
for negotiating trunking on a link between two devices as well as
negotiating the encapsulation type of either 802.1q or ISL. I use the
\texttt{nonegotiate} command when I want dedicated trunk ports; no
questions asked.

\begin{center}\rule{0.5\linewidth}{0.5pt}\end{center}

\protect\hypertarget{c11.xhtmlux5cux23Page_463}{}{}To disable trunking
on an interface, use the \texttt{switchport\ mode\ access} command,
which sets the port back to a dedicated layer 2 access switch port.

\subsubsection[Defining the Allowed VLANs on a
Trunk]{\texorpdfstring{\protect\hypertarget{c11.xhtmlux5cux23c11-sec-12}{}{}Defining
the Allowed VLANs on a Trunk}{Defining the Allowed VLANs on a Trunk}}

As I've mentioned, trunk ports send and receive information from all
VLANs by default, and if a frame is untagged, it's sent to the
management VLAN. Understand that this applies to the extended range
VLANs too.

But we can remove VLANs from the allowed list to prevent traffic from
certain VLANs from traversing a trunked link. I'll show you how you'd do
that, but first let me again demonstrate that all VLANs are allowed
across the trunk link by default:

\begin{verbatim}
S1#sh int trunk
[output cut]
Port        Vlans allowed on trunk
Fa0/15      1-4094
Fa0/16      1-4094
Fa0/17      1-4094
Fa0/18      1-4094
S1(config)#S1(config)#
S1(config-if)#S1(config-if)#
S1(config-if)#S1(config-if)#
[output cut]
Port        Vlans allowed on trunk
Fa0/15      4,6,12,15
Fa0/16      1-4094
Fa0/17      1-4094
Fa0/18      1-4094
\end{verbatim}

The preceding command affected the trunk link configured on S1 port
F0/15, causing it to permit all traffic sent and received for VLANs 4,
6, 12, and 15. You can try to remove VLAN 1 on a trunk link, but it will
still send and receive management like CDP, DTP, and VTP, so what's the
point?

To remove a range of VLANs, just use the hyphen:

\begin{verbatim}
S1(config-if)#switchport trunk allowed vlan remove 4-8
\end{verbatim}

If by chance someone has removed some VLANs from a trunk link and you
want to set the trunk back to default, just use this command:

\begin{verbatim}
S1(config-if)#switchport trunk allowed vlan all
\end{verbatim}

Next, I want to show you how to configure a native VLAN for a trunk
before we start routing between VLANs.

\subsubsection[Changing or Modifying the Trunk Native
VLAN]{\texorpdfstring{\protect\hypertarget{c11.xhtmlux5cux23c11-sec-13}{}{}\protect\hypertarget{c11.xhtmlux5cux23Page_464}{}{}Changing
or Modifying the Trunk Native
VLAN}{Changing or Modifying the Trunk Native VLAN}}

You can change the trunk port native VLAN from VLAN 1, which many people
do for security reasons. To change the native VLAN, use the following
command:

\begin{verbatim}
S1(config)#int f0/15
S1(config-if)#switchport trunk native vlan ?
  <1-4094>  VLAN ID of the native VLAN when this port is in trunking mode
\end{verbatim}

\begin{verbatim}
S1(config-if)#switchport trunk native vlan 4
1w6d: %CDP-4-NATIVE_VLAN_MISMATCH: Native VLAN mismatch discovered on FastEthernet0/15 (4), with S3 FastEthernet0/1 (1).
\end{verbatim}

So we've changed our native VLAN on our trunk link to 4, and by using
the\texttt{show\ running-config} command, I can see the configuration
under the trunk link:

\begin{verbatim}
S1#sh run int f0/15
Building configuration...
\end{verbatim}

\begin{verbatim}
Current configuration : 202 bytes
!
interface FastEthernet0/15
 description 1st connection to S3
 switchport trunk encapsulation dot1q
 switchport trunk native vlan 4
 switchport trunk allowed vlan 4,6,12,15
 switchport mode trunk
end
\end{verbatim}

\begin{verbatim}
S1#!
\end{verbatim}

Oops---wait a minute! You didn't think it would be this easy and would
just start working, did you? Of course not! Here's the rub: If all
switches don't have the same native VLAN configured on the given trunk
links, then we'll start to receive this error, which happened
immediately after I entered the command:

\begin{verbatim}
1w6d: %CDP-4-NATIVE_VLAN_MISMATCH: Native VLAN mismatch discovered
on FastEthernet0/15 (4), with S3 FastEthernet0/1 (1).
\end{verbatim}

Actually, this is a good, noncryptic error, so either we can go to the
other end of our trunk link(s) and change the native VLAN or we set the
native VLAN back to the default to fix it. Here's how we'd do that:

\begin{verbatim}
S1(config-if)#no switchport trunk native vlan
1w6d: %SPANTREE-2-UNBLOCK_CONSIST_PORT: Unblocking FastEthernet0/15
on VLAN0004. Port consistency restored.
\end{verbatim}

\protect\hypertarget{c11.xhtmlux5cux23Page_465}{}{}Now our trunk link is
using the default VLAN 1 as the native VLAN. Just remember that all
switches on a given trunk must use the same native VLAN or you'll have
some serious management problems. These issues won't affect user data,
just management traffic between switches. Now, let's mix it up by
connecting a router into our switched network and configure inter-VLAN
communication.

\subsubsection[Configuring Inter-VLAN
Routing]{\texorpdfstring{\protect\hypertarget{c11.xhtmlux5cux23c11-sec-14}{}{}Configuring
Inter-VLAN Routing}{Configuring Inter-VLAN Routing}}

By default, only hosts that are members of the same VLAN can
communicate. To change this and allow inter-VLAN communication, you need
a router or a layer 3 switch. I'm going to start with the router
approach.

To support ISL or 802.1q routing on a FastEthernet interface, the
router's interface is divided into logical interfaces---one for each
VLAN---as was shown in
\protect\hyperlink{c11.xhtmlux5cux23figure11-10}{Figure 11.10}. These
are called \emph{subinterfaces}. From a FastEthernet or Gigabit
interface, you can set the interface to trunk with the
\texttt{encapsulation} command:

\begin{verbatim}
ISR#config t
ISR(config)#int f0/0.1
ISR(config-subif)#encapsulation ?
  dot1Q  IEEE 802.1Q Virtual LAN
ISR(config-subif)#encapsulation dot1Q ?
  <1-4094>  IEEE 802.1Q VLAN ID
\end{verbatim}

Notice that my 2811 router (named ISR) only supports 802.1q. We'd need
an older-model router to run the ISL encapsulation, but why bother?

The subinterface number is only locally significant, so it doesn't
matter which subinterface numbers are configured on the router. Most of
the time, I'll configure a subinterface with the same number as the VLAN
I want to route. It's easy to remember that way since the subinterface
number is used only for administrative purposes.

It's really important that you understand that each VLAN is actually a
separate subnet. True, I know---they don't \emph{have} to be. But it
really is a good idea to configure your VLANs as separate subnets, so
just do that. Before we move on, I want to define \emph{upstream
routing}. This is a term used to define the router on a stick. This
router will provide inter-VLAN routing, but it can also be used to
forward traffic upstream from the switched network to other parts of the
corporate network or Internet.

Now, I need to make sure you're fully prepared to configure inter-VLAN
routing as well as determine the IP addresses of hosts connected in a
switched VLAN environment. And as always, it's also a good idea to be
able to fix any problems that may arise. To set you up for success, let
me give you few examples.

First, start by looking at
\protect\hyperlink{c11.xhtmlux5cux23figure11-12}{Figure 11.12} and read
the router and switch configuration within it. By this point in the
book, you should be able to determine the IP address, masks, and default
gateways of each of the hosts in the VLANs.

\protect\hypertarget{c11.xhtmlux5cux23Page_466}{}{}

\begin{figure}
\centering
\includegraphics{images/c11f012.jpg}
\caption{{\protect\hyperlink{c11.xhtmlux5cux23figureanchor11-12}{\textbf{FIGURE
11.12}} Configuring inter-VLAN example 1}}
\end{figure}

The next step is to figure out which subnets are being used. By looking
at the router configuration in the figure, you can see that we're using
192.168.10.0/28 for VLAN1, 192.168.1.64/26 with VLAN 2, and
192.168.1.128/27 for VLAN 10.

By looking at the switch configuration, you can see that ports 2 and 3
are in VLAN 2 and port 4 is in VLAN 10. This means that Host A and Host
B are in VLAN 2 and Host C is in VLAN 10.

But wait---what's that IP address doing there under the physical
interface? Can we even do that? Sure we can! If we place an IP address
under the physical interface, the result is that frames sent from the IP
address would be untagged. So what VLAN would those frames be a member
of? By default, they would belong to VLAN 1, our management VLAN. This
means the address 192.168.10.1 /28 is my native VLAN IP address for this
switch.

Here's what the hosts' IP addresses should be:

\begin{enumerate}
\tightlist
\item
  \textbf{Host A:} 192.168.1.66, 255.255.255.192, default gateway
  192.168.1.65
\item
  \textbf{Host B:} 192.168.1.67, 255.255.255.192, default gateway
  192.168.1.65
\item
  \textbf{Host C:} 192.168.1.130, 255.255.255.224, default gateway
  192.168.1.129
\end{enumerate}

The hosts could be any address in the range---I just chose the first
available IP address after the default gateway address. That wasn't so
hard, was it?

Now, again using \protect\hyperlink{c11.xhtmlux5cux23figure11-12}{Figure
11.12}, let's go through the commands necessary to configure switch port
1 so it will establish a link with the router and provide inter-VLAN
communication using the IEEE version for encapsulation. Keep in mind
that the commands can vary slightly depending on what type of switch
you're dealing with.

For a 2960 switch, use the following:

\begin{verbatim}
2960#config t
2960(config)#interface fa0/1
2960(config-if)#switchport mode trunk
\end{verbatim}

\protect\hypertarget{c11.xhtmlux5cux23Page_467}{}{}That's it! As you
already know, the 2960 switch can only run the 802.1q encapsulation, so
there's no need to specify it. You can't anyway. For a 3560, it's
basically the same, but because it can run ISL and 802.1q, you have to
specify the trunking encapsulation protocol you're going to use.

\begin{center}\rule{0.5\linewidth}{0.5pt}\end{center}

\includegraphics{images/note.png}Remember that when you create a trunked
link, all VLANs are allowed to pass data by default.

\begin{center}\rule{0.5\linewidth}{0.5pt}\end{center}

Let's take a look at
\protect\hyperlink{c11.xhtmlux5cux23figure11-13}{Figure 11.13} and see
what we can determine. This figure shows three VLANs, with two hosts in
each of them. The router in
\protect\hyperlink{c11.xhtmlux5cux23figure11-13}{Figure 11.13} is
connected to the Fa0/1 switch port, and VLAN 4 is configured on port
F0/6.

When looking at this diagram, keep in mind that these three factors are
what Cisco expects you to know:

\begin{enumerate}
\tightlist
\item
  The router is connected to the switch using subinterfaces.
\item
  The switch port connecting to the router is a trunk port.
\item
  The switch ports connecting to the clients and the hub are access
  ports, not trunk ports.
\end{enumerate}

\begin{figure}
\centering
\includegraphics{images/c11f013.jpg}
\caption{{\protect\hyperlink{c11.xhtmlux5cux23figureanchor11-13}{\textbf{FIGURE
11.13}} Inter-VLAN example 2}}
\end{figure}

The configuration of the switch would look something like this:

\begin{verbatim}
2960#config t
2960(config)#int f0/1
2960(config-if)#switchport mode trunk
2960(config-if)#int f0/2
2960(config-if)#switchport access vlan 2
2960(config-if)#int f0/3
2960(config-if)#switchport access vlan 2
2960(config-if)#int f0/4
2960(config-if)#switchport access vlan 3
2960(config-if)#int f0/5
2960(config-if)#switchport access vlan 3
2960(config-if)#int f0/6
2960(config-if)#switchport access vlan 4
\end{verbatim}

Before we configure the router, we need to design our logical network:

\begin{enumerate}
\tightlist
\item
  \textbf{VLAN 1:} 192.168.10.0/28
\item
  \textbf{VLAN 2:} 192.168.10.16/28
\item
  \textbf{VLAN 3:} 192.168.10.32/28
\item
  \textbf{VLAN 4:} 192.168.10.48/28
\end{enumerate}

The configuration of the router would then look like this:

\begin{verbatim}
ISR#config t
ISR(config)#int fa0/0
ISR(config-if)#ip address 192.168.10.1 255.255.255.240
ISR(config-if)#no shutdown
ISR(config-if)#int f0/0.2
ISR(config-subif)#encapsulation dot1q 2
ISR(config-subif)#ip address 192.168.10.17 255.255.255.240
ISR(config-subif)#int f0/0.3
ISR(config-subif)#encapsulation dot1q 3
ISR(config-subif)#ip address 192.168.10.33 255.255.255.240
ISR(config-subif)#int f0/0.4
ISR(config-subif)#encapsulation dot1q 4
ISR(config-subif)#ip address 192.168.10.49 255.255.255.240
\end{verbatim}

Notice I didn't tag VLAN 1. Even though I could have created a
subinterface and tagged VLAN 1, it's not necessary with 802.1q because
untagged frames are members of the native VLAN.

The hosts in each VLAN would be assigned an address from their subnet
range, and the default gateway would be the IP address assigned to the
router's subinterface in that VLAN.

Now, let's take a look at another figure and see if you can determine
the switch and router configurations without looking at the answer---no
cheating! \protect\hyperlink{c11.xhtmlux5cux23figure11-14}{Figure 11.14}
shows a router connected to a 2960 switch with two VLANs. One host in
each VLAN is assigned
\protect\hypertarget{c11.xhtmlux5cux23Page_469}{}{}an IP address. What
would your router and switch configurations be based on these IP
addresses?

\begin{figure}
\centering
\includegraphics{images/c11f014.jpg}
\caption{{\protect\hyperlink{c11.xhtmlux5cux23figureanchor11-14}{\textbf{FIGURE
11.14}} Inter-VLAN example 3}}
\end{figure}

Since the hosts don't list a subnet mask, you have to look for the
number of hosts used in each VLAN to figure out the block size. VLAN 2
has 85 hosts and VLAN 3 has 115 hosts. Each of these will fit in a block
size of 128, which is a /25 mask, or 255.255.255.128.

You should know by now that the subnets are 0 and 128; the 0 subnet
(VLAN 2) has a host range of 1--126, and the 128 subnet (VLAN 3) has a
range of 129--254. You can almost be fooled since Host A has an IP
address of 126, which makes it \emph{almost} seem that Host A and B are
in the same subnet. But they're not, and you're way too smart by now to
be fooled by this one!

Here is the switch configuration:

\begin{verbatim}
2960#config t
2960(config)#int f0/1
2960(config-if)#switchport mode trunk
2960(config-if)#int f0/2
2960(config-if)#switchport access vlan 2
2960(config-if)#int f0/3
2960(config-if)#switchport access vlan 3
\end{verbatim}

Here is the router configuration:

\begin{verbatim}
ISR#config t
ISR(config) # int f0/0
ISR(config-if)#ip address 192.168.10.1 255.255.255.0
ISR(config-if)#no shutdown
ISR(config-if)#int f0/0.2
ISR(config-subif)#encapsulation dot1q 2
ISR(config-subif)#ip address 172.16.10.1 255.255.255.128
ISR(config-subif)#int f0/0.3
ISR(config-subif)#encapsulation dot1q 3
ISR(config-subif)#ip address 172.16.10.254 255.255.255.128
\end{verbatim}

I used the first address in the host range for VLAN 2 and the last
address in the range for VLAN 3, but any address in the range would
work. You would just have to configure the host's default gateway to
whatever you make the router's address. Also, I used a different subnet
for my physical interface, which is my management VLAN router's address.

Now, before we go on to the next example, I need to make sure you know
how to set the IP address on the switch. Since VLAN 1 is typically the
administrative VLAN, we'll use an IP address from out of that pool of
addresses. Here's how to set the IP address of the switch (not nagging,
but you really should already know this!):

\begin{verbatim}
2960#config t
2960(config)#int vlan 1
2960(config-if)#ip address 192.168.10.2 255.255.255.0
2960(config-if)#no shutdown
2960(config-if)#exit
2960(config)#ip default-gateway 192.168.10.1
\end{verbatim}

Yes, you have to execute a \texttt{no\ shutdown} on the VLAN interface
and set the \texttt{ip\ default-gateway} address to the router.

One more example, and then we'll move on to IVR using a multilayer
switch---another important subject that you definitely don't want to
miss! In \protect\hyperlink{c11.xhtmlux5cux23figure11-15}{Figure 11.15}
there are two VLANs, plus the management VLAN 1. By looking at the
router configuration, what's the IP address, subnet mask, and default
gateway of Host A? Use the last IP address in the range for Host A's
address.

If you really look carefully at the router configuration (the hostname
in this configuration is just Router), there's a simple and quick
answer. All subnets are using a /28, which is a 255.255.255.240 mask.
This is a block size of 16. The router's address for VLAN 2 is in subnet
128. The next subnet is 144, so the broadcast address of VLAN 2 is 143
and the valid host range is 129--142. So the host address would be this:

\begin{enumerate}
\tightlist
\item
  \textbf{IP address:} 192.168.10.142
\item
  \textbf{Mask:} 255.255.255.240
\item
  \textbf{Default gateway:} 192.168.10.129
\end{enumerate}

\protect\hypertarget{c11.xhtmlux5cux23Page_471}{}{}

\begin{figure}
\centering
\includegraphics{images/c11f015.jpg}
\caption{{\protect\hyperlink{c11.xhtmlux5cux23figureanchor11-15}{\textbf{FIGURE
11.15}} Inter-VLAN example 4}}
\end{figure}

This section was probably the hardest part of this entire book, and I
honestly created the simplest configuration you can possibly get away
with using to help you through it!

I'll use \protect\hyperlink{c11.xhtmlux5cux23figure11-16}{Figure 11.16}
to demonstrate configuring inter-VLAN routing (IVR) with a multi­layer
switch, which is often referred to as a switched virtual interface
(SVI). I'm going to use the same network that I used to discuss a
multilayer switch back in
\protect\hyperlink{c11.xhtmlux5cux23figure11-11}{Figure 11.11}, and I'll
use this IP address scheme: 192.168.\emph{x}.0/24, where \emph{x}
represents the VLAN subnet. In my example this will be the same as the
VLAN number.

\begin{figure}
\centering
\includegraphics{images/c11f016.jpg}
\caption{{\protect\hyperlink{c11.xhtmlux5cux23figureanchor11-16}{\textbf{FIGURE
11.16}} Inter-VLAN routing with a multilayer switch}}
\end{figure}

The hosts are already configured with the IP address, subnet mask, and
default gateway address using the first address in the range. Now I just
need to configure the routing on the switch, which is pretty simple
actually:

\begin{verbatim}
S1(config)#ip routing
S1(config)#int vlan 10
S1(config-if)#ip address 192.168.10.1 255.255.255.0
S1(config-if)#int vlan 20
S1(config-if)#ip address 192.168.20.1 255.255.255.0
\end{verbatim}

And that's it! Enable IP routing and create one logical interface for
each VLAN using the \texttt{interface\ vlan\ number} command and voilà!
You've now accomplished making inter-VLAN routing work on the backplane
of the switch!

\subsection[Summary]{\texorpdfstring{\protect\hypertarget{c11.xhtmlux5cux23c11-sec-15}{}{}Summary}{Summary}}

In this chapter, I introduced you to the world of virtual LANs and
described how Cisco switches can use them. We talked about how VLANs
break up broadcast domains in a switched internetwork---a very
important, necessary thing because layer 2 switches only break up
collision domains, and by default, all switches make up one large
broadcast domain. I also described access links to you, and we went over
how trunked VLANs work across a FastEthernet or faster link.

Trunking is a crucial technology to understand really well when you're
dealing with a network populated by multiple switches that are running
several VLANs.

You were also presented with some key troubleshooting and configuration
examples for access and trunk ports, configuring trunking options, and a
huge section on IVR.

\subsection[Exam
Essentials]{\texorpdfstring{\protect\hypertarget{c11.xhtmlux5cux23c11-sec-16}{}{}Exam
Essentials}{Exam Essentials}}

\textbf{Understand the term}\emph{frame tagging}. \emph{Frame tagging}
refers to VLAN identification; this is what switches use to keep track
of all those frames as they're traversing a switch fabric. It's how
switches identify which frames belong to which VLANs.

\textbf{Understand the 802.1q VLAN identification method.} This is a
nonproprietary IEEE method of frame tagging. If you're trunking between
a Cisco switched link and a different brand of switch, you have to use
802.1q for the trunk to work.

\textbf{Remember how to set a trunk port on a 2960 switch.} To set a
port to trunking on a 2960, use the \texttt{switchport\ mode\ trunk}
command.

\textbf{Remember to check a switch port's VLAN assignment when plugging
in a new host.} If you plug a new host into a switch, then you must
verify the VLAN membership of that port. If the membership is different
than what is needed for that host, the host will not be able to reach
the needed network services, such as a workgroup server or printer.

\textbf{Remember how to create a Cisco router on a stick to provide
inter-VLAN communication.} You can use a Cisco FastEthernet or Gigabit
Ethernet interface to provide inter-VLAN routing. The switch port
connected to the router must be a trunk port; then you must create
virtual \protect\hypertarget{c11.xhtmlux5cux23Page_473}{}{}interfaces
(subinterfaces) on the router port for each VLAN connecting to it. The
hosts in each VLAN will use this subinterface address as their default
gateway address.

\textbf{Remember how to provide inter-VLAN routing with a layer 3
switch.} You can use a layer 3 (multilayer) switch to provide IVR just
as with a router on a stick, but using a layer 3 switch is more
efficient and faster. First you start the routing process with the
command\texttt{ip\ routing}, then create a virtual interface for each
VLAN using the command \texttt{interface\ vlan\ vlan}, and then apply
the IP address for that VLAN under that logical interface.

\subsection[Written Lab
11]{\texorpdfstring{\protect\hypertarget{c11.xhtmlux5cux23c11-sec-17}{}{}Written
Lab 11}{Written Lab 11}}

In this section, you'll complete the following lab to make sure you've
got the information and concepts contained within them fully dialed in:

Lab 11.1: VLANs

You can find the answers to this lab in Appendix A, ``Answers to Written
Labs.''

Write the answers to the following questions:

\begin{enumerate}
\tightlist
\item
  True/False: To provide IVR with a layer 3 switch, you place an IP
  address on each interface of the switch.
\item
  What protocol will stop loops in a layer 2 switched network?
\item
  VLANs break up \_\_\_\_\_\_\_\_\_\_\_ domains in a layer 2 switched
  network.
\item
  Which VLAN numbers are reserved by default?
\item
  If you have a switch that provides both ISL and 802.1q frame tagging,
  what command under the trunk interface will make the trunk use 802.1q?
\item
  What does trunking provide?
\item
  How many VLANs can you create on an IOS switch by default?
\item
  True/False: The 802.1q encapsulation is removed from the frame if the
  frame is forwarded out an access link.
\item
  What type of link on a switch is a member of only one VLAN?
\item
  You want to change from the default of VLAN 1 to VLAN 4 for untagged
  traffic. What command will you use?
\end{enumerate}

\subsection[Hands-on
Labs]{\texorpdfstring{\protect\hypertarget{c11.xhtmlux5cux23c11-sec-18}{}{}Hands-on
Labs}{Hands-on Labs}}

In these labs, you will use three switches and a router. To perform the
last lab, you'll need a layer 3 switch.

\begin{enumerate}
\tightlist
\item
  Lab 11.1: Configuring and Verifying VLANs
\item
  Lab 11.2: Configuring and Verifying Trunk Links
\item
  \protect\hypertarget{c11.xhtmlux5cux23Page_474}{}{}Lab 11.3:
  Configuring Router on a Stick Routing
\item
  Lab 11.4: Configuring IVR with a Layer 3 Switch
\end{enumerate}

In these labs, I'll use the following layout:

\begin{figure}
\centering
\includegraphics{images/c11f017.jpg}
\caption{}
\end{figure}

\subsubsection[Hands-on Lab 11.1: Configuring and Verifying
VLANs]{\texorpdfstring{\protect\hypertarget{c11.xhtmlux5cux23c11-sec-19}{}{}Hands-on
Lab 11.1: Configuring and Verifying
VLANs}{Hands-on Lab 11.1: Configuring and Verifying VLANs}}

This lab will have you configure VLANs from global configuration mode
and then verify the VLANs.

\begin{enumerate}
\item
  Configure two VLANs on each switch, VLAN 10 and VLAN 20.

\begin{verbatim}
S1(config)#vlan 10
S1(config-vlan)#vlan 20
\end{verbatim}

\begin{verbatim}
S2(config)#vlan 10
S2(config-vlan)#vlan 20
\end{verbatim}

\begin{verbatim}
S3(config)#vlan 10
S3(config-vlan)#vlan 20
\end{verbatim}
\item
  Use the \texttt{show\ vlan} and \texttt{show\ vlan\ brief} commands to
  verify your VLANs. Notice that all interfaces are in VLAN 1 by
  default.

\begin{verbatim}
S1#sh vlan
S1#sh vlan brief
\end{verbatim}
\end{enumerate}

\subsubsection[Hands-on Lab 11.2: Configuring and Verifying Trunk
Links]{\texorpdfstring{\protect\hypertarget{c11.xhtmlux5cux23c11-sec-20}{}{}Hands-on
Lab 11.2: Configuring and Verifying Trunk
Links}{Hands-on Lab 11.2: Configuring and Verifying Trunk Links}}

This lab will have you configure trunk links and then verify them.

\begin{enumerate}
\item
  \protect\hypertarget{c11.xhtmlux5cux23Page_475}{}{}Connect to each
  switch and configure trunking on all switch links. If you are using a
  switch that supports both 802.1q and ISL frame tagging, then use the
  encapsulation command; if not, then skip that command.

\begin{verbatim}
S1#config t
S1(config)#interface fa0/15
S1(config-if)#switchport trunk encapsulation ?
  dot1q  Interface uses only 802.1q trunking encapsulation when trunking
  isl    Interface uses only ISL trunking encapsulation when trunking
  negotiate   Device will negotiate trunking encapsulation with peer on interface
\end{verbatim}

  Again, if you typed the previous and received an error, then your
  switch does not support both encapsulation methods:

\begin{verbatim}
S1 (config-if)#switchport trunk encapsulation dot1q
S1 (config-if)#switchport mode trunk
S1 (config-if)#interface fa0/16
S1 (config-if)#switchport trunk encapsulation dot1q
S1 (config-if)#switchport mode trunk
S1 (config-if)#interface fa0/17
S1 (config-if)#switchport trunk encapsulation dot1q
S1 (config-if)#switchport mode trunk
S1 (config-f)#interface fa0/18
S1 (config-if)#switchport trunk encapsulation dot1q
S1 (config-if)#switchport mode trunk
\end{verbatim}
\item
  Configure the trunk links on your other switches.
\item
  On each switch, verify your trunk ports with the
  \texttt{show\ interface\ trunk} command:

\begin{verbatim}
S1#show interface trunk
\end{verbatim}
\item
  Verify the switchport configuration with the following:

\begin{verbatim}
S1#show interface interface switchport
\end{verbatim}
\end{enumerate}

The second \texttt{interface}in the command is a variable, such as
Fa0/15.

\subsubsection[Hands-on Lab 11.3: Configuring Router on a Stick
Routing]{\texorpdfstring{\protect\hypertarget{c11.xhtmlux5cux23c11-sec-21}{}{}Hands-on
Lab 11.3: Configuring Router on a Stick
Routing}{Hands-on Lab 11.3: Configuring Router on a Stick Routing}}

In this lab, you'll use the router connected to port F0/8 of switch S1
to configure ROAS.

\begin{enumerate}
\item
  Configure the F0/0 of the router with two subinterfaces to provide
  inter-VLAN routing using 802.1q encapsulation. Use 172.16.10.0/24 for
  your management VLAN, 10.10.10.0/24 for VLAN 10, and 20.20.20.0/24 for
  VLAN 20.

\begin{verbatim}
Router#config t
Router (config)#int f0/0
Router (config-if)#ip address 172.16.10.1 255.255.255.0
Router (config-if)#interface f0/0.10
Router (config-subif)#encapsulation dot1q 10
Router (config-subif)#ip address 10.10.10.1 255.255.255.0
Router (config-subif)#interface f0/0.20
Router (config-subif)#encapsulation dot1q 20
Router (config-subif)#ip address 20.20.20.1 255.255.255.0
\end{verbatim}
\item
  Verify the configuration with the \texttt{show\ running-config}
  command.
\item
  Configure trunking on interface F0/8 of the S1 switch connecting to
  your router.
\item
  Verify that your VLANs are still configured on your switches with the
  \texttt{sh\ vlan} command.
\item
  Configure your hosts to be in VLAN 10 and VLAN 20 with the
  \texttt{switchport\ access\ vlan\ x} command.
\item
  Ping from your PC to the router's subinterface configured for your
  VLAN.
\item
  Ping from your PC to your PC in the other VLAN. You are now routing
  through the router!
\end{enumerate}

\subsubsection[Hands-on Lab 11.4: Configuring IVR with a Layer 3
Switch]{\texorpdfstring{\protect\hypertarget{c11.xhtmlux5cux23c11-sec-22}{}{}Hands-on
Lab 11.4: Configuring IVR with a Layer 3
Switch}{Hands-on Lab 11.4: Configuring IVR with a Layer 3 Switch}}

In this lab, you will disable the router and use the S1 switch to
provide inter-VLAN routing by creating SVI's.

\begin{enumerate}
\item
  Connect to the S1 switch and make interface F0/8 an access port, which
  will make the router stop providing inter-VLAN routing.
\item
  Enable IP routing on the S1 switch.

\begin{verbatim}
S1(config)#ip routing
\end{verbatim}
\item
  Create two new interfaces on the S1 switch to provide IVR.

\begin{verbatim}
S1(config)#interface vlan 10
S1(config-if)#ip address 10.10.10.1 255.255.255.0
S1(config-if)#interface vlan 20
S1(config-if)#ip address 20.20.20.1 255.255.255.0
\end{verbatim}
\item
  Clear the ARP cache on the switch and hosts.

\begin{verbatim}
S1#clear arp
\end{verbatim}
\item
  Ping from your PC to the router's subinterface configured for your
  VLAN.
\item
  Ping from your PC to your PC in the other VLAN. You are now routing
  through the S1 switch!
\end{enumerate}

\subsection[Review
Questions]{\texorpdfstring{\protect\hypertarget{c11.xhtmlux5cux23c11-sec-23}{}{}\protect\hypertarget{c11.xhtmlux5cux23Page_477}{}{}Review
Questions}{Review Questions}}

\begin{center}\rule{0.5\linewidth}{0.5pt}\end{center}

\includegraphics{images/note.png}The following questions are designed to
test your understanding of this chapter's material. For more information
on how to get additional questions, please
see\texttt{www.lammle.com/ccna}.

\begin{center}\rule{0.5\linewidth}{0.5pt}\end{center}

You can find the answers to these questions in Appendix B, ``Answers to
Review Questions.''

\begin{enumerate}
\item
  Which of the following statements is true with regard to VLANs?

  \begin{enumerate}
  \tightlist
  \item
    VLANs greatly reduce network security.
  \item
    VLANs increase the number of collision domains while decreasing
    their size.
  \item
    VLANs decrease the number of broadcast domains while decreasing
    their size.
  \item
    Network adds, moves, and changes are achieved with ease by just
    configuring a port into the appropriate VLAN.
  \end{enumerate}
\item
  Write the command that must be present for this layer 3 switch to
  provide inter-VLAN routing between the two VLANs created with these
  commands:

\begin{verbatim}
S1(config)#int vlan 10
S1(config-if)#ip address 192.168.10.1 255.255.255.0
S1(config-if)#int vlan 20
S1(config-if)#ip address 192.168.20.1 255.255.255.0
\end{verbatim}
\item
  In the following diagram, how must the port on each end of the line be
  configured to carry traffic between the four hosts?

  \begin{figure}
  \centering
  \includegraphics{images/c11f018.jpg}
  \caption{}
  \end{figure}

  \begin{enumerate}
  \tightlist
  \item
    \protect\hypertarget{c11.xhtmlux5cux23Page_478}{}{} Access port
  \item
    10 GB
  \item
    Trunk
  \item
    Spanning
  \end{enumerate}
\item
  What is the only type of \emph{second} VLAN of which an access port
  can be a member?

  \begin{enumerate}
  \tightlist
  \item
    Secondary
  \item
    Voice
  \item
    Primary
  \item
    Trunk
  \end{enumerate}
\item
  In the following configuration, what command is missing in the
  creation of the VLAN interface?

\begin{verbatim}
2960#config t
2960(config)#int vlan 1
2960(config-if)#ip address 192.168.10.2 255.255.255.0
2960(config-if)#exit
2960(config)#ip default-gateway 192.168.10.1
\end{verbatim}

  \begin{enumerate}
  \tightlist
  \item
    \texttt{no\ shutdown} under int vlan 1
  \item
    \texttt{encapsulation\ dot1q\ 1} under int vlan 1
  \item
    \texttt{switchport\ access\ vlan\ 1}
  \item
    \texttt{passive-interface}
  \end{enumerate}
\item
  Which of the following statements is true with regard to ISL and
  802.1q?

  \begin{enumerate}
  \tightlist
  \item
    802.1q encapsulates the frame with control information; ISL inserts
    an ISL field along with tag control information.
  \item
    802.1q is Cisco proprietary.
  \item
    ISL encapsulates the frame with control information; 802.1q inserts
    an 802.1q field along with tag control information.
  \item
    ISL is a standard.
  \end{enumerate}
\item
  What concept is depicted in the diagram?

  \begin{figure}
  \centering
  \includegraphics{images/c11f019.jpg}
  \caption{}
  \end{figure}

  \begin{enumerate}
  \tightlist
  \item
    \protect\hypertarget{c11.xhtmlux5cux23Page_479}{}{} Multiprotocol
    routing
  \item
    Passive interface
  \item
    Gateway redundancy
  \item
    Router on a stick
  \end{enumerate}
\item
  Write the command that places an interface into VLAN 2. Write only the
  command and not the prompt.
\item
  Write the command that generated the following output:

\begin{verbatim}
VLAN Name                       Status    Ports
---- ------------------------- --------- ------------------------
1    default                    active   Fa0/1, Fa0/2, Fa0/3, Fa0/4
                                         Fa0/5, Fa0/6, Fa0/7, Fa0/8
                                       Fa0/9, Fa0/10, Fa0/11, Fa0/12
                                      Fa0/13, Fa0/14, Fa0/19, Fa0/20
                                      Fa0/21, Fa0/22, Fa0/23, Gi0/1
                                          Gi0/2
2    Sales                      active
3    Marketing                  active
4    Accounting                 active
[output cut]
\end{verbatim}
\item
  In the configuration and diagram shown, what command is missing to
  enable inter-VLAN routing between VLAN 2 and VLAN 3?

  \begin{figure}
  \centering
  \includegraphics{images/c11f020.jpg}
  \caption{}
  \end{figure}

  \begin{enumerate}
  \tightlist
  \item
    \texttt{encapsulation\ dot1q\ 3} under int f0/0.2
  \item
    \texttt{encapsulation\ dot1q\ 2} under int f0/0.2
  \item
    \texttt{no\ shutdown} under int f0/0.2
  \item
    \texttt{no\ shutdown} under int f0/0.3
  \end{enumerate}
\item
  \protect\hypertarget{c11.xhtmlux5cux23Page_480}{}{} Based on the
  configuration shown here, what statement is true?

\begin{verbatim}
S1(config)#ip routing
S1(config)#int vlan 10
S1(config-if)#ip address 192.168.10.1 255.255.255.0
S1(config-if)#int vlan 20
S1(config-if)#ip address 192.168.20.1 255.255.255.0
\end{verbatim}

  \begin{enumerate}
  \tightlist
  \item
    This is a multilayer switch.
  \item
    The two VLANs are in the same subnet.
  \item
    Encapsulation must be configured.
  \item
    VLAN 10 is the management VLAN.
  \end{enumerate}
\item
  What is true of the output shown here?

\begin{verbatim}
S1#sh vlan
VLAN Name                    Status    Ports
---- ---------------------- --------- -------------------------------
1    default                 active    Fa0/1, Fa0/2, Fa0/3, Fa0/4
                                       Fa0/5, Fa0/6, Fa0/7, Fa0/8
                                       Fa0/9, Fa0/10, Fa0/11, Fa0/12
                                       Fa0/13, Fa0/14, Fa0/19, Fa0/20,
                                       Fa0/22, Fa0/23, Gi0/1, Gi0/2
2    Sales                   active
3    Marketing               active    Fa0/21
4    Accounting              active
[output cut]
\end{verbatim}

  \begin{enumerate}
  \tightlist
  \item
    Interface F0/15 is a trunk port.
  \item
    Interface F0/17 is an access port.
  \item
    Interface F0/21 is a trunk port.
  \item
    VLAN 1 was populated manually.
  \end{enumerate}
\item
  802.1q untagged frames are members of the \_\_\_\_\_\_\_\_\_\_ VLAN.

  \begin{enumerate}
  \tightlist
  \item
    Auxiliary
  \item
    Voice
  \item
    Native
  \item
    Private
  \end{enumerate}
\item
  \protect\hypertarget{c11.xhtmlux5cux23Page_481}{}{} Write the command
  that generated the following output. Write only the command and not
  the prompt:

\begin{verbatim}
Name: Fa0/15
Switchport: Enabled
Administrative Mode: dynamic desirable
Operational Mode: trunk
Administrative Trunking Encapsulation: negotiate
Operational Trunking Encapsulation: isl
Negotiation of Trunking: On
Access Mode VLAN: 1 (default)
Trunking Native Mode VLAN: 1 (default)
Administrative Native VLAN tagging: enabled
Voice VLAN: none
[output cut]
\end{verbatim}
\item
  In the switch output of question 12, how many broadcast domains are
  shown?

  \begin{enumerate}
  \tightlist
  \item
    1
  \item
    2
  \item
    4
  \item
    1001
  \end{enumerate}
\item
  In the diagram, what should be the default gateway address of Host B?

  \begin{figure}
  \centering
  \includegraphics{images/c11f021.jpg}
  \caption{}
  \end{figure}

  \begin{enumerate}
  \tightlist
  \item
    192.168.10.1
  \item
    192.168.1.65
  \item
    192.168.1.129
  \item
    192.168.1.2
  \end{enumerate}
\item
  \protect\hypertarget{c11.xhtmlux5cux23Page_482}{}{} What is the
  purpose of frame tagging in virtual LAN (VLAN) configurations?

  \begin{enumerate}
  \tightlist
  \item
    Inter-VLAN routing
  \item
    Encryption of network packets
  \item
    Frame identification over trunk links
  \item
    Frame identification over access links
  \end{enumerate}
\item
  Write the command to create VLAN 2 on a layer 2 switch. Write only the
  command and not the prompt.
\item
  Which statement is true regarding 802.1q frame tagging?

  \begin{enumerate}
  \tightlist
  \item
    802.1q adds a 26-byte trailer and 4-byte header.
  \item
    802.1q uses a native VLAN.
  \item
    The original Ethernet frame is not modified.
  \item
    802.1q only works with Cisco switches.
  \end{enumerate}
\item
  Write the command that prevents an interface from generating DTP
  frames. Write only the command and not the prompt.
\end{enumerate}

\protect\hypertarget{c12.xhtml}{}{}

\section[{Chapter 12}\\
{Security}]{\texorpdfstring{\protect\hypertarget{c12.xhtmlux5cux23c12}{}{}\protect\hypertarget{c12.xhtmlux5cux23Page_483}{}{}{Chapter
12}\\
{Security}}{Chapter 12 Security}}

\begin{center}\rule{0.5\linewidth}{0.5pt}\end{center}

\subsection{THE FOLLOWING ICND1 EXAM TOPICS ARE COVERED IN THIS
CHAPTER:}

\begin{enumerate}
\tightlist
\item
  \includegraphics{images/tick.png} \textbf{4.0 Infrastructure Services}

  \begin{enumerate}
  \tightlist
  \item
    \includegraphics{images/square.png} 4.6 Configure, verify, and
    troubleshoot IPv4 standard numbered and named access list for routed
    interfaces
  \end{enumerate}
\end{enumerate}

\protect\hypertarget{c12.xhtmlux5cux23Page_484}{}{}\includegraphics{images/intro.png}If
you're a sys admin, it's my guess that shielding sensitive, critical
data, as well as your network's resources, from every possible evil
exploit is a top priority of yours, right? Good to know you're on the
right page because Cisco has some really effective security solutions to
equip you with the tools you'll need to make this happen in a very real
way!

The first power tool I'm going to hand you is known as the access
control list (ACL). Being able to execute an ACL proficiently is an
integral part of Cisco's security solution, so I'm going to begin by
showing you how to create and implement simple ACLs. From there, I'll
move to demonstrating more advanced ACLs and describe how to implement
them strategically to provide serious armor for an internetwork in
today's challenging, high-risk environment.

In Appendix C, ``Disabling and Configuring Network Services,'' I'll show
you how to mitigate most security-oriented network threats. Make sure
you don't skip this appendix because it is chock full of great security
information, and the information it contains is part of the Cisco exam
objectives as well!

The proper use and configuration of access lists is a vital part of
router configuration because access lists are such versatile networking
accessories. Contributing mightily to the efficiency and operation of
your network, access lists give network managers a huge amount of
control over traffic flow throughout the enterprise. With access lists,
we can gather basic statistics on packet flow and security policies can
be implemented. These dynamic tools also enable us to protect sensitive
devices from the dangers of unauthorized access.

In this chapter, we'll cover ACLs for TCP/IP as well as explore
effective ways available to us for testing and monitoring how well
applied access lists are functioning. We'll begin now by discussing key
security measures deployed using hardware devices and VLANs and then
I'll introduce you to ACLs.

\begin{center}\rule{0.5\linewidth}{0.5pt}\end{center}

\includegraphics{images/note.png}To find up-to-the-minute updates for
this chapter, please see
\href{http://www.lammle.com/ccna}{www.lammle.com/ccna} or the book's web
page at \href{http://www.sybex.com/go/ccna}{www.sybex.com/go/ccna}.

\begin{center}\rule{0.5\linewidth}{0.5pt}\end{center}

\subsection[Perimeter, Firewall, and Internal
Routers]{\texorpdfstring{\protect\hypertarget{c12.xhtmlux5cux23c12-sec-1}{}{}Perimeter,
Firewall, and Internal
Routers}{Perimeter, Firewall, and Internal Routers}}

You see this a lot---typically, in medium to large enterprise
networks---the various strategies for security are based on some mix of
internal and perimeter routers plus firewall devices. Internal routers
provide additional security by screening traffic to various parts of
\protect\hypertarget{c12.xhtmlux5cux23Page_485}{}{}the protected
corporate network, and they achieve this using access lists. You can see
where each of these types of devices would be found in
\protect\hyperlink{c12.xhtmlux5cux23figure12-1}{Figure 12.1}.

\begin{figure}
\centering
\includegraphics{images/c12f001.jpg}
\caption{{\protect\hyperlink{c12.xhtmlux5cux23figureanchor12-1}{\textbf{FIGURE
12.1}} A typical secured network}}
\end{figure}

I'll use the terms \emph{trusted network} and \emph{untrusted network}
throughout this chapter, so it's important that you can see where
they're found in a typical secured network. The demilitarized zone (DMZ)
can be global (real) Internet addresses or private addresses, depending
on how you configure your firewall, but this is typically where you'll
find the HTTP, DNS, email, and other Internet-type corporate servers.

As you now know, instead of using routers, we can create VLANs with
switches on the inside trusted network. Multilayer switches containing
their own security features can sometimes replace internal (LAN) routers
to provide higher performance in VLAN architectures.

Let's look at some ways of protecting the internetwork using access
lists.

\subsection[Introduction to Access
Lists]{\texorpdfstring{\protect\hypertarget{c12.xhtmlux5cux23c12-sec-2}{}{}Introduction
to Access Lists}{Introduction to Access Lists}}

An \emph{access list} is essentially a list of conditions that
categorize packets, and they really come in handy when you need to
exercise control over network traffic. An ACL would be your tool of
choice for decision making in these situations.

One of the most common and easiest-to-understand uses of access lists is
to filter unwanted packets when implementing security policies. For
example, you can set them up to make very specific decisions about
regulating traffic patterns so that they'll allow only
\protect\hypertarget{c12.xhtmlux5cux23Page_486}{}{}certain hosts to
access web resources on the Internet while restricting others. With the
right combination of access lists, network managers arm themselves with
the power to enforce nearly any security policy they can invent.

Creating access lists is really a lot like programming a series of
if-then statements---if a given condition is met, then a given action is
taken. If the specific condition isn't met, nothing happens and the next
statement is evaluated. Access-list statements are basically packet
filters that packets are compared against, categorized by, and acted
upon accordingly. Once the lists are built, they can be applied to
either inbound or outbound traffic on any interface. Applying an access
list causes the router to analyze every packet crossing that interface
in the specified direction and take the appropriate action.

There are three important rules that a packet follows when it's being
compared with an access list:

\begin{enumerate}
\tightlist
\item
  The packet is always compared with each line of the access list in
  sequential order---it will always start with the first line of the
  access list, move on to line 2, then line 3, and so on.
\item
  The packet is compared with lines of the access list only until a
  match is made. Once it matches the condition on a line of the access
  list, the packet is acted upon and no further comparisons take place.
\item
  There is an implicit ``deny'' at the end of each access list---this
  means that if a packet doesn't match the condition on any of the lines
  in the access list, the packet will be discarded.
\end{enumerate}

Each of these rules has some powerful implications when filtering IP
packets with access lists, so keep in mind that creating effective
access lists definitely takes some practice.

There are two main types of access lists:

\textbf{Standard access lists} These ACLs use only the source IP address
in an IP packet as the condition test. All decisions are made based on
the source IP address. This means that standard access lists basically
permit or deny an entire suite of protocols. They don't distinguish
between any of the many types of IP traffic such as Web, Telnet, UDP,
and so on.

\textbf{Extended access lists} Extended access lists can evaluate many
of the other fields in the layer 3 and layer 4 headers of an IP packet.
They can evaluate source and destination IP addresses, the Protocol
field in the Network layer header, and the port number at the Transport
layer header. This gives extended access lists the ability to make much
more granular decisions when controlling traffic.

\textbf{Named access lists} Hey, wait a minute---I said there were only
two types of access lists but listed three! Well, technically there
really are only two since \emph{named access lists} are either standard
or extended and not actually a distinct type. I'm just distinguishing
them because they're created and referred to differently than standard
and extended access lists are, but they're still functionally the same.

\begin{center}\rule{0.5\linewidth}{0.5pt}\end{center}

\includegraphics{images/note.png}We'll cover these types of access lists
in more depth later in the chapter.

\begin{center}\rule{0.5\linewidth}{0.5pt}\end{center}

\protect\hypertarget{c12.xhtmlux5cux23Page_487}{}{}Once you create an
access list, it's not really going to do anything until you apply it.
Yes, they're there on the router, but they're inactive until you tell
that router what to do with them. To use an access list as a packet
filter, you need to apply it to an interface on the router where you
want the traffic filtered. And you've got to specify which direction of
traffic you want the access list applied to. There's a good reason for
this---you may want different controls in place for traffic leaving your
enterprise destined for the Internet than you'd want for traffic coming
into your enterprise from the Internet. So, by specifying the direction
of traffic, you can and must use different access lists for inbound and
outbound traffic on a single interface:

\textbf{Inbound access lists} When an access list is applied to inbound
packets on an interface, those packets are processed through the access
list before being routed to the outbound interface. Any packets that are
denied won't be routed because they're discarded before the routing
process is invoked.

\textbf{Outbound access lists} When an access list is applied to
outbound packets on an interface, packets are routed to the outbound
interface and then processed through the access list before being
queued.

There are some general access-list guidelines that you should keep in
mind when creating and implementing access lists on a router:

\begin{enumerate}
\tightlist
\item
  You can assign only one access list per interface per protocol per
  direction. This means that when applying IP access lists, you can have
  only one inbound access list and one outbound access list per
  interface.
\end{enumerate}

\begin{center}\rule{0.5\linewidth}{0.5pt}\end{center}

\includegraphics{images/note.png}When you consider the implications of
the implicit deny at the end of any access list, it makes sense that you
can't have multiple access lists applied on the same interface in the
same direction for the same protocol. That's because any packets that
don't match some condition in the first access list would be denied and
there wouldn't be any packets left over to compare against a second
access list!

\begin{center}\rule{0.5\linewidth}{0.5pt}\end{center}

\begin{enumerate}
\tightlist
\item
  Organize your access lists so that the more specific tests are at the
  top.
\item
  Anytime a new entry is added to the access list, it will be placed at
  the bottom of the list, which is why I highly recommend using a text
  editor for access lists.
\item
  You can't remove one line from an access list. If you try to do this,
  you will remove the entire list. This is why it's best to copy the
  access list to a text editor before trying to edit the list. The only
  exception is when you're using named access lists.
\end{enumerate}

\begin{center}\rule{0.5\linewidth}{0.5pt}\end{center}

\includegraphics{images/note.png}You can edit, add, or delete a single
line from a named access list. I'll show you how shortly.

\begin{center}\rule{0.5\linewidth}{0.5pt}\end{center}

\begin{enumerate}
\tightlist
\item
  Unless your access list ends with a \texttt{permit\ any} command, all
  packets will be discarded if they do not meet any of the list's tests.
  This means every list should have at least one \texttt{permit}
  statement or it will deny all traffic.
\item
  \protect\hypertarget{c12.xhtmlux5cux23Page_488}{}{}Create access lists
  and then apply them to an interface. Any access list applied to an
  interface without access-list test statements present will not filter
  traffic.
\item
  Access lists are designed to filter traffic going through the router.
  They will not filter traffic that has originated from the router.
\item
  Place IP standard access lists as close to the destination as
  possible. This is the reason we don't really want to use standard
  access lists in our networks. You can't put a standard access list
  close to the source host or network because you can only filter based
  on source address and all destinations would be affected as a result.
\item
  Place IP extended access lists as close to the source as possible.
  Since extended access lists can filter on very specific addresses and
  protocols, you don't want your traffic to traverse the entire network
  just to be denied. By placing this list as close to the source address
  as possible, you can filter traffic before it uses up precious
  bandwidth.
\end{enumerate}

Before I move on to demonstrate how to configure basic and extended
ACLs, let's talk about how they can be used to mitigate the security
threats I mentioned earlier.

\subsubsection[Mitigating Security Issues with
ACLs]{\texorpdfstring{\protect\hypertarget{c12.xhtmlux5cux23c12-sec-3}{}{}Mitigating
Security Issues with ACLs}{Mitigating Security Issues with ACLs}}

The most common attack is a denial of service (DoS) attack. Although
ACLs can help with a DoS, you really need an intrusion detection system
(IDS) and intrusion prevention system (IPS) to help prevent these common
attacks. Cisco sells the Adaptive Security Appliance (ASA), which has
IDS/IPS modules, but lots of other companies sell IDS/IPS products too.

Here's a list of the many security threats you can mitigate with ACLs:

\begin{enumerate}
\tightlist
\item
  IP address spoofing, inbound
\item
  IP address spoofing, outbound
\item
  Denial of service (DoS) TCP SYN attacks, blocking external attacks
\item
  DoS TCP SYN attacks, using TCP Intercept
\item
  DoS smurf attacks
\item
  Denying/filtering ICMP messages, inbound
\item
  Denying/filtering ICMP messages, outbound
\item
  Denying/filtering Traceroute
\end{enumerate}

\begin{center}\rule{0.5\linewidth}{0.5pt}\end{center}

\includegraphics{images/note.png}This is not an ``introduction to
security'' book, so you may have to research some of the preceding terms
if you don't understand them.

\begin{center}\rule{0.5\linewidth}{0.5pt}\end{center}

It's generally a bad idea to allow into a private network any external
IP packets that contain the source address of any internal hosts or
networks---just don't do it!

Here's a list of rules to live by when configuring ACLs from the
Internet to your production network to mitigate security problems:

\begin{enumerate}
\tightlist
\item
  Deny any source addresses from your internal networks.
\item
  Deny any local host addresses (127.0.0.0/8).
\item
  \protect\hypertarget{c12.xhtmlux5cux23Page_489}{}{}Deny any reserved
  private addresses (RFC 1918).
\item
  Deny any addresses in the IP multicast address range (224.0.0.0/4).
\end{enumerate}

None of these source addresses should be ever be allowed to enter your
internetwork. Now finally, let's get our hands dirty and configure some
basic and advanced access lists!

\subsection[Standard Access
Lists]{\texorpdfstring{\protect\hypertarget{c12.xhtmlux5cux23c12-sec-4}{}{}Standard
Access Lists}{Standard Access Lists}}

Standard IP access lists filter network traffic by examining the source
IP address in a packet. You create a \emph{standard IP access list} by
using the access-list numbers 1--99 or numbers in the expanded range of
1300--1999 because the type of ACL is generally differentiated using a
number. Based on the number used when the access list is created, the
router knows which type of syntax to expect as the list is entered. By
using numbers 1--99 or 1300--1999, you're telling the router that you
want to create a standard IP access list, so the router will expect
syntax specifying only the source IP address in the test lines.

The following output displays a good example of the many access-list
number ranges that you can use to filter traffic on your network. The
IOS version delimits the protocols you can specify access for:

\begin{verbatim}
Corp(config)#access-list ?
\end{verbatim}

\begin{verbatim}
 <1-99>            IP standard access list
 <100-199>         IP extended access list
 <1000-1099>       IPX SAP access list
 <1100-1199>       Extended 48-bit MAC address access list
 <1200-1299>       IPX summary address access list
 <1300-1999>       IP standard access list (expanded range)
 <200-299>         Protocol type-code access list
 <2000-2699>       IP extended access list (expanded range)
 <2700-2799>       MPLS access list
 <300-399>         DECnet access list
 <700-799>         48-bit MAC address access list
 <800-899>         IPX standard access list
 <900-999>         IPX extended access list
 dynamic-extended  Extend the dynamic ACL absolute timer
 rate-limit        Simple rate-limit specific access list
\end{verbatim}

Wow---there certainly are lot of old protocols listed in that output!
IPX and DECnet would no longer be used in any of today's networks. Let's
take a look at the syntax used when creating a standard IP access list:

\begin{verbatim}
Corp(config)#access-list 10 ?
  deny    Specify packets to reject
  permit  Specify packets to forward
  remark  Access list entry comment
\end{verbatim}

As I said, by using the access-list numbers 1--99 or 1300--1999, you're
telling the router that you want to create a standard IP access list,
which means you can only filter on source IP address.

Once you've chosen the access-list number, you need to decide whether
you're creating a \texttt{permit} or \texttt{deny} statement. I'm going
to create a \texttt{deny} statement now:

\begin{verbatim}
Corp(config)#access-list 10 deny ?
  Hostname or A.B.C.D  Address to match
  any                  Any source host
  host                 A single host address
\end{verbatim}

The next step is more detailed because there are three options available
in it:

\begin{enumerate}
\tightlist
\item
  The first option is the \texttt{any} parameter, which is used to
  permit or deny any source host or network.
\item
  The second choice is to use an IP address to specify either a single
  host or a range of them.
\item
  The last option is to use the \texttt{host} command to specify a
  specific host only.
\end{enumerate}

The \texttt{any} command is pretty obvious---any source address matches
the statement, so every packet compared against this line will match.
The \texttt{host} command is relatively simple too, as you can see here:

\begin{verbatim}
Corp(config)#access-list 10 deny host ?
  Hostname or A.B.C.D  Host address
Corp(config)#access-list 10 deny host 172.16.30.2
\end{verbatim}

This tells the list to deny any packets from host 172.16.30.2. The
default parameter is \texttt{host}. In other words, if you type
\texttt{access-list\ 10\ deny\ 172.16.30.2}, the router assumes you mean
host 172.16.30.2 and that's exactly how it will show in your
running-config.

But there's another way to specify either a particular host or a range
of hosts, and it's known as wildcard masking. In fact, to specify any
range of hosts, you must use wildcard masking in the access list.

So exactly what is wildcard masking? Coming up, I'm going to show you
using a standard access list example. I'll also guide you through how to
control access to a virtual terminal.

\subsubsection[Wildcard
Masking]{\texorpdfstring{\protect\hypertarget{c12.xhtmlux5cux23c12-sec-5}{}{}Wildcard
Masking}{Wildcard Masking}}

Wildcards are used with access lists to specify an individual host, a
network, or a specific range of a network or networks. The block sizes
you learned about earlier used to specify a range of addresses are key
to understanding wildcards.

\protect\hypertarget{c12.xhtmlux5cux23Page_491}{}{}Let me pause here for
a quick review of block sizes before we go any further. I'm sure you
remember that the different block sizes available are 64, 32, 16, 8, and
4. When you need to specify a range of addresses, you choose the
next-largest block size for your needs. So if you need to specify 34
networks, you need a block size of 64. If you want to specify 18 hosts,
you need a block size of 32. If you specify only 2 networks, then go
with a block size of 4.

Wildcards are used with the host or network address to tell the router a
range of available addresses to filter. To specify a host, the address
would look like this:

\begin{verbatim}
172.16.30.5 0.0.0.0
\end{verbatim}

The four zeros represent each octet of the address. Whenever a zero is
present, it indicates that the octet in the address must match the
corresponding reference octet exactly. To specify that an octet can be
any value, use the value 255. Here's an example of how a /24 subnet is
specified with a wildcard mask:

\begin{verbatim}
172.16.30.0 0.0.0.255
\end{verbatim}

This tells the router to match up the first three octets exactly, but
the fourth octet can be any value.

Okay---that was the easy part. But what if you want to specify only a
small range of subnets? This is where block sizes come in. You have to
specify the range of values in a block size, so you can't choose to
specify 20 networks. You can only specify the exact amount that the
block size value allows. This means that the range would have to be
either 16 or 32, but not 20.

Let's say that you want to block access to the part of the network that
ranges from 172.16.8.0 through 172.16.15.0. To do that, you would go
with a block size of 8, your network number would be 172.16.8.0, and the
wildcard would be 0.0.7.255. The 7.255 equals the value the router will
use to determine the block size. So together, the network number and the
wildcard tell the router to begin at 172.16.8.0 and go up a block size
of eight addresses to network 172.16.15.0.

This really is easier than it looks! I could certainly go through the
binary math for you, but no one needs that kind of pain because all you
have to do is remember that the wildcard is always one number less than
the block size. So, in our example, the wildcard would be 7 since our
block size is 8. If you used a block size of 16, the wildcard would be
15. Easy, right?

Just to make you've got this, we'll go through some examples that will
definitely help you nail it down. The following example tells the router
to match the first three octets exactly but that the fourth octet can be
anything:

\begin{verbatim}
Corp(config)#access-list 10 deny 172.16.10.0 0.0.0.255
\end{verbatim}

The next example tells the router to match the first two octets and that
the last two octets can be any value:

\begin{verbatim}
Corp(config)#access-list 10 deny 172.16.0.0 0.0.255.255
\end{verbatim}

\protect\hypertarget{c12.xhtmlux5cux23Page_492}{}{}Now, try to figure
out this next line:

\begin{verbatim}
Corp(config)#access-list 10 deny 172.16.16.0 0.0.3.255
\end{verbatim}

This configuration tells the router to start at network 172.16.16.0 and
use a block size of 4. The range would then be 172.16.16.0 through
172.16.19.255, and by the way, the Cisco objectives seem to really like
this one!

Let's keep practicing. What about this next one?

\begin{verbatim}
Corp(config)#access-list 10 deny 172.16.16.0 0.0.7.255
\end{verbatim}

This example reveals an access list starting at 172.16.16.0 going up a
block size of 8 to 172.16.23.255.

Let's keep at it\ldots{} What do you think the range of this one is?

\begin{verbatim}
Corp(config)#access-list 10 deny 172.16.32.0 0.0.15.255
\end{verbatim}

This one begins at network 172.16.32.0 and goes up a block size of 16 to
172.16.47.255.

You're almost done practicing! After a couple more, we'll configure some
real ACLs.

\begin{verbatim}
Corp(config)#access-list 10 deny 172.16.64.0 0.0.63.255
\end{verbatim}

This example starts at network 172.16.64.0 and goes up a block size of
64 to 172.16.127.255.

What about this last example?

\begin{verbatim}
Corp(config)#access-list 10 deny 192.168.160.0 0.0.31.255
\end{verbatim}

This one shows us that it begins at network 192.168.160.0 and goes up a
block size of 32 to 192.168.191.255.

Here are two more things to keep in mind when working with block sizes
and wildcards:

\begin{enumerate}
\tightlist
\item
  Each block size must start at 0 or a multiple of the block size. For
  example, you can't say that you want a block size of 8 and then start
  at 12. You must use 0--7, 8--15, 16--23, etc. For a block size of 32,
  the ranges are 0--31, 32--63, 64--95, etc.
\item
  The command \texttt{any} is the same thing as writing out the wildcard
  0.0.0.0 255.255.255.255.
\end{enumerate}

\begin{center}\rule{0.5\linewidth}{0.5pt}\end{center}

\includegraphics{images/note.png}Wildcard masking is a crucial skill to
master when creating IP access lists, and it's used identically when
creating standard and extended IP access lists.

\begin{center}\rule{0.5\linewidth}{0.5pt}\end{center}

\subsubsection[Standard Access List
Example]{\texorpdfstring{\protect\hypertarget{c12.xhtmlux5cux23c12-sec-6}{}{}Standard
Access List Example}{Standard Access List Example}}

In this section, you'll learn how to use a standard access list to stop
specific users from gaining access to the Finance department LAN.

\protect\hypertarget{c12.xhtmlux5cux23Page_493}{}{}In
\protect\hyperlink{c12.xhtmlux5cux23figure12-2}{Figure 12.2}, a router
has three LAN connections and one WAN connection to the Internet. Users
on the Sales LAN should not have access to the Finance LAN, but they
should be able to access the Internet and the marketing department
files. The Marketing LAN needs to access the Finance LAN for application
services.

\begin{figure}
\centering
\includegraphics{images/c12f002.jpg}
\caption{{\protect\hyperlink{c12.xhtmlux5cux23figureanchor12-2}{\textbf{FIGURE
12.2}} IP access list example with three LANs and a WAN connection}}
\end{figure}

We can see that the following standard IP access list is configured on
the router:

\begin{verbatim}
Lab_A#config t
Lab_A(config)#access-list 10 deny 172.16.40.0 0.0.0.255
Lab_A(config)#access-list 10 permit any
\end{verbatim}

It's very important to remember that the \texttt{any} command is the
same thing as saying the following using wildcard masking:

\begin{verbatim}
Lab_A(config)#access-list 10 permit 0.0.0.0 255.255.255.255
\end{verbatim}

Since the wildcard mask says that none of the octets are to be
evaluated, every address matches the test condition, so this is
functionally doing the same as using the \texttt{any} keyword.

At this point, the access list is configured to deny source addresses
from the Sales LAN to the Finance LAN and to allow everyone else. But
remember, no action will be taken until the access list is applied on an
interface in a specific direction!

But where should this access list be placed? If you place it as an
incoming access list on Fa0/0, you might as well shut down the
FastEthernet interface because all of the Sales LAN devices will be
denied access to all networks attached to the router. The best place to
apply this access list is on the Fa0/1 interface as an outbound list:

\begin{verbatim}
Lab_A(config)#int fa0/1
Lab_A(config-if)#ip access-group 10 out
\end{verbatim}

\protect\hypertarget{c12.xhtmlux5cux23Page_494}{}{}Doing this completely
stops traffic from 172.16.40.0 from getting out FastEthernet0/1. It has
no effect on the hosts from the Sales LAN accessing the Marketing LAN
and the Internet because traffic to those destinations doesn't go
through interface Fa0/1. Any packet trying to exit out Fa0/1 will have
to go through the access list first. If there were an inbound list
placed on F0/0, then any packet trying to enter interface F0/0 would
have to go through the access list before being routed to an exit
interface.

Now, let's take a look at another standard access list example.
\protect\hyperlink{c12.xhtmlux5cux23figure12-3}{Figure 12.3} shows an
internetwork of two routers with four LANs.

\begin{figure}
\centering
\includegraphics{images/c12f003.jpg}
\caption{{\protect\hyperlink{c12.xhtmlux5cux23figureanchor12-3}{\textbf{FIGURE
12.3}} IP standard access list example 2}}
\end{figure}

Now we're going to stop the Accounting users from accessing the Human
Resources server attached to the Lab\_B router but allow all other users
access to that LAN using a standard ACL. What kind of standard access
list would we need to create and where would we place it to achieve our
goals?

The real answer is that we should use an extended access list and place
it closest to the source! But this question specifies using a standard
access list, and as a rule, standard ACLs are placed closest to the
destination. In this example, Ethernet 0 is the outbound interface on
the Lab\_B router and here's the access list that should be placed on
it:

\begin{verbatim}
Lab_B#config t
Lab_B(config)#access-list 10 deny 192.168.10.128 0.0.0.31
Lab_B(config)#access-list 10 permit any
Lab_B(config)#interface Ethernet 0
Lab_B(config-if)#ip access-group 10 out
\end{verbatim}

Keep in mind that to be able to answer this question correctly, you
really need to understand subnetting, wildcard masks, and how to
configure and implement ACLs. The
\protect\hypertarget{c12.xhtmlux5cux23Page_495}{}{}accounting subnet is
the 192.168.10.128/27, which is a 255.255.255.224, with a block size of
32 in the fourth octet.

With all this in mind and before we move on to restricting Telnet access
on a router, let's take a look at one more standard access list example.
This one is going to require some thought. In
\protect\hyperlink{c12.xhtmlux5cux23figure12-4}{Figure 12.4}, you have a
router with four LAN connections and one WAN connection to the Internet.

\begin{figure}
\centering
\includegraphics{images/c12f004.jpg}
\caption{{\protect\hyperlink{c12.xhtmlux5cux23figureanchor12-4}{\textbf{FIGURE
12.4}} IP standard access list example 3}}
\end{figure}

Okay---you need to write an access list that will stop access from each
of the four LANs shown in the diagram to the Internet. Each of the LANs
reveals a single host's IP address, which you need to use to determine
the subnet and wildcards of each LAN to configure the access list.

Here is an example of what your answer should look like, beginning with
the network on E0 and working through to E3:

\begin{verbatim}
Router(config)#access-list 1 deny 172.16.128.0 0.0.31.255
Router(config)#access-list 1 deny 172.16.48.0 0.0.15.255
Router(config)#access-list 1 deny 172.16.192.0 0.0.63.255
Router(config)#access-list 1 deny 172.16.88.0 0.0.7.255
Router(config)#access-list 1 permit any
Router(config)#interface serial 0
Router(config-if)#ip access-group 1 out
\end{verbatim}

Sure, you could have done this with one line:

\begin{verbatim}
Router(config)#access-list 1 deny 172.16.0.0 0.0.255.255
\end{verbatim}

But what fun is that?

\protect\hypertarget{c12.xhtmlux5cux23Page_496}{}{}And remember the
reasons for creating this list. If you actually applied this ACL on the
router, you'd effectively shut down access to the Internet, so why even
have an Internet connection? I included this exercise so you can
practice how to use block sizes with access lists, which is vital for
succeeding when you take the Cisco exam!

\subsubsection[Controlling VTY (Telnet/SSH)
Access]{\texorpdfstring{\protect\hypertarget{c12.xhtmlux5cux23c12-sec-7}{}{}Controlling
VTY (Telnet/SSH) Access}{Controlling VTY (Telnet/SSH) Access}}

Trying to stop users from telnetting or trying to SSH to a router is
really challenging because any active interface on a router is fair game
for VTY/SSH access. Creating an extended IP ACL that limits access to
every IP address on the router may sound like a solution, but if you did
that, you'd have to apply it inbound on every interface, which really
wouldn't scale well if you happen to have dozens, even hundreds, of
interfaces, now would it? And think of all the latency dragging down
your network as a result of each and every router checking every packet
just in case the packet was trying to access your VTY lines---horrible!

Don't give up---there's always a solution! And in this case, a much
better one, which employs a standard IP access list to control access to
the VTY lines themselves.

Why does this work so well? Because when you apply an access list to the
VTY lines, you don't need to specify the protocol since access to the
VTY already implies terminal access via the Telnet or SSH protocols. You
also don't need to specify a destination address because it really
doesn't matter which interface address the user used as a target for the
Telnet session. All you really need control of is where the user is
coming from, which is betrayed by their source IP address.

You need to do these two things to make this happen:

\begin{enumerate}
\tightlist
\item
  Create a standard IP access list that permits only the host or hosts
  you want to be able to telnet into the routers.
\item
  Apply the access list to the VTY line with the
  \texttt{access-class\ in} command.
\end{enumerate}

Here, I'm allowing only host 172.16.10.3 to telnet into a router:

\begin{verbatim}
Lab_A(config)#access-list 50 permit host 172.16.10.3
Lab_A(config)#line vty 0 4
Lab_A(config-line)#access-class 50 in
\end{verbatim}

Because of the implied \texttt{deny\ any} at the end of the list, the
ACL stops any host from telnetting into the router except the host
172.16.10.3, regardless of the individual IP address on the router being
used as a target. It's a good idea to include an admin subnet address as
the source instead of a single host, but the reason I demonstrated this
was to show you how to create security on your VTY lines without adding
latency to your router.

\begin{center}\rule{0.5\linewidth}{0.5pt}\end{center}

\protect\hypertarget{c12.xhtmlux5cux23Page_497}{}{}\includegraphics{images/globe1.png}\\
\textbf{Should You Secure Your VTY Lines on a Router?}

You're monitoring your network and notice that someone has telnetted
into your core router by using the \texttt{show\ users} command. You use
the \texttt{disconnect} command and they're disconnected from the
router, but you notice that they're right back in there a few minutes
later. You consider putting an ACL on the router interfaces, but you
don't want to add latency on each interface since your router is already
pushing a lot of packets. At this point, you think about putting an
access list on the VTY lines themselves, but not having done this
before, you're not sure if this is a safe alternative to putting an ACL
on each interface. Would placing an ACL on the VTY lines be a good idea
for this network?

Yes---absolutely! And the \texttt{access-class} command covered in this
chapter is the way to do it. Why? Because it doesn't use an access list
that just sits on an interface looking at every packet, resulting in
unnecessary overhead and latency.

When you put the \texttt{access-class\ in} command on the VTY lines,
only packets trying to telnet into the router will be checked and
compared, providing easy-to-configure yet solid security for your
router!

\begin{center}\rule{0.5\linewidth}{0.5pt}\end{center}

\begin{center}\rule{0.5\linewidth}{0.5pt}\end{center}

\includegraphics{images/tip.png} Just a reminder---Cisco recommends
using Secure Shell (SSH) instead of Telnet on the VTY lines of a router,
as we covered in Chapter 6, ``Cisco's Internetworking Operating System
(IOS),'' so review that chapter if you need a refresher on SSH and how
to configure it on your routers and switches.

\begin{center}\rule{0.5\linewidth}{0.5pt}\end{center}

\subsection[Extended Access
Lists]{\texorpdfstring{\protect\hypertarget{c12.xhtmlux5cux23c12-sec-8}{}{}Extended
Access Lists}{Extended Access Lists}}

Let's go back to the standard IP access list example where you had to
block all access from the Sales LAN to the finance department and add a
new requirement. You now must allow Sales to gain access to a certain
server on the Finance LAN but not to other network services for security
reasons. What's the solution? Applying a standard IP access list won't
allow users to get to one network service but not another because a
standard ACL won't allow you to make decisions based on both source and
destination addresses. It makes decisions based only on source address,
so we need another way to achieve our new goal---but what is it?

Using an \emph{extended access list} will save the day because extended
ACLs allow us to specify source and destination addresses as well as the
protocol and port number that identify the upper-layer protocol or
application. An extended ACL is just what we need to affectively allow
users access to a physical LAN while denying them access to specific
hosts---even specific services on those hosts!

\begin{center}\rule{0.5\linewidth}{0.5pt}\end{center}

\protect\hypertarget{c12.xhtmlux5cux23Page_498}{}{}\includegraphics{images/note.png}Yes,
I am well aware there are no ICND1 objectives for extended access lists,
but you need to understand Extended ACL's for when you get to ICND2
troubleshooting, so I added foundation here.

\begin{center}\rule{0.5\linewidth}{0.5pt}\end{center}

We're going to take a look at the commands we have in our arsenal, but
first, you need to know that you must use the extended access-list range
from 100 to 199. The 2000--2699 range is also available for extended IP
access lists.

After choosing a number in the extended range, you need to decide what
type of list entry to make. For this example, I'm going with a
\texttt{deny} list entry:

\begin{verbatim}
Corp(config)#access-list 110 ?
  deny     Specify packets to reject
  dynamic  Specify a DYNAMIC list of PERMITs or DENYs
  permit   Specify packets to forward
  remark   Access list entry comment
\end{verbatim}

And once you've settled on the type of ACL, you then need to select a
protocol field entry:

\begin{verbatim}
Corp(config)#access-list 110 deny ?
  <0-255>  An IP protocol number
  ahp      Authentication Header Protocol
  eigrp    Cisco's EIGRP routing protocol
  esp      Encapsulation Security Payload
  gre      Cisco's GRE tunneling
  icmp     Internet Control Message Protocol
  igmp     Internet Gateway Message Protocol
  ip       Any Internet Protocol
  ipinip   IP in IP tunneling
  nos      KA9Q NOS compatible IP over IP tunneling
  ospf     OSPF routing protocol
  pcp      Payload Compression Protocol
  pim      Protocol Independent Multicast
  tcp      Transmission Control Protocol
  udp      User Datagram Protocol
\end{verbatim}

\begin{center}\rule{0.5\linewidth}{0.5pt}\end{center}

\includegraphics{images/note.png}If you want to filter by Application
layer protocol, you have to choose the appropriate layer 4 transport
protocol after the \texttt{permit} or \texttt{deny} statement. For
example, to filter Telnet or FTP, choose TCP since both Telnet and FTP
use TCP at the Transport layer. Selecting IP wouldn't allow you to
specify a particular application protocol later and only filter based on
source and destination addresses.

\begin{center}\rule{0.5\linewidth}{0.5pt}\end{center}

\protect\hypertarget{c12.xhtmlux5cux23Page_499}{}{}So now, let's filter
an Application layer protocol that uses TCP by selecting TCP as the
protocol and indicating the specific destination TCP port at the end of
the line. Next, we'll be prompted for the source IP address of the host
or network and we'll choose the \texttt{any} command to allow any source
address:

\begin{verbatim}
Corp(config)#access-list 110 deny tcp ?
  A.B.C.D  Source address
  any      Any source host
  host     A single source host
\end{verbatim}

After we've selected the source address, we can then choose the specific
destination address:

\begin{verbatim}
Corp(config)#access-list 110 deny tcp any ?
  A.B.C.D  Destination address
  any      Any destination host
  eq       Match only packets on a given port number
  gt       Match only packets with a greater port number
  host     A single destination host
  lt       Match only packets with a lower port number
  neq      Match only packets not on a given port number
  range    Match only packets in the range of port numbers
\end{verbatim}

In this output, you can see that any source IP address that has a
destination IP address of 172.16.30.2 has been denied:

\begin{verbatim}
Corp(config)#access-list 110 deny tcp any host 172.16.30.2 ?
  ack          Match on the ACK bit
  dscp         Match packets with given dscp value
  eq           Match only packets on a given port number
  established  Match established connections
  fin          Match on the FIN bit
  fragments    Check non-initial fragments
  gt           Match only packets with a greater port number
  log          Log matches against this entry
  log-input    Log matches against this entry, including input interface
  lt           Match only packets with a lower port number
  neq          Match only packets not on a given port number
  precedence   Match packets with given precedence value
  psh          Match on the PSH bit
  range        Match only packets in the range of port numbers
  rst          Match on the RST bit
  syn          Match on the SYN bit
  time-range   Specify a time-range
  tos          Match packets with given TOS value
  urg          Match on the URG bit
  <cr>
\end{verbatim}

And once we have the destination host addresses in place, we just need
to specify the type of service to deny using the \texttt{equal\ to}
command, entered as \texttt{eq}. The following help screen reveals the
options available now. You can choose a port number or use the
application name:

\begin{verbatim}
Corp(config)#access-list 110 deny tcp any host 172.16.30.2 eq ?
  <0-65535>    Port number
  bgp          Border Gateway Protocol (179)
  chargen      Character generator (19)
  cmd          Remote commands (rcmd, 514)
  daytime      Daytime (13)
  discard      Discard (9)
  domain       Domain Name Service (53)
  drip         Dynamic Routing Information Protocol (3949)
  echo         Echo (7)
  exec         Exec (rsh, 512)
  finger       Finger (79)
  ftp          File Transfer Protocol (21)
  ftp-data     FTP data connections (20)
  gopher       Gopher (70)
  hostname     NIC hostname server (101)
  ident        Ident Protocol (113)
  irc          Internet Relay Chat (194)
  klogin       Kerberos login (543)
  kshell       Kerberos shell (544)
  login        Login (rlogin, 513)
  lpd          Printer service (515)
  nntp         Network News Transport Protocol (119)
  pim-auto-rp  PIM Auto-RP (496)
  pop2         Post Office Protocol v2 (109)
  pop3         Post Office Protocol v3 (110)
  smtp         Simple Mail Transport Protocol (25)
  sunrpc       Sun Remote Procedure Call (111)
  syslog       Syslog (514)
  tacacs       TAC Access Control System (49)
  talk         Talk (517)
  telnet       Telnet (23)
  time         Time (37)
  uucp         Unix-to-Unix Copy Program (540)
  whois        Nicname (43)
  www          World Wide Web (HTTP, 80)
\end{verbatim}

Now let's block Telnet (port 23) to host 172.16.30.2 only. If the users
want to use FTP, fine---that's allowed. The \texttt{log} command is used
to log messages every time the access list entry is hit. This can be an
extremely cool way to monitor inappropriate access attempts, but be
careful because in a large network, this command can overload your
console's screen with messages!

Here's our result:

\begin{verbatim}
Corp(config)#access-list 110 deny tcp any host 172.16.30.2 eq 23 log
\end{verbatim}

This line says to deny any source host trying to telnet to destination
host 172.16.30.2. Keep in mind that the next line is an implicit
\texttt{deny} by default. If you apply this access list to an interface,
you might as well just shut the interface down because by default,
there's an implicit \texttt{deny\ all} at the end of every access list.
So we've got to follow up the access list with the following command:

\begin{verbatim}
Corp(config)#access-list 110 permit ip any any
\end{verbatim}

The IP in this line is important because it will permit the IP stack. If
TCP was used instead of IP in this line, then UDP, etc. would all be
denied. Remember, the \texttt{0.0.0.0\ 255.255.255.255} is the same
command as \texttt{any}, so the command could also look like this:

\begin{verbatim}
Corp(config)#access-list 110 permit ip 0.0.0.0 255.255.255.255
0.0.0.0 255.255.255.255
\end{verbatim}

But if you did this, when you looked at the running-config, the commands
would be replaced with the \texttt{any\ any}. I like efficiency so I'll
just use the \texttt{any} command because it requires less typing.

As always, once our access list is created, we must apply it to an
interface with the same command used for the IP standard list:

\begin{verbatim}
Corp(config-if)#ip access-group 110 in
\end{verbatim}

Or this:

\begin{verbatim}
Corp(config-if)#ip access-group 110 out
\end{verbatim}

Next, we'll check out some examples of how to use an extended access
list.

\subsubsection[Extended Access List Example
1]{\texorpdfstring{\protect\hypertarget{c12.xhtmlux5cux23c12-sec-9}{}{}Extended
Access List Example 1}{Extended Access List Example 1}}

For our first scenario, we'll use
\protect\hyperlink{c12.xhtmlux5cux23figure12-5}{Figure 12.5}. What do we
need to do to deny access to a host at 172.16.50.5 on the finance
department LAN for both Telnet and FTP services? All other services on
this and all other hosts are acceptable for the sales and marketing
departments to access.

\protect\hypertarget{c12.xhtmlux5cux23Page_502}{}{}

\begin{figure}
\centering
\includegraphics{images/c12f005.jpg}
\caption{{\protect\hyperlink{c12.xhtmlux5cux23figureanchor12-5}{\textbf{FIGURE
12.5}} Extended ACL example 1}}
\end{figure}

Here's the ACL we must create:

\begin{verbatim}
Lab_A#config t
Lab_A(config)#access-list 110 deny tcp any host 172.16.50.5 eq 21
Lab_A(config)#access-list 110 deny tcp any host 172.16.50.5 eq 23
Lab_A(config)#access-list 110 permit ip any any
\end{verbatim}

The \texttt{access-list\ 110} tells the router we're creating an
extended IP ACL. The \texttt{tcp} is the protocol field in the Network
layer header. If the list doesn't say \texttt{tcp} here, you cannot
filter by TCP port numbers 21 and 23 as shown in the example. Remember
that these values indicate FTP and Telnet, which both use TCP for
connection-oriented services. The \texttt{any} command is the source,
which means any source IP address, and the \texttt{host} is the
destination IP address. This ACL says that all IP traffic will be
permitted from any host except FTP and Telnet to host 172.16.50.5 from
any source.

\begin{center}\rule{0.5\linewidth}{0.5pt}\end{center}

\includegraphics{images/note.png}Remember that instead of the
\texttt{host\ 172.16.50.5} command when we created the extended access
list, we could have entered \texttt{172.16.50.5\ 0.0.0.0}. There would
be no difference in the result other than the router would change the
command to \texttt{host\ 172.16.50.5} in the running-config.

\begin{center}\rule{0.5\linewidth}{0.5pt}\end{center}

After the list is created, it must be applied to the FastEthernet 0/1
interface outbound because we want to block all traffic from getting to
host 172.16.50.5 and performing FTP and Telnet. If this list was created
to block access only from the Sales LAN to host 172.16.50.5, then we'd
have put this list closer to the source, or on FastEthernet 0/0. In that
situation, we'd apply the list to inbound traffic. This highlights the
fact that you really need to analyze each situation carefully before
creating and applying ACLs!

\protect\hypertarget{c12.xhtmlux5cux23Page_503}{}{}Now let's go ahead
and apply the list to interface Fa0/1 to block all outside FTP and
Telnet access to the host 172.16.50.5:

\begin{verbatim}
Lab_A(config)#int fa0/1
Lab_A(config-if)#ip access-group 110 out
\end{verbatim}

\subsubsection[Extended Access List Example
2]{\texorpdfstring{\protect\hypertarget{c12.xhtmlux5cux23c12-sec-10}{}{}Extended
Access List Example 2}{Extended Access List Example 2}}

We're going to use
\protect\hyperlink{c12.xhtmlux5cux23figure12-4}{Figure 12.4} again,
which has four LANs and a serial connection. We need to prevent Telnet
access to the networks attached to the E1 and E2 interfaces.

The configuration on the router would look something like this, although
the answer can vary:

\begin{verbatim}
Router(config)#access-list 110 deny tcp any 172.16.48.0 0.0.15.255
eq 23
Router(config)#access-list 110 deny tcp any 172.16.192.0 0.0.63.255
eq 23
Router(config)#access-list 110 permit ip any any
Router(config)#interface Ethernet 1
Router(config-if)#ip access-group 110 out
Router(config-if)#interface Ethernet 2
Router(config-if)#ip access-group 110 out
\end{verbatim}

Here are the key factors to understand from this list:

\begin{enumerate}
\tightlist
\item
  First, you need to verify that the number range is correct for the
  type of access list you are creating. In this example, it's extended,
  so the range must be 100--199.
\item
  Second, you must verify that the protocol field matches the
  upper-layer process or application, which in this case, is TCP port 23
  (Telnet).
\end{enumerate}

\begin{center}\rule{0.5\linewidth}{0.5pt}\end{center}

\includegraphics{images/tip.png}The protocol parameter must be TCP since
Telnet uses TCP. If it were TFTP instead, then the protocol parameter
would have to be UDP because TFTP uses UDP at the Transport layer.

\begin{center}\rule{0.5\linewidth}{0.5pt}\end{center}

\begin{enumerate}
\tightlist
\item
  Third, verify that the destination port number matches the application
  you're filtering for. In this case, port 23 matches Telnet, which is
  correct, but know that you can also type \texttt{telnet} at the end of
  the line instead of 23.
\item
  Finally, the test statement \texttt{permit\ ip\ any\ any} is important
  to have there at the end of the list because it means to enable all
  packets other than Telnet packets destined for the LANs connected to
  Ethernet 1 and Ethernet 2.
\end{enumerate}

\subsubsection[Extended Access List Example
3]{\texorpdfstring{\protect\hypertarget{c12.xhtmlux5cux23c12-sec-11}{}{}\protect\hypertarget{c12.xhtmlux5cux23Page_504}{}{}Extended
Access List Example 3}{Extended Access List Example 3}}

I want to guide you through one more extended ACL example before we move
on to named ACLs. \protect\hyperlink{c12.xhtmlux5cux23figure12-6}{Figure
12.6} displays the network we're going to use for this last scenario.

\begin{figure}
\centering
\includegraphics{images/c12f006.jpg}
\caption{{\protect\hyperlink{c12.xhtmlux5cux23figureanchor12-6}{\textbf{FIGURE
12.6}} Extended ACL example 3}}
\end{figure}

In this example, we're going to allow HTTP access to the Finance server
from source Host B only. All other traffic will be permitted. We need to
be able to configure this in only three test statements, and then we'll
need to add the interface configuration.

Let's take what we've learned and knock this one out:

\begin{verbatim}
Lab_A#config t
Lab_A(config)#access-list 110 permit tcp host 192.168.177.2 host 172.22.89.26 eq 80
Lab_A(config)#access-list 110 deny tcp any host 172.22.89.26 eq 80
Lab_A(config)#access-list 110 permit ip any any
\end{verbatim}

This is really pretty simple! First we need to permit Host B HTTP access
to the Finance server. But since all other traffic must be allowed, we
must detail who cannot HTTP to the Finance server, so the second test
statement is there to deny anyone else from using HTTP on the Finance
server. Finally, now that Host B can HTTP to the Finance server and
everyone else can't, we'll permit all other traffic with our third test
statement.

Not so bad---this just takes a little thought! But wait---we're not done
yet because we still need to apply this to an interface. Since extended
access lists are typically applied closest to the source, we should
simply place this inbound on F0/0, right? Well, this is one time we're
not going to follow the rules. Our challenge required us to allow only
HTTP traffic \protect\hypertarget{c12.xhtmlux5cux23Page_505}{}{}to the
Finance server from Host B. If we apply the ACL inbound on Fa0/0, then
the branch office would be able to access the Finance server and perform
HTTP. So in this example, we need to place the ACL closest to the
destination:

\begin{verbatim}
Lab_A(config)#interface fastethernet 0/1
Lab_A(config-if)#ip access-group 110 out
\end{verbatim}

Perfect! Now let's get into how to create ACLs using names.

\paragraph{Named ACLs}

As I said earlier, \emph{named} access lists are just another way to
create standard and extended access lists. In medium to large
enterprises, managing ACLs can become a real hassle over time! A handy
way to make things easier is to copy the access list to a text editor,
edit the list, then paste the new list back into the router, which works
pretty well if it weren't for the ``pack rat'' mentality. It's really
common to think things like, ``What if I find a problem with the new
list and need to back out of the change?'' This and other factors cause
people to hoard unapplied ACLs, and over time, they can seriously build
up on a router, leading to more questions, like, ``What were these ACLs
for? Are they important? Do I need them?'' All good questions, and named
access lists are the answer to this problem!

And of course, this kind of thing can also apply to access lists that
are up and running. Let's say you come into an existing network and are
looking at access lists on a router. Suppose you find an access list
177, which happens to be an extended access list that's a whopping 93
lines long. This leads to more of the same bunch of questions and can
even lead to needless existential despair! Instead, wouldn't it be a
whole lot easier to identify an access with a name like ``FinanceLAN''
rather than one mysteriously dubbed ``177''?

To our collective relief, named access lists allow us to use names for
creating and applying either standard or extended access lists. There's
really nothing new or different about these ACLs aside from being
readily identifiable in a way that makes sense to humans, but there are
some subtle changes to the syntax. So let's re-create the standard
access list we created earlier for our test network in
\protect\hyperlink{c12.xhtmlux5cux23figure12-2}{Figure 12.2} using a
named access list:

\begin{verbatim}
Lab_A#config t
Lab_A(config)# ip access-list ?
  extended    Extended Access List
  log-update  Control access list log updates
  logging     Control access list logging
  resequence  Resequence Access List
  standard    Standard Access List
\end{verbatim}

Notice that I started by typing \texttt{ip\ access-list}, not
\texttt{access-list}. Doing this allows me to enter a named access list.
Next, I'll need to specify it as a standard access list:

\begin{verbatim}
Lab_A(config)#ip access-list standard ?
  <1-99>       Standard IP access-list number
  <1300-1999>  Standard IP access-list number (expanded range)
  WORD         Access-list name
\end{verbatim}

\begin{verbatim}
Lab_A(config)#ip access-list standard BlockSales
Lab_A(config-std-nacl)#
\end{verbatim}

I've specified a standard access list, then added the name, BlockSales.
I definitely could've used a number for a standard access list, but
instead, I chose to use a nice, clear, descriptive name. And notice that
after entering the name, I hit Enter and the router prompt changed. This
confirms that I'm now in named access list configuration mode and that
I'm entering the named access list:

\begin{verbatim}
Lab_A(config-std-nacl)#?
Standard Access List configuration commands:
  default  Set a command to its defaults
  deny     Specify packets to reject
  exit     Exit from access-list configuration mode
  no       Negate a command or set its defaults
  permit   Specify packets to forward
\end{verbatim}

\begin{verbatim}
Lab_A(config-std-nacl)#deny 172.16.40.0 0.0.0.255
Lab_A(config-std-nacl)#permit any
Lab_A(config-std-nacl)#exit
Lab_A(config)#^Z
Lab_A#
\end{verbatim}

So I've entered the access list and then exited configuration mode.
Next, I'll take a look at the running configuration to verify that the
access list is indeed in the router:

\begin{verbatim}
Lab_A#sh running-config | begin ip access
ip access-list standard BlockSales
 deny   172.16.40.0 0.0.0.255
 permit any
!
\end{verbatim}

And there it is: the BlockSales access list has truly been created and
is in the running-config of the router. Next, I'll need to apply the
access list to the correct interface:

\begin{verbatim}
Lab_A#config t
Lab_A(config)#int fa0/1
Lab_A(config-if)#ip access-group BlockSales out
\end{verbatim}

Clear skies! At this point, we've re-created the work done earlier using
a named access list. But let's take our IP extended example, shown in
\protect\hyperlink{c12.xhtmlux5cux23figure12-6}{Figure 12.6}, and redo
that list using a named ACL instead as well.

\protect\hypertarget{c12.xhtmlux5cux23Page_507}{}{}Same business
requirements: Allow HTTP access to the Finance server from source Host B
only. All other traffic is permitted.

\begin{verbatim}
Lab_A#config t
Lab_A(config)#ip access-list extended 110
Lab_A(config-ext-nacl)#permit tcp host 192.168.177.2 host 172.22.89.26 eq 80
Lab_A(config-ext-nacl)#deny tcp any host 172.22.89.26 eq 80
Lab_A(config-ext-nacl)#permit ip any any
Lab_A(config-ext-nacl)#int fa0/1
Lab_A(config-if)#ip access-group 110 out
\end{verbatim}

Okay---true---I named the extended list with a number, but sometimes
it's okay to do that! I'm guessing that named ACLs don't seem all that
exciting or different to you, do they? Maybe not in this configuration,
except that I don't need to start every line with
\texttt{access-list\ 110}, which is nice. But where named ACLs really
shine is that they allow us to insert, delete, or edit a single line.
That isn't just nice, it's wonderful! Numbered ACLs just can't compare
with that, and I'll demonstrate this in a minute.

\paragraph{Remarks}

The \texttt{remark} keyword is really important because it arms you with
the ability to include comments---remarks---regarding the entries you've
made in both your IP standard and extended ACLs. Remarks are very cool
because they efficiently increase your ability to examine and understand
your ACLs to superhero level! Without them, you'd be caught in a
quagmire of potentially meaningless numbers without anything to help you
recall what all those numbers mean.

Even though you have the option of placing your remarks either before or
after a \texttt{permit} or \texttt{deny} statement, I totally recommend
that you choose to position them consistently so you don't get confused
about which remark is relevant to a specific \texttt{permit} or
\texttt{deny} statement.

To get this going for both standard and extended ACLs, just use the
\texttt{access-list\ access-list\ number\ remark} remark global
configuration command like this:

\begin{verbatim}
R2#config t
R2(config)#access-list 110 remark Permit Bob from Sales Only To Finance
R2(config)#access-list 110 permit ip host 172.16.40.1 172.16.50.0 0.0.0.255
R2(config)#access-list 110 deny ip 172.16.40.0 0.0.0.255 172.16.50.0 0.0.0.255
R2(config)#ip access-list extended No_Telnet
R2(config-ext-nacl)#remark Deny all of Sales from Telnetting to Marketing
R2(config-ext-nacl)#deny tcp 172.16.40.0 0.0.0.255 172.16.60.0 0.0.0.255 eq 23
R2(config-ext-nacl)#permit ip any any
R2(config-ext-nacl)#do show run
[output cut]
!
ip access-list extended No_Telnet
 remark Stop all of Sales from Telnetting to Marketing
 deny   tcp 172.16.40.0 0.0.0.255 172.16.60.0 0.0.0.255 eq telnet
 permit ip any any
!
access-list 110 remark Permit Bob from Sales Only To Finance
access-list 110 permit ip host 172.16.40.1 172.16.50.0 0.0.0.255
access-list 110 deny   ip 172.16.40.0 0.0.0.255 172.16.50.0 0.0.0.255
access-list 110 permit ip any any
!
\end{verbatim}

Sweet---I was able to add a \texttt{remark} to both an extended list and
a named access list. Keep in mind that you cannot see these remarks in
the output of the \texttt{show\ access-list} command, which we'll cover
next, because they only show up in the running-config.

Speaking of ACLs, I still need to show you how to monitor and verify
them. This is an important topic, so pay attention!

\subsection[Monitoring Access
Lists]{\texorpdfstring{\protect\hypertarget{c12.xhtmlux5cux23c12-sec-12}{}{}Monitoring
Access Lists}{Monitoring Access Lists}}

It's always good to be able to verify a router's configuration.
\protect\hyperlink{c12.xhtmlux5cux23table12-1}{Table 12.1} lists the
commands that we can use to achieve that.

{\protect\hyperlink{c12.xhtmlux5cux23tableanchor12-1}{\textbf{TABLE
12.1}} Commands used to verify access-list configuration}

\begin{longtable}[]{@{}ll@{}}
\toprule
Command & Effect\tabularnewline
\midrule
\endhead
\texttt{show\ access-list} & Displays all access lists and their
parameters configured on the router. Also shows statistics about how
many times the line either permitted or denied a packet. This command
does not show you which interface the list is applied on.\tabularnewline
\texttt{show\ access-list\ 110} & Reveals only the parameters for access
list 110. Again, this command will not reveal the specific interface the
list is set on.\tabularnewline
\texttt{show\ ip\ access-list} & Shows only the IP access lists
configured on the router.\tabularnewline
\texttt{show\ ip\ interface} & Displays which interfaces have access
lists set on them.\tabularnewline
\texttt{show\ running-config} & Shows the access lists and the specific
interfaces that have ACLs applied on them.\tabularnewline
\bottomrule
\end{longtable}

\protect\hypertarget{c12.xhtmlux5cux23Page_509}{}{}We've already used
the \texttt{show\ running-config} command to verify that a named access
list was in the router, so now let's take a look at the output from some
of the other commands.

The \texttt{show\ access-list} command will list all ACLs on the router,
whether they're applied to an interface or not:

\begin{verbatim}
Lab_A#show access-list
Standard IP access list 10
    10 deny   172.16.40.0, wildcard bits 0.0.0.255
    20 permit any
Standard IP access list BlockSales
    10 deny   172.16.40.0, wildcard bits 0.0.0.255
    20 permit any
Extended IP access list 110
    10 deny tcp any host 172.16.30.5 eq ftp
    20 deny tcp any host 172.16.30.5 eq telnet
    30 permit ip any any
    40 permit tcp host 192.168.177.2 host 172.22.89.26 eq www
    50 deny tcp any host 172.22.89.26 eq www
Lab_A#
\end{verbatim}

First, notice that access list 10 as well as both of our named access
lists appear on this list---remember, my extended named ACL was named
110! Second, notice that even though I entered actual numbers for TCP
ports in access list 110, the \texttt{show} command gives us the
protocol names rather than TCP ports for serious clarity.

But wait! The best part is those numbers on the left side: 10, 20, 30,
etc. Those are called sequence numbers, and they allow us to edit our
named ACL. Here's an example where I added a line into the named
extended ACL 110:

\begin{verbatim}
Lab_A (config)#ip access-list extended 110
Lab_A (config-ext-nacl)#21 deny udp any host 172.16.30.5 eq 69
Lab_A#show access-list
[output cut]
Extended IP access list 110
    10 deny tcp any host 172.16.30.5 eq ftp
    20 deny tcp any host 172.16.30.5 eq telnet
    21 deny udp any host 172.16.30.5 eq tftp
    30 permit ip any any
    40 permit tcp host 192.168.177.2 host 172.22.89.26 eq www
    50 deny tcp any host 172.22.89.26 eq www
\end{verbatim}

You can see that I added line 21. I could have deleted a line or edited
an existing line as well---very nice!

\protect\hypertarget{c12.xhtmlux5cux23Page_510}{}{}Here's the output of
the \texttt{show\ ip\ interface} command:

\begin{verbatim}
Lab_A#show ip interface fa0/1
FastEthernet0/1 is up, line protocol is up
  Internet address is 172.16.30.1/24
  Broadcast address is 255.255.255.255
  Address determined by non-volatile memory
  MTU is 1500 bytes
  Helper address is not set
  Directed broadcast forwarding is disabled
  Outgoing access list is 110
  Inbound access list is not set
  Proxy ARP is enabled
  Security level is default
  Split horizon is enabled
[output cut]
\end{verbatim}

Be sure to notice the bold line indicating that the outgoing list on
this interface is 110, yet the inbound access list isn't set. What
happened to BlockSales? I had configured that outbound on Fa0/1! That's
true, I did, but I configured my extended named ACL 110 and applied it
to Fa0/1 as well. You can't have two lists on the same interface, in the
same direction, so what happened here is that my last configuration
overwrote the BlockSales configuration.

And as I've already mentioned, you can use the
\texttt{show\ running-config} command to see any and all access lists.

\subsection[Summary]{\texorpdfstring{\protect\hypertarget{c12.xhtmlux5cux23c12-sec-13}{}{}Summary}{Summary}}

In this chapter you learned how to configure standard access lists to
properly filter IP traffic. You discovered what a standard access list
is and how to apply it to a Cisco router to add security to your
network. You also learned how to configure extended access lists to
further filter IP traffic. We also covered the key differences between
standard and extended access lists as well as how to apply them to Cisco
routers.

Moving on, you found out how to configure named access lists and apply
them to interfaces on the router and learned that named access lists
offer the huge advantage of being easily identifiable and, therefore, a
whole lot easier to manage than mysterious access lists that are simply
referred to by obscure numbers.

Appendix C, ``Disabling and Configuring Network Services,'' which takes
off from this chapter, has a fun section in it: turning off default
services. I've always found performing this administration task fun, and
the \texttt{auto\ secure} command can help us configure basic,
much-needed security on our routers.

The chapter wrapped up by showing you how to monitor and verify selected
access-list configurations on a router.

\subsection[Exam
Essentials]{\texorpdfstring{\protect\hypertarget{c12.xhtmlux5cux23c12-sec-14}{}{}\protect\hypertarget{c12.xhtmlux5cux23Page_511}{}{}Exam
Essentials}{Exam Essentials}}

\textbf{Remember the standard and extended IP access-list number
ranges.} The number ranges you can use to configure a standard IP access
list are 1--99 and 1300--1999. The number ranges for an extended IP
access list are 100--199 and 2000--2699.

\textbf{Understand the term}\emph{implicit deny}. At the end of every
access list is an \emph{implicit deny}. What this means is that if a
packet does not match any of the lines in the access list, it will be
discarded. Also, if you have nothing but \texttt{deny} statements in
your list, the list will not permit any packets.

\textbf{Understand the standard IP access-list configuration command.}
To configure a standard IP access list, use the access-list numbers
1--99 or 1300--1999 in global configuration mode. Choose \texttt{permit}
or \texttt{deny}, then choose the source IP address you want to filter
on using one of the three techniques covered in this chapter.

\textbf{Understand the extended IP access-list configuration command.}
To configure an extended IP access list, use the access-list numbers
100--199 or 2000--2699 in global configuration mode. Choose
\texttt{permit} or \texttt{deny}, the Network layer protocol field, the
source IP address you want to filter on, the destination address you
want to filter on, and finally, the Transport layer port number if TCP
or UDP has been specified as the protocol.

\textbf{Remember the command to verify an access list on a router
interface.} To see whether an access list is set on an interface and in
which direction it is filtering, use the \texttt{show\ ip\ interface}
command. This command will not show you the contents of the access list,
merely which access lists are applied on the interface.

\textbf{Remember the command to verify the access-list configuration.}
To see the configured access lists on your router, use the
\texttt{show\ access-list} command. This command will not show you which
interfaces have an access list set.

\subsection[Written Lab
12]{\texorpdfstring{\protect\hypertarget{c12.xhtmlux5cux23c12-sec-15}{}{}Written
Lab 12}{Written Lab 12}}

In this section, you'll complete the following lab to make sure you've
got the information and concepts contained within them fully dialed in:

Lab 12.1: Security

The answers to this lab can be found in Appendix A, ``Answers to Written
Labs.''

In this section, write the answers to the following questions:

\begin{enumerate}
\tightlist
\item
  What command would you use to configure a standard IP access list to
  prevent all machines on network 172.16.0.0/16 from accessing your
  Ethernet network?
\item
  What command would you use to apply the access list you created in
  question 1 to an Ethernet interface outbound?
\item
  \protect\hypertarget{c12.xhtmlux5cux23Page_512}{}{}What command(s)
  would you use to create an access list that denies host 192.168.15.5
  access to an Ethernet network?
\item
  Which command verifies that you've entered the access list correctly?
\item
  What two tools can help notify and prevent DoS attacks?
\item
  What command(s) would you use to create an extended access list that
  stops host 172.16.10.1 from telnetting to host 172.16.30.5?
\item
  What command would you use to set an access list on a VTY line?
\item
  Write the same standard IP access list you wrote in question 1 but
  this time as a named access list.
\item
  Write the command to apply the named access list you created in
  question 8 to an Ethernet interface outbound.
\item
  Which command verifies the placement and direction of an access list?
\end{enumerate}

\subsection[Hands-on
Labs]{\texorpdfstring{\protect\hypertarget{c12.xhtmlux5cux23c12-sec-16}{}{}Hands-on
Labs}{Hands-on Labs}}

In this section, you will complete two labs. To complete these labs, you
will need at least three routers. You can easily perform these labs with
the Cisco Packet Tracer program. If you are studying to take your Cisco
exam, you really need to do these labs!

\begin{enumerate}
\tightlist
\item
  Lab 12.1: Standard IP Access Lists
\item
  Lab 12.2: Extended IP Access Lists
\end{enumerate}

All of the labs will use the following diagram for configuring the
routers.

\begin{figure}
\centering
\includegraphics{images/c12f007.jpg}
\caption{}
\end{figure}

\subsubsection[Hands-on Lab 12.1: Standard IP Access
Lists]{\texorpdfstring{\protect\hypertarget{c12.xhtmlux5cux23c12-sec-17}{}{}\protect\hypertarget{c12.xhtmlux5cux23Page_513}{}{}Hands-on
Lab 12.1: Standard IP Access
Lists}{Hands-on Lab 12.1: Standard IP Access Lists}}

In this lab, you will allow only packets from a single host on the SF
LAN to enter the LA LAN.

\begin{enumerate}
\item
  Go to LA router and enter global configuration mode by typing
  \texttt{config\ t}.
\item
  From global configuration mode, type \texttt{access-list\ ?} to get a
  list of all the different access lists available.
\item
  Choose an access-list number that will allow you to create an IP
  standard access list. This is a number between 1 and 99 or 1300 and
  1399.
\item
  Choose to permit host 192.168.10.2, which is the host address:

\begin{verbatim}
LA(config)#access-list 10 permit 192.168.20.2 ?
  A.B.C.D  Wildcard bits
  <cr>
\end{verbatim}

  To specify only host 192.168.20.2, use the wildcards 0.0.0.0:

\begin{verbatim}
LA(config)#access-list 10 permit 192.168.20.2
  0.0.0.0
\end{verbatim}
\item
  Now that the access list is created, you must apply it to an interface
  to make it work:

\begin{verbatim}
LA(config)#int f0/0
Lab_A(config-if)#ip access-group 10 out
\end{verbatim}
\item
  Verify your access list with the following commands:

\begin{verbatim}
LA#sh access-list
Standard IP access list 10
    permit 192.168.20.2
LA#sh run
[output cut]
interface FastEthernet0/0
 ip address 192.168.20.1 255.255.255.0
 ip access-group 10 out
\end{verbatim}
\item
  Test your access list by pinging from 192.168.10.2 to 192.168.20.2.
\item
  If you have another host on the LA LAN, ping that address, which
  should fail if your ACL is working.
\end{enumerate}

\subsubsection[Hands-on Lab 12.2: Extended IP Access
Lists]{\texorpdfstring{\protect\hypertarget{c12.xhtmlux5cux23c12-sec-18}{}{}\protect\hypertarget{c12.xhtmlux5cux23Page_514}{}{}Hands-on
Lab 12.2: Extended IP Access
Lists}{Hands-on Lab 12.2: Extended IP Access Lists}}

In this lab, you will use an extended IP access list to stop host
192.168.10.2 from creating a Telnet session to router LA (172.16.10.6).
However, the host still should be able to ping the LA router. IP
extended lists should be placed close to the source, so add the extended
list on router SF. Pay attention to the \texttt{log} command used in
step 6. It is a Cisco objective!

\begin{enumerate}
\item
  Remove any access lists on SF and add an extended list to SF.
\item
  Choose a number to create an extended IP list. The IP extended lists
  use 100--199 or 2000--2699.
\item
  Use a \texttt{deny} statement. (You'll add a \texttt{permit} statement
  in step 7 to allow other traffic to still work.)

\begin{verbatim}
SF(config)#access-list 110 deny ?
  <0-255>  An IP protocol number
  ahp      Authentication Header Protocol
  eigrp    Cisco's EIGRP routing protocol
  esp      Encapsulation Security Payload
  gre      Cisco's GRE tunneling
  icmp     Internet Control Message Protocol
  igmp     Internet Gateway Message Protocol
  igrp     Cisco's IGRP routing protocol
  ip       Any Internet Protocol
  ipinip   IP in IP tunneling
  nos      KA9Q NOS compatible IP over IP tunneling
  ospf     OSPF routing protocol
  pcp      Payload Compression Protocol
  tcp      Transmission Control Protocol
  udp      User Datagram Protocol
\end{verbatim}
\item
  Since you are going to deny Telnet, you must choose TCP as a Transport
  layer protocol:

\begin{verbatim}
SF(config)#access-list 110 deny tcp ?
  A.B.C.D  Source address
  any      Any source host
  host     A single source host
\end{verbatim}
\item
  Add the source IP address you want to filter on, then add the
  destination host IP address. Use the \texttt{host} command instead of
  wildcard bits.

\begin{verbatim}
SF(config)#access-list 110 deny tcp host
  192.168.10.2 host 172.16.10.6 ?
  ack          Match on the ACK bit
eq           Match only packets on a given port
               number
  established  Match established connections
  fin          Match on the FIN bit
  fragments    Check fragments
  gt           Match only packets with a greater
               port number
  log          Log matches against this entry
  log-input    Log matches against this entry,
               including input interface
  lt           Match only packets with a lower port
               number
  neq          Match only packets not on a given
               port number
  precedence   Match packets with given precedence
               value
  psh          Match on the PSH bit
  range        Match only packets in the range of
               port numbers
  rst          Match on the RST bit
  syn          Match on the SYN bit
  tos          Match packets with given TOS value
  urg          Match on the URG bit
  <cr>
\end{verbatim}
\item
  At this point, you can add the \texttt{eq\ telnet} command to filter
  host 192.168.10.2 from telnetting to 172.16.10.6. The \texttt{log}
  command can also be used at the end of the command so that whenever
  the access-list line is hit, a log will be generated on the console.

\begin{verbatim}
SF(config)#access-list 110 deny tcp host
  192.168.10.2 host 172.16.10.6 eq telnet log
\end{verbatim}
\item
  It is important to add this line next to create a \texttt{permit}
  statement. (Remember that 0.0.0.0 255.255.255.255 is the same as the
  \texttt{any} command.)

\begin{verbatim}
SF(config)#access-list 110 permit ip any 0.0.0.0
  255.255.255.255
\end{verbatim}

  You must create a \texttt{permit} statement; if you just add a
  \texttt{deny} statement, nothing will be permitted at all. Please see
  the sections earlier in this chapter for more detailed information on
  the \texttt{deny\ any} command implied at the end of every ACL.
\item
  \protect\hypertarget{c12.xhtmlux5cux23Page_516}{}{}Apply the access
  list to the FastEthernet0/0 on SF to stop the Telnet traffic as soon
  as it hits the first router interface.

\begin{verbatim}
SF(config)#int f0/0
SF(config-if)#ip access-group 110 in
SF(config-if)#^Z
\end{verbatim}
\item
  Try telnetting from host 192.168.10.2 to LA using the destination IP
  address of 172.16.10.6. This should fail, but the \texttt{ping}
  command should work.
\item
  On the console of SF, because of the \texttt{log} command, the output
  should appear as follows:

\begin{verbatim}
01:11:48: %SEC-6-IPACCESSLOGP: list 110 denied tcp
  192.168.10.2(1030) -> 172.16.10.6(23), 1 packet
01:13:04: %SEC-6-IPACCESSLOGP: list 110 denied tcp
  192.168.10.2(1030) -> 172.16.10.6(23), 3 packets
\end{verbatim}
\end{enumerate}

\subsection[Review
Questions]{\texorpdfstring{\protect\hypertarget{c12.xhtmlux5cux23c12-sec-19}{}{}\protect\hypertarget{c12.xhtmlux5cux23Page_517}{}{}Review
Questions}{Review Questions}}

\begin{center}\rule{0.5\linewidth}{0.5pt}\end{center}

\includegraphics{images/note.png}The following questions are designed to
test your understanding of this chapter's material. For more information
on how to get additional questions, please see
\href{http://www.lammle.com/ccna}{www.lammle.com/ccna}.

\begin{center}\rule{0.5\linewidth}{0.5pt}\end{center}

You can find the answers to these questions in Appendix B, ``Answers to
Review Questions.''

\begin{enumerate}
\item
  Which of the following statements is false when a packet is being
  compared to an access list?

  \begin{enumerate}
  \tightlist
  \item
    It's always compared with each line of the access list in sequential
    order.
  \item
    Once the packet matches the condition on a line of the access list,
    the packet is acted upon and no further comparisons take place.
  \item
    There is an implicit ``deny'' at the end of each access list.
  \item
    Until all lines have been analyzed, the comparison is not over.
  \end{enumerate}
\item
  You need to create an access list that will prevent hosts in the
  network range of 192.168.160.0 to 192.168.191.0. Which of the
  following lists will you use?

  \begin{enumerate}
  \tightlist
  \item
    \texttt{access-list\ 10\ deny\ 192.168.160.0\ 255.255.224.0}
  \item
    \texttt{access-list\ 10\ deny\ 192.168.160.0\ 0.0.191.255}
  \item
    \texttt{access-list\ 10\ deny\ 192.168.160.0\ 0.0.31.255}
  \item
    \texttt{access-list\ 10\ deny\ 192.168.0.0\ 0.0.31.255}
  \end{enumerate}
\item
  You have created a named access list called BlockSales. Which of the
  following is a valid command for applying this to packets trying to
  enter interface Fa0/0 of your router?

  \begin{enumerate}
  \tightlist
  \item
    \texttt{(config)\#ip\ access-group\ 110\ in}
  \item
    \texttt{(config-if)\#ip\ access-group\ 110\ in}
  \item
    \texttt{(config-if)\#ip\ access-group\ Blocksales\ in}
  \item
    \texttt{(config-if)\#BlockSales\ ip\ access-list\ in}
  \end{enumerate}
\item
  Which access list statement will permit all HTTP sessions to network
  192.168.144.0/24 containing web servers?

  \begin{enumerate}
  \tightlist
  \item
    \texttt{access-list\ 110\ permit\ tcp\ 192.168.144.0\ 0.0.0.255\ any\ eq\ 80}
  \item
    \texttt{access-list\ 110\ permit\ tcp\ any\ 192.168.144.0\ 0.0.0.255\ eq\ 80}
  \item
    \texttt{access-list\ 110\ permit\ tcp\ 192.168.144.0\ 0.0.0.255\ 192.168.144.0\ 0.0.0.255\ any\ eq\ 80}
  \item
    \texttt{access-list\ 110\ permit\ udp\ any\ 192.168.144.0\ eq\ 80}
  \end{enumerate}
\item
  Which of the following access lists will allow only HTTP traffic into
  network 196.15.7.0?

  \begin{enumerate}
  \tightlist
  \item
    \texttt{access-list\ 100\ permit\ tcp\ any\ 196.15.7.0\ 0.0.0.255\ eq\ www}
  \item
    \texttt{access-list\ 10\ deny\ tcp\ any\ 196.15.7.0\ eq\ www}
  \item
    \texttt{access-list\ 100\ permit\ 196.15.7.0\ 0.0.0.255\ eq\ www}
  \item
    \protect\hypertarget{c12.xhtmlux5cux23Page_518}{}{}\texttt{access-list\ 110\ permit\ ip\ any\ 196.15.7.0\ 0.0.0.255}
  \item
    \texttt{access-list\ 110\ permit\ www\ 196.15.7.0\ 0.0.0.255}
  \end{enumerate}
\item
  What router command allows you to determine whether an IP access list
  is enabled on a particular interface?

  \begin{enumerate}
  \tightlist
  \item
    \texttt{show\ ip\ port}
  \item
    \texttt{show\ access-lists}
  \item
    \texttt{show\ ip\ interface}
  \item
    \texttt{show\ access-lists\ interface}
  \end{enumerate}
\item
  In the work area, connect the \texttt{show} command to its function on
  the right.

  \begin{longtable}[]{@{}ll@{}}
  \toprule
  \endhead
  show access-list & Shows only the parameters for the access list 110.
  This command does not show you the interface the list is set
  on.\tabularnewline
  show access-list 110 & Shows only the IP access lists configured on
  the router.\tabularnewline
  show ip access-list & Shows which interfaces have access lists
  set.\tabularnewline
  show ip interface & Displays all access lists and their parameters
  configured on the router. This command does not show you which
  interface the list is set on.\tabularnewline
  \bottomrule
  \end{longtable}
\item
  If you wanted to deny all Telnet connections to only network
  192.168.10.0, which command could you use?

  \begin{enumerate}
  \tightlist
  \item
    \texttt{access-list\ 100\ deny\ tcp\ 192.168.10.0\ 255.255.255.0\ eq\ telnet}
  \item
    \texttt{access-list\ 100\ deny\ tcp\ 192.168.10.0\ 0.255.255.255\ eq\ telnet}
  \item
    \texttt{access-list\ 100\ deny\ tcp\ any\ 192.168.10.0\ 0.0.0.255\ eq\ 23}
  \item
    \texttt{access-list\ 100\ deny\ 192.168.10.0\ 0.0.0.255\ any\ eq\ 23}
  \end{enumerate}
\item
  If you wanted to deny FTP access from network 200.200.10.0 to network
  200.199.11.0 but allow everything else, which of the following command
  strings is valid?

  \begin{enumerate}
  \item
    \texttt{access-list\ 110\ deny\ 200.200.10.0\ to\ network\ 200.199.11.0\ eq\ ftp}
  \item
    \texttt{access-list\ 111\ permit\ ip\ any\ 0.}\texttt{0.0.0\ 255.255.255.255}
  \item
    \texttt{access-list\ 1\ deny\ ftp\ 200.200.10.0\ 200.199.11.0\ any\ any}
  \item
    \texttt{access-list\ 100\ deny\ tcp\ 200.200.10.0\ 0.0.0.255\ 200.199.11.0\ 0.0.0.255\ eq\ ftp}
  \item
    \texttt{access-list\ 198\ deny\ tcp\ 200.200.10.0\ 0.0.0.255\ 200.199.11.0\ 0.0.0.255\ eq\ ftp}

    \texttt{access-list\ 198\ permit\ ip\ any\ 0.0.0.0\ 255.255.255.255}
  \end{enumerate}
\item
  You want to create an extended access list that denies the subnet of
  the following host: 172.16.50.172/20. Which of the following would you
  start your list with?

  \begin{enumerate}
  \tightlist
  \item
    \texttt{access-list\ 110\ deny\ ip\ 172.16.48.0\ 255.255.240.0\ any}
  \item
    \texttt{access-list\ 110\ udp\ deny\ 172.16.0.0\ 0.0.255.255\ ip\ any}
  \item
    \protect\hypertarget{c12.xhtmlux5cux23Page_519}{}{}\texttt{access-list\ 110\ deny\ tcp\ 172.16.64.0\ 0.0.31.255\ any\ eq\ 80}
  \item
    \texttt{access-list\ 110\ deny\ ip\ 172.16.48.0\ 0.0.15.255\ any}
  \end{enumerate}
\item
  Which of the following is the wildcard (inverse) version of a /27
  mask?

  \begin{enumerate}
  \tightlist
  \item
    0.0.0.7
  \item
    0.0.0.31
  \item
    0.0.0.27
  \item
    0.0.31.255
  \end{enumerate}
\item
  You want to create an extended access list that denies the subnet of
  the following host: 172.16.198.94/19. Which of the following would you
  start your list with?

  \begin{enumerate}
  \tightlist
  \item
    \texttt{access-list\ 110\ deny\ ip\ 172.16.192.0\ 0.0.31.255\ any}
  \item
    \texttt{access-list\ 110\ deny\ ip\ 172.16.0.0\ 0.0.255.255\ any}
  \item
    \texttt{access-list\ 10\ deny\ ip\ 172.16.172.0\ 0.0.31.255\ any}
  \item
    \texttt{access-list\ 110\ deny\ ip\ 172.16.188.0\ 0.0.15.255\ any}
  \end{enumerate}
\item
  The following access list has been applied to an interface on a
  router:

  \texttt{access-list\ 101\ deny\ tcp\ 199.111.16.32\ 0.0.0.31\ host\ 199.168.5.60}

  Which of the following IP addresses will be blocked because of this
  single rule in the list? (Choose all that apply.)

  \begin{enumerate}
  \tightlist
  \item
    199.111.16.67
  \item
    199.111.16.38
  \item
    199.111.16.65
  \item
    199.11.16.54
  \end{enumerate}
\item
  Which of the following commands connects access list 110 inbound to
  interface Ethernet0?

  \begin{enumerate}
  \tightlist
  \item
    \texttt{Router(config)\#}\texttt{ip\ access-group\ 110\ in}
  \item
    \texttt{Router(config)\#}\texttt{ip\ access-list\ 110\ in}
  \item
    \texttt{Router(config-if)\#}\texttt{ip\ access-group\ 110\ in}
  \item
    \texttt{Router(config-if)\#}\texttt{ip\ access-list\ 110\ in}
  \end{enumerate}
\item
  What is the effect of this single-line access list?

  \texttt{access-list\ 110\ deny\ ip\ 172.16.10.0\ 0.0.0.255\ host\ 1.1.1.1}

  \begin{enumerate}
  \tightlist
  \item
    Denies only the computer at 172.16.10
  \item
    Denies all traffic
  \item
    Denies the subnet 172.16.10.0/26
  \item
    Denies the subnet 172.16.10.0/25
  \end{enumerate}
\item
  You configure the following access list. What will the result of this
  access list be?

  \texttt{access-list\ 110\ deny\ tcp\ 10.1.1.128\ 0.0.0.63\ any\ eq\ smtp}

  \texttt{access-list\ 110\ deny\ tcp\ any\ any\ eq\ 23}

  \texttt{int\ ethernet\ 0}

  \texttt{ip\ access-group\ 110\ out}

  \begin{enumerate}
  \tightlist
  \item
    Email and Telnet will be allowed out E0.
  \item
    Email and Telnet will be allowed in E0.
  \item
    Everything but email and Telnet will be allowed out E0.
  \item
    No IP traffic will be allowed out E0.
  \end{enumerate}
\item
  Which of the following series of commands will restrict Telnet access
  to the router?

  \begin{enumerate}
  \item
\begin{verbatim}
Lab_A(config)#access-list 10 permit 172.16.1.1
Lab_A(config)#line con 0
Lab_A(config-line)#ip access-group 10 in
\end{verbatim}
  \item
\begin{verbatim}
Lab_A(config)#access-list 10 permit 172.16.1.1
Lab_A(config)#line vty 0 4
Lab_A(config-line)#access-class 10 out
\end{verbatim}
  \item
\begin{verbatim}
Lab_A(config)#access-list 10 permit 172.16.1.1
Lab_A(config)#line vty 0 4
Lab_A(config-line)#access-class 10 in
\end{verbatim}
  \item
\begin{verbatim}
Lab_A(config)#access-list 10 permit 172.16.1.1
Lab_A(config)#line vty 0 4
Lab_A(config-line)#ip access-group 10 in
\end{verbatim}
  \end{enumerate}
\item
  Which of the following is true regarding access lists applied to an
  interface?

  \begin{enumerate}
  \tightlist
  \item
    You can place as many access lists as you want on any interface
    until you run out of memory.
  \item
    You can apply only one access list on any interface.
  \item
    One access list may be configured, per direction, for each layer 3
    protocol configured on an interface.
  \item
    You can apply two access lists to any interface.
  \end{enumerate}
\item
  What is the most common attack on a network today?

  \begin{enumerate}
  \tightlist
  \item
    Lock picking
  \item
    Naggle
  \item
    DoS
  \item
    \texttt{auto\ secure}
  \end{enumerate}
\item
  You need to stop DoS attacks in real time and have a log of anyone who
  has tried to attack your network. What should you do your network?

  \begin{enumerate}
  \tightlist
  \item
    Add more routers.
  \item
    Use the \texttt{auto\ secure} command.
  \item
    Implement IDS/IPS.
  \item
    Configure Naggle.
  \end{enumerate}
\end{enumerate}

\protect\hypertarget{c13.xhtml}{}{}

\section[{Chapter 13}\\
{Network Address Translation
(NAT)}]{\texorpdfstring{\protect\hypertarget{c13.xhtmlux5cux23c13}{}{}\protect\hypertarget{c13.xhtmlux5cux23Page_521}{}{}{Chapter
13}\\
{Network Address Translation
(NAT)}}{Chapter 13 Network Address Translation (NAT)}}

\begin{center}\rule{0.5\linewidth}{0.5pt}\end{center}

\subsection{THE FOLLOWING ICND1 EXAM TOPICS ARE COVERED IN THIS
CHAPTER:}

\begin{enumerate}
\tightlist
\item
  \includegraphics{images/tick.png} \textbf{4.0 Infrastructure Services}

  \begin{enumerate}
  \tightlist
  \item
    \includegraphics{images/square.png} 4.7 Configure, verify, and
    troubleshoot inside source NAT

    \begin{enumerate}
    \tightlist
    \item
      \includegraphics{images/square.png} 4.7.a Static
    \item
      \includegraphics{images/square.png} 4.7.b Pool
    \item
      \includegraphics{images/square.png} 4.7.c PAT
    \end{enumerate}
  \end{enumerate}
\end{enumerate}

\protect\hypertarget{c13.xhtmlux5cux23Page_522}{}{}\includegraphics{images/intro.png}In
this chapter, we're going to dig into Network Address Translation (NAT),
Dynamic NAT, and Port Address Translation (PAT), also known as NAT
Overload. Of course, I'll demonstrate all the NAT commands. I also
provided some fantastic hands-on labs for you to configure at the end of
this chapter, so be sure not to miss those!

It's important to understand the Cisco objectives for this chapter. They
are very straightforward: you have hosts on your inside Corporate
network using RFC 1918 addresses and you need to allow those hosts
access to the Internet by configuring NAT translations. With that
objective in mind, that will be my direction with this chapter.

Because we'll be using ACLs in our NAT configurations, it's important
that you're really comfortable with the skills you learned in the
previous chapter before proceeding with this one.

\begin{center}\rule{0.5\linewidth}{0.5pt}\end{center}

\includegraphics{images/note.png}To find up-to-the-minute updates for
this chapter, please see \texttt{www.lammle.com/ccna} or the book's web
page at \texttt{www.sybex.com/go/ccna}.

\begin{center}\rule{0.5\linewidth}{0.5pt}\end{center}

\subsection[When Do We Use
NAT?]{\texorpdfstring{\protect\hypertarget{c13.xhtmlux5cux23c13-sec-1}{}{}When
Do We Use NAT?}{When Do We Use NAT?}}

\emph{Network Address Translation (NAT)} is similar to Classless
Inter-Domain Routing (CIDR) in that the original intention for NAT was
to slow the depletion of available IP address space by allowing multiple
private IP addresses to be represented by a much smaller number of
public IP addresses.

Since then, it's been discovered that NAT is also a useful tool for
network migrations and mergers, server load sharing, and creating
``virtual servers.'' So in this chapter, I'm going to describe the
basics of NAT functionality and the terminology common to NAT.

Because NAT really decreases the overwhelming amount of public IP
addresses required in a networking environment, it comes in really handy
when two companies that have duplicate internal addressing schemes
merge. NAT is also a great tool to use when an organization changes its
Internet service provider (ISP) but the networking manager needs to
avoid the hassle of changing the internal address scheme.

Here's a list of situations when NAT can be especially helpful:

\begin{enumerate}
\tightlist
\item
  When you need to connect to the Internet and your hosts don't have
  globally unique IP addresses
\item
  \protect\hypertarget{c13.xhtmlux5cux23Page_523}{}{}When you've changed
  to a new ISP that requires you to renumber your network
\item
  When you need to merge two intranets with duplicate addresses
\end{enumerate}

You typically use NAT on a border router. For example, in
\protect\hyperlink{c13.xhtmlux5cux23figure13-1}{Figure 13.1}, NAT is
used on the Corporate router connected to the Internet.

\begin{figure}
\centering
\includegraphics{images/c13f001.jpg}
\caption{{\protect\hyperlink{c13.xhtmlux5cux23figureanchor13-1}{\textbf{FIGURE
13.1}} Where to configure NAT}}
\end{figure}

Now you may be thinking, ``NAT's totally cool and I just gotta have
it!'' But don't get too excited yet because there are some serious snags
related to using NAT that you need to understand first. Don't get me
wrong---it can truly be a lifesaver sometimes, but NAT has a bit of a
dark side you need to know about too. For the pros and cons linked to
using NAT, check out
\protect\hyperlink{c13.xhtmlux5cux23table13-1}{Table 13.1}.

{\protect\hyperlink{c13.xhtmlux5cux23tableanchor13-1}{\textbf{TABLE
13.1}} Advantages and disadvantages of implementing NAT}

\begin{longtable}[]{@{}ll@{}}
\toprule
Advantages & Disadvantages\tabularnewline
\midrule
\endhead
Conserves legally registered addresses. & Translation results in
switching path delays.\tabularnewline
Remedies address overlap events. & Causes loss of end-to-end IP
traceability\tabularnewline
Increases flexibility when connecting to the Internet. & Certain
applications will not function with NAT enabled\tabularnewline
Eliminates address renumbering as a network evolves. & Complicates
tunneling protocols such as IPsec because NAT modifies the values in the
header\tabularnewline
\bottomrule
\end{longtable}

\begin{center}\rule{0.5\linewidth}{0.5pt}\end{center}

\protect\hypertarget{c13.xhtmlux5cux23Page_524}{}{}\includegraphics{images/note.png}The
most obvious advantage associated with NAT is that it allows you to
conserve your legally registered address scheme. But a version of it
known as PAT is also why we've only just recently run out of IPv4
addresses. Without NAT/PAT, we'd have run out of IPv4 addresses more
than a decade ago!

\begin{center}\rule{0.5\linewidth}{0.5pt}\end{center}

\subsection[Types of Network Address
Translation]{\texorpdfstring{\protect\hypertarget{c13.xhtmlux5cux23c13-sec-2}{}{}Types
of Network Address Translation}{Types of Network Address Translation}}

In this section, I'm going to go over the three types of NATs with you:

\textbf{Static NAT (one-to-one)} This type of NAT is designed to allow
one-to-one mapping between local and global addresses. Keep in mind that
the static version requires you to have one real Internet IP address for
every host on your network.

\textbf{Dynamic NAT (many-to-many)} This version gives you the ability
to map an unregistered IP address to a registered IP address from out of
a pool of registered IP addresses. You don't have to statically
configure your router to map each inside address to an individual
outside address as you would using static NAT, but you do have to have
enough real, bona fide IP addresses for everyone who's going to be
sending packets to and receiving them from the Internet at the same
time.

\textbf{Overloading (one-to-many)} This is the most popular type of NAT
configuration. Understand that overloading really is a form of dynamic
NAT that maps multiple unregistered IP addresses to a single registered
IP address (many-to-one) by using different source ports. Now, why is
this so special? Well, because it's also known as \emph{Port Address
Translation (PAT)}, which is also commonly referred to as NAT Overload.
Using PAT allows you to permit thousands of users to connect to the
Internet using only one real global IP address---pretty slick, right?
Seriously, NAT Overload is the real reason we haven't run out of valid
IP addresses on the Internet. Really---I'm not joking!

\begin{center}\rule{0.5\linewidth}{0.5pt}\end{center}

\includegraphics{images/note.png}I'll show you how to configure all
three types of NAT throughout this chapter and at the end of this
chapter with the hands-on labs.

\begin{center}\rule{0.5\linewidth}{0.5pt}\end{center}

\subsection[NAT
Names]{\texorpdfstring{\protect\hypertarget{c13.xhtmlux5cux23c13-sec-3}{}{}NAT
Names}{NAT Names}}

The names we use to describe the addresses used with NAT are fairly
straightforward. Addresses used after NAT translations are called
\emph{global addresses}. These are usually the public addresses used on
the Internet, which you don't need if you aren't going on the Internet.

\emph{Local addresses} are the ones we use before NAT translation. This
means that the inside local address is actually the private address of
the sending host that's attempting to get to the Internet. The outside
local address would typically be the router interface connected to your
ISP and is also usually a public address used as the packet begins its
journey.

\protect\hypertarget{c13.xhtmlux5cux23Page_525}{}{}After translation,
the inside local address is then called the \emph{inside global address}
and the outside global address then becomes the address of the
destination host. Check out
\protect\hyperlink{c13.xhtmlux5cux23table13-2}{Table 13.2}, which lists
all this terminology and offers a clear picture of the various names
used with NAT. Keep in mind that these terms and their definitions can
vary somewhat based on implementation. The table shows how they're used
according to the Cisco exam objectives.

{\protect\hyperlink{c13.xhtmlux5cux23tableanchor13-2}{\textbf{TABLE
13.2}} NAT terms}

\begin{longtable}[]{@{}ll@{}}
\toprule
Names & Meaning\tabularnewline
\midrule
\endhead
Inside local & Source host inside address before translation---typically
an RFC 1918 address.\tabularnewline
Outside local & Address of an outside host as it appears to the inside
network. This is usually the address of the router interface connected
to ISP---the actual Internet address.\tabularnewline
Inside global & Source host address used after translation to get onto
the Internet. This is also the actual Internet address.\tabularnewline
Outside global & Address of outside destination host and, again, the
real Internet address.\tabularnewline
\bottomrule
\end{longtable}

\subsection[How NAT
Works]{\texorpdfstring{\protect\hypertarget{c13.xhtmlux5cux23c13-sec-4}{}{}How
NAT Works}{How NAT Works}}

Okay, it's time to look at how this whole NAT thing works. I'm going to
start by using \protect\hyperlink{c13.xhtmlux5cux23figure13-2}{Figure
13.2} to describe basic NAT translation.

\begin{figure}
\centering
\includegraphics{images/c13f002.jpg}
\caption{{\protect\hyperlink{c13.xhtmlux5cux23figureanchor13-2}{\textbf{FIGURE
13.2}} Basic NAT translation}}
\end{figure}

\protect\hypertarget{c13.xhtmlux5cux23Page_526}{}{}In this figure, we
can see host 10.1.1.1 sending an Internet-bound packet to the border
router configured with NAT. The router identifies the source IP address
as an inside local IP address destined for an outside network,
translates the source IP address in the packet, and documents the
translation in the NAT table.

The packet is sent to the outside interface with the new translated
source address. The external host returns the packet to the destination
host and the NAT router translates the inside global IP address back to
the inside local IP address using the NAT table. This is as simple as it
gets!

Let's take a look at a more complex configuration using overloading,
also referred to as PAT. I'll use
\protect\hyperlink{c13.xhtmlux5cux23figure13-3}{Figure 13.3} to
demonstrate how PAT works by having an inside host HTTP to a server on
the Internet.

\begin{figure}
\centering
\includegraphics{images/c13f003.jpg}
\caption{{\protect\hyperlink{c13.xhtmlux5cux23figureanchor13-3}{\textbf{FIGURE
13.3}} NAT overloading example (PAT)}}
\end{figure}

With PAT, all inside hosts get translated to one single IP address,
hence the term \emph{overloading}. Again, the reason we've just run out
of available global IP addresses on the Internet is because of
overloading (PAT).

Take a look at the NAT table
in\protect\hyperlink{c13.xhtmlux5cux23figure13-3}{Figure 13.3} again. In
addition to the inside local IP address and inside global IP address, we
now have port numbers. These port numbers help the router identify which
host should receive the return traffic. The router uses the source port
number from each host to differentiate the traffic from each of them.
Understand that the packet has a destination port number of 80 when it
leaves the router, and the HTTP server sends back the data with a
destination port number of 1026, in this example. This allows the NAT
translation router to differentiate between hosts in the NAT table and
then translate the destination IP address back to the inside local
address.

Port numbers are used at the Transport layer to identify the local host
in this example. If we had to use real global IP addresses to identify
the source hosts, that's called \emph{static NAT}
\protect\hypertarget{c13.xhtmlux5cux23Page_527}{}{}and we would run out
of addresses. PAT allows us to use the Transport layer to identify the
hosts, which in turn allows us to theoretically use up to about 65,000
hosts with only one real IP address!

\subsubsection[Static NAT
Configuration]{\texorpdfstring{\protect\hypertarget{c13.xhtmlux5cux23c13-sec-5}{}{}Static
NAT Configuration}{Static NAT Configuration}}

Let's take a look at a simple example of a basic static NAT
configuration:

\begin{verbatim}
ip nat inside source static 10.1.1.1 170.46.2.2
!
interface Ethernet0
 ip address 10.1.1.10 255.255.255.0
 ip nat inside
!
interface Serial0
 ip address 170.46.2.1 255.255.255.0
 ip nat outside
!
\end{verbatim}

In the preceding router output, the \texttt{ip\ nat\ inside\ source}
command identifies which IP addresses will be translated. In this
configuration example, the \texttt{ip\ nat\ inside\ source} command
configures a static translation between the inside local IP address
10.1.1.1 and the outside global IP address 170.46.2.2.

Scrolling farther down in the configuration, we find an \texttt{ip\ nat}
command under each interface. The \texttt{ip\ nat\ inside} command
identifies that interface as the inside interface. The
\texttt{ip\ nat\ outside} command identifies that interface as the
outside interface. When you look back at the
\texttt{ip\ nat\ inside\ source} command, you can see that the command
is referencing the inside interface as the source or starting point of
the translation. You could also use the command like this:
\texttt{ip\ nat\ outside\ source}. This option indicates the interface
that you designated as the outside interface should become the source or
starting point for the translation.

\subsubsection[Dynamic NAT
Configuration]{\texorpdfstring{\protect\hypertarget{c13.xhtmlux5cux23c13-sec-6}{}{}Dynamic
NAT Configuration}{Dynamic NAT Configuration}}

Basically, dynamic NAT really means we have a pool of addresses that
we'll use to provide real IP addresses to a group of users on the
inside. Because we don't use port numbers, we must have real IP
addresses for every user who's trying to get outside the local network
simultaneously.

Here is a sample output of a dynamic NAT configuration:

\begin{verbatim}
ip nat pool todd 170.168.2.3 170.168.2.254
netmask 255.255.255.0
ip nat inside source list 1 pool todd
!
interface Ethernet0
 ip address 10.1.1.10 255.255.255.0
 ip nat inside
!
interface Serial0
 ip address 170.168.2.1 255.255.255.0
 ip nat outside
!
access-list 1 permit 10.1.1.0 0.0.0.255
!
\end{verbatim}

The \texttt{ip\ nat\ inside\ source\ list\ 1\ pool\ todd} command tells
the router to translate IP addresses that match \texttt{access-list\ 1}
to an address found in the IP NAT pool named \texttt{todd}. Here the ACL
isn't there to filter traffic for security reasons by permitting or
denying traffic. In this case, it's there to select or designate what we
often call interesting traffic. When interesting traffic has been
matched with the access list, it's pulled into the NAT process to be
translated. This is actually a common use for access lists, which aren't
always just stuck with the dull job of just blocking traffic at an
interface!

The command
\texttt{ip\ nat\ pool\ todd\ 170.168.2.3\ 170.168.2.254\ netmask\ 255.255.255.0}
creates a pool of addresses that will be distributed to the specific
hosts that require global addresses. When troubleshooting NAT for the
Cisco objectives, always check this pool to confirm that there are
enough addresses in it to provide translation for all the inside hosts.
Last, check to make sure the pool names match exactly on both lines,
remembering that they are case sensitive; if they don't, the pool won't
work!

\subsubsection[PAT (Overloading)
Configuration]{\texorpdfstring{\protect\hypertarget{c13.xhtmlux5cux23c13-sec-7}{}{}PAT
(Overloading) Configuration}{PAT (Overloading) Configuration}}

This last example shows how to configure inside global address
overloading. This is the typical form of NAT that we would use today.
It's actually now rare to use static or dynamic NAT unless it is for
something like statically mapping a server, for example.

Here is a sample output of a PAT configuration:

\begin{verbatim}
ip nat pool globalnet 170.168.2.1 170.168.2.1 netmask 255.255.255.0
ip nat inside source list 1 pool globalnet overload
!
interface Ethernet0/0
 ip address 10.1.1.10 255.255.255.0
 ip nat inside
!
interface Serial0/0
 ip address 170.168.2.1 255.255.255.0
 ip nat outside
!
access-list 1 permit 10.1.1.0 0.0.0.255
\end{verbatim}

\protect\hypertarget{c13.xhtmlux5cux23Page_529}{}{}The nice thing about
PAT is that these are only a few differences between this configuration
and the previous dynamic NAT configuration:

\begin{enumerate}
\tightlist
\item
  Our pool of addresses has shrunk to only one IP address.
\item
  We included the \texttt{overload} keyword at the end of our
  \texttt{ip\ nat\ inside\ source} command.
\end{enumerate}

A really key factor to see in the example is that the one IP address
that's in the pool for us to use is the IP address of the outside
interface. This is perfect if you are configuring NAT Overload for
yourself at home or for a small office that only has one IP address from
your ISP. You could, however, use an additional address such as
170.168.2.2 if you had that address available to you as well, and doing
that could prove very helpful in a very large implementation where
you've got such an abundance of simultaneously active internal users
that you need to have more than one overloaded IP address on the
outside!

\subsubsection[Simple Verification of
NAT]{\texorpdfstring{\protect\hypertarget{c13.xhtmlux5cux23c13-sec-8}{}{}Simple
Verification of NAT}{Simple Verification of NAT}}

As always, once you've chosen and configured the type of NAT you're
going to run, which is typically PAT, you must be able to verify your
configuration.

To see basic IP address translation information, use the following
command:

\begin{verbatim}
Router#show ip nat translations
\end{verbatim}

When looking at the IP NAT translations, you may see many translations
from the same host to the corresponding host at the destination.
Understand that this is typical when there are many connections to the
same server.

You can also verify your NAT configuration via the
\texttt{debug\ ip\ nat} command. This output will show the sending
address, the translation, and the destination address on each debug
line:

\begin{verbatim}
Router#debug ip nat
\end{verbatim}

But wait---how do you clear your NAT entries from the translation table?
Just use the \texttt{clear\ ip\ nat\ translation} command, and if you
want to clear all entries from the NAT table, just use an asterisk
(\texttt{*}) at the end of the command.

\subsection[Testing and Troubleshooting
NAT]{\texorpdfstring{\protect\hypertarget{c13.xhtmlux5cux23c13-sec-9}{}{}Testing
and Troubleshooting NAT}{Testing and Troubleshooting NAT}}

Cisco's NAT gives you some serious power---and it does so without much
effort, because the configurations are really pretty simple. But we all
know nothing's perfect, so in case something goes wrong, you can figure
out some of the more common culprits by running through this list of
potential causes:

\begin{enumerate}
\tightlist
\item
  Check the dynamic pools. Are they composed of the right scope of
  addresses?
\item
  Check to see if any dynamic pools overlap.
\item
  Check to see if the addresses used for static mapping and those in the
  dynamic pools overlap.
\item
  \protect\hypertarget{c13.xhtmlux5cux23Page_530}{}{}Ensure that your
  access lists specify the correct addresses for translation.
\item
  Make sure there aren't any addresses left out that need to be there,
  and ensure that none are included that shouldn't be.
\item
  Check to make sure you've got both the inside and outside interfaces
  delimited properly.
\end{enumerate}

A key thing to keep in mind is that one of the most common problems with
a new NAT configuration often isn't specific to NAT at all---it usually
involves a routing blooper. So, because you're changing a source or
destination address in a packet, make sure your router still knows what
to do with the new address after the translation!

The first command you should typically use is the
\texttt{show\ ip\ nat\ translations} command:

\begin{verbatim}
Router#show ip nat trans
Pro   Inside global   Inside local   Outside local   Outside global
---   192.2.2.1       10.1.1.1       ---             ---
---   192.2.2.2       10.1.1.2       ---             ---
\end{verbatim}

After checking out this output, can you tell me if the configuration on
the router is static or dynamic NAT? The answer is yes, either static or
dynamic NAT is configured because there's a one-to-one translation from
the inside local to the inside global. Basically, by looking at the
output, you can't tell if it's static or dynamic per se, but you
absolutely can tell that you're not using PAT because there are no port
numbers.

Let's take a look at another output:

\begin{verbatim}
Router#sh ip nat trans
Pro Inside global      Inside local       Outside local      Outside global
tcp 170.168.2.1:11003  10.1.1.1:11003     172.40.2.2:23      172.40.2.2:23
tcp 170.168.2.1:1067   10.1.1.1:1067      172.40.2.3:23      172.40.2.3:23
\end{verbatim}

Okay, you can easily see that the previous output is using NAT Overload
(PAT). The protocol in this output is TCP, and the inside global address
is the same for both entries.

Supposedly the sky's the limit regarding the number of mappings the NAT
table can hold. But this is reality, so things like memory and CPU, or
even the boundaries set in place by the scope of available addresses or
ports, can cause limitations on the actual number of entries. Consider
that each NAT mapping devours about 160 bytes of memory. And sometimes
the amount of entries must be limited for the sake of performance or
because of policy restrictions, but this doesn't happen very often. In
situations like these, just go to the
\texttt{ip\ nat\ translation\ max-entries} command for help.

Another handy command for troubleshooting is
\texttt{show\ ip\ nat\ statistics}. Deploying this gives you a summary
of the NAT configuration, and it will count the number of active
translation types too. Also counted are hits to an existing mapping as
well any misses, with the latter causing an attempt to create a mapping.
This command will also reveal expired translations. If you want to check
into dynamic pools, their types, the total available addresses, how many
addresses have been allocated and how many have failed, plus the number
of translations that have occurred, just use the\texttt{pool} keyword
after statistics.

\protect\hypertarget{c13.xhtmlux5cux23Page_531}{}{}Here is an example of
the basic NAT debugging command:

\begin{verbatim}
Router#debug ip nat
NAT: s=10.1.1.1->192.168.2.1, d=172.16.2.2 [0]
NAT: s=172.16.2.2, d=192.168.2.1->10.1.1.1 [0]
NAT: s=10.1.1.1->192.168.2.1, d=172.16.2.2 [1]
NAT: s=10.1.1.1->192.168.2.1, d=172.16.2.2 [2]
NAT: s=10.1.1.1->192.168.2.1, d=172.16.2.2 [3]
NAT*: s=172.16.2.2, d=192.168.2.1->10.1.1.1 [1]
\end{verbatim}

Notice the last line in the output and how the \texttt{NAT} at the
beginning of the line has an asterisk (\texttt{*}). This means the
packet was translated and fast-switched to the destination. What's
fast-switched? Well in brief, fast-switching has gone by several aliases
such as cache-based switching and this nicely descriptive name, ``route
once switch many.'' The fast-switching process is used on Cisco routers
to create a cache of layer 3 routing information to be accessed at layer
2 so packets can be forwarded quickly through a router without the
routing table having to be parsed for every packet. As packets are
packet switched (looked up in the routing table), this information is
stored in the cache for later use if needed for faster routing
processing.

Let's get back to verifying NAT. Did you know you can manually clear
dynamic NAT entries from the NAT table? You can, and doing this can come
in seriously handy if you need to get rid of a specific rotten entry
without sitting around waiting for the timeout to expire! A manual clear
is also really useful when you want to clear the whole NAT table to
reconfigure a pool of addresses.

You also need to know that the Cisco IOS software just won't allow you
to change or delete an address pool if any of that pool's addresses are
mapped in the NAT table. The\texttt{clear\ ip\ nat\ translations}
command clears entries---you can indicate a single entry via the global
and local address and through TCP and UDP translations, including ports,
or you can just type in an asterisk (\texttt{*}) to wipe out the entire
table. But know that if you do that, only dynamic entries will be
cleared because this command won't remove static entries.

Oh, and there's more---any outside device's packet destination address
that happens to be responding to any inside device is known as the
inside global (IG) address. This means that the initial mapping has to
be held in the NAT table so that all packets arriving from a specific
connection get translated consistently. Holding entries in the NAT table
also cuts down on repeated translation operations happening each time
the same inside machine sends packets to the same outside destinations
on a regular basis.

Let me clarify: When an entry is placed into the NAT table the first
time, a timer begins ticking and its duration is known as the
translation timeout. Each time a packet for a given entry translates
through the router, the timer gets reset. If the timer expires, the
entry will be unceremoniously removed from the NAT table and the
dynamically assigned address will then be returned to the pool. Cisco's
default translation timeout is 86,400 seconds (24hours), but you can
change that with the \texttt{ip\ nat\ translation\ timeout} command.

Before we move on to the configuration section and actually use the
commands I just talked about, let's go through a couple of NAT examples
and see if you can figure out the best configuration to go with. To
start, look at \protect\hyperlink{c13.xhtmlux5cux23figure13-4}{Figure
13.4} and ask yourself two things: Where would you implement NAT in this
design? What type of NAT would you configure?

\protect\hypertarget{c13.xhtmlux5cux23Page_532}{}{}

\begin{figure}
\centering
\includegraphics{images/c13f004.jpg}
\caption{{\protect\hyperlink{c13.xhtmlux5cux23figureanchor13-4}{\textbf{FIGURE
13.4}} NAT example}}
\end{figure}

In \protect\hyperlink{c13.xhtmlux5cux23figure13-4}{Figure 13.4}, the NAT
configuration would be placed on the corporate router, just as I
demonstrated with \protect\hyperlink{c13.xhtmlux5cux23figure13-1}{Figure
13.1}, and the configuration would be dynamic NAT with overload (PAT).
In this next NAT example, what type of NAT is being used?

\begin{verbatim}
ip nat pool todd-nat 170.168.10.10 170.168.10.20 netmask 255.255.255.0
ip nat inside source list 1 pool todd-nat
\end{verbatim}

The preceding command uses dynamic NAT without PAT. The \texttt{pool} in
the command gives the answer away as dynamic, plus there's more than one
address in the pool and there is no \texttt{overload} command at the end
of our \texttt{ip\ nat\ inside\ source} command. This means we are not
using PAT!

In the next NAT example, refer to
\protect\hyperlink{c13.xhtmlux5cux23figure13-5}{Figure 13.5} and see if
you can come up with the configuration needed.

\begin{figure}
\centering
\includegraphics{images/c13f005.jpg}
\caption{{\protect\hyperlink{c13.xhtmlux5cux23figureanchor13-5}{\textbf{FIGURE
13.5}} Another NAT example}}
\end{figure}

\protect\hypertarget{c13.xhtmlux5cux23Page_533}{}{}\protect\hyperlink{c13.xhtmlux5cux23figure13-5}{Figure
13.5} shows a border router that needs to be configured with NAT and
allow the use of six public IP addresses to the inside locals,
192.1.2.109 through 192.1.2.114. However, on the inside network, you
have 62 hosts that use the private addresses of 192.168.10.65 through
192.168.10.126. What would your NAT configuration be on the border
router?

Actually, two different answers would both work here, but the following
would be my first choice based on the exam objectives:

\begin{verbatim}
ip nat pool Todd 192.1.2.109 192.1.2.109 netmask 255.255.255.248
access-list 1 permit 192.168.10.64 0.0.0.63
ip nat inside source list 1 pool Todd overload
\end{verbatim}

The command
\texttt{ip\ nat\ pool\ Todd\ 192.1.2.109\ 192.1.2.109\ netmask\ 255.255.255.248}
sets the pool name as Todd and creates a dynamic pool of only one
address using NAT address 192.1.2.109. Instead of the \texttt{netmask}
command, you can use the \texttt{prefix-length\ 29} statement. Just in
case you're wondering, you cannot do this on router interfaces as well!

The second answer would get you the exact same result of having only
192.1.2.109 as your inside global, but you can type this in and it will
also work:
\texttt{ip\ nat\ pool\ Todd\ 192.1.2.109\ 192.1.2.114\ netmask\ 255.255.255.248}.
But this option really is a waste because the second through sixth
addresses would only be used if there was a conflict with a TCP port
number. You would use something like what I've shown in this example if
you literally had about ten thousand hosts with one Internet connection!
You would need it to help with the TCP-Reset issue when two hosts are
trying to use the same source port number and get a negative
acknowledgment (NAK). But in our example, we've only got up to 62 hosts
connecting to the Internet at the same time, so having more than one
inside global gets us nothing!

If you're fuzzy on the second line where the access list is set in the
NAT configuration, do a quick review of Chapter 12, ``Security.'' But
this isn't difficult to grasp because it's easy to see in this
access-list line that it's just the \emph{\texttt{network\ number}} and
\emph{\texttt{wildcard}} used with that command. I always say, ``Every
question is a subnet question,'' and this one is no exception. The
inside locals in this example were 192.168.10.65--126, which is a block
of 64, or a 255.255.255.192 mask. As I've said in pretty much every
chapter, you really need to be able to subnet quickly!

The command
\texttt{ip\ nat\ inside\ source\ list\ 1\ pool\ Todd\ overload} sets the
dynamic pool to use PAT by using the \texttt{overload} command.

And be sure to add the \texttt{ip\ nat\ inside} and
\texttt{ip\ nat\ outside} statements on the appropriate interfaces.

\begin{center}\rule{0.5\linewidth}{0.5pt}\end{center}

\includegraphics{images/note.png}If you're planning on testing for any
Cisco exam, configure the hands-on labs at the end of this chapter until
you're really comfortable with doing that!

\begin{center}\rule{0.5\linewidth}{0.5pt}\end{center}

One more example, and then you are off to the written lab, hands-on
labs, and review questions.

\protect\hypertarget{c13.xhtmlux5cux23Page_534}{}{}The network in
\protect\hyperlink{c13.xhtmlux5cux23figure13-6}{Figure 13.6} is already
configured with IP addresses as shown in the figure, and there is only
one configured host. However, you need to add 25 more hosts to the LAN.
Now, all 26 hosts must be able to get to the Internet at the same time.

\begin{figure}
\centering
\includegraphics{images/c13f006.jpg}
\caption{{\protect\hyperlink{c13.xhtmlux5cux23figureanchor13-6}{\textbf{FIGURE
13.6}} Last NAT example}}
\end{figure}

By looking at the configured network, use only the following inside
addresses to configure NAT on the Corp router to allow all hosts to
reach the Internet:

\begin{enumerate}
\tightlist
\item
  Inside globals: 198.18.41.129 through 198.18.41.134
\item
  Inside locals: 192.168.76.65 through 192.168.76.94
\end{enumerate}

This one is a bit more challenging because all we have to help us figure
out the configuration is the inside globals and the inside locals. But
even meagerly armed with these crumbs of information, plus the IP
addresses of the router interfaces shown in the figure, we can still
configure this correctly.

To do that, we must first determine what our block sizes are so we can
get our subnet mask for our NAT pool. This will also equip us to
configure the wildcard for the access list.

You should easily be able to see that the block size of the inside
globals is 8 and the block size of the inside locals is 32. Know that
it's critical not to stumble on this foundational information!

So we can configure NAT now that we have our block sizes:

\begin{verbatim}
ip nat pool Corp 198.18.41.129 198.18.41.134 netmask 255.255.255.248
ip nat inside source list 1 pool Corp overload
access-list 1 permit 192.168.76.64 0.0.0.31
\end{verbatim}

Since we had a block of only 8 for our pool, we had to use the
\texttt{overload} command to make sure all 26 hosts can get to the
Internet at the same time.

There is one other simple way to configure NAT, and I use this command
at my home office to connect to my ISP. One command line and it's done!
Here it is:

\begin{verbatim}
ip nat inside source list 1 int s0/0/0 overload
\end{verbatim}

I can't say enough how much I love efficiency, and being able to achieve
something cool using one measly line always makes me happy! My one
little powerfully elegant line essentially says, ``Use my outside local
as my inside global and overload it.'' Nice! Of course, I still had to
create ACL 1 and add the inside and outside interface commands to the
configuration, but this is a really nice, fast way to configure NAT if
you don't have a pool of addresses to use.

\subsection[Summary]{\texorpdfstring{\protect\hypertarget{c13.xhtmlux5cux23c13-sec-10}{}{}\protect\hypertarget{c13.xhtmlux5cux23Page_535}{}{}Summary}{Summary}}

Now this really was a fun chapter. Come on---admit it! You learned a lot
about Network Address Translation (NAT) and how it's configured as
static and dynamic as well as with Port Address Translation (PAT), also
called NAT Overload.

I also described how each flavor of NAT is used in a network as well as
how each type is configured.

We finished up by going through some verification and troubleshooting
commands. Now don't forget to practice all the wonderfully helpful labs
until you've got them nailed down tight!

\subsection[Exam
Essentials]{\texorpdfstring{\protect\hypertarget{c13.xhtmlux5cux23c13-sec-11}{}{}Exam
Essentials}{Exam Essentials}}

\textbf{Understand the term\emph{NAT}.} This may come as news to you,
because I didn't---okay, failed to---mention it earlier, but NAT has a
few nicknames. In the industry, it's referred to as network
masquerading, IP-masquerading, and (for those who are besieged with OCD
and compelled to spell everything out) Network Address Translation.
Whatever you want to dub it, basically, they all refer to the process of
rewriting the source/destination addresses of IP packets when they go
through a router or firewall. Just focus on the process that's occurring
and your understanding of it (i.e., the important part) and you're on it
for sure!

\textbf{Remember the three methods of NAT.} The three methods are
static, dynamic, and overloading; the latter is also called PAT.

\textbf{Understand static NAT.} This type of NAT is designed to allow
one-to-one mapping between local and global addresses.

\textbf{Understand dynamic NAT.} This version gives you the ability to
map a range of unregistered IP addresses to a registered IP address from
out of a pool of registered IP addresses.

\textbf{Understand overloading.} Overloading really is a form of dynamic
NAT that maps multiple unregistered IP addresses to a single registered
IP address (many-to-one) by using different ports. It's also known as
\emph{PAT}.

\subsection[Written Lab
13]{\texorpdfstring{\protect\hypertarget{c13.xhtmlux5cux23c13-sec-12}{}{}Written
Lab 13}{Written Lab 13}}

In this section, you'll complete the following lab to make sure you've
got the information and concepts contained within it fully dialed in:

Lab 13.1: NAT

You can find the answers to this lab in Appendix A, ``Answers to Written
Labs.''

\protect\hypertarget{c13.xhtmlux5cux23Page_536}{}{}In this section,
write the answers to the following questions:

\begin{enumerate}
\item
  What type of address translation can use only one address to allow
  thousands of hosts to be translated globally?
\item
  What command can you use to show the NAT translations as they occur on
  your router?
\item
  What command will show you the translation table?
\item
  What command will clear all your NAT entries from the translation
  table?
\item
  An inside local is before or after translation?
\item
  An inside global is before or after translation?
\item
  Which command can be used for troubleshooting and displays a summary
  of the NAT configuration as well as counts of active translation types
  and hits to an existing mapping?
\item
  What commands must be used on your router interfaces before NAT will
  translate addresses?
\item
  In the following output, what type of NAT is being used?

\begin{verbatim}
ip nat pool todd-nat 170.168.10.10 170.168.10.20 netmask 255.255.255.0
\end{verbatim}
\item
  Instead of the \texttt{netmask} command, you can use the
  \_\_\_\_\_\_\_\_\_\_\_\_\_ statement.
\end{enumerate}

\subsection[Hands-on
Labs]{\texorpdfstring{\protect\hypertarget{c13.xhtmlux5cux23c13-sec-13}{}{}Hands-on
Labs}{Hands-on Labs}}

I am going to use some basic routers for these labs, but really, almost
any Cisco router will work. Also, you can use the LammleSim IOS version
to run through all the labs in this (and every) chapter in this book.

Here is a list of the labs in this chapter:

\begin{enumerate}
\tightlist
\item
  Lab 13.1: Preparing for NAT
\item
  Lab 13.2: Configuring Dynamic NAT
\item
  Lab 13.3: Configuring PAT
\end{enumerate}

I am going to use the network shown in the following diagram for our
hands-on labs. I highly recommend you connect up some routers and run
through these labs. You will configure NAT on router Lab\_A to translate
the private IP address of 192.168.10.0 to a public address of
171.16.10.0.

\begin{figure}
\centering
\includegraphics{images/c13f007.jpg}
\caption{}
\end{figure}

\protect\hyperlink{c13.xhtmlux5cux23table13-3}{Table 13.3} shows the
commands we will use and the purpose of each command.

\protect\hypertarget{c13.xhtmlux5cux23Page_537}{}{}

{\protect\hyperlink{c13.xhtmlux5cux23tableanchor13-3}{\textbf{TABLE
13.3}} Command summary for NAT/PAT hands-on labs}

\begin{longtable}[]{@{}ll@{}}
\toprule
Command & Purpose\tabularnewline
\midrule
\endhead
\texttt{ip\ nat\ inside\ source\ list}
\emph{\texttt{acl\ pool}\texttt{name}} & Translates IPs that match the
ACL to the pool\tabularnewline
\texttt{ip\ nat\ inside\ source\ static}
\emph{\texttt{inside\_addr\ outside\_addr}} & Statically maps an inside
local address to an outside global address\tabularnewline
\texttt{ip\ nat\ pool} \emph{\texttt{name}} & Creates an address
pool\tabularnewline
\texttt{ip\ nat\ inside} & Sets an interface to be an inside
interface\tabularnewline
\texttt{ip\ nat\ outside} & Sets an interface to be an outside
interface\tabularnewline
\texttt{show\ ip\ nat\ translations} & Shows current NAT
translations\tabularnewline
\bottomrule
\end{longtable}

\subsubsection[Lab 13.1: Preparing for
NAT]{\texorpdfstring{\protect\hypertarget{c13.xhtmlux5cux23c13-sec-14}{}{}Lab
13.1: Preparing for NAT}{Lab 13.1: Preparing for NAT}}

In this lab, you'll set up your routers with IP addresses and RIP
routing.

\begin{enumerate}
\item
  Configure the routers with the IP addresses listed in the following
  table:

  \begin{longtable}[]{@{}lll@{}}
  \toprule
  \textbf{Router} & \textbf{Interface} & \textbf{IP
  Address}\tabularnewline
  \midrule
  \endhead
  ISP & S0 & 171.16.10.1/24\tabularnewline
  Lab\_A & S0/2 & 171.16.10.2/24\tabularnewline
  Lab\_A & S0/0 & 192.168.20.1/24\tabularnewline
  Lab\_B & S0 & 192.168.20.2/24\tabularnewline
  Lab\_B & E0 & 192.168.30.1/24\tabularnewline
  Lab\_C & E0 & 192.168.30.2/24\tabularnewline
  \bottomrule
  \end{longtable}

  After you configure IP addresses on the routers, you should be able to
  ping from router to router, but since we do not have a routing
  protocol running until the next step, you can verify only from one
  router to another but not through the network until RIP is set up. You
  can use any routing protocol you wish; I am just using RIP for
  simplicity's sake to get this up and running.
\item
  On Lab\_A, configure RIP routing, set a passive interface, and
  configure the default network.

\begin{verbatim}
Lab_A#config t
Lab_A(config)#router rip
Lab_A(config-router)#network 192.168.20.0
Lab_A(config-router)#network 171.16.0.0
Lab_A(config-router)#passive-interface s0/2
Lab_A(config-router)#exit
Lab_A(config)#ip default-network 171.16.10.1
\end{verbatim}

  The \texttt{passive-interface} command stops RIP updates from being
  sent to the ISP and the \texttt{ip\ default-network} command
  advertises a default network to the other routers so they know how to
  get to the Internet.
\item
  On Lab\_B, configure RIP routing:

\begin{verbatim}
Lab_B#config t
Lab_B(config)#router rip
Lab_B(config-router)#network 192.168.30.0
Lab_B(config-router)#network 192.168.20.0
\end{verbatim}
\item
  On Lab\_C, configure RIP routing:

\begin{verbatim}
Lab_C#config t
Lab_C(config)#router rip
Lab_C(config-router)#network 192.168.30.0
\end{verbatim}
\item
  On the ISP router, configure a default route to the corporate network:

\begin{verbatim}
ISP#config t
ISP(config)#ip route 0.0.0.0 0.0.0.0 s0
\end{verbatim}
\item
  Configure the ISP router so you can telnet into the router without
  being prompted for a password:

\begin{verbatim}
ISP#config t
ISP(config)#line vty 0 4
ISP(config-line)#no login
\end{verbatim}
\item
  Verify that you can ping from the ISP router to the Lab\_C router and
  from the Lab\_C router to the ISP router. If you cannot, troubleshoot
  your network.
\end{enumerate}

\subsubsection[Lab 13.2: Configuring Dynamic
NAT]{\texorpdfstring{\protect\hypertarget{c13.xhtmlux5cux23c13-sec-15}{}{}Lab
13.2: Configuring Dynamic NAT}{Lab 13.2: Configuring Dynamic NAT}}

In this lab, you'll configure dynamic NAT on the Lab\_A router.

\begin{enumerate}
\item
  Create a pool of addresses called GlobalNet on the Lab\_A router. The
  pool should contain a range of addresses of 171.16.10.50 through
  171.16.10.55.

\begin{verbatim}
Lab_A(config)#ip nat pool GlobalNet 171.16.10.50 171.16.10.55
net 255.255.255.0
\end{verbatim}
\item
  \protect\hypertarget{c13.xhtmlux5cux23Page_539}{}{}Create access list
  1. This list permits traffic from the 192.168.20.0 and 192.168.30.0
  network to be translated.

\begin{verbatim}
Lab_A(config)#access-list 1 permit 192.168.20.0 0.0.0.255
Lab_A(config)#access-list 1 permit 192.168.30.0 0.0.0.255
\end{verbatim}
\item
  Map the access list to the pool that was created.

\begin{verbatim}
Lab_A(config)#ip nat inside source list 1 pool GlobalNet
\end{verbatim}
\item
  Configure serial 0/0 as an inside NAT interface.

\begin{verbatim}
Lab_A(config)#int s0/0
Lab_A(config-if)#ip nat inside
\end{verbatim}
\item
  Configure serial 0/2 as an outside NAT interface.

\begin{verbatim}
Lab_A(config-if)#int s0/2
Lab_A(config-if)#ip nat outside
\end{verbatim}
\item
  Move the console connection to the Lab\_C router. Log in to the Lab\_C
  router. Telnet from the Lab\_C router to the ISP router.

\begin{verbatim}
Lab_C#telnet 171.16.10.1
\end{verbatim}
\item
  Move the console connection to the Lab\_B router. Log in to the Lab\_B
  router. Telnet from the Lab\_B router to the ISP router.

\begin{verbatim}
Lab_B#telnet 171.16.10.1
\end{verbatim}
\item
  Execute the command \texttt{show\ users} from the ISP router. (This
  shows who is accessing the VTY lines.)

\begin{verbatim}
ISP#show users
\end{verbatim}

  \begin{enumerate}
  \item
    What does it show as your source IP
    address?\_\_\_\_\_\_\_\_\_\_\_\_\_\_\_\_
  \item
    What is your real source IP
    address?\_\_\_\_\_\_\_\_\_\_\_\_\_\_\_\_\_\_

    The \texttt{show\ users} output should look something like this:

\begin{verbatim}
ISP>sh users
    Line       User       Host(s)              Idle       Location
   0 con 0                idle                 00:03:32
   2 vty 0                idle                 00:01:33 171.16.10.50
*  3 vty 1                idle                 00:00:09 171.16.10.51
  Interface  User      Mode                     Idle Peer Address
ISP>
\end{verbatim}

    \begin{center}\rule{0.5\linewidth}{0.5pt}\end{center}

    \protect\hypertarget{c13.xhtmlux5cux23Page_540}{}{}\includegraphics{images/note.png}Notice
    that there is a one-to-one translation. This means you must have a
    real IP address for every host that wants to get to the Internet,
    which is not typically possible.

    \begin{center}\rule{0.5\linewidth}{0.5pt}\end{center}
  \end{enumerate}
\item
  Leave the session open on the ISP router and connect to Lab\_A. (Use
  \textbf{Ctrl+Shift+6}, let go, and then press \textbf{X}.)
\item
  Log in to your Lab\_A router and view your current translations by
  entering the \texttt{show\ ip\ nat\ translations} command. You should
  see something like this:

\begin{verbatim}
Lab_A#sh ip nat translations
Pro Inside global      Inside local       Outside local      Outside global
--- 171.16.10.50       192.168.30.2       ---                ---
--- 171.16.10.51       192.168.20.2       ---                ---
Lab_A#
\end{verbatim}
\item
  If you turn on \texttt{debug\ ip\ nat} on the Lab\_A router and then
  ping through the router, you will see the actual NAT process take
  place, which will look something like this:

\begin{verbatim}
00:32:47: NAT*: s=192.168.30.2->171.16.10.50, d=171.16.10.1 [5]
00:32:47: NAT*: s=171.16.10.1, d=171.16.10.50->192.168.30.2
\end{verbatim}
\end{enumerate}

\subsubsection[Lab 13.3: Configuring
PAT]{\texorpdfstring{\protect\hypertarget{c13.xhtmlux5cux23c13-sec-16}{}{}Lab
13.3: Configuring PAT}{Lab 13.3: Configuring PAT}}

In this lab, you'll configure PAT on the Lab\_A router. We will use PAT
because we don't want a one-to-one translation, which uses just one IP
address for every user on the network.

\begin{enumerate}
\item
  On the Lab\_A router, delete the translation table and remove the
  dynamic NAT pool.

\begin{verbatim}
Lab_A#clear ip nat translations *
Lab_A#config t
Lab_A(config)#no ip nat pool GlobalNet 171.16.10.50
171.16.10.55 netmask 255.255.255.0
Lab_A(config)#no ip nat inside source list 1 pool GlobalNet
\end{verbatim}
\item
  On the Lab\_A router, create a NAT pool with one address called
  Lammle. The pool should contain a single address, 171.16.10.100. Enter
  the following command:

\begin{verbatim}
Lab_A#config t
Lab_A(config)#ip nat pool Lammle 171.16.10.100 171.16.10.100
net 255.255.255.0
\end{verbatim}
\item
  Create access list 2. It should permit networks 192.168.20.0 and
  192.168.30.0 to be translated.

\begin{verbatim}
Lab_A(config)#access-list 2 permit 192.168.20.0 0.0.0.255
Lab_A(config)#access-list 2 permit 192.168.30.0 0.0.0.255
\end{verbatim}
\item
  Map access list 2 to the new pool, allowing PAT to occur by using the
  \texttt{overload} command.

\begin{verbatim}
Lab_A(config)#ip nat inside source list 2 pool Lammle overload
\end{verbatim}
\item
  Log in to the Lab\_C router and telnet to the ISP router; also, log in
  to the Lab\_B router and telnet to the ISP router.
\item
  From the ISP router, use the \texttt{show\ users} command. The output
  should look like this:

\begin{verbatim}
ISP>sh users
    Line       User       Host(s)              Idle       Location
*  0 con 0                idle                 00:00:00
   2 vty 0                idle                 00:00:39 171.16.10.100
   4 vty 2                idle                 00:00:37 171.16.10.100
 
  Interface  User      Mode               Idle Peer Address
 
ISP>
\end{verbatim}
\item
  From the Lab\_A router, use the \texttt{show\ ip\ nat\ translations}
  command.

\begin{verbatim}
Lab_A#sh ip nat translations
Pro Inside global  Inside local  Outside local Outside global
tcp 171.16.10.100:11001 192.168.20.2:11001 171.16.10.1:23    
171.16.10.1:23
tcp 171.16.10.100:11002 192.168.30.2:11002 171.16.10.1:23    
171.16.10.1:23
\end{verbatim}
\item
  Also make sure the \texttt{debug\ ip\ nat} command is on for the
  Lab\_A router. If you ping from the Lab\_C router to the ISP router,
  the output will look like this:

\begin{verbatim}
01:12:36: NAT: s=192.168.30.2->171.16.10.100, d=171.16.10.1 [35]
01:12:36: NAT*: s=171.16.10.1, d=171.16.10.100->192.168.30.2 [35]
01:12:36: NAT*: s=192.168.30.2->171.16.10.100, d=171.16.10.1 [36]
01:12:36: NAT*: s=171.16.10.1, d=171.16.10.100->192.168.30.2 [36]
01:12:36: NAT*: s=192.168.30.2->171.16.10.100, d=171.16.10.1 [37]
01:12:36: NAT*: s=171.16.10.1, d=171.16.10.100->192.168.30.2 [37]
01:12:36: NAT*: s=192.168.30.2->171.16.10.100, d=171.16.10.1 [38]
01:12:36: NAT*: s=171.16.10.1, d=171.16.10.100->192.168.30.2 [38]
01:12:37: NAT*: s=192.168.30.2->171.16.10.100, d=171.16.10.1 [39]
01:12:37: NAT*: s=171.16.10.1, d=171.16.10.100->192.168.30.2 [39]
\end{verbatim}
\end{enumerate}

\subsection[Review
Questions]{\texorpdfstring{\protect\hypertarget{c13.xhtmlux5cux23c13-sec-17}{}{}Review
Questions}{Review Questions}}

\begin{center}\rule{0.5\linewidth}{0.5pt}\end{center}

\protect\hypertarget{c13.xhtmlux5cux23Page_542}{}{}\includegraphics{images/note.png}The
following questions are designed to test your understanding of this
chapter's material. For more information on how to get additional
questions, please see \texttt{www.lammle.com/ccna}.

\begin{center}\rule{0.5\linewidth}{0.5pt}\end{center}

You can find the answers to these questions in Appendix B, ``Answers to
Review Questions.''

\begin{enumerate}
\item
  Which of the following are disadvantages of using NAT? (Choose three.)

  \begin{enumerate}
  \tightlist
  \item
    Translation introduces switching path delays.
  \item
    NAT conserves legally registered addresses.
  \item
    NAT causes loss of end-to-end IP traceability.
  \item
    NAT increases flexibility when connecting to the Internet.
  \item
    Certain applications will not function with NAT enabled.
  \item
    NAT reduces address overlap occurrence.
  \end{enumerate}
\item
  Which of the following are advantages of using NAT? (Choose three.)

  \begin{enumerate}
  \tightlist
  \item
    Translation introduces switching path delays.
  \item
    NAT conserves legally registered addresses.
  \item
    NAT causes loss of end-to-end IP traceability.
  \item
    NAT increases flexibility when connecting to the Internet.
  \item
    Certain applications will not function with NAT enabled.
  \item
    NAT remedies address overlap occurrence.
  \end{enumerate}
\item
  Which command will allow you to see real-time translations on your
  router?

  \begin{enumerate}
  \tightlist
  \item
    \texttt{show\ ip\ nat\ translations}
  \item
    \texttt{show\ ip\ nat\ statistics}
  \item
    \texttt{debug\ ip\ nat}
  \item
    \texttt{clear\ ip\ nat\ translations\ *}
  \end{enumerate}
\item
  Which command will show you all the translations active on your
  router?

  \begin{enumerate}
  \tightlist
  \item
    \texttt{show\ ip\ nat\ translations}
  \item
    \texttt{show\ ip\ nat\ statistics}
  \item
    \texttt{debug\ ip\ nat}
  \item
    \texttt{clear\ ip\ nat\ translations\ *}
  \end{enumerate}
\item
  Which command will clear all the translations active on your router?

  \begin{enumerate}
  \tightlist
  \item
    \texttt{show\ ip\ nat\ translations}
  \item
    \texttt{show\ ip\ nat\ statistics}
  \item
    \protect\hypertarget{c13.xhtmlux5cux23Page_543}{}{}\texttt{debug\ ip\ nat}
  \item
    \texttt{clear\ ip\ nat\ translations\ *}
  \end{enumerate}
\item
  Which command will show you the summary of the NAT configuration?

  \begin{enumerate}
  \tightlist
  \item
    \texttt{show\ ip\ nat\ translations}
  \item
    \texttt{show\ ip\ nat\ statistics}
  \item
    \texttt{debug\ ip\ nat}
  \item
    \texttt{clear\ ip\ nat\ translations\ *}
  \end{enumerate}
\item
  Which command will create a dynamic pool named Todd that will provide
  you with 30 global addresses?

  \begin{enumerate}
  \tightlist
  \item
    \texttt{ip\ nat\ pool\ Todd\ 171.16.10.65\ 171.16.10.94\ net\ 255.255.255.240}
  \item
    \texttt{ip\ nat\ pool\ Todd\ 171.16.10.65\ 171.16.10.94\ net\ 255.255.255.224}
  \item
    \texttt{ip\ nat\ pool\ todd\ 171.16.10.65\ 171.16.10.94\ net\ 255.255.255.224}
  \item
    \texttt{ip\ nat\ pool\ Todd\ 171.16.10.1\ 171.16.10.254\ net\ 255.255.255.0}
  \end{enumerate}
\item
  Which of the following are methods of NAT? (Choose three.)

  \begin{enumerate}
  \tightlist
  \item
    Static
  \item
    IP NAT pool
  \item
    Dynamic
  \item
    NAT double-translation
  \item
    Overload
  \end{enumerate}
\item
  When creating a pool of global addresses, which of the following can
  be used instead of the \texttt{netmask} command?

  \begin{enumerate}
  \tightlist
  \item
    \texttt{/} (slash notation)
  \item
    \texttt{prefix-length}
  \item
    \texttt{no\ mask}
  \item
    \texttt{block-size}
  \end{enumerate}
\item
  Which of the following would be a good starting point for
  troubleshooting if your router is not translating?

  \begin{enumerate}
  \tightlist
  \item
    Reboot.
  \item
    Call Cisco.
  \item
    Check your interfaces for the correct configuration.
  \item
    Run the \texttt{debug\ all} command.
  \end{enumerate}
\item
  Which of the following would be good reasons to run NAT? (Choose
  three.)

  \begin{enumerate}
  \tightlist
  \item
    You need to connect to the Internet and your hosts don't have
    globally unique IP addresses.
  \item
    You change to a new ISP that requires you to renumber your network.
  \item
    You don't want any hosts connecting to the Internet.
  \item
    You require two intranets with duplicate addresses to merge.
  \end{enumerate}
\item
  \protect\hypertarget{c13.xhtmlux5cux23Page_544}{}{}Which of the
  following is considered to be the inside host's address after
  translation?

  \begin{enumerate}
  \tightlist
  \item
    Inside local
  \item
    Outside local
  \item
    Inside global
  \item
    Outside global
  \end{enumerate}
\item
  Which of the following is considered to be the inside host's address
  before translation?

  \begin{enumerate}
  \tightlist
  \item
    Inside local
  \item
    Outside local
  \item
    Inside global
  \item
    Outside global
  \end{enumerate}
\item
  By looking at the following output, determine which of the following
  commands would allow dynamic translations?

\begin{verbatim}
Router#show ip nat trans
Pro   Inside global   Inside local   Outside local Outside global
---   1.1.128.1       10.1.1.1       ---           ---
---   1.1.130.178     10.1.1.2       ---           ---
---   1.1.129.174     10.1.1.10      ---           ---
---   1.1.130.101     10.1.1.89      ---           ---
---   1.1.134.169     10.1.1.100     ---           ---
---   1.1.135.174     10.1.1.200      ---           ---
\end{verbatim}

  \begin{enumerate}
  \tightlist
  \item
    \texttt{ip\ nat\ inside\ source\ pool\ todd\ 1.1.128.1\ 1.1.135.254\ prefix-length\ 19}
  \item
    \texttt{ip\ nat\ pool\ todd\ 1.1.128.1\ 1.1.135.254\ prefix-length\ 19}
  \item
    \texttt{ip\ nat\ pool\ todd\ 1.1.128.1\ 1.1.135.254\ prefix-length\ 18}
  \item
    \texttt{ip\ nat\ pool\ todd\ 1.1.128.1\ 1.1.135.254\ prefix-length\ 21}
  \end{enumerate}
\item
  Your inside locals are not being translated to the inside global
  addresses. Which of the following commands will show you if your
  inside globals are allowed to use the NAT pool?

\begin{verbatim}
ip nat pool Corp 198.18.41.129 198.18.41.134 netmask 255.255.255.248
ip nat inside source list 100 int s0/0 Corp overload
\end{verbatim}

  \begin{enumerate}
  \tightlist
  \item
    \texttt{debug\ ip\ nat}
  \item
    \texttt{show\ access-list}
  \item
    \texttt{show\ ip\ nat\ translation}
  \item
    \texttt{show\ ip\ nat\ statistics}
  \end{enumerate}
\item
  \protect\hypertarget{c13.xhtmlux5cux23Page_545}{}{}Which command would
  you place on the interface of a private network?

  \begin{enumerate}
  \tightlist
  \item
    \texttt{ip\ nat\ inside}
  \item
    \texttt{ip\ nat\ outside}
  \item
    \texttt{ip\ outside\ global}
  \item
    \texttt{ip\ inside\ local}
  \end{enumerate}
\item
  Which command would you place on an interface connected to the
  Internet?

  \begin{enumerate}
  \tightlist
  \item
    \texttt{ip\ nat\ inside}
  \item
    \texttt{ip\ nat\ outside}
  \item
    \texttt{ip\ outside\ global}
  \item
    \texttt{ip\ inside\ local}
  \end{enumerate}
\item
  Port Address Translation is also called what?

  \begin{enumerate}
  \tightlist
  \item
    NAT Fast
  \item
    NAT Static
  \item
    NAT Overload
  \item
    Overloading Static
  \end{enumerate}
\item
  What does the asterisk (*) represent in the following output?

\begin{verbatim}
NAT*: s=172.16.2.2, d=192.168.2.1->10.1.1.1 [1]
\end{verbatim}

  \begin{enumerate}
  \tightlist
  \item
    The packet was destined for a local interface on the router.
  \item
    The packet was translated and fast-switched to the destination.
  \item
    The packet attempted to be translated but failed.
  \item
    The packet was translated but there was no response from the remote
    host.
  \end{enumerate}
\item
  Which of the following needs to be added to the configuration to
  enable PAT?

\begin{verbatim}
ip nat pool Corp 198.18.41.129 198.18.41.134 netmask 255.255.255.248
access-list 1 permit 192.168.76.64 0.0.0.31
\end{verbatim}

  \begin{enumerate}
  \tightlist
  \item
    \texttt{ip\ nat\ pool\ inside\ overload}
  \item
    \texttt{ip\ nat\ inside\ source\ list\ 1\ pool\ Corp\ overload}
  \item
    \texttt{ip\ nat\ pool\ outside\ overload}
  \item
    \texttt{ip\ nat\ pool\ Corp\ 198.41.129\ net\ 255.255.255.0\ overload}
  \end{enumerate}
\end{enumerate}

\protect\hypertarget{c14.xhtml}{}{}

\section[{Chapter 14}\\
{Internet Protocol Version 6
(IPv6)}]{\texorpdfstring{\protect\hypertarget{c14.xhtmlux5cux23c14}{}{}\protect\hypertarget{c14.xhtmlux5cux23Page_547}{}{}{Chapter
14}\\
{Internet Protocol Version 6
(IPv6)}}{Chapter 14 Internet Protocol Version 6 (IPv6)}}

\begin{center}\rule{0.5\linewidth}{0.5pt}\end{center}

\subsection{THE FOLLOWING ICND1 EXAM TOPICS ARE COVERED IN THIS
CHAPTER:}

\begin{enumerate}
\tightlist
\item
  \includegraphics{images/tick.png} \textbf{1.11 Identify the
  appropriate IPv6 addressing scheme to satisfy addressing requirements
  in a LAN/WAN environment}
\item
  \includegraphics{images/tick.png} \textbf{1.12 Configure, verify, and
  troubleshoot IPv6 addressing}
\item
  \includegraphics{images/tick.png} \textbf{1.13 Configure and verify
  IPv6 Stateless Address Auto Configuration}
\item
  \includegraphics{images/tick.png} \textbf{1.14 Compare and contrast
  IPv6 address types}

  \begin{enumerate}
  \tightlist
  \item
    \includegraphics{images/square.png} 1.14.a Global unicast
  \item
    \includegraphics{images/square.png} 1.14.b Unique local
  \item
    \includegraphics{images/square.png} 1.14.c Link local
  \item
    \includegraphics{images/square.png} 1.14.d Multicast
  \item
    \includegraphics{images/square.png} 1.14.e Modified EUI 64
  \item
    \includegraphics{images/square.png} 1.14.f Autoconfiguration
  \item
    \includegraphics{images/square.png} 1.14.g Anycast
  \end{enumerate}
\item
  \includegraphics{images/tick.png} \textbf{3.6 Configure, verify, and
  troubleshoot IPv4 and IPv6 static routing}

  \begin{enumerate}
  \tightlist
  \item
    \includegraphics{images/square.png} 3.6.a Default route
  \end{enumerate}
\end{enumerate}

\protect\hypertarget{c14.xhtmlux5cux23Page_548}{}{}\includegraphics{images/intro.png}We've
covered a lot of ground in this book, and though the journey has been
tough at times, it's been well worth it! But our networking expedition
isn't quite over yet because we still have the vastly important frontier
of IPv6 to explore. There's still some expansive territory to cover with
this sweeping new subject, so gear up and get ready to discover all you
need to know about IPv6. Understanding IPv6 is vital now, so you'll be
much better equipped and prepared to meet today's real-world networking
challenges as well as to ace the exam. This final chapter is packed and
brimming with all the IPv6 information you'll need to complete your
Cisco exam trek successfully, so get psyched---we're in the home
stretch!

I probably don't need to say this, but I will anyway because I really
want to go the distance and do everything I can to ensure that you
arrive and achieve . . . You absolutely must have a solid hold on IPv4
by now, but if you're still not confident with it, or feel you could use
a refresher, just page back to the chapters on TCP/IP and subnetting.
And if you're not crystal clear on the address problems inherent to
IPv4, you really need to review Chapter 13, ``Network Address
Translation (NAT)'', before we decamp for this chapter's IPv6 summit
push!

People refer to IPv6 as ``the next-generation Internet protocol,'' and
it was originally created as the solution to IPv4's inevitable and
impending address-exhaustion crisis. Though you've probably heard a
thing or two about IPv6 already, it has been improved even further in
the quest to bring us the flexibility, efficiency, capability, and
optimized functionality that can effectively meet our world's seemingly
insatiable thirst for ever-evolving technologies and increasing access.
The capacity of its predecessor, IPv4, pales wan and ghostly in
comparison, which is why IPv4 is destined to fade into history
completely, making way for IPv6 and the future.

The IPv6 header and address structure has been completely overhauled,
and many of the features that were basically just afterthoughts and
addenda in IPv4 are now included as full-blown standards in IPv6. It's
power-packed, well equipped with robust and elegant features, poised and
prepared to manage the mind-blowing demands of the Internet to come!

After an introduction like that, I understand if you're a little
apprehensive, but I promise---really---to make this chapter and its VIP
topic pretty painless for you. In fact, you might even find yourself
actually enjoying it---I definitely did! Because IPv6 is so complex,
while still being so elegant, innovative, and powerful, it fascinates me
like some weird combination of a sleek, new Aston Martin and a riveting
futuristic novel. Hopefully you'll experience this chapter as an awesome
ride and enjoy reading it as much as I did writing it!

\begin{center}\rule{0.5\linewidth}{0.5pt}\end{center}

\includegraphics{images/note.png}To find up-to-the-minute updates for
this chapter, please see \texttt{www.lammle.com/ccna} or the book's web
page at \texttt{www.sybex.com/go/ccna}.

\begin{center}\rule{0.5\linewidth}{0.5pt}\end{center}

\subsection[Why Do We Need
IPv6?]{\texorpdfstring{\protect\hypertarget{c14.xhtmlux5cux23c14-sec-1}{}{}\protect\hypertarget{c14.xhtmlux5cux23Page_549}{}{}Why
Do We Need IPv6?}{Why Do We Need IPv6?}}

Well, the short answer is because we need to communicate and our current
system isn't really cutting it anymore. It's kind of like the Pony
Express trying to compete with airmail! Consider how much time and
effort we've been investing for years while we scratch our heads to
resourcefully come up with slick new ways to conserve bandwidth and IP
addresses. Sure, variable length subnet masks (VLSMs) are wonderful and
cool, but they're really just another invention to help us cope while we
desperately struggle to overcome the worsening address drought.

I'm not exaggerating, at all, about how dire things are getting, because
it's simply reality. The number of people and devices that connect to
networks increases dramatically each and every day, which is not a bad
thing. We're just finding new and exciting ways to communicate to more
people, more often, which is good thing. And it's not likely to go away
or even decrease in the littlest bit, because communicating and making
connections are, in fact, basic human needs---they're in our very
nature. But with our numbers increasing along with the rising tide of
people joining the communications party increasing as well, the forecast
for our current system isn't exactly clear skies and smooth sailing.
IPv4, upon which our ability to do all this connecting and communicating
is presently dependent, is quickly running out of addresses for us to
use.

IPv4 has only about 4.3 billion addresses available---in theory---and we
know that we don't even get to use most of those! Sure, the use of
Classless Inter-Domain Routing (CIDR) and Network Address Translation
(NAT) has helped to extend the inevitable dearth of addresses, but we
will still run out of them, and it's going to happen within a few years.
China is barely online, and we know there's a huge population of people
and corporations there that surely want to be. There are myriad reports
that give us all kinds of numbers, but all you really need to think
about to realize that I'm not just being an alarmist is this: there are
about 7 billion people in the world today, and it's estimated that only
just over 10 percent of that population is currently connected to the
Internet---wow!

That statistic is basically screaming at us the ugly truth that based on
IPv4's capacity, every person can't even have a computer, let alone all
the other IP devices we use with them! I have more than one computer,
and it's pretty likely that you do too, and I'm not even including
phones, laptops, game consoles, fax machines, routers, switches, and a
mother lode of other devices we use every day into the mix! So I think
I've made it pretty clear that we've got to do something before we run
out of addresses and lose the ability to connect with each other as we
know it. And that ``something'' just happens to be implementing IPv6.

\subsection[The Benefits and Uses of
IPv6]{\texorpdfstring{\protect\hypertarget{c14.xhtmlux5cux23c14-sec-2}{}{}The
Benefits and Uses of IPv6}{The Benefits and Uses of IPv6}}

So what's so fabulous about IPv6? Is it really the answer to our coming
dilemma? Is it really worth it to upgrade from IPv4? All good
questions---you may even think of a few more. Of course, there's going
to be that group of people with the time-tested ``resistance
\protect\hypertarget{c14.xhtmlux5cux23Page_550}{}{}to change syndrome,''
but don't listen to them. If we had done that years ago, we'd still be
waiting weeks, even months for our mail to arrive via horseback.
Instead, just know that the answer is a resounding \emph{yes}, it is
really the answer, and it is worth the upgrade! Not only does IPv6 give
us lots of addresses (3.4 × 10\textsuperscript{38} = definitely enough),
there are tons of other features built into this version that make it
well worth the cost, time, and effort required to migrate to it.

Today's networks, as well as the Internet, have a ton of unforeseen
requirements that simply weren't even considerations when IPv4 was
created. We've tried to compensate with a collection of add-ons that can
actually make implementing them more difficult than they would be if
they were required by a standard. By default, IPv6 has improved upon and
included many of those features as standard and mandatory. One of these
sweet new standards is IPsec---a feature that provides end-to-end
security.

But it's the efficiency features that are really going to rock the
house! For starters, the headers in an IPv6 packet have half the fields,
and they are aligned to 64 bits, which gives us some seriously souped-up
processing speed. Compared to IPv4, lookups happen at light speed! Most
of the information that used to be bound into the IPv4 header was taken
out, and now you can choose to put it, or parts of it, back into the
header in the form of optional extension headers that follow the basic
header fields.

And of course there's that whole new universe of addresses---the 3.4 ×
10\textsuperscript{38} I just mentioned---but where did we get them? Did
some genie just suddenly arrive and make them magically appear? That
huge proliferation of addresses had to come from somewhere! Well it just
so happens that IPv6 gives us a substantially larger address space,
meaning the address itself is a whole lot bigger---four times bigger as
a matter of fact! An IPv6 address is actually 128 bits in length, and no
worries---I'm going to break down the address piece by piece and show
you exactly what it looks like coming up in the section ``IPv6
Addressing and Expressions.'' For now, let me just say that all that
additional room permits more levels of hierarchy inside the address
space and a more flexible addressing architecture. It also makes routing
much more efficient and scalable because the addresses can be aggregated
a lot more effectively. And IPv6 also allows multiple addresses for
hosts and networks. This is especially important for enterprises
veritably drooling for enhanced access and availability. Plus, the new
version of IP now includes an expanded use of multicast
communication---one device sending to many hosts or to a select
group---that joins in to seriously boost efficiency on networks because
communications will be more specific.

IPv4 uses broadcasts quite prolifically, causing a bunch of problems,
the worst of which is of course the dreaded broadcast storm. This is
that uncontrolled deluge of forwarded broadcast traffic that can bring
an entire network to its knees and devour every last bit of bandwidth!
Another nasty thing about broadcast traffic is that it interrupts each
and every device on the network. When a broadcast is sent out, every
machine has to stop what it's doing and respond to the traffic whether
the broadcast is relevant to it or not.

But smile assuredly, everyone. There's no such thing as a broadcast in
IPv6 because it uses multicast traffic instead. And there are two other
types of communications as well: unicast, which is the same as it is in
IPv4, and a new type called \emph{anycast}. Anycast communication allows
the same address to be placed on more than one device so that when
traffic is sent to the device service addressed in this way, it's routed
to the nearest host that shares
\protect\hypertarget{c14.xhtmlux5cux23Page_551}{}{}the same address. And
this is just the beginning---we'll get into the various types of
communication later in the section called ``Address Types.''

\subsection[IPv6 Addressing and
Expressions]{\texorpdfstring{\protect\hypertarget{c14.xhtmlux5cux23c14-sec-3}{}{}IPv6
Addressing and Expressions}{IPv6 Addressing and Expressions}}

Just as understanding how IP addresses are structured and used is
critical with IPv4 addressing, it's also vital when it comes to IPv6.
You've already read about the fact that at 128 bits, an IPv6 address is
much larger than an IPv4 address. Because of this, as well as the new
ways the addresses can be used, you've probably guessed that IPv6 will
be more complicated to manage. But no worries! As I said, I'll break
down the basics and show you what the address looks like and how you can
write it as well as many of its common uses. It's going to be a little
weird at first, but before you know it, you'll have it nailed!

So let's take a look at
\protect\hyperlink{c14.xhtmlux5cux23figure14-1}{Figure 14.1}, which has
a sample IPv6 address broken down into sections.

\begin{figure}
\centering
\includegraphics{images/c14f001.jpg}
\caption{{\protect\hyperlink{c14.xhtmlux5cux23figureanchor14-1}{\textbf{FIGURE
14.1}} IPv6 address example}}
\end{figure}

As you can clearly see, the address is definitely much larger. But what
else is different? Well, first, notice that it has eight groups of
numbers instead of four and also that those groups are separated by
colons instead of periods. And hey, wait a second . . . there are
letters in that address! Yep, the address is expressed in hexadecimal
just like a MAC address is, so you could say this address has eight
16-bit hexadecimal colon-delimited blocks. That's already quite a
mouthful, and you probably haven't even tried to say the address out
loud yet!

\begin{center}\rule{0.5\linewidth}{0.5pt}\end{center}

\includegraphics{images/note.png}There are four hexadecimal characters
(16 bits) in each IPv6 field (with eight fields total), separated by
colons.

\begin{center}\rule{0.5\linewidth}{0.5pt}\end{center}

\subsubsection[Shortened
Expression]{\texorpdfstring{\protect\hypertarget{c14.xhtmlux5cux23c14-sec-4}{}{}Shortened
Expression}{Shortened Expression}}

The good news is there are a few tricks to help rescue us when writing
these monster addresses. For one thing, you can actually leave out parts
of the address to abbreviate it, but to get away with doing that you
have to follow a couple of rules. First, you can drop
\protect\hypertarget{c14.xhtmlux5cux23Page_552}{}{}any leading zeros in
each of the individual blocks. After you do that, the sample address
from earlier would then look like this:

\begin{verbatim}
2001:db8:3c4d:12:0:0:1234:56ab
\end{verbatim}

That's a definite improvement---at least we don't have to write all of
those extra zeros! But what about whole blocks that don't have anything
in them except zeros? Well, we can kind of lose those too---at least
some of them. Again referring to our sample address, we can remove the
two consecutive blocks of zeros by replacing them with a doubled colon,
like this:

\begin{verbatim}
2001:db8:3c4d:12::1234:56ab
\end{verbatim}

Cool---we replaced the blocks of all zeros with a doubled colon. The
rule you have to follow to get away with this is that you can replace
only one contiguous block of such zeros in an address. So if my address
has four blocks of zeros and each of them were separated, I just don't
get to replace them all because I can replace only one contiguous block
with a doubled colon. Check out this example:

\begin{verbatim}
2001:0000:0000:0012:0000:0000:1234:56ab
\end{verbatim}

And just know that you \emph{can't} do this:

\begin{verbatim}
2001::12::1234:56ab
\end{verbatim}

Instead, the best you can do is this:

\begin{verbatim}
2001::12:0:0:1234:56ab
\end{verbatim}

The reason the preceding example is our best shot is that if we remove
two sets of zeros, the device looking at the address will have no way of
knowing where the zeros go back in. Basically, the router would look at
the incorrect address and say, ``Well, do I place two blocks into the
first set of doubled colons and two into the second set, or do I place
three blocks into the first set and one block into the second set?'' And
on and on it would go because the information the router needs just
isn't there.

\subsubsection[Address
Types]{\texorpdfstring{\protect\hypertarget{c14.xhtmlux5cux23c14-sec-5}{}{}Address
Types}{Address Types}}

We're all familiar with IPv4's unicast, broadcast, and multicast
addresses that basically define who or at least how many other devices
we're talking to. But as I mentioned, IPv6 modifies that trio and
introduces the anycast. Broadcasts, as we know them, have been
eliminated in IPv6 because of their cumbersome inefficiency and basic
tendency to drive us insane!

So let's find out what each of these types of IPv6 addressing and
communication methods do for us:

\textbf{Unicast} Packets addressed to a unicast address are delivered to
a single interface. For load balancing, multiple interfaces across
several devices can use the same address, but we'll call
\protect\hypertarget{c14.xhtmlux5cux23Page_553}{}{}that an anycast
address. There are a few different types of unicast addresses, but we
don't need to get further into that here.

\textbf{Global unicast addresses (2000::/3)} These are your typical
publicly routable addresses and they're the same as in IPv4. Global
addresses start at 2000::/3.
\protect\hyperlink{c14.xhtmlux5cux23figure14-2}{Figure 14.2} shows how a
unicast address breaks down. The ISP can provide you with a minimum /48
network ID, which in turn provides you 16-bits to create a unique 64-bit
router interface address. The last 64-bits are the unique host ID.

\begin{figure}
\centering
\includegraphics{images/c14f002.jpg}
\caption{{\protect\hyperlink{c14.xhtmlux5cux23figureanchor14-2}{\textbf{FIGURE
14.2}} IPv6 global unicast addresses}}
\end{figure}

\textbf{Link-local addresses (FE80::/10)} These are like the Automatic
Private IP Address (APIPA) addresses that Microsoft uses to
automatically provide addresses in IPv4 in that they're not meant to be
routed. In IPv6 they start with FE80::/10, as shown in
\protect\hyperlink{c14.xhtmlux5cux23figure14-3}{Figure 14.3}. Think of
these addresses as handy tools that give you the ability to throw a
temporary LAN together for meetings or create a small LAN that's not
going to be routed but still needs to share and access files and
services locally.

\begin{figure}
\centering
\includegraphics{images/c14f003.jpg}
\caption{{\protect\hyperlink{c14.xhtmlux5cux23figureanchor14-3}{\textbf{FIGURE
14.3}} IPv6 link local FE80::/10: The first 10 bits define the address
type.}}
\end{figure}

\textbf{Unique local addresses (FC00::/7)} These addresses are also
intended for nonrouting purposes over the Internet, but they are nearly
globally unique, so it's unlikely you'll ever have one of them overlap.
Unique local addresses were designed to replace site-local addresses, so
they basically do almost exactly what IPv4 private addresses do: allow
communication throughout a site while being routable to multiple local
networks. Site-local addresses were deprecated as of September 2004.

\textbf{\protect\hypertarget{c14.xhtmlux5cux23Page_554}{}{}Multicast
(FF00::/8)} Again, as in IPv4, packets addressed to a multicast address
are delivered to all interfaces tuned into the multicast address.
Sometimes people call them ``one-to-many'' addresses. It's really easy
to spot a multicast address in IPv6 because they always start with
\emph{FF}. We'll get deeper into multicast operation coming up, in ``How
IPv6 Works in an Internetwork.''

\textbf{Anycast} Like multicast addresses, an anycast address identifies
multiple interfaces on multiple devices. But there's a big difference:
the anycast packet is delivered to only one device---actually, to the
closest one it finds defined in terms of routing distance. And again,
this address is special because you can apply a single address to more
than one host. These are referred to as ``one-to-nearest'' addresses.
Anycast addresses are typically only configured on routers, never hosts,
and a source address could never be an anycast address. Of note is that
the IETF did reserve the top 128 addresses for each /64 for use with
anycast addresses.

You're probably wondering if there are any special, reserved addresses
in IPv6 because you know they're there in IPv4. Well there are---plenty
of them! Let's go over those now.

\subsubsection[Special
Addresses]{\texorpdfstring{\protect\hypertarget{c14.xhtmlux5cux23c14-sec-6}{}{}Special
Addresses}{Special Addresses}}

I'm going to list some of the addresses and address ranges (in
\protect\hyperlink{c14.xhtmlux5cux23table14-1}{Table 14.1}) that you
should definitely make sure to remember because you'll eventually use
them. They're all special or reserved for a specific use, but unlike
IPv4, IPv6 gives us a galaxy of addresses, so reserving a few here and
there doesn't hurt at all!

{\protect\hyperlink{c14.xhtmlux5cux23tableanchor14-1}{\textbf{TABLE
14.1}} Special IPv6 addresses}

\begin{longtable}[]{@{}ll@{}}
\toprule
Address & Meaning\tabularnewline
\midrule
\endhead
0:0:0:0:0:0:0:0 & Equals ::. This is the equivalent of IPv4's 0.0.0.0
and is typically the source address of a host before the host receives
an IP address when you're using DHCP-driven stateful
configuration.\tabularnewline
0:0:0:0:0:0:0:1 & Equals ::1. The equivalent of 127.0.0.1 in
IPv4.\tabularnewline
0:0:0:0:0:0:192.168.100.1 & This is how an IPv4 address would be written
in a mixed IPv6/IPv4 network environment.\tabularnewline
2000::/3 & The global unicast address range.\tabularnewline
FC00::/7 & The unique local unicast range.\tabularnewline
FE80::/10 & The link-local unicast range.\tabularnewline
FF00::/8 & The multicast range.\tabularnewline
\protect\hypertarget{c14.xhtmlux5cux23Page_555}{}{}3FFF:FFFF::/32 &
Reserved for examples and documentation.\tabularnewline
2001:0DB8::/32 & Also reserved for examples and
documentation.\tabularnewline
2002::/16 & Used with 6-to-4 tunneling, which is an IPv4-to-IPv6
transition system. The structure allows IPv6 packets to be transmitted
over an IPv4 network without the need to configure explicit
tunnels.\tabularnewline
\bottomrule
\end{longtable}

\begin{center}\rule{0.5\linewidth}{0.5pt}\end{center}

\includegraphics{images/note.png}When you run IPv4 and IPv6 on a router,
you have what is called ``dual-stack.''

\begin{center}\rule{0.5\linewidth}{0.5pt}\end{center}

Let me show you how IPv6 actually works in an internetwork. We all know
how IPv4 works, so let's see what's new!

\subsection[How IPv6 Works in an
Internetwork]{\texorpdfstring{\protect\hypertarget{c14.xhtmlux5cux23c14-sec-7}{}{}How
IPv6 Works in an Internetwork}{How IPv6 Works in an Internetwork}}

It's time to explore the finer points of IPv6. A great place to start is
by showing you how to address a host and what gives it the ability to
find other hosts and resources on a network.

I'll also demonstrate a device's ability to automatically address
itself---something called stateless autoconfiguration---plus another
type of autoconfiguration known as stateful. Keep in mind that stateful
autoconfiguration uses a DHCP server in a very similar way to how it's
used in an IPv4 configuration. I'll also show you how Internet Control
Message Protocol (ICMP) and multicasting works for us in an IPv6 network
environment.

\subsubsection[Manual Address
Assignment]{\texorpdfstring{\protect\hypertarget{c14.xhtmlux5cux23c14-sec-8}{}{}Manual
Address Assignment}{Manual Address Assignment}}

In order to enable IPv6 on a router, you have to use the
\texttt{ipv6\ unicast-routing} global configuration command:

\begin{verbatim}
Corp(config)#ipv6 unicast-routing
\end{verbatim}

By default, IPv6 traffic forwarding is disabled, so using this command
enables it. Also, as you've probably guessed, IPv6 isn't enabled by
default on any interfaces either, so we have to go to each interface
individually and enable it.

There are a few different ways to do this, but a really easy way is to
just add an address to the interface. You use the interface
configuration command \texttt{ipv6\ address}
\emph{\texttt{\textless{}ipv6prefix\textgreater{}}}\texttt{/}\emph{\texttt{\textless{}prefix-length\textgreater{}}}\texttt{\ {[}eui-64{]}}to
get this done.

Here's an example:

\begin{verbatim}
Corp(config-if)#ipv6 address 2001:db8:3c4d:1:0260:d6FF.FE73:1987/64
\end{verbatim}

\protect\hypertarget{c14.xhtmlux5cux23Page_556}{}{}You can specify the
entire 128-bit global IPv6 address as I just demonstrated with the
preceding command, or you can use the EUI-64 option. Remember, the
EUI-64 (extended unique identifier) format allows the device to use its
MAC address and pad it to make the interface ID. Check it out:

\begin{verbatim}
Corp(config-if)#ipv6 address 2001:db8:3c4d:1::/64 eui-64
\end{verbatim}

As an alternative to typing in an IPv6 address on a router, you can
enable the interface instead to permit the application of an automatic
link-local address.

To configure a router so that it uses only link-local addresses, use the
\texttt{ipv6\ enable} interface configuration command:

\begin{verbatim}
Corp(config-if)#ipv6 enable
\end{verbatim}

\begin{center}\rule{0.5\linewidth}{0.5pt}\end{center}

\includegraphics{images/note.png}Remember, if you have only a link-local
address, you will be able to communicate only on that local subnet.

\begin{center}\rule{0.5\linewidth}{0.5pt}\end{center}

\subsubsection[Stateless Autoconfiguration
(eui-64)]{\texorpdfstring{\protect\hypertarget{c14.xhtmlux5cux23c14-sec-9}{}{}Stateless
Autoconfiguration (eui-64)}{Stateless Autoconfiguration (eui-64)}}

Autoconfiguration is an especially useful solution because it allows
devices on a network to address themselves with a link-local unicast
address as well as with a global unicast address. This process happens
through first learning the prefix information from the router and then
appending the device's own interface address as the interface ID. But
where does it get that interface ID? Well, you know every device on an
Ethernet network has a physical MAC address, which is exactly what's
used for the interface ID. But since the interface ID in an IPv6 address
is 64 bits in length and a MAC address is only 48 bits, where do the
extra 16 bits come from? The MAC address is padded in the middle with
the extra bits---it's padded with FFFE.

For example, let's say I have a device with a MAC address that looks
like this: 0060:d673:1987. After it's been padded, it would look like
this: 0260:d6FF:FE73:1987.
\protect\hyperlink{c14.xhtmlux5cux23figure14-4}{Figure 14.4} illustrates
what an EUI-64 address looks like.

\begin{figure}
\centering
\includegraphics{images/c14f004.jpg}
\caption{{\protect\hyperlink{c14.xhtmlux5cux23figureanchor14-4}{\textbf{FIGURE
14.4}} EUI-64 interface ID assignment}}
\end{figure}

\protect\hypertarget{c14.xhtmlux5cux23Page_557}{}{}So where did that 2
in the beginning of the address come from? Another good question. You
see that part of the process of padding, called modified EUI-64 format,
changes a bit to specify if the address is locally unique or globally
unique. And the bit that gets changed is the 7th bit in the address.

The reason for modifying the U/L bit is that, when using manually
assigned addresses on an interface, it means you can simply assign the
address 2001:db8:1:9::1/64 instead of the much longer
2001:db8:1:9:0200::1/64. Also, if you are going to manually assign a
link-local address, you can assign the short address fe80::1 instead of
the long fe80::0200:0:0:1 or fe80:0:0:0:0200::1. So, even though at
first glance it seems the IETF made this harder for you to simply
understand IPv6 addressing by flipping the 7th bit, in reality this made
addressing much simpler. Also, since most people don't typically
override the burned-in address, the U/L bit is a 0, which means that
you'll see this inverted to a 1 most of the time. But because you're
studying the Cisco exam objectives, you'll need to look at inverting it
both ways.

Here are a few examples:

\begin{enumerate}
\item
  MAC address 0\textbf{0}90:2716:fd0f
\item
  IPv6 EUI-64 address: 2001:0db8:0:1:0\textbf{2}90:27ff:fe16:fd0f

  That one was easy! Too easy for the Cisco exam, so let's do another:
\item
  MAC address a\textbf{a}12:bcbc:1234
\item
  IPv6 EUI-64 address: 2001:0db8:0:1:a\textbf{8}12:bcff:febc:1234

  101010\textbf{1}0 represents the first 8 bits of the MAC address (aa),
  which when inverting the 7th bit becomes 101010\textbf{0}0. The answer
  becomes A8. I can't tell you how important this is for you to
  understand, so bear with me and work through a couple more!
\item
  MAC address 0\textbf{c}0c:dede:1234
\item
  IPv6 EUI-64 address: 2001:0db8:0:1:0\textbf{e}0c:deff:fede:1234

  0c is 00001100 in the first 8 bits of the MAC address, which then
  becomes 00001110 when flipping the 7th bit. The answer is then 0e.
  Let's practice one more:
\item
  MAC address 0\textbf{b}34:ba12:1234
\item
  IPv6 EUI-64 address: 2001:0db8:0:1:0\textbf{9}34:baff:fe12:1234

  0b in binary is 000010\textbf{1}1, the first 8 bits of the MAC
  address, which then becomes 000010\textbf{0}1. The answer is 09.
\end{enumerate}

\begin{center}\rule{0.5\linewidth}{0.5pt}\end{center}

\includegraphics{images/tip.png}Pay extra-special attention to this
EUI-64 address assignment and be able to convert the 7th bit based on
the EUI-64 rules! Written Lab 14.2 will help you practice this.

\begin{center}\rule{0.5\linewidth}{0.5pt}\end{center}

To perform autoconfiguration, a host goes through a basic two-step
process:

\begin{enumerate}
\tightlist
\item
  First, the host needs the prefix information, similar to the network
  portion of an IPv4 address, to configure its interface, so it sends a
  router solicitation (RS) request for it. This RS is then sent out as a
  multicast to all routers (FF02::2). The actual information
  \protect\hypertarget{c14.xhtmlux5cux23Page_558}{}{}being sent is a
  type of ICMP message, and like everything in networking, this ICMP
  message has a number that identifies it. The RS message is ICMP type
  133.
\item
  The router answers back with the required prefix information via a
  router advertisement (RA). An RA message also happens to be a
  multicast packet that's sent to the all-nodes multicast address
  (FF02::1) and is ICMP type 134. RA messages are sent on a periodic
  basis, but the host sends the RS for an immediate response so it
  doesn't have to wait until the next scheduled RA to get what it needs.
\end{enumerate}

These two steps are shown in
\protect\hyperlink{c14.xhtmlux5cux23figure14-5}{Figure 14.5}.

\begin{figure}
\centering
\includegraphics{images/c14f005.jpg}
\caption{{\protect\hyperlink{c14.xhtmlux5cux23figureanchor14-5}{\textbf{FIGURE
14.5}} Two steps to IPv6 autoconfiguration}}
\end{figure}

By the way, this type of autoconfiguration is also known as stateless
autoconfiguration because it doesn't contact or connect to and receive
any further information from the other device. We'll get to stateful
configuration when we talk about DHCPv6 next.

But before we do that, first take a look at
\protect\hyperlink{c14.xhtmlux5cux23figure14-6}{Figure 14.6}. In this
figure, the Branch router needs to be configured, but I just don't feel
like typing in an IPv6 address on the interface connecting to the Corp
router. I also don't feel like typing in any routing commands, but I
need more than a link-local address on that interface, so I'm going to
have to do something! So basically, I want to have the Branch router
work with IPv6 on the internetwork with the least amount of effort from
me. Let's see if I can get away with that.

\begin{figure}
\centering
\includegraphics{images/c14f006.jpg}
\caption{{\protect\hyperlink{c14.xhtmlux5cux23figureanchor14-6}{\textbf{FIGURE
14.6}} IPv6 autoconfiguration example}}
\end{figure}

Ah ha---there is an easy way! I love IPv6 because it allows me to be
relatively lazy when dealing with some parts of my network, yet it still
works really well. By using the command
\texttt{ipv6\ address\ autoconfig}, the interface will listen for RAs
and then, via the EUI-64 format, it will assign itself a global
address---sweet!

This is all really great, but you're hopefully wondering what that
\texttt{default} is doing there at the end of the command. If so, good
catch! It happens to be a wonderful, optional part of the command that
smoothly delivers a default route received from the Corp router, which
will be automatically injected into my routing table and set as the
default route---so easy!

\subsubsection[DHCPv6
(Stateful)]{\texorpdfstring{\protect\hypertarget{c14.xhtmlux5cux23c14-sec-10}{}{}DHCPv6
(Stateful)}{DHCPv6 (Stateful)}}

DHCPv6 works pretty much the same way DHCP does in v4, with the obvious
difference that it supports IPv6's new addressing scheme. And it might
come as a surprise, but there are a couple of other options that DHCP
still provides for us that autoconfiguration doesn't. And no, I'm not
kidding--- in autoconfiguration, there's absolutely no mention of DNS
servers, domain names, or many of the other options that DHCP has always
generously provided for us via IPv4. This is a big reason that the odds
favor DHCP's continued use into the future in IPv6 at least
partially---maybe even most of the time!

Upon booting up in IPv4, a client sends out a DHCP Discover message
looking for a server to give it the information it needs. But remember,
in IPv6, the RS and RA process happens first, so if there's a DHCPv6
server on the network, the RA that comes back to the client will tell it
if DHCP is available for use. If a router isn't found, the client will
respond by sending out a DHCP Solicit message, which is actually a
multicast message addressed with a destination of ff02::1:2 that calls
out, ``All DHCP agents, both servers and relays.''

It's good to know that there's some support for DHCPv6 in the Cisco IOS
even though it's limited. This rather miserly support is reserved for
stateless DHCP servers and tells us it doesn't offer any address
management of the pool or the options available for configuring that
address pool other than the DNS, domain name, default gateway, and SIP
servers.

This means that you're definitely going to need another server around to
supply and dispense all the additional, required information---maybe to
even manage the address assignment, if needed!

\begin{center}\rule{0.5\linewidth}{0.5pt}\end{center}

\includegraphics{images/tip.png}Remember for the objectives that both
stateless and stateful autoconfiguration can dynamically assign IPv6
addresses.

\begin{center}\rule{0.5\linewidth}{0.5pt}\end{center}

\subsubsection[IPv6
Header]{\texorpdfstring{\protect\hypertarget{c14.xhtmlux5cux23c14-sec-11}{}{}IPv6
Header}{IPv6 Header}}

An IPv4 header is 20 bytes long, so since an IPv6 address is four times
the size of IPv4 at 128 bits, its header must then be 80 bytes long,
right? That makes sense and is totally intuitive, but it's also
completely wrong! When IPv6 designers devised the header, they created
fewer, streamlined fields that would also result in a faster routed
protocol at the same time. Let's take a look at the streamlined IPv6
header using \protect\hyperlink{c14.xhtmlux5cux23figure14-7}{Figure
14.7}.

\protect\hypertarget{c14.xhtmlux5cux23Page_560}{}{}

\begin{figure}
\centering
\includegraphics{images/c14f007.jpg}
\caption{{\protect\hyperlink{c14.xhtmlux5cux23figureanchor14-7}{\textbf{FIGURE
14.7}} IPv6 header}}
\end{figure}

The basic IPv6 header contains eight fields, making it only twice as
large as an IP header at 40 bytes. Let's zoom in on these fields:

\textbf{Version} This 4-bit field contains the number 6, instead of the
number 4 as in IPv4.

\textbf{Traffic Class} This 8-bit field is like the Type of Service
(ToS) field in IPv4.

\textbf{Flow Label} This new field, which is 24 bits long, is used to
mark packets and traffic flows. A flow is a sequence of packets from a
single source to a single destination host, an anycast or multicast
address. The field enables efficient IPv6 flow classification.

\textbf{Payload Length} IPv4 had a total length field delimiting the
length of the packet. IPv6's payload length describes the length of the
payload only.

\textbf{Next Header} Since there are optional extension headers with
IPv6, this field defines the next header to be read. This is in contrast
to IPv4, which demands static headers with each packet.

\textbf{Hop Limit} This field specifies the maximum number of hops that
an IPv6 packet can traverse.

\begin{center}\rule{0.5\linewidth}{0.5pt}\end{center}

\includegraphics{images/tip.png}For objectives remember that the Hop
Limit field is equivalent to the TTL field in IPv4's header, and the
Extension header (after the destination address and not shown in the
figure) is used instead of the IPv4 Fragmentation field.

\begin{center}\rule{0.5\linewidth}{0.5pt}\end{center}

\textbf{Source Address} This field of 16 bytes, or 128 bits, identifies
the source of the packet.

\textbf{Destination Address} This field of 16 bytes, or 128 bits,
identifies the destination of the packet.

There are also some optional extension headers following these eight
fields, which carry other Network layer information. These header
lengths are not a fixed number---they're of variable size.

\protect\hypertarget{c14.xhtmlux5cux23Page_561}{}{}So what's different
in the IPv6 header from the IPv4 header? Let's look at that:

\begin{enumerate}
\tightlist
\item
  The Internet Header Length field was removed because it is no longer
  required. Unlike the variable-length IPv4 header, the IPv6 header is
  fixed at 40 bytes.
\item
  Fragmentation is processed differently in IPv6 and does not need the
  Flags field in the basic IPv4 header. In IPv6, routers no longer
  process fragmentation; the host is responsible for fragmentation.
\item
  The Header Checksum field at the IP layer was removed because most
  Data Link layer technologies already perform checksum and error
  control, which forces formerly optional upper-layer checksums (UDP,
  for example) to become mandatory.
\end{enumerate}

\begin{center}\rule{0.5\linewidth}{0.5pt}\end{center}

\includegraphics{images/tip.png}For the objectives, remember that unlike
IPv4 headers, IPv6 headers have a fixed length, use an extension header
instead of the IPv4 Fragmentation field, and eliminate the IPv4 checksum
field.

\begin{center}\rule{0.5\linewidth}{0.5pt}\end{center}

It's time to move on to talk about another IPv4 familiar face and find
out how a certain very important, built-in protocol has evolved in IPv6.

\subsubsection[ICMPv6]{\texorpdfstring{\protect\hypertarget{c14.xhtmlux5cux23c14-sec-12}{}{}ICMPv6}{ICMPv6}}

IPv4 used the ICMP workhorse for lots of tasks, including error messages
like destination unreachable and troubleshooting functions like Ping and
Traceroute. ICMPv6 still does those things for us, but unlike its
predecessor, the v6 flavor isn't implemented as a separate layer 3
protocol. Instead, it's an integrated part of IPv6 and is carried after
the basic IPv6 header information as an extension header. And ICMPv6
gives us another really cool feature---by default, it prevents IPv6 from
doing any fragmentation through an ICMPv6 process called path MTU
discovery. \protect\hyperlink{c14.xhtmlux5cux23figure14-8}{Figure 14.8}
shows how ICMPv6 has evolved to become part of the IPv6 packet itself.

\begin{figure}
\centering
\includegraphics{images/c14f008.jpg}
\caption{{\protect\hyperlink{c14.xhtmlux5cux23figureanchor14-8}{\textbf{FIGURE
14.8}} ICMPv6}}
\end{figure}

\protect\hypertarget{c14.xhtmlux5cux23Page_562}{}{}The ICMPv6 packet is
identified by the value 58 in the Next Header field, located inside the
ICMPv6 packet. The Type field identifies the particular kind of ICMP
message that's being carried, and the Code field further details the
specifics of the message. The Data field contains the ICMPv6 payload.

\protect\hyperlink{c14.xhtmlux5cux23table14-2}{Table 14.2} shows the
ICMP Type codes.

{\protect\hyperlink{c14.xhtmlux5cux23tableanchor14-2}{\textbf{TABLE
14.2}} ICMPv6 types}

\begin{longtable}[]{@{}ll@{}}
\toprule
ICMPv6 Type & Description\tabularnewline
\midrule
\endhead
1 & Destination Unreachable\tabularnewline
128 & Echo Request\tabularnewline
129 & Echo Reply\tabularnewline
133 & Router Solicitation\tabularnewline
134 & Router Advertisement\tabularnewline
135 & Neighbor Solicitation\tabularnewline
136 & Neighbor Advertisement\tabularnewline
\bottomrule
\end{longtable}

And this is how it works: The source node of a connection sends a packet
that's equal to the MTU size of its local link's MTU. As this packet
traverses the path toward its destination, any link that has an MTU
smaller than the size of the current packet will force the intermediate
router to send a ``packet too big'' message back to the source machine.
This message tells the source node the maximum size the restrictive link
will allow and asks the source to send a new, scaled-down packet that
can pass through. This process will continue until the destination is
finally reached, with the source node now sporting the new path's MTU.
So now, when the rest of the data packets are transmitted, they'll be
protected from fragmentation.

ICMPv6 is used for router solicitation and advertisement, for neighbor
solicitation and advertisement (i.e., finding the MAC data addresses for
IPv6 neighbors), and for redirecting the host to the best router
(default gateway).

\paragraph{Neighbor Discovery (NDP)}

ICMPv6 also takes over the task of finding the address of other devices
on the local link. The Address Resolution Protocol is used to perform
this function for IPv4, but that's been renamed neighbor discovery (ND)
in ICMPv6. This process is now achieved via a multicast address called
the solicited-node address because all hosts join this multicast group
upon connecting to the network.

\protect\hypertarget{c14.xhtmlux5cux23Page_563}{}{}Neighbor discovery
enables these functions:

\begin{enumerate}
\tightlist
\item
  Determining the MAC address of neighbors
\item
  Router solicitation (RS) FF02::2 type code 133
\item
  Router advertisements (RA) FF02::1 type code 134
\item
  Neighbor solicitation (NS) Type code 135
\item
  Neighbor advertisement (NA) Type code 136
\item
  Duplicate address detection (DAD)
\end{enumerate}

The part of the IPv6 address designated by the 24 bits farthest to the
right is added to the end of the multicast address FF02:0:0:0:0:1:FF/104
prefix and is referred to as the \emph{solicited-node address}. When
this address is queried, the corresponding host will send back its layer
2 address.

Devices can find and keep track of other neighbor devices on the network
in pretty much the same way. When I talked about RA and RS messages
earlier and told you that they use multicast traffic to request and send
address information, that too is actually a function of
ICMPv6---specifically, neighbor discovery.

In IPv4, the protocol IGMP was used to allow a host device to tell its
local router that it was joining a multicast group and would like to
receive the traffic for that group. This IGMP function has been replaced
by ICMPv6, and the process has been renamed multicast listener
discovery.

With IPv4, our hosts could have only one default gateway configured, and
if that router went down we had to either fix the router, change the
default gateway, or run some type of virtual default gateway with other
protocols created as a solution for this inadequacy in IPv4.
\protect\hyperlink{c14.xhtmlux5cux23figure14-9}{Figure 14.9} shows how
IPv6 devices find their default gateways using neighbor discovery.

\begin{figure}
\centering
\includegraphics{images/c14f009.jpg}
\caption{{\protect\hyperlink{c14.xhtmlux5cux23figureanchor14-9}{\textbf{FIGURE
14.9}} Router solicitation (RS) and router advertisement (RA)}}
\end{figure}

IPv6 hosts send a router solicitation (RS) onto their data link asking
for all routers to respond, and they use the multicast address FF02::2
to achieve this. Routers on the same link respond with a unicast to the
requesting host, or with a router advertisement (RA) using FF02::1.

\protect\hypertarget{c14.xhtmlux5cux23Page_564}{}{}But that's not all!
Hosts also can send solicitations and advertisements between themselves
using a neighbor solicitation (NS) and neighbor advertisement (NA), as
shown in \protect\hyperlink{c14.xhtmlux5cux23figure14-10}{Figure 14.10}.
Remember that RA and RS gather or provide information about routers, and
NS and NA gather information about hosts. Remember that a ``neighbor''
is a host on the same data link or VLAN.

\begin{figure}
\centering
\includegraphics{images/c14f010.jpg}
\caption{{\protect\hyperlink{c14.xhtmlux5cux23figureanchor14-10}{\textbf{FIGURE
14.10}} Neighbor solicitation (NS) and neighbor advertisement (NA)}}
\end{figure}

\subparagraph{Solicited-Node and Multicast Mapping over Ethernet}

If an IPv6 address is known, then the associated IPv6 solicited-node
multicast address is known, and if an IPv6 multicast address is known,
then the associated Ethernet MAC address is known.

For example, the IPv6 address 2001:DB8:2002:F:2C0:10FF:FE18:FC0F will
have a known solicited-node address of FF02::1:FF18:FC0F.

Now we'll form the multicast Ethernet addresses by adding the last 32
bits of the IPv6 multicast address to 33:33.

For example, if the IPv6 solicited-node multicast address is
FF02::1:FF18:FC0F, the associated Ethernet MAC address is
33:33:FF:18:FC:0F and is a virtual address.

\subparagraph{Duplicate Address Detection (DAD)}

So what do you think are the odds that two hosts will assign themselves
the same random IPv6 address? Personally, I think you could probably win
the lotto every day for a year and still not come close to the odds
against two hosts on the same data link duplicating an IPv6 address!
Still, to make sure this doesn't ever happen, duplicate address
detection (DAD) was created, which isn't an actual protocol, but a
function of the NS/NA messages.
\protect\hyperlink{c14.xhtmlux5cux23figure14-11}{Figure 14.11} shows how
a host sends an NDP NS when it receives or creates an IPv6 address.

\begin{figure}
\centering
\includegraphics{images/c14f011.jpg}
\caption{{\protect\hyperlink{c14.xhtmlux5cux23figureanchor14-11}{\textbf{FIGURE
14.11}} Duplicate address detection (DAD)}}
\end{figure}

\protect\hypertarget{c14.xhtmlux5cux23Page_565}{}{}When hosts make up or
receive an IPv6 address, they send three DADs out via NDP NS asking if
anyone has this same address. The odds are unlikely that this will ever
happen, but they ask anyway.

\begin{center}\rule{0.5\linewidth}{0.5pt}\end{center}

\includegraphics{images/tip.png}Remember for the objectives that ICMPv6
uses type 134 for router advertisement messages, and the advertised
prefix must be 64 bits in length.

\begin{center}\rule{0.5\linewidth}{0.5pt}\end{center}

\subsection[IPv6 Routing
Protocols]{\texorpdfstring{\protect\hypertarget{c14.xhtmlux5cux23c14-sec-13}{}{}IPv6
Routing Protocols}{IPv6 Routing Protocols}}

All of the routing protocols we've already discussed have been tweaked
and upgraded for use in IPv6 networks, so it figures that many of the
functions and configurations that you've already learned will be used in
almost the same way as they are now. Knowing that broadcasts have been
eliminated in IPv6, it's safe to conclude that any protocols relying
entirely on broadcast traffic will go the way of the dodo. But unlike
with the dodo, it'll be really nice to say goodbye to these
bandwidth-hogging, performance-annihilating little gremlins!

The routing protocols we'll still use in IPv6 have been renovated and
given new names. Even though this chapter's focus is on the Cisco exam
objectives, which cover only static and default routing, I want to
discuss a few of the more important ones too.

First on the list is the IPv6 RIPng (next generation). Those of you
who've been in IT for a while know that RIP has worked pretty well for
us on smaller networks. This happens to be the very reason it didn't get
whacked and will still be around in IPv6. And we still have EIGRPv6
because EIGRP already had protocol-dependent modules and all we had to
do was add a new one to it to fit in nicely with the IPv6 protocol.
Rounding out our group of protocol survivors is OSPFv3---that's not a
typo, it really is v3! OSPF for IPv4 was actually v2, so when it got its
upgrade to IPv6, it became OSPFv3. Lastly, for the new objectives, we'll
list MP-BGP4 as a multiprotocol BGP-4 protocol for IPv6. Please
understand for the objectives at this point in the book, we only need to
understand static and default routing.

\subsubsection[Static Routing with
IPv6]{\texorpdfstring{\protect\hypertarget{c14.xhtmlux5cux23c14-sec-14}{}{}Static
Routing with IPv6}{Static Routing with IPv6}}

Okay, now don't let the heading of this section scare you into looking
on Monster.com for some job that has nothing to do with networking! I
know that static routing has always run a chill up our collective spines
because it's cumbersome, difficult, and really easy to screw up. And I
won't lie to you---it's certainly not any easier with IPv6's longer
addresses, but you can do it!

We know that to make static routing work, whether in IP or IPv6, you
need these three tools:

\begin{enumerate}
\tightlist
\item
  An accurate, up-to-date network map of your entire internetwork
\item
  Next-hop address and exit interface for each neighbor connection
\item
  All the remote subnet IDs
\end{enumerate}

\protect\hypertarget{c14.xhtmlux5cux23Page_566}{}{}Of course, we don't
need to have any of these for dynamic routing, which is why we mostly
use dynamic routing. It's just so awesome to have the routing protocol
do all that work for us by finding all the remote subnets and
automatically placing them into the routing table!

\protect\hyperlink{c14.xhtmlux5cux23figure14-12}{Figure 14.12} shows a
really good example of how to use static routing with IPv6. It really
doesn't have to be that hard, but just as with IPv4, you absolutely need
an accurate network map to make static routing work!

\begin{figure}
\centering
\includegraphics{images/c14f012.jpg}
\caption{{\protect\hyperlink{c14.xhtmlux5cux23figureanchor14-12}{\textbf{FIGURE
14.12}} IPv6 static and default routing}}
\end{figure}

So here's what I did: First, I created a static route on the Corp router
to the remote network 2001:1234:4321:1::/64 using the next hop address.
I could've just as easily used the Corp router's exit interface. Next, I
just set up a default route for the Branch router with ::/0 and the
Branch exit interface of Gi0/0---not so bad!

\subsection[Configuring IPv6 on Our
Internetwork]{\texorpdfstring{\protect\hypertarget{c14.xhtmlux5cux23c14-sec-15}{}{}Configuring
IPv6 on Our Internetwork}{Configuring IPv6 on Our Internetwork}}

We're going to continue working on the same internetwork we've been
configuring throughout this book, as shown in
\protect\hyperlink{c14.xhtmlux5cux23figure14-13}{Figure 14.13}. Let's
add IPv6 to the Corp, SF, and LA routers by using a simple subnet scheme
of 11, 12, 13, 14, and 15. After that, we'll add the OSPFv3 routing
protocol. Notice in
\protect\hyperlink{c14.xhtmlux5cux23figure14-13}{Figure 14.13} how the
subnet numbers are the same on each end of the WAN links. Keep in mind
that we'll finish this chapter by running through some verification
commands.

As usual, I'll start with the Corp router:

\begin{verbatim}
Corp#config t
Corp(config)#ipv6 unicast-routing
Corp(config)#int f0/0
Corp(config-if)#ipv6 address 2001:db8:3c4d:11::/64 eui-64
Corp(config-if)#int s0/0
Corp(config-if)#ipv6 address 2001:db8:3c4d:12::/64 eui-64
Corp(config-if)#int s0/1
Corp(config-if)#ipv6 address 2001:db8:3c4d:13::/64 eui-64
Corp(config-if)#^Z
Corp#copy run start
Destination filename [startup-config]?[enter]
Building configuration...
[OK]
\end{verbatim}

\begin{figure}
\centering
\includegraphics{images/c14f013.jpg}
\caption{{\protect\hyperlink{c14.xhtmlux5cux23figureanchor14-13}{\textbf{FIGURE
14.13}} Our internetwork}}
\end{figure}

Pretty simple! In the previous configuration, I only changed the subnet
address for each interface slightly. Let's take a look at the routing
table now:

\begin{verbatim}
Corp(config-if)#do sho ipv6 route
C   2001:DB8:3C4D:11::/64 [0/0]
     via ::, FastEthernet0/0
L   2001:DB8:3C4D:11:20D:BDFF:FE3B:D80/128 [0/0]
     via ::, FastEthernet0/0
C   2001:DB8:3C4D:12::/64 [0/0]
     via ::, Serial0/0
L   2001:DB8:3C4D:12:20D:BDFF:FE3B:D80/128 [0/0]
     via ::, Serial0/0
C   2001:DB8:3C4D:13::/64 [0/0]
     via ::, Serial0/1
L   2001:DB8:3C4D:13:20D:BDFF:FE3B:D80/128 [0/0]
     via ::, Serial0/1
L   FE80::/10 [0/0]
     via ::, Null0
L   FF00::/8 [0/0]
     via ::, Null0
Corp(config-if)#
\end{verbatim}

Alright, but what's up with those two addresses for each interface? One
shows C for connected, one shows L. The connected address indicates the
IPv6 address I configured on each interface and the L is the link-local
that's been automatically assigned. Notice in the link-local address
that the FF:FE is inserted into the address to create the EUI-64
address.

Let's configure the SF router now:

\begin{verbatim}
SF#config t
SF(config)#ipv6 unicast-routing
SF(config)#int s0/0/0
SF(config-if)#ipv6 address 2001:db8:3c4d:12::/64
% 2001:DB8:3C4D:12::/64 should not be configured on Serial0/0/0, a subnet router anycast
SF(config-if)#ipv6 address 2001:db8:3c4d:12::/64 eui-64
SF(config-if)#int fa0/0
SF(config-if)#ipv6 address 2001:db8:3c4d:14::/64 eui-64
SF(config-if)#^Z
SF#show ipv6 route
C   2001:DB8:3C4D:12::/64 [0/0]
     via ::, Serial0/0/0
L   2001:DB8:3C4D:12::/128 [0/0]
     via ::, Serial0/0/0
L   2001:DB8:3C4D:12:21A:2FFF:FEE7:4398/128 [0/0]
     via ::, Serial0/0/0
C   2001:DB8:3C4D:14::/64 [0/0]
     via ::, FastEthernet0/0
L   2001:DB8:3C4D:14:21A:2FFF:FEE7:4398/128 [0/0]
     via ::, FastEthernet0/0
L   FE80::/10 [0/0]
     via ::, Null0
L   FF00::/8 [0/0]
     via ::, Null0
\end{verbatim}

Did you notice that I used the exact IPv6 subnet addresses on each side
of the serial link? Good . . . but wait---what's with that anycast error
I received when trying to configure the interfaces on the SF router? I
didn't meant to create that error; it happened because I forgot to add
the \texttt{eui-64} at the end of the address. Still, what's behind that
error? An anycast address is a host address of all 0s, meaning the last
64 bits are all off, but by typing in \texttt{/64}
\protect\hypertarget{c14.xhtmlux5cux23Page_569}{}{}without the
\texttt{eui-64}, I was telling the interface that the unique identifier
would be nothing but zeros, and that's not allowed!

Let's configure the LA router now, and then add OSPFv3:

\begin{verbatim}
SF#config t
SF(config)#ipv6 unicast-routing
SF(config)#int s0/0/1
SF(config-if)#ipv6 address 2001:db8:3c4d:13::/64 eui-64
SF(config-if)#int f0/0
SF(config-if)#ipv6 address 2001:db8:3c4d:15::/64 eui-64
SF(config-if)#do show ipv6 route
C   2001:DB8:3C4D:13::/64 [0/0]
     via ::, Serial0/0/1
L   2001:DB8:3C4D:13:21A:6CFF:FEA1:1F48/128 [0/0]
     via ::, Serial0/0/1
C   2001:DB8:3C4D:15::/64 [0/0]
     via ::, FastEthernet0/0
L   2001:DB8:3C4D:15:21A:6CFF:FEA1:1F48/128 [0/0]
     via ::, FastEthernet0/0
L   FE80::/10 [0/0]
     via ::, Null0
L   FF00::/8 [0/0]
     via ::, Null0
\end{verbatim}

This looks good, but I want you to notice that I used the exact same
IPv6 subnet addresses on each side of the links from the Corp router to
the SF router as well as from the Corp to the LA router.

\subsection[Configuring Routing on Our
Internetwork]{\texorpdfstring{\protect\hypertarget{c14.xhtmlux5cux23c14-sec-16}{}{}Configuring
Routing on Our Internetwork}{Configuring Routing on Our Internetwork}}

I'll start at the Corp router and add simple static routes. Check it
out:

\begin{verbatim}
Corp(config)#ipv6 route 2001:db8:3c4d:14::/64  2001:DB8:3C4D:12:21A:2FFF:FEE7:4398 150
Corp(config)#ipv6 route 2001:DB8:3C4D:15::/64 s0/1 150
Corp(config)#do sho ipv6 route static
[output cut]
S   2001:DB8:3C4D:14::/64 [150/0]
     via 2001:DB8:3C4D:12:21A:2FFF:FEE7:4398
\end{verbatim}

\protect\hypertarget{c14.xhtmlux5cux23Page_570}{}{}Okay---I agree that
first static route line was pretty long because I used the next-hop
address, but notice that I used the exit interface on the second entry.
But it still wasn't really all that hard to create the longer static
route entry. I just went to the SF router, used the
command\texttt{show\ ipv6\ int\ brief}, and then copied and pasted the
interface address used for the next hop. You'll get used to IPv6
addresses (You'll get used to doing a lot of copy/paste moves!).

Now since I put an AD of 150 on the static routes, once I configure a
routing protocol such as OSPF, they'll be replaced with an OSPF injected
route. Let's go to the SF and LA routers and put a single entry in each
router to get to remote subnet 11.

\begin{verbatim}
SF(config)#ipv6 route 2001:db8:3c4d:11::/64 s0/0/0 150
\end{verbatim}

That's it! I'm going to head over to LA and put a default route on that
router now:

\begin{verbatim}
LA(config)#ipv6 route ::/0 s0/0/1
\end{verbatim}

Let's take a peek at the Corp router's routing table and see if our
static routes are in there.

\begin{verbatim}
Corp#sh ipv6 route static
[output cut]
S   2001:DB8:3C4D:14::/64 [150/0]
     via 2001:DB8:3C4D:12:21A:2FFF:FEE7:4398
S   2001:DB8:3C4D:15::/64 [150/0]
     via ::, Serial0/1
\end{verbatim}

Voilà! I can see both of my static routes in the routing table, so IPv6
can now route to those networks. But we're not done because we still
need to test our network! First I'm going to go to the SF router and get
the IPv6 address of the Fa0/0 interface:

\begin{verbatim}
SF#sh ipv6 int brief
FastEthernet0/0            [up/up]
    FE80::21A:2FFF:FEE7:4398
    2001:DB8:3C4D:14:21A:2FFF:FEE7:4398
FastEthernet0/1            [administratively down/down]
Serial0/0/0                [up/up]
    FE80::21A:2FFF:FEE7:4398
    2001:DB8:3C4D:12:21A:2FFF:FEE7:4398
\end{verbatim}

Next, I'm going to go back to the Corporate router and ping that remote
interface by copying and pasting in the address. No sense doing all that
typing when copy/paste works great!

\begin{verbatim}
Corp#ping ipv6 2001:DB8:3C4D:14:21A:2FFF:FEE7:4398
Type escape sequence to abort.
Sending 5, 100-byte ICMP Echos to 2001:DB8:3C4D:14:21A:2FFF:FEE7:4398, timeout is 2 seconds:
!!!!!
Success rate is 100 percent (5/5), round-trip min/avg/max = 0/0/0 ms
Corp#
\end{verbatim}

We can see that static route worked, so next, I'll go get the IPv6
address of the LA router and ping that remote interface as well:

\begin{verbatim}
LA#sh ipv6 int brief
FastEthernet0/0            [up/up]
    FE80::21A:6CFF:FEA1:1F48
    2001:DB8:3C4D:15:21A:6CFF:FEA1:1F48
Serial0/0/1                [up/up]
    FE80::21A:6CFF:FEA1:1F48
    2001:DB8:3C4D:13:21A:6CFF:FEA1:1F48
\end{verbatim}

It's time to head over to Corp and ping LA:

\begin{verbatim}
Corp#ping ipv6 2001:DB8:3C4D:15:21A:6CFF:FEA1:1F48
Type escape sequence to abort.
Sending 5, 100-byte ICMP Echos to 2001:DB8:3C4D:15:21A:6CFF:FEA1:1F48, timeout is 2 seconds:
!!!!!
Success rate is 100 percent (5/5), round-trip min/avg/max = 4/4/4 ms
Corp#
\end{verbatim}

Now let's use one of my favorite commands:

\begin{verbatim}
Corp#sh ipv6 int brief
FastEthernet0/0            [up/up]
    FE80::20D:BDFF:FE3B:D80
    2001:DB8:3C4D:11:20D:BDFF:FE3B:D80
Serial0/0                  [up/up]
    FE80::20D:BDFF:FE3B:D80
    2001:DB8:3C4D:12:20D:BDFF:FE3B:D80
FastEthernet0/1            [administratively down/down]
    unassigned
Serial0/1                  [up/up]
    FE80::20D:BDFF:FE3B:D80
    2001:DB8:3C4D:13:20D:BDFF:FE3B:D80
Loopback0                  [up/up]
    unassigned
Corp#
\end{verbatim}

What a nice output! All our interfaces are \texttt{up/up}, and we can
see the link-local and assigned global address.

\protect\hypertarget{c14.xhtmlux5cux23Page_572}{}{}Static routing really
isn't so bad with IPv6! I'm not saying I'd like to do this in a
ginormous network---no way---I wouldn't want to opt for doing that with
IPv4 either! But you can see that it can be done. Also, notice how easy
it was to ping an IPv6 address. Copy/paste really is your friend!

Before we finish the chapter, let's add another router to our network
and connect it to the Corp Fa0/0 LAN. For our new router I really don't
feel like doing any work, so I'll just type this:

\begin{verbatim}
Boulder#config t
Boulder(config)#int f0/0
Boulder(config-if)#ipv6 address autoconfig default
\end{verbatim}

Nice and easy! This configures stateless autoconfiguration on the
interface, and the \texttt{default} keyword will advertise itself as the
default route for the local link!

I hope you found this chapter as rewarding as I did. The best thing you
can do to learn IPv6 is to get some routers and just go at it. Don't
give up because it's seriously worth your time!

\subsection[Summary]{\texorpdfstring{\protect\hypertarget{c14.xhtmlux5cux23c14-sec-17}{}{}Summary}{Summary}}

This last chapter introduced you to some very key IPv6 structural
elements as well as how to make IPv6 work within a Cisco internetwork.
You now know that even when covering and configuring IPv6 basics,
there's still a great deal to understand---and we just scratched the
surface! But you're still well equipped with all you need to meet the
Cisco exam objectives.

You learned the vital reasons why we need IPv6 and the benefits
associated with it.I covered IPv6 addressing and the importance of using
the shortened expressions. As I covered addressing with IPv6, I also
showed you the different address types, plus the special addresses
reserved in IPv6.

IPv6 will mostly be deployed automatically, meaning hosts will employ
autoconfiguration. I demonstrated how IPv6 utilizes autoconfiguration
and how it comes into play when configuring a Cisco router. You also
learned that in IPv6, we can and still should use a DHCP server to the
router to provide options to hosts just as we've been doing for years
with IPv4---not necessarily IPv6 addresses, but other mission-critical
options like providing a DNS server address.

From there, I discussed the evolution of some more integral and familiar
protocols like ICMP and OSPF. They've been upgraded to work in the IPv6
environment, but these networking workhorses are still vital and
relevant to operations, and I detailed how ICMP works with IPv6,
followed by how to configure OSPFv3. I wrapped up this pivotal chapter
by demonstrating key methods to use when verifying that all is running
correctly in your IPv6 network. So take some time and work through all
the essential study material, espe cially the written labs, to ensure
that you meet your networking goals!

\subsection[Exam
Essentials]{\texorpdfstring{\protect\hypertarget{c14.xhtmlux5cux23c14-sec-18}{}{}\protect\hypertarget{c14.xhtmlux5cux23Page_573}{}{}Exam
Essentials}{Exam Essentials}}

\textbf{Understand why we need IPv6.} Without IPv6, the world would be
depleted of IP addresses.

\textbf{Understand link-local.} Link-local is like an IPv4 private IP
address, but it can't be routed at all, not even in your organization.

\textbf{Understand unique local.} This, like link-local, is like a
private IP address in IPv4 and cannot be routed to the Internet.
However, the difference between link-local and unique local is that
unique local can be routed within your organization or company.

\textbf{Remember IPv6 addressing.} IPv6 addressing is not like IPv4
addressing. IPv6 addressing has much more address space, is 128 bits
long, and represented in hexadecimal, unlike IPv4, which is only 32 bits
long and represented in decimal.

\textbf{Understand and be able to read a EUI-64 address with the 7th bit
inverted.} Hosts can use autoconfiguration to obtain an IPv6 address,
and one of the ways it can do that is through what is called EUI-64.
This takes the unique MAC address of a host and inserts FF:FE in the
middle of the address to change a 48-bit MAC address to a 64-bit
interface ID. In addition to inserting the 16 bits into the interface
ID, the 7th bit of the 1st byte is inverted, typically from a 0 to a 1.
Practice this with Written Lab 14.2.

\subsection[Written Labs
14]{\texorpdfstring{\protect\hypertarget{c14.xhtmlux5cux23c14-sec-19}{}{}Written
Labs 14}{Written Labs 14}}

In this section, you'll complete the following labs to make sure you've
got the information and concepts contained within them fully dialed in:

\begin{enumerate}
\tightlist
\item
  Lab 14.1: IPv6
\item
  Lab 14.2: Converting EUI addresses
\end{enumerate}

You can find the answers to these labs in Appendix A, ``Answers to
Written Labs.''

\subsubsection[Written Lab
14.1]{\texorpdfstring{\protect\hypertarget{c14.xhtmlux5cux23c14-sec-20}{}{}Written
Lab 14.1}{Written Lab 14.1}}

In this section, write the answers to the following IPv6 questions:

\begin{enumerate}
\tightlist
\item
  Which two ICMPv6 types are used for testing IPv6 reachability?
\item
  What is the corresponding Ethernet address for
  FF02:0000:0000:0000:0000:0001:FF17:FC0F?
\item
  Which type of address is not meant to be routed?
\item
  What type of address is this: FE80::/10?
\item
  Which type of address is meant to be delivered to multiple interfaces?
\item
  \protect\hypertarget{c14.xhtmlux5cux23Page_574}{}{}Which type of
  address identifies multiple interfaces, but packets are delivered only
  to the first address it finds?
\item
  Which routing protocol uses multicast address FF02::5?
\item
  IPv4 had a loopback address of 127.0.0.1. What is the IPv6 loopback
  address?
\item
  What does a link-local address always start with?
\item
  Which IPv6 address is the all-router multicast group?
\end{enumerate}

\subsubsection[Written Lab
14.2]{\texorpdfstring{\protect\hypertarget{c14.xhtmlux5cux23c14-sec-21}{}{}Written
Lab 14.2}{Written Lab 14.2}}

In this section, you will practice inverting the 7th bit of a EUI-64
address. Use the prefix 2001:db8:1:1/64 for each address.

\begin{enumerate}
\tightlist
\item
  Convert the following MAC address into a EUI-64 address:
  0b0c:abcd:1234.
\item
  Convert the following MAC address into a EUI-64 address:
  060c:32f1:a4d2.
\item
  Convert the following MAC address into a EUI-64 address:
  10bc:abcd:1234.
\item
  Convert the following MAC address into a EUI-64 address:
  0d01:3a2f:1234.
\item
  Convert the following MAC address into a EUI-64 address:
  0a0c.abac.caba.
\end{enumerate}

\subsection[Hands-on
Labs]{\texorpdfstring{\protect\hypertarget{c14.xhtmlux5cux23c14-sec-22}{}{}Hands-on
Labs}{Hands-on Labs}}

You'll need at least three routers to complete these labs; five would be
better, but if you are using the LammleSim IOS version, then these lab
layouts are preconfigured for you. This section will have you configure
the following labs:

\begin{enumerate}
\tightlist
\item
  Lab 14.1: Manual and Stateful Autoconfiguration
\item
  Lab 14.2: Static and Default Routing
\end{enumerate}

Here is our network:

\begin{figure}
\centering
\includegraphics{images/c14f014.jpg}
\caption{}
\end{figure}

\subsubsection[Hands-on Lab 14.1: Manual and Stateful
Autoconfiguration]{\texorpdfstring{\protect\hypertarget{c14.xhtmlux5cux23c14-sec-23}{}{}Hands-on
Lab 14.1: Manual and Stateful
Autoconfiguration}{Hands-on Lab 14.1: Manual and Stateful Autoconfiguration}}

In this lab, you will configure the C router with manual IPv6 addresses
on the Fa0/0 and Fa0/1 interfaces and then configure the other routers
to automatically assign themselves an IPv6 address.

\begin{enumerate}
\item
  \protect\hypertarget{c14.xhtmlux5cux23Page_575}{}{}Log in to the C
  router and configure IPv6 addresses on each interface based on the
  subnets (1 and 2) shown in the graphic.

\begin{verbatim}
C(config)#ipv6 unicast-routing
C(config)#int fa0/0
C(config-if)#ipv6 address 2001:db8:3c4d:1::1/64
C(config-if)#int fa0/1
C(config-if)#ipv6 address 2001:db8:3c4d:2::1/64
\end{verbatim}
\item
  Verify the interfaces with the \texttt{show\ ipv6\ route\ connected}
  and \texttt{sho\ ipv6\ int\ brief} commands.

\begin{verbatim}
C(config-if)#do show ipv6 route connected
[output cut]
C   2001:DB8:3C4D:1::/64 [0/0]
     via ::, FastEthernet0/0
C   2001:DB8:3C4D:2::/64 [0/0]
     via ::, FastEthernet0/0
C(config-if)#sh ipv6 int brief
FastEthernet0/0            [up/up]
    FE80::20D:BDFF:FE3B:D80
    2001:DB8:3C4D:1::1
FastEthernet0/1            [up/up]
    FE80::20D:BDFF:FE3B:D81
    2001:DB8:3C4D:2::1
Loopback0                  [up/up]
    Unassigned
\end{verbatim}
\item
  Go to your other routers and configure the Fa0/0 on each router to
  autoconfigure an IPv6 address.

\begin{verbatim}
A(config)#ipv6 unicast-routing
A(config)#int f0/0
A(config-if)#ipv6 address autoconfig
A(config-if)#no shut
 
B(config)#ipv6 unicast-routing
B(config)#int fa0/0
B(config-if)#ipv6 address autoconfig
B(config-if)#no shut
 
D(config)#ipv6 unicast-routing
D(config)#int fa0/0
D(config-if)#ipv6 address autoconfig
D(config-if)#no shut
 
E(config)#ipv6 unicast-routing
E(config)#int fa0/0
E(config-if)#ipv6 address autoconfig
E(config-if)#no shut
\end{verbatim}
\item
  Verify that your routers received an IPv6 address.

\begin{verbatim}
A#sh ipv6 int brief
FastEthernet0/0            [up/up]
    FE80::20D:BDFF:FE3B:C20
    2001:DB8:3C4D:1:20D:BDFF:FE3B:C20
\end{verbatim}
\end{enumerate}

Continue to verify your addresses on all your other routers.

\subsubsection[Hands-on Lab 14.2: Static and Default
Routing]{\texorpdfstring{\protect\hypertarget{c14.xhtmlux5cux23c14-sec-24}{}{}Hands-on
Lab 14.2: Static and Default
Routing}{Hands-on Lab 14.2: Static and Default Routing}}

Router C is directly connected to both subnets, so no routing of any
type needs to be configured. However, all the other routers are
connected to only one subnet, so at least one route needs to be
configured on each router.

\begin{enumerate}
\item
  On the A router, configure a static route to the 2001:db8:3c4d:2::/64
  subnet.

\begin{verbatim}
A(config)#ipv6 route 2001:db8:3c4d:2::/64 fa0/0
\end{verbatim}
\item
  On the B router, configure a default route.

\begin{verbatim}
B(config)#ipv6 route ::/0 fa0/0
\end{verbatim}
\item
  On the D router, create a static route to the remote subnet.

\begin{verbatim}
D(config)#ipv6 route 2001:db8:3c4d:1::/64 fa0/0
\end{verbatim}
\item
  On the E router, create a static route to the remote subnet.

\begin{verbatim}
E(config)#ipv6 route 2001:db8:3c4d:1::/64 fa0/0
\end{verbatim}
\item
  Verify your configurations with a \texttt{show\ running-config} and
  \texttt{show\ ipv6\ route}.
\item
  Ping from router D to router A. First, you need to get router A's IPv6
  address with a \texttt{show\ ipv6\ int\ brief} command. Here is an
  example:

\begin{verbatim}
A#sh ipv6 int brief
FastEthernet0/0            [up/up]
    FE80::20D:BDFF:FE3B:C20
    2001:DB8:3C4D:1:20D:BDFF:FE3B:C20
\end{verbatim}
\item
  Now go to router D and ping the IPv6 address from router A:

\begin{verbatim}
D#ping ipv6 2001:DB8:3C4D:1:20D:BDFF:FE3B:C20
Type escape sequence to abort.
Sending 5, 100-byte ICMP Echos to 2001:DB8:3C4D:1:20D:BDFF:FE3B:C20, timeout is 2 seconds:
!!!!!
Success rate is 100 percent (5/5), round-trip min/avg/max = 0/2/4 ms
\end{verbatim}
\end{enumerate}

\subsection[Review
Questions]{\texorpdfstring{\protect\hypertarget{c14.xhtmlux5cux23c14-sec-25}{}{}\protect\hypertarget{c14.xhtmlux5cux23Page_577}{}{}Review
Questions}{Review Questions}}

\begin{center}\rule{0.5\linewidth}{0.5pt}\end{center}

\includegraphics{images/note.png}The following questions are designed to
test your understanding of this chapter's material. For more information
on how to get additional questions, please see
\texttt{www.lammle.com/ccna}.

\begin{center}\rule{0.5\linewidth}{0.5pt}\end{center}

The answers to these questions can be found in Appendix B, ``Answers to
Chapter Review Questions.''

\begin{enumerate}
\tightlist
\item
  How is an EUI-64 format interface ID created from a 48-bit MAC
  address?

  \begin{enumerate}
  \tightlist
  \item
    By appending 0xFF to the MAC address
  \item
    By prefixing the MAC address with 0xFFEE
  \item
    By prefixing the MAC address with 0xFF and appending 0xFF to it
  \item
    By inserting 0xFFFE between the upper 3 bytes and the lower 3 bytes
    of the MAC address
  \item
    By prefixing the MAC address with 0xF and inserting 0xF after each
    of its first three bytes
  \end{enumerate}
\item
  Which option is a valid IPv6 address?

  \begin{enumerate}
  \tightlist
  \item
    2001:0000:130F::099a::12a
  \item
    2002:7654:A1AD:61:81AF:CCC1
  \item
    FEC0:ABCD:WXYZ:0067::2A4
  \item
    2004:1:25A4:886F::1
  \end{enumerate}
\item
  Which three statements about IPv6 prefixes are true? (Choose three.)

  \begin{enumerate}
  \tightlist
  \item
    FF00:/8 is used for IPv6 multicast.
  \item
    FE80::/10 is used for link-local unicast.
  \item
    FC00::/7 is used in private networks.
  \item
    2001::1/127 is used for loopback addresses.
  \item
    FE80::/8 is used for link-local unicast.
  \item
    FEC0::/10 is used for IPv6 broadcast.
  \end{enumerate}
\item
  What are three approaches that are used when migrating from an IPv4
  addressing scheme to an IPv6 scheme? (Choose three.)

  \begin{enumerate}
  \tightlist
  \item
    Enable dual-stack routing.
  \item
    Configure IPv6 directly.
  \item
    Configure IPv4 tunnels between IPv6 islands.
  \item
    Use proxying and translation to translate IPv6 packets into IPv4
    packets.
  \item
    Statically map IPv4 addresses to IPv6 addresses.
  \item
    Use DHCPv6 to map IPv4 addresses to IPv6 addresses.
  \end{enumerate}
\item
  \protect\hypertarget{c14.xhtmlux5cux23Page_578}{}{}Which two
  statements about IPv6 router advertisement messages are true? (Choose
  two.)

  \begin{enumerate}
  \tightlist
  \item
    They use ICMPv6 type 134.
  \item
    The advertised prefix length must be 64 bits.
  \item
    The advertised prefix length must be 48 bits.
  \item
    They are sourced from the configured IPv6 interface address.
  \item
    Their destination is always the link-local address of the
    neighboring node.
  \end{enumerate}
\item
  Which of the following is true when describing an IPv6 anycast
  address?

  \begin{enumerate}
  \tightlist
  \item
    One-to-many communication model
  \item
    One-to-nearest communication model
  \item
    Any-to-many communication model
  \item
    A unique IPv6 address for each device in the group
  \item
    The same address for multiple devices in the group
  \item
    Delivery of packets to the group interface that is closest to the
    sending device
  \end{enumerate}
\item
  You want to ping the loopback address of your IPv6 local host. What
  will you type?

  \begin{enumerate}
  \tightlist
  \item
    \texttt{ping\ 127.0.0.1}
  \item
    \texttt{ping\ 0.0.0.0}
  \item
    \texttt{ping\ ::1}
  \item
    \texttt{trace\ 0.0.::1}
  \end{enumerate}
\item
  What are three features of the IPv6 protocol? (Choose three.)

  \begin{enumerate}
  \tightlist
  \item
    Optional IPsec
  \item
    Autoconfiguration
  \item
    No broadcasts
  \item
    Complicated header
  \item
    Plug-and-play
  \item
    Checksums
  \end{enumerate}
\item
  Which two statements describe characteristics of IPv6 unicast
  addressing? (Choose two.)

  \begin{enumerate}
  \tightlist
  \item
    Global addresses start with 2000::/3.
  \item
    Link-local addresses start with FE00:/12.
  \item
    Link-local addresses start with FF00::/10.
  \item
    There is only one loopback address and it is ::1.
  \item
    If a global address is assigned to an interface, then that is the
    only allowable address for the interface.
  \end{enumerate}
\item
  A host sends a router solicitation (RS) on the data link. What
  destination address is sent with this request?

  \begin{enumerate}
  \tightlist
  \item
    FF02::A
  \item
    FF02::9
  \item
    \protect\hypertarget{c14.xhtmlux5cux23Page_579}{}{}FF02::2
  \item
    FF02::1
  \item
    FF02::5
  \end{enumerate}
\item
  What are two valid reasons for adopting IPv6 over IPv4? (Choose two.)

  \begin{enumerate}
  \tightlist
  \item
    No broadcast
  \item
    Change of source address in the IPv6 header
  \item
    Change of destination address in the IPv6 header
  \item
    No password required for Telnet access
  \item
    Autoconfiguration
  \item
    NAT
  \end{enumerate}
\item
  A host sends a type of NDP message providing the MAC address that was
  requested. Which type of NDP was sent?

  \begin{enumerate}
  \tightlist
  \item
    NA
  \item
    RS
  \item
    RA
  \item
    NS
  \end{enumerate}
\item
  Which is known as ``one-to-nearest'' addressing in IPv6?

  \begin{enumerate}
  \tightlist
  \item
    Global unicast
  \item
    Anycast
  \item
    Multicast
  \item
    Unspecified address
  \end{enumerate}
\item
  Which of the following statements about IPv6 addresses are true?
  (Choose two.)

  \begin{enumerate}
  \tightlist
  \item
    Leading zeros are required.
  \item
    Two colons (::) are used to represent successive hexadecimal fields
    of zeros.
  \item
    Two colons (::) are used to separate fields.
  \item
    A single interface will have multiple IPv6 addresses of different
    types.
  \end{enumerate}
\item
  Which three ways are an IPv6 header simpler than an IPv4 header?
  (Choose three.)

  \begin{enumerate}
  \tightlist
  \item
    Unlike IPv4 headers, IPv6 headers have a fixed length.
  \item
    IPv6 uses an extension header instead of the IPv4 Fragmentation
    field.
  \item
    IPv6 headers eliminate the IPv4 Checksum field.
  \item
    IPv6 headers use the Fragment Offset field in place of the IPv4
    Fragmentation field.
  \item
    IPv6 headers use a smaller Option field size than IPv4 headers.
  \item
    IPv6 headers use a 4-bit TTL field, and IPv4 headers use an 8-bit
    TTL field.
  \end{enumerate}
\item
  Which of the following descriptions about IPv6 is correct?

  \begin{enumerate}
  \tightlist
  \item
    Addresses are not hierarchical and are assigned at random.
  \item
    Broadcasts have been eliminated and replaced with multicasts.
  \item
    \protect\hypertarget{c14.xhtmlux5cux23Page_580}{}{}There are 2.7
    billion addresses.
  \item
    An interface can only be configured with one IPv6 address.
  \end{enumerate}
\item
  How many bits are in an IPv6 address field?

  \begin{enumerate}
  \tightlist
  \item
    24
  \item
    4
  \item
    3
  \item
    16
  \item
    32
  \item
    128
  \end{enumerate}
\item
  Which of the following correctly describe characteristics of IPv6
  unicast addressing? (Choose two.)

  \begin{enumerate}
  \tightlist
  \item
    Global addresses start with 2000::/3.
  \item
    Link-local addresses start with FF00::/10.
  \item
    Link-local addresses start with FE00:/12.
  \item
    There is only one loopback address and it is ::1.
  \end{enumerate}
\item
  Which of the following statements are true of IPv6 address
  representation? (Choose two.)

  \begin{enumerate}
  \tightlist
  \item
    The first 64 bits represent the dynamically created interface ID.
  \item
    A single interface may be assigned multiple IPv6 addresses of any
    type.
  \item
    Every IPv6 interface contains at least one loopback address.
  \item
    Leading zeroes in an IPv6 16-bit hexadecimal field are mandatory.
  \end{enumerate}
\item
  Which command enables IPv6 forwarding on a Cisco router?

  \begin{enumerate}
  \tightlist
  \item
    \texttt{ipv6\ local}
  \item
    \texttt{ipv6\ host}
  \item
    \texttt{ipv6\ unicast-routing}
  \item
    \texttt{ipv6\ neighbor}
  \end{enumerate}
\end{enumerate}

\protect\hypertarget{p02.xhtml}{}{}

\section[{PART II}\\
{ICND
2}]{\texorpdfstring{\protect\hypertarget{p02.xhtmlux5cux23p02}{}{}\protect\hypertarget{p02.xhtmlux5cux23Page_581}{}{}{PART
II}\\
{ICND 2}}{PART II ICND 2}}

\begin{center}\rule{0.5\linewidth}{0.5pt}\end{center}

\protect\hypertarget{c15.xhtml}{}{}

\section[{Chapter 15}\\
{Enhanced Switched
Technologies}]{\texorpdfstring{\protect\hypertarget{c15.xhtmlux5cux23c15}{}{}\protect\hypertarget{c15.xhtmlux5cux23Page_583}{}{}{Chapter
15}\\
{Enhanced Switched
Technologies}}{Chapter 15 Enhanced Switched Technologies}}

\begin{center}\rule{0.5\linewidth}{0.5pt}\end{center}

\subsection{THE FOLLOWING ICND2 EXAM TOPICS ARE COVERED IN THIS
CHAPTER:}

\begin{enumerate}
\tightlist
\item
  \includegraphics{images/tick.png} \textbf{1.0 LAN Switching
  Technologies}

  \begin{enumerate}
  \tightlist
  \item
    \includegraphics{images/square.png} 1.1 Configure, verify, and
    troubleshoot VLANs (normal/extended range) spanning multiple
    switches

    \begin{enumerate}
    \tightlist
    \item
      \includegraphics{images/square.png} 1.1.a Access ports (data and
      voice)
    \item
      \includegraphics{images/square.png} 1.1.b Default VLAN
    \end{enumerate}
  \item
    \includegraphics{images/square.png} 1.2 Configure, verify, and
    troubleshoot interswitch connectivity

    \begin{enumerate}
    \tightlist
    \item
      \includegraphics{images/square.png} 1.2.a Add and remove VLANs on
      a trunk
    \item
      \includegraphics{images/square.png} 1.2.b DTP and VTP (v1\&v2)
    \end{enumerate}
  \item
    \includegraphics{images/square.png} 1.3 Configure, verify, and
    troubleshoot STP protocols

    \begin{enumerate}
    \tightlist
    \item
      \includegraphics{images/square.png} 1.3.a STP mode (PVST+ and
      RPVST+)
    \item
      \includegraphics{images/square.png} 1.3.b STP root bridge
      selection
    \end{enumerate}
  \item
    \includegraphics{images/square.png} 1.4 Configure, verify, and
    troubleshoot STP-related optional features

    \begin{enumerate}
    \tightlist
    \item
      \includegraphics{images/square.png} 1.4.a PortFast
    \item
      \includegraphics{images/square.png} 1.4.b BPDU guard
    \end{enumerate}
  \item
    \includegraphics{images/square.png} 1.5 Configure, verify, and
    troubleshoot (Layer 2/Layer 3) EtherChannel

    \begin{enumerate}
    \tightlist
    \item
      \includegraphics{images/square.png} 1.5.a Static
    \item
      \includegraphics{images/square.png} 1.5.b PAGP
    \item
      \includegraphics{images/square.png} 1.5.c LACP
    \end{enumerate}
  \item
    \includegraphics{images/square.png} 1.7 Describe common access layer
    threat mitigation techniques

    \begin{enumerate}
    \tightlist
    \item
      \includegraphics{images/square.png} 1.7.c Nondefault native VLAN
    \end{enumerate}
  \end{enumerate}
\item
  \includegraphics{images/tick.png} 2.0 Routing Technologies

  \begin{enumerate}
  \tightlist
  \item
    \includegraphics{images/square.png} 2.1 Configure, verify, and
    troubleshoot Inter-VLAN routing

    \begin{enumerate}
    \tightlist
    \item
      \includegraphics{images/square.png} 2.1.a Router on a stick
    \item
      \includegraphics{images/square.png} 2.1.b SVI
    \end{enumerate}
  \end{enumerate}
\end{enumerate}

\protect\hypertarget{c15.xhtmlux5cux23Page_584}{}{}\includegraphics{images/intro.png}Long
ago, a company called Digital Equipment Corporation (DEC) created the
original version of \emph{Spanning Tree Protocol (STP)}. The IEEE later
created its own version of STP called 802.1d. Cisco has moved toward
another industry standard in its newer switches called 802.1w. We'll
explore both the old and new versions of STP in this chapter, but first,
I'll define some important STP basics.

Routing protocols like RIP, EIGRP, and OSPF have processes for
preventing loops from occurring at the Network layer, but if you have
redundant physical links between your switches, these protocols won't do
a thing to stop loops from occurring at the Data Link layer. That's
exactly why STP was developed---to put an end to loop issues in a layer
2 switched network. It's also why we'll be thoroughly exploring the key
features of this vital protocol as well as how it works within a
switched network in this chapter.

After covering STP in detail, we'll move on to explore EtherChannel.

\begin{center}\rule{0.5\linewidth}{0.5pt}\end{center}

\includegraphics{images/note.png}To find up-to-the-minute updates for
this chapter, please see \texttt{www.lammle.com/ccna} or the book's web
page at \texttt{www.sybex.com/go/ccna}.

\begin{center}\rule{0.5\linewidth}{0.5pt}\end{center}

\subsection[VLAN
Review]{\texorpdfstring{\protect\hypertarget{c15.xhtmlux5cux23c15-sec-1}{}{}VLAN
Review}{VLAN Review}}

As you may remember from ICND1, configuring VLANs is actually pretty
easy. It's just that figuring out which users you want in each VLAN is
not, and doing that can eat up a lot of your time! But once you've
decided on the number of VLANs you want to create and established which
users you want to belong to each one, it's time to bring your first VLAN
into the world.

To configure VLANs on a Cisco Catalyst switch, use the global config
\texttt{vlan} command. In the following example, I'm going to
demonstrate how to configure VLANs on the S1 switch by creating three
VLANs for three different departments---again, remember that VLAN 1 is
the native and management VLAN by default:

\begin{verbatim}
S1(config)#vlan ?
  WORD        ISL VLAN IDs 1-4094
  access-map  Create vlan access-map or enter vlan access-map command mode
  dot1q       dot1q parameters
  filter      Apply a VLAN Map
  group       Create a vlan group
  internal    internal VLAN
S1(config)#vlan 2
S1(config-vlan)#name Sales
S1(config-vlan)#vlan 3
S1(config-vlan)#name Marketing
S1(config-vlan)#vlan 4
S1(config-vlan)#name Accounting
S1(config-vlan)#^Z
S1#
\end{verbatim}

In this output, you can see that you can create VLANs from 1 to 4094.
But this is only mostly true. As I said, VLANs can really only be
created up to 1001, and you can't use, change, rename, or delete VLANs 1
or 1002 through 1005 because they're reserved. The VLAN with numbers
above 1005 are called extended VLANs and won't be saved in the database
unless your switch is set to what is called VLAN Trunking Protocol (VTP)
transparent mode. You won't see these VLAN numbers used too often in
production. Here's an example of me attempting to set my S1 switch to
VLAN 4000 when my switch is set to VTP server mode (the default VTP
mode, which we'll talk about shortly):

\begin{verbatim}
S1#config t
S1(config)#vlan 4000
S1(config-vlan)#^Z
% Failed to create VLANs 4000
Extended VLAN(s) not allowed in current VTP mode.
%Failed to commit extended VLAN(s) changes.
\end{verbatim}

After you create the VLANs that you want, you can use the
\texttt{show\ vlan} command to check them out. But notice that, by
default, all ports on the switch are in VLAN 1. To change the VLAN
associated with a port, you need to go to each interface and
specifically tell it which VLAN to be a part of.

\begin{center}\rule{0.5\linewidth}{0.5pt}\end{center}

\includegraphics{images/note.png}Remember that a created VLAN is unused
until it is assigned to a switch port or ports and that all ports are
always assigned in VLAN 1 unless set otherwise.

\begin{center}\rule{0.5\linewidth}{0.5pt}\end{center}

Once the VLANs are created, verify your configuration with the
\texttt{show\ vlan} command (\texttt{sh\ vlan} for short):

\begin{verbatim}
S1#sh vlan

VLAN Name                       Status    Ports
---- ------------------------- --------- -------------------------------
1    default                    active    Fa0/1, Fa0/2, Fa0/3, Fa0/4
                                         Fa0/5, Fa0/6, Fa0/7, Fa0/8
                                         Fa0/9, Fa0/10, Fa0/11, Fa0/12
                                         Fa0/13, Fa0/14, Fa0/19, Fa0/20
                                         Fa0/21, Fa0/22, Fa0/23, Gi0/1
                                         Gi0/2
2    Sales                            active
3    Marketing                        active
4    Accounting                       active
[output cut]
\end{verbatim}

If you want to see which ports are assigned to a particular VLAN (for
example, VLAN 200), you can obviously use the \texttt{show\ vlan}
command as shown above, or you can use the \texttt{show\ vlan\ id\ 200}
command to get ports assigned only to VLAN 200.

This may seem repetitive, but it's important, and I want you to remember
it: You can't change, delete, or rename VLAN 1 because it's the default
VLAN and you just can't change that---period. It's also the native VLAN
of all switches by default, and Cisco recommends that you use it as your
management VLAN. If you're worried about security issues, then change
the native VLAN! Basically, any ports that aren't specifically assigned
to a different VLAN will be sent down to the native VLAN---VLAN 1.

In the preceding S1 output, you can see that ports Fa0/1 through Fa0/14,
Fa0/19 through 23, and the Gi0/1 and Gi02 uplinks are all in VLAN 1. But
where are ports 15 through 18? First, understand that the
command\texttt{show\ vlan} only displays access ports, so now that you
know what you're looking at with the \texttt{show\ vlan} command, where
do you think ports Fa15--18 are? That's right! They are trunked ports.
Cisco switches run a proprietary protocol called \emph{Dynamic Trunk
Protocol (DTP)}, and if there is a compatible switch connected, they
will start trunking automatically, which is precisely where my four
ports are. You have to use the\texttt{show\ interfaces\ trunk} command
to see your trunked ports like this:

\begin{verbatim}
S1# show interfaces trunk
Port        Mode             Encapsulation  Status        Native vlan
Fa0/15      desirable        n-isl          trunking      1
Fa0/16      desirable        n-isl          trunking      1
Fa0/17      desirable        n-isl          trunking      1
Fa0/18      desirable        n-isl          trunking      1
 
Port        Vlans allowed on trunk
Fa0/15      1-4094
Fa0/16      1-4094
Fa0/17      1-4094
Fa0/18      1-4094
 
 [output cut]
\end{verbatim}

\protect\hypertarget{c15.xhtmlux5cux23Page_587}{}{}This output reveals
that the VLANs from 1 to 4094 are allowed across the trunk by default.
Another helpful command, which is also part of the Cisco exam
objectives, is the \texttt{show\ interfaces}
\texttt{interface}\texttt{\ switchport} command:

\begin{verbatim}
S1#sh interfaces fastEthernet 0/15 switchport
Name: Fa0/15
Switchport: Enabled
Administrative Mode: dynamic desirable
Operational Mode: trunk
Administrative Trunking Encapsulation: negotiate
Operational Trunking Encapsulation: isl
Negotiation of Trunking: On
Access Mode VLAN: 1 (default)
Trunking Native Mode VLAN: 1 (default)
Administrative Native VLAN tagging: enabled
Voice VLAN: none
[output cut]
\end{verbatim}

The highlighted output shows us the administrative mode of
\texttt{dynamic\ desirable}, that the port is a trunk port, and that DTP
was used to negotiate the frame-tagging method of ISL. It also
predictably shows that the native VLAN is the default of 1.

Now that we can see the VLANs created, we can assign switch ports to
specific ones. Each port can be part of only one VLAN, with the
exception of voice access ports. Using trunking, you can make a port
available to traffic from all VLANs. I'll cover that next.

\subsubsection[Assigning Switch Ports to
VLANs]{\texorpdfstring{\protect\hypertarget{c15.xhtmlux5cux23c15-sec-2}{}{}Assigning
Switch Ports to VLANs}{Assigning Switch Ports to VLANs}}

You configure a port to belong to a VLAN by assigning a membership mode
that specifies the kind of traffic the port carries plus the number of
VLANs it can belong to. You can also configure each port on a switch to
be in a specific VLAN (access port) by using the
\texttt{inter}\texttt{face} \texttt{switchport} command. You can even
configure multiple ports at the same time with the
\texttt{interface\ range} command.

In the next example, I'll configure interface Fa0/3 to VLAN 3. This is
the connection from the S3 switch to the host device:

\begin{verbatim}
S3#config t
S3(config)#int fa0/3
S3(config-if)#switchport ?
  access         Set access mode characteristics of the interface
  autostate      Include or exclude this port from vlan link up calculation
  backup         Set backup for the interface
  block          Disable forwarding of unknown uni/multi cast addresses
  host           Set port host
  mode           Set trunking mode of the interface
  nonegotiate    Device will not engage in negotiation protocol on this
                 interface
  port-security  Security related command
  priority       Set appliance 802.1p priority
  private-vlan   Set the private VLAN configuration
  protected      Configure an interface to be a protected port
  trunk          Set trunking characteristics of the interface
  voice          Voice appliance attributes  voice
\end{verbatim}

Well now, what do we have here? There's some new stuff showing up in our
output now. We can see various commands---some that I've already
covered, but no worries because I'm going to cover the \texttt{access},
\texttt{mode}, \texttt{nonegotiate}, and \texttt{trunk} commands very
soon. Let's start with setting an access port on S1, which is probably
the most widely used type of port you'll find on production switches
that have VLANs configured:

\begin{verbatim}
S3(config-if)#switchport mode ?
    access        Set trunking mode to ACCESS unconditionally
  dot1q-tunnel  set trunking mode to TUNNEL unconditionally
  dynamic       Set trunking mode to dynamically negotiate access or trunk mode
  private-vlan  Set private-vlan mode
  trunk         Set trunking mode to TRUNK unconditionally
 
S3(config-if)#switchport mode access
S3(config-if)#switchport access vlan 3
\end{verbatim}

By starting with the \texttt{switchport\ mode\ access} command, you're
telling the switch that this is a nontrunking layer 2 port. You can then
assign a VLAN to the port with the \texttt{switchport\ access} command.
Remember, you can choose many ports to configure simultaneously with the
\texttt{interface\ range} command.

Let's take a look at our VLANs now:

\begin{verbatim}
S3#show vlan
VLAN Name                       Status    Ports
---- ------------------------ --------- -------------------------------
1    default                   active     Fa0/4, Fa0/5, Fa0/6, Fa0/7
                                          Fa0/8, Fa0/9, Fa0/10, Fa0/11,
                                          Fa0/12, Fa0/13, Fa0/14, Fa0/19,
                                          Fa0/20, Fa0/21, Fa0/22, Fa0/23,
                                          Gi0/1,Gi0/2
 
2    Sales                     active
3    Marketing                 active    Fa0/3
\end{verbatim}

\protect\hypertarget{c15.xhtmlux5cux23Page_589}{}{}Notice that port
Fa0/3 is now a member of VLAN 3. But, can you tell me where ports 1 and
2 are? And why aren't they showing up in the output of
\texttt{show\ vlan}? That's right, because they are trunk ports!

We can also see this with the
\texttt{show\ interfaces\ interface\ switchport} command:

\begin{verbatim}
S3#sh int fa0/3 switchport
Name: Fa0/3
Switchport: Enabled
Administrative Mode: static access
Operational Mode: static access
Administrative Trunking Encapsulation: negotiate
Negotiation of Trunking: Off
Access Mode VLAN: 3 (Marketing)
\end{verbatim}

The highlighted output shows that Fa0/3 is an access port and a member
of VLAN 3 (Marketing).

Before we move onto trunking and VTP, let's add a voice VLAN on our
switch. When an IP phone is connected to a switch port, this port should
have a voice VLAN associated with it. By creating a separate VLAN for
voice traffic, which of course you would do, what happens when you have
a PC or laptop that connects via Ethernet into an IP phone? The phone
connects to the Ethernet port and into one port on the switch. You're
now sending both voice and data to the single switch port.

All you need to do is add another VLAN to the same switch port like so
to fix this issue and separate the data at the switch port into two
VLANs:

\begin{verbatim}
S1(config)#vlan 10
S1(config-vlan)#name Voice
S1(config-vlan)#int g0/1
S1(config-if)#switchport voice vlan 10
\end{verbatim}

That's it. Well, sort of. If you plugged devices into each VLAN port,
they can only talk to other devices in the same VLAN. But as soon as you
learn a bit more about trunking, we're going to enable inter-VLAN
communication!

\subsubsection[Configuring Trunk
Ports]{\texorpdfstring{\protect\hypertarget{c15.xhtmlux5cux23c15-sec-3}{}{}Configuring
Trunk Ports}{Configuring Trunk Ports}}

The 2960 switch only runs the IEEE 802.1q encapsulation method. To
configure trunking on a FastEthernet port, use the interface command
\texttt{switchport\ mode\ trunk}. It's a tad different on the 3560
switch.

The following switch output shows the trunk configuration on interfaces
Fa0/15--18 as set to \texttt{trunk}:

\begin{verbatim}
S1(config)#int range f0/15-18
S1(config-if-range)#switchport trunk encapsulation dot1q
S1(config-if-range)#switchport mode trunk
\end{verbatim}

\protect\hypertarget{c15.xhtmlux5cux23Page_590}{}{}If you have a switch
that only runs the 802.1q encapsulation method, then you wouldn't use
the encapsulation command as I did in the preceding output. Let's check
out our trunk ports now:

\begin{verbatim}
S1(config-if-range)#do sh int f0/15 switchport
Name: Fa0/15
Switchport: Enabled
Administrative Mode: trunk
Operational Mode: trunk
Administrative Trunking Encapsulation: dot1q
Operational Trunking Encapsulation: dot1q
Negotiation of Trunking: On
Access Mode VLAN: 1 (default)
Trunking Native Mode VLAN: 1 (default)
Administrative Native VLAN tagging: enabled
Voice VLAN: none
\end{verbatim}

Notice that port Fa0/15 is a trunk and running 802.1q. Let's take
another look:

\begin{verbatim}
S1(config-if-range)#do sh int trunk
Port        Mode             Encapsulation  Status        Native vlan
Fa0/15      on               802.1q         trunking      1
Fa0/16      on               802.1q         trunking      1
Fa0/17      on               802.1q         trunking      1
Fa0/18      on               802.1q         trunking      1
Port        Vlans allowed on trunk
Fa0/15      1-4094
Fa0/16      1-4094
Fa0/17      1-4094
Fa0/18      1-4094
\end{verbatim}

Take note of the fact that ports 15--18 are now in the trunk mode of on
and the encapsulation is now 802.1q instead of the negotiated ISL.
Here's a description of the different options available when configuring
a switch interface:

\texttt{switchport\ mode\ access} I discussed this in the previous
section, but this puts the interface (access port) into permanent
nontrunking mode and negotiates to convert the link into a nontrunk
link. The interface becomes a nontrunk interface regardless of whether
the neighboring interface is a trunk interface. The port would be a
dedicated layer 2 access port.

\texttt{switchport\ mode\ dynamic\ auto} This mode makes the interface
able to convert the link to a trunk link. The interface becomes a trunk
interface if the neighboring
\protect\hypertarget{c15.xhtmlux5cux23Page_591}{}{}interface is set to
trunk or desirable mode. The default is \texttt{dynamic\ auto} on a lot
of Cisco switches, but that default trunk method is changing to
\texttt{dynamic\ desirable} on most new models.

\texttt{switchport\ mode\ dynamic\ desirable} This one makes the
interface actively attempt to convert the link to a trunk link. The
interface becomes a trunk interface if the neighboring interface is set
to \texttt{trunk}, \texttt{desirable}, or \texttt{auto} mode. This is
now the default switch port mode for all Ethernet interfaces on all new
Cisco switches.

\texttt{switchport\ mode\ trunk} Puts the interface into permanent
trunking mode and negotiates to convert the neighboring link into a
trunk link. The interface becomes a trunk interface even if the
neighboring interface isn't a trunk interface.

\texttt{switchport\ nonegotiate} Prevents the interface from generating
DTP frames. You can use this command only when the interface switchport
mode is access or trunk. You must manually configure the neighboring
interface as a trunk interface to establish a trunk link.

\begin{center}\rule{0.5\linewidth}{0.5pt}\end{center}

\includegraphics{images/note.png}Dynamic Trunking Protocol (DTP) is used
for negotiating trunking on a link between two devices as well as
negotiating the encapsulation type of either 802.1q or ISL. I use the
\texttt{nonegotiate} command when I want dedicated trunk ports; no
questions asked.

\begin{center}\rule{0.5\linewidth}{0.5pt}\end{center}

To disable trunking on an interface, use the
\texttt{switchport\ mode\ access} command, which sets the port back to a
dedicated layer 2 access switch port.

\paragraph{Defining the Allowed VLANs on a Trunk}

As I've mentioned, trunk ports send and receive information from all
VLANs by default, and if a frame is untagged, it's sent to the
management VLAN. Understand that this applies to the extended range
VLANs too.

But we can remove VLANs from the allowed list to prevent traffic from
certain VLANs from traversing a trunked link. I'll show you how you'd do
that, but first let me again demonstrate that all VLANs are allowed
across the trunk link by default:

\begin{verbatim}
S1#sh int trunk
[output cut]
Port        Vlans allowed on trunk
Fa0/15      1-4094
Fa0/16      1-4094
Fa0/17      1-4094
Fa0/18      1-4094
S1(config)#int f0/15
S1(config-if)#switchport trunk allowed vlan 4,6,12,15
S1(config-if)#do show int trunk
[output cut]
Port        Vlans allowed on trunk
Fa0/15      4,6,12,15
Fa0/16      1-4094
Fa0/17      1-4094
Fa0/18      1-4094
\end{verbatim}

\protect\hypertarget{c15.xhtmlux5cux23Page_592}{}{}The preceding command
affected the trunk link configured on S1 port Fa0/15, causing it to
permit all traffic sent and received for VLANs 4, 6, 12, and 15. You can
try to remove VLAN 1 on a trunk link, but it will still send and receive
management data like CDP, DTP, and VTP, so what's the point?

To remove a range of VLANs, just use the hyphen:

\begin{verbatim}
S1(config-if)#switchport trunk allowed vlan remove 4-8
\end{verbatim}

If by chance someone has removed some VLANs from a trunk link and you
want to set the trunk back to default, just use this command:

\begin{verbatim}
S1(config-if)#switchport trunk allowed vlan all
\end{verbatim}

Next, I want to show you how to configure a native VLAN for a trunk
before we start routing between VLANs.

\paragraph{Changing or Modifying the Trunk Native VLAN}

You can change the trunk port native VLAN from VLAN 1, which many people
do for security reasons. To change the native VLAN, use the following
command:

\begin{verbatim}
S1(config)#int f0/15
S1(config-if)#switchport trunk native vlan ?
  <1-4094>  VLAN ID of the native VLAN when this port is in trunking mode
\end{verbatim}

\begin{verbatim}
S1(config-if)#switchport trunk native vlan 4
1w6d: %CDP-4-NATIVE_VLAN_MISMATCH: Native VLAN mismatch discovered on FastEthernet0/15 (4), with S3 FastEthernet0/1 (1).
\end{verbatim}

So we've changed our native VLAN on our trunk link to 4, and by using
the \texttt{show\ running-config} command, I can see the configuration
under the trunk link:

\begin{verbatim}
S1#sh run int f0/15
Building configuration...
 
Current configuration : 202 bytes
!
interface FastEthernet0/15
 description 1st connection to S3
 switchport trunk encapsulation dot1q
 switchport trunk native vlan 4
 switchport trunk allowed vlan 4,6,12,15
 switchport mode trunk
end
 
S1#!
\end{verbatim}

\protect\hypertarget{c15.xhtmlux5cux23Page_593}{}{}Oops---wait a minute!
You didn't think it would be this easy and would just start working, did
you? Of course not! Here's the rub: If all switches don't have the same
native VLAN configured on the given trunk links, then we'll start to
receive this error, which happened immediately after I entered the
command:

\begin{verbatim}
1w6d: %CDP-4-NATIVE_VLAN_MISMATCH: Native VLAN mismatch discovered
on FastEthernet0/15 (4), with S3 FastEthernet0/1 (1).
\end{verbatim}

Actually, this is a good, noncryptic error, so either we can go to the
other end of our trunk link(s) and change the native VLAN or we set the
native VLAN back to the default to fix it. Here's how we'd do that:

\begin{verbatim}
S1(config-if)#no switchport trunk native vlan
1w6d: %SPANTREE-2-UNBLOCK_CONSIST_PORT: Unblocking FastEthernet0/15
on VLAN0004. Port consistency restored.
\end{verbatim}

Now our trunk link is using the default VLAN 1 as the native VLAN. Just
remember that all switches on a given trunk must use the same native
VLAN or you'll have some serious management problems. These issues won't
affect user data, just management traffic between switches. Now, let's
mix it up by connecting a router into our switched network and configure
inter-VLAN communication.

\subsection[VLAN Trunking Protocol
(VTP)]{\texorpdfstring{\protect\hypertarget{c15.xhtmlux5cux23c15-sec-4}{}{}VLAN
Trunking Protocol (VTP)}{VLAN Trunking Protocol (VTP)}}

Cisco created this one too. The basic goals of \emph{VLAN Trunking
Protocol (VTP)} are to manage all configured VLANs across a switched
internetwork and to maintain consistency throughout that network. VTP
allows you to add, delete, and rename VLANs---information that is then
propagated to all other switches in the VTP domain.

Here's a list of some of the cool features VTP has to offer:

\begin{enumerate}
\tightlist
\item
  Consistent VLAN configuration across all switches in the network
\item
  VLAN trunking over mixed networks, such as Ethernet to ATM LANE or
  even FDDI
\item
  Accurate tracking and monitoring of VLANs
\item
  Dynamic reporting of added VLANs to all switches in the VTP domain
\item
  Adding VLANs using Plug and Play
\end{enumerate}

Very nice, but before you can get VTP to manage your VLANs across the
network, you have to create a VTP server (really, you don't need to even
do that since all switches default to VTP server mode, but just make
sure you have a server). All servers that need to share VLAN information
must use the same domain name, and a switch can be in only one domain at
a time. So basically, this means that a switch can share VTP domain
information with other switches only if they're configured into the same
VTP domain. You can use a VTP domain if you have more than one switch
connected in a network, but if you've got
\protect\hypertarget{c15.xhtmlux5cux23Page_594}{}{}all your switches in
only one VLAN, you just don't need to use VTP. Do keep in mind that VTP
information is sent between switches only via a trunk port.

Switches advertise VTP management domain information as well as a
configuration revision number and all known VLANs with any specific
parameters. But there's also something called \emph{VTP transparent
mode}. In it, you can configure switches to forward VTP information
through trunk ports but not to accept information updates or update
their VLAN databases.

If you've got sneaky users adding switches to your VTP domain behind
your back, you can include passwords, but don't forget---every switch
must be set up with the same password. And as you can imagine, this
little snag can be a real hassle administratively!

Switches detect any added VLANs within a VTP advertisement and then
prepare to send information on their trunk ports with the newly defined
VLAN in tow. Updates are sent out as revision numbers that consist of
summary advertisements. Anytime a switch sees a higher revision number,
it knows the information it's getting is more current, so it will
overwrite the existing VLAN database with the latest information.

You should know these four requirements for VTP to communicate VLAN
information between switches:

\begin{enumerate}
\tightlist
\item
  The VTP version must be set the same
\item
  The VTP management domain name of both switches must be set the same.
\item
  One of the switches has to be configured as a VTP server.
\item
  Set a VTP password if used.
\end{enumerate}

No router is necessary and is not a requirement. Now that you've got
that down, we're going to delve deeper into the world of VTP with VTP
modes and VTP pruning.

\subsubsection[VTP Modes of
Operation]{\texorpdfstring{\protect\hypertarget{c15.xhtmlux5cux23c15-sec-5}{}{}VTP
Modes of Operation}{VTP Modes of Operation}}

\protect\hyperlink{c15.xhtmlux5cux23figure15-1}{Figure 15.1} shows you
how a VTP server will update the connected VTP client's VLAN database
when a change occurs in the VLAN database on the server.

\begin{figure}
\centering
\includegraphics{images/c15f001.jpg}
\caption{{\protect\hyperlink{c15.xhtmlux5cux23figureanchor15-1}{\textbf{FIGURE
15.1}} VTP modes}}
\end{figure}

\protect\hypertarget{c15.xhtmlux5cux23Page_595}{}{}\textbf{Server} This
is the default mode for all Catalyst switches. You need at least one
server in your VTP domain to propagate VLAN information throughout that
domain. Also important: The switch must be in server mode to be able to
create, add, and delete VLANs in a VTP domain. VLAN information has to
be changed in server mode, and any change made to VLANs on a switch in
server mode will be advertised to the entire VTP domain. In VTP server
mode, VLAN configurations are saved in NVRAM on the switch.

\textbf{Client} In client mode, switches receive information from VTP
servers, but they also receive and forward updates, so in this way, they
behave like VTP servers. The difference is that they can't create,
change, or delete VLANs. Plus, none of the ports on a client switch can
be added to a new VLAN before the VTP server notifies the client switch
of the new VLAN and the VLAN exists in the client's VLAN database. Also
good to know is that VLAN information sent from a VTP server isn't
stored in NVRAM, which is important because it means that if the switch
is reset or reloaded, the VLAN information will be deleted. Here's a
hint: If you want a switch to become a server, first make it a client so
it receives all the correct VLAN information, then change it to a
server---so much easier!

So basically, a switch in VTP client mode will forward VTP summary
advertisements and process them. This switch will learn about but won't
save the VTP configuration in the running configuration, and it won't
save it in NVRAM. Switches that are in VTP client mode will only learn
about and pass along VTP information---that's it!

\begin{center}\rule{0.5\linewidth}{0.5pt}\end{center}

\includegraphics{images/globe1.png}\\
\textbf{So, When Do I Need to Consider Using VTP?}

Here's a scenario for you. Bob, a senior network administrator at Acme
Corporation in San Francisco, has about 25 switches all connected
together, and he wants to configure VLANs to break up broadcast domains.
When do you think he should start to consider using VTP?

If you answered that he should have used VTP the moment he had more than
one switch and multiple VLANs, you're right. If you have only one
switch, then VTP is irrelevant. It also isn't a player if you're not
configuring VLANs in your network. But if you do have multiple switches
that use multiple VLANs, you'd better configure your VTP server and
clients, and you better do it right!

When you first bring up your switched network, verify that your main
switch is a VTP server and that all the other ones are VTP clients. When
you create VLANs on the main VTP server, all switches will receive the
VLAN database.

If you have an existing switched network and you want to add a new
switch, make sure to configure it as a VTP client before you install it.
If you don't, it's possible---okay, highly probable---that your new
little beauty will send out a new VTP database to all your other
switches, effectively wiping out all your existing VLANs like a nuclear
blast. No one needs that!

\protect\hypertarget{c15.xhtmlux5cux23Page_596}{}{}\textbf{Transparent}
Switches in transparent mode don't participate in the VTP domain or
share its VLAN database, but they'll still forward VTP advertisements
through any configured trunk links. They can create, modify, and delete
VLANs because they keep their own database---one they keep secret from
the other switches. Despite being kept in NVRAM, the VLAN database in
transparent mode is actually only locally significant. The whole purpose
of transparent mode is to allow remote switches to receive the VLAN
database from a VTP Server configured switch through a switch that is
not participating in the same VLAN assignments.

\begin{center}\rule{0.5\linewidth}{0.5pt}\end{center}

VTP only learns about normal-range VLANs, with VLAN IDs 1 to 1005; VLANs
with IDs greater than 1005 are called extended-range VLANs and they're
not stored in the VLAN database. The switch must be in VTP transparent
mode when you create VLAN IDs from 1006 to 4094, so it would be pretty
rare that you'd ever use these VLANs. One other thing: VLAN IDs 1 and
1002 to 1005 are automatically created on all switches and can't be
removed.

\subsubsection[VTP
Pruning]{\texorpdfstring{\protect\hypertarget{c15.xhtmlux5cux23c15-sec-6}{}{}VTP
Pruning}{VTP Pruning}}

VTP gives you a way to preserve bandwidth by configuring it to reduce
the amount of broadcasts, multicasts, and unicast packets. This is
called \emph{pruning}. Switches enabled for VTP pruning send broadcasts
only to trunk links that actually must have the information.

Here's what this means: If Switch A doesn't have any ports configured
for VLAN 5 and a broadcast is sent throughout VLAN 5, that broadcast
wouldn't traverse the trunk link to Switch A. By default, VTP pruning is
disabled on all switches. Seems to me this would be a good default
parameter. When you enable pruning on a VTP server, you enable it for
the entire domain. By default, VLANs 2 through 1001 are pruning
eligible, but VLAN 1 can never be pruned because it's an administrative
VLAN. VTP pruning is supported with both VTP version 1 and version 2.

By using the \texttt{show\ interface\ trunk} command, we can see that
all VLANs are allowed across a trunked link by default:

\begin{verbatim}
S1#sh int trunk

Port        Mode         Encapsulation  Status        Native vlan
Fa0/1       auto         802.1q         trunking      1
Fa0/2       auto         802.1q         trunking      1

Port        Vlans allowed on trunk
Fa0/1       1-4094
Fa0/2       1-4094

Port        Vlans allowed and active in management domain
Fa0/1       1
Fa0/2       1

Port        Vlans in spanning tree forwarding state and not pruned
Fa0/1       1
Fa0/2       none
S1#
\end{verbatim}

Looking at the preceding output, you can see that VTP pruning is
disabled by default. I'm going to go ahead and enable pruning. It only
takes one command and it is enabled on your entire switched network for
the listed VLANs. Let's see what happens:

\begin{verbatim}
S1#config t
S1(config)#int f0/1
S1(config-if)#switchport trunk ?
  allowed  Set allowed VLAN characteristics when interface is
  in trunking mode
  native   Set trunking native characteristics when interface
  is in trunking mode
  pruning  Set pruning VLAN characteristics when interface is
  in trunking mode
S1(config-if)#switchport trunk pruning ?
  vlan  Set VLANs enabled for pruning when interface is in
  trunking mode
S1(config-if)#switchport trunk pruning vlan 3-4
\end{verbatim}

The valid VLANs that can be pruned are 2 to 1001. Extended-range VLANs
(VLAN IDs 1006 to 4094) can't be pruned, and these pruning-ineligible
VLANs can receive a flood of traffic.

\subsection[Configuring
VTP]{\texorpdfstring{\protect\hypertarget{c15.xhtmlux5cux23c15-sec-7}{}{}Configuring
VTP}{Configuring VTP}}

All Cisco switches are configured to be VTP servers by default. To
configure VTP, first you have to configure the domain name you want to
use. And of course, once you configure the VTP information on a switch,
you need to verify it.

When you create the VTP domain, you have a few options, including
setting the VTP version, domain name, password, operating mode, and
pruning capabilities of the switch. Use the \texttt{vtp} global
configuration mode command to set all this information. In the following
example, I'll set the S1 switch to \texttt{vtp\ server}, the VTP domain
to \texttt{Lammle}, and the VTP password to \texttt{todd}:

\begin{verbatim}
S1#config t
S1#(config)#vtp mode server
Device mode already VTP SERVER.
S1(config)#vtp domain Lammle
Changing VTP domain name from null to Lammle
S1(config)#vtp password todd
Setting device VLAN database password to todd
S1(config)#do show vtp password
VTP Password: todd
S1(config)#do show vtp status
VTP Version                     : 2
Configuration Revision          : 0
Maximum VLANs supported locally : 255
Number of existing VLANs        : 8
VTP Operating Mode              : Server
VTP Domain Name                 : Lammle
VTP Pruning Mode                : Disabled
VTP V2 Mode                     : Disabled
VTP Traps Generation            : Disabled
MD5 digest                      : 0x15 0x54 0x88 0xF2 0x50 0xD9 0x03 0x07
Configuration last modified by 192.168.24.6 at 3-14-93 15:47:32
Local updater ID is 192.168.24.6 on interface Vl1 (lowest numbered VLAN interface found)
\end{verbatim}

Please make sure you remember that all switches are set to VTP server
mode by default, and if you want to change and distribute any VLAN
information on a switch, you absolutely must be in VTP server mode.
After you configure the VTP information, you can verify it with the
\texttt{show\ vtp\ status} command as shown in the preceding output.

The preceding switch output shows the VTP Version, Configuration
Revision, Maximum VLANs supported locally, Number of existing VLANs, VTP
Operating Mode, VTP domain, the VTP domain, and the VTP password listed
as an MD5 Digest. You can use\texttt{show\ vtp\ password}in privileged
mode to see the password.

\subsubsection[Troubleshooting
VTP]{\texorpdfstring{\protect\hypertarget{c15.xhtmlux5cux23c15-sec-8}{}{}Troubleshooting
VTP}{Troubleshooting VTP}}

You connect your switches with crossover cables, the lights go green on
both ends, and you're up and running! Yeah---in a perfect world, right?
Don't you wish it was that easy? Well, actually, it pretty much
is---without VLANs, of course. But if you're using VLANs---and you
definitely should be---then you need to use VTP if you have multiple
VLANs configured in your switched network.

But here there be monsters: If VTP is not configured correctly, it
(surprise!) will not work, so you absolutely must be capable of
troubleshooting VTP. Let's take a look at a couple of configurations and
solve the problems. Study the output from the two following switches:

\begin{verbatim}
SwitchA#sh vtp status
VTP Version                     : 2
Configuration Revision          : 0
Maximum VLANs supported locally : 64
Number of existing VLANs        : 7
VTP Operating Mode              : Server
VTP Domain Name                 : Lammle
VTP Pruning Mode                : Disabled
VTP V2 Mode                     : Disabled
VTP Traps Generation            : Disabled
 
SwitchB#sh vtp status
VTP Version                     : 2
Configuration Revision          : 1
Maximum VLANs supported locally : 64
Number of existing VLANs        : 7
VTP Operating Mode              : Server
VTP Domain Name                 : GlobalNet
VTP Pruning Mode                : Disabled
VTP V2 Mode                     : Disabled
VTP Traps Generation            : Disabled
\end{verbatim}

So what's happening with these two switches? Why won't they share VLAN
information? At first glance, it seems that both servers are in VTP
server mode, but that's not the problem. Servers in VTP server mode will
share VLAN information using VTP. The problem is that they're in two
different VTP\emph{domains}. SwitchA is in VTP domain Lammle and SwitchB
is in VTP domain GlobalNet. They will never share VTP information
because the VTP domain names are configured differently.

Now that you know how to look for common VTP domain configuration errors
in your switches, let's take a look at another switch configuration:

\begin{verbatim}
SwitchC#sh vtp status
VTP Version                     : 2
Configuration Revision          : 1
Maximum VLANs supported locally : 64
Number of existing VLANs        : 7
VTP Operating Mode              : Client
VTP Domain Name                 : Todd
VTP Pruning Mode                : Disabled
VTP V2 Mode                     : Disabled
VTP Traps Generation            : Disabled
\end{verbatim}

Here's what will happen when you have the preceding VTP configuration:

\begin{verbatim}
SwitchC(config)#vlan 50
VTP VLAN configuration not allowed when device is in CLIENT mode.
\end{verbatim}

\protect\hypertarget{c15.xhtmlux5cux23Page_600}{}{}There you are just
trying to create a new VLAN on SwitchC and what do you get for your
trouble? A loathsome error! Why can't you create a VLAN on SwitchC?
Well, the VTP domain name isn't the important thing in this example.
What is critical here is the VTP \emph{mode}. The VTP mode is client,
and a VTP client cannot create, delete, or change VLANs, remember? VTP
clients only keep the VTP database in RAM, and that's not saved to
NVRAM. So, in order to create a VLAN on this switch, you've got to make
the switch a VTP server first.

So to fix this problem, here's what you need to do:

\begin{verbatim}
SwitchC(config)#vtp mode server
Setting device to VTP SERVER mode
SwitchC(config)#vlan 50
SwitchC(config-vlan)#
\end{verbatim}

Wait, we're not done. Now take a look at the output from these two
switches and determine why SwitchB is not receiving VLAN information
from SwitchA:

\begin{verbatim}
SwitchA#sh vtp status
VTP Version                     : 2
Configuration Revision          : 4
Maximum VLANs supported locally : 64
Number of existing VLANs        : 7
VTP Operating Mode              : Server
VTP Domain Name                 : GlobalNet
VTP Pruning Mode                : Disabled
VTP V2 Mode                     : Disabled
VTP Traps Generation            : Disabled
 
SwitchB#sh vtp status
VTP Version                     : 2
Configuration Revision          : 14
Maximum VLANs supported locally : 64
Number of existing VLANs        : 7
VTP Operating Mode              : Server
VTP Domain Name                 : GlobalNet
VTP Pruning Mode                : Disabled
VTP V2 Mode                     : Disabled
VTP Traps Generation            : Disabled
\end{verbatim}

You may be tempted to say it's because they're both VTP servers, but
that is not the problem. All your switches can be servers and they can
still share VLAN information. As a matter of fact, Cisco actually
suggests that all switches stay VTP servers and that you just make sure
\protect\hypertarget{c15.xhtmlux5cux23Page_601}{}{}the switch you want
to advertise VTP VLAN information has the highest revision number. If
all switches are VTP servers, then all of the switches will save the
VLAN database. But SwitchB isn't receiving VLAN information from SwitchA
because SwitchB has a higher revision number than SwitchA. It's very
important that you can recognize this problem.

There are a couple ways to go about resolving this issue. The first
thing you could do is to change the VTP domain name on SwitchB to
another name, then set it back to GlobalNet, which will reset the
revision number to zero (0) on SwitchB. The second approach would be to
create or delete VLANs on SwitchA until the revision number passes the
revision number on SwitchB. I didn't say the second way was better; I
just said it's another way to fix it!

Let's look at one more. Why is SwitchB not receiving VLAN information
from SwitchA?

\begin{verbatim}
SwitchA#sh vtp status
VTP Version                     : 1
Configuration Revision          : 4
Maximum VLANs supported locally : 64
Number of existing VLANs        : 7
VTP Operating Mode              : Server
VTP Domain Name                 : GlobalNet
VTP Pruning Mode                : Disabled
VTP V2 Mode                     : Disabled
VTP Traps Generation            : Disabled
 
SwitchB#sh vtp status
VTP Version                     : 2
Configuration Revision          : 3
Maximum VLANs supported locally : 64
Number of existing VLANs        : 5
VTP Operating Mode              : Server
VTP Domain Name                 :
VTP Pruning Mode                : Disabled
VTP V2 Mode                     : Disabled
VTP Traps Generation            : Disabled
\end{verbatim}

I know your first instinct is to notice that SwitchB doesn't have a
domain name set and consider that the issue. That's not the problem!
When a switch comes up, a VTP server with a domain name set will send
VTP advertisements, and a new switch out of the box will configure
itself using the advertisement with the received domain name and also
download the VLAN database.

The problem with the above switches is that they are set to different
VTP versions---but that still isn't the full problem.

\protect\hypertarget{c15.xhtmlux5cux23Page_602}{}{}By default, VTP
operates in version 1. You can configure VTP version 2 if you want
support for these features, which are not supported in version 1:

\begin{enumerate}
\tightlist
\item
  Token Ring support---Hmmm\ldots doesn't seem like much of a reason to
  go to version 2 today. Let's look at some other reasons.
\item
  Unrecognized Type-Length-Value (TLV) support---A VTP server or client
  propagates configuration changes to its other trunks, even for TLVs it
  is not able to parse. The unrecognized TLV is saved in NVRAM when the
  switch is operating in VTP server mode.
\item
  Version-Dependent Transparent Mode---In VTP version 1, a VTP
  transparent switch inspects VTP messages for the domain name and
  version and forwards a message only if the version and domain name
  match. Because VTP version 2 supports only one domain, it forwards VTP
  messages in transparent mode without inspecting the version and domain
  name.
\item
  Consistency Checks---In VTP version 2, VLAN consistency checks (such
  as checking the consistency of VLAN names and values) are performed
  only when you enter new information through the CLI or SNMP.
  Consistency checks are not performed when new information is obtained
  from a VTP message or when information is read from NVRAM. If the MD5
  digest on a received VTP message is correct, its information is
  accepted.
\end{enumerate}

Wait! Nothing is that easy. Just set SwitchA to version 2 and we're up
and running? Nope! The interesting thing about VTP version 2 is that if
you set one switch in your network (VTP domain) to version 2, all
switches would set their version to 2 automatically---very cool! So then
what is the problem? SwitchA doesn't support VTP version 2, which is the
actual answer to this question. Crazy! I think you can see that VTP will
drive you to drink if you're not careful!

Okay, get a coffee, expresso or Mountain Dew and hold onto your
hats---it's spanning tree time!

\subsection[Spanning Tree Protocol
(STP)]{\texorpdfstring{\protect\hypertarget{c15.xhtmlux5cux23c15-sec-9}{}{}Spanning
Tree Protocol (STP)}{Spanning Tree Protocol (STP)}}

Spanning Tree Protocol (STP) achieves its primary objective of
preventing network loops on layer 2 network bridges or switches by
monitoring the network to track all links and shut down the redundant
ones. STP uses the spanning-tree algorithm (STA) to first create a
topology database and then search out and disable redundant links. With
STP running, frames will be forwarded on only premium, STP-chosen links.

The Spanning Tree Protocol is a great protocol to use in networks like
the one shown in \protect\hyperlink{c15.xhtmlux5cux23figure15-2}{Figure
15.2}.

\protect\hypertarget{c15.xhtmlux5cux23Page_603}{}{}

\begin{figure}
\centering
\includegraphics{images/c15f002.jpg}
\caption{{\protect\hyperlink{c15.xhtmlux5cux23figureanchor15-2}{\textbf{FIGURE
15.2}} A switched network with switching loops}}
\end{figure}

This is a switched network with a redundant topology that includes
switching loops. Without some type of layer 2 mechanism in place to
prevent a network loop, this network is vulnerable to nasty issues like
broadcast storms, multiple frame copies, and MAC table thrashing!
\protect\hyperlink{c15.xhtmlux5cux23figure15-3}{Figure 15.3} shows how
this network would work with STP working on the switches.

\begin{figure}
\centering
\includegraphics{images/c15f003.jpg}
\caption{{\protect\hyperlink{c15.xhtmlux5cux23figureanchor15-3}{\textbf{FIGURE
15.3}} A switched network with STP}}
\end{figure}

There a few types of spanning-tree protocols, but I'll start with the
IEEE version 802.1d, which happens to be the default on all Cisco IOS
switches.

\subsubsection[Spanning-Tree
Terms]{\texorpdfstring{\protect\hypertarget{c15.xhtmlux5cux23c15-sec-10}{}{}Spanning-Tree
Terms}{Spanning-Tree Terms}}

Now, before I get into describing the details of how STP works within a
network, it would be good for you to have these basic ideas and terms
down first:

\textbf{Root bridge} The \emph{root bridge} is the bridge with the
lowest and, therefore, the best bridge ID. The switches within the STP
network elect a root bridge, which becomes the focal
\protect\hypertarget{c15.xhtmlux5cux23Page_604}{}{}point in the network.
All other decisions in the network, like which ports on the non-root
bridges should be blocked or put in forwarding mode, are made from the
perspective of the root bridge, and once it has been elected, all other
bridges must create a single path to it. The port with the best path to
the root bridge is called the root port.

\textbf{Non-root bridges} These are all bridges that aren't the root
bridge. Non-root bridges exchange BPDUs with all the other bridges and
update the STP topology database on all switches. This prevents loops
and helps defend against link failures.

\textbf{BPDU} All switches exchange information to use for the
subsequent configuration of the network. Each switch compares the
parameters in the \emph{Bridge Protocol Data Unit (BPDU)} that it sends
to a neighbor with the parameters in the BPDU that it receives from
other neighbors. Inside the BPDU is the bridge ID.

\textbf{Bridge ID} The bridge ID is how STP keeps track of all the
switches in the network. It's determined by a combination of the bridge
priority, which is 32,768 by default on all Cisco switches, and the base
MAC address. The bridge with the lowest bridge ID becomes the root
bridge in the network. Once the root bridge is established, every other
switch must make a single path to it. Most networks benefit by forcing a
specific bridge or switch to be on the root bridge by setting its bridge
priority lower than the default value.

\textbf{Port cost} Port cost determines the best path when multiple
links are used between two switches. The cost of a link is determined by
the bandwidth of a link, and this path cost is the deciding factor used
by every bridge to find the most efficient path to the root bridge.

\textbf{Path cost} A switch may encounter one or more switches on its
path to the root bridge, and there may be more than one possible path.
All unique paths are analyzed individually, and a path cost is
calculated for each unique path by adding the individual port costs
encountered on the way to the root bridge.

\paragraph{Bridge Port Roles}

STP uses roles to determine how a port on a switch will act within the
spanning-tree algorithm.

\textbf{Root port} The root port is the link with the lowest path cost
to the root bridge. If more than one link connects to the root bridge,
then a port cost is found by checking the bandwidth of each link. The
lowest-cost port becomes the root port. When multiple links connect to
the same device, the port connected to the lowest port number on the
upstream switch will be the one that's used. The root bridge can never
have a root port designation, while every other switch in a network must
have one and only one root port.

\textbf{Designated port} A \emph{designated port} is one that's been
determined to have the best (lowest) cost to get to on a given network
segment, compared to other ports on that segment. A designated port will
be marked as a forwarding port, and you can have only one forwarding
port per network segment.

\textbf{Non-designated port} A \emph{non-designated port} is one with a
higher cost than the designated port. These are basically the ones left
over after the root ports and designated ports have
\protect\hypertarget{c15.xhtmlux5cux23Page_605}{}{}been determined.
Non-designated ports are put in blocking or discarding mode---they are
not forwarding ports!

\textbf{Forwarding port} A forwarding port forwards frames and will be
either a root port or a designated port.

\textbf{Blocked port} A blocked port won't forward frames in order to
prevent loops. A blocked port will still always listen to BPDU frames
from neighbor switches, but it will drop any and all other frames
received and will never transmit a frame.

\textbf{Alternate port} This corresponds to the blocking state of 802.1d
and is a term used with the newer 802.1w (Rapid Spanning Tree Protocol).
An alternative port is located on a switch connected to a LAN segment
with two or more switches connected, and one of the other switches holds
the designated port.

\textbf{Backup port} This corresponds to the blocking state of 802.1d
and is a term now used with the newer 802.1w. A backup port is connected
to a LAN segment where another port on that switch is acting as the
designated port.

\paragraph{Spanning-Tree Port States}

Okay, so you plug your host into a switch port and the light turns amber
and your host doesn't get a DHCP address from the server. You wait and
wait and finally the light goes green after almost a full
minute---that's an eternity in today's networks! This is the STA
transitioning through the different port states verifying that you
didn't just create a loop with the device you just plugged in. STP would
rather time out your new host than allow a loop into the network because
that would effectively bring your network to its knees. Let's talk about
the transition states; then later in this chapter we'll talk about how
to speed this process up.

The ports on a bridge or switch running IEEE 802.1d STP can transition
through five different states:

\textbf{Disabled (technically, not a transition state)} A port in the
administratively disabled state doesn't participate in frame forwarding
or STP. A port in the disabled state is virtually nonoperational.

\textbf{Blocking} As I mentioned, a blocked port won't forward frames;
it just listens to BPDUs. The purpose of the blocking state is to
prevent the use of looped paths. All ports are in blocking state by
default when the switch is powered up.

\textbf{Listening} This port listens to BPDUs to make sure no loops
occur on the network before passing data frames. A port in listening
state prepares to forward data frames without populating the MAC address
table.

\textbf{Learning} The switch port listens to BPDUs and learns all the
paths in the switched network. A port in learning state populates the
MAC address table but still doesn't forward data frames. Forward delay
refers to the time it takes to transition a port from listening to
learning mode, or from learning to forwarding mode, which is set to 15
seconds by default and can be seen in the \texttt{show\ spanning-tree}
output.

\protect\hypertarget{c15.xhtmlux5cux23Page_606}{}{}\textbf{Forwarding}
This port sends and receives all data frames on the bridged port. If the
port is still a designated or root port at the end of the learning
state, it will enter the forwarding state.

\begin{center}\rule{0.5\linewidth}{0.5pt}\end{center}

\includegraphics{images/note.png}Switches populate the MAC address table
in learning and forwarding modes only.

\begin{center}\rule{0.5\linewidth}{0.5pt}\end{center}

Switch ports are most often in either the blocking or forwarding state.
A forwarding port is typically the one that's been determined to have
the lowest (best) cost to the root bridge. But when and if the network
experiences a topology change due to a failed link or because someone
has added in a new switch, you'll see the ports on a switch
transitioning through listening and learning states.

As I said earlier, blocking ports is a strategy for preventing network
loops. Once a switch determines the best path to the root bridge for its
root port and any designated ports, all other redundant ports will be in
blocking mode. Blocked ports can still receive BPDUs---they just don't
send out any frames.

If a switch determines that a blocked port should become the designated
or root port because of a topology change, it will go into listening
mode and check all BPDUs it receives to make sure it won't create a loop
once the port moves into forwarding mode.

\paragraph{Convergence}

Convergence occurs when all ports on bridges and switches have
transitioned to either forwarding or blocking modes. No data will be
forwarded until convergence is complete. Yes---you read that right: When
STP is converging, all host data stops transmitting through the
switches! So if you want to remain on speaking terms with your network's
users, or remain employed for any length of time, you must make sure
that your switched network is physically designed really well so that
STP can converge quickly!

Convergence is vital because it ensures that all devices have a coherent
database. And making sure this happens efficiently will definitely
require your time and attention. The original STP (802.1d) takes 50
seconds to go from blocking to forwarding mode by default and I don't
recommend changing the default STP timers. You can adjust those timers
for a large network, but the better solution is simply to opt out of
using 802.1d at all! We'll get to the various STP versions in a minute.

\paragraph{Link Costs}

Now that you know about the different port roles and states, you need to
really understand all about path cost before we put this all together.
Port cost is based on the speed of the link, and
\protect\hyperlink{c15.xhtmlux5cux23table15-1}{Table 15.1} breaks down
the need-to-know path costs for you. Port cost is the cost of a single
link, whereas path cost is the sum of the various port costs to the root
bridge.

\protect\hypertarget{c15.xhtmlux5cux23Page_607}{}{}

{\protect\hyperlink{c15.xhtmlux5cux23tableanchor15-1}{\textbf{Table
15.1}} IEEE STP link costs}

\begin{longtable}[]{@{}ll@{}}
\toprule
Speed & Cost\tabularnewline
\midrule
\endhead
10 Mb/s & 100\tabularnewline
100 Mb/s & 19\tabularnewline
1000 Mb/s & 4\tabularnewline
10,000 Mb/s & 2\tabularnewline
\bottomrule
\end{longtable}

These costs will be used in the STP calculations to choose a single root
port on each bridge. You absolutely need to memorize this table, but no
worries---I'll guide you through lots of examples in this chapter to
help you do that quite easily! Now it's time to take everything we've
learned so far and put it all together.

\subsubsection[Spanning-Tree
Operations]{\texorpdfstring{\protect\hypertarget{c15.xhtmlux5cux23c15-sec-11}{}{}Spanning-Tree
Operations}{Spanning-Tree Operations}}

Let's start neatly summarizing what you've learned so far using the
simple three-switch network connected together as shown in
\protect\hyperlink{c15.xhtmlux5cux23figure15-4}{Figure 15.4}.

\begin{figure}
\centering
\includegraphics{images/c15f004.jpg}
\caption{{\protect\hyperlink{c15.xhtmlux5cux23figureanchor15-4}{\textbf{FIGURE
15.4}} STP operations}}
\end{figure}

Basically, STP's job is to find all the links in the network and shut
down any redundant ones, thereby preventing network loops from
occurring. It achieves this by first electing a root bridge that will
have all ports forwarding and will also act as a point of reference for
all other devices within the STP domain. In
\protect\hyperlink{c15.xhtmlux5cux23figure15-4}{Figure 15.4}, S1 has
been elected the root
\protect\hypertarget{c15.xhtmlux5cux23Page_608}{}{}bridge based on
bridge ID. Since the priorities are all equal to 32,768, we'll compare
MAC addresses and find that the MAC address of S1 is lower than that of
S2 and S3, meaning that S1 has a better bridge ID.

Once all switches agree on the root bridge, they must then determine
their one and only root port---the single path to the root bridge. It's
really important to remember that a bridge can go through many other
bridges to get to the root, so it's not always the shortest path that
will be chosen. That role will be given to the port that happens to
offer the fastest, highest bandwidth.
\protect\hyperlink{c15.xhtmlux5cux23figure15-5}{Figure 15.5} shows the
root ports for both non-root bridges (the \emph{RP} signifies a root
port and the \emph{F} signifies a designated forwarding port).

\protect\hypertarget{c15.xhtmlux5cux23Page_609}{}{}

\begin{figure}
\centering
\includegraphics{images/c15f005.jpg}
\caption{{\protect\hyperlink{c15.xhtmlux5cux23figureanchor15-5}{\textbf{FIGURE
15.5}} STP operations}}
\end{figure}

Looking at the cost of each link, it's clear why S2 and S3 are using
their directly connected links, because a gigabit link has a cost of 4.
For example, if S3 chose the path through S2 as its root port, we'd have
to add up each port cost along the way to the root, which would be 4 + 4
for a total cost of 8.

Every port on the root bridge is a designated, or forwarding, port for a
segment, and after the dust settles on all other non-root bridges, any
port connection between switches that isn't either a root port or a
designated port will predictably become a non-designated port. These
will again be put into the blocking state to prevent switching loops.

Okay---at this point, we have our root bridge with all ports in
forwarding state and we've found our root ports for each non-root
bridge. Now the only thing left to do is to choose the one forwarding
port on the segment between S2 and S3. Both bridges can't be forwarding
on a segment because that's exactly how we would end up with loops. So,
based on the bridge ID, the port with the best and lowest would become
the only bridge forwarding on that segment, with the one having the
highest, worst bridge ID put into blocking mode.
\protect\hyperlink{c15.xhtmlux5cux23figure15-6}{Figure 15.6} shows the
network after STP has converged.

\begin{figure}
\centering
\includegraphics{images/c15f006.jpg}
\caption{{\protect\hyperlink{c15.xhtmlux5cux23figureanchor15-6}{\textbf{FIGURE
15.6}} STP operations}}
\end{figure}

Since S3 had a lower bridge ID (better), S2's port went into blocking
mode. Let's discuss the root bridge election process more completely
now.

\paragraph{Selecting the Root Bridge}

The bridge ID is used to elect the root bridge in the STP domain and to
determine the root port for each of the remaining devices when there's
more than one potential root port available because they have equal-cost
paths. This key bridge ID is 8 bytes long and includes both the priority
and the MAC address of the device, as illustrated in
\protect\hyperlink{c15.xhtmlux5cux23figure15-7}{Figure 15.7}.
Remember---the default priority on all devices running the IEEE STP
version is 32,768.

\begin{figure}
\centering
\includegraphics{images/c15f007.jpg}
\caption{{\protect\hyperlink{c15.xhtmlux5cux23figureanchor15-7}{\textbf{FIGURE
15.7}} STP operations}}
\end{figure}

So, to determine the root bridge, you combine the priority of each
bridge with its MAC address. If two switches or bridges happen to have
the same priority value, the MAC address becomes the tiebreaker for
figuring out which one has the lowest and, therefore, best ID. This
means that because the two switches in
\protect\hyperlink{c15.xhtmlux5cux23figure15-7}{Figure 15.7} are both
using the default priority of 32,768, the MAC address will be the
determining factor instead. And because Switch A's MAC address is
0000.0cab.3274 and Switch B's MAC address is 0000.0cf6.9370, Switch
\protect\hypertarget{c15.xhtmlux5cux23Page_610}{}{}A wins and will
become the root bridge. A really easy way to figure out the lowest MAC
address is to just start reading from the left toward the right until
you find a lesser value. For Switch A, I only needed to get to 0000.0ca
before stopping. Switch A wins since switch B is 0000.0cf. Never forget
that the lower value is always the better one when it comes to electing
a root bridge!

I want to point out that prior to the election of the root bridge, BPDUs
are sent every 2 seconds out all active ports on a bridge/switch by
default, and they're received and processed by all bridges. The root
bridge is elected based on this information. You can change the bridge's
ID by lowering its priority so that it will become a root bridge
automatically. Being able to do that is important in a large switched
network because it ensures that the best paths will actually be the ones
chosen. Efficiency is always awesome in networking!

\subsection[Types of Spanning-tree
Protocols]{\texorpdfstring{\protect\hypertarget{c15.xhtmlux5cux23c15-sec-12}{}{}Types
of Spanning-tree Protocols}{Types of Spanning-tree Protocols}}

There are several varieties of spanning-tree protocols in use today:

\textbf{IEEE 802.1d} The original standard for bridging and STP, which
is really slow but requires very little bridge resources. It's also
referred to as Common Spanning Tree (CST).

\textbf{PVST+ (Cisco default version)} Per-VLAN Spanning Tree+ (PVST+)
is the Cisco proprietary enhancement for STP that provides a separate
802.1d spanning-tree instance for each VLAN. Know that this is just as
slow as the CST protocol, but with it, we get to have multiple root
bridges. This creates more efficiency of the links in the network, but
it does use more bridge resources than CST does.

\textbf{IEEE 802.1w} Also called Rapid Spanning Tree Protocol (RSTP),
this iteration enhanced the BPDU exchange and paved the way for much
faster network convergence, but it still only allows for one root bridge
per network like CST. The bridge resources used with RSTP are higher
than CST's but less than PVST+.

\textbf{802.1s (MSTP)} IEEE standard that started out as Cisco propriety
MISTP. Maps multiple VLANs into the same spanning-tree instance to save
processing on the switch. It's basically a spanning-tree protocol that
rides on top of another spanning-tree protocol.

\textbf{Rapid PVST+} Cisco's version of RSTP that also uses PVST+ and
provides a separate instance of 802.1w per VLAN. It gives us really fast
convergence times and optimal traffic flow but predictably requires the
most CPU and memory of all.

\subsubsection[Common Spanning
Tree]{\texorpdfstring{\protect\hypertarget{c15.xhtmlux5cux23c15-sec-13}{}{}Common
Spanning Tree}{Common Spanning Tree}}

If you're running Common Spanning Tree (CST) in your switched network
with redundant links, there will be an election to choose what STP
considers to be the best root bridge for your network. That switch will
also become the root for all VLANs in your network and all bridges in
your network will create a single path to it. You can manually override
this selection and pick whichever bridge you want if it makes sense for
your particular network.

\protect\hypertarget{c15.xhtmlux5cux23Page_611}{}{}\protect\hyperlink{c15.xhtmlux5cux23figure15-8}{Figure
15.8} shows how a typical root bridge would look on your switched
network when running CST.

\begin{figure}
\centering
\includegraphics{images/c15f008.jpg}
\caption{{\protect\hyperlink{c15.xhtmlux5cux23figureanchor15-8}{\textbf{FIGURE
15.8}} Common STP example}}
\end{figure}

Notice that switch A is the root bridge for all VLANs even though it's
really not the best path for some VLANs because all switches must make a
single path to it! This is where Per-VLAN Spanning Tree+ (PVST+) comes
into play. Because it allows for a separate instance of STP for each
VLAN, it frees up the individual selection of the most optimal path.

\subsubsection[Per-VLAN Spanning
Tree+]{\texorpdfstring{\protect\hypertarget{c15.xhtmlux5cux23c15-sec-14}{}{}Per-VLAN
Spanning Tree+}{Per-VLAN Spanning Tree+}}

PVST+ is a Cisco proprietary extension to 801.2d STP that provides a
separate 802.1 spanning-tree instance for each VLAN configured on your
switches. All of the Cisco proprietary extensions were created to
improve convergence times, which is 50 seconds by default. Cisco IOS
switches run 802.1d PVST+ by default, which means you'll have optimal
path selection, but the convergence time will still be slow.

Creating a per-VLAN STP instance for each VLAN is worth the increased
CPU and memory requirements, because it allows for per-VLAN root
bridges. This feature allows the STP tree to be optimized for the
traffic of each VLAN by allowing you to configure the root bridge in the
center of each of them.
\protect\hyperlink{c15.xhtmlux5cux23figure15-9}{Figure 15.9} shows how
PVST+ would look in an optimized switched network with multiple
redundant links.

This root bridge placement clearly enables faster convergence as well as
optimal path determination. This version's convergence is really similar
to 802.1 CST's, which has one instance of STP no matter how many VLANs
you have configured on your network. The difference is that with PVST+,
convergence happens on a per-VLAN basis, with each VLAN running its own
instance of STP. \protect\hyperlink{c15.xhtmlux5cux23figure15-9}{Figure
15.9} shows us that we now have a nice, efficient root bridge selection
for each VLAN.

\protect\hypertarget{c15.xhtmlux5cux23Page_612}{}{}

\begin{figure}
\centering
\includegraphics{images/c15f009.jpg}
\caption{{\protect\hyperlink{c15.xhtmlux5cux23figureanchor15-9}{\textbf{FIGURE
15.9}} PVST+ provides efficient root bridge selection.}}
\end{figure}

To allow for the PVST+ to operate, there's a field inserted into the
BPDU to accommodate the extended system ID so that PVST+ can have a root
bridge configured on a per-STP instance, shown in
\protect\hyperlink{c15.xhtmlux5cux23figure15-10}{Figure 15.10}. The
bridge ID actually becomes smaller---only 4 bits---which means that we
would configure the bridge priority in blocks of 4,096 rather than in
increments of 1 as we did with CST. The extended system ID (VLAN ID) is
a 12-bit field, and we can even see what this field is carrying via
\texttt{show\ spanning-tree} command output, which I'll show you soon.

\begin{figure}
\centering
\includegraphics{images/c15f010.jpg}
\caption{{\protect\hyperlink{c15.xhtmlux5cux23figureanchor15-10}{\textbf{FIGURE
15.10}} PVST+ unique bridge ID}}
\end{figure}

But still, isn't there a way we can do better than a 50-second
convergence time? That's a really long time in today's world!

\paragraph{Rapid Spanning Tree Protocol 802.1w}

Wouldn't it be wonderful to have a solid STP configuration running on
your switched network, regardless of switch type, and still have all the
features we just discussed built
\protect\hypertarget{c15.xhtmlux5cux23Page_613}{}{}in and enabled on
every one of your switches too? Rapid Spanning Tree Protocol (RSTP)
serves up exactly this amazing capacity right to our networking table!

Cisco created proprietary extensions to ``fix'' all the sinkholes and
liabilities the IEEE 802.1d standard threw at us, with the main drawback
to them being they require extra configuration because they're Cisco
proprietary. But RSTP, the new 802.1w standard, brings us most of the
patches needed in one concise solution. Again, efficiency is golden!

RSTP, or IEEE 802.1w, is essentially an evolution of STP that allows for
much faster convergence. But even though it does address all the
convergence issues, it still only permits a single STP instance, so it
doesn't help to take the edge off suboptimal traffic flow issues. And as
I mentioned, to support that faster convergence, the CPU usage and
memory demands are slightly higher than CST's. The good news is that
Cisco IOS can run the Rapid PVST+ protocol---a Cisco enhancement of RSTP
that provides a separate 802.1w spanning-tree instance for each VLAN
configured within the network. But all that power needs fuel, and
although this version addresses both convergence and traffic flow
issues, it also demands the most CPU and memory of all solutions. And
it's also good news that Cisco's newest switches don't have a problem
with this protocol running on them.

\begin{center}\rule{0.5\linewidth}{0.5pt}\end{center}

\includegraphics{images/note.png}Keep in mind that Cisco documentation
may say STP 802.1d and RSTP 802.1w, but it is referring to the PVST+
enhancement of each version.

\begin{center}\rule{0.5\linewidth}{0.5pt}\end{center}

Understand that RSTP wasn't meant to be something completely new and
different. The protocol is more of an evolution than an innovation of
the 802.1d standard, which offers faster convergence whenever a topology
change occurs. Backward compatibility was a must when 802.1w was
created.

So, RSTP helps with convergence issues that were the bane of traditional
STP. Rapid PVST+ is based on the 802.1w standard in the same way that
PVST+ is based on 802.1d. The operation of Rapid PVST+ is simply a
separate instance of 802.1w for each VLAN. Here's a list to clarify how
this all breaks down:

\begin{enumerate}
\tightlist
\item
  RSTP speeds the recalculation of the spanning tree when the layer 2
  network topology changes.
\item
  It's an IEEE standard that redefines STP port roles, states, and
  BPDUs.
\item
  RSTP is extremely proactive and very quick, so it doesn't need the
  802.1d delay timers.
\item
  RSTP (802.1w) supersedes 802.1d while remaining backward compatible.
\item
  Much of the 802.1d terminology and most parameters remain unchanged.
\item
  802.1w is capable of reverting to 802.1d to interoperate with
  traditional switches on a per-port basis.
\end{enumerate}

\protect\hypertarget{c15.xhtmlux5cux23Page_614}{}{}And to clear up
confusion, there are also five terminology adjustments between 802.1d's
five port states and 802.1w's, compared here, respectively:

\begin{longtable}[]{@{}lll@{}}
\toprule
802.1d State & & 802.1w State\tabularnewline
\midrule
\endhead
Disabled & = & Discarding\tabularnewline
Blocking & = & Discarding\tabularnewline
Listening & = & Discarding\tabularnewline
Learning & = & Learning\tabularnewline
Forwarding & = & Forwarding\tabularnewline
\bottomrule
\end{longtable}

Make note of the fact that RSTP basically just goes from discarding to
learning to forwarding, whereas 802.1d requires five states to
transition.

The task of determining the root bridge, root ports, and designated
ports hasn't changed from 802.1d to RSTP, and understanding the cost of
each link is still key to making these decisions well. Let's take a look
at an example of how to determine ports using the revised IEEE cost
specifications in
\protect\hyperlink{c15.xhtmlux5cux23figure15-11}{Figure 15.11}.

\begin{figure}
\centering
\includegraphics{images/c15f011.jpg}
\caption{{\protect\hyperlink{c15.xhtmlux5cux23figureanchor15-11}{\textbf{FIGURE
15.11}} RSTP example 1}}
\end{figure}

Can you figure out which is the root bridge? How about which port is the
root and which ones are designated? Well, because SC has the lowest MAC
address, it becomes the root bridge, and since all ports on a root
bridge are forwarding designated ports, well, that's easy, right? Ports
Gi0/1 and Gi0/10 become designated forwarding ports on SC.

But which one would be the root port for SA? To figure that out, we must
first find the port cost for the direct link between SA and SC. Even
though the root bridge (SC) has a Gigabit Ethernet port, it's running at
100 Mbps because SA's port is a 100-Mbps port, giving it a cost of 19.
If the paths between SA and SC were both Gigabit Ethernet, their
\protect\hypertarget{c15.xhtmlux5cux23Page_615}{}{}costs would only be
4, but because they're running 100 Mbps links instead, the cost jumps to
a whopping 19!

Can you find SD's root port? A quick glance at the link between SC and
SD tells us that's a Gigabit Ethernet link with a cost of 4, so the root
port for SD would be its Gi0/9 port.

The cost of the link between SB and SD is also 19 because it's also a
Fast Ethernet link, bringing the full cost from SB to SD to the root
(SC) to a total cost of 19 + 4 = 23. If SB were to go through SA to get
to SC, then the cost would be 19 + 19, or 38, so the root port of SB
becomes the Fa0/3 port.

The root port for SA would be the Fa0/0 port since that's a direct link
with a cost of 19. Going through SB to SD would be 19 + 19 + 4 = 42, so
we'll use that as a backup link for SA to get to the root just in case
we need to.

Now all we need is a forwarding port on the link between SA and SB.
Because SA has the lowest bridge ID, Fa0/1 on SA wins that role. Also,
the Gi0/1 port on SD would become a designated forwarding port. This is
because the SB Fa0/3 port is a designed root port and you must have a
forwarding port on a network segment! This leaves us with the Fa0/2 port
on SB. Since it isn't a root port or designated forwarding port, it will
be placed into blocking mode, which will prevent looks in our network.

Let's take a look at this example network when it has converged in
\protect\hyperlink{c15.xhtmlux5cux23figure15-12}{Figure 15.12}.

\begin{figure}
\centering
\includegraphics{images/c15f012.jpg}
\caption{{\protect\hyperlink{c15.xhtmlux5cux23figureanchor15-12}{\textbf{FIGURE
15.12}} RSTP example 1 answer}}
\end{figure}

If this isn't clear and still seems confusing, just remember to always
tackle this process following these three steps:

\begin{enumerate}
\item
  Find your root bridge by looking at bridge IDs.
\item
  Determine your root ports by finding the lowest path cost to the root
  bridge.
\item
  Find your designated ports by looking at bridge IDs.

  As usual, the best way to nail this down is to practice, so let's
  explore another scenario, shown in
  \protect\hyperlink{c15.xhtmlux5cux23figure15-13}{Figure 15.13}.
\end{enumerate}

\protect\hypertarget{c15.xhtmlux5cux23Page_616}{}{}

\begin{figure}
\centering
\includegraphics{images/c15f013.jpg}
\caption{{\protect\hyperlink{c15.xhtmlux5cux23figureanchor15-13}{\textbf{FIGURE
15.13}} RSTP example 2}}
\end{figure}

So which bridge is our root bridge? Checking priorities first tells us
that SC is the root bridge, which means all ports on SC are designated
forwarding ports. Now we need to find our root ports.

We can quickly see that SA has a 10-gigabit port to SC, so that would be
a port cost of 2, and it would be our root port. SD has a direct Gigabit
Ethernet port to SC, so that would be the root port for SD with a port
cost of 4. SB's best path would also be the direct Gigabit Ethernet port
to SC with a port cost of 4.

Now that we've determined our root bridge and found the three root ports
we need, we've got to find our designated ports next. Whatever is left
over simply goes into the discarding role. Let's take a look at
\protect\hyperlink{c15.xhtmlux5cux23figure15-14}{Figure 15.14} and see
what we have.

\begin{figure}
\centering
\includegraphics{images/c15f014.jpg}
\caption{{\protect\hyperlink{c15.xhtmlux5cux23figureanchor15-14}{\textbf{FIGURE
15.14}} RSTP example 2, answer 1}}
\end{figure}

All right, it looks like there are two links to choose between to find
one designated port per segment. Let's start with the link between SA
and SD. Which one has the best bridge
\protect\hypertarget{c15.xhtmlux5cux23Page_617}{}{}ID? They're both
running the same default priority, so by looking at the MAC address, we
can see that SD has the better bridge ID (lower), so the SA port toward
SD will go into a discarding role, or will it? The SD port will go into
discarding mode, because the link from SA to the root has the lowest
accumulated path costs to the root bridge, and that is used before the
bridge ID in this circumstance. It makes sense to let the bridge with
the fastest path to the root bridge be a designated forwarding port.
Let's talk about this a little more in depth.

As you know, once your root bridge and root ports have been chosen,
you're left with finding your designated ports. Anything left over goes
into a discarding role. But how are the designated ports chosen? Is it
just bridge ID? Here are the rules:

\begin{enumerate}
\item
  To choose the switch that will forward on the segment, we select the
  switch with the lowest accumulated path cost to the root bridge. We
  want the fast path to the root bridge.
\item
  If there is a tie on the accumulated path cost from both switches to
  the root bridge, then we'll use bridge ID, which was what we used in
  our previous example (but not with this latest RSTP example; not with
  a 10-Gigabit Ethernet link to the root bridge available!).
\item
  Port priorities can be set manually if we want a specific port chosen.
  The default priority is 32, but we can lower that if needed.
\item
  If there are two links between switches, and the bridge ID and
  priority are tied, the port with the lowest number will be
  chosen---for example, Fa0/1 would be chosen over Fa0/2.

  Let's take a look at our answer now, but before we do, can you find
  the forwarding port between SA and SB? Take a look at
  \protect\hyperlink{c15.xhtmlux5cux23figure15-15}{Figure 15.15} for the
  answer.
\end{enumerate}

\begin{figure}
\centering
\includegraphics{images/c15f015.jpg}
\caption{{\protect\hyperlink{c15.xhtmlux5cux23figureanchor15-15}{\textbf{FIGURE
15.15}} RSTP example 2, answer 2}}
\end{figure}

Again, to get the right answer to this question we're going to let the
switch on the network segment with the lowest accumulated path cost to
the root bridge forward on that segment. This is definitely SA, meaning
the SB port goes into discarding role---not so hard at all!

\paragraph[802.1s
(MSTP)]{\texorpdfstring{\protect\hypertarget{c15.xhtmlux5cux23Page_618}{}{}802.1s
(MSTP)}{802.1s (MSTP)}}

Multiple Spanning Tree Protocol (MSTP), also known as IEEE 802.ls, gives
us the same fast convergence as RSTP but reduces the number of required
STP instances by allowing us to map multiple VLANs with the same traffic
flow requirements into the same spanning-tree instance. It essentially
allows us to create VLAN sets and basically is a spanning-tree protocol
that runs on top of another spanning-tree protocol.

So clearly, you would opt to use MSTP over RSTP when you've got a
configuration involving lots of VLANs, resulting in CPU and memory
requirements that would be too high otherwise. But there's no free
lunch---though MSTP reduces the demands of Rapid PVST+, you've got to
configure it correctly because MSTP does nothing by itself!

\subsection[Modifying and Verifying the Bridge
ID]{\texorpdfstring{\protect\hypertarget{c15.xhtmlux5cux23c15-sec-15}{}{}Modifying
and Verifying the Bridge ID}{Modifying and Verifying the Bridge ID}}

To verify spanning tree on a Cisco switch, just use the command
\texttt{show\ spanning-tree}. From its output, we can determine our root
bridge, priorities, root ports, and designated and blocking/discarding
ports.

Let's use the same simple three-switch network we used earlier as the
base to play around with the configuration of STP.
\protect\hyperlink{c15.xhtmlux5cux23figure15-16}{Figure 15.16} shows the
network we'll work with in this section.

\begin{figure}
\centering
\includegraphics{images/c15f016.jpg}
\caption{{\protect\hyperlink{c15.xhtmlux5cux23figureanchor15-16}{\textbf{FIGURE
15.16}} Our simple three-switch network}}
\end{figure}

Let's start by taking a look at the output from S1:

\begin{verbatim}
S1#sh spanning-tree vlan 1
VLAN0001
  Spanning tree enabled protocol ieee
  Root ID    Priority    32769
             Address     0001.42A7.A603
             This bridge is the root
             Hello Time  2 sec  Max Age 20 sec  Forward Delay 15 sec
 
  Bridge ID  Priority    32769  (priority 32768 sys-id-ext 1)
             Address     0001.42A7.A603 him
             Hello Time  2 sec  Max Age 20 sec  Forward Delay 15 sec
             Aging Time  20
 
Interface        Role Sts Cost      Prio.Nbr Type
---------------- ---- --- --------- -------- --------------------------------
Gi1/1            Desg FWD 4         128.25   P2p
Gi1/2            Desg FWD 4         128.26   P2p
\end{verbatim}

First, we can see that we're running the IEEE 802.1d STP version by
default, and don't forget that this is really 802.1d PVST+! Looking at
the output, we can see that S1 is the root bridge for VLAN 1. When you
use this command, the top information is about the root bridge, and the
Bridge ID output refers to the bridge you're looking at. In this
example, they are one and the same. Notice the
\texttt{sys-id-ext\ 1\ (for\ VLAN\ 1)}. This is the 12-bit PVST+ field
that is placed into the BPDU so it can carry multiple-VLAN information.
You add the priority and \texttt{sys-id-ext} to come up with the true
priority for the VLAN. We can also see from the output that both Gigabit
Ethernet interfaces are designated forwarding ports. You will not see a
blocked/discarding port on a root bridge. Now let's take a look at S3's
output:

\begin{verbatim}
S3#sh spanning-tree
VLAN0001
  Spanning tree enabled protocol ieee
  Root ID    Priority    32769
             Address     0001.42A7.A603
             Cost        4
             Port        26(GigabitEthernet1/2)
             Hello Time  2 sec  Max Age 20 sec  Forward Delay 15 sec
 
  Bridge ID  Priority    32769  (priority 32768 sys-id-ext 1)
             Address     000A.41D5.7937
             Hello Time  2 sec  Max Age 20 sec  Forward Delay 15 sec
             Aging Time  20
 
Interface        Role Sts Cost      Prio.Nbr Type
---------------- ---- --- --------- -------- --------------------------------
Gi1/1            Desg FWD 4         128.25   P2p
Gi1/2            Root FWD 4         128.26   P2p
\end{verbatim}

Looking at the Root ID, it's easy to see that S3 isn't the root bridge,
but the output tells us it's a cost of 4 to get to the root bridge and
also that it's located out port 26 of the switch (Gi1/2). This tells us
that the root bridge is one Gigabit Ethernet link away,
\protect\hypertarget{c15.xhtmlux5cux23Page_620}{}{}which we already know
is S1, but we can confirm this with the\texttt{show\ cdp\ neighbors}
command:

\begin{verbatim}
Switch#sh cdp nei
Capability Codes: R - Router, T - Trans Bridge, B - Source Route Bridge
                  S - Switch, H - Host, I - IGMP, r - Repeater, P - Phone
Device ID    Local Intrfce   Holdtme    Capability   Platform    Port ID
S3           Gig 1/1          135            S       2960        Gig 1/1
S1           Gig 1/2          135            S       2960        Gig 1/1
\end{verbatim}

That's how simple it is to find your root bridge if you don't have the
nice figure as we do. Use the \texttt{show\ spanning-tree} command, find
your root port, and then use the \texttt{show\ cdp\ neighbors} command.
Let's see what S2's output has to tell us now:

\begin{verbatim}
S2#sh spanning-tree
VLAN0001
  Spanning tree enabled protocol ieee
  Root ID    Priority    32769
             Address     0001.42A7.A603
             Cost        4
             Port        26(GigabitEthernet1/2)
             Hello Time  2 sec  Max Age 20 sec  Forward Delay 15 sec
 
  Bridge ID  Priority    32769  (priority 32768 sys-id-ext 1)
             Address     0030.F222.2794
             Hello Time  2 sec  Max Age 20 sec  Forward Delay 15 sec
             Aging Time  20
 
Interface        Role Sts Cost      Prio.Nbr Type
---------------- ---- --- --------- -------- --------------------------------
Gi1/1            Altn BLK 4         128.25   P2p
Gi1/2            Root FWD 4         128.26   P2p
\end{verbatim}

We're certainly not looking at a root bridge since we're seeing a
blocked port, which is S2's connection to S3!

Let's have some fun by making S2 the root bridge for VLAN 2 and for VLAN
3. Here's how easy that is to do:

\begin{verbatim}
S2#sh spanning-tree vlan 2
VLAN0002
  Spanning tree enabled protocol ieee
  Root ID    Priority    32770
             Address     0001.42A7.A603
             Cost        4
             Port        26(GigabitEthernet1/2)
             Hello Time  2 sec  Max Age 20 sec  Forward Delay 15 sec
 
  Bridge ID  Priority    32770  (priority 32768 sys-id-ext 2)
             Address     0030.F222.2794
             Hello Time  2 sec  Max Age 20 sec  Forward Delay 15 sec
             Aging Time  20
 
Interface        Role Sts Cost      Prio.Nbr Type
---------------- ---- --- --------- -------- --------------------------------
Gi1/1            Altn BLK 4         128.25   P2p
Gi1/2            Root FWD 4         128.26   P2p
\end{verbatim}

We can see that the root bridge cost is 4, meaning that the root bridge
is one gigabit link away. One more key factor I want to talk about
before making S2 the root bridge for VLANs 2 and 3 is the
\texttt{sys-id-ext}, which shows up as 2 in this output because this
output is for VLAN 2. This \texttt{sys-id-ext} is added to the bridge
priority, which in this case is 32768 + 2, which makes the priority
32770. Now that you understand what that output is telling us, let's
make S2 the root bridge:

\begin{verbatim}
S2(config)#spanning-tree vlan 2 ?
  priority  Set the bridge priority for the spanning tree
  root      Configure switch as root
  <cr>
S2(config)#spanning-tree vlan 2 priority ?
  <0-61440>  bridge priority in increments of 4096
S2(config)#spanning-tree vlan 2 priority 16384
\end{verbatim}

You can set the priority to any value from 0 through 61440 in increments
of 4096. Setting it to zero (0) means that the switch will always be a
root as long as it has a lower MAC address than another switch that also
has its bridge ID set to 0. If you want to set a switch to be the root
bridge for every VLAN in your network, then you have to change the
priority for each VLAN, with 0 being the lowest priority you can use.
But trust me---it's never a good idea to set all switches to a priority
of 0!

Furthermore, you don't actually need to change priorities because there
is yet another way to configure the root bridge. Take a look:

\begin{verbatim}
S2(config)#spanning-tree vlan 3 root ?
  primary    Configure this switch as primary root for this spanning tree
  secondary  Configure switch as secondary root
S2(config)#spanning-tree vlan 3 root primary
S3(config)#spanning-tree vlan 3 root secondary
\end{verbatim}

\protect\hypertarget{c15.xhtmlux5cux23Page_622}{}{}Notice that you can
set a bridge to either primary or secondary---very cool! If both the
primary and secondary switches go down, then the next highest priority
will take over as root.

Let's check to see if S2 is actually the root bridge for VLANs 2 and 3
now:

\begin{verbatim}
S2#sh spanning-tree vlan 2
VLAN0002
  Spanning tree enabled protocol ieee
  Root ID    Priority    16386
             Address     0030.F222.2794
             This bridge is the root
             Hello Time  2 sec  Max Age 20 sec  Forward Delay 15 sec
 
  Bridge ID  Priority    16386  (priority 16384 sys-id-ext 2)
             Address     0030.F222.2794
             Hello Time  2 sec  Max Age 20 sec  Forward Delay 15 sec
             Aging Time  20
 
Interface        Role Sts Cost      Prio.Nbr Type
---------------- ---- --- --------- -------- --------------------------------
Gi1/1            Desg FWD 4         128.25   P2p
Gi1/2            Desg FWD 4         128.26   P2p
\end{verbatim}

Nice---S2 is the root bridge for VLAN 2, with a priority of 16386 (16384
+ 2). Let's take a look to see the root bridge for VLAN 3. I'll use a
different command for that this time. Check it out:

\begin{verbatim}
S2#sh spanning-tree summary
Switch is in pvst mode
Root bridge for: VLAN0002 VLAN0003
Extended system ID           is enabled
Portfast Default             is disabled
PortFast BPDU Guard Default  is disabled
Portfast BPDU Filter Default is disabled
Loopguard Default            is disabled
EtherChannel misconfig guard is disabled
UplinkFast                   is disabled
BackboneFast                 is disabled
Configured Pathcost method used is short
 
Name                   Blocking Listening Learning Forwarding STP Active
---------------------- -------- --------- -------- ---------- ----------
VLAN0001                     1         0        0          1          2
VLAN0002                     0         0        0          2          2
VLAN0003                     0         0        0          2          2
 
---------------------- -------- --------- -------- ---------- ----------
3 vlans                      1         0        0          5          6
\end{verbatim}

The preceding output tells us that S2 is the root for the two VLANs, but
we can see we have a blocked port for VLAN 1 on S2, so it's not the root
bridge for VLAN 1. This is because there's another bridge with a better
bridge ID for VLAN 1 than S2's.

One last burning question: How do you enable RSTP on a Cisco switch?
Well, doing that is actually the easiest part of this chapter! Take a
look:

\begin{verbatim}
S2(config)#spanning-tree mode rapid-pvst
\end{verbatim}

Is that really all there is to it? Yes, because it's a global command,
not per VLAN. Let's verify we're running RSTP now:

\begin{verbatim}
S2#sh spanning-tree
VLAN0001
  Spanning tree enabled protocol rstp
  Root ID    Priority    32769
             Address     0001.42A7.A603
             Cost        4
             Port        26(GigabitEthernet1/2)
             Hello Time  2 sec  Max Age 20 sec  Forward Delay 15 sec
[output cut
S2#sh spanning-tree summary
Switch is in rapid-pvst mode
Root bridge for: VLAN0002 VLAN0003
\end{verbatim}

Looks like we're set! We're running RSTP, S1 is our root bridge for VLAN
1, and S2 is the root bridge for VLANs 2 and 3. I know this doesn't seem
hard, and it really isn't, but you still need to practice what we've
covered so far in this chapter to really get your skills solid!

\subsection[Spanning-Tree Failure
Consequences]{\texorpdfstring{\protect\hypertarget{c15.xhtmlux5cux23c15-sec-16}{}{}Spanning-Tree
Failure Consequences}{Spanning-Tree Failure Consequences}}

Clearly, there will be consequences when a routing protocol fails on a
single router, but mainly, you'll just lose connectivity to the networks
directly connected to that router, and it usually does not affect the
rest of your network. This definitely makes it easier to troubleshoot
and fix the issue!

There are two failure types with STP. One of them causes the same type
of issue I mentioned with a routing protocol; when certain ports have
been placed in a blocking state they should be forwarding on a network
segment instead. This situation makes the network
\protect\hypertarget{c15.xhtmlux5cux23Page_624}{}{}segment unusable, but
the rest of the network will still be working. But what happens when
blocked ports are placed into forwarding state when they should be
blocking? Let's work through this second failure issue now, using the
same layout we used in the last section. Let's start with
\protect\hyperlink{c15.xhtmlux5cux23figure15-17}{Figure 15.17} and then
find out what happens when STP fails. Squeamish readers be warned---this
isn't pretty!

Looking at \protect\hyperlink{c15.xhtmlux5cux23figure15-17}{Figure
15.17}, what do you think will happen if SD transitions its blocked port
to the forwarding state?

\begin{figure}
\centering
\includegraphics{images/c15f017.jpg}
\caption{{\protect\hyperlink{c15.xhtmlux5cux23figureanchor15-17}{\textbf{FIGURE
15.17}} STP stopping loops}}
\end{figure}

Clearly, the consequences to the entire network will be pretty
devastating! Frames that already had a destination address recorded in
the MAC address table of the switches are forwarded to the port they're
associated with; however, any broadcast, multicast, and unicasts not in
the CAM are now in an endless loop.
\protect\hyperlink{c15.xhtmlux5cux23figure15-18}{Figure 15.18} shows us
the carnage---when you see all the lights on each port blinking
super-fast amber/green, this means serious errors are occurring, and
lots of them!

\begin{figure}
\centering
\includegraphics{images/c15f018.jpg}
\caption{{\protect\hyperlink{c15.xhtmlux5cux23figureanchor15-18}{\textbf{FIGURE
15.18}} STP failure}}
\end{figure}

\protect\hypertarget{c15.xhtmlux5cux23Page_625}{}{}As frames begin
building up on the network, the bandwidth starts getting saturated. The
CPU percentage goes way up on the switches until they'll just give up
and stop working completely, and all this within a few seconds!

Here is a list of the problems that will occur in a failed STP network
that you must be aware of and you must be able to find in your
production network---and of course, you must know them to meet the exam
objectives:

\begin{enumerate}
\tightlist
\item
  The load on all links begins increasing and more and more frames enter
  the loop. Remember, this loop affects all the other links in the
  network because these frames are always flooded out all ports. This
  scenario is a little less dire if the loop occurs within a single
  VLAN. In that case, the snag will be isolated to ports only in that
  VLAN membership, plus all trunk links that carry information for that
  VLAN.
\item
  If you have more than one loop, traffic will increase on the switches
  because all the circling frames actually get duplicated. Switches
  basically receive a frame, make a copy of it, and send it out all
  ports. And they do this over and over and over again with the same
  frame, as well as for any new ones!
\item
  The MAC address table is now completely unstable. It no longer knows
  where any source MAC address hosts are actually located because the
  same source address comes in via multiple ports on the switch.
\item
  With the overwhelmingly high load on the links and the CPUs, now
  possibly at 100\% or close to that, the devices become unresponsive,
  making it impossible to troubleshoot---it's a terrible thing!
\end{enumerate}

At this point your only option is to systematically remove every
redundant link between switches until you can find the source of the
problem. And don't freak because, eventually, your ravaged network will
calm down and come back to life after STP converges. Your fried switches
will regain consciousness, but the network will need some serious
therapy, so you're not out of the woods yet!

Now is when you start troubleshooting to find out what caused the
disaster in the first place. A good strategy is to place the redundant
links back into your network one at a time and wait to see when a
problem begins to occur. You could have a failing switch port, or even a
dead switch. Once you've replaced all your redundant links, you need to
carefully monitor the network and have a back-out plan to quickly
isolate the problem if it reoccurs. You don't want to go through this
again!

You're probably wondering how to prevent these STP problems from ever
darkening your doorstep in the first place. Well, just hang on, because
after the next section, I'll tell you all about EtherChannel, which can
stop ports from being placed in the blocked/discarding state on
redundant links to save the day! But before we add more links to our
switches and then bundle them, let's talk about PortFast.

\subsection[PortFast and BPDU
Guard]{\texorpdfstring{\protect\hypertarget{c15.xhtmlux5cux23c15-sec-17}{}{}PortFast
and BPDU Guard}{PortFast and BPDU Guard}}

If you have a server or other devices connected into your switch that
you're totally sure won't create a switching loop if STP is disabled,
you can use a Cisco proprietary extension to the 802.1d standard called
PortFast on these ports. With this tool, the port won't spend
\protect\hypertarget{c15.xhtmlux5cux23Page_626}{}{}the usual 50 seconds
to come up into forwarding mode while STP is converging, which is what
makes it so cool.

Since ports will transition from blocking to forwarding state
immediately, PortFast can prevent our hosts from being potentially
unable to receive a DHCP address due to STP's slow convergence. If the
host's DHCP request times out, or if every time you plug a host in
you're just tired of looking at the switch port being amber for almost a
minute before it transitions to forwarding state and turns green,
PortFast can really help you out!

\protect\hyperlink{c15.xhtmlux5cux23figure15-19}{Figure 15.19}
illustrates a network with three switches, each with a trunk to each of
the others and a host and server off the S1 switch.

\begin{figure}
\centering
\includegraphics{images/c15f019.jpg}
\caption{{\protect\hyperlink{c15.xhtmlux5cux23figureanchor15-19}{\textbf{FIGURE
15.19}} PortFast}}
\end{figure}

We can use PortFast on the ports on S1 to help them transition to the
STP forwarding state immediately upon connecting to the switch.

Here are the commands, first from global config mode---they're pretty
simple:

\begin{verbatim}
S1(config)#spanning-tree portfast ?
  bpdufilter  Enable portfast bdpu filter on this switch
  bpduguard   Enable portfast bpdu guard on this switch
  default     Enable portfast by default on all access ports
\end{verbatim}

If you were to type \texttt{spanning-tree\ portfast\ default}, you would
enable all nontrunking ports with PortFast. From interface mode, you can
be more specific, which is the better way to go:

\begin{verbatim}
S1(config-if)#spanning-tree portfast ?
  disable  Disable portfast for this interface
  trunk    Enable portfast on the interface even in trunk mode
  <cr>
\end{verbatim}

From interface mode you can actually configure PortFast on a trunk port,
but you would do that only if the port connects to a server or router,
not to another switch, so we won't use that here. So let's take a look
at the message I get when I turn on PortFast on interface Gi0/1:

\begin{verbatim}
S1#config t
S1#config)#int range gi0/1 - 2
S1(config-if)#spanning-tree portfast
%Warning: portfast should only be enabled on ports connected to a single
 host. Connecting hubs, concentrators, switches, bridges, etc... to this
 interface  when portfast is enabled, can cause temporary bridging loops.
 Use with CAUTION
\end{verbatim}

\begin{verbatim}
 
%Portfast has been configured on GigabitEthernet0/1 but will only
 have effect when the interface is in a non-trunking mode.
\end{verbatim}

PortFast is enabled on port Gi0/1 and Gi0/2, but notice that you get a
pretty long message that's essentially telling you to be careful. This
is because when using PortFast, you definitely don't want to create a
network loop by plugging another switch or hub into a port that's also
configured with PortFast! Why? Because if you let this happen, even
though the network may still sort of work, data will pass super slowly,
and worse, it could take you a really long time to find the source of
the problem, making you very unpopular. So proceed with caution!

At this juncture, you would be happy to know that there are some
safeguard commands to have handy when using PortFast just in case
someone causes a loop in a port that's configured with PortFast enabled.
Let's talk about a really key safeguard command now.

\subsubsection[BPDU
Guard]{\texorpdfstring{\protect\hypertarget{c15.xhtmlux5cux23c15-sec-18}{}{}BPDU
Guard}{BPDU Guard}}

If you turn on PortFast for a switch port, it's a really good idea to
turn on BPDU Guard as well. In fact, it's such a great idea, I
personally feel that it should be enabled by default whenever a port is
configured with PortFast!

This is because if a switch port that has PortFast enabled receives a
BPDU on that port, it will place the port into error disabled (shutdown)
state, effectively preventing anyone from accidentally connecting
another switch or hub port into a switch port configured with PortFast.
Basically, you're preventing (guarding) your network from being severely
crippled or even brought down. So let's configure our S1 interface,
which is already configured with PortFast, with BPDU Guard now---it's
easy!

Here's how to set it globally:

\begin{verbatim}
S1(config)# spanning-tree portfast bpduguard default
\end{verbatim}

And specifically on an interface:

\begin{verbatim}
S1(config-if)#spanning-tree bpduguard enable
\end{verbatim}

It's important to know that you would only configure this command on
your access layer switches---switches where users are directly
connected.

\begin{center}\rule{0.5\linewidth}{0.5pt}\end{center}

\protect\hypertarget{c15.xhtmlux5cux23Page_628}{}{}\includegraphics{images/globe1.png}\\
\textbf{Hedging My Bets Created Bad Switch Ports during the Super Bowl}

A junior admin called me frantically telling me all switch ports had
just gone bad on the core switch, which was located at the data center
where I was lead consultant for a data center upgrade. Now these things
happen, but keep in mind that I just happened to be at a Super Bowl
party having a great time watching my favorite team play in the ``Big
One'' when I received this call! So I took a deep breath to refocus. I
needed to find out some key information to determine just how bad the
situation really was, and my client was in as big of a hurry as I was to
get to a solution!

First I asked the junior admin exactly what he did. Of course, he said,
``Nothing, I swear!'' I figured that's what he'd say, so I pressed him
for more info and finally asked for stats on the switch. The admin told
me that all the ports on the 10/100/1000 line card went amber at the
same time---finally some information I could use! I confirmed that, as
suspected, these ports trunked to uplink distribution switches.
Wow---this was not good!

At this point, though, I found it hard to believe that all 24 ports
would suddenly go bad, but it's possible, so I asked if he had a spare
card to try. He told me that he had already put in the new card but the
same thing was still happening. Well, it's not the card, or the ports,
but maybe something happened with the other switches. I knew there were
a lot of switches involved, so someone must have screwed something up to
make this catastrophe happen! Or, maybe the fiber distribution closet
went down somehow? If so, how? Was there a fire in the closet or
something? Some serious internal shenanigans would be the only answer if
that were the cause!

So remaining ever patient (because, to quote Dr. House, ``Patients
lie''), I again had to ask the admin exactly what he did, and sure
enough, he finally admitted that he tried to plug his personal laptop
into the core switch so he could watch the Super Bowl, and he quickly
added, ``\ldots but that's it, I didn't do anything else!'' I'll skip
over the fact that this guy was about to have the ugliest Monday ever,
but something still didn't make sense, and here's why.

Knowing that the ports on that card would all connect to distribution
switches, I configured the ports with PortFast so they wouldn't have to
transition through the STP process. And because I wanted to make sure no
one plugged a switch into any of those ports, I enabled BPDU Guard on
the entire line card.

But a host would not bring down those ports, so I asked him if he had
plugged in the laptop directly or used something in between. He admitted
that he had indeed used another switch because, turns out, there were
lots of people from the office who wanted to plug into the core switch
and watch the game too. Was he kidding me? The security policy wouldn't
allow connecting from their offices, so wouldn't you think they'd
consider the core even more off-limits? Some people!

\protect\hypertarget{c15.xhtmlux5cux23Page_629}{}{}But wait\ldots{} This
doesn't explain all ports turning amber, because only the one he plugged
into should be doing that. It took me a second, but I figured out what
he did and finally got him to confess. When he plugged the switch in,
the port turned amber so he thought it went bad. So what do think he
did? Well, if at first you don't succeed, try, try again, and that's
just what he did---he actually kept trying ports---all 24 of them to be
exact! Now that's what I call determined!

Sad to say, I got back to the party in time just to watch my team lose
in the last few minutes! A dark day, indeed!

\begin{center}\rule{0.5\linewidth}{0.5pt}\end{center}

\subsection[EtherChannel]{\texorpdfstring{\protect\hypertarget{c15.xhtmlux5cux23c15-sec-19}{}{}EtherChannel}{EtherChannel}}

Know that almost all Ethernet networks today will typically have
multiple links between switches because this kind of design provides
redundancy and resiliency. On a physical design that includes multiple
links between switches, STP will do its job and put a port or ports into
blocking mode. In addition to that, routing protocols like OSPF and
EIGRP could see all these redundant links as individual ones, depending
on the configuration, which can mean an increase in routing overhead.

We can gain the benefits from multiple links between switches by using
port channeling. EtherChannel is a port channel technology that was
originally developed by Cisco as a switch-to-switch technique for
grouping several Fast Ethernet or Gigabit Ethernet ports into one
logical channel.

Also important to note is that once your port channel (EtherChannel) is
up and working, layer 2 STP and layer 3 routing protocols will treat
those bundled links as a single one, which would stop STP from
performing blocking. An additional nice result is that because the
routing protocols now only see this as a single link, a single adjacency
across the link can be formed---elegant!

\protect\hyperlink{c15.xhtmlux5cux23figure15-20}{Figure 15.20} shows how
a network would look if we had four connections between switches, before
and after configuring port channels.

\begin{figure}
\centering
\includegraphics{images/c15f020.jpg}
\caption{{\protect\hyperlink{c15.xhtmlux5cux23figureanchor15-20}{\textbf{FIGURE
15.20}} Before and after port channels}}
\end{figure}

\protect\hypertarget{c15.xhtmlux5cux23Page_630}{}{}Now as usual, there's
the Cisco version and the IEEE version of port channel negotiation
protocols to choose from---take your pick. Cisco's version is called
Port Aggregation Protocol (PAgP), and the IEEE 802.3ad standard is
called Link Aggregation Control Protocol (LACP). Both versions work
equally well, but the way you configure each is slightly different. Keep
in mind that both PAgP and LACP are negotiation protocols and that
EtherChannel can actually be statically configured without PAgP or LACP.
Still, it's better to use one of these protocols to help with
compatibility issues as well as to manage link additions and failures
between two switches.

Cisco EtherChannel allows us to bundle up to eight ports active between
switches. The links must have the same speed, duplex setting, and VLAN
configuration---in other words, you can't mix interface types and
configurations into the same bundle.

There are a few differences in configuring PAgP and LACP, but first,
let's go over some terms so you don't get confused:

\textbf{Port channeling} Refers to combining two to eight Fast Ethernet
or two Gigabit Ethernet ports together between two switches into one
aggregated logical link to achieve more bandwidth and resiliency.

\textbf{EtherChannel} Cisco's proprietary term for port channeling.

\textbf{PAgP} This is a Cisco proprietary port channel negotiation
protocol that aids in the automatic creation for EtherChannel links. All
links in the bundle must match the same parameters (speed, duplex, VLAN
info), and when PAgP identifies matched links, it groups the links into
an EtherChannel. This is then added to STP as a single bridge port. At
this point, PAgP's job is to send packets every 30 seconds to manage the
link for consistency, any link additions, and failures.

\textbf{LACP (802.3ad)} This has the exact same purpose as PAgP, but
it's nonproprietary so it can work between multi-vendor networks.

\texttt{channel-group} This is a command on Ethernet interfaces used to
add the specified interface to a single EtherChannel. The number
following this command is the port channel ID.

\texttt{interface\ port-channel} Here's a command that creates the
bundled interface. Ports can be added to this interface with the
\texttt{channel-group} command. Keep in mind that the interface number
must match the group number.

Now let's see if you can make some sense out of all these terms by
actually configuring something!

\subsubsection[Configuring and Verifying Port
Channels]{\texorpdfstring{\protect\hypertarget{c15.xhtmlux5cux23c15-sec-20}{}{}Configuring
and Verifying Port Channels}{Configuring and Verifying Port Channels}}

Let's use \protect\hyperlink{c15.xhtmlux5cux23figure15-21}{Figure 15.21}
for our simple example of how to configure port channels.

\begin{figure}
\centering
\includegraphics{images/c15f021.jpg}
\caption{{\protect\hyperlink{c15.xhtmlux5cux23figureanchor15-21}{\textbf{FIGURE
15.21}} EtherChannel example}}
\end{figure}

\protect\hypertarget{c15.xhtmlux5cux23Page_631}{}{}You can enable your
\texttt{channel-group} for each channel by setting the channel mode for
each interface to either \texttt{active} or \texttt{passive} if using
LACP. When a port is configured in \texttt{passive} mode, it will
respond to the LACP packets it receives, but it won't initiate an LACP
negotiation. When a port is configured for \texttt{active} mode, the
port initiates negotiations with other ports by sending LACP packets.

Let me show you a simple example of configuring port channels and then
verifying them. First I'll go to global configuration mode and create a
port channel interface, and then I'll add this port channel to the
physical interfaces.

Remember, all parameters and configurations of the ports must be the
same, so I'll start by trunking the interfaces before I configure
EtherChannel, like this:

\begin{verbatim}
S1(config)#int range g0/1 - 2
S1(config-if-range)#switchport trunk encapsulation dot1q
S1(config-if-range)#switchport mode trunk
\end{verbatim}

All ports in your bundles must be configured the same, so I'll configure
both sides with the same trunking configuration. Now I can assign these
ports to a bundle:

\begin{verbatim}
S1(config-if-range)#channel-group 1 mode ?
  active     Enable LACP unconditionally
  auto       Enable PAgP only if a PAgP device is detected
  desirable  Enable PAgP unconditionally
  on         Enable Etherchannel only
  passive    Enable LACP only if a LACP device is detected
S1(config-if-range)#channel-group 1 mode active
S1(config-if-range)#exit
\end{verbatim}

To configure the IEEE nonproprietary LACP, I'll use the \texttt{active}
or \texttt{passive} command; if I wanted to use Cisco's PAgP, I'd use
the \texttt{auto} or \texttt{desirable} command. You can't mix and match
these on either end of the bundle, and really, it doesn't matter which
one you use in a pure Cisco environment, as long as you configure them
the same on both ends (setting the mode to \texttt{on} would be
statically configuring your EtherChannel bundle).

At this point in the configuration, I'd have to set the mode to
\texttt{active} on the S2 interfaces if I wanted the bundle to come up
with LACP because, again, all parameters must be the same on both ends
of the link. Let's configure our port channel interface, which was
created when we used the channel-group command:

\begin{verbatim}
S1(config)#int port-channel 1
S1(config-if)#switchport trunk encapsulation dot1q
S1(config-if)#switchport mode trunk
S1(config-if)#switchport trunk allowed vlan 1,2,3
\end{verbatim}

Notice that I set the same trunking method under the port channel
interface as I did the physical interfaces, as well as VLAN information
too. Nicely, all command performed under the port-channel are inherited
at the interface level, so you can just easily configure the
port-channel with all parameters.

\protect\hypertarget{c15.xhtmlux5cux23Page_632}{}{}Time to configure the
interfaces, channel groups, and port channel interface on the S2 switch:

\begin{verbatim}
S2(config)#int range g0/13 - 14
S2(config-if-range)#switchport trunk encapsulation dot1q
S2(config-if-range)#switchport mode trunk
S2(config-if-range)#channel-group 1 mode active
S2(config-if-range)#exit
S2(config)#int port-channel 1
S2(config-if)#switchport trunk encapsulation dot1q
S2(config-if)#switchport mode trunk
S2(config-if)#switchport trunk allowed vlan 1,2,3
\end{verbatim}

On each switch, I configured the ports I wanted to bundle with the same
configuration, then created the port channel. After that, I added the
ports into the port channel with the \texttt{channel-group} command.

Remember, for LACP we'll use either active/active on each side of the
bundle or active/passive, but you can't use passive/passive. Same goes
for PAgP; you can use desirable/desirable or auto/desirable but not
auto/auto.

Let's verify our EtherChannel with a few commands. We'll start with the
\texttt{show\ etherchannel\ port-channel} command to see information
about a specific port channel interface:

\begin{verbatim}
S2#sh etherchannel port-channel
               Channel-group listing:
                ----------------------
 
Group: 1
----------
                Port-channels in the group:
                ---------------------------
 
Port-channel: Po1    (Primary Aggregator)
------------
 
Age of the Port-channel   = 00d:00h:46m:49s
Logical slot/port   = 2/1       Number of ports = 2
GC                  = 0x00000000      HotStandBy port = null
Port state          = Port-channel
Protocol            =   LACP
Port Security       = Disabled
 
Ports in the Port-channel:
 
Index   Load   Port     EC state        No of bits
------+------+------+------------------+-----------
  0     00     Gig0/2   Active             0
  0     00     Gig0/1   Active             0
Time since last port bundled:    00d:00h:46m:47s    Gig0/1
S2#
\end{verbatim}

Notice that we have one group and that we're running the IEEE LACP
version of port channeling. We're in \texttt{Active} mode, and that
\texttt{Port-channel:\ Po1} interface has two physical interfaces. The
heading \texttt{Load} is not the load over the interfaces, it's a
hexadecimal value that decides which interface will be chosen to specify
the flow of traffic.

The \texttt{show\ etherchannel\ summary} command displays one line of
information per port channel:

\begin{verbatim}
S2#sh etherchannel summary
Flags:  D - down        P - in port-channel
        I - stand-alone s - suspended
        H - Hot-standby (LACP only)
        R - Layer3      S - Layer2
        U - in use      f - failed to allocate aggregator
        u - unsuitable for bundling
        w - waiting to be aggregated
        d - default port
 
Number of channel-groups in use: 1
Number of aggregators:           1
 
Group  Port-channel  Protocol    Ports
------+-------------+-----------+----------------------------------------------
 
1      Po1(SU)           LACP   Gig0/1(P) Gig0/2(P)
\end{verbatim}

This command shows that we have one group, that we're running LACP, and
Gig0/1 and Gig0/2 or (P), which means these ports are
\texttt{in\ port-channel} mode. This command isn't really all that
helpful unless you have multiple channel groups, but it does tell us our
group is working well!

\paragraph{Layer 3 EtherChannel}

One last item to discuss before we finish this chapter and that is layer
3 EtherChannel. You'd use layer 3 EtherChannel when connecting a switch
to multiple ports on a router, for example. It's important to understand
that you wouldn't put IP addresses under the
\protect\hypertarget{c15.xhtmlux5cux23Page_634}{}{}physical interfaces
of the router, instead you'd actually add the IP address of the bundle
under the logical port-channel interface.

Here is an example on how to create the logical port channel 1 and
assign 20.2.2.2 as its IP address:

\begin{verbatim}
Router#config t
Router(config)#int port-channel 1
Router(config-if)#ip address 20.2.2.2 255.255.255.0
\end{verbatim}

Now we need to add the physical ports into port channel 1:

\begin{verbatim}
Router(config-if)#int range g0/0-1
Router(config-if-range)#channel-group 1
GigabitEthernet0/0 added as member-1 to port-channel1
GigabitEthernet0/1 added as member-2 to port-channel1
\end{verbatim}

Now let's take a look at the running-config. Notice there are no IP
addresses under the physical interface of the router:

\begin{verbatim}
!
interface Port-channel1
 ip address 20.2.2.2 255.255.255.0
 load-interval 30
!
 interface GigabitEthernet0/0
 no ip address
 load-interval 30
 duplex auto
 speed auto
 channel-group 1
!
 interface GigabitEthernet0/1
 no ip address
 load-interval 30
 duplex auto
 speed auto
 channel-group 1
\end{verbatim}

\subsection[Summary]{\texorpdfstring{\protect\hypertarget{c15.xhtmlux5cux23c15-sec-21}{}{}Summary}{Summary}}

This chapter was all about switching technologies, with a particular
focus on the Spanning Tree Protocol (STP) and its evolution to newer
versions like RSTP and then Cisco's PVST+.

\protect\hypertarget{c15.xhtmlux5cux23Page_635}{}{}You learned about the
problems that can occur if you have multiple links between bridges
(switches) and the solutions attained with STP.

I also talked about and demonstrated issues that can occur if you have
multiple links between bridges (switches), plus how to solve these
problems by using the Spanning Tree Protocol (STP).

I covered a detailed configuration of Cisco's Catalyst switches,
including verifying the configuration, setting the Cisco STP extensions,
and changing the root bridge by setting a bridge priority.

Finally, we discussed, configured, and verified the EtherChannel
technology that helps us bundle multiple links between switches.

\subsection[Exam
Essentials]{\texorpdfstring{\protect\hypertarget{c15.xhtmlux5cux23c15-sec-22}{}{}Exam
Essentials}{Exam Essentials}}

\textbf{Understand the main purpose of the Spanning Tree Protocol in a
switched LAN.} The main purpose of STP is to prevent switching loops in
a network with redundant switched paths.

\textbf{Remember the states of STP.} The purpose of the blocking state
is to prevent the use of looped paths. A port in listening state
prepares to forward data frames without populating the MAC address
table. A port in learning state populates the MAC address table but
doesn't forward data frames. A port in forwarding state sends and
receives all data frames on the bridged port. Also, a port in the
disabled state is virtually nonoperational.

\textbf{Remember the command}\texttt{show\ spanning-tree}. You must be
familiar with the command \texttt{show\ spanning-tree} and how to
determine the root bridge of each VLAN. Also, you can use the
\texttt{show\ spanning-tree\ summary} command to help you get a quick
glimpse of your STP network and root bridges.

\textbf{Understand what PortFast and BPDU Guard provide.} PortFast
allows a port to transition to the forwarding state immediately upon a
connection. Because you don't want other switches connecting to this
port, BPDU Guard will shut down a PortFast port if it receives a BPDU.

\textbf{Understand what EtherChannel is and how to configure it.}
EtherChannel allows you to bundle links to get more bandwidth, instead
of allowing STP to shut down redundant ports. You can configure Cisco's
PAgP or the IEEE version, LACP, by creating a port channel interface and
assigning the port channel group number to the interfaces you are
bundling.

\subsection[Written Lab
15]{\texorpdfstring{\protect\hypertarget{c15.xhtmlux5cux23c15-sec-23}{}{}Written
Lab 15}{Written Lab 15}}

You can find the answers to this lab in Appendix A, ``Answers to Written
Labs.''

Write the answers to the following questions:

\begin{enumerate}
\tightlist
\item
  Which of the following is Cisco proprietary: LACP or PAgP?
\item
  What command will show you the STP root bridge for a VLAN?
\item
  \protect\hypertarget{c15.xhtmlux5cux23Page_636}{}{}What standard is
  RSTP PVST+ based on?
\item
  Which protocol is used in a layer 2 network to maintain a loop-free
  network?
\item
  Which proprietary Cisco STP extension would put a switch port into
  error disabled mode if a BPDU is received on this port?
\item
  You want to configure a switch port to not transition through the STP
  port states but to go immediately to forwarding mode. What command
  will you use on a per-port basis?
\item
  What command will you use to see information about a specific port
  channel interface?
\item
  What command can you use to set a switch so that it will be the root
  bridge for VLAN 3 over any other switch?
\item
  You need to find the VLANs for which your switch is the root bridge.
  What two commands can you use?
\item
  What are the two modes you can set with LACP?
\end{enumerate}

\subsection[Hands-on
Labs]{\texorpdfstring{\protect\hypertarget{c15.xhtmlux5cux23c15-sec-24}{}{}Hands-on
Labs}{Hands-on Labs}}

In this section, you will configure and verify STP, as well as configure
PortFast and BPDU Guard, and finally, bundle links together with
EtherChannel.

Note that the labs in this chapter were written to be used with real
equipment using 2960 switches. However, you can use the free LammleSim
IOS version simulator or Cisco's Packet Tracer to run through these
labs.

The labs in this chapter are as follows:

\begin{enumerate}
\tightlist
\item
  Lab 15.1: Verifying STP and Finding Your Root Bridge
\item
  Lab 15.2: Configuring and Verifying Your Root Bridge
\item
  Lab 15.3: Configuring PortFast and BPDU Guard
\item
  Lab 15.4: Configuring and Verifying EtherChannel
\item
  We'll use the following illustration for all four labs:
\end{enumerate}

\begin{figure}
\centering
\includegraphics{images/c15f022.jpg}
\caption{}
\end{figure}

\subsubsection[Hands-on Lab 15.1: Verifying STP and Finding Your Root
Bridge]{\texorpdfstring{\protect\hypertarget{c15.xhtmlux5cux23c15-sec-25}{}{}\protect\hypertarget{c15.xhtmlux5cux23Page_637}{}{}Hands-on
Lab 15.1: Verifying STP and Finding Your Root
Bridge}{Hands-on Lab 15.1: Verifying STP and Finding Your Root Bridge}}

This lab will assume that you have added VLANs 2 and 3 to each of your
switches and all of your links are trunked.

\begin{enumerate}
\item
  From one of your switches, use the
  \texttt{show\ spanning-tree\ vlan\ 2} command. Verify the output.

\begin{verbatim}
S3#sh spanning-tree vlan 2
VLAN0002
  Spanning tree enabled protocol ieee
  Root ID    Priority    32770
             Address     0001.C9A5.8748
             Cost        19
             Port        1(FastEthernet0/1)
             Hello Time  2 sec  Max Age 20 sec  Forward Delay 15 sec
 
  Bridge ID  Priority    32770  (priority 32768 sys-id-ext 2)
             Address     0004.9A04.ED97
             Hello Time  2 sec  Max Age 20 sec  Forward Delay 15 sec
             Aging Time  20
\end{verbatim}

\begin{verbatim}
Interface        Role Sts Cost      Prio.Nbr Type
---------------- ---- --- --------- -------- --------------------------------
Fa0/1            Root FWD 19        128.1    P2p
Fa0/2            Desg FWD 19        128.2    P2p
Gi1/1            Altn BLK 4         128.25   P2p
Gi1/2            Altn BLK 4         128.26   P2p
\end{verbatim}

  Notice that S3 is not the root bridge, so to find your root bridge,
  just follow the root port and see what bridge is connected to that
  port. Port Fa0/1 is the root port with a cost of 19, which means the
  switch that is off the Fa0/1 port is the root port connecting to the
  root bridge because it is a cost of 19, meaning one Fast Ethernet link
  away.
\item
  Find the bridge that is off of Fa0/1, which will be our root.

\begin{verbatim}
S3#sh cdp neighbors
Capability Codes: R - Router, T - Trans Bridge, B - Source Route Bridge
                  S - Switch, H - Host, I - IGMP, r - Repeater, P - Phone
Device ID    Local Intrfce   Holdtme    Capability   Platform    Port ID
S1           Fas 0/1          158            S       2960        Fas 0/1
S2           Gig 1/1          151            S       2960        Gig 1/1
S2           Gig 1/2          151            S       2960        Gig 1/2
S3#
\end{verbatim}

  Notice that S1 is connected to the local interface Fa0/1, so let's go
  to S1 and verify our root bridge.
\item
  Verify the root bridge for each of the three VLANs. From S1, use the
  \texttt{show\ spanning-tree\ summary} command.

\begin{verbatim}
S1#sh spanning-tree summary
Switch is in pvst mode
Root bridge for: default VLAN0002 VLAN0003
Extended system ID           is enabled
Portfast Default             is disabled
PortFast BPDU Guard Default  is disabled
Portfast BPDU Filter Default is disabled
Loopguard Default            is disabled
EtherChannel misconfig guard is disabled
UplinkFast                   is disabled
BackboneFast                 is disabled
Configured Pathcost method used is short
 
Name                   Blocking Listening Learning Forwarding STP Active
---------------------- -------- --------- -------- ---------- ----------
VLAN0001                     0         0        0          2          2
VLAN0002                     0         0        0          2          2
VLAN0003                     0         0        0          2          2
 
---------------------- -------- --------- -------- ---------- ----------
3 vlans                      0         0        0          6          6
 
S1#
\end{verbatim}

  Notice that S1 is the root bridge for all three VLANs.
\item
  Make note of all your root bridges, for all three VLANs, if you have
  more than one root bridge.
\end{enumerate}

\subsubsection[Hands-on Lab 15.2: Configuring and Verifying Your Root
Bridge]{\texorpdfstring{\protect\hypertarget{c15.xhtmlux5cux23c15-sec-26}{}{}Hands-on
Lab 15.2: Configuring and Verifying Your Root
Bridge}{Hands-on Lab 15.2: Configuring and Verifying Your Root Bridge}}

This lab will assume you have performed Lab 1 and now know who your root
bridge is for each VLAN.

\begin{enumerate}
\item
  \protect\hypertarget{c15.xhtmlux5cux23Page_639}{}{}Go to one of your
  non-root bridges and verify the bridge ID with the
  \texttt{show\ spanning-tree\ vlan} command.

\begin{verbatim}
S3#sh spanning-tree vlan 1
VLAN0001
  Spanning tree enabled protocol ieee
  Root ID    Priority    32769
             Address     0001.C9A5.8748
             Cost        19
             Port        1(FastEthernet0/1)
             Hello Time  2 sec  Max Age 20 sec  Forward Delay 15 sec
 
  Bridge ID  Priority    32769  (priority 32768 sys-id-ext 1)
             Address     0004.9A04.ED97
             Hello Time  2 sec  Max Age 20 sec  Forward Delay 15 sec
             Aging Time  20
 
Interface        Role Sts Cost      Prio.Nbr Type
---------------- ---- --- --------- -------- --------------------------------
Fa0/1            Root FWD 19        128.1    P2p
Fa0/2            Desg FWD 19        128.2    P2p
Gi1/1            Altn BLK 4         128.25   P2p
Gi1/2            Altn BLK 4         128.26   P2p
\end{verbatim}

  Notice that this bridge is not the root bridge for VLAN 1 and the root
  port is Fa0/1 with a cost of 19, which means the root bridge is
  directly connected one Fast Ethernet link away.
\item
  Make one of your non-root bridges the root bridge for VLAN 1. Use
  priority 16,384, which is lower than the 32,768 of the current root.

\begin{verbatim}
S3(config)#spanning-tree vlan 1 priority ?
  <0-61440>  bridge priority in increments of 4096
S3(config)#spanning-tree vlan 1 priority 16384
\end{verbatim}
\item
  Verify the root bridge for VLAN 1.

\begin{verbatim}
S3#sh spanning-tree vlan 1
VLAN0001
  Spanning tree enabled protocol ieee
  Root ID    Priority    16385
             Address     0004.9A04.ED97
             This bridge is the root
             Hello Time  2 sec  Max Age 20 sec  Forward Delay 15 sec
 
  Bridge ID  Priority    16385  (priority 16384 sys-id-ext 1)
             Address     0004.9A04.ED97
             Hello Time  2 sec  Max Age 20 sec  Forward Delay 15 sec
             Aging Time  20
 
Interface        Role Sts Cost      Prio.Nbr Type
---------------- ---- --- --------- -------- --------------------------------
Fa0/1            Desg FWD 19        128.1    P2p
Fa0/2            Desg FWD 19        128.2    P2p
Gi1/1            Desg FWD 4         128.25   P2p
Gi1/2            Desg FWD 4         128.26   P2p
\end{verbatim}
\end{enumerate}

Notice that this bridge is indeed the root and all ports are in Desg FWD
mode.

\subsubsection[Hands-on Lab 15.3: Configuring PortFast and BPDU
Guard]{\texorpdfstring{\protect\hypertarget{c15.xhtmlux5cux23c15-sec-27}{}{}Hands-on
Lab 15.3: Configuring PortFast and BPDU
Guard}{Hands-on Lab 15.3: Configuring PortFast and BPDU Guard}}

This lab will have you configure ports on switches S3 and S2 to allow
the PC and server to automatically go into forward mode when they
connect into the port.

\begin{enumerate}
\item
  Connect to your switch that has a host connected and enable PortFast
  for the interface.

\begin{verbatim}
S3#config t
S3(config)#int fa0/2
S3(config-if)#spanning-tree portfast
%Warning: portfast should only be enabled on ports connected to a single
host. Connecting hubs, concentrators, switches, bridges, etc... to this
interface  when portfast is enabled, can cause temporary bridging loops.
Use with CAUTION
\end{verbatim}

\begin{verbatim}
 
%Portfast has been configured on FastEthernet0/2 but will only
have effect when the interface is in a non-trunking mode.
\end{verbatim}
\item
  Verify that the switch port will be shut down if another switch
  Ethernet cable plugs into this port.

\begin{verbatim}
S3(config-if)#spanning-tree bpduguard enable
\end{verbatim}
\item
  Verify your configuration with the \texttt{show\ running-config}
  command.

\begin{verbatim}
!
interface FastEthernet0/2
 switchport mode trunk
 spanning-tree portfast
 spanning-tree bpduguard enable
!
\end{verbatim}
\end{enumerate}

\subsubsection[Hands-on Lab 15.4: Configuring and Verifying
EtherChannel]{\texorpdfstring{\protect\hypertarget{c15.xhtmlux5cux23c15-sec-28}{}{}\protect\hypertarget{c15.xhtmlux5cux23Page_641}{}{}Hands-on
Lab 15.4: Configuring and Verifying
EtherChannel}{Hands-on Lab 15.4: Configuring and Verifying EtherChannel}}

This lab will have you configure the Cisco EtherChannel PAgP version on
the switches used in this lab. Because I have preconfigured the
switches, I have set up the trunks on all inter-switch ports. We'll use
the Gigabit Ethernet ports between switches S3 and S2.

\begin{enumerate}
\item
  Configure the S3 switch with EtherChannel by creating a port channel
  interface.

\begin{verbatim}
S3#config t
S3(config)#inter port-channel 1
\end{verbatim}
\item
  Configure the ports to be in the bundle with the
  \texttt{channel-group} command.

\begin{verbatim}
S3(config-if)#int range g1/1 - 2
S3(config-if-range)#channel-group 1 mode ?
  active     Enable LACP unconditionally
  auto       Enable PAgP only if a PAgP device is detected
  desirable  Enable PAgP unconditionally
  on         Enable Etherchannel only
  passive    Enable LACP only if a LACP device is detected
S3(config-if-range)#channel-group 1 mode desirable
\end{verbatim}

  I chose the PAgP desirable mode for the S3 switch.
\item
  Configure the S2 switch with EtherChannel, using the same parameters
  as S3.

\begin{verbatim}
S2#config t
S2(config)#interface port-channel 1
S2(config-if)#int rang g1/1 - 2
S2(config-if-range)#channel-group 1 mode desirable
%LINK-5-CHANGED: Interface Port-channel 1, changed state to up
\end{verbatim}

\begin{verbatim}
 %LINEPROTO-5-UPDOWN: Line protocol on Interface Port-channel 1, changed state to up
\end{verbatim}

  Pretty simple, really. Just a couple of commands.
\item
  Verify with the \texttt{show\ etherchannel\ port-channel} command.

\begin{verbatim}
S3#sh etherchannel port-channel
                Channel-group listing:
                ----------------------

Group: 1
----------
                Port-channels in the group:
                ---------------------------
 
Port-channel: Po1
------------
 
Age of the Port-channel   = 00d:00h:06m:43s
Logical slot/port   = 2/1       Number of ports = 2
GC                  = 0x00000000      HotStandBy port = null
Port state          = Port-channel
Protocol            =   PAGP
Port Security       = Disabled
 
Ports in the Port-channel:
 
Index   Load   Port     EC state        No of bits
------+------+------+------------------+-----------
  0     00     Gig1/1   Desirable-Sl       0
  0     00     Gig1/2   Desirable-Sl       0
Time since last port bundled:    00d:00h:01m:30s    Gig1/2
\end{verbatim}
\item
  Verify with the \texttt{show\ etherchannel\ summary} command.

\begin{verbatim}
S3#sh etherchannel summary
Flags:  D - down        P - in port-channel
        I - stand-alone s - suspended
        H - Hot-standby (LACP only)
        R - Layer3      S - Layer2
        U - in use      f - failed to allocate aggregator
        u - unsuitable for bundling
        w - waiting to be aggregated
        d - default port
 
Number of channel-groups in use: 1
Number of aggregators:           1
 
Group  Port-channel  Protocol    Ports
------+-------------+-----------+----------------------------------
 
1      Po1(SU)           PAgP   Gig1/1(P) Gig1/2(P)
S3#
\end{verbatim}
\end{enumerate}

\subsection[Review
Questions]{\texorpdfstring{\protect\hypertarget{c15.xhtmlux5cux23c15-sec-29}{}{}\protect\hypertarget{c15.xhtmlux5cux23Page_643}{}{}Review
Questions}{Review Questions}}

\begin{center}\rule{0.5\linewidth}{0.5pt}\end{center}

\includegraphics{images/note.png}The following questions are designed to
test your understanding of this chapter's material. For more information
on how to get additional questions, please see
\texttt{www.lammle.com/ccna}.

\begin{center}\rule{0.5\linewidth}{0.5pt}\end{center}

You can find the answers to these questions in Appendix B, ``Answers to
Review Questions.''

\begin{enumerate}
\item
  You receive the following output from a switch:

\begin{verbatim}
S2#sh spanning-tree
VLAN0001
  Spanning tree enabled protocol rstp
  Root ID    Priority    32769
             Address     0001.42A7.A603
             Cost        4
             Port        26(GigabitEthernet1/2)
             Hello Time  2 sec  Max Age 20 sec  Forward Delay 15 sec
[output cut]
\end{verbatim}

  Which are true regarding this switch? (Choose two.)

  \begin{enumerate}
  \tightlist
  \item
    The switch is a root bridge.
  \item
    The switch is a non-root bridge.
  \item
    The root bridge is four switches away.
  \item
    The switch is running 802.1w.
  \item
    The switch is running STP PVST+.
  \end{enumerate}
\item
  You have configured your switches with the
  \texttt{spanning-tree\ vlan\ x\ root\ primary} and
  \texttt{spanning-tree\ vlan\ x\ root\ secondary}commands. Which of the
  following tertiary switch will take over if both switches fail?

  \begin{enumerate}
  \tightlist
  \item
    A switch with priority 4096
  \item
    A switch with priority 8192
  \item
    A switch with priority 12288
  \item
    A switch with priority 20480
  \end{enumerate}
\item
  Which of the following would you use to find the VLANs for which your
  switch is the root bridge? (Choose two.)

  \begin{enumerate}
  \tightlist
  \item
    \texttt{show\ spanning-tree}
  \item
    \texttt{show\ root\ all}
  \item
    \texttt{show\ spanning-tree\ port\ root\ VLAN}
  \item
    \texttt{show\ spanning-tree\ summary}
  \end{enumerate}
\item
  \protect\hypertarget{c15.xhtmlux5cux23Page_644}{}{}You want to run the
  new 802.1w on your switches. Which of the following would enable this
  protocol?

  \begin{enumerate}
  \tightlist
  \item
    \texttt{Switch(config)\#spanning-tree\ mode\ rapid-pvst}
  \item
    \texttt{Switch\#spanning-tree\ mode\ rapid-pvst}
  \item
    \texttt{Switch(config)\#spanning-tree\ mode\ 802.1w}
  \item
    \texttt{Switch\#spanning-tree\ mode\ 802.1w}
  \end{enumerate}
\item
  Which of the following is a layer 2 protocol used to maintain a
  loop-free network?

  \begin{enumerate}
  \tightlist
  \item
    VTP
  \item
    STP
  \item
    RIP
  \item
    CDP
  \end{enumerate}
\item
  Which statement describes a spanning-tree network that has converged?

  \begin{enumerate}
  \tightlist
  \item
    All switch and bridge ports are in the forwarding state.
  \item
    All switch and bridge ports are assigned as either root or
    designated ports.
  \item
    All switch and bridge ports are in either the forwarding or blocking
    state.
  \item
    All switch and bridge ports are either blocking or looping.
  \end{enumerate}
\item
  Which of the following modes enable LACP EtherChannel? (Choose two.)

  \begin{enumerate}
  \tightlist
  \item
    On
  \item
    Prevent
  \item
    Passive
  \item
    Auto
  \item
    Active
  \item
    Desirable
  \end{enumerate}
\item
  Which of the following are true regarding RSTP? (Choose three.)

  \begin{enumerate}
  \tightlist
  \item
    RSTP speeds the recalculation of the spanning tree when the layer 2
    network topology changes.
  \item
    RSTP is an IEEE standard that redefines STP port roles, states, and
    BPDUs.
  \item
    RSTP is extremely proactive and very quick, and therefore it
    absolutely needs the 802.1 delay timers.
  \item
    RSTP (802.1w) supersedes 802.1d while remaining proprietary.
  \item
    All of the 802.1d terminology and most parameters have been changed.
  \item
    802.1w is capable of reverting to 802.1d to interoperate with
    traditional switches on a per-port basis.
  \end{enumerate}
\item
  What does BPDU Guard perform?

  \begin{enumerate}
  \tightlist
  \item
    Makes sure the port is receiving BPDUs from the correct upstream
    switch.
  \item
    Makes sure the port is not receiving BPDUs from the upstream switch,
    only the root.
  \item
    \protect\hypertarget{c15.xhtmlux5cux23Page_645}{}{}If a BPDU is
    received on a BPDU Guard port, PortFast is used to shut down the
    port.
  \item
    Shuts down a port if a BPDU is seen on that port.
  \end{enumerate}
\item
  How many bits is the \texttt{sys-id-ext} field in a BPDU?

  \begin{enumerate}
  \tightlist
  \item
    4
  \item
    8
  \item
    12
  \item
    16
  \end{enumerate}
\item
  There are four connections between two switches running RSTP PVST+ and
  you want to figure out how to achieve higher bandwidth without
  sacrificing the resiliency that RSTP provides. What can you configure
  between these two switches to achieve higher bandwidth than the
  default configuration is already providing?

  \begin{enumerate}
  \tightlist
  \item
    Set PortFast and BPDU Guard, which provides faster convergence.
  \item
    Configure unequal cost load balancing with RSTP PVST+.
  \item
    Place all four links into the same EtherChannel bundle.
  \item
    Configure PPP and use multilink.
  \end{enumerate}
\item
  In which circumstance are multiple copies of the same unicast frame
  likely to be transmitted in a switched LAN?

  \begin{enumerate}
  \tightlist
  \item
    During high-traffic periods
  \item
    After broken links are reestablished
  \item
    When upper-layer protocols require high reliability
  \item
    In an improperly implemented redundant topology
  \end{enumerate}
\item
  You want to configure LACP. Which do you need to make sure are
  configured exactly the same on all switch interfaces you are using?
  (Choose three.)

  \begin{enumerate}
  \tightlist
  \item
    Virtual MAC address
  \item
    Port speeds
  \item
    Duplex
  \item
    PortFast enabled
  \item
    VLAN information
  \end{enumerate}
\item
  Which of the following modes enable PAgP EtherChannel? (Choose two.)

  \begin{enumerate}
  \tightlist
  \item
    On
  \item
    Prevent
  \item
    Passive
  \item
    Auto
  \item
    Active
  \item
    Desirable
  \end{enumerate}
\item
  \protect\hypertarget{c15.xhtmlux5cux23Page_646}{}{}For this question,
  refer to the following illustration. SB's RP to the root bridge has
  failed.

  \begin{figure}
  \centering
  \includegraphics{images/c15f023.jpg}
  \caption{}
  \end{figure}

  What is the new cost for SB to make a single path to the root bridge?

  \begin{enumerate}
  \tightlist
  \item
    4
  \item
    8
  \item
    23
  \item
    12
  \end{enumerate}
\item
  Which of the following would put switch interfaces into EtherChannel
  port number 1, using LACP? (Choose two.)

  \begin{enumerate}
  \tightlist
  \item
    \texttt{Switch(config)\#interface\ port-channel\ 1}
  \item
    \texttt{Switch(config)\#channel-group\ 1\ mode\ active}
  \item
    \texttt{Switch\#interface\ port-channel\ 1}
  \item
    \texttt{Switch(config-if)\#channel-group\ 1\ mode\ active}
  \end{enumerate}
\item
  Which two commands would guarantee your switch to be the root bridge
  for VLAN 30? (Choose two.)

  \begin{enumerate}
  \tightlist
  \item
    \texttt{spanning-tree\ vlan\ 30\ priority\ 0}
  \item
    \texttt{spanning-tree\ vlan\ 30\ priority\ 16384}
  \item
    \texttt{spanning-tree\ vlan\ 30\ root\ guarantee}
  \item
    \texttt{spanning-tree\ vlan\ 30\ root\ primary}
  \end{enumerate}
\item
  Why does Cisco use its proprietary extension of PVST+ with STP and
  RSTP?

  \begin{enumerate}
  \tightlist
  \item
    Root bridge placement enables faster convergence as well as optimal
    path determination.
  \item
    Non-root bridge placement clearly enables faster convergence as well
    as optimal path determination.
  \item
    \protect\hypertarget{c15.xhtmlux5cux23Page_647}{}{}PVST+ allows for
    faster discarding of non-IP frames.
  \item
    PVST+ is actually an IEEE standard called 802.1w.
  \end{enumerate}
\item
  Which are states in 802.1d? (Choose all that apply.)

  \begin{enumerate}
  \tightlist
  \item
    Blocking
  \item
    Discarding
  \item
    Listening
  \item
    Learning
  \item
    Forwarding
  \item
    Alternate
  \end{enumerate}
\item
  Which of the following are roles in STP? (Choose all that apply.)

  \begin{enumerate}
  \tightlist
  \item
    Blocking
  \item
    Discarding
  \item
    Root
  \item
    Non-designated
  \item
    Forwarding
  \item
    Designated
  \end{enumerate}
\end{enumerate}

\protect\hypertarget{c16.xhtml}{}{}

\section[{Chapter 16}\\
{Network Device Management and
Security}]{\texorpdfstring{\protect\hypertarget{c16.xhtmlux5cux23c16}{}{}\protect\hypertarget{c16.xhtmlux5cux23Page_649}{}{}{Chapter
16}\\
{Network Device Management and
Security}}{Chapter 16 Network Device Management and Security}}

\begin{center}\rule{0.5\linewidth}{0.5pt}\end{center}

\subsection{THE FOLLOWING ICND2 EXAM TOPICS ARE COVERED IN THIS
CHAPTER:}

\begin{enumerate}
\tightlist
\item
  \includegraphics{images/rarr.png}\textbf{1.7 Describe common access
  layer threat mitigation techniques}

  \begin{enumerate}
  \tightlist
  \item
    \includegraphics{images/squ.png}1.7.a 802.1x
  \item
    \includegraphics{images/squ.png} 1.7.b DHCP snooping
  \end{enumerate}
\item
  \includegraphics{images/rarr.png}\textbf{4.0 Infrastructure Services}
\item
  \includegraphics{images/rarr.png}\textbf{4.1 Configure, verify, and
  troubleshoot basic HSRP}

  \begin{enumerate}
  \tightlist
  \item
    \includegraphics{images/squ.png}4.1.a Priority
  \item
    \includegraphics{images/squ.png} 4.1.b Preemption
  \item
    \includegraphics{images/squ.png} 4.1.c Version
  \end{enumerate}
\item
  \includegraphics{images/rarr.png}\textbf{5.0 Infrastructure
  Maintenance}
\item
  \includegraphics{images/rarr.png}\textbf{5.1 Configure and verify
  device-monitoring protocols}

  \begin{enumerate}
  \tightlist
  \item
    \includegraphics{images/squ.png} 5.1.a SNMPv2
  \item
    \includegraphics{images/squ.png} 5.1.b SNMPv3
  \end{enumerate}
\item
  \includegraphics{images/rarr.png}\textbf{5.4 Describe device
  management using AAA with TACACS+ and RADIUS}
\end{enumerate}

\protect\hypertarget{c16.xhtmlux5cux23Page_650}{}{}\includegraphics{images/intro.png}
We're going to start this chapter by discussing how to mitigate threats
at the access layer using various security techniques. Keeping our
discussion on security, we're then going to turn our attention to
external authentication with authentication, authorization, and
accounting (AAA) of our network devices using RADIUS and TACACS+.

Next, we're going to look at Simple Network Management Protocol (SNMP)
and the type of alerts sent to the network management station (NMS).

Last, I'm going to show you how to integrate redundancy and
load-balancing features into your network elegantly with the routers
that you likely have already. Acquiring some overpriced load-balancing
device just isn't always necessary because knowing how to properly
configure and use Hot Standby Router Protocol (HSRP) can often meet your
needs instead.

\begin{center}\rule{0.5\linewidth}{0.5pt}\end{center}

\includegraphics{images/note.png}\textbf{To find up-to-the-minute
updates for this chapter, please see
\emph{\href{http://www.lammle.com/ccna}{www.lammle.com/ccna}} or the
book's web page at
\emph{\href{http://www.sybex.com/go/ccna}{www.sybex.com/go/ccna}}}.

\begin{center}\rule{0.5\linewidth}{0.5pt}\end{center}

\subsection[Mitigating Threats at the Access
Layer]{\texorpdfstring{\protect\hypertarget{c16.xhtmlux5cux23c16-sec-1}{}{}Mitigating
Threats at the Access Layer}{Mitigating Threats at the Access Layer}}

The Cisco hierarchical model can help you design, implement, and
maintain a scalable, reliable, cost-effective hierarchical internetwork.

The access layer controls user and workgroup access to internetwork
resources and is also sometimes referred to as the desktop layer. The
network resources most users need at this layer will be available
locally because the distribution layer handles any traffic for remote
services.

The following are some of the functions to be included at the access
layer:

\begin{enumerate}
\tightlist
\item
  Continued (from the distribution layer) use of access control and
  policies
\item
  Creation of separate collision domains (microsegmentation/switches)
\item
  Workgroup connectivity into the distribution layer
\item
  Device connectivity
\item
  Resiliency and security services
\item
  Advanced technology capabilities (voice/video, PoE, port-security,
  etc.)
\item
  Interfaces like Gigabit or FastEthernet switching frequently seen in
  the access layer
\end{enumerate}

\protect\hypertarget{c16.xhtmlux5cux23Page_651}{}{}Since the access
layer is both the point at which user devices connect to the network and
the connection point between the network and client device, protecting
it plays an important role in protecting other users, applications, and
the network itself from attacks.

Here are some of the ways to protect the access layer (also shown in
\protect\hyperlink{c16.xhtmlux5cux23figure16-1}{Figure 16.1}):

\begin{figure}
\centering
\includegraphics{images/c16f001.jpg}
\caption{{\protect\hyperlink{c16.xhtmlux5cux23figureanchor16-1}{\textbf{Figure
16.1}} Mitigating threats at the access layer}}
\end{figure}

\textbf{Port security} You're already very familiar with port security
(or you should be!), but restricting a port to a specific set of MAC
addresses is the most common way to defend the access layer.

\textbf{DHCP snooping} DHCP snooping is a layer 2 security feature that
validates DHCP messages by acting like a firewall between untrusted
hosts and trusted DHCP servers.

In order to stop rogue DHCP servers in the network, switch interfaces
are configured as trusted or untrusted, where trusted interfaces allow
all types of DHCP messages and untrusted interfaces allow only requests.
Trusted interfaces are interfaces that connect to a DHCP server or are
an uplink toward the DHCP server, as shown in
\protect\hyperlink{c16.xhtmlux5cux23figure16-2}{Figure 16.2}.

\begin{figure}
\centering
\includegraphics{images/c16f002.jpg}
\caption{{\protect\hyperlink{c16.xhtmlux5cux23figureanchor16-2}{\textbf{Figure
16.2}} DHCP snooping and DAI}}
\end{figure}

\protect\hypertarget{c16.xhtmlux5cux23Page_652}{}{}\textbf{With DHCP}
snooping enabled, a switch also builds a DHCP snooping binding database,
where each entry includes the MAC and IP address of the host as well as
the DHCP lease time, binding type, VLAN, and interface. Dynamic ARP
inspection also uses this DHCP snooping binding database.

\textbf{Dynamic ARP inspection (DAI)} DAI, used with DHCP snooping,
tracks IP-to-MAC bindings from DHCP transactions to protect against ARP
poisoning (which is an attacker trying to have your traffic be sent to
him instead of to your valid destination). DHCP snooping is required in
order to build the MAC-to-IP bindings for DAI validation.

\textbf{Identity-based networking} Identity-based networking is a
concept that ties together several authentication, access control, and
user policy components in order to provide users with the network
services you want them to have.

In the past, for a user to connect to the Finance services, for example,
a user had to be plugged into the Finance LAN or VLAN. However, with
user mobility as one of the core requirements of modern networks, this
is no longer practical, nor does it provide sufficient security.

Identity-based networking allows you to verify users when they connect
to a switch port by authenticating them and placing them in the right
VLAN based on their identity. Should any users fail to pass the
authentication process, their access can be rejected, or they might be
simply put in a guest VLAN.
\protect\hyperlink{c16.xhtmlux5cux23figure16-3}{Figure 16.3} shows this
process.

\begin{figure}
\centering
\includegraphics{images/c16f003.jpg}
\caption{{\protect\hyperlink{c16.xhtmlux5cux23figureanchor16-3}{\textbf{Figure
16.3}} Identity-based networking}}
\end{figure}

The IEEE 802.1x standard allows you to implement identity-based
networking on wired and wireless hosts by using client/server access
control. There are three roles:

\begin{enumerate}
\tightlist
\item
  \textbf{Client} Also referred to as a supplicant, this software runs
  on a client that is 802.1x compliant.
\item
  \textbf{Authenticator} Typically a switch, this controls physical
  access to the network and is a proxy between the client and the
  authentication server.
\item
  \textbf{Authentication server (RADIUS)} This is a server that
  authenticates each client before making available any services.
\end{enumerate}

\subsection[External Authentication
Options]{\texorpdfstring{\protect\hypertarget{c16.xhtmlux5cux23c16-sec-2}{}{}\protect\hypertarget{c16.xhtmlux5cux23Page_653}{}{}External
Authentication Options}{External Authentication Options}}

Of course we only want authorized IT folks to have administrative access
to our network devices such as routers and switches, and in a small to
medium-sized network, just using local authentication is sufficient.

However, if you have hundreds of devices, managing administrative
connectivity would be nearly impossible since you'd have to configure
local authentication on each device by hand, and if you changed just one
password, it can take hours to update your network.

Since maintaining the local database for each network device for the
size of the network is usually not feasible, you can use an external AAA
server that will manage all user and administrative access needs for an
entire network.

The two most popular options for external AAA are RADIUS and TACACS+,
both covered next.

\subsubsection[RADIUS]{\texorpdfstring{\protect\hypertarget{c16.xhtmlux5cux23c16-sec-3}{}{}RADIUS}{RADIUS}}

\emph{Remote Authentication Dial-In User Service, or RADIUS}, was
developed by the Internet Engineering Task Force---the IETF---and is
basically a security system that works to guard the network against
unauthorized access. RADIUS, which uses only UDP, is an open standard
implemented by most major vendors, and it's one of the most popular
types of security servers around because it combines authentication and
authorization services into a single process. So after users are
authenticated, they are then authorized for network services.

RADIUS implements a client/server architecture, where the typical client
is a router, switch, or AP and the typical server is a Windows or Unix
device that's running RADIUS software.

The authentication process has three distinct stages:

\begin{enumerate}
\tightlist
\item
  The user is prompted for a username and password.
\item
  The username and encrypted password are sent over the network to the
  RADIUS server.
\item
  The RADIUS server replies with one of the following:
\end{enumerate}

\begin{longtable}[]{@{}ll@{}}
\toprule
\textbf{Response} & \textbf{Meaning}\tabularnewline
\midrule
\endhead
Accept & The user has been successfully authenticated.\tabularnewline
Reject & The username and password are not valid.\tabularnewline
Challenge & The RADIUS server requests additional
information.\tabularnewline
Change Password & The user should select a new password.\tabularnewline
\bottomrule
\end{longtable}

It's important to remember that RADIUS encrypts only the password in the
access-request packet from the client to the server. The remainder of
the packet is unencrypted.

\paragraph[Configuring
RADIUS]{\texorpdfstring{\protect\hypertarget{c16.xhtmlux5cux23Page_654}{}{}Configuring
RADIUS}{Configuring RADIUS}}

To configure a RADIUS server for console and VTY access, first you need
to enable AAA services in order to configure all the AAA commands.
Configure the \texttt{aaa\ new-model} command in the global
configuration mode.

\begin{verbatim}
Router(config)# aaa new-model
\end{verbatim}

The \texttt{aaa\ new-model} command immediately applies local
authentication to all lines and interfaces (except
\texttt{line\ con\ 0}). So, to avoid being locked out of the router or
switch, you should define a local username and password before starting
the AAA configuration.

Now, configure a local user:

\begin{verbatim}
Router(config)#username Todd password Lammle
\end{verbatim}

Creating this user is super important because you can then use this same
locally created user if the external authentication server fails! If you
don't create this and you can't get to the server, you're going to end
up doing a password recovery.

Next, configure a RADIUS server using any name and the RADIUS key that
is configured on the server.

\begin{verbatim}
Router(config)#radius server SecureLogin
Router(config-radius-server)#address ipv4 10.10.10.254
Router(config-radius-server)#key MyRadiusPassword
\end{verbatim}

Now, add your newly created RADIUS server to an AAA group of any name.

\begin{verbatim}
Router(config)#aaa group server radius MyRadiusGroup
Router(config-sg-radius)#server name SecureLogin
\end{verbatim}

Last, configure this newly created group to be used for AAA login
authentication. If the RADIUS server fails, the fallback to local
authentication should be set.

\begin{verbatim}
Router(config)# aaa authentication login default group MyRadiusGroup local
\end{verbatim}

\subsubsection[TACACS+]{\texorpdfstring{\protect\hypertarget{c16.xhtmlux5cux23c16-sec-5}{}{}TACACS+}{TACACS+}}

\emph{Terminal Access Controller Access Control System (TACACS+)} is
also a security server that's Cisco proprietary and uses TCP. It's
really similar in many ways to RADIUS; however, it does all that RADIUS
does and more, including multiprotocol support.

TACACS+ was developed by Cisco Systems, so it's specifically designed to
interact with Cisco's AAA services. If you're using TACACS+, you have
the entire menu of AAA features available to you---and it handles each
security aspect separately, unlike RADIUS:

\begin{enumerate}
\tightlist
\item
  Authentication includes messaging support in addition to login and
  password functions.
\item
  Authorization enables explicit control over user capabilities.
\item
  Accounting supplies detailed information about user activities.
\end{enumerate}

\paragraph[Configuring
TACACS+]{\texorpdfstring{\protect\hypertarget{c16.xhtmlux5cux23Page_655}{}{}Configuring
TACACS+}{Configuring TACACS+}}

This is pretty much identical to the RADIUS configuration.

To configure a TACACS+ server for console and VTY access, first you need
to enable AAA services in order to configure all the AAA commands.
Configure the \texttt{aaa\ new-model} command in the global
configuration mode (if it isn't already enabled).

\begin{verbatim}
Router(config)# aaa new-model
\end{verbatim}

Now, configure a local user if you haven't already.

\begin{verbatim}
Router(config)#username Todd password Lammle
\end{verbatim}

Next, configure a TACACS+ server using any name and the key that is
configured on the server.

\begin{verbatim}
Router(config)#radius server SecureLoginTACACS+
Router(config-radius-server)#address ipv4 10.10.10.254
Router(config-radius-server)#key MyTACACS+Password
\end{verbatim}

Now, add your newly created TACACS+ server to a AAA group of any name.

\begin{verbatim}
Router(config)#aaa group server radius MyTACACS+Group
Router(config-sg-radius)#server name SecureLoginTACACS+
\end{verbatim}

Last configure this newly created group to be used for AAA login
authentication. If the TACACS+ server fails, the fallback to local
authentication should be set.

\begin{verbatim}
Router(config)# aaa authentication login default group MyTACACS+Group local
\end{verbatim}

\subsubsection[SNMP]{\texorpdfstring{\protect\hypertarget{c16.xhtmlux5cux23c16-sec-7}{}{}SNMP}{SNMP}}

Although \emph{Simple Network Management Protocol (SNMP)} certainly
isn't the oldest protocol ever, it's still pretty old, considering it
was created way back in 1988 (RFC 1065)!

SNMP is an Application layer protocol that provides a message format for
agents on a variety of devices to communicate with network management
stations (NMSs)---for example, Cisco Prime or HP Openview. These agents
send messages to the NMS station, which then either reads or writes
information in the database that's stored on the NMS and called a
management information base (MIB).

The NMS periodically queries or polls the SNMP agent on a device to
gather and analyze statistics via GET messages. End devices running SNMP
agents would send an SNMP trap to the NMS if a problem occurs. This is
demonstrated in \protect\hyperlink{c16.xhtmlux5cux23figure16-4}{Figure
16.4}.

\protect\hypertarget{c16.xhtmlux5cux23Page_656}{}{}

\begin{figure}
\centering
\includegraphics{images/c16f004.jpg}
\caption{{\protect\hyperlink{c16.xhtmlux5cux23figureanchor16-4}{\textbf{Figure
16.4}} SNMP GET and TRAP messages}}
\end{figure}

Admins can also use SNMP to provide some configurations to agents as
well, called SET messages. In addition to polling to obtain statistics,
SNMP can be used for analyzing information and compiling the results in
a report or even a graph. Thresholds can be used to trigger a
notification process when exceeded. Graphing tools are used to monitor
the CPU statistics of Cisco devices like a core router. The CPU should
be monitored continuously and the NMS can graph the statistics.
Notification will be sent when any threshold you've set has been
exceeded.

SNMP has three versions, with version 1 being rarely, if ever,
implemented today. Here's a summary of these three versions:

\textbf{SNMPv1} Supports plaintext authentication with community strings
and uses only UDP.

\textbf{SNMPv2c} Supports plaintext authentication with community
strings with no encryption but provides GET BULK, which is a way to
gather many types of information at once and minimize the number of GET
requests. It offers a more detailed error message reporting method
called INFORM, but it's not more secure than v1. It uses UDP even though
it can be configured to use TCP.

\textbf{SNMPv3} Supports strong authentication with MD5 or SHA,
providing confidentiality (encryption) and data integrity of messages
via DES or DES-256 encryption between agents and managers. GET BULK is a
supported feature of SNMPv3, and this version also uses TCP.

\subsubsection[Management Information Base
(MIB)]{\texorpdfstring{\protect\hypertarget{c16.xhtmlux5cux23c16-sec-8}{}{}Management
Information Base (MIB)}{Management Information Base (MIB)}}

With so many kinds of devices and so much data that can be accessed,
there needed to be a standard way to organize this plethora of data, so
MIB to the rescue! A \emph{management information base (MIB)} is a
collection of information that's organized hierarchically and can be
accessed by protocols like SNMP. RFCs define some common public
variables, but most organizations define their own private branches
along with basic SNMP standards. Organizational IDs (OIDs) are laid out
as a tree with different levels assigned by different organizations,
with top-level MIB OIDs belonging to various standards organizations.

\protect\hypertarget{c16.xhtmlux5cux23Page_657}{}{}Vendors assign
private branches in their own products. Let's take a look at Cisco's
OIDs, which are described in words or numbers to locate a particular
variable in the tree, as shown in
\protect\hyperlink{c16.xhtmlux5cux23figure16-5}{Figure 16.5}.

\begin{figure}
\centering
\includegraphics{images/c16f005.jpg}
\caption{{\protect\hyperlink{c16.xhtmlux5cux23figureanchor16-5}{\textbf{Figure
16.5}} Cisco's MIB OIDs}}
\end{figure}

Luckily, you don't need to memorize the OIDs in
\protect\hyperlink{c16.xhtmlux5cux23figure16-5}{Figure 16.5} for the
Cisco exams!

To obtain information from the MIB on the SNMP agent, you can use
several different operations:

\begin{enumerate}
\tightlist
\item
  GET: This operation is used to get information from the MIB to an SNMP
  agent.
\item
  SET: This operation is used to get information to the MIB from an SNMP
  manager.
\item
  WALK: This operation is used to list information from successive MIB
  objects within a specified MIB.
\item
  TRAP: This operation is used by the SNMP agent to send a triggered
  piece of information to the SNMP manager.
\item
  INFORM: This operation is the same as a trap, but it adds an
  acknowledgment that a trap does not provide.
\end{enumerate}

\subsubsection[Configuring
SNMP]{\texorpdfstring{\protect\hypertarget{c16.xhtmlux5cux23c16-sec-9}{}{}Configuring
SNMP}{Configuring SNMP}}

Configuring SNMP is a pretty straightforward process for which you only
need a few commands. These five steps are all you need to run through to
configure a Cisco device for SNMP access:

\begin{enumerate}
\tightlist
\item
  Configure where the traps are to be sent.
\item
  Enable SNMP read-write access to the router.
\item
  \protect\hypertarget{c16.xhtmlux5cux23Page_658}{}{}Configure SNMP
  contact information.
\item
  Configure SNMP location.
\item
  Configure an ACL to restrict SNMP access to the NMS hosts.
\end{enumerate}

The only required configuration is the IP address of the NMS station and
the community string (which acts as a password or authentication string)
because the other three are optional. Here's an example of a typical
SNMP router configuration:

\begin{verbatim}
Router(config)#snmp-server host 1.2.3.4
Router(config)#snmp-server community ?
 WORD SNMP community string
 
Router(config)#snmp-server community Todd ?
 <1-99>		Std IP accesslist allowing access with this community string
 <1300-1999>	Expanded IP accesslist allowing access with this community  string
 WORD		Access-list name
 ipv6		Specify IPv6 Named Access-List
 ro		Read-only access with this community string
 rw		Read-write access with this community string
 view		Restrict this community to a named MIB view
 <cr>
 
Router(config)#snmp-server community Todd rw
Router(config)#snmp-server location Boulder
Router(config)#snmp-server contact Todd Lammle
Router(config)#ip access-list standard Protect_NMS_Station
Router(config-std-nacl)#permit host 192.168.10.254
\end{verbatim}

Entering the \texttt{snmp-server} command enables SNMPv1 on the Cisco
device.

You can enter the ACL directly in the SNMP configuration to provide
security, using either a number or a name. Here is an example:

\begin{verbatim}
Router(config)#snmp-server community Todd Protect_NMS_Station rw
\end{verbatim}

Notice that even though there's a boatload of configuration options
under SNMP, you only really need to work with a few of them to configure
a basic SNMP trap setup on a router. First, I set the IP address of the
NMS station where the router will send the traps; then I chose the
community name of \texttt{Todd} with RW access (read-write), which means
the NMS will be able to retrieve and modify MIB objects from the router.
Location and contact information comes in really handy for
troubleshooting the configuration. Make sure you understand that the ACL
protects the NMS from access, not the devices with the agents!

\protect\hypertarget{c16.xhtmlux5cux23Page_659}{}{}Let's define the SNMP
read and write options.

\textbf{Read-only} Gives authorized management stations read access to
all objects in the MIB except the community strings and doesn't allow
write access

\textbf{Read-write} Gives authorized management stations read and write
access to all objects in the MIB but doesn't allow access to the
community strings

Next we'll explore a Cisco proprietary method of configuring redundant
default gateways for hosts.

\subsection[Client Redundancy
Issues]{\texorpdfstring{\protect\hypertarget{c16.xhtmlux5cux23c16-sec-10}{}{}Client
Redundancy Issues}{Client Redundancy Issues}}

If you're wondering how you can possibly configure a client to send data
off its local link when its default gateway router has gone down, you've
targeted a key issue because the answer is that, usually, you can't!
Most host operating systems just don't allow you to change data routing.
Sure, if a host's default gateway router goes down, the rest of the
network will still converge, but it won't share that information with
the hosts. Take a look at
\protect\hyperlink{c16.xhtmlux5cux23figure16-6}{Figure 16.6} to see what
I am talking about. There are actually two routers available to forward
data for the local subnet, but the hosts know about only one of them.
They learn about this router when you provide them with the default
gateway either statically or through DHCP.

\begin{figure}
\centering
\includegraphics{images/c16f006.jpg}
\caption{{\protect\hyperlink{c16.xhtmlux5cux23figureanchor16-6}{\textbf{Figure
16.6}} Default gateway}}
\end{figure}

\protect\hypertarget{c16.xhtmlux5cux23Page_660}{}{}This begs the
question: Is there another way to use the second active router? The
answer is a bit complicated, but bear with me. There is a feature that's
enabled by default on Cisco routers called Proxy Address Resolution
Protocol (Proxy ARP). Proxy ARP enables hosts, which have no knowledge
of routing options, to obtain the MAC address of a gateway router that
can forward packets for them.

You can see how this happens in
\protect\hyperlink{c16.xhtmlux5cux23figure16-7}{Figure 16.7}. If a Proxy
ARP--enabled router receives an ARP request for an IP address that it
knows isn't on the same subnet as the requesting host, it will respond
with an ARP reply packet to the host. The router will give its own local
MAC address---the MAC address of its interface on the host's subnet---as
the destination MAC address for the IP address that the host is seeking
to be resolved. After receiving the destination MAC address, the host
will then send all the packets to the router, not knowing that what it
sees as the destination host is really a router. The router will then
forward the packets toward the intended host.

\begin{figure}
\centering
\includegraphics{images/c16f007.jpg}
\caption{{\protect\hyperlink{c16.xhtmlux5cux23figureanchor16-7}{\textbf{Figure
16.7}} Proxy ARP}}
\end{figure}

So with Proxy ARP, the host device sends traffic as if the destination
device were located on its own network segment. If the router that
responded to the ARP request fails, the source host continues to send
packets for that destination to the same MAC address. But because
they're being sent to a failed router, the packets will be sent to the
other router on the network that is also responding to ARP requests for
remote hosts.

After the time-out period on the host, the proxy ARP MAC address ages
out of the ARP cache. The host can then make a new ARP request for the
destination and get the address of
\protect\hypertarget{c16.xhtmlux5cux23Page_661}{}{}another proxy ARP
router. Still, keep in mind that the host cannot send packets off of its
subnet during the failover time. This isn't exactly a perfect situation,
so there has to be a better way, right? Well, there is, and that's
precisely where redundancy protocols come to the rescue!

\subsection[Introducing First Hop Redundancy Protocols
(FHRPs)]{\texorpdfstring{\protect\hypertarget{c16.xhtmlux5cux23c16-sec-11}{}{}Introducing
First Hop Redundancy Protocols
(FHRPs)}{Introducing First Hop Redundancy Protocols (FHRPs)}}

\emph{First hop redundancy protocols (FHRPs)} work by giving you a way
to configure more than one physical router to appear as if they were
only a single logical one. This makes client configuration and
communication easier because you can simply configure a single default
gateway and the host machine can use its standard protocols to
communicate. \emph{First hop} is a reference to the default router being
the first router, or first router hop, through which a packet must pass.

So how does a redundancy protocol accomplish this? The protocols I'm
going to describe to you do this basically by presenting a virtual
router to all of the clients. The virtual router has its own IP and MAC
addresses. The virtual IP address is the address that's configured on
each of the host machines as the default gateway. The virtual MAC
address is the address that will be returned when an ARP request is sent
by a host. The hosts don't know or care which physical router is
actually forwarding the traffic, as you can see in
\protect\hyperlink{c16.xhtmlux5cux23figure16-8}{Figure 16.8}.

\begin{figure}
\centering
\includegraphics{images/c16f008.jpg}
\caption{{\protect\hyperlink{c16.xhtmlux5cux23figureanchor16-8}{\textbf{Figure
16.8}} FHRPs use a virtual router with a virtual IP address and virtual
MAC address.}}
\end{figure}

\protect\hypertarget{c16.xhtmlux5cux23Page_662}{}{}It's the
responsibility of the redundancy protocol to decide which physical
router will actively forward traffic and which one will be placed on
standby in case the active router fails. Even if the active router
fails, the transition to the standby router will be transparent to the
hosts because the virtual router, which is identified by the virtual IP
and MAC addresses, is now used by the standby router. The hosts never
change default gateway information, so traffic keeps flowing.

\begin{center}\rule{0.5\linewidth}{0.5pt}\end{center}

\includegraphics{images/note.png}\textbf{Fault-tolerant solutions
provide continued operation in the event of a device failure, and
load-balancing solutions distribute the workload over multiple devices.}

\begin{center}\rule{0.5\linewidth}{0.5pt}\end{center}

There are three important redundancy protocols, but only HSRP is covered
on the CCNA objectives now:

\textbf{Hot Standby Router Protocol (HSRP)} HSRP is by far Cisco's
favorite protocol ever! Don't buy just one router; buy up to eight
routers to provide the same service, and keep seven as backup in case of
failure! HSRP is a Cisco proprietary protocol that provides a redundant
gateway for hosts on a local subnet, but this isn't a load-balanced
solution. HSRP allows you to configure two or more routers into a
standby group that shares an IP address and MAC address and provides a
default gateway. When the IP and MAC addresses are independent from the
routers' physical addresses (on a virtual interface, not tied to a
specific interface), HSRP can swap control of an address if the current
forwarding and active router fails. But there is actually a way you can
sort of achieve load balancing with HSRP---by using multiple VLANs and
designating a specific router active for one VLAN, then an alternate
router as active for the other VLAN via trunking. This still isn't a
true load-balancing solution and it's not nearly as solid as what you
can achieve with GLBP!

\textbf{Virtual Router Redundancy Protocol (VRRP)} Also provides a
redundant---but again, not load-balanced---gateway for hosts on a local
subnet. It's an open standard protocol that functions almost identically
to HSRP.

\textbf{Gateway Load Balancing Protocol (GLBP)} For the life of me I
can't figure out how GLBP isn't a CCNA objective anymore! GLBP doesn't
just stop at providing us with a redundant gateway; it's a true
load-balancing solution for routers. GLBP allows a maximum of four
routers in each forwarding group. By default, the active router directs
the traffic from hosts to each successive router in the group using a
round-robin algorithm. The hosts are directed to send their traffic
toward a specific router by being given the MAC address of the next
router in line to be used.

\subsection[Hot Standby Router Protocol
(HSRP)]{\texorpdfstring{\protect\hypertarget{c16.xhtmlux5cux23c16-sec-12}{}{}Hot
Standby Router Protocol (HSRP)}{Hot Standby Router Protocol (HSRP)}}

Again, HSRP is a Cisco proprietary protocol that can be run on most, but
not all, of Cisco's router and multilayer switch models. It defines a
standby group, and each standby group that you define includes the
following routers:

\begin{enumerate}
\tightlist
\item
  \protect\hypertarget{c16.xhtmlux5cux23Page_663}{}{}Active router
\item
  Standby router
\item
  Virtual router
\item
  Any other routers that maybe attached to the subnet
\end{enumerate}

The problem with HSRP is that with it, only one router is active and two
or more routers just sit there in standby mode and won't be used unless
a failure occurs---not very cost effective or efficient!
\protect\hyperlink{c16.xhtmlux5cux23figure16-9}{Figure 16.9} shows how
only one router is used at a time in an HSRP group.

The standby group will always have at least two routers participating in
it. The primary players in the group are the one active router and one
standby router that communicate to each other using multicast Hello
messages. The Hello messages provide all of the required communication
for the routers. The Hellos contain the information required to
accomplish the election that determines the active and standby router
positions. They also hold the key to the failover process. If the
standby router stops receiving Hello packets from the active router, it
then takes over the active router role, as shown in
\protect\hyperlink{c16.xhtmlux5cux23figure16-9}{Figure 16.9} and
\protect\hyperlink{c16.xhtmlux5cux23figure16-10}{Figure 16.10}.

\begin{figure}
\centering
\includegraphics{images/c16f009.jpg}
\caption{{\protect\hyperlink{c16.xhtmlux5cux23figureanchor16-9}{\textbf{Figure
16.9}} HSRP active and standby routers}}
\end{figure}

\protect\hypertarget{c16.xhtmlux5cux23Page_664}{}{}

\begin{figure}
\centering
\includegraphics{images/c16f010.jpg}
\caption{{\protect\hyperlink{c16.xhtmlux5cux23figureanchor16-10}{\textbf{Figure
16.10}} Example of HSRP active and standby routers swapping interfaces}}
\end{figure}

As soon as the active router stops responding to Hellos, the standby
router automatically becomes the active router and starts responding to
host requests.

\subsubsection[Virtual MAC
Address]{\texorpdfstring{\protect\hypertarget{c16.xhtmlux5cux23c16-sec-13}{}{}Virtual
MAC Address}{Virtual MAC Address}}

A virtual router in an HSRP group has a virtual IP address and a virtual
MAC address. So where does that virtual MAC come from? The virtual IP
address isn't that hard to figure out; it just has to be a unique IP
address on the same subnet as the hosts defined in the configuration.
But MAC addresses are a little different, right? Or are they? The answer
is yes---sort of. With HSRP, you create a totally new, made-up MAC
address in addition to the IP address.

The HSRP MAC address has only one variable piece in it. The first 24
bits still identify the vendor who manufactured the device (the
organizationally unique identifier, or OUI).
\protect\hypertarget{c16.xhtmlux5cux23Page_665}{}{}The next 16 bits in
the address tell us that the MAC address is a well-known HSRP MAC
address. Finally, the last 8 bits of the address are the hexadecimal
representation of the HSRP group number.

Let me clarify all this with an example of what an HSRP MAC address
would look like:

\begin{verbatim}
0000.0c07.ac0a
\end{verbatim}

\begin{enumerate}
\tightlist
\item
  The first 24 bits (0000.0c) are the vendor ID of the address; in the
  case of HSRP being a Cisco protocol, the ID is assigned to Cisco.
\item
  The next 16 bits (07.ac) are the well-known HSRP ID. This part of the
  address was assigned by Cisco in the protocol, so it's always easy to
  recognize that this address is for use with HSRP.
\item
  The last 8 bits (0a) are the only variable bits and represent the HSRP
  group number that you assign. In this case, the group number is 10 and
  converted to hexadecimal when placed in the MAC address, where it
  becomes the 0a that you see.
\end{enumerate}

You can see this displayed with every MAC address added to the ARP cache
of every router in the HSRP group. There will be the translation from
the IP address to the MAC address, as well as the interface on which
it's located.

\subsubsection[HSRP
Timers]{\texorpdfstring{\protect\hypertarget{c16.xhtmlux5cux23c16-sec-14}{}{}HSRP
Timers}{HSRP Timers}}

Before we get deeper into the roles that each of the routers can have in
an HSRP group, I want to define the HSRP timers for HSRP to function
because they ensure communication between the routers, and if something
goes wrong, they allow the standby router to take over. The HSRP timers
include \emph{hello}, \emph{hold}, \emph{active}, and \emph{standby}.

\textbf{Hello timer} The hello timer is the defined interval during
which each of the routers send out Hello messages. Their default
interval is 3 seconds and they identify the state that each router is
in. This is important because the particular state determines the
specific role of each router and, as a result, the actions each will
take within the group.
\protect\hyperlink{c16.xhtmlux5cux23figure16-11}{Figure 16.11} shows the
Hello messages being sent and the router using the hello timer to keep
the network flowing in case of a failure.

This timer can be changed, and people used to avoid doing so because it
was thought that lowering the hello value would place an unnecessary
load on the routers. That isn't true with most of the routers today; in
fact, you can configure the timers in milliseconds, meaning the failover
time can be in milliseconds! Still, keep in mind that increasing the
value will make the standby router wait longer before taking over for
the active router when it fails or can't communicate.

\protect\hypertarget{c16.xhtmlux5cux23Page_666}{}{}

\begin{figure}
\centering
\includegraphics{images/c16f011.jpg}
\caption{{\protect\hyperlink{c16.xhtmlux5cux23figureanchor16-11}{\textbf{Figure
16.11}} HSRP Hellos}}
\end{figure}

\textbf{Hold timer} The hold timer specifies the interval the standby
router uses to determine whether the active router is offline or out of
communication. By default, the hold timer is 10 seconds, roughly three
times the default for the hello timer. If one timer is changed for some
reason, I recommend using this multiplier to adjust the other timers
too. By setting the hold timer at three times the hello timer, you
ensure that the standby router doesn't take over the active role every
time there's a short break in communication.

\textbf{Active timer} The active timer monitors the state of the active
router. The timer resets each time a router in the standby group
receives a Hello packet from the active router. This timer expires based
on the hold time value that's set in the corresponding field of the HSRP
Hello message.

\textbf{Standby timer} The standby timer is used to monitor the state of
the standby router. The timer resets anytime a router in the standby
group receives a Hello packet from the standby router and expires based
on the hold time value that's set in the respective Hello packet.

\protect\hypertarget{c16.xhtmlux5cux23Page_667}{}{}

\begin{center}\rule{0.5\linewidth}{0.5pt}\end{center}

\includegraphics{images/c16inline02.png}

\subsection{Large Enterprise Network Outages with FHRPs}

Years ago when HSRP was all the rage, and before VRRP and GLBP,
enterprises used hundreds of HSRP groups. With the hello timer set to 3
seconds and a hold time of 10 seconds, these timers worked just fine and
we had great redundancy with our core routers.

However, as we've seen in the last few years and certainly will see in
the future, 10 seconds is now a lifetime! Some of my customers have been
complaining with the failover time and loss of connectivity to their
virtual server farm.

So lately I've been changing the timers to well below the defaults.
Cisco had changed the timers so you could use sub-second times for
failover. Because these are multicast packets, the overhead that is seen
on a current high-speed network is almost nothing.

The hello timer is typically set to 200 msec and the hold time is 700
msec. The command is as follows:

\begin{verbatim}
(config-if)#Standby 1 timers msec 200 msec 700
\end{verbatim}

This almost ensures that not even a single packet is lost when there is
an outage.

\begin{center}\rule{0.5\linewidth}{0.5pt}\end{center}

\subsubsection[Group
Roles]{\texorpdfstring{\protect\hypertarget{c16.xhtmlux5cux23c16-sec-15}{}{}Group
Roles}{Group Roles}}

Each of the routers in the standby group has a specific function and
role to fulfill. The three main roles are as virtual router, active
router, and standby router. Additional routers can also be included in
the group.

\textbf{Virtual router} As its name implies, the virtual router is not a
physical entity. It really just defines the role that's held by one of
the physical routers. The physical router that communicates as the
virtual router is the current active router. The virtual router is
nothing more than a separate IP address and MAC address to which packets
are sent.

\textbf{Active router} The active router is the physical router that
receives data sent to the virtual router address and routes it onward to
its various destinations. As I mentioned, this router accepts all the
data sent to the MAC address of the virtual router in addition to the
data that's been sent to its own physical MAC address. The active router
processes the data that's being forwarded and will also answer any ARP
requests destined for the virtual router's IP address.

\textbf{Standby router} The standby router is the backup to the active
router. Its job is to monitor the status of the HSRP group and quickly
take over packet-forwarding responsibilities if the active router fails
or loses communication. Both the active and standby routers transmit
Hello messages to inform all other routers in the group of their role
and status.

\protect\hypertarget{c16.xhtmlux5cux23Page_668}{}{}\textbf{Other
routers} An HSRP group can include additional routers, which are members
of the group but don't take the primary roles of either active or
standby states. These routers monitor the Hello messages sent by the
active and standby routers to ensure that an active and standby router
exists for the HSRP group that they belong to. They will forward data
that's specifically addressed to their own IP addresses, but they will
never forward data addressed to the virtual router unless elected to the
active or standby state. These routers send ``speak'' messages based on
the hello timer interval that informs other routers of their position in
an election.

\paragraph{Interface Tracking}

By now, you probably understand why having a virtual router on a LAN is
a great idea. You also know why it's a very good thing that the active
router can change dynamically, giving us much needed redundancy on our
inside network. But what about the links to the upstream network or the
Internet connection off of those HSRP-enabled routers? And how will the
inside hosts know if an outside interface goes down or if they are
sending packets to an active router that can't route to a remote
network? Key questions and HSRP do provide a solution for them; it's
called interface tracking.

\protect\hyperlink{c16.xhtmlux5cux23figure16-12}{Figure 16.12} shows how
HSRP-enabled routers can keep track of the interface status of the
outside interfaces and how they can switch the inside active router as
needed to keep the inside hosts from losing connectivity upstream.

\begin{figure}
\centering
\includegraphics{images/c16f012.jpg}
\caption{{\protect\hyperlink{c16.xhtmlux5cux23figureanchor16-12}{\textbf{Figure
16.12}} Interface tracking setup}}
\end{figure}

\protect\hypertarget{c16.xhtmlux5cux23Page_669}{}{}If the outside link
of the active router goes down, the standby router will take over and
become the active router. There is a default priority of 100 on routers
configured with an HSRP interface, and if you raise this priority (we'll
do this in a minute), it means your router has a higher priority to
become the active router. The reason I am bringing this up now is
because when a tracked interface goes down, it decrements the priority
of this router.

\subsubsection[Configuring and Verifying
HSRP]{\texorpdfstring{\protect\hypertarget{c16.xhtmlux5cux23c16-sec-17}{}{}Configuring
and Verifying HSRP}{Configuring and Verifying HSRP}}

Configuring and verifying the different FHRPs can be pretty simple,
especially regarding the Cisco objectives, but as with most
technologies, you can quickly get into advanced configurations and
territory with the different FHRPs if you're not careful, so I'll show
you exactly what you need to know.

The Cisco objectives don't cover much about the configuration of FHRPs,
but verification and troubleshooting is important, so I'll use a simple
configuration on two routers here.
\protect\hyperlink{c16.xhtmlux5cux23figure16-13}{Figure 16.13} shows the
network I'll use to demonstrate HSRP.

\begin{figure}
\centering
\includegraphics{images/c16f013.jpg}
\caption{{\protect\hyperlink{c16.xhtmlux5cux23figureanchor16-13}{\textbf{Figure
16.13}} HSRP configuration and verification}}
\end{figure}

This is a simple configuration for which you really need only one
command: \texttt{standby\ group\ ip\ virtual\_ip.} After using this
single mandatory command, I'll name the group and
\protect\hypertarget{c16.xhtmlux5cux23Page_670}{}{}set the interface on
router HSRP1 so it wins the election and becomes the active router by
default.

\begin{verbatim}
HSRP1#config t
HSRP1(config)#int fa0/0
HSRP1(config-if)#standby ?
 <0-255>	      group number
 authentication       Authentication
 delay                HSRP initialisation delay
 ip                   Enable HSRP and set the virtual IP address
 mac-address          Virtual MAC address
 name                 Redundancy name string
 preempt              Overthrow lower priority Active routers
 priority             Priority level
 redirect             Configure sending of ICMP Redirect messages with an HSRP
                      virtual IP address as the gateway IP address
 timers               Hello and hold timers
 track                Priority tracking
 use-bia              HSRP uses interface's burned in address
 version              HSRP version
 
HSRP1(config-if)#standby 1 ip 10.1.1.10
HSRP1(config-if)#standby 1 name HSRP_Test
HSRP1(config-if)#standby 1 priority ?
 <0-255> Priority value
 
HSRP1(config-if)#standby 1 priority 110
000047: %HSRP-5-STATECHANGE: FastEthernet0/0 Grp 1 state Speak -> Standby
000048: %HSRP-5-STATECHANGE: FastEthernet0/0 Grp 1 state Standby -> Active110
\end{verbatim}

There are quite a few commands available to use in an advanced setting
with the \texttt{standby} command, but we'll stick with the simple
commands that follow the Cisco objectives. First, I numbered the group
(1), which must be the same on all routers sharing HSRP duties; then I
added the virtual IP address shared by all routers in the HSRP group.
Optionally, I named the group and then set the priority of HSRP1 to 110,
and I left HSRP2 to a default of 100. The router with the highest
priority will win the election to become the active router. Let's
configure the HSRP2 router now:

\begin{verbatim}
HSRP2#config t
HSRP2(config)#int fa0/0
HSRP2(config-if)#standby 1 ip 10.1.1.10
HSRP2(config-if)#standby 1 name HSRP_Test
*Jun 23 21:40:10.699:%HSRP-5-STATECHANGE:FastEthernet0/0 Grp 1 state
Speak -> Standby
\end{verbatim}

I really only needed the first command---naming it was for
administrative purposes only. Notice that the link came up and HSRP2
became the standby router because it had the lower priority of 100 (the
default). Make a note that this priority comes into play only if both
routers were to come up at the same time. This means that HSRP2 would be
the active router, regardless of the priority, if it comes up first.

Let's take a look at the configurations with the \texttt{show\ standby}
and \texttt{show\ standby\ brief} commands:

\begin{verbatim}
HSRP1(config-if)#do show standby
FastEthernet0/0 - Group 1
  State is Active
    2 state changes, last state change 00:03:40
  Virtual IP address is 10.1.1.10
  Active virtual MAC address is 0000.0c07.ac01
    Local virtual MAC address is 0000.0c07.ac01 (v1 default)
  Hello time 3 sec, hold time 10 sec
    Next hello sent in 1.076 secs
  Preemption disabled
  Active router is local
  Standby router is 10.1.1.2, priority 100 (expires in 7.448 sec)
  Priority 110 (configured 110)
  IP redundancy name is "HSRP_Test" (cfgd)
 
HSRP1(config-if)#do show standby brief
                     P indicates configured to preempt.
                     |
Interface   Grp Prio P State    Active          Standby         Virtual IP
Fa0/0       1   110    Active   local           10.1.1.2        10.1.1.10
\end{verbatim}

Notice the group number in each output---it's a key troubleshooting
spot! Each router must be configured in the same group or they won't
work. Also, you can see the virtual MAC and configured virtual IP
address, as well as the hello time of 3 seconds. The standby and virtual
IP addresses are also displayed.

HSRP2's output tells us that it's in standby mode:

\begin{verbatim}
HSRP2(config-if)#do show standby brief
                       P indicates configured to preempt.
                       |
Interface      Grp Prio	P	State	Active		Standby		Virtual IP
Fa0/0           1 100			Standby	10.1.1.1		local		10.1.1.10
HRSP2(config-if)#
\end{verbatim}

\protect\hypertarget{c16.xhtmlux5cux23Page_672}{}{}Notice so far that
you have seen HSRP states of active and standby, but watch what happens
when I disable Fa0/0:

\begin{verbatim}
HSRP1#config t
HSRP1(config)#interface Fa0/0
HSRP1(config-if)#shutdown
*Nov 20 10:06:52.369: %HSRP-5-STATECHANGE: Ethernet0/0 Grp 1 state Active -> Init
\end{verbatim}

The HSRP went into \texttt{Init} state, meaning it's trying to
initialize with a peer. The possible interface states for HSRP are shown
in \protect\hyperlink{c16.xhtmlux5cux23table16-1}{Table 16.1}.

{\protect\hyperlink{c16.xhtmlux5cux23tableanchor16-1}{\textbf{Table~16.1}}
HSRP states}

\begin{longtable}[]{@{}ll@{}}
\toprule
State & Definition\tabularnewline
\midrule
\endhead
Initial (INIT) & This is~the state at the start. This state indicates
that HSRP does not run.~This state is entered through a configuration
change or when an~interface first becomes available.\tabularnewline
Learn & The~router has not determined the virtual IP address and has not
yet seen an~authenticated Hello message from the active router. In this
state, the~router still waits to hear from the active
router.\tabularnewline
Listen & The~router knows the virtual IP address, but the router is
neither the~active router nor the standby router. It listens for Hello
messages from~those routers.\tabularnewline
Speak & The~router sends periodic Hello messages and actively
participates in the~election of the active and/or standby router. A
router cannot enter speak state unless the router has the virtual IP
address.\tabularnewline
Standby & The~router is a candidate to become the next active router and
sends~periodic Hello messages. With the exclusion of transient
conditions,~there is, at most, one router in the group in standby
state.\tabularnewline
Active & The~router currently forwards packets that are sent to the
group virtual MAC~address. The router sends periodic Hello messages.
With the exclusion~of transient conditions, there must be, at most, one
router in active state in the group.\tabularnewline
\bottomrule
\end{longtable}

There is one other command that I want to cover. If you're studying and
want to understand HSRP, you should learn to use this \texttt{debug}
command and have your active and standby routers move. You'll really get
to see what is going on.

\begin{verbatim}
HSRP2#debug standby
*Sep 15 00:07:32.344:HSRP:Fa0/0 Interface UP
*Sep 15 00:07:32.344:HSRP:Fa0/0 Initialize swsb, Intf state Up
*Sep 15 00:07:32.344:HSRP:Fa0/0 Starting minimum intf delay (1 secs)
*Sep 15 00:07:32.344:HSRP:Fa0/0 Grp 1 Set virtual MAC 0000.0c07.ac01
type: v1 default
*Sep 15 00:07:32.344:HSRP:Fa0/0 MAC hash entry 0000.0c07.ac01, Added
Fa0/0 Grp 1 to list
*Sep 15 00:07:32.348:HSRP:Fa0/0 Added 10.1.1.10 to hash table
*Sep 15 00:07:32.348:HSRP:Fa0/0 Grp 1 Has mac changed? cur 0000.0c07.ac01
new 0000.0c07.ac01
*Sep 15 00:07:32.348:HSRP:Fa0/0 Grp 1 Disabled -> Init
*Sep 15 00:07:32.348:HSRP:Fa0/0 Grp 1 Redundancy "hsrp-Fa0/0-1" state
Disabled -> Init
*Sep 15 00:07:32.348:HSRP:Fa0/0 IP Redundancy "hsrp-Fa0/0-1" added
*Sep 15 00:07:32.348:HSRP:Fa0/0 IP Redundancy "hsrp-Fa0/0-1" update,
Disabled -> Init
*Sep 15 00:07:33.352:HSRP:Fa0/0 Intf min delay expired
*Sep 15 00:07:39.936:HSRP:Fa0/0 Grp 1 MAC addr update Delete from SMF 0000.0c07.ac01
*Sep 15 00:07:39.936:HSRP:Fa0/0 Grp 1 MAC addr update Delete from SMF 0000.0c07.ac01
*Sep 15 00:07:39.940:HSRP:Fa0/0 ARP reload
\end{verbatim}

\paragraph{HSRP Load Balancing}

As you know, HSRP doesn't really perform true load balancing, but it can
be configured to use more than one router at a time for use with
different VLANs. This is different from the true load balancing that's
possible with GLBP, which I'll demonstrate in a minute, but HSRP still
performs a load-balancing act of sorts.
\protect\hyperlink{c16.xhtmlux5cux23figure16-14}{Figure 16.14} shows how
load balancing would look with HSRP.

How can you get two HSRP routers active at the same time? Well for the
same subnet with this simple configuration, you can't, but if you trunk
the links to each router, they'll run and be configured with a ``router
on a stick'' (ROAS) configuration. This means that each router can be
the default gateway for different VLANs, but you still can have only one
active router per VLAN. Typically, in a more advanced setting you won't
use HSRP for load balancing; you'll use GLBP, but you can do
load-sharing with HSRP, and that is the topic of an objective, so we'll
remember that, right? It comes in handy because it prevents situations
where a single point of failure causes traffic interruptions. This HSRP
feature improves network resilience by allowing for load-balancing and
redundancy capabilities between subnets and VLANs.

\protect\hypertarget{c16.xhtmlux5cux23Page_674}{}{}

\begin{figure}
\centering
\includegraphics{images/c16f014.jpg}
\caption{{\protect\hyperlink{c16.xhtmlux5cux23figureanchor16-14}{\textbf{Figure
16.14}} HSRP load balancing per VLAN}}
\end{figure}

\paragraph{HSRP Troubleshooting}

Besides HSRP verification, the troubleshooting of HSRP is the Cisco
objective hotspot, so let's go through this.

Most of your HSRP misconfiguration issues can be solved by checking the
output of the \texttt{show\ standby} command. In the output, you can see
the active IP and the MAC address, the timers, the active router, and
more, as shown earlier in the verification section.

There are several possible misconfigurations of HSRP, but these are what
you need to pay attention to for your CCNA:

\textbf{Different HSRP virtual IP addresses configured on the peers}
Console messages will notify you about this, of course, but if you
configure it this way and the active router fails, the standby router
takes over with a virtual IP address, which is different than the one
used previously, and different than the one configured as the
default-gateway address for end devices, so your hosts stop working,
which defeats the purpose of a FHRP.

\protect\hypertarget{c16.xhtmlux5cux23Page_675}{}{}\textbf{Different
HSRP groups configured on the peers} This misconfiguration leads to both
peers becoming active, and you'll start receiving duplicate IP address
warnings. It seems like this would be easy to troubleshoot, but the next
issue has the same warnings.

\textbf{Different HSRP versions configured on the peers or ports
blocked} HSRP comes in two versions, 1 and 2. If there is a version
mismatch, both routers will become active and you'll again have
duplicate IP address warnings.

In version 1, HSRP messages are sent to the multicast IP address
224.0.0.2 and UDP port 1985. HSRP version 2 uses the multicast IP
address 224.0.0.102 and UDP port 1985. These IP addresses and ports need
to be permitted in the inbound access lists. If the packets are blocked,
the peers will not see each other and there will be no HSRP redundancy.

\subsection[Summary]{\texorpdfstring{\protect\hypertarget{c16.xhtmlux5cux23c16-sec-20}{}{}Summary}{Summary}}

I started this chapter by discussing how to mitigate security threats at
the access layer and then also discussed external authentication for our
network devices for ease of management.

SNMP is an Application layer protocol that provides a message format for
agents on a variety of devices to communicate to network management
stations (NMSs). I discussed the basic information you need to use
syslog and SNMP, that is, configuration and verification.

Last, I showed you how to integrate redundancy and load-balancing
features into your network elegantly with the routers that you likely
have already. HSRP is Cisco proprietary; acquiring some overpriced
load-balancing device just isn't always necessary because knowing how to
properly configure and use Hot Standby Router Protocol (HSRP) can often
meet your needs instead.

\subsection[Exam
Essentials]{\texorpdfstring{\protect\hypertarget{c16.xhtmlux5cux23c16-sec-21}{}{}Exam
Essentials}{Exam Essentials}}

\textbf{Understand how to mitigate threats at the access layer.} You can
mitigate threats at the access layer by using port security, DHCP
snooping, dynamic ARP inspection, and identity-based networking.

\textbf{Understand TACACS+ and RADIUS.} TACACS+ is Cisco proprietary,
uses TCP, and can separate services. RADIUS is an open standard, uses
UDP, and cannot separate services.

\textbf{Remember the differences between SNMPv2 and SNMPv3.} SNMPv2 uses
UDP but can use TCP; however, v2 still sends data to the NMS station in
clear text, exactly like SNMPv1, plus SNMPv2 implemented GETBULK and
INFORM messages. SNMPv3 uses TCP and authenticates users, plus it can
use ACLs in the SNMP strings to protect the NMS station from
unauthorized use.

\protect\hypertarget{c16.xhtmlux5cux23Page_676}{}{}\textbf{Understand
FHRPs, especially HSRP.} The FHRPs are HSRP, VRRP, and GLBP, with HSRP
and GLBP being Cisco proprietary.

\textbf{Remember the HSRP virtual address.} The HSRP MAC address has
only one variable piece in it. The first 24 bits still identify the
vendor who manufactured the device (the organizationally unique
identifier, or OUI). The next 16 bits in the address tell us that the
MAC address is a well-known HSRP MAC address. Finally, the last 8 bits
of the address are the hexadecimal representation of the HSRP group
number.

Let me clarify all this with an example of what an HSRP MAC address
would look like:

\begin{verbatim}
0000.0c07.ac0a
\end{verbatim}

\subsection[Written Lab
16]{\texorpdfstring{\protect\hypertarget{c16.xhtmlux5cux23c16-sec-22}{}{}Written
Lab 16}{Written Lab 16}}

You can find the answers to this lab in Appendix A, ``Answers to Written
Labs.''

\begin{enumerate}
\tightlist
\item
  Which operation used by SNMP is the same as a trap but adds an
  acknowledgment that a trap does not provide?
\item
  Which operation is used by SNMP to get information from the MIB to an
  SNMP agent?
\item
  Which operation used by the SNMP agent to send a triggered piece of
  information to the SNMP manager?
\item
  Which operation is used to get information to the MIB from an SNMP
  manager?
\item
  This operation is used to list information from successive MIB objects
  within a specified MIB.
\item
  You have different HSRP virtual IP addresses configured on peers. What
  is the result?
\item
  You configure HSRP on peers with different group numbers. What is the
  result?
\item
  You configure your HSRP peers with different versions (v1 and v2).
  What is the result?
\item
  What is the multicast and port number used for both HSRP versions 1
  and 2?
\item
  The two most popular options for external AAA are what, and which one
  of them is Cisco proprietary?
\end{enumerate}

\subsection[Review
Questions]{\texorpdfstring{\protect\hypertarget{c16.xhtmlux5cux23c16-sec-23}{}{}\protect\hypertarget{c16.xhtmlux5cux23Page_677}{}{}Review
Questions}{Review Questions}}

\begin{center}\rule{0.5\linewidth}{0.5pt}\end{center}

\includegraphics{images/note.png}\textbf{The following questions are
designed to test your understanding of this chapter's material. For more
information on how to get additional questions, please see
\href{http://www.lammle.com/ccna}{www.lammle.com/ccna}.}

\begin{center}\rule{0.5\linewidth}{0.5pt}\end{center}

You can find the answers to these questions in Appendix B, ``Answers to
Review Questions.''

\begin{enumerate}
\tightlist
\item
  How can you efficiently restrict the read-only function of a
  requesting SNMP management station based on the IP address?

  \begin{enumerate}
  \def\labelenumii{\Alph{enumii}.}
  \tightlist
  \item
    Place an ACL on the logical control plane.
  \item
    Place an ACL on the line when configuring the RO community string.
  \item
    Place an ACL on the VTY line.
  \item
    Place an ACL on all router interfaces.
  \end{enumerate}
\item
  What is the default priority setting on an HSRP router?

  \begin{enumerate}
  \def\labelenumii{\Alph{enumii}.}
  \tightlist
  \item
    25
  \item
    50
  \item
    100
  \item
    125
  \end{enumerate}
\item
  Which of the following commands will enable AAA on a router?

  \begin{enumerate}
  \def\labelenumii{\Alph{enumii}.}
  \tightlist
  \item
    \texttt{aaa\ enable}
  \item
    \texttt{enable\ aaa}
  \item
    \texttt{new-model\ aaa}
  \item
    \texttt{aaa\ new-model}
  \end{enumerate}
\item
  Which of the following will mitigate access layer threats? (Choose
  two.)

  \begin{enumerate}
  \def\labelenumii{\Alph{enumii}.}
  \tightlist
  \item
    Port security
  \item
    Access lists
  \item
    Dynamic ARP inspection
  \item
    AAA
  \end{enumerate}
\item
  Which of the following is not true about DHCP snooping?

  \begin{enumerate}
  \def\labelenumii{\Alph{enumii}.}
  \tightlist
  \item
    DHCP snooping validates DHCP messages received from untrusted
    sources and filters out invalid messages.
  \item
    DHCP snooping builds and maintains the DHCP snooping binding
    database, which contains information about untrusted hosts with
    leased IP addresses.
  \item
    DHCP snooping rate-limits DHCP traffic from trusted and untrusted
    sources.
  \item
    DHCP snooping is a layer 2 security feature that acts like a
    firewall between hosts.
  \end{enumerate}
\item
  \protect\hypertarget{c16.xhtmlux5cux23Page_678}{}{}Which of the
  following are true about TACACS+? (Choose two.)

  \begin{enumerate}
  \def\labelenumii{\Alph{enumii}.}
  \tightlist
  \item
    TACACS+ is a Cisco proprietary security mechanism.
  \item
    TACACS+ uses UDP.
  \item
    TACACS+ combines authentication and authorization services as a
    single process---after users are authenticated, they are also
    authorized.
  \item
    TACACS+ offers multiprotocol support.
  \end{enumerate}
\item
  Which of the following is not true about RADIUS?

  \begin{enumerate}
  \def\labelenumii{\Alph{enumii}.}
  \tightlist
  \item
    RADIUS is an open standard protocol.
  \item
    RADIUS separates AAA services.
  \item
    RADIUS uses UDP.
  \item
    RADIUS encrypts only the password in the access-request packet from
    the client to the server. The remainder of the packet is
    unencrypted.
  \end{enumerate}
\item
  A switch is configured with the
  \texttt{snmp-server\ community\ Cisco\ RO} command running SNMPv2c. An
  NMS is trying to communicate to this router via SNMP, so what can be
  performed by the NMS? (Choose two.)

  \begin{enumerate}
  \def\labelenumii{\Alph{enumii}.}
  \tightlist
  \item
    The NMS can only graph obtained results.
  \item
    The NMS can graph obtained results and change the hostname of the
    router.
  \item
    The NMS can only change the hostname of the router.
  \item
    The NMS can use GETBULK and return many results.
  \end{enumerate}
\item
  What is true regarding any type of FHRP?

  \begin{enumerate}
  \def\labelenumii{\Alph{enumii}.}
  \tightlist
  \item
    The FHRP supplies hosts with routing information.
  \item
    The FHRP is a routing protocol.
  \item
    The FHRP provides default gateway redundancy.
  \item
    The FHRP is only standards-based.
  \end{enumerate}
\item
  Which of the following are HSRP states? (Choose two.)

  \begin{enumerate}
  \def\labelenumii{\Alph{enumii}.}
  \tightlist
  \item
    INIT
  \item
    Active
  \item
    Established
  \item
    Idle
  \end{enumerate}
\item
  Which command configures an interface to enable HSRP with the virtual
  router IP address 10.1.1.10?

  \begin{enumerate}
  \def\labelenumii{\Alph{enumii}.}
  \tightlist
  \item
    \texttt{standby\ 1\ ip\ 10.1.1.10}
  \item
    \texttt{ip\ hsrp\ 1\ standby\ 10.1.1.10}
  \item
    \texttt{hsrp\ 1\ ip\ 10.1.1.10}
  \item
    \texttt{standby\ 1\ hsrp\ ip\ 10.1.1.10}
  \end{enumerate}
\item
  \protect\hypertarget{c16.xhtmlux5cux23Page_679}{}{}Which command
  displays the status of all HSRP groups on a Cisco router or layer 3
  switch?

  \begin{enumerate}
  \def\labelenumii{\Alph{enumii}.}
  \tightlist
  \item
    \texttt{show\ ip\ hsrp}
  \item
    \texttt{show\ hsrp}
  \item
    \texttt{show\ standby\ hsrp}
  \item
    \texttt{show\ standby}
  \item
    \texttt{show\ hsrp\ groups}
  \end{enumerate}
\item
  Two routers are part of a HSRP standby group and there is no priority
  configured on the routers for the HSRP group. Which of the statements
  below is correct?

  \begin{enumerate}
  \def\labelenumii{\Alph{enumii}.}
  \tightlist
  \item
    Both routers will be in the active state.
  \item
    Both routers will be in the standby state.
  \item
    Both routers will be in the listen state.
  \item
    One router will be active, the other standby.
  \end{enumerate}
\item
  Which of the following statement is true about the HSRP version 1
  Hello packet?

  \begin{enumerate}
  \def\labelenumii{\Alph{enumii}.}
  \tightlist
  \item
    HSRP Hello packets are sent to multicast address 224.0.0.5.
  \item
    HSRP RP Hello packets are sent to multicast address 224.0.0.2 with
    TCP port 1985.
  \item
    HSRP Hello packets are sent to multicast address 224.0.0.2 with UDP
    port 1985.
  \item
    HSRP Hello packets are sent to multicast address 224.0.0.10 with UDP
    port 1986.
  \end{enumerate}
\item
  Routers HSRP1 and HSRP2 are in HSRP group 1. HSRP1 is the active
  router with a priority of 120 and HSRP2 has the default priority. When
  HSRP1 reboots, HSRP2 will become the active router. Once HSRP1 comes
  back up, which of the following statements will be true? (Choose two.)

  \begin{enumerate}
  \def\labelenumii{\Alph{enumii}.}
  \tightlist
  \item
    HSRP1 will become the active router.
  \item
    HSRP2 will stay the active router.
  \item
    HSRP1 will become the active router if it is also configured to
    preempt.
  \item
    Both routers will go into speak state.
  \end{enumerate}
\item
  What is the multicast address and port number used for HSRP version 2?

  \begin{enumerate}
  \def\labelenumii{\Alph{enumii}.}
  \tightlist
  \item
    224.0.0.2, UDP port 1985
  \item
    224.0.0.2, TCP port 1985
  \item
    224.0.0.102, UDP port 1985
  \item
    224.0.0.102, TCP port 1985
  \end{enumerate}
\item
  Which is true regarding SNMP? (Choose two.)

  \begin{enumerate}
  \def\labelenumii{\Alph{enumii}.}
  \tightlist
  \item
    SNMPv2c offers more security than SNMPv1.
  \item
    SNMPv3 uses TCP and introduced the GETBULK operation.
  \item
    SNMPv2c introduced the INFORM operation.
  \item
    SNMPv3 provides the best security of the three versions.
  \end{enumerate}
\item
  \protect\hypertarget{c16.xhtmlux5cux23Page_680}{}{}You want to
  configure RADIUS so your network devices have external authentication,
  but you also need to make sure you can fall back to local
  authentication. Which command will you use?

  \begin{enumerate}
  \def\labelenumii{\Alph{enumii}.}
  \tightlist
  \item
    \texttt{aaa\ authentication\ login\ local\ group\ MyRadiusGroup}
  \item
    \texttt{aaa\ authentication\ login\ group\ MyRadiusGroup\ fallback\ local}
  \item
    \texttt{aaa\ authentication\ login\ default\ group\ MyRadiusGroup\ external\ local}
  \item
    \texttt{aaa\ authentication\ login\ default\ group\ MyRadiusGroup\ local}
  \end{enumerate}
\item
  Which is true about DAI?

  \begin{enumerate}
  \def\labelenumii{\Alph{enumii}.}
  \tightlist
  \item
    It must use TCP, BootP, and DHCP snooping in order to work.
  \item
    DHCP snooping is required in order to build the MAC-to-IP bindings
    for DAI validation.
  \item
    DAI is required in order to build the MAC-to-IP bindings, which
    protect against man-in-the-middle attacks.
  \item
    DAI tracks ICMP-to-MAC bindings from DHCP.
  \end{enumerate}
\item
  The IEEE 802.1x standard allows you to implement identity-based
  networking on wired and wireless hosts by using client/server access
  control. There are three roles. Which of the following are these three
  roles?

  \begin{enumerate}
  \def\labelenumii{\Alph{enumii}.}
  \tightlist
  \item
    Client
  \item
    Forwarder
  \item
    Security access control
  \item
    Authenticator
  \item
    Authentication server
  \end{enumerate}
\end{enumerate}

\protect\hypertarget{c17.xhtml}{}{}

\section[{Chapter 17}\\
{Enhanced
IGRP}]{\texorpdfstring{\protect\hypertarget{c17.xhtmlux5cux23c17}{}{}\protect\hypertarget{c17.xhtmlux5cux23Page_681}{}{}{Chapter
17}\\
{Enhanced IGRP}}{Chapter 17 Enhanced IGRP}}

\begin{center}\rule{0.5\linewidth}{0.5pt}\end{center}

\subsection{THE FOLLOWING ICND2 EXAM TOPICS ARE COVERED IN THIS
CHAPTER:}

\begin{enumerate}
\tightlist
\item
  \includegraphics{images/rarr.png}2.0 Routing Technologies

  \begin{enumerate}
  \tightlist
  \item
    \includegraphics{images/squ.png} 2.2 Compare and contrast distance
    vector and link-state routing protocols
  \item
    \includegraphics{images/squ.png} 2.3 Compare and contrast interior
    and exterior routing protocols
  \item
    \includegraphics{images/squ.png} 2.6 Configure, verify, and
    troubleshoot EIGRP for IPv4 (excluding authentication, filtering,
    manual summarization, redistribution, stub)
  \item
    \includegraphics{images/squ.png} 2.7 Configure, verify, and
    troubleshoot EIGRP for IPv6 (excluding authentication, filtering,
    manual summarization, redistribution, stub
  \end{enumerate}
\end{enumerate}

\protect\hypertarget{c17.xhtmlux5cux23Page_682}{}{}\includegraphics{images/intro.png}
\emph{Enhanced Interior Gateway Routing Protocol (EIGRP)} is a Cisco
protocol that runs on Cisco routers and on some Cisco switches. In this
chapter, I'll cover the many features and functions of EIGRP, with an
added focus on the unique way that it discovers, selects, and
­advertises routes.

EIGRP has a number of features that make it especially useful within
large, complex networks. A real standout among these is its support of
VLSM, which is crucial to its ultra-efficient scalability. EIGRP even
includes benefits gained through other common protocols like OSPF and
RIPv2, such as the ability to create route summaries at any location you
choose.

I'll also cover key EIGRP configuration details and give you examples of
each, as well as demonstrate the various commands required to verify
that EIGRP is working properly. Finally, I'll wrap up the chapter by
showing you how to configure and verify EIGRPv6. I promise that after
you get through it, you'll agree that EIGRPv6 is truly the easiest part
of this chapter!

\begin{center}\rule{0.5\linewidth}{0.5pt}\end{center}

\includegraphics{images/note.png}To find up-to-the-minute updates for
this chapter, please see \href{http://www.lammle.com/}{www.lammle.com/}
\texttt{ccna} or the book's web page at
\href{http://www.sybex.com/go/ccna}{www.sybex.com/go/ccna}.

\begin{center}\rule{0.5\linewidth}{0.5pt}\end{center}

\subsection[EIGRP Features and
Operations]{\texorpdfstring{\protect\hypertarget{c17.xhtmlux5cux23c17-sec-1}{}{}EIGRP
Features and Operations}{EIGRP Features and Operations}}

EIGRP is a classless, distance-vector protocol that uses the concept of
an autonomous system to describe a set of contiguous routers that run
the same routing protocol and share routing information; it also
includes the subnet mask in its route updates. This is a very big deal
because by advertising subnet information, this robust protocol enables
us to use VLSM and permits summarization to be included within the
design of EIGRP networks.

EIGRP is sometimes referred to as a \emph{hybrid routing protocol} or an
\emph{advanced distance-vector protocol} because it has characteristics
of both distance-vector and some link-state protocols. For example,
EIGRP doesn't send link-state packets like OSPF does. Instead, it sends
traditional distance-vector updates that include information about
networks plus the cost of reaching them from the perspective of the
advertising router.

EIGRP has link-state characteristics as well---it synchronizes network
topology information between neighbors at startup and then sends
specific updates only when topology changes occur (bounded updates).
This particular feature is a huge advancement over RIP and is a big
reason that EIGRP works so well in very large networks.

EIGRP has a default hop count of 100, with a maximum of 255, but don't
let this confuse you because EIGRP doesn't rely on hop count as a metric
like RIP does. In
\protect\hypertarget{c17.xhtmlux5cux23Page_683}{}{}EIGRP-speak, hop
count refers to how many routers an EIGRP route update packet can go
through before it will be discarded, which limits the size of the
autonomous system (AS). So don't forget that this isn't how metrics are
calculated with EIGRP!

There are a bunch of powerful features that make EIGRP a real standout
from other protocols. Here's a list of some of the major ones:

\begin{enumerate}
\tightlist
\item
  Support for IP and IPv6 (and some other useless routed protocols) via
  protocol-dependent modules
\item
  Considered classless (same as RIPv2 and OSPF)
\item
  Support for VLSM/CIDR
\item
  Support for summaries and discontiguous networks
\item
  Efficient neighbor discovery
\item
  Communication via Reliable Transport Protocol (RTP)
\item
  Best path selection via Diffusing Update Algorithm (DUAL)
\item
  Reduced bandwidth usage with bounded updates
\item
  No broadcasts
\end{enumerate}

\begin{center}\rule{0.5\linewidth}{0.5pt}\end{center}

\includegraphics{images/note.png}Cisco refers to EIGRP as a
distance-vector routing protocol but also as an advanced distance-vector
or even a hybrid routing protocol.

\begin{center}\rule{0.5\linewidth}{0.5pt}\end{center}

\subsubsection[Neighbor
Discovery]{\texorpdfstring{\protect\hypertarget{c17.xhtmlux5cux23c17-sec-2}{}{}Neighbor
Discovery}{Neighbor Discovery}}

Before EIGRP routers can exchange routes with each other, they must
become neighbors, and there are three conditions that must be met before
this can happen, as shown in
\protect\hyperlink{c17.xhtmlux5cux23figure17-1}{Figure 17.1}.

\begin{figure}
\centering
\includegraphics{images/c17f001.jpg}
\caption{{\protect\hyperlink{c17.xhtmlux5cux23figureanchor17-1}{\textbf{FIGURE
17.1}} EIGRP neighbor discovery}}
\end{figure}

And these three things will be exchanged with directly connected
neighbors:

\begin{enumerate}
\tightlist
\item
  Hello or ACK received
\item
  AS numbers match
\item
  Identical metrics (K values)
\end{enumerate}

\protect\hypertarget{c17.xhtmlux5cux23Page_684}{}{}Link-state protocols
often use Hello messages to establish who their neighbors are because
they usually don't send out periodic route updates but still need a way
to help neighbors know when a new peer has arrived or an old one has
gone down. And because Hellos are also used to maintain neighbor
relationships, it follows that EIGRP routers must also continuously
receive Hellos from their neighbors.

But EIGRP routers that belong to different ASs don't automatically share
routing information and, therefore, don't become neighbors. This factor
is really helpful operating in larger networks because it reduces the
amount of route information propagated through a specific AS. But it
also means that manual redistribution can sometimes be required between
different ASs as a result. Because metrics play a big role in choosing
between the five possible factors to be evaluated when choosing the best
possible route, it's important that all EIGRP neighbors agree on how a
specific route is chosen. This is vital because the calculations on one
router depend upon the calculations of its neighbors.

Hellos between EIGRP routers are set to 5 seconds by default. Another
timer that's related to the \emph{hello timer} is the \emph{hold timer}.
The hold timer determines the amount of time a router is willing to wait
to get a Hello from a neighbor before declaring it dead. Once a neighbor
is declared dead, it's removed from the neighbor table and all routes
that depended upon it are recalculated. Interestingly, the hold timer
configuration doesn't determine how long a router waits before it
declares neighbors dead; it establishes how long the router will tell
others to wait before they can declare it dead. This means that the hold
timers on neighboring routers don't need to match because they only tell
the others how long to wait.

The only time EIGRP advertises its entire information is when it
discovers a new neighbor and forms a relationship or adjacency with it
by exchanging Hello packets. When this happens, both neighbors then
advertise their complete information to one another. After each has
learned its neighbor's routes, only changes to the routing table will be
propagated.

During each EIGRP session running on a router, a neighbor table is
created in which the router stores information about all routers known
to be directly connected neighbors. Each neighboring router's IP
address, hold time interval, \emph{smooth round-trip timer (SRTT)}, and
queue information are all kept in this table. It's an important
reference used to establish that topology changes have occurred that
neighboring routers need to know about.

To sum this all up, remember that EIGRP routers receive their neighbors'
updates and store them in a local topology table that contains all known
routes from all known neighbors and serves as the raw material from
which the best routes are selected.

Let's define some terms before we move on:

\textbf{Reported/advertised distance (RD/AD)} This is the metric of a
remote network, as reported by a neighbor. It's also the routing table
metric of the neighbor and is the same as the second number in
parentheses as displayed in the topology table. The first number is the
administrative distance, and I'll discuss more about these values in a
minute. In \protect\hyperlink{c17.xhtmlux5cux23figure17-2}{Figure 17.2},
routers SF and NY are both advertising the path to network 10.0.0.0 to
the Corp router, but the cost through SF to network 10.0.0.0 is less
than NY.

\protect\hypertarget{c17.xhtmlux5cux23Page_685}{}{}

\begin{figure}
\centering
\includegraphics{images/c17f002.jpg}
\caption{{\protect\hyperlink{c17.xhtmlux5cux23figureanchor17-2}{\textbf{FIGURE
17.2}} Advertised distance}}
\end{figure}

We're not done yet because the Corp router still needs to calculate its
cost to each neighbor.

\textbf{Feasible distance (FD)} This is the best metric among all paths
to a remote network, including the metric to the neighbor that's
advertising the remote network. The route with the lowest FD is the
route that you'll find in the routing table because it's considered the
best path. The metric of a feasible distance is calculated using the
metric reported by the neighbor that's referred to as the reported or
advertised distance plus the metric to the neighbor reporting the route.
In \protect\hyperlink{c17.xhtmlux5cux23figure17-3}{Figure 17.3}, the
Corp router will have the path through router SF to network 10.0.0.0 in
the routing table since it's the lowest feasible distance. It's the
lowest true cost from end to end.

Take a look at an EIGRP route that's been injected into a routing table
and find the FD listed in the entry.

\begin{verbatim}
D    10.0.0.0/8 [90/2195456] via 172.16.10.2, 00:27:06,Serial0/0
\end{verbatim}

First, the \texttt{D} means Dual, and it's an EIGRP injected route and
the route used by EIGRP to forward traffic to the 10.0.0.0 network via
its neighbor, 172.16.10.2. But that's not what I want to focus on right
now. See the \texttt{{[}90/2195456{]}} entry in the line? The first
number (\texttt{90}) is the administrative distance (AD), which is not
to be confused with advertised distance (AD), which is why a lot of
people call it the reported distance! The second number, is the feasible
distance (FD), or the entire cost for this router to get to network
10.0.0.0. To sum this up, the neighbor router sends a reported, or
advertised, distance (RD/AD) for network 10.0.0.0, and EIGRP calculates
the cost to get to that neighbor and then adds those two numbers
together to get the FD, or total cost.

\protect\hypertarget{c17.xhtmlux5cux23Page_686}{}{}

\begin{figure}
\centering
\includegraphics{images/c17f003.jpg}
\caption{{\protect\hyperlink{c17.xhtmlux5cux23figureanchor17-3}{\textbf{FIGURE
17.3}} Feasible distance}}
\end{figure}

\textbf{Neighbor table} Each router keeps state information about
adjacent neighbors. When a newly discovered neighbor is found, its
address and interface are recorded and the information is held in the
neighbor table, stored in RAM. Sequence numbers are used to match
acknowledgments with update packets. The last sequence number received
from the neighbor is recorded so that out-of-order packets can be
detected. We'll get into this more, later in the chapter, when we look
at the neighbor table and find out how it's useful for troubleshooting
links between neighbor routers.

\textbf{Topology table} The topology table is populated by the neighbor
table and the Diffusing Update Algorithm (DUAL) calculates the best
loop-free path to each remote network. It contains all destinations
advertised by neighboring routers, holding each destination address and
a list of neighbors that have advertised the destination. For each
neighbor, the advertised metric (distance), which comes only from the
neighbor's routing table, is recorded, as well as the FD. The best path
to each remote network is copied and placed in the routing table and
then IP will use this route to forward traffic to the remote network.
The path copied to the routing table is called a successor
router---think ``successful'' to help you remember. The path with a
good, but less desirable, cost will be entered in the topology table as
a backup link and called the feasible successor. Let's talk more about
these terms now.

\begin{center}\rule{0.5\linewidth}{0.5pt}\end{center}

\includegraphics{images/note.png}The neighbor and topology tables are
stored in RAM and maintained through the use of Hello and update
packets. While the routing table is also stored in RAM, the information
stored in the routing table is gathered only from the topology table.

\begin{center}\rule{0.5\linewidth}{0.5pt}\end{center}

\protect\hypertarget{c17.xhtmlux5cux23Page_687}{}{}\textbf{Feasible
successor (FS)} So a feasible successor is basically an entry in the
topology table that represents a path that's inferior to the successor
route(s). An FS is defined as a path whose advertised distance is less
than the feasible distance of the current successor and considered a
backup route. EIGRP will keep up to 32 feasible successors in the
topology table in 15.0 code but only up to 16 in previous IOS versions,
which is still a lot! Only the path with the best metric---the
successor---is copied and placed in the routing table. The
\texttt{show\ ip\ eigrp\ topology} command will display all the EIGRP
feasible successor routes known to the router.

\begin{center}\rule{0.5\linewidth}{0.5pt}\end{center}

\includegraphics{images/note.png}A feasible successor is a backup route
and is stored in the topology table. A successor route is stored in the
topology table and is copied and placed in the routing table.

\begin{center}\rule{0.5\linewidth}{0.5pt}\end{center}

\textbf{Successor} A successor route---again, think ``successful''---is
the best route to a remote network. A successor route is the lowest cost
to a destination and stored in the topology table along with everything
else. However, this particular best route is copied and placed in the
routing table so IP can use it to get to the remote network. The
successor route is backed up by a feasible successor route, which is
also stored in the topology table, if there's one available. The routing
table contains only successor routes; the topology table contains
successor and feasible successor routes.

\protect\hyperlink{c17.xhtmlux5cux23figure17-4}{Figure 17.4} illustrates
that the SF and NY routers each have subnets of the 10.0.0.0 network and
the Corp router has two paths to get to this network.

\begin{figure}
\centering
\includegraphics{images/c17f004.jpg}
\caption{{\protect\hyperlink{c17.xhtmlux5cux23figureanchor17-4}{\textbf{FIGURE
17.4}} The tables used by EIGRP}}
\end{figure}

As shown in \protect\hyperlink{c17.xhtmlux5cux23figure17-4}{Figure
17.4}, there are two paths to network 10.0.0.0 that can be used by the
Corp router. EIGRP picks the best path and places it in the routing
table, but if both links
\protect\hypertarget{c17.xhtmlux5cux23Page_688}{}{}have equal-cost
paths, EIGRP would load-balance between them---up to four links, by
default. By using the successor, and having feasible successors in the
topology table as backup links, the network can converge instantly and
updates to any neighbor make up the only traffic sent from EIGRP---very
clean!

\subsubsection[Reliable Transport Protocol
(RTP)]{\texorpdfstring{\protect\hypertarget{c17.xhtmlux5cux23c17-sec-3}{}{}Reliable
Transport Protocol (RTP)}{Reliable Transport Protocol (RTP)}}

EIGRP depends on a proprietary protocol, called \emph{Reliable Transport
Protocol (RTP)}, to manage the communication of messages between
EIGRP-speaking routers. As the name suggests, reliability is a key
concern of this protocol, so Cisco designed this mechanism, which
leverages multicasts and unicasts, to ensure that updates are delivered
quickly and that data reception is tracked accurately.

But how does this really work? Well, when EIGRP sends multicast traffic,
it uses the Class D address 224.0.0.10, and each EIGRP router knows who
its neighbors are. For each multicast it sends out, a list is built and
maintained that includes all the neighbors who have replied. If a router
doesn't get a reply from a neighbor via the multicast, EIGRP will then
try using unicasts to resend the same data. If there's no reply from a
neighbor after 16 unicast attempts, that neighbor will then be declared
dead. This process is often referred to as \emph{reliable multicast}.

Routers keep track of the information they send by assigning a sequence
number to each packet that enables them to identify old, redundant
information and data that's out of sequence. You'll get to actually see
this information in the neighbor table coming up when we get into
configuring EIGRP.

Remember, EIGRP is all about topology changes and updates, making it the
quiet, performance-optimizing protocol it is. Its ability to synchronize
routing databases at startup time, while maintaining the consistency of
databases over time, is achieved quietly by communicating only necessary
changes. The downside here is that you can end up with a corrupted
routing database if any packets have been permanently lost or if packets
have been mishandled out of order!

Here's a description of the five different types of packets used by
EIGRP:

\textbf{Update} An \emph{Update packet} contains route information. When
these are sent in response to metric or topology changes, they use
reliable multicasts. In the event that only one router needs an update,
like when a new neighbor is discovered, it's sent via unicasts. Keep in
mind that the unicast method still requires an acknowledgment, so
updates are always reliable regardless of their underlying delivery
mechanism.

\textbf{Query} A \emph{Query packet} is a request for specific routes
and always uses the reliable multicast method. Routers send queries when
they realize they've lost the path to a particular network and are
searching for alternatives.

\textbf{Reply} A \emph{Reply packet} is sent in response to a query via
the unicast method. Replies either include a specific route to the
queried destination or declare that there's no known route.

\textbf{Hello} A \emph{Hello packet} is used to discover EIGRP neighbors
and is sent via unreliable multicast, meaning it doesn't require an
acknowledgment.

\protect\hypertarget{c17.xhtmlux5cux23Page_689}{}{}\textbf{ACK} An
\emph{ACK packet} is sent in response to an update and is always
unicast. ACKs are never sent reliably because this would require another
ACK sent for acknowledgment, which would just create a ton of useless
traffic!

It's helpful to think of all these different packet types like
envelopes. They're really just types of containers that EIGRP routers
use to communicate with their neighbors. What's really interesting is
the actual content envelopes these communications and the procedures
that guide their conversations, and that's what we'll be exploring next!

\subsubsection[Diffusing Update Algorithm
(DUAL)]{\texorpdfstring{\protect\hypertarget{c17.xhtmlux5cux23c17-sec-4}{}{}Diffusing
Update Algorithm (DUAL)}{Diffusing Update Algorithm (DUAL)}}

I mentioned that EIGRP uses \emph{Diffusing Update Algorithm (DUAL)} for
selecting and maintaining the best path to each remote network. DUAL
allows EIGRP to carry out these vital tasks:

\begin{enumerate}
\tightlist
\item
  Figure out a backup route if there's one available.
\item
  Support variable length subnet masks (VLSMs).
\item
  Perform dynamic route recoveries.
\item
  Query neighbors for unknown alternate routes.
\item
  Send out queries for an alternate route.
\end{enumerate}

Quite an impressive list, but what really makes DUAL so great is that it
enables EIGRP to converge amazingly fast! The key to the speed is
twofold: First, EIGRP routers maintain a copy of all of their neighbors'
routes to refer to for calculating their own cost to each remote
network. So if the best path goes down, all it often takes to find
another one is a quick scan of the topology table looking for a feasible
successor. Second, if that quick table survey doesn't work out, EIGRP
routers immediately ask their neighbors for help finding the best path.
It's exactly this, ahem, DUAL strategy of reliance upon, and the
leveraging of, other routers' information that accounts for the
algorithm's ``diffusing'' character. Unlike other routing protocols
where the change is propagated through the entire network, EIGRP bounded
updates are propagated only as far as needed.

Three critical conditions must be met for DUAL to work properly:

\begin{enumerate}
\tightlist
\item
  Neighbors are discovered or noted as dead within a finite time.
\item
  All transmitted messages are received correctly.
\item
  All changes and messages are processed in the order in which they're
  detected.
\end{enumerate}

As you already know, the Hello protocol ensures the rapid detection of
new or dead neighbors, and RTP provides a reliable method of conveying
and sequencing messages. Based upon this solid foundation, DUAL can then
select and maintain information about the best paths. Let's check
further into the process of route discovery and maintenance next.

\subsubsection[Route Discovery and
Maintenance]{\texorpdfstring{\protect\hypertarget{c17.xhtmlux5cux23c17-sec-5}{}{}Route
Discovery and Maintenance}{Route Discovery and Maintenance}}

The hybrid nature of EIGRP is fully revealed in its approach to route
discovery and maintenance. Like many link-state protocols, EIGRP
supports the concept of neighbors that are formally discovered via a
Hello process and whose state is monitored thereafter. And like
\protect\hypertarget{c17.xhtmlux5cux23Page_690}{}{}many distance-vector
protocols, EIGRP uses the routing-by-rumor approach, which implies that
many routers within an AS never actually hear about a route update
firsthand. Instead, these devices rely on ``network gossip'' to hear
about neighbors and their respective status via another router that may
have also gotten the info from yet another router and so on.

Given all of the information that EIGRP routers have to collect, it
follows that they must have a place to store it, and they do this in the
tables I referred to earlier in this chapter. As you know, EIGRP doesn't
depend on just one table---it actually uses three of them to store
important information about its environment:

\textbf{Neighbor table} Contains information about the specific routers
with whom neighbor relationships have been formed. It also displays
information about the Hello transmit interval and queue counts for
unaccounted Hello acknowledgment.

\textbf{Topology table} Stores the route advertisements received from
each neighbor. All routes in the AS are stored in the topology table,
both successors and feasible successors.

\textbf{Route table} Stores the routes that are currently in use to make
local routing decisions. Anything in the routing table is considered a
successor route.

We'll explore more of EIGRP's features in greater detail soon, beginning
with a look at the metrics associated with particular routes. After
that, I'll cover the decision-making process that's used to select the
best routes, and then we'll review the procedures followed when routes
change.

\subsection[Configuring
EIGRP]{\texorpdfstring{\protect\hypertarget{c17.xhtmlux5cux23c17-sec-6}{}{}Configuring
EIGRP}{Configuring EIGRP}}

I know what you're thinking! ``We're going to jump in to configuring
EIGRP already when I've heard how complex it is?'' No worries
here---what I'm about to show is basic, and I know you won't have a
problem with it at all! We're going to start with the easy part of
EIGRP, and by configuring it on our little internetwork, you'll learn a
lot more this way than you would if I just continued explaining more at
this point. After we've completed the initial configuration, we'll
fine-tune it and have fun experimenting with it throughout this chapter!

Okay, there are two modes for entering EIGRP commands: router
configuration mode and interface configuration mode. In router
configuration mode, we'll enable the protocol, determine which networks
will run EIGRP, and set global factors. When in interface configuration
mode, we'll customize summaries and bandwidth.

To initiate an EIGRP session on a router, I'll use the
\texttt{router\ eigrp} command followed by our network's AS number.
After that, we'll enter the specific numbers of the networks that we
want to connect to the router using the \texttt{network} command
followed by the network number. This is pretty straightforward
stuff---if you can configure RIP, then you can configure EIGRP!

Just so you know, we'll use the same network I used in the previous
CCENT routing chapters, but I'm going to connect more networks so we can
look deeper into EIGRP. With that, I'm going to enable EIGRP for
autonomous system 20 on our Corp router connected to four networks.

\protect\hypertarget{c17.xhtmlux5cux23Page_691}{}{}\protect\hyperlink{c17.xhtmlux5cux23figure17-5}{Figure
17.5} shows the network we'll be configuring throughout this chapter and
the next chapter. Here's the Corp configuration:

\begin{figure}
\centering
\includegraphics{images/c17f005.jpg}
\caption{{\protect\hyperlink{c17.xhtmlux5cux23figureanchor17-5}{\textbf{FIGURE
17.5}} Configuring our little internetwork with EIGRP}}
\end{figure}

\begin{verbatim}
Corp#config t
Corp(config)#router eigrp 20
Corp(config-router)#network 172.16.0.0
Corp(config-router)#network 10.0.0.0
\end{verbatim}

Remember, just as we would when configuring RIP, we need to use the
classful network address, which is all subnet and host bits turned off.
This is another thing that makes EIGRP so great: it has the complexity
of a link-state protocol running in the background and the same easy
configuration process used for RIP!

\begin{center}\rule{0.5\linewidth}{0.5pt}\end{center}

\includegraphics{images/note.png}Understand that the AS number is
irrelevant---that is, as long as all routers use the same number! You
can use any number from 1 to 65,535.

\begin{center}\rule{0.5\linewidth}{0.5pt}\end{center}

But wait, the EIGRP configuration can't be that easy, can it? A few
simple EIGRP commands and my network just works? Well, it can be and
usually is, but not always. Remember the wildcards you learned about in
your access list configurations in your preparation for the Cisco exam?
Let's say, for example, that we wanted to advertise all the directly
connected networks with EIGRP off the Corp router. By using the command
\texttt{network\ 10.0.0.0}, we can effectively advertise to all subnets
within that classful network; however, take a look at this configuration
now:

\begin{verbatim}
Corp#config t
Corp(config)#router eigrp 20
Corp(config-router)#network 10.10.11.0 0.0.0.255
Corp(config-router)#network 172.16.10.0 0.0.0.3
Corp(config-router)#network 172.16.10.4 0.0.0.3
\end{verbatim}

\protect\hypertarget{c17.xhtmlux5cux23Page_692}{}{}This configuration
should look pretty familiar to you because by now you should have a
solid understanding of how wildcards are configured. This configuration
will advertise the network connected to g0/1 on the Corp router as well
as the two WAN links. Still, all we accomplished with this configuration
was to stop the g0/0 interface from being placed into the EIGRP process,
and unless you have tens of thousands of networks worldwide, then there
is really no need to use wildcards because they don't provide any other
administrative purpose other than what I've already described.

Now let's take a look at the simple configuration needed for the SF and
NY routers in our internetwork:

\begin{verbatim}
SF(config)#router eigrp 20
SF(config-router)#network 172.16.0.0
SF(config-router)#network 10.0.0.0
000060:&#x00025;DUAL-5-NBRCHANGE:IP-EIGRP(0) 20:Neighbor 172.16.10.1 (Serial0/0/0) is up:
new adjacency
 
NY(config)#router eigrp 20
NY(config-router)#network 172.16.0.0
NY(config-router)#network 10.0.0.0
*Jun 26 02:41:36:%DUAL-5-NBRCHANGE:IP-EIGRP(0) 20:Neighbor 172.16.10.5 (Serial0/0/1) is up: new adjacency
\end{verbatim}

Nice and easy---or is it? We can see that the SF and NY router created
an adjacency to the Corp router, but are they actually sharing routing
information? To find out, let's take a look at the number that I pointed
out as the autonomous system (AS) number in the configuration.

EIGRP uses ASs to identify the group of routers that will share route
information. Only routers that have the same AS share routes. The range
of values we can use to create an AS with EIGRP is 1--65535:

\begin{verbatim}
Corp(config)#router eigrp ?
  <1-65535>  Autonomous System
  WORD       EIGRP Virtual-Instance Name
Corp(config)#router eigrp 20
\end{verbatim}

Notice that I could have used any number from 1 to 65,535, but I chose
to use 20 because it just felt good at the time. As long as all routers
use the same number, they'll create an adjacency. Okay, now the AS makes
sense, but it looks like I can type a \texttt{word} in the place of the
AS number, and I can! Let's take a look at the configuration:

\begin{verbatim}
Corp(config)#router eigrp Todd
Corp(config-router)#address-family ipv4 autonomous-system 20
Corp(config-router-af)#network 10.0.0.0
Corp(config-router-af)#network 172.16.0.0
\end{verbatim}

\protect\hypertarget{c17.xhtmlux5cux23Page_693}{}{}What I just showed
you is not part of the Cisco exam objectives, but it's also not really
necessary for any IPv4 routing configuration in your network. The
previous configuration examples I've gone through so far in this chapter
covers the objectives and work just fine, but I included this last
configuration example because it's now an option in IOS 15.0 code.

\subsubsection[VLSM Support and
Summarization]{\texorpdfstring{\protect\hypertarget{c17.xhtmlux5cux23c17-sec-7}{}{}VLSM
Support and Summarization}{VLSM Support and Summarization}}

Being one of the more sophisticated classless routing protocols, EIGRP
supports using variable length subnet masks. This is good because it
allows us to conserve address space by using subnet masks that map to
specific host requirements in a much better way. Being able to use
30-bit subnet masks for the point-to-point networks that I configured in
our internetwork is a great example. Plus, because the subnet mask is
propagated with every route update, EIGRP also supports the use of
discontiguous subnets, giving us greater administrative flexibility when
designing a network IP address scheme. Another versatile feature is that
EIGRP allows us to use and place route summaries at strategically
optimal locations throughout the EIGRP network to reduce the size of the
routing table.

Keep in mind that EIGRP automatically summarizes networks at their
classful boundaries and supports the manual creation of summaries at any
and all EIGRP routers. This is usually a good thing, but by checking out
the routing table in the Corp router, you can see the possible
complications that auto-summarization can cause:

\begin{verbatim}
Corp#sh ip route
[output cut]
     172.16.0.0/16 is variably subnetted, 3 subnets, 2 masks
C       172.16.10.4/30 is directly connected, Serial0/1
C       172.16.10.0/30 is directly connected, Serial0/0
D       172.16.0.0/16 is a summary, 00:01:37, Null0
     10.0.0.0/8 is variably subnetted, 3 subnets, 2 masks
C       10.10.10.0/24 is directly connected, GigabitEthernet0/0
D       10.0.0.0/8 is a summary, 00:01:19, Null0
C       10.10.11.0/24 is directly connected, GigabitEthernet0/1
\end{verbatim}

Now this just doesn't look so good---both 172.16.0.0 and 10.0.0.0/8 are
being advertised as summary routes injected by EIGRP, but we have
multiple subnets in the 10.0.0.0/8 classful network address, so how
would the Corp router know how to route to a specific network like
10.10.20.0? The answer is, it wouldn't. Let's see why in
\protect\hyperlink{c17.xhtmlux5cux23figure17-6}{Figure 17.6}.

The networks we're using make up what is considered a discontinuous
network because we have the 10.0.0.0/8 network subnetted across a
different class of address, the 172.16.0.0 network, with 10.0.0.0/8
subnets on both sides of the WAN links.

You can see that the SF and NY routers will both create an automatic
summary of 10.0.0.0/8 and then inject it into their routing tables. This
is a common problem, and an important one that Cisco really wants you to
understand (by including it in the objectives)! With this type of
topology, disabling automatic summarization is definitely the better
option. Actually, it's the only option if we want this network to work.

\protect\hypertarget{c17.xhtmlux5cux23Page_694}{}{}

\begin{figure}
\centering
\includegraphics{images/c17f006.jpg}
\caption{{\protect\hyperlink{c17.xhtmlux5cux23figureanchor17-6}{\textbf{FIGURE
17.6}} Discontiguous networks}}
\end{figure}

Let's take a look at the routing tables on the NY and SF routers to find
out what they're seeing:

\begin{verbatim}
SF>sh ip route
[output cut]
     172.16.0.0/16 is variably subnetted, 3 subnets, 3 masks
C       172.16.10.0/30 is directly connected, Serial0/0/0
D       172.16.10.0/24 [90/2681856] via 172.16.10.1, 00:54:58, Serial0/0/0
D       172.16.0.0/16 is a summary, 00:55:12, Null0
     10.0.0.0/8 is variably subnetted, 3 subnets, 2 masks
D       10.0.0.0/8 is a summary, 00:54:58, Null0
C       10.10.20.0/24 is directly connected, FastEthernet0/0
C       10.10.30.0/24 is directly connected, Loopback0
SF>
 
NY>sh ip route
[output cut]
     172.16.0.0/16 is variably subnetted, 2 subnets, 2 masks
C       172.16.10.4/30 is directly connected, Serial0/0/1
D       172.16.0.0/16 is a summary, 00:55:56, Null0
     10.0.0.0/8 is variably subnetted, 3 subnets, 2 masks
D       10.0.0.0/8 is a summary, 00:55:26, Null0
C       10.10.40.0/24 is directly connected, FastEthernet0/0
C       10.10.50.0/24 is directly connected, Loopback0
NY>ping 10.10.10.1
Type escape sequence to abort.
Sending 5, 100-byte ICMP Echos to 10.10.10.1, timeout is 2 seconds:
.....
Success rate is 0 percent (0/5)
NY>
\end{verbatim}

The confirmed answer is that our network isn't working because we're
discontiguous and our classful boundaries are auto-summarizing. We can
see that EIGRP is injecting summary routes into both the SF and NY
routing tables.

We need to advertise our subnets in order to make this work, and here's
how we make that happen, starting with the Corp router:

\begin{verbatim}
Corp#config t
Corp(config)#router eigrp 20
Corp(config-router)#no auto-summary
Corp(config-router)#
*Feb 25 18:29:30%DUAL-5-NBRCHANGE:IP-EIGRP(0) 20:Neighbor 172.16.10.6 (Serial0/1)
 is resync: summary configured
*Feb 25 18:29:30%DUAL-5-NBRCHANGE:IP-EIGRP(0) 20:Neighbor 172.16.10.2 (Serial0/0)
 is resync: summary configured
Corp(config-router)#
\end{verbatim}

Okay---our network still isn't working because the other routers are
still sending a summary. So let's configure the SF and NY routers to
advertise subnets:

\begin{verbatim}
SF#config t
SF(config)#router eigrp 20
SF(config-router)#no auto-summary
SF(config-router)#
000090:%DUAL-5-NBRCHANGE:IP-EIGRP(0) 20:Neighbor 172.16.10.1 (Serial0/0/0) is resync: summary configured
 
NY#config t
NY(config)#router eigrp 20
NY(config-router)#no auto-summary
NY(config-router)#
*Jun 26 21:31:08%DUAL-5-NBRCHANGE:IP-EIGRP(0) 20:Neighbor 172.16.10.5 (Serial0/0/1)
is resync: summary configured
\end{verbatim}

Let's take a look at the Corp router's output now:

\begin{verbatim}
Corp(config-router)#do show ip route
[output cut]
     172.16.0.0/30 is subnetted, 2 subnets
C       172.16.10.4 is directly connected, Serial0/1
C       172.16.10.0 is directly connected, Serial0/0
     10.0.0.0/24 is subnetted, 6 subnets
C       10.10.10.0 is directly connected, GigabitEthernet0/0
C       10.10.11.0 is directly connected, GigabitEthernet0/1
D       10.10.20.0 [90/3200000] via 172.16.10.2, 00:00:27, Serial0/0
D       10.10.30.0 [90/3200000] via 172.16.10.2, 00:00:27, Serial0/0
D       10.10.40.0 [90/2297856] via 172.16.10.6, 00:00:29, Serial0/1
D       10.10.50.0 [90/2297856] via 172.16.10.6, 00:00:30, Serial0/1
Corp# ping 10.10.20.1
 
Type escape sequence to abort.
Sending 5, 100-byte ICMP Echos to 10.10.20.1, timeout is 2 seconds:
!!!!!
Success rate is 100 percent (5/5), round-trip min/avg/max = 1/2/4 ms
\end{verbatim}

Wow, what a difference compared to the previous routing table output! We
can see all the subnets now. It would be hard to justify using
auto-summarization today. If you want to summarize, it should definitely
be done manually. Always typing in \texttt{no\ auto-summary} under RIPv2
and EIGRP is common practice today.

\begin{center}\rule{0.5\linewidth}{0.5pt}\end{center}

\includegraphics{images/note.png}The new 15.\emph{x} code
auto-summarization feature is disabled by default, as it should be. But
don't think that discontiguous networks and disabling auto-summary are
no longer topics in the Cisco exam objectives, because they most
certainly are! When troubleshooting EIGRP on the exam, verify the code
version, and if it is 15.\emph{x} code, then you can assume that
auto-summary is not a problem.

\begin{center}\rule{0.5\linewidth}{0.5pt}\end{center}

\subsubsection[Controlling EIGRP
Traffic]{\texorpdfstring{\protect\hypertarget{c17.xhtmlux5cux23c17-sec-8}{}{}Controlling
EIGRP Traffic}{Controlling EIGRP Traffic}}

But what if you need to stop EIGRP from working on a specific interface?
Maybe it's a connection to your ISP, or where we didn't want to have the
g0/0 interface be part of the EIGRP process as in our earlier example.
All you need to do is to flag the interface as passive, and to do this
from an EIGRP session, just use this command:

\begin{verbatim}
passive-interface interface-type interface-number
\end{verbatim}

This works because the \texttt{interface-type} portion defines the type
of interface and the \texttt{interface-number} portion defines the
number of the interface. The following command makes interface serial
0/0 into a passive interface:

\begin{verbatim}
Corp(config)#router eigrp 20
Corp(config-router)#passive-interface g0/0
\end{verbatim}

\protect\hypertarget{c17.xhtmlux5cux23Page_697}{}{}What we've
accomplished here is to prevent this interface from sending or reading
received Hello packets so that it will no longer form adjacencies or
send or receive route information. But this still won't stop EIGRP from
advertising the subnet of this interface out all other interfaces
without using wildcards. This really illustrates the reason you must
understand why and when to use wildcards as well as what the
\texttt{passive-interface} command does. This knowledge really helps you
to make an informed decision on which command you need to use to meet
your specific business requirements!

\begin{center}\rule{0.5\linewidth}{0.5pt}\end{center}

\includegraphics{images/note.png}The impact of the
\texttt{passive-interface} command depends upon the routing protocol
under which the command is issued. For example, on an interface running
RIP, the \texttt{passive-interface} command will prohibit sending route
updates but will permit receiving them. An RIP router with a passive
interface will still learn about the networks advertised by other
routers. This is different from EIGRP, where an interface configured
with the ­\texttt{passive-interface} command will neither send nor read
received Hellos.

\begin{center}\rule{0.5\linewidth}{0.5pt}\end{center}

Typically, EIGRP neighbors use multicast to exchange routing updates.
You can change this by specifically telling the router about a
particular neighbor, which will ensure that unicast packets will only be
used for the routing updates with that specific neighbor. To take
advantage of this feature, apply the \texttt{neighbor} command and
execute it under the EIGRP process.

I'm going to configure the Corp router with information about routers SF
and NY:

\begin{verbatim}
Corp(config)#router eigrp 20
Corp(config-router)#neighbor 172.16.10.2
Corp(config-router)#neighbor 172.16.10.6
\end{verbatim}

Understand that you don't need to use the preceding commands to create
neighbor relationships, but they're available if you need them.

\paragraph{EIGRP Metrics}

Unlike many other protocols that use a single element to compare routes
and select the best possible path, EIGRP uses a combination of these
four factors:

\begin{enumerate}
\tightlist
\item
  \emph{Bandwidth}
\item
  \emph{Delay}
\item
  \emph{Load}
\item
  \emph{Reliability}
\end{enumerate}

It's worth noting that there's a fifth element, \emph{maximum
transmission unit (MTU)}, which has never been used in EIGRP metrics
calculations though it's still a required parameter in some
EIGRP-related commands---especially those involving redistribution. The
value of the MTU element represents the smallest MTU value encountered
along the path to the destination network.

\protect\hypertarget{c17.xhtmlux5cux23Page_698}{}{}Also good to know is
that there's a mathematical formula that combines the four main elements
to create a single value representing just how good a given route
actually is. The higher the metric associated with it, the less
desirable the route. Here's that formula:

\includegraphics{images/c17ue001.png}

The formula's components break down like this:

\begin{enumerate}
\tightlist
\item
  By default, K\textsubscript{1} = 1, K\textsubscript{2} = 0,
  K\textsubscript{3} = 1, K\textsubscript{4} = 0, K\textsubscript{5} =
  0.
\item
  \emph{Delay} equals the sum of all the delays of the links along the
  path.

  \begin{enumerate}
  \tightlist
  \item
    \emph{Delay} = {[}Delay in 10s of microseconds{]} × 256.
  \end{enumerate}
\item
  \emph{Bandwidth} is the lowest bandwidth of the links along the path.

  \begin{enumerate}
  \tightlist
  \item
    \emph{Bandwidth} = {[}10000000 / (bandwidth in Kbps){]} × 256.
  \end{enumerate}
\item
  By default, \emph{metric} = lowest bandwidth along path + sum of all
  delays along path.
\end{enumerate}

If necessary, you can adjust the constant K values on a per-interface
basis, but I would recommend that you only do this under the direction
of the Cisco Technical Assistance Center (TAC). Metrics are tuned to
change the manner in which routes are calculated. The K values can be
seen with a \texttt{show\ ip\ protocols} output:

\begin{verbatim}
Corp#sh ip protocols
*** IP Routing is NSF aware ***
 
Routing Protocol is "eigrp 1"
  Outgoing update filter list for all interfaces is not set
  Incoming update filter list for all interfaces is not set
  Default networks flagged in outgoing updates
  Default networks accepted from incoming updates
  EIGRP-IPv4 Protocol for AS(1)
    Metric weight K1=1, K2=0, K3=1, K4=0, K5=0
\end{verbatim}

Notice that that the K1 and K3 values are enabled by default---for
example, K1 = 1. \protect\hyperlink{c17.xhtmlux5cux23table17-1}{Table
17.1} shows the relationship between each constant and the metric it
affects.

Each constant is used to assign a weight to a specific variable, meaning
that when the metric is calculated, the algorithm will assign a greater
importance to the specified metric. This is very cool because it means
that by assigning a weight, you get to specify the factor that's most
important to you. For example, if bandwidth is your priority, you would
assign K1 to weight it accordingly, but if delay is totally
unacceptable, then K3 would be assigned a greater weight. A word of
caution though: Always remember that any changes to the default values
could result in instability and convergence problems, particularly if
delay or reliability values are constantly changing! But if you're
looking for something to do on a rainy Saturday, it's an interesting
experiment to pass some time and gain some nice networking insight!

\protect\hypertarget{c17.xhtmlux5cux23Page_699}{}{}

{\protect\hyperlink{c17.xhtmlux5cux23tableanchor17-1}{\textbf{TABLE~17.1}}
Metric association of K values}

\begin{longtable}[]{@{}ll@{}}
\toprule
Constant & Metric\tabularnewline
\midrule
\endhead
K1 & Bandwidth (B\textsubscript{e})\tabularnewline
K2 & Load (utilization on path)\tabularnewline
K3 & Delay (D\textsubscript{c})\tabularnewline
K4 & Reliability (r)\tabularnewline
K5 & MTU\tabularnewline
\bottomrule
\end{longtable}

\paragraph{Maximum Paths and Hop Count}

By default, EIGRP can provide equal-cost load balancing across up to 4
links. RIP and OSPF do this too. But you can have EIGRP actually
load-balance across up to 32 links with 15.0 code (equal or unequal) by
using the following command:

\begin{verbatim}
Corp(config)#router eigrp 10
Corp(config-router)#maximum-paths ?
  <1-32>  Number of paths
\end{verbatim}

As I mentioned, pre--15.0 code routers allowed up to 16 paths to remote
networks, which is still a lot!

EIGRP has a default maximum hop count of 100 for route update packets,
but it can be set up to 255. Chances are you wouldn't want to ever
change this, but if you did, here is how you would do it:

\begin{verbatim}
Corp(config)#router eigrp 10
Corp(config-router)#metric maximum-hops ?
  <1-255>  Hop count
\end{verbatim}

As you can see from this router output, EIGRP can be set to a maximum of
255 hops. Even though it doesn't use hop count in the path metric
calculation, it still uses the maximum hop count to limit the scope of
the AS.

\paragraph{Route Selection}

Now that you've got a good idea how EIGRP works and also how easy it
actually is to configure, it's probably clear that determining the best
path simply comes down to seeing which one gets awarded the lowest
metric. But it's not the winning path that really sets EIGRP apart from
other protocols. You know that EIGRP stores route information from its
neighbors in its topology table and that as long as a given neighbor
remains alive, it will
\protect\hypertarget{c17.xhtmlux5cux23Page_700}{}{}rarely throw out
anything it has learned from that neighbor. This makes EIGRP able to
flag the best routes in its topology table for positioning in its local
routing table, enabling it to flag the next-best routes as alternatives
if the best route goes down.

In \protect\hyperlink{c17.xhtmlux5cux23figure17-7}{Figure 17.7}, you can
see that I added another Fast Ethernet link between the SF and NY
routers. This will give us a great opportunity to play with the topology
and routing tables!

\begin{figure}
\centering
\includegraphics{images/c17f007.jpg}
\caption{{\protect\hyperlink{c17.xhtmlux5cux23figureanchor17-7}{\textbf{FIGURE
17.7}} EIGRP route selection process}}
\end{figure}

First, let's take another look at the routing table on the Corp router
before I bring up the new interfaces:

\begin{verbatim}
172.16.0.0/30 is subnetted, 2 subnets
C       172.16.10.4 is directly connected, Serial0/1
C       172.16.10.0 is directly connected, Serial0/0
     10.0.0.0/24 is subnetted, 6 subnets
C       10.10.10.0 is directly connected, GigabitEthernet0/0
C       10.10.11.0 is directly connected, GigabitEthernet0/1
D       10.10.20.0 [90/3200000] via 172.16.10.2, 00:00:27, Serial0/0
D       10.10.30.0 [90/3200000] via 172.16.10.2, 00:00:27, Serial0/0
D       10.10.40.0 [90/2297856] via 172.16.10.6, 00:00:29, Serial0/1
D       10.10.50.0 [90/2297856] via 172.16.10.6, 00:00:30, Serial0/1
\end{verbatim}

We can see the three directly connected interfaces as well as the other
four networks injected into the routing table by EIGRP. Now I'll add the
network 192.168.10.0/24 between the SF and NY routers, then enable the
interfaces.

And let's check out the routing table of the Corp router now that I've
configured that link:

\begin{verbatim}
D    192.168.10.0/24 [90/2172416] via 172.16.10.6, 00:04:27, Serial0/1
 172.16.0.0/30 is subnetted, 2 subnets
C       172.16.10.4 is directly connected, Serial0/1
C       172.16.10.0 is directly connected, Serial0/0
     10.0.0.0/24 is subnetted, 6 subnets
C       10.10.10.0 is directly connected, GigabitEthernet0/0
C       10.10.11.0 is directly connected, GigabitEthernet0/1
D       10.10.20.0 [90/3200000] via 172.16.10.2, 00:00:27, Serial0/0
D       10.10.30.0 [90/3200000] via 172.16.10.2, 00:00:27, Serial0/0
D       10.10.40.0 [90/2297856] via 172.16.10.6, 00:00:29, Serial0/1
D       10.10.50.0 [90/2297856] via 172.16.10.6, 00:00:30, Serial0/1
\end{verbatim}

Okay---that's weird. The only thing different I see is one path to the
192.168.10.0/24 network listed first. Glad it is there, which means that
we can route to that network. Notice that we can reach the network from
the Serial0/1 interface, but what happened to my link to the SF
router---shouldn't we have an advertisement from that router and be
load-balancing? Let's take a look the topology table to find out what's
going on:

\begin{verbatim}
Corp#sh ip eigrp topology
IP-EIGRP Topology Table for AS(20)/ID(10.10.11.1)
 
Codes: P - Passive, A - Active, U - Update, Q - Query, R - Reply,
       r - reply Status, s - sia Status
 
P 10.10.10.0/24, 1 successors, FD is 128256
        via Connected, GigbitEthernet0/0
P 10.10.11.0/24, 1 successors, FD is 128256
        via Connected, GigbitEthernet0/1
P 10.10.20.0/24, 1 successors, FD is 2300416
        via 172.16.10.6 (2300416/156160), Serial0/1
        via 172.16.10.2 (3200000/128256), Serial0/0
P 10.10.30.0/24, 1 successors, FD is 2300416
        via 172.16.10.6 (2300416/156160), Serial0/1
        via 172.16.10.2 (3200000/128256), Serial0/0
P 10.10.40.0/24, 1 successors, FD is 2297856
        via 172.16.10.6 (2297856/128256), Serial0/1
        via 172.16.10.2 (3202560/156160), Serial0/0
P 10.10.50.0/24, 1 successors, FD is 2297856
        via 172.16.10.6 (2297856/128256), Serial0/1
        via 172.16.10.2 (3202560/156160), Serial0/0
P 192.168.10.0/24, 1 successors, FD is 2172416
        via 172.16.10.6 (2172416/28160), Serial0/1
        via 172.16.10.2 (3074560/28160), Serial0/0
P 172.16.10.4/30, 1 successors, FD is 2169856
        via Connected, Serial0/1
P 172.16.10.0/30, 1 successors, FD is 3072000
        via Connected, Serial0/0
\end{verbatim}

\protect\hypertarget{c17.xhtmlux5cux23Page_702}{}{}Okay, we can see
there are two paths to the 192.168.10.0/24 network, but it's using the
next hop of 172.16.10.6 (NY) because the feasible distance (FD) is less!
The advertised distance from both routers is 28160, but the cost to get
to each router via the WAN links is not the same. This means the FD is
not the same, meaning we're not load-balancing by default.

Both WAN links are a T1, so this should have load-balanced by default,
but EIGRP has determined that it costs more to go through SF than
through NY. Since EIGRP uses bandwidth and delay of the line to
determine the best path, we can use the \texttt{show\ interfaces}
command to verify our stats like this:

\begin{verbatim}
Corp#sh int s0/0
Serial0/0 is up, line protocol is up
  Hardware is PowerQUICC Serial
  Description: <<Connection to CR1>>
  Internet address is 172.16.10.1/30
  MTU 1500 bytes, BW 1000 Kbit, DLY 20000 usec,
     reliability 255/255, txload 1/255, rxload 1/255
  Encapsulation HDLC, loopback not set Keepalive set (10 sec)
 
Corp#sh int s0/1
Serial0/1 is up, line protocol is up
  Hardware is PowerQUICC Serial
  Internet address is 172.16.10.5/30
  MTU 1500 bytes, BW 1544 Kbit, DLY 20000 usec,
     reliability 255/255, txload 1/255, rxload 1/255
  Encapsulation HDLC, loopback not set Keepalive set (10 sec)
\end{verbatim}

I highlighted the statistics that EIGRP uses to determine the metrics to
a next-hop router: MTU, bandwidth, delay, reliability, and load, with
bandwidth and delay enabled by default. We can see that the bandwidth on
the Serial0/0 interface is set to 1000 Kbit, which is not the default
bandwidth. Serial0/1 is set to the default bandwidth of 1544 Kbit.

Let's set the bandwidth back to the default on the s0/0 interface and we
should start load-balancing to the 192.168.10.0 network. I'll just use
the \texttt{no\ bandwidth} command, which will set it back to its
default of 1544 Mbps:

\begin{verbatim}
Corp#config t
Corp(config)#int s0/0
Corp(config-if)#no bandwidth
Corp(config-if)#^Z
\end{verbatim}

Now let's take a look at the topology table and see if we're equal.

\begin{verbatim}
Corp#sh ip eigrp topo | section 192.168.10.0
P 192.168.10.0/24, 2 successors, FD is 2172416
        via 172.16.10.2 (2172416/28160), Serial0/0
        via 172.16.10.6 (2172416/28160), Serial0/1
\end{verbatim}

\protect\hypertarget{c17.xhtmlux5cux23Page_703}{}{}Since the topology
tables can get really huge in most networks, the
\texttt{show\ ip\ eigrp\ topology\ \textbar{}\ section} \texttt{network}
command comes in handy because it allows us to see information about the
network we want to look into in a couple of lines.

Let's use the \texttt{show\ ip\ route} \texttt{network} command and
check out what is going on there:

\begin{verbatim}
Corp#sh ip route 192.168.10.0
Routing entry for 192.168.10.0/24
  Known via "eigrp 20", distance 90, metric 2172416, type internal
  Redistributing via eigrp 20
  Last update from 172.16.10.2 on Serial0/0, 00:05:18 ago
  Routing Descriptor Blocks:
  * 172.16.10.6, from 172.16.10.6, 00:05:18 ago, via Serial0/1
      Route metric is 2172416, traffic share count is 1
      Total delay is 20100 microseconds, minimum bandwidth is 1544 Kbit
      Reliability 255/255, minimum MTU 1500 bytes
      Loading 1/255, Hops 1
    172.16.10.2, from 172.16.10.2, 00:05:18 ago, via Serial0/0
      Route metric is 2172416, traffic share count is 1
      Total delay is 20100 microseconds, minimum bandwidth is 1544 Kbit
      Reliability 255/255, minimum MTU 1500 bytes
      Loading 1/255, Hops 1
\end{verbatim}

Lots of detail about our routes to the 192.168.10.0 network! The Corp
route has two equal-cost links to the 192.168.10.0 network. And to
reveal load balancing even better, we'll just use the plain, ever useful
\texttt{show\ ip\ route} command:

\begin{verbatim}
Corp#sh ip route
[output cut]
D    192.168.10.0/24 [90/2172416] via 172.16.10.6, 00:05:35, Serial0/1
                     [90/2172416] via 172.16.10.2, 00:05:35, Serial0/0
\end{verbatim}

Now we can see that there are two successor routes to the 192.168.10.0
network. Pretty sweet! But in the routing table, there's one path to
192.168.20.0 and 192.168.30.0, with the link between the SF and NY
routers being feasible successors. And it's the same with the
192.168.40.0 and 192.168.50.0 networks. Let's take a look at the
topology table to examine this more closely:

\begin{verbatim}
Corp#sh ip eigrp topology
IP-EIGRP Topology Table for AS(20)/ID(10.10.11.1)
 
Codes: P - Passive, A - Active, U - Update, Q - Query, R - Reply,
       r - reply Status, s - sia Status
 
P 10.10.10.0/24, 1 successors, FD is 128256
        via Connected, GigabitEthernet0/0
P 10.10.11.0/24, 1 successors, FD is 128256
        via Connected, GigabitEthernet0/1
P 10.10.20.0/24, 1 successors, FD is 2297856
        via 172.16.10.2 (2297856/128256), Serial0/0
        via 172.16.10.6 (2300416/156160), Serial0/1
P 10.10.30.0/24, 1 successors, FD is 2297856
        via 172.16.10.2 (2297856/128256), Serial0/0
        via 172.16.10.6 (2300416/156160), Serial0/1
P 10.10.40.0/24, 1 successors, FD is 2297856
        via 172.16.10.6 (2297856/128256), Serial0/1
        via 172.16.10.2 (2300416/156160), Serial0/0
P 10.10.50.0/24, 1 successors, FD is 2297856
        via 172.16.10.6 (2297856/128256), Serial0/1
        via 172.16.10.2 (2300416/156160), Serial0/0
P 192.168.10.0/24, 2 successors, FD is 2172416
        via 172.16.10.2 (2172416/28160), Serial0/0
        via 172.16.10.6 (2172416/28160), Serial0/1
P 172.16.10.4/30, 1 successors, FD is 2169856
        via Connected, Serial0/1
P 172.16.10.0/30, 1 successors, FD is 2169856
        via Connected, Serial0/0
\end{verbatim}

It is nice that we can see that we have a successor and a feasible
successor to each network, so we know that EIGRP is doing its job. Let's
take a close look at the links to 10.10.20.0 now and dissect what it's
telling us:

\begin{verbatim}
P 10.10.20.0/24, 1 successors, FD is 2297856
        via 172.16.10.2 (2297856/128256), Serial0/0
        via 172.16.10.6 (2300416/156160), Serial0/1
\end{verbatim}

Okay---first, we can see that it's passive (P), which means that it has
found all the usable paths to the network 10.10.20.0 and is happy! If we
see active (A), that means that EIGRP is not happy at all and is
querying its neighbors for a new path to that network. The
\texttt{(2297856/128256)} is the FD/AD, meaning that the SF router is
advertising the 10.10.20.0 network as a cost of 128256, which is the AD.
The Corp router adds the bandwidth and delay of the line to get to the
SF router and then adds that number to the AD (128256) to come up with a
total cost (FD) of 2297856 to get to network 10.10.20.0.

\begin{center}\rule{0.5\linewidth}{0.5pt}\end{center}

\includegraphics{images/worning.png}To become a CCNA R/S, you must
understand how to read a topology table!

\begin{center}\rule{0.5\linewidth}{0.5pt}\end{center}

\paragraph[Unequal-Cost Load
Balancing]{\texorpdfstring{\protect\hypertarget{c17.xhtmlux5cux23Page_705}{}{}Unequal-Cost
Load Balancing}{Unequal-Cost Load Balancing}}

As with all routing protocols running on Cisco routers, EIGRP
automatically supports load balancing over four equal-cost routes and
can be configured to support up to 32 equal-cost paths with IOS 15.0
code. As you know, previous IOS versions supported up to 16. I've
mentioned this a few times in this chapter already, but I want to show
you how to configure unequal-cost load balancing with EIGRP. First let's
take a look at the Corp router by typing in the
\texttt{show\ ip\ protocols} command:

\begin{verbatim}
Corp#sh ip protocols
Routing Protocol is "eigrp 20"
  Outgoing update filter list for all interfaces is not set
  Incoming update filter list for all interfaces is not set
  Default networks flagged in outgoing updates
  Default networks accepted from incoming updates
  EIGRP metric weight K1=1, K2=0, K3=1, K4=0, K5=0
  EIGRP maximum hopcount 100
  EIGRP maximum metric variance 1
  Redistributing: eigrp 20
  EIGRP NSF-aware route hold timer is 240s
  Automatic network summarization is not in effect
  Maximum path: 4
  Routing for Networks:
    10.0.0.0
    172.16.0.0
  Routing Information Sources:
    Gateway         Distance      Last Update
    (this router)         90      19:15:10
    172.16.10.6           90      00:25:38
    172.16.10.2           90      00:25:38
  Distance: internal 90 external 170
\end{verbatim}

The \texttt{variance\ 1} means equal-path load balancing with the
maximum paths set to 4 by default. Unlike most other protocols, EIGRP
also supports unequal-cost load balancing through the use of the
\texttt{variance} parameter.

To clarify, let's say the parameter has been set to a variance of 2.
This would effectively load-balance traffic across the best route plus
any route with a feasible distance of up to twice as large. But still
keep in mind that load balancing occurs in proportion with and relative
to the cost of the route, meaning that more traffic would travel across
the best route than the suboptimal one.

\protect\hypertarget{c17.xhtmlux5cux23Page_706}{}{}Let's configure the
variance on the Corp router and see if we can load-balance across our
feasible successors now:

\begin{verbatim}
Corp# config t
Corp(config)#router eigrp 20
Corp(config-router)#variance 2
Corp(config-router)#
*Feb 26 22:24:24:IP-EIGRP(Default-IP-Routing-Table:20):route installed for 10.10.20.0
*Feb 26 22:24:24:IP-EIGRP(Default-IP-Routing-Table:20):route installed for 10.10.20.0
*Feb 26 22:24:24:IP-EIGRP(Default-IP-Routing-Table:20):route installed for 10.10.30.0
*Feb 26 22:24:24:IP-EIGRP(Default-IP-Routing-Table:20):route installed for 10.10.30.0
*Feb 26 22:24:24:IP-EIGRP(Default-IP-Routing-Table:20):route installed for 10.10.40.0
*Feb 26 22:24:24:IP-EIGRP(Default-IP-Routing-Table:20):route installed for 10.10.40.0
*Feb 26 22:24:24:IP-EIGRP(Default-IP-Routing-Table:20):route installed for 10.10.50.0
*Feb 26 22:24:24:IP-EIGRP(Default-IP-Routing-Table:20):route installed for 10.10.50.0
*Feb 26 22:24:24:IP-EIGRP(Default-IP-Routing-Table:20):route installed for 192.168.10.0
*Feb 26 22:24:24:IP-EIGRP(Default-IP-Routing-Table:20):route installed for 192.168.10.0
Corp(config-router)#do show ip route
[output cut]
D    192.168.10.0/24 [90/2172416] via 172.16.10.6, 00:00:18, Serial0/1
                     [90/2172416] via 172.16.10.2, 00:00:18, Serial0/0
     172.16.0.0/30 is subnetted, 2 subnets
C       172.16.10.4 is directly connected, Serial0/1
C       172.16.10.0 is directly connected, Serial0/0
     10.0.0.0/24 is subnetted, 6 subnets
C       10.10.10.0 is directly connected, GigabitEthernet0/0
C       10.10.11.0 is directly connected, GigabitEthernet0/1
D       10.10.20.0 [90/2300416] via 172.16.10.6, 00:00:18, Serial0/1
                   [90/2297856] via 172.16.10.2, 00:00:19, Serial0/0
D       10.10.30.0 [90/2300416] via 172.16.10.6, 00:00:19, Serial0/1
                   [90/2297856] via 172.16.10.2, 00:00:19, Serial0/0
D       10.10.40.0 [90/2297856] via 172.16.10.6, 00:00:19, Serial0/1
                   [90/2300416] via 172.16.10.2, 00:00:19, Serial0/0
D       10.10.50.0 [90/2297856] via 172.16.10.6, 00:00:20, Serial0/1
                   [90/2300416] via 172.16.10.2, 00:00:20, Serial0/0
Corp(config-router)#
\end{verbatim}

Nice---it worked! Now we have two paths to each remote network in the
routing table, even though the feasible distances to each route aren't
equal. Don't forget that unequal load balancing is not enabled by
default and that you can perform load balancing through paths that have
up to 128 times worse metrics than the successor route!

\subsubsection[Split
Horizon]{\texorpdfstring{\protect\hypertarget{c17.xhtmlux5cux23c17-sec-9}{}{}Split
Horizon}{Split Horizon}}

Split horizon is enabled on interfaces by default, which means that if a
route update is received on an interface from a neighbor router, this
interface will not advertise those networks back out to the neighbor
router who sent them. Let's take a look at an interface and then go
through an example:

\begin{verbatim}
Corp#sh ip int s0/0
Serial0/0 is up, line protocol is up
  Internet address is 172.16.10.1/24
  Broadcast address is 255.255.255.255
  Address determined by setup command
  MTU is 1500 bytes
  Helper address is not set
  Directed broadcast forwarding is disabled
  Multicast reserved groups joined: 224.0.0.10
  Outgoing access list is not set
  Inbound  access list is not set
  Proxy ARP is enabled
  Local Proxy ARP is disabled
  Security level is default
  Split horizon is enabled
[output cut]
\end{verbatim}

Okay---we can see that split horizon is enabled by default. But what
does this really mean? Most of the time it's more helpful than harmful,
but let's check out our internetwork in
\protect\hyperlink{c17.xhtmlux5cux23figure17-8}{Figure 17.8} so I can
really explain what split horizon is doing.

Notice that the SF and NY routers are each advertising their routes to
the Corp router. Now, let's see what the Corp router sends back to each
router in \protect\hyperlink{c17.xhtmlux5cux23figure17-9}{Figure 17.9}.

Can you see that the Corp router is not advertising back out the
advertised networks that it received on each interface? This is saving
the SF and NY routers from receiving the incorrect route information
that they could possibly get to their own network through the Corp
router, which we know is wrong.

\protect\hypertarget{c17.xhtmlux5cux23Page_708}{}{}

\begin{figure}
\centering
\includegraphics{images/c17f008.jpg}
\caption{{\protect\hyperlink{c17.xhtmlux5cux23figureanchor17-8}{\textbf{FIGURE
17.8}} Split horizon in action, part 1}}
\end{figure}

\begin{figure}
\centering
\includegraphics{images/c17f009.jpg}
\caption{{\protect\hyperlink{c17.xhtmlux5cux23figureanchor17-9}{\textbf{FIGURE
17.9}} Split horizon in action, part 2}}
\end{figure}

\protect\hypertarget{c17.xhtmlux5cux23Page_709}{}{}So how can this cause
a problem? After all, it seems reasonable not to send misinformation
back to an originating router, right? You'll see this create a problem
on point-to-multipoint links, such as Frame Relay, when multiple remote
routers connect to a single interface at the Corp location. We can use
logical interfaces, called subinterfaces, which I'll tell you all about
in Chapter 21, ``Wide Area Networks,'' to solve the split horizon issue
on a point-to-multipoint interface.

\subsection[Verifying and Troubleshooting
EIGRP]{\texorpdfstring{\protect\hypertarget{c17.xhtmlux5cux23c17-sec-10}{}{}Verifying
and Troubleshooting EIGRP}{Verifying and Troubleshooting EIGRP}}

Even though EIGRP usually runs smoothly and is relatively low
maintenance, there are several commands you need to memorize for using
on a router that can be super helpful when troubleshooting EIGRP! I've
already shown you a few of them, but I'm going to demonstrate all the
tools you'll need to verify and troubleshoot EIGRP now.
\protect\hyperlink{c17.xhtmlux5cux23table17-2}{Table 17.2} contains all
of the commands you need to know for verifying that EIGRP is functioning
well and offers a brief description of what each command does.

{\protect\hyperlink{c17.xhtmlux5cux23tableanchor17-2}{\textbf{TABLE~17.2}}
EIGRP troubleshooting commands}

\begin{longtable}[]{@{}ll@{}}
\toprule
Command & Description/Function\tabularnewline
\midrule
\endhead
\texttt{show\ ip\ eigrp\ neighbors} & Shows all EIGRP neighbors, their
IP addresses, and the retransmit interval and queue counts for the
adjacent routers\tabularnewline
\texttt{show\ ip\ eigrp\ interfaces} & Lists the interfaces on which the
router has actually enabled EIGRP\tabularnewline
\texttt{show\ ip\ route\ eigrp} & Shows EIGRP entries in the routing
table\tabularnewline
\texttt{show\ ip\ eigrp\ topology} & Shows entries in the EIGRP topology
table\tabularnewline
\texttt{show\ ip\ eigrp\ traffic} & Shows the packet count for EIGRP
packets sent and received\tabularnewline
\texttt{show\ ip\ protocols} & Shows information about the active
protocol sessions\tabularnewline
\bottomrule
\end{longtable}

When troubleshooting an EIGRP problem, it's always a good idea to start
by getting an accurate map of the network, and the best way to do that
is by using the \texttt{show\ ip\ eigrp\ neighbors} command to find out
who your directly connected neighbors are. This command shows all
adjacent routers that share route information within a given AS. If
neighbors are missing, check the configuration, AS number, and link
status on both routers to verify that the protocol has been configured
correctly.

\protect\hypertarget{c17.xhtmlux5cux23Page_710}{}{}Let's execute the
command on the Corp router:

\begin{verbatim}
Corp#sh ip eigrp neighbors
IP-EIGRP neighbors for process 20
H   Address                 Interface       Hold Uptime   SRTT   RTO  Q  Seq
                                            (sec)         (ms)       Cnt Num
1   172.16.10.2             Se0/0             11 03:54:25    1   200  0  127
0   172.16.10.6             Se0/1             11 04:14:47    1   200  0  2010
\end{verbatim}

Here's a breakdown of the important information we can see in the
preceding output:

\begin{enumerate}
\tightlist
\item
  H indicates the order in which the neighbor was discovered.
\item
  Hold time in seconds is how long this router will wait for a Hello
  packet to arrive from a specific neighbor.
\item
  The Uptime value indicates how long the neighbor relationship has been
  established.
\item
  The SRTT field is the smooth round-trip timer and represents how long
  it takes to complete a round-trip from this router to its neighbor and
  back. This value delimits how long to wait after a multicast for a
  reply from this neighbor. As mentioned earlier, the router will
  attempt to establish communication via unicasts if it doesn't receive
  a reply.
\item
  The time between multicast attempts is specified by the Retransmission
  Time Out (RTO) field, which is based upon the SRTT values.
\item
  The Q value tells us if there are any outstanding messages in the
  queue. We can make a mental note that there's a problem if we see
  consistently large values here!
\item
  Finally, the Seq field shows the sequence number of the last update
  from that neighbor, which is used to maintain synchronization and
  avoid duplicate messages or their out-of-sequence processing.
\end{enumerate}

The \texttt{neighbors} command is a great command, but we can get local
status of our router by also using the
\texttt{show\ ip\ eigrp\ interface} command like this:

\begin{verbatim}
Corp#sh ip eigrp interfaces
IP-EIGRP interfaces for process 20
 

                        Xmit Queue   Mean   Pacing Time   Multicast    Pending
Interface        Peers  Un/Reliable  SRTT   Un/Reliable   Flow Timer   Routes
Gi0/0              0        0/0         0       0/1            0           0
Se0/1              1        0/0         1       0/15          50           0
Se0/0              1        0/0         1       0/15          50           0
Gi0/1              0        0/0         0       0/1            0           0
 
Corp#sh ip eigrp interface detail s0/0
IP-EIGRP interfaces for process 20
 
                        Xmit Queue   Mean   Pacing Time   Multicast    Pending
Interface        Peers  Un/Reliable  SRTT   Un/Reliable   Flow Timer   Routes
Se0/0              1        0/0         1       0/15          50           0
  Hello interval is 5 sec
  Next xmit serial <none>
  Un/reliable mcasts: 0/0  Un/reliable ucasts: 21/26
  Mcast exceptions: 0  CR packets: 0  ACKs suppressed: 9
  Retransmissions sent: 0  Out-of-sequence rcvd: 0
  Authentication mode is not set
\end{verbatim}

The first command, \texttt{show\ ip\ eigrp\ interfaces}, lists all
interfaces for which EIGRP is enabled as well as those the router is
currently sending Hello messages to in an attempt to find new EIGRP
neighbors. The \texttt{show\ ip\ eigrp\ interface\ detail\ interface}
command lists more details per interface, including the local router's
own Hello interval. Understand that you can use these commands to verify
that all your interfaces are within the AS process used by EIGRP, but
also note that the passive interfaces won't show up in these outputs. So
be sure to also check to see if an interface has been configured as
passive if is not present in the outputs.

Okay, if all neighbors are present, then verify the routes learned. By
executing the \texttt{show\ ip\ route\ eigrp} command, you're given a
quick picture of the routes in the routing table. If a certain route
doesn't appear in the routing table, you need to verify its source. If
the source is functioning properly, then check the topology table.

The routing table according to Corp looks like this:

\begin{verbatim}
D    192.168.10.0/24 [90/2172416] via 172.16.10.6, 02:29:09, Serial0/1
                     [90/2172416] via 172.16.10.2, 02:29:09, Serial0/0
     172.16.0.0/30 is subnetted, 2 subnets
C       172.16.10.4 is directly connected, Serial0/1
C       172.16.10.0 is directly connected, Serial0/0
     10.0.0.0/24 is subnetted, 6 subnets
C       10.10.10.0 is directly connected, Loopback0
C       10.10.11.0 is directly connected, Loopback1
D       10.10.20.0 [90/2300416] via 172.16.10.6, 02:29:09, Serial0/1
                   [90/2297856] via 172.16.10.2, 02:29:10, Serial0/0
D       10.10.30.0 [90/2300416] via 172.16.10.6, 02:29:10, Serial0/1
                   [90/2297856] via 172.16.10.2, 02:29:10, Serial0/0
D       10.10.40.0 [90/2297856] via 172.16.10.6, 02:29:10, Serial0/1
                   [90/2300416] via 172.16.10.2, 02:29:10, Serial0/0
D       10.10.50.0 [90/2297856] via 172.16.10.6, 02:29:11, Serial0/1
                   [90/2300416] via 172.16.10.2, 02:29:11, Serial0/0
\end{verbatim}

You can see here that most EIGRP routes are referenced with a D and that
their administrative distance is 90. Remember that the
\texttt{{[}90/2300416{]}} represents AD/FD, and in the preceding output,
EIGRP is performing equal- and unequal-cost load balancing between two
links to our remote networks.

\protect\hypertarget{c17.xhtmlux5cux23Page_712}{}{}We can see this by
looking closer at two different networks. Pay special attention to the
FD of each output:

\begin{verbatim}
Corp#sh ip route | section 192.168.10.0
D    192.168.10.0/24 [90/2172416] via 172.16.10.6, 01:15:44, Serial0/1
                     [90/2172416] via 172.16.10.2, 01:15:44, Serial0/0
\end{verbatim}

The preceding output shows equal-cost load balancing, and here's our
unequal-cost load balancing in action:

\begin{verbatim}
Corp#sh ip route | section 10.10.50.0
D       10.10.50.0 [90/2297856] via 172.16.10.6, 01:16:16, Serial0/1
                   [90/2300416] via 172.16.10.2, 01:16:16, Serial0/0
\end{verbatim}

We can get the topology table displayed for us via the
\texttt{show\ ip\ eigrp\ topology} command. If the route is in the
topology table but not in the routing table, it's a pretty safe
assumption that there's a problem between the topology database and the
routing table. After all, there must be a good reason the topology
database isn't adding the route into the routing table, right? We
discussed this issue in detail earlier in the chapter, and it's oh so
important!

Corp's topology table looks like this:

\begin{verbatim}
P 10.10.10.0/24, 1 successors, FD is 128256
        via Connected, GigabitEthernet0/0
P 10.10.11.0/24, 1 successors, FD is 128256
        via Connected, GigabitEthernet0/1
P 10.10.20.0/24, 1 successors, FD is 2297856
        via 172.16.10.2 (2297856/128256), Serial0/0
        via 172.16.10.6 (2300416/156160), Serial0/1
P 10.10.30.0/24, 1 successors, FD is 2297856
        via 172.16.10.2 (2297856/128256), Serial0/0
        via 172.16.10.6 (2300416/156160), Serial0/1
P 10.10.40.0/24, 1 successors, FD is 2297856
        via 172.16.10.6 (2297856/128256), Serial0/1
        via 172.16.10.2 (2300416/156160), Serial0/0
P 10.10.50.0/24, 1 successors, FD is 2297856
        via 172.16.10.6 (2297856/128256), Serial0/1
        via 172.16.10.2 (2300416/156160), Serial0/0
P 192.168.10.0/24, 2 successors, FD is 2172416
        via 172.16.10.2 (2172416/28160), Serial0/0
        via 172.16.10.6 (2172416/28160), Serial0/1
P 172.16.10.4/30, 1 successors, FD is 2169856
        via Connected, Serial0/1
P 172.16.10.0/30, 1 successors, FD is 2169856
        via Connected, Serial0/0
\end{verbatim}

\protect\hypertarget{c17.xhtmlux5cux23Page_713}{}{}Notice that every
route in this output is preceded by a P, which shows that these routes
are in a \emph{passive state}. This is good because routes in the active
state indicate that the router has lost its path to this network and is
searching for a replacement. Each entry also reveals the feasible
distance, or FD, to each remote network as well as the next-hop neighbor
through which packets will travel to this destination. Each entry also
has two numbers in brackets, with the first indicating the feasible
distance and the second, the advertised distance to a remote network.

Again, here's our equal- and unequal-cost load-balancing output shown in
the topology table:

\begin{verbatim}
Corp#sh ip eigrp top | section 192.168.10.0
P 192.168.10.0/24, 2 successors, FD is 2172416
        via 172.16.10.2 (2172416/28160), Serial0/0
        via 172.16.10.6 (2172416/28160), Serial0/1
\end{verbatim}

The preceding output shows equal-cost load balancing, and here is our
unequal-cost load balancing in action:

\begin{verbatim}
Corp#sh ip eigrp top | section 10.10.50.0
P 10.10.50.0/24, 1 successors, FD is 2297856
        via 172.16.10.6 (2297856/128256), Serial0/1
        via 172.16.10.2 (2300416/156160), Serial0/0
\end{verbatim}

The command \texttt{show\ ip\ eigrp\ traffic} enables us to see if
updates are being sent. If the counters for EIGRP input and output
packets don't increase, it means that no EIGRP information is being sent
between peers. The following output indicates that the Corp router is
experiencing normal traffic:

\begin{verbatim}
Corp#show ip eigrp traffic
IP-EIGRP Traffic Statistics for process 200
  Hellos sent/received: 2208/2310
  Updates sent/received: 184/183
  Queries sent/received: 17/4
  Replies sent/received: 4/18
  Acks sent/received: 62/65
  Input queue high water mark 2, 0 drops
\end{verbatim}

All of the packet types I talked about in the section on RTP are
represented in the output of this command. And we can't forget the
always useful troubleshooting command \texttt{show\ ip\ protocols}.
Here's the output the Corp router gives us after using it:

\begin{verbatim}
Routing Protocol is "eigrp 20"
  Outgoing update filter list for all interfaces is not set
  Incoming update filter list for all interfaces is not set
  Default networks flagged in outgoing updates
  Default networks accepted from incoming updates
  EIGRP metric weight K1=1, K2=0, K3=1, K4=0, K5=0
  EIGRP maximum hopcount 100
  EIGRP maximum metric variance 2
  Redistributing: eigrp 20
  EIGRP NSF-aware route hold timer is 240s
  Automatic network summarization is not in effect
  Maximum path: 4
  Routing for Networks:
    10.0.0.0
    172.16.0.0
  Routing Information Sources:
    Gateway         Distance      Last Update
    (this router)         90      04:23:51
    172.16.10.6           90      02:30:48
    172.16.10.2           90      02:30:48
  Distance: internal 90 external 170
\end{verbatim}

In this output, we can see that EIGRP is enabled for autonomous system
20 and that the K values are set to their defaults. The variance is 2,
so both equal- and unequal-cost load balancing is happening here.
Automatic summarization has been turned off. We can also see that EIGRP
is advertising two classful networks and that it sees two neighbors.

The \texttt{show\ ip\ eigrp\ events} command displays a log of every
EIGRP event: when routes are injected and removed from the routing table
and when EIGRP adjacencies are reset or fail. This information is so
helpful in determining if there are routing instabilities in the
network! Be advised that this command can result in quite a flood of
information even for really simple configurations like ours. To
demonstrate, here's the output the Corp router divulged after I used it:

\begin{verbatim}
Corp#show ip eigrp events
Event information for AS 20:
1    22:24:24.258 Metric set: 172.16.10.0/30 2169856
2    22:24:24.258 FC sat rdbmet/succmet: 2169856 0
3    22:24:24.258 FC sat nh/ndbmet: 0.0.0.0 2169856
4    22:24:24.258 Find FS: 172.16.10.0/30 2169856
5    22:24:24.258 Metric set: 172.16.10.4/30 2169856
6    22:24:24.258 FC sat rdbmet/succmet: 2169856 0
7    22:24:24.258 FC sat nh/ndbmet: 0.0.0.0 2169856
8    22:24:24.258 Find FS: 172.16.10.4/30 2169856
9    22:24:24.258 Metric set: 192.168.10.0/24 2172416
10   22:24:24.258 Route install: 192.168.10.0/24 172.16.10.2
11   22:24:24.258 Route install: 192.168.10.0/24 172.16.10.6
12   22:24:24.254 FC sat rdbmet/succmet: 2172416 28160
13   22:24:24.254 FC sat nh/ndbmet: 172.16.10.6 2172416
14   22:24:24.254 Find FS: 192.168.10.0/24 2172416
15   22:24:24.254 Metric set: 10.10.50.0/24 2297856
16   22:24:24.254 Route install: 10.10.50.0/24 172.16.10.6
17   22:24:24.254 FC sat rdbmet/succmet: 2297856 128256
18   22:24:24.254 FC sat nh/ndbmet: 172.16.10.6 2297856
19   22:24:24.254 Find FS: 10.10.50.0/24 2297856
20   22:24:24.254 Metric set: 10.10.40.0/24 2297856
21   22:24:24.254 Route install: 10.10.40.0/24 172.16.10.6
22   22:24:24.250 FC sat rdbmet/succmet: 2297856 128256
 --More--
\end{verbatim}

\subsubsection[Troubleshooting Example with
EIGRP]{\texorpdfstring{\protect\hypertarget{c17.xhtmlux5cux23c17-sec-11}{}{}Troubleshooting
Example with EIGRP}{Troubleshooting Example with EIGRP}}

Throughout this chapter I've covered many of the problems that commonly
occur with EIGRP and how to verify and troubleshoot these issues. Make
sure you clearly understand what I have shown you so far in this chapter
so you're prepared to answer any question the Cisco exam could possibly
throw at you!

Just to make sure you're solidly armed with all the skills you need to
ace the exam as well as successfully administer a network, I'm going to
provide even more examples about verifying EIGRP. We'll be dealing with
mostly the same commands and problems we've already covered, but this is
so important, and the best way to get this all nailed down is to
practice troubleshooting an EIGRP network as much as possible!

With that, after you've configured EIGRP, you would first test
connectivity to the remote network by using the Ping program. If that
fails, you need to check whether the directly connected router is in the
neighbor table.

Here are some key things to look for if neighbors haven't formed an
adjacency:

\begin{enumerate}
\tightlist
\item
  Interfaces between the devices are down.
\item
  The two routers have mismatching EIGRP autonomous system numbers.
\item
  Proper interfaces are not enabled for the EIGRP process.
\item
  An interface is configured as passive.
\item
  The K values are mismatched.
\item
  EIGRP authentication is misconfigured.
\end{enumerate}

Also, if the adjacency is up but you're not receiving remote network
updates, there may be a routing problem, likely caused by these issues:

\begin{enumerate}
\tightlist
\item
  The proper networks aren't being advertised under the EIGRP process.
\item
  An access list is blocking the advertisements from remote networks.
\item
  Automatic summary is causing confusion in your discontiguous network.
\end{enumerate}

Let's use \protect\hyperlink{c17.xhtmlux5cux23figure17-10}{Figure 17.10}
as our example network and run through some troubleshooting scenarios.
I've preconfigured the routers with IP addresses, and without having to
try too hard, I also snuck in a few snags for us to find and fix. Let's
see what we're facing.

\protect\hypertarget{c17.xhtmlux5cux23Page_716}{}{}

\begin{figure}
\centering
\includegraphics{images/c17f010.jpg}
\caption{{\protect\hyperlink{c17.xhtmlux5cux23figureanchor17-10}{\textbf{FIGURE
17.10}} Troubleshooting scenario}}
\end{figure}

A good place to start is by checking to see if we have an adjacency with
\texttt{show\ ip\ eigrp\ neighbors} and
\texttt{show\ ip\ eigrp\ interfaces}. It's also smart to see what
information the \texttt{show\ ip\ eigrp\ topology} command reveals:

\begin{verbatim}
Corp#sh ip eigrp neighbors
IP-EIGRP neighbors for process 20
Corp#
 
Corp#sh ip eigrp interfaces
IP-EIGRP interfaces for process 20
 
                        Xmit Queue   Mean   Pacing Time   Multicast    Pending
Interface        Peers  Un/Reliable  SRTT   Un/Reliable   Flow Timer   Routes
Se0/1              0        0/0         0       0/15          50           0
Fa0/0              0        0/0         0       0/1            0           0
Se0/0              0        0/0         0       0/15          50           0
 
Corp#sh ip eigrp top
IP-EIGRP Topology Table for AS(20)/ID(10.10.11.1)
 
Codes: P - Passive, A - Active, U - Update, Q - Query, R - Reply,
       r - reply Status, s - sia Status
 
P 10.1.1.0/24, 1 successors, FD is 28160
        via Connected, FastEthernet0/0
\end{verbatim}

Alright---we can see by looking at the neighbor and the interface as
well as the topology table command that our LAN is up on the Corp router
but the serial link isn't working between routers because we don't have
an adjacency. From the\texttt{show\ ip\ eigrp\ interfaces} command, we
can establish that EIGRP is running on all interfaces, so that means our
network statements under the EIGRP process are probably correct, but
we'll verify that later.

Let's move on by checking into our Physical and Data Link status with
the \texttt{show\ ip\ int\ brief} command because maybe there's a
physical problem between routers:

\begin{verbatim}
Corp#sh ip int brief
Interface                IP-Address    OK? Method Status             Protocol

FastEthernet0/0          10.1.1.1      YES manual up                    up
Serial0/0                192.168.1.1   YES manual up                    up
FastEthernet0/1          unassigned    YES manual administratively down down
Serial0/1                172.16.10.5   YES manual administratively down down
Corp#
Corp#sh protocols s0/0
Serial0/0 is up, line protocol is up
  Internet address is 192.168.1.1/30
\end{verbatim}

Well, since the Serial0/0 interface shows the correct IP address and the
status is up/up, it means we have a good Data Link connection between
routers, so it's not a physical link issue between the routers, which is
good! Notice I also used the \texttt{show\ protocols} command, which
gave me the subnet mask for the link. Remember, the information obtained
via the two commands gives us only layer 1 and layer 2 status and
doesn't mean we can ping across the link. In other words, we might have
a layer 3 issue, so let's check the Branch router with the same
commands:

\begin{verbatim}
Branch#sh ip int brief
Interface                 IP-Address    OK? Method Status             Protocol
FastEthernet0/0           10.2.2.2      YES manual up                    up
FastEthernet0/1           unassigned    YES manual administratively down down
Serial0/0/0               192.168.1.2   YES manual up                    up
Serial0/0/1               unassigned    YES unset  administratively down down
Branch#sh proto s0/0/0
Serial0/0/0 is up, line protocol is up
  Internet address is 192.168.1.2/30
\end{verbatim}

Okay, well, we can see that our IP address and mask are correct, and
that the link shows up/up, so we're looking pretty good! Let's try to
ping from the Corp router to the Branch router now:

\begin{verbatim}
Corp#ping 192.168.1.2
 
Type escape sequence to abort.
Sending 5, 100-byte ICMP Echos to 192.168.1.2, timeout is 2 seconds:
!!!!!
Success rate is 100 percent (5/5), round-trip min/avg/max = 1/3/4 ms
\end{verbatim}

Now because that was successful, we've ruled out layer 1, 2, or 3 issues
between routers at this point! Since everything seems to be working
between the routers, except
\protect\hypertarget{c17.xhtmlux5cux23Page_718}{}{}EIGRP, checking our
EIGRP configurations is our next move. Let's start with the
\texttt{show\ ip} \texttt{protocols} command:

\begin{verbatim}
Corp#sh ip protocols
Routing Protocol is "eigrp 20"
  Outgoing update filter list for all interfaces is not set
  Incoming update filter list for all interfaces is not set
  Default networks flagged in outgoing updates
  Default networks accepted from incoming updates
  EIGRP metric weight K1=1, K2=0, K3=1, K4=0, K5=0
  EIGRP maximum hopcount 100
  EIGRP maximum metric variance 2
  Redistributing: eigrp 20
  EIGRP NSF-aware route hold timer is 240s
  Automatic network summarization is in effect
  Maximum path: 4
  Routing for Networks:
    10.0.0.0
    172.16.0.0
    192.168.1.0
Passive Interface(s):
    FastEthernet0/1
  Routing Information Sources:
    Gateway         Distance      Last Update
    (this router)         90      20:51:48
    192.168.1.2           90      00:22:58
    172.16.10.6           90      01:58:46
    172.16.10.2           90      01:59:52
  Distance: internal 90 external 170
\end{verbatim}

This output shows us we're using AS 20, that we don't have an
access-list filter list set on the routing tables, and that our K values
are set to default. We can see that we're routing for the 10.0.0.0,
172.16.0.0, and 192.168.1.0 networks and that we have a passive
interface on interface FastEthernet0/1. We don't have an interface
configured for the 172.16.0.0 network, which means that this entry is an
extra network statement under EIGRP. But that won't hurt anything, so
this is not causing our issue. Last, the passive interface is not
causing a problem with this network either, because we're not using
interface Fa0/1. Still, keep in mind that when troubleshooting, it's
always good to see if there are any interfaces set to passive.

Let's see what the \texttt{show\ interfaces} command will tell us:

\begin{verbatim}
Corp#sh interfaces s0/0
Serial0/0 is up, line protocol is up
  Hardware is PowerQUICC Serial
  Description: <<Connection to Branch>>
  Internet address is 192.168.1.1/30
  MTU 1500 bytes, BW 1544 Kbit, DLY 20000 usec,
     reliability 255/255, txload 1/255, rxload 1/255
  Encapsulation HDLC, loopback not set
[output cut]
\end{verbatim}

Looks like our statistics are set to defaults, so nothing really pops as
a problem here. But remember when I covered the steps to check if there
is no adjacency back at the beginning of this section? In case you
forgot, here's a list of things to investigate:

\begin{enumerate}
\tightlist
\item
  The interface between the devices are down.
\item
  The two routers have mismatching EIGRP autonomous system numbers.
\item
  The proper interfaces aren't enabled for the EIGRP process.
\item
  An interface is configured as passive.
\item
  K values are mismatched.
\item
  EIGRP authentication is misconfigured.
\end{enumerate}

Okay, our interfaces are not down, our AS number matches, layer 3 is
working between routers, all the interfaces show up under the EIGRP
process, and none of our needed interfaces are passive, so now we'll
have to look even deeper into the EIGRP configuration to uncover the
problem.

Since the Corp router has the basic default configurations, we need to
check the Branch router's EIGRP configuration:

\begin{verbatim}
Branch#sh ip protocols
Routing Protocol is "eigrp 20"
  Outgoing update filter list for all interfaces is 10
  Incoming update filter list for all interfaces is not set
  Default networks flagged in outgoing updates
  Default networks accepted from incoming updates
  EIGRP metric weight K1=1, K2=0, K3=0, K4=0, K5=0
  EIGRP maximum hopcount 100
  EIGRP maximum metric variance 1
  Redistributing: eigrp 20
  EIGRP NSF-aware route hold timer is 240s
  Automatic network summarization is not in effect
  Maximum path: 4
  Routing for Networks:
    10.0.0.0
    192.168.1.0
  Routing Information Sources:
    Gateway         Distance      Last Update
    192.168.1.1           90      00:27:09
  Distance: internal 90 external 170
\end{verbatim}

\protect\hypertarget{c17.xhtmlux5cux23Page_720}{}{}This router has the
correct AS---always check this first---and we're routing for the correct
networks. But I see two possible snags here, do you? First, the outgoing
ACL filter list is set, but the metrics are not set to default.
Remember, just because an ACL is set doesn't mean it's automatically
giving you grief. Second, the K values must match, and we know these
values are not matching the Corp router!

Let's take a look at the Branch interface statistics to see what else
might be wrong:

\begin{verbatim}
Branch>sh int s0/0/0
Serial0/0/0 is up, line protocol is up
  Hardware is GT96K Serial
  Internet address is 192.168.1.2/30
  MTU 1500 bytes, BW 512 Kbit, DLY 30000 usec,
     reliability 255/255, txload 1/255, rxload 1/255
  Encapsulation HDLC, loopback not set
  [output cut]
\end{verbatim}

Aha! The bandwidth and delay are not set to their defaults and don't
match the directly connected Corp router. Let's start by changing those
back to the default and see if that fixes our problem:

\begin{verbatim}
Branch#config t
Branch(config)#int s0/0/0
Branch(config-if)#no bandwidth
Branch(config-if)#no delay
\end{verbatim}

And let's check out our stats now to see if we're back to defaults:

\begin{verbatim}
Branch#sh int s0/0/0
Serial0/0/0 is up, line protocol is up
  Hardware is GT96K Serial
  Internet address is 192.168.1.2/30
  MTU 1500 bytes, BW 1544 Kbit, DLY 20000 usec,
     reliability 255/255, txload 1/255, rxload 1/255
  Encapsulation HDLC, loopback not set
[output cut]
\end{verbatim}

The bandwidth and delay are now at the defaults, so let's check our
adjacencies next:

\begin{verbatim}
Corp#sh ip eigrp neighbors
IP-EIGRP neighbors for process 20
\end{verbatim}

Okay, so it wasn't the bandwidth and delay settings because our
adjacency didn't come up, so let's set our K values back to default like
this:

\begin{verbatim}
Branch#config t
Branch(config)#router eigrp 20
Branch(config-router)#metric weights 0 1 0 1 0 0
Branch(config-router)#do sho ip proto
Routing Protocol is "eigrp 20"
  Outgoing update filter list for all interfaces is 10
  Incoming update filter list for all interfaces is not set
  Default networks flagged in outgoing updates
  Default networks accepted from incoming updates
  EIGRP metric weight K1=1, K2=0, K3=1, K4=0, K5=0
[output cut]
\end{verbatim}

I know this probably seems a little complicated at first, but it's
something you shouldn't have to do much, if ever. Remember, there are
five K values, so why 6 numbers? The first number listed is type of
service (ToS), so always just set that to 0, which means you must type
in six numbers as shown in my configuration example. After we chose the
default of 0 first, the default K values are then 1 0 1 0 0, which is
bandwidth and delay enabled. Let's check our adjacency now:

\begin{verbatim}
Corp#sh ip eigrp neighbors
IP-EIGRP neighbors for process 20
H   Address                 Interface       Hold Uptime   SRTT   RTO  Q  Seq
                                            (sec)         (ms)       Cnt Num
0   192.168.1.2             Se0/0             14 00:02:09    7   200  0  18
\end{verbatim}

Bam! There we go! Looks like mismatched K values were our problem. Now
let's just check to make sure we can ping from end to end and we're
done:

\begin{verbatim}
Corp#ping 10.2.2.2
 
Type escape sequence to abort.
Sending 5, 100-byte ICMP Echos to 10.2.2.2, timeout is 2 seconds:
.....
Success rate is 0 percent (0/5)
Corp#
\end{verbatim}

Rats! It looks like even though we have our adjacency, we still can't
reach our remote network. Next step? Let's see what the routing table
shows us:

\begin{verbatim}
Corp#sh ip route
[output cut]
 
     10.0.0.0/8 is variably subnetted, 2 subnets, 2 masks
C       10.1.1.0/24 is directly connected, FastEthernet0/0
D       10.0.0.0/8 is a summary, 00:18:55, Null0
     192.168.1.0/24 is variably subnetted, 2 subnets, 2 masks
C       192.168.1.0/30 is directly connected, Serial0/0
D       192.168.1.0/24 is a summary, 00:18:55, Null0
\end{verbatim}

\protect\hypertarget{c17.xhtmlux5cux23Page_722}{}{}The problem is
screamingly clear now because I went through this in detail throughout
this chapter. But just in case you still can't find it, let's look at
the \texttt{show\ ip\ protocols} command output:

\begin{verbatim}
Routing Protocol is "eigrp 20"
  Outgoing update filter list for all interfaces is not set
  Incoming update filter list for all interfaces is not set
  Default networks flagged in outgoing updates
  Default networks accepted from incoming updates
  EIGRP metric weight K1=1, K2=0, K3=1, K4=0, K5=0
  EIGRP maximum hopcount 100
  EIGRP maximum metric variance 2
  Redistributing: eigrp 20
  EIGRP NSF-aware route hold timer is 240s
  Automatic network summarization is in effect
  Automatic address summarization:
    192.168.1.0/24 for FastEthernet0/0
      Summarizing with metric 2169856
    10.0.0.0/8 for Serial0/0
      Summarizing with metric 28160
 [output cut]
\end{verbatim}

By looking at the
\protect\hyperlink{c17.xhtmlux5cux23figure17-10}{Figure 17.10}, you
should have noticed right away that we had a discontiguous network. This
means that unless they are running 15.0 IOS code, the routers will
auto-summarize, so we need to disable auto-summary:

\begin{verbatim}
Branch(config)#router eigrp 20
Branch(config-router)#no auto-summary
008412:%DUAL-5-NBRCHANGE:IP-EIGRP(0) 20:Neighbor 192.168.1.1 (Serial0/0/0) is resync:
peer graceful-restart
 
Corp(config)#router eigrp 20
Corp(config-router)#no auto-summary
Corp(config-router)#
*Feb 27 19:52:54:%DUAL-5-NBRCHANGE: IP-EIGRP(0) 20:Neighbor 192.168.1.2 (Serial0/0)
 is resync: summary configured
*Feb 27 19:52:54.177:IP-EIGRP(Default-IP-Routing-Table:20):10.1.1.0/24 - do advertise
 out Serial0/0
*Feb 27 19:52:54:IP-EIGRP(Default-IP-Routing-Table:20):Int 10.1.1.0/24 metric 2816
0 - 25600 2560
*Feb 27 19:52:54:IP-EIGRP(Default-IP-Routing-Table:20):192.168.1.0/30 - do advertise out Serial0/0
*Feb 27 19:52:54:IP-EIGRP(Default-IP-Routing-Table:20):192.168.1.0/24 - do advertise out Serial0/0
*Feb 27 19:52:54:IP-EIGRP(Default-IP-Routing-Table:20):Int 192.168.1.0/24 metric 4294967295 - 0 4294967295
*Feb 27 19:52:54:IP-EIGRP(Default-IP-Routing-Table:20):10.0.0.0/8 - do advertise
 out Serial0/0
Corp(config-router)#
*Feb 27 19:52:54:IP-EIGRP(Default-IP-Routing-Table:20):Int 10.0.0.0/8 metric 4294967295 - 0 4294967295
*Feb 27 19:52:54:IP-EIGRP(Default-IP-Routing-Table:20):Processing incoming REPLY packet
*Feb 27 19:52:54:IP-EIGRP(Default-IP-Routing-Table:20):Int 192.168.1.0/24 M 4294967295 - 1657856 4294967295 SM 4294967295 - 1657856 4294967295
*Feb 27 19:52:54:IP-EIGRP(Default-IP-Routing-Table:20):Int 10.0.0.0/8 M 4294967295 - 25600 4294967295 SM 4294967295 - 25600 4294967295
*Feb 27 19:52:54:IP-EIGRP(Default-IP-Routing-Table:20):Processing incoming UPDATE packet
\end{verbatim}

Finally the Corp looks happy, so it looks like we're good to go! Let's
just check our routing table to be sure:

\begin{verbatim}
Corp#sh ip route
[output cut]
     10.0.0.0/24 is subnetted, 1 subnets
C       10.1.1.0 is directly connected, FastEthernet0/0
     192.168.1.0/30 is subnetted, 1 subnets
C       192.168.1.0 is directly connected, Serial0/0
\end{verbatim}

What the heck? How can this be! We saw all those updates on the Corp
console, right? Let's check the configuration of EIGRP by looking at the
active configuration on the Branch router:

\begin{verbatim}
Branch#sh run
[output cut]
!
router eigrp 20
 network 10.0.0.0
 network 192.168.1.0
 distribute-list 10 out
 no auto-summary
!
\end{verbatim}

\protect\hypertarget{c17.xhtmlux5cux23Page_724}{}{}We can see that the
access list is set outbound on the routing table of the Branch router.
This may be preventing us from receiving the updates from remote
networks! Let's see what the ACL 10 list is doing:

\begin{verbatim}
Branch#sh access-lists
Standard IP access list 10
    10 deny   any (40 matches)
    20 permit any
\end{verbatim}

Now who in the world would stick an access list like this on a router?
This ACL says to deny every packet, which makes the second line of the
ACL irrelevant since every single packet will match the first line! This
has got to be the source of our troubles, so let's remove that list and
see if the Corp router starts working:

\begin{verbatim}
Branch#config t
Branch(config)#router eigrp 20
Branch(config-router)#no distribute-list 10 out
\end{verbatim}

Okay, with that ugly thing gone, let's check to see if we're receiving
our remote networks now:

\begin{verbatim}
Corp#sh ip route
[output cut]
     10.0.0.0/24 is subnetted, 2 subnets
D       10.2.2.0 [90/2172416] via 192.168.1.2, 00:00:24, Serial0/0
C       10.1.1.0 is directly connected, FastEthernet0/0
     192.168.1.0/30 is subnetted, 1 subnets
C       192.168.1.0 is directly connected, Serial0/0
Corp#
Corp#ping 10.2.2.2
 
Type escape sequence to abort.
Sending 5, 100-byte ICMP Echos to 10.2.2.2, timeout is 2 seconds:
!!!!!
Success rate is 100 percent (5/5), round-trip min/avg/max = 1/3/4 ms
Corp#
\end{verbatim}

Clear skies! We're up and running. We had mismatched K values,
discontiguous networking, and a nasty ACL on our routing table. For the
CCNA R/S objectives, always check for an ACL on the actual interface as
well, not just in the routing table. It could be set on the interface or
routing table, either one, or both! And never forget to check for
passive interfaces when troubleshooting a routing protocol issue!

All of these commands are seriously powerful tools in the hands of a
savvy professional faced with the task of troubleshooting myriad network
issues. I could go on and on about the profusion of information these
commands can generate and how well they can equip
\protect\hypertarget{c17.xhtmlux5cux23Page_725}{}{}us to solve virtually
every networking ill, but that would be way outside the scope of this
book. Even so, I have no doubt that the foundation I've given you here
will prove practical and valuable for certification purposes as well as
for working in the real networking world.

\subsubsection[Simple Troubleshooting EIGRP for the
CCNA]{\texorpdfstring{\protect\hypertarget{c17.xhtmlux5cux23c17-sec-12}{}{}Simple
Troubleshooting EIGRP for the
CCNA}{Simple Troubleshooting EIGRP for the CCNA}}

Let's do one more troubleshooting scenario. You have two routers not
forming an adjacency. What would you do first? Well, we went through a
lot in this chapter, but let me make it super easy for you when you're
troubleshooting on the CCNA exam.

All you need to do is perform a \texttt{show\ running-config} on each
router. That's it! I can then fix anything regarding EIGRP. Remember
that dynamic routing is all about the router you are looking at---it's
not important to be looking at another router's configuration to get
EIGRP correct on the router you're configuring as long as you know your
AS number.

Let's look at each router's configuration and determine what the problem
is---no network figure needed here because this is all about the router
you're looking at.

Here is the first router's configuration:

\begin{verbatim}
R1#sh run
Building configuration...
 
Current configuration : 737 bytes
!
version 15.1
!
interface Loopback0
 ip address 10.1.1.1 255.255.255.255
int FastEthernet0/0
 ip address 192.168.16.1 255.255.255.0
int Serial1/1
 ip address 192.168.13.1 255.255.255.0
 bandwidth 1000
int Serial1/3
 ip address 192.168.12.1 255.255.255.0
!
router eigrp 1
 network 192.168.12.0
 network 192.168.13.0
 network 192.168.16.0
\end{verbatim}

Here is the neighbor router's configuration:

\begin{verbatim}
R2#sh run
Building configuration...
 
Current configuration : 737 bytes
!
version 15.1
!
interface Loopback0
 ip address 10.2.2.2 255.255.255.255
interface Loopback1
 ip address 10.5.5.5 255.255.255.255
interface Loopback2
 ip address 10.5.5.55 255.255.255.255
int FastEthernet0/0
 ip address 192.168.123.2 255.255.255.0
int Serial2/1
 ip address 192.168.12.2 255.255.255.0
!
router eigrp 2
 network 10.2.2.2 0.0.0.0
 network 192.168.12.0
 network 192.168.123.0
\end{verbatim}

Can you see the problems? Pretty simple. First, notice that we're
running 15.1 codeso we don't need to worry about discontiguous networks
or need to configure the\texttt{no\ auto-summary} command. One thing
down!

Now, let's look at each interface and either remember or write down the
network numbers under each interface, including the loopback interfaces.
Once we do that we can then make sure our EIGRP configuration is
correct.

Here is the new configuration for R1:

\begin{verbatim}
R1#config t
R1(config)#router eigrp 1
R1(config-router)#network 10.1.1.1 0.0.0.0
\end{verbatim}

That's it! I just added the missing network statement from the loopback0
interface under the EIGRP process; all the other networks were already
under the EIGRP process. We're golden on R1. Let's fix R2 now:

\begin{verbatim}
R2#config t
R2(config)#no router eigrp 2
R2(config)#router eigrp 1
R2(config-router)#network 10.2.2.2 0.0.0
R2(config-router)#network 10.5.5.5 0.0.0.0
R2(config-router)#network 10.5.5.55 0.0.0.0
R2(config-router)#network 192.168.123.0
R2(config-router)#network 192.168.12.0
\end{verbatim}

\protect\hypertarget{c17.xhtmlux5cux23Page_727}{}{}Notice I started by
deleting the wrong AS number---they have to match! I then created
another EIGRP process using AS 1 and then added all the networks found
under every interface, including the loopback interfaces.

It's that easy! Just perform a \texttt{Show\ running-config} on each
router, add any missing networks found under each interface to the EIGRP
process, make sure the AS numbers match, and you're set!

Now it's time to relax a bit as we move into the easiest part of this
chapter, seriously---not joking! You still need to pay attention though.

\subsection[EIGRPv6]{\texorpdfstring{\protect\hypertarget{c17.xhtmlux5cux23c17-sec-13}{}{}EIGRPv6}{EIGRPv6}}

As I was just saying, welcome to the easiest part of the chapter! Of
course, I only mostly mean that, and here's why: I talked about IPv6 in
the earlier ICND1 chapters, and in order to continue on with this
section of the chapter, you need to have that vital, foundational part
of IPv6 down solidly before you dare to dwell here! If you do, you're
pretty much set and this will all be pretty simple for you.

EIGRPv6 works much the same way as its IPv4 predecessor does---most of
the features that EIGRP provided before EIGRPv6 will still be available.

EIGRPv6 is still an advanced distance-vector protocol that has some
link-state features. The neighbor discovery process using Hellos still
happens, and it still provides reliable communication with Reliable
Transport Protocol that gives us loop-free fast convergence using the
Diffusing Update Algorithm (DUAL).

Hello packets and updates are sent using multicast transmission, and as
with RIPng, EIGRPv6's multicast address stayed almost the same. In IPv4
it was 224.0.0.10; in IPv6, it's FF02::A (A = 10 in hexadecimal
notation).

But clearly, there are key differences between the two versions. Most
notably the use of the pesky \texttt{network} command is gone, so it's
hard to make a mistake with EIGRPv6. Also, the network and interface to
be advertised must be enabled from interface configuration mode with one
simple command.

But you still have to use the router configuration mode to enable the
routing protocol in EIGRPv6 because the routing process must be
literally enabled like an interface with the \texttt{no\ shutdown}
command---interesting! However, the 15.0 code does enable this by
default, so this command actually may or may not be needed.

Here's an example of enabling EIGRPv6 on the Corp router:

\begin{verbatim}
Corp(config)#ipv6 unicast-routing
Corp(config)#ipv6 router eigrp 10
\end{verbatim}

The 10 in this case is still the AS number. The prompt changes to
\texttt{(config-rtr)}, and from here, just initiate a
\texttt{no\ shutdown} if needed:

\begin{verbatim}
Corp(config-rtr)#no shutdown
\end{verbatim}

\protect\hypertarget{c17.xhtmlux5cux23Page_728}{}{}Other options also
can be configured in this mode, like redistribution and router ID (RID).
So now, let's go to the interface and enable IPv6:

\begin{verbatim}
Corp(config-if)#ipv6 eigrp 10
\end{verbatim}

The 10 in the interface command again references the AS number that was
enabled in the configuration mode.

\protect\hyperlink{c17.xhtmlux5cux23figure17-11}{Figure 17.11} shows the
layout we've been using throughout this chapter, only with IPv6
addresses now assigned to interfaces. I used the EUI-64 option on each
interface so each router assigned itself an IPv6 address after I typed
in the 64-bit network/subnet address.

\begin{figure}
\centering
\includegraphics{images/c17f011.jpg}
\caption{{\protect\hyperlink{c17.xhtmlux5cux23figureanchor17-11}{\textbf{FIGURE
17.11}} Configuring EIGRPv6 on our internetwork}}
\end{figure}

We'll start with the Corp router. Really, all we need to know in order
to enable EIGRPv6 are which interfaces we're using and want to advertise
our networks.

\begin{verbatim}
Corp#config t
Corp(config)#ipv6 router eigrp 10
Corp(config-rtr)#no shut
Corp(config-rtr)#router-id 1.1.1.1
Corp(config-rtr)#int s0/0/0
Corp(config-if)#ipv6 eigrp 10
Corp(config-if)#int s0/0/1
Corp(config-if)#ipv6 eigrp 10
Corp(config-if)#int g0/0
Corp(config-if)#ipv6 eigrp 10
Corp(config-if)#int g0/1
Corp(config-if)#ipv6 eigrp 10
\end{verbatim}

I had erased and reloaded the routers before I started this EIGRPv6
section of the chapter. What this means is that there were no 32-bit
addresses on the router in order to create the RID for EIGRP, so I had
to set it under the IPv6 router global command, which is the same
command used with EIGRP and EIGRPv6. Unlike with OSPF, the RID
\protect\hypertarget{c17.xhtmlux5cux23Page_729}{}{}isn't that important,
and it can actually be the same address on every router. You just can't
get away with doing this with OSPF! The configuration for EIGRPv6 was
pretty straightforward because unless you type the AS number wrong, it's
pretty hard to screw this up!

Okay, let's configure the SF and NY routers now, and then we'll verify
our networks:

\begin{verbatim}
SF#config t
SF(config)#ipv6 router eigrp 10
SF(config-rtr)#no shut
SF(config-rtr)#router-id 2.2.2.2
SF(config-rtr)#int s0/0/0
SF(config-if)#ipv6 eigrp 10
SF(config-if)#int g0/0
SF(config-if)#ipv6 eigrp 10
SF(config-if)#int g0/1
SF(config-if)#ipv6 eigrp 10
 
NY#config t
NY(config)#ipv6 router eigrp 10
NY(config-rtr)#no shut
NY(config-rtr)#router-id 3.3.3.3
NY(config-rtr)#int s0/0/0
NY(config-if)#ipv6 eigrp 10
NY(config-if)#int g0/0
NY(config-if)#ipv6 eigrp 10
NY(config-if)#int g0/1
\end{verbatim}

Since we configured EIGRPv6 on a per-interface basis, no worries about
having to use the \texttt{passive-interface} command. This is because if
we don't enable the routing protocol on an interface, it's just not part
of the EIGRPv6 process. We can see which interfaces are part of the
EIGRPv6 process with the \texttt{show\ ipv6\ eigrp\ interfaces} command
like this:

\begin{verbatim}
Corp#sh ipv6 eigrp interfaces
IPv6-EIGRP interfaces for process 10
                        Xmit Queue   Mean   Pacing Time   Multicast    Pending
Interface        Peers  Un/Reliable  SRTT   Un/Reliable   Flow Timer   Routes
Se0/0/0            1        0/0      1236       0/10           0           0
Se0/0/1            1        0/0      1236       0/10           0           0
Gig0/1             0        0/0      1236       0/10           0           0
Gig0/0             0        0/0      1236       0/10           0           0
Corp#
\end{verbatim}

\protect\hypertarget{c17.xhtmlux5cux23Page_730}{}{}Looks great so
far---all the interfaces we want in our AS are listed, so we're looking
good for our Corp's local configuration. Now it's time to check if our
adjacencies came up with the \texttt{show\ ipv6\ eigrp\ neighbors}
command:

\begin{verbatim}
Corp#sh ipv6 eigrp neighbors
IPv6-EIGRP neighbors for process 10
H   Address                 Interface     Hold   Uptime   SRTT   RTO  Q  Seq
                                          (sec)           (ms)       Cnt Num
0   Link-local address:       Se0/0/0      10   00:01:40  40     1000  0   11
    FE80::201:C9FF:FED0:3301
1   Link-local address:       Se0/0/1      14   00:01:24  40     1000  0   11
    FE80::209:7CFF:FE51:B401
\end{verbatim}

It's great that we can see neighbors listed off of each serial
interface, but do you notice something missing from the preceding
output? That's right, the actual IPv6 network/subnet addresses of the
links aren't listed in the neighbor table! Only the link-local addresses
are used for forming EIGRP neighbor adjacencies. With IPv6, neighbor
interfaces and next-hop addresses are always link-local.

We can verify our configuration with the \texttt{show\ ip\ protocols}
command:

\begin{verbatim}
Corp#sh ipv6 protocols
IPv6 Routing Protocol is "connected"
IPv6 Routing Protocol is "static
IPv6 Routing Protocol is "eigrp  10 "
  EIGRP metric weight K1=1, K2=0, K3=1, K4=0, K5=0
  EIGRP maximum hopcount 100
  EIGRP maximum metric variance 1
  Interfaces:
    Serial0/0/0
    Serial0/0/1
    GigabitEthernet0/0
    GigabitEthernet0/1
Redistributing: eigrp 10
  Maximum path: 16
  Distance: internal 90 external 170
\end{verbatim}

You can verify the AS number from this output, but be sure to verify
your K values, variance, and interfaces too. Remember that the AS number
and interfaces are the first factors to check when troubleshooting.

The topology table lists all feasible routes in the network, so this
output can be rather long, but let's see what this shows us:

\begin{verbatim}
Corp#sh ipv6 eigrp topology
IPv6-EIGRP Topology Table for AS 10/ID(1.1.1.1)
 
Codes: P - Passive, A - Active, U - Update, Q - Query, R - Reply,
       r - Reply status
 
P 2001:DB8:C34D:11::/64, 1 successors, FD is 2169856
         via Connected, Serial0/0/0
P 2001:DB8:C34D:12::/64, 1 successors, FD is 2169856
         via Connected, Serial0/0/1
P 2001:DB8:C34D:14::/64, 1 successors, FD is 2816
         via Connected, GigabitEthernet0/1
P 2001:DB8:C34D:13::/64, 1 successors, FD is 2816
         via Connected, GigabitEthernet0/0
P 2001:DB8:C34D:17::/64, 1 successors, FD is 2170112
         via FE80::201:C9FF:FED0:3301 (2170112/2816), Serial0/0/0
P 2001:DB8:C34D:18::/64, 1 successors, FD is 2170112
         via FE80::201:C9FF:FED0:3301 (2170112/2816), Serial0/0/0
P 2001:DB8:C34D:15::/64, 1 successors, FD is 2170112
         via FE80::209:7CFF:FE51:B401 (2170112/2816), Serial0/0/1
P 2001:DB8:C34D:16::/64, 1 successors, FD is 2170112
         via FE80::209:7CFF:FE51:B401 (2170112/2816), Serial0/0/1
\end{verbatim}

Since we only have eight networks in our internetwork, we can see all
eight networks in the topology table, which clearly is as it should be.
I've highlighted a couple of things I want to discuss, and the first is
that you need to be able to read and understand a topology table. This
includes understanding which routes are directly connected and which are
being advertised via neighbors. The \texttt{via\ Connected} shows us our
directly connected networks. The second item I want to show you is
\texttt{(2170112/2816)}, which is the FD/AD, and by the way, it's no
different than if you're working with IPv4.

So let's wrap up this chapter by taking a look at a routing table:

\begin{verbatim}
Corp#sh ipv6 route eigrp
IPv6 Routing Table - 13 entries
Codes: C - Connected, L - Local, S - Static, R - RIP, B - BGP
       U - Per-user Static route, M - MIPv6
       I1 - ISIS L1, I2 - ISIS L2, IA - ISIS interarea, IS - ISIS summary
       O - OSPF intra, OI - OSPF inter, OE1 - OSPF ext 1, OE2 - OSPF ext 2
       ON1 - OSPF NSSA ext 1, ON2 - OSPF NSSA ext 2
       D - EIGRP, EX - EIGRP external
C   2001:DB8:C34D:11::/64 [0/0]
     via ::, Serial0/0/0
L   2001:DB8:C34D:11:230:A3FF:FE36:B101/128 [0/0]
     via ::, Serial0/0/0
C   2001:DB8:C34D:12::/64 [0/0]
     via ::, Serial0/0/1
L   2001:DB8:C34D:12:230:A3FF:FE36:B102/128 [0/0]
     via ::, Serial0/0/1
C   2001:DB8:C34D:13::/64 [0/0]
     via ::, GigabitEthernet0/0
L   2001:DB8:C34D:13:2E0:F7FF:FEDA:7501/128 [0/0]
     via ::, GigabitEthernet0/0
C   2001:DB8:C34D:14::/64 [0/0]
     via ::, GigabitEthernet0/1
L   2001:DB8:C34D:14:2E0:F7FF:FEDA:7502/128 [0/0]
     via ::, GigabitEthernet0/1
D   2001:DB8:C34D:15::/64 [90/2170112]
     via FE80::209:7CFF:FE51:B401, Serial0/0/1
D   2001:DB8:C34D:16::/64 [90/2170112]
     via FE80::209:7CFF:FE51:B401, Serial0/0/1
D   2001:DB8:C34D:17::/64 [90/2170112]
     via FE80::201:C9FF:FED0:3301, Serial0/0/0
D   2001:DB8:C34D:18::/64 [90/2170112]
     via FE80::201:C9FF:FED0:3301, Serial0/0/0
L   FF00::/8 [0/0]
     via ::, Null0
\end{verbatim}

I highlighted the EIGRPv6 injected routes that were injected into the
routing table. It's important to notice that in order for IPv6 to get to
a remote network, the router uses the next-hop link-local address. Do
you see that in the table? For example,
\texttt{via\ FE80::209:7CFF:FE51:B401,\ Serial0/0/1} is the link-local
address of the NY router.

See? I told you it was easy!

\subsection[Summary]{\texorpdfstring{\protect\hypertarget{c17.xhtmlux5cux23c17-sec-14}{}{}Summary}{Summary}}

It's true that this chapter has been pretty extensive, so let's briefly
recap what we covered in it. EIGRP, the main focus of the chapter, is a
hybrid of link-state routing and typically referred to as an advanced
distance-vector protocol. It allows for unequal-cost load balancing,
controlled routing updates, and formal neighbor adjacencies called
relationships to be formed.

EIGRP uses the capabilities of the Reliable Transport Protocol (RTP) to
communicate between neighbors and utilizes the Diffusing Update
Algorithm (DUAL) to compute the best path to each remote network.

We also covered the configuration of EIGRP and explored a number of
troubleshooting commands plus key ways and means to help solve some
common networking issues.

Moving on, EIGRP facilitates unequal-cost load balancing, controlled
routing updates, and formal neighbor adjacencies.

I also went over the configuration of EIGRP and explored a number of
troubleshooting commands as well as taking you through a highly
informative scenario that will not only
\protect\hypertarget{c17.xhtmlux5cux23Page_733}{}{}help you to ace the
exam, it will help you confront and overcome many troubleshooting issues
common to today's internetworks!

Finally, I went over the easiest topic at the end of this long chapter:
EIGRPv6. Easy to understand, configure, and verify!

\subsection[Exam
Essentials]{\texorpdfstring{\protect\hypertarget{c17.xhtmlux5cux23c17-sec-15}{}{}Exam
Essentials}{Exam Essentials}}

\textbf{Know EIGRP features.} EIGRP is a classless, advanced
distance-vector protocol that supports IP and now IPv6. EIGRP uses a
unique algorithm, called DUAL, to maintain route information and uses
RTP to communicate with other EIGRP routers reliably.

\textbf{Know how to configure EIGRP.} Be able to configure basic EIGRP.
This is configured the same as RIP with classful addresses.

\textbf{Know how to verify EIGRP operation.} Know all of the EIGRP
\texttt{show} commands and be familiar with their output and the
interpretation of the main components of their output.

\textbf{Be able to read an EIGRP topology table.} Understand which are
successors, which are feasible successors, and which routes will become
successors if the main successor fails.

\textbf{You must be able to troubleshoot EIGRP.} Go through the EIGRP
troubleshooting scenario and make sure you understand to look for the AS
number, ACLs, passive interfaces, variance, and other factors.

\textbf{Be able to read an EIGRP neighbor table.} Understand the output
of the \texttt{show\ ip\ eigrp\ neighbor} command.

\textbf{Understand how to configure EIGRPv6.} To configure EIGRPv6,
first create the autonomous system from global configuration mode and
perform a \texttt{no\ shutdown}. Then enable EIGRPv6 on each interface
individually.

\subsection[Written Lab
17]{\texorpdfstring{\protect\hypertarget{c17.xhtmlux5cux23c17-sec-16}{}{}Written
Lab 17}{Written Lab 17}}

You can find the answers to this lab in Appendix A, ``Answers to Written
Labs.''

\begin{enumerate}
\tightlist
\item
  What is the command to enable EIGRPv6 from global configuration mode?
\item
  What is the EIGRPv6 multicast address?
\item
  True/False: Each router within an EIGRP domain must use different AS
  numbers.
\item
  If you have two routers with various K values assigned, what will this
  do to the link?
\item
  What type of EIGRP interface will neither send nor receive Hello
  packets?
\item
  Which type of EIGRP route entry describes a feasible successor?
\end{enumerate}

\subsection[Hands-on
Labs]{\texorpdfstring{\protect\hypertarget{c17.xhtmlux5cux23c17-sec-17}{}{}\protect\hypertarget{c17.xhtmlux5cux23Page_734}{}{}Hands-on
Labs}{Hands-on Labs}}

In this section, you will use the following network and add EIGRP and
EIGRPv6 routing.

\begin{figure}
\centering
\includegraphics{images/c17f012.jpg}
\caption{}
\end{figure}

The first lab requires you to configure two routers for EIGRP and then
view the configuration. In the last lab, you will be asked to enable
EIGRPv6 routing on the same network. Note that the labs in this chapter
were written to be used with real equipment---real cheap equipment, that
is. I wrote these labs with the cheapest, oldest routers I had lying
around so you can see that you don't need expensive gear to get through
some of the hardest labs in this book. However, you can use the free
LammleSim IOS version simulator or Cisco's Packet Tracer to run through
these labs.

The labs in this chapter are as follows:

\begin{enumerate}
\tightlist
\item
  Lab 17.1: Configuring and Verifying EIGRP
\item
  Lab 17.2: Configuring and Verifying EIGRPv6
\end{enumerate}

\subsubsection[Hands-on Lab 17.1: Configuring and Verifying
EIGRP]{\texorpdfstring{\protect\hypertarget{c17.xhtmlux5cux23c17-sec-18}{}{}Hands-on
Lab 17.1: Configuring and Verifying
EIGRP}{Hands-on Lab 17.1: Configuring and Verifying EIGRP}}

This lab will assume you have configured the IP addresses on the
interfaces as shown in the preceding diagram.

\begin{enumerate}
\item
  Implement EIGRP on RouterA.

\begin{verbatim}
RouterA#conf t
Enter configuration commands, one per line.
  End with CNTL/Z.
RouterA(config)#router eigrp 100
RouterA(config-router)#network 192.168.1.0
RouterA(config-router)#network 10.0.0.0
RouterA(config-router)#^Z
RouterA#
\end{verbatim}
\item
  Implement EIGRP on RouterB.

\begin{verbatim}
RouterB#conf t
Enter configuration commands, one per line.
  End with CNTL/Z.
RouterB(config)#router eigrp 100
RouterB(config-router)#network 192.168.1.0
RouterA(config-router)#network 10.0.0.0
RouterB(config-router)#exit
RouterB#
\end{verbatim}
\item
  Display the topology table for RouterA.

\begin{verbatim}
RouterA#show ip eigrp topology
\end{verbatim}
\item
  Display the routing table for RouterA.

\begin{verbatim}
RouterA #show ip route
\end{verbatim}
\item
  Display the neighbor table for RouterA.

\begin{verbatim}
RouterA show ip eigrp neighbor
\end{verbatim}
\item
  Type the command on each router to fix the routing problem. You did
  see a problem, didn't you? Yes, the network is discontiguous.

\begin{verbatim}
RouterA#config t
RouterA(config)#router eigrp 100
RouterA(config-router)#no auto-summary
 
RouterB#config t
RouterA(config)#router eigrp 100
RouterA(config-router)#no auto-summary
\end{verbatim}
\item
  Verify your routes with the \texttt{show\ ip\ route} command.
\end{enumerate}

\subsubsection[Hands-on Lab 17.2: Configuring and Verifying
EIGRPv6]{\texorpdfstring{\protect\hypertarget{c17.xhtmlux5cux23c17-sec-19}{}{}Hands-on
Lab 17.2: Configuring and Verifying
EIGRPv6}{Hands-on Lab 17.2: Configuring and Verifying EIGRPv6}}

This lab will assume you configured the IPv6 address as shown in the
diagram preceding Lab 5.1.

\begin{enumerate}
\item
  Implement EIGRPv6 on RouterA with AS 100.

\begin{verbatim}
RouterA#config t
RouterA (config)#ipv6 router eigrp 100
RouterA (config-rtr)#no shut
RouterA (config-rtr)#router-id 2.2.2.2
RouterA (config-rtr)#int s0/0
RouterA (config-if)#ipv6 eigrp 100
RouterA (config-if)#int g0/0
RouterA (config-if)#ipv6 eigrp 100
\end{verbatim}
\item
  Implement EIGRP on RouterB.

\begin{verbatim}
RouterA#config t
RouterB(config)#ipv6 router eigrp 100
RouterB(config-rtr)#no shut
RouterB(config-rtr)#router-id 2.2.2.2
RouterB(config-rtr)#int s0/0
RouterB(config-if)#ipv6 eigrp 100
RouterB(config-if)#int g0/0
RouterB(config-if)#ipv6 eigrp 100
\end{verbatim}
\item
  Display the topology table RouterA.

\begin{verbatim}
RouterA#show ipv6 eigrp topology
\end{verbatim}
\item
  Display the routing table for RouterA.

\begin{verbatim}
RouterA #show ipv6 route
\end{verbatim}
\item
  Display the neighbor table for RouterA.

\begin{verbatim}
RouterA show ipv6 eigrp neighbor
\end{verbatim}
\end{enumerate}

\subsection[Review
Questions]{\texorpdfstring{\protect\hypertarget{c17.xhtmlux5cux23c17-sec-20}{}{}\protect\hypertarget{c17.xhtmlux5cux23Page_737}{}{}Review
Questions}{Review Questions}}

\begin{center}\rule{0.5\linewidth}{0.5pt}\end{center}

\includegraphics{images/note.png}The following questions are designed to
test your understanding of this chapter's material. For more information
on how to get additional questions, please see
\href{http://www.lammle.com/ccna}{www.lammle.com/ccna}.

\begin{center}\rule{0.5\linewidth}{0.5pt}\end{center}

You can find the answers to these questions in Appendix B, ``Answers to
Review Questions.''

\begin{enumerate}
\item
  There are three possible routes for a router to reach a destination
  network. The first route is from OSPF with a metric of 782. The second
  route is from RIPv2 with a metric of 4. The third is from EIGRP with a
  composite metric of 20514560. Which route will be installed by the
  router in its routing table?

  \begin{enumerate}
  \def\labelenumii{\Alph{enumii}.}
  \tightlist
  \item
    RIPv2
  \item
    EIGRP
  \item
    OSPF
  \item
    All three
  \end{enumerate}
\item
  Which EIGRP information is held in RAM and maintained through the use
  of Hello and update packets? (Choose two.)

  \begin{enumerate}
  \def\labelenumii{\Alph{enumii}.}
  \tightlist
  \item
    Neighbor table
  \item
    STP table
  \item
    Topology table
  \item
    DUAL table
  \end{enumerate}
\item
  What will be the reported distance to a downstream neighbor router for
  the 10.10.30.0 network, with the neighbor adding the cost to find the
  true FD?

\begin{verbatim}
P 10.10.30.0/24, 1 successors, FD is 2297856
        via 172.16.10.2 (2297856/128256), Serial0/0
\end{verbatim}

  \begin{enumerate}
  \def\labelenumii{\Alph{enumii}.}
  \tightlist
  \item
    Four hops
  \item
    2297856
  \item
    128256
  \item
    EIGRP doesn't use reported distances.
  \end{enumerate}
\item
  Where are EIGRP successor routes stored?

  \begin{enumerate}
  \def\labelenumii{\Alph{enumii}.}
  \tightlist
  \item
    In the routing table only
  \item
    In the neighbor table only
  \item
    In the topology table only
  \item
    In the routing table and the neighbor table
  \item
    \protect\hypertarget{c17.xhtmlux5cux23Page_738}{}{}In the routing
    table and the topology table
  \item
    In the topology table and the neighbor table
  \end{enumerate}
\item
  Which command will display all the EIGRP feasible successor routes
  known to a router?

  \begin{enumerate}
  \def\labelenumii{\Alph{enumii}.}
  \tightlist
  \item
    \texttt{show\ ip\ routes\ *}
  \item
    \texttt{show\ ip\ eigrp\ summary}
  \item
    \texttt{show\ ip\ eigrp\ topology}
  \item
    \texttt{show\ ip\ eigrp\ adjacencies}
  \item
    \texttt{show\ ip\ eigrp\ neighbors\ detail}
  \end{enumerate}
\item
  Which of the following commands are used when routing with EIGRP or
  EIGRPv6? (Choose three.)

  \begin{enumerate}
  \def\labelenumii{\Alph{enumii}.}
  \tightlist
  \item
    \texttt{network\ 10.0.0.0}
  \item
    \texttt{eigrp\ router-id}
  \item
    \texttt{variance}
  \item
    \texttt{router\ eigrp}
  \item
    \texttt{maximum-paths}
  \end{enumerate}
\item
  Serial0/0 goes down. How will EIGRP send packets to the 10.1.1.0
  network?

\begin{verbatim}
Corp#show ip eigrp topology
[output cut]
P 10.1.1.0/24, 2 successors, FD is 2681842
         via 10.1.2.2 (2681842/2169856), Serial0/0
         via 10.1.3.1 (2973467/2579243), Serial0/2
         via 10.1.3.3 (2681842/2169856), Serial0/1
\end{verbatim}

  \begin{enumerate}
  \def\labelenumii{\Alph{enumii}.}
  \tightlist
  \item
    EIGRP will put the 10.1.1.0 network into active mode.
  \item
    EIGRP will drop all packets destined for 10.1.1.0.
  \item
    EIGRP will just keep sending packets out s0/1.
  \item
    EIGRP will use s0/2 as the successor and keep routing to 10.1.1.0.
  \end{enumerate}
\item
  What command do you use to enable EIGRPv6 on an interface?

  \begin{enumerate}
  \def\labelenumii{\Alph{enumii}.}
  \tightlist
  \item
    \texttt{router\ eigrp\ as}
  \item
    \texttt{ip\ router\ eigrp\ as}
  \item
    \texttt{router\ eigrpv6\ as}
  \item
    \texttt{ipv6\ eigrp\ as}
  \end{enumerate}
\item
  \protect\hypertarget{c17.xhtmlux5cux23Page_739}{}{}What command was
  typed in to have these two paths to network 10.10.50.0 in the routing
  table?

\begin{verbatim}
D       10.10.50.0 [90/2297856] via 172.16.10.6, 00:00:20, Serial0/1
                   [90/6893568] via 172.16.10.2, 00:00:20, Serial0/0
\end{verbatim}

  \begin{enumerate}
  \def\labelenumii{\Alph{enumii}.}
  \tightlist
  \item
    \texttt{maximum-paths\ 2}
  \item
    \texttt{variance\ 2}
  \item
    \texttt{variance\ 3}
  \item
    \texttt{maximum-hops\ 2}
  \end{enumerate}
\item
  A route to network 10.10.10.0 goes down. How does EIGRP respond in the
  local routing table? (Choose two.)

  \begin{enumerate}
  \def\labelenumii{\Alph{enumii}.}
  \tightlist
  \item
    It sends a poison reverse with a maximum hop of 16.
  \item
    If there is a feasible successor, that is copied and placed into the
    routing table.
  \item
    If a feasible successor is not found, a query will be sent to all
    neighbors asking for a path to network 10.10.10.0.
  \item
    EIGRP will broadcast out all interfaces that the link to network
    10.10.10.0 is down and that it is looking for a feasible successor.
  \end{enumerate}
\item
  You need the IP address of the devices with which the router has
  established an adjacency. Also, the retransmit interval and the queue
  counts for the adjacent routers need to be checked. What command will
  display the required information?

  \begin{enumerate}
  \def\labelenumii{\Alph{enumii}.}
  \tightlist
  \item
    \texttt{show\ ip\ eigrp\ adjacency}
  \item
    \texttt{show\ ip\ eigrp\ topology}
  \item
    \texttt{show\ ip\ eigrp\ interfaces}
  \item
    \texttt{show\ ip\ eigrp\ neighbors}
  \end{enumerate}
\item
  For some reason, you cannot establish an adjacency relationship on a
  common Ethernet link between two routers. Looking at the output shown
  here, what are the causes of the problem? (Choose two.)

\begin{verbatim}
RouterA##show ip protocols
Routing Protocol is "eigrp 20"
 Outgoing update filter list for all interfaces is not set
 Incoming update filter list for all interfaces is not set
 Default networks flagged in outgoing updates
 Default networks accepted from incoming updates
 EIGRP metric weight K1=1, K2=0, K3=1, K4=0, K5=0
RouterB##show ip protocols
Routing Protocol is "eigrp 220"
 Outgoing update filter list for all interfaces is not set
 Incoming update filter list for all interfaces is not set
 Default networks flagged in outgoing updates
 Default networks accepted from incoming updates
 EIGRP metric weight K1=1, K2=1, K3=1, K4=0, K5=0
\end{verbatim}

  \begin{enumerate}
  \def\labelenumii{\Alph{enumii}.}
  \tightlist
  \item
    \protect\hypertarget{c17.xhtmlux5cux23Page_740}{}{}EIGRP is running
    on RouterA and OSPF is running on RouterB.
  \item
    There is an ACL set on the routing protocol.
  \item
    The AS numbers don't match.
  \item
    There is no default network accepted from incoming updates.
  \item
    The K values don't match.
  \item
    There is a passive interface set.
  \end{enumerate}
\item
  Which are true regarding EIGRP successor routes? (Choose two.)

  \begin{enumerate}
  \def\labelenumii{\Alph{enumii}.}
  \tightlist
  \item
    A successor route is used by EIGRP to forward traffic to a
    destination.
  \item
    Successor routes are saved in the topology table to be used if the
    primary route fails.
  \item
    Successor routes are flagged as ``active'' in the routing table.
  \item
    A successor route may be backed up by a feasible successor route.
  \item
    Successor routes are stored in the neighbor table following the
    discovery process.
  \end{enumerate}
\item
  The remote RouterB router has a directly connected network of
  10.255.255.64/27. Which two of the following EIGRP network statements
  could you use so this directly connected network will be advertised
  under the EIGRP process? (Choose two.)

  \begin{enumerate}
  \def\labelenumii{\Alph{enumii}.}
  \tightlist
  \item
    \texttt{network\ 10.255.255.64}
  \item
    \texttt{network\ 10.255.255.64\ 0.0.0.31}
  \item
    \texttt{network\ 10.255.255.64\ 0.0.0.0}
  \item
    \texttt{network\ 10.255.255.64\ 0.0.0.15}
  \end{enumerate}
\item
  RouterA and RouterB are connected via their Serial 0/0 interfaces, but
  they have not formed an adjacency. Based on the following output, what
  could be the problem?

  \begin{figure}
  \centering
  \includegraphics{images/c17f014.jpg}
  \caption{}
  \end{figure}

  \protect\hypertarget{c17.xhtmlux5cux23Page_741}{}{}

  \begin{figure}
  \centering
  \includegraphics{images/c17f015.jpg}
  \caption{}
  \end{figure}

  \begin{enumerate}
  \def\labelenumii{\Alph{enumii}.}
  \tightlist
  \item
    The metric K values don't match.
  \item
    The AS numbers don't match.
  \item
    There is a passive interface on RouterB.
  \item
    There is an ACL set on RouterA.
  \end{enumerate}
\item
  How many paths will EIGRPv6 load-balance by default?

  \begin{enumerate}
  \def\labelenumii{\Alph{enumii}.}
  \tightlist
  \item
    16
  \item
    32
  \item
    4
  \item
    None
  \end{enumerate}
\item
  What would your configurations be on RouterB based on the
  illustration? (Choose two.)

  \begin{figure}
  \centering
  \includegraphics{images/c17f016.jpg}
  \caption{}
  \end{figure}

  \begin{enumerate}
  \def\labelenumii{\Alph{enumii}.}
  \tightlist
  \item
    \texttt{(config)\#router\ eigrp\ 10}
  \item
    \texttt{(config)\#ipv6\ router\ eigrp\ 10}
  \item
    \texttt{(config)\#ipv6\ router\ 2001:db8:3c4d:15::/64}
  \item
    \texttt{(config-if)\#ip\ eigrp\ 10}
  \item
    \texttt{(config-if)\#ipv6\ eigrp\ 10}
  \item
    \texttt{(config-if)\#ipv6\ router\ eigrp\ 10}
  \end{enumerate}
\item
  \protect\hypertarget{c17.xhtmlux5cux23Page_742}{}{}RouterA has a
  feasible successor not shown in the following output. Based on what
  you can learn from the output, which one of the following will be the
  successor for 2001:db8:c34d:18::/64 if the current successor fails?

\begin{verbatim}
via FE80::201:C9FF:FED0:3301 (29110112/33316), Serial0/0/0
via FE80::209:7CFF:FE51:B401 (4470112/42216), Serial0/0/1
via FE80::209:7CFF:FE51:B401 (2170112/2816), Serial0/0/2
\end{verbatim}

  \begin{enumerate}
  \def\labelenumii{\Alph{enumii}.}
  \tightlist
  \item
    \texttt{Serial0/0/0}
  \item
    \texttt{Serial0/0/1}
  \item
    \texttt{Serial0/0/2}
  \item
    There is no feasible successor.
  \end{enumerate}
\item
  You have router output as shown in the following illustrations with
  routers running IOS 12.4. However, the two networks are not sharing
  routing table route entries. What is the problem?

  \begin{figure}
  \centering
  \includegraphics{images/c17f017.jpg}
  \caption{}
  \end{figure}

  \protect\hypertarget{c17.xhtmlux5cux23Page_743}{}{}

  \begin{figure}
  \centering
  \includegraphics{images/c17f018.jpg}
  \caption{}
  \end{figure}

  \begin{enumerate}
  \def\labelenumii{\Alph{enumii}.}
  \tightlist
  \item
    The variances don't match between routers.
  \item
    The metrics are not valid between neighbors.
  \item
    There is a discontiguous network.
  \item
    There is a passive interface on RouterB.
  \item
    An ACL is set on the router.
  \end{enumerate}
\item
  Which should you look for when troubleshooting an adjacency? (Choose
  four.)

  \begin{enumerate}
  \def\labelenumii{\Alph{enumii}.}
  \tightlist
  \item
    Verify the AS numbers.
  \item
    Verify that you have the proper interfaces enabled for EIGRP.
  \item
    Make sure there are no mismatched K values.
  \item
    Check your passive interface settings.
  \item
    Make sure your remote routers are not connected to the Internet.
  \item
    If authentication is configured, make sure all routers use different
    passwords.
  \end{enumerate}
\end{enumerate}

\protect\hypertarget{c18.xhtml}{}{}

\section[{Chapter 18}\\
{Open Shortest Path First
(OSPF)}]{\texorpdfstring{\protect\hypertarget{c18.xhtmlux5cux23c18}{}{}\protect\hypertarget{c18.xhtmlux5cux23Page_745}{}{}{Chapter
18}\\
{Open Shortest Path First
(OSPF)}}{Chapter 18 Open Shortest Path First (OSPF)}}

\begin{center}\rule{0.5\linewidth}{0.5pt}\end{center}

\subsection{THE FOLLOWING ICND1 EXAM TOPICS ARE COVERED IN THIS
CHAPTER:}

\begin{enumerate}
\tightlist
\item
  \includegraphics{images/rarr.png}\textbf{2.0 Routing Technologies}
\item
  \includegraphics{images/rarr.png}\textbf{2.2 Compare and contrast
  distance vector and link-state routing protocols}
\item
  \includegraphics{images/rarr.png}\textbf{2.3 Compare and contrast
  interior and exterior routing protocols}
\item
  \includegraphics{images/rarr.png}\textbf{2.4 Configure, verify, and
  troubleshoot single area and multiarea OSPFv2 for IPv4 (excluding
  authentication, filtering, manual summarization, redistribution, stub,
  virtual-link, and LSAs)}
\item
  \includegraphics{images/rarr.png}\textbf{2.5 Configure, verify, and
  troubleshoot single area and multiarea OSPFv3 for IPv6 (excluding
  authentication, filtering, manual summarization, redistribution, stub,
  virtual-link, and LSAs)}
\end{enumerate}

\protect\hypertarget{c18.xhtmlux5cux23Page_746}{}{}\includegraphics{images/intro.png}
Open Shortest Path First (OSPF) is by far the most popular and important
routing protocol in use today---so important, I'm devoting this entire
chapter to it! Sticking with the same approach we've adhered to
throughout this book, we'll begin with the basics by completely
familiarizing you with key OSPF terminology. Once we've covered that
thoroughly, I'll guide you through OSPF's internal operation and then
move on to tell you all about OSPF's many advantages over RIP.

This chapter is going to be more than chock full of vitally important
information and it's also going to be really exciting because together,
we'll explore some seriously critical factors and issues innate to
implementing OSPF! I'll walk you through exactly how to implement
single-area OSPF in a variety of networking environments and then
demonstrate some great techniques you'll need to verify that everything
is configured correctly and running smoothly.

\begin{center}\rule{0.5\linewidth}{0.5pt}\end{center}

\includegraphics{images/note.png}To find up-to-the-minute updates for
this chapter, please see
\href{http://www.lammle.com/ccna}{www.lammle.com/ccna} or the book's web
page at \href{http://www.sybex.com/go/ccna}{www.sybex.com/go/ccna}.

\begin{center}\rule{0.5\linewidth}{0.5pt}\end{center}

\subsection[Open Shortest Path First (OSPF)
Basics]{\texorpdfstring{\protect\hypertarget{c18.xhtmlux5cux23c18-sec-1}{}{}Open
Shortest Path First (OSPF)
Basics}{Open Shortest Path First (OSPF) Basics}}

\emph{Open Shortest Path First} is an open standard routing protocol
that's been implemented by a wide variety of network vendors, including
Cisco. And it's that open standard characteristic that's the key to
OSPF's flexibility and popularity.

Most people opt for OSPF, which works by using the Dijkstra algorithm to
initially construct a shortest path tree and follows that by populating
the routing table with the resulting best paths. EIGRP's convergence
time may be blindingly fast, but OSPF isn't that far behind, and its
quick convergence is another reason it's a favorite. Another two great
advantages OSPF offers are that it supports multiple, equal-cost routes
to the same destination and, like EIGRP, it also supports both IP and
IPv6 routed protocols.

Here's a list that summarizes some of OSPF's best features:

\begin{enumerate}
\tightlist
\item
  Allows for the creation of areas and autonomous systems
\item
  Minimizes routing update traffic
\item
  Is highly flexible, versatile, and scalable
\item
  Supports VLSM/CIDR
\item
  \protect\hypertarget{c18.xhtmlux5cux23Page_747}{}{}Offers an unlimited
  hop count
\item
  Is open standard and supports multi-vendor deployment
\end{enumerate}

Because OSPF is the first link-state routing protocol that most people
run into, it's a good idea to size it up against more traditional
distance-vector protocols like RIPv2 and RIPv1.
\protect\hyperlink{c18.xhtmlux5cux23table18-1}{Table 18.1} presents a
nice comparison of all three of these common protocols.

{\protect\hyperlink{c18.xhtmlux5cux23tableanchor18-1}{\textbf{TABLE~18.1}}
OSPF and RIP comparison}

\begin{longtable}[]{@{}llll@{}}
\toprule
Characteristic & OSPF & RIPv2 & RIPv1\tabularnewline
\midrule
\endhead
Type of protocol & Link state & Distance vector & Distance
vector\tabularnewline
Classless support & Yes & Yes & No\tabularnewline
VLSM support & Yes & Yes & No\tabularnewline
Auto-summarization & No & Yes & Yes\tabularnewline
Manual summarization & Yes & Yes & No\tabularnewline
Noncontiguous support & Yes & Yes & No\tabularnewline
Route propagation & Multicast on change & Periodic multicast & Periodic
broadcast\tabularnewline
Path metric & Bandwidth & Hops & Hops\tabularnewline
Hop count limit & None & 15 & 15\tabularnewline
Convergence & Fast & Slow & Slow\tabularnewline
Peer authentication & Yes & Yes & No\tabularnewline
Hierarchical network requirement & Yes (using areas) & No (flat only) &
No (flat only)\tabularnewline
Updates & Event triggered & Periodic & Periodic\tabularnewline
Route computation & Dijkstra & Bellman-Ford &
Bellman-Ford\tabularnewline
\bottomrule
\end{longtable}

I want you know that OSPF has many features beyond the few I've listed
in \protect\hyperlink{c18.xhtmlux5cux23table18-1}{Table 18.1}, and all
of them combine to produce a fast, scalable, robust protocol that's also
flexible enough to be actively deployed in a vast array of production
networks!

One of OSPF's most useful traits is that its design is intended to be
hierarchical in use, meaning that it allows us to subdivide the larger
internetwork into smaller internetworks
\protect\hypertarget{c18.xhtmlux5cux23Page_748}{}{}called areas. It's a
really powerful feature that I recommend using, and I promise to show
you how to do that later in the chapter.

Here are three of the biggest reasons to implement OSPF in a way that
makes full use of its intentional, hierarchical design:

\begin{enumerate}
\tightlist
\item
  To decrease routing overhead
\item
  To speed up convergence
\item
  To confine network instability to single areas of the network
\end{enumerate}

Because free lunches are invariably hard to come by, all this wonderful
functionality predictably comes at a price and doesn't exactly make
configuring OSPF any easier. But no worries---we'll crush it!

Let's start by checking out
\protect\hyperlink{c18.xhtmlux5cux23figure18-1}{Figure 18.1}, which
shows a very typical, yet simple OSPF design. I really want to point out
the fact that some routers connect to the backbone---called area 0---the
backbone area. OSPF absolutely must have an area 0, and all other areas
should connect to it except for those connected via virtual links, which
are beyond the scope of this book. A router that connects other areas to
the backbone area within an AS is called an \emph{area border router
(ABR)}, and even these must have at least one of their interfaces
connected to area 0.

\begin{figure}
\centering
\includegraphics{images/c18f001.jpg}
\caption{{\protect\hyperlink{c18.xhtmlux5cux23figureanchor18-1}{\textbf{FIGURE
18.1}} OSPF design example. An OSPF hierarchical design minimizes
routing table entries and keeps the impact of any topology changes
contained within a specific area.}}
\end{figure}

OSPF runs great inside an autonomous system, but it can also connect
multiple autonomous systems together. The router that connects these ASs
is called an \emph{autonomous system boundary router (ASBR)}. Ideally,
your aim is to create other areas of networks to help keep route updates
to a minimum, especially in larger networks. Doing this also keeps
problems from propagating throughout the network, effectively isolating
them to a single area.

But let's pause here to cover some key OSPF terms that are really
essential for you to nail down before we move on any further.

\subsubsection[OSPF
Terminology]{\texorpdfstring{\protect\hypertarget{c18.xhtmlux5cux23c18-sec-2}{}{}\protect\hypertarget{c18.xhtmlux5cux23Page_749}{}{}OSPF
Terminology}{OSPF Terminology}}

Imagine being given a map and compass with no prior concept of east,
west, north or south---not even what rivers, mountains, lakes, or
deserts are. I'm guessing that without any ability to orient yourself in
a basic way, your cool, new tools wouldn't help you get anywhere but
completely lost, right? This is exactly why we're going to begin
exploring OSPF by getting you solidly acquainted with a fairly long list
of terms before setting out from base camp into the great unknown! Here
are those vital terms to commit to memory now:

\textbf{Link} A \emph{link} is a network or router interface assigned to
any given network. When an interface is added to the OSPF process, it's
considered to be a link. This link, or interface, will have up or down
state information associated with it as well as one or more IP
addresses.

\textbf{Router ID} The \emph{router ID (RID)} is an IP address used to
identify the router. Cisco chooses the router ID by using the highest IP
address of all configured loopback interfaces. If no loopback interfaces
are configured with addresses, OSPF will choose the highest IP address
out of all active physical interfaces. To OSPF, this is basically the
``name'' of each router.

\textbf{Neighbor} \emph{Neighbors} are two or more routers that have an
interface on a common network, such as two routers connected on a
point-to-point serial link. OSPF neighbors must have a number of common
configuration options to be able to successfully establish a neighbor
relationship, and all of these options must be configured exactly the
same way:

\begin{enumerate}
\tightlist
\item
  Area ID
\item
  Stub area flag
\item
  Authentication password (if using one)
\item
  Hello and Dead intervals
\end{enumerate}

\textbf{Adjacency} An \emph{adjacency} is a relationship between two
OSPF routers that permits the direct exchange of route updates. Unlike
EIGRP, which directly shares routes with all of its neighbors, OSPF is
really picky about sharing routing information and will directly share
routes only with neighbors that have also established adjacencies. And
not all neighbors will become adjacent---this depends upon both the type
of network and the configuration of the routers. In multi-access
networks, routers form adjacencies with designated and backup designated
routers. In point-to-point and point-to-multipoint networks, routers
form adjacencies with the router on the opposite side of the connection.

\textbf{Designated router} A \emph{designated router (DR)} is elected
whenever OSPF routers are connected to the same broadcast network to
minimize the number of adjacencies formed and to publicize received
routing information to and from the remaining routers on the broadcast
network or link. Elections are won based upon a router's priority level,
with the one having the highest priority becoming the winner. If there's
a tie, the router ID will be used to break it. All routers on the shared
network will establish adjacencies with the DR and the BDR, which
ensures that all routers' topology tables are synchronized.

\protect\hypertarget{c18.xhtmlux5cux23Page_750}{}{}\textbf{Backup
designated router} A \emph{backup designated router (BDR)} is a hot
standby for the DR on broadcast, or multi-access, links. The BDR
receives all routing updates from OSPF adjacent routers but does not
disperse LSA updates.

\textbf{Hello protocol} The OSPF Hello protocol provides dynamic
neighbor discovery and maintains neighbor relationships. Hello packets
and Link State Advertisements (LSAs) build and maintain the topological
database. Hello packets are addressed to multicast address 224.0.0.5.

\textbf{Neighborship database} The \emph{neighborship database} is a
list of all OSPF routers for which Hello packets have been seen. A
variety of details, including the router ID and state, are maintained on
each router in the neighborship database.

\textbf{Topological database} The \emph{topological database} contains
information from all of the Link State Advertisement packets that have
been received for an area. The router uses the information from the
topology database as input into the Dijkstra algorithm that computes the
shortest path to every network.

\begin{center}\rule{0.5\linewidth}{0.5pt}\end{center}

\includegraphics{images/note.png}LSA packets are used to update and
maintain the topological database.

\begin{center}\rule{0.5\linewidth}{0.5pt}\end{center}

\textbf{Link State Advertisement} A \emph{Link State Advertisement
(LSA)} is an OSPF data packet containing link-state and routing
information that's shared among OSPF routers. An OSPF router will
exchange LSA packets only with routers to which it has established
adjacencies.

\textbf{OSPF areas} An \emph{OSPF area} is a grouping of contiguous
networks and routers. All routers in the same area share a common area
ID. Because a router can be a member of more than one area at a time,
the area ID is associated with specific interfaces on the router. This
would allow some interfaces to belong to area 1 while the remaining
interfaces can belong to area 0. All of the routers within the same area
have the same topology table. When configuring OSPF with multiple areas,
you've got to remember that there must be an area 0 and that this is
typically considered the backbone area. Areas also play a role in
establishing a hierarchical network organization---something that really
enhances the scalability of OSPF!

\textbf{Broadcast (multi-access)} \emph{Broadcast (multi-access)
networks} such as Ethernet allow multiple devices to connect to or
access the same network, enabling a \emph{broadcast} ability in which a
single packet is delivered to all nodes on the network. In OSPF, a DR
and BDR must be elected for each broadcast multi-access network.

\textbf{Nonbroadcast multi-access} \emph{Nonbroadcast multi-access
(NBMA)} networks are networks such as Frame Relay, X.25, and
Asynchronous Transfer Mode (ATM). These types of networks allow for
multi-access without broadcast ability like Ethernet. NBMA networks
require special OSPF configuration to function properly.

\protect\hypertarget{c18.xhtmlux5cux23Page_751}{}{}\textbf{Point-to-point}
\emph{Point-to-point} refers to a type of network topology made up of a
direct connection between two routers that provides a single
communication path. The point-to-point connection can be physical---for
example, a serial cable that directly connects two routers---or logical,
where two routers thousands of miles apart are connected by a circuit in
a Frame Relay network. Either way, point-to-point configurations
eliminate the need for DRs or BDRs.

\textbf{Point-to-multipoint} \emph{Point-to-multipoint} refers to a type
of network topology made up of a series of connections between a single
interface on one router and multiple destination routers. All interfaces
on all routers share the point-to-multipoint connection and belong to
the same network. Point-to-multipoint networks can be further classified
according to whether they support broadcasts or not. This is important
because it defines the kind of OSPF configurations you can deploy.

All of these terms play a critical role when you're trying to understand
how OSPF actually works, so again, make sure you're familiar with each
of them. Having these terms down will enable you to confidently place
them in their proper context as we progress on our journey through the
rest of this chapter!

\subsubsection[OSPF
Operation]{\texorpdfstring{\protect\hypertarget{c18.xhtmlux5cux23c18-sec-3}{}{}OSPF
Operation}{OSPF Operation}}

Fully equipped with your newly acquired knowledge of the terms and
technologies we just covered, it's now time to delve into how OSPF
discovers, propagates, and ultimately chooses routes. Once you know how
OSPF achieves these tasks, you'll understand how OSPF operates
internally really well.

OSPF operation is basically divided into these three categories:

\begin{enumerate}
\tightlist
\item
  Neighbor and adjacency initialization
\item
  LSA flooding
\item
  SPF tree calculation
\end{enumerate}

The beginning neighbor/adjacency formation stage is a very big part of
OSPF operation. When OSPF is initialized on a router, the router
allocates memory for it, as well as for the maintenance of both neighbor
and topology tables. Once the router determines which interfaces have
been configured for OSPF, it will then check to see if they're active
and begin sending Hello packets as shown in
\protect\hyperlink{c18.xhtmlux5cux23figure18-2}{Figure 18.2}.

\begin{figure}
\centering
\includegraphics{images/c18f002.jpg}
\caption{{\protect\hyperlink{c18.xhtmlux5cux23figureanchor18-2}{\textbf{FIGURE
18.2}} The Hello protocol}}
\end{figure}

The Hello protocol is used to discover neighbors, establish adjacencies,
and maintain relationships with other OSPF routers. Hello packets are
periodically sent out of each enabled OSPF interface and in environments
that support multicast.

\protect\hypertarget{c18.xhtmlux5cux23Page_752}{}{}The address used for
this is 224.0.0.5, and the frequency with which Hello packets are sent
out depends upon the network type and topology. Broadcast and
point-to-point networks send Hellos every 10 seconds, whereas
non-broadcast and point-to-multipoint networks send them every 30
seconds.

\paragraph{LSA Flooding}

\emph{LSA flooding} is the method OSPF uses to share routing
information. Via Link State Updates (LSU's) packets, LSA information
containing link-state data is shared with all OSPF routers within an
area. The network topology is created from the LSA updates, and flooding
is used so that all OSPF routers have the same topology map to make SPF
calculations with.

Efficient flooding is achieved through the use of a reserved multicast
address: 224.0.0.5 (AllSPFRouters). LSA updates, which indicate that
something in the topology has changed, are handled a bit differently.
The network type determines the multicast address used for sending
updates. \protect\hyperlink{c18.xhtmlux5cux23table18-2}{Table 18.2}
contains the multicast addresses associated with LSA flooding.
Point-to-multipoint networks use the adjacent router's unicast IP
address.

{\protect\hyperlink{c18.xhtmlux5cux23tableanchor18-2}{\textbf{TABLE~18.2}}
LSA update multicast addresses}

\begin{longtable}[]{@{}lll@{}}
\toprule
Network Type & Multicast Address & Description\tabularnewline
\midrule
\endhead
Point-to-point & 224.0.0.5 & AllSPFRouters\tabularnewline
Broadcast & 224.0.0.6 & AllDRouters\tabularnewline
Point-to-multipoint & NA & NA\tabularnewline
\bottomrule
\end{longtable}

Once the LSA updates have been flooded throughout the network, each
recipient must acknowledge that the flooded update has been received.
It's also important for recipients to validate the LSA update.

\paragraph{SPF Tree Calculation}

Within an area, each router calculates the best/shortest path to every
network in that same area. This calculation is based upon the
information collected in the topology database and an algorithm called
shortest path first (SPF). Picture each router in an area constructing a
tree---much like a family tree---where the router is the root and all
other networks are arranged along the branches and leaves. This is the
shortest-path tree used by the router to insert OSPF routes into the
routing table.

It's important to understand that this tree contains only networks that
exist in the same area as the router itself does. If a router has
interfaces in multiple areas, then separate trees will be constructed
for each area. One of the key criteria considered during the route
selection process of the SPF algorithm is the metric or cost of each
potential path to a network. But this SPF calculation doesn't apply to
routes from other areas.

\subparagraph[OSPF
Metrics]{\texorpdfstring{\protect\hypertarget{c18.xhtmlux5cux23c18-sec-4}{}{}\protect\hypertarget{c18.xhtmlux5cux23Page_753}{}{}OSPF
Metrics}{OSPF Metrics}}

OSPF uses a metric referred to as \emph{cost}. A cost is associated with
every outgoing interface included in an SPF tree. The cost of the entire
path is the sum of the costs of the outgoing interfaces along the path.
Because cost is an arbitrary value as defined in RFC 2338, Cisco had to
implement its own method of calculating the cost for each OSPF-enabled
interface. Cisco uses a simple equation of
10\textsuperscript{8}/\emph{bandwidth}, where \emph{bandwidth} is the
configured bandwidth for the interface. Using this rule, a 100 Mbps Fast
Ethernet interface would have a default OSPF cost of 1 and a 1,000 Mbps
Ethernet interface would have a cost of 1.

Important to note is that this value can be overridden with the
\texttt{ip\ ospf\ cost} command. The cost is manipulated by changing the
value to a number within the range of 1 to 65,535. Because the cost is
assigned to each link, the value must be changed on the specific
interface you want to change the cost on.

\begin{center}\rule{0.5\linewidth}{0.5pt}\end{center}

\includegraphics{images/note.png}Cisco bases link cost on bandwidth.
Other vendors may use other metrics to calculate a given link's cost.
When connecting links between routers from different vendors, you'll
probably have to adjust the cost to match another vendor's router
because both routers must assign the same cost to the link for OSPF to
work properly.

\begin{center}\rule{0.5\linewidth}{0.5pt}\end{center}

\subsection[Configuring
OSPF]{\texorpdfstring{\protect\hypertarget{c18.xhtmlux5cux23c18-sec-5}{}{}Configuring
OSPF}{Configuring OSPF}}

Configuring basic OSPF isn't as simple as configuring RIP and EIGRP, and
it can get really complex once the many options that are allowed within
OSPF are factored in. But that's okay because you really only need to
focus on basic, single-area OSPF configuration at this point. Coming up,
I'll show you how to configure single-area OSPF.

The two factors that are foundational to OSPF configuration are enabling
OSPF and configuring OSPF areas.

\subsubsection[Enabling
OSPF]{\texorpdfstring{\protect\hypertarget{c18.xhtmlux5cux23c18-sec-6}{}{}Enabling
OSPF}{Enabling OSPF}}

The easiest and also least scalable way to configure OSPF is to just use
a single area. Doing this requires a minimum of two commands.

The first command used to activate the OSPF routing process is as
follows:

\begin{verbatim}
Router(config)#router ospf ?
<1-65535> Process ID
\end{verbatim}

A value in the range from 1 to 65,535 identifies the OSPF process ID.
It's a unique number on this router that groups a series of OSPF
configuration commands under a specific running process. Different OSPF
routers don't have to use the same process ID to communicate. It's a
purely local value that doesn't mean a lot, but you still need to
remember that it cannot start at 0; it has to start at a minimum of 1.

\protect\hypertarget{c18.xhtmlux5cux23Page_754}{}{}You can have more
than one OSPF process running simultaneously on the same router if you
want, but this isn't the same as running multi-area OSPF. The second
process will maintain an entirely separate copy of its topology table
and manage its communications independently of the first one and you use
it when you want OSPF to connect multiple ASs together. Also, because
the Cisco exam objectives only cover single-area OSPF with each router
running a single OSPF process, that's what we'll focus on in this book.

\begin{center}\rule{0.5\linewidth}{0.5pt}\end{center}

\includegraphics{images/note.png}The OSPF process ID is needed to
identify a unique instance of an OSPF database and is locally
significant.

\begin{center}\rule{0.5\linewidth}{0.5pt}\end{center}

\subsubsection[Configuring OSPF
Areas]{\texorpdfstring{\protect\hypertarget{c18.xhtmlux5cux23c18-sec-7}{}{}Configuring
OSPF Areas}{Configuring OSPF Areas}}

After identifying the OSPF process, you need to identify the interfaces
that you want to activate OSPF communications on as well as the area in
which each resides. This will also configure the networks you're going
to advertise to others.

Here's an example of a basic OSPF configuration for you, showing our
second minimum command needed, the \texttt{network} command:

\begin{verbatim}
Router#config t
Router(config)#router ospf 1
Router(config-router)#network 10.0.0.0 0.255.255.255 area ?
  <0-4294967295>  OSPF area ID as a decimal value
  A.B.C.D         OSPF area ID in IP address format
Router(config-router)#network 10.0.0.0 0.255.255.255 area 0
\end{verbatim}

\begin{center}\rule{0.5\linewidth}{0.5pt}\end{center}

\includegraphics{images/note.png}The areas can be any number from 0 to
4.2 billion. Don't get these numbers confused with the process ID, which
ranges from 1 to 65,535.

\begin{center}\rule{0.5\linewidth}{0.5pt}\end{center}

Remember, the OSPF process ID number is irrelevant. It can be the same
on every router on the network, or it can be different---doesn't matter.
It's locally significant and just enables the OSPF routing on the
router.

The arguments of the \texttt{network} command are the network number
(10.0.0.0) and the wildcard mask (0.255.255.255). The combination of
these two numbers identifies the interfaces that OSPF will operate on
and will also be included in its OSPF LSA advertisements. Based on my
sample configuration, OSPF will use this command to find any interface
on the router configured in the 10.0.0.0 network and will place any
interface it finds into area 0.
\protect\hypertarget{c18.xhtmlux5cux23Page_755}{}{}Notice that you can
create about 4.2 billion areas! In reality, a router wouldn't let you
create that many, but you can certainly name them using the numbers up
to 4.2 billion. You can also label an area using an IP address format.

Let me stop here a minute to give you a quick explanation of wildcards:
A 0 octet in the wildcard mask indicates that the corresponding octet in
the network must match exactly. On the other hand, a 255 indicates that
you don't care what the corresponding octet is in the network number. A
network and wildcard mask combination of 1.1.1.1 0.0.0.0 would match an
interface configured exactly with 1.1.1.1 only, and nothing else. This
is really useful if you want to activate OSPF on a specific interface in
a very clear and simple way. If you insist on matching a range of
networks, the network and wildcard mask combination of 1.1.0.0
0.0.255.255 would match any interface in the range of 1.1.0.0 to
1.1.255.255. Because of this, it's simpler and safer to stick to using
wildcard masks of 0.0.0.0 and identify each OSPF interface individually.
Once configured, they'll function exactly the same---one way really
isn't better than the other.

The final argument is the area number. It indicates the area to which
the interfaces identified in the network and wildcard mask portion
belong. Remember that OSPF routers will become neighbors only if their
interfaces share a network that's configured to belong to the same area
number. The format of the area number is either a decimal value from the
range 0 to 4,294,967,295 or a value represented in standard
dotted-decimal notation. For example, area 0.0.0.0 is a legitimate area
and is identical to area 0.

\paragraph{Wildcard Example}

Before getting down to configuring our network, let's take a quick peek
at a more complex OSPF network configuration to find out what our OSPF
network statements would be if we were using subnets and wildcards.

In this scenario, you have a router with these four subnets connected to
four different interfaces:

\begin{enumerate}
\tightlist
\item
  192.168.10.64/28
\item
  192.168.10.80/28
\item
  192.168.10.96/28
\item
  192.168.10.8/30
\end{enumerate}

All interfaces need to be in area 0, so it seems to me the easiest
configuration would look like this:

\begin{verbatim}
Test#config t
Test(config)#router ospf 1
Test(config-router)#network 192.168.10.0 0.0.0.255 area 0
\end{verbatim}

I'll admit that the preceding example is actually pretty simple, but
easy isn't always best---especially when dealing with OSPF! So even
though this is an easy-button way to configure OSPF, it doesn't make
good use of its capabilities and what fun is that? Worse
\protect\hypertarget{c18.xhtmlux5cux23Page_756}{}{}yet, the objectives
aren't very likely to present something this simple for you! So let's
create a separate network statement for each interface using the subnet
numbers and wildcards. Doing that would look something like this:

\begin{verbatim}
Test#config t
Test(config)#router ospf 1
Test(config-router)#network 192.168.10.64 0.0.0.15 area 0
Test(config-router)#network 192.168.10.80 0.0.0.15 area 0
Test(config-router)#network 192.168.10.96 0.0.0.15 area 0
Test(config-router)#network 192.168.10.8 0.0.0.3 area 0
\end{verbatim}

Wow, now that's a different looking config! Truthfully, OSPF would work
exactly the same way as it would with the easy configuration I showed
you first---but unlike the easy configuration, this one covers the
objectives!

And although this looks a bit complicated, trust me, it really isn't.
All you need for clarity is to fully understand your block sizes! Just
remember that when configuring wildcards, they're always one less than
the block size. A /28 is a block size of 16, so we would add our network
statement using the subnet number and then add a wildcard of 15 in the
interesting octet. For the /30, which is a block size of 4, we would go
with a wildcard of 3. Once you practice this a few times, it gets really
easy. And do practice because we'll deal with them again when we get to
access lists later on!

Let's use \protect\hyperlink{c18.xhtmlux5cux23figure18-3}{Figure 18.3}
as an example and configure that network with OSPF using wildcards to
make sure you have a solid grip on this. The figure shows a three-router
network with the IP addresses of each interface.

\begin{figure}
\centering
\includegraphics{images/c18f003.jpg}
\caption{{\protect\hyperlink{c18.xhtmlux5cux23figureanchor18-3}{\textbf{FIGURE
18.3}} Sample OSPF wildcard configuration}}
\end{figure}

The very first thing you need to be able to do is to look at each
interface and determine the subnet that the addresses are in. Hold on, I
know what you're thinking: ``Why don't I just use the exact IP addresses
of the interface with the 0.0.0.0 wildcard?'' Well, you can, but we're
paying attention to Cisco exam objectives here, not just what's easiest,
remember?

\protect\hypertarget{c18.xhtmlux5cux23Page_757}{}{}The IP addresses for
each interface are shown in the figure. The Lab\_A router has two
directly connected subnets: 192.168.10.64/29 and 10.255.255.80/30.
Here's the OSPF configuration using wildcards:

\begin{verbatim}
Lab_A#config t
Lab_A(config)#router ospf 1
Lab_A(config-router)#network 192.168.10.64 0.0.0.7 area 0
Lab_A(config-router)#network 10.255.255.80 0.0.0.3 area 0
\end{verbatim}

The Lab\_A router is using a /29, or 255.255.255.248, mask on the Fa0/0
interface. This isa block size of 8, which is a wildcard of 7. The G0/0
interface is a mask of 255.255.255.252---block size of 4, with a
wildcard of 3. Notice that I typed in the network number, not the
interface number. You can't configure OSPF this way if you can't look at
the IP address and slash notation and then figure out the subnet, mask,
and wildcard, can you? So don't take your exam until you can do this.

Here are other two configurations to help you practice:

\begin{verbatim}
Lab_B#config t
Lab_B(config)#router ospf 1
Lab_B(config-router)#network 192.168.10.48 0.0.0.7 area 0
Lab_B(config-router)#network 10.255.255.80 0.0.0.3 area 0
Lab_B(config-router)#network 10.255.255.8 0.0.0.3 area 0
 
Lab_C#config t
Lab_C(config)#router ospf 1
Lab_C(config-router)#network 192.168.10.16 0.0.0.7 area 0
Lab_C(config-router)#network 10.255.255.8 0.0.0.3 area 0
\end{verbatim}

As I mentioned with the Lab\_A configuration, you've got to be able to
determine the subnet, mask, and wildcard just by looking at the IP
address and mask of an interface. If you can't do that, you won't be
able to configure OSPF using wildcards as I just demonstrated. So go
over this until you're really comfortable with it!

\subsubsection[Configuring Our Network with
OSPF]{\texorpdfstring{\protect\hypertarget{c18.xhtmlux5cux23c18-sec-8}{}{}Configuring
Our Network with OSPF}{Configuring Our Network with OSPF}}

Now we get to have some fun! Let's configure our internetwork with OSPF
using just area 0. OSPF has an administrative distance of 110, but let's
remove RIP while we're at it because I don't want you to get in the
habit of having RIP running on your network.

There's a bunch of different ways to configure OSPF, and as I said, the
simplest and easiest is to use the wildcard mask 0.0.0.0. But I want to
demonstrate that we can configure each router differently with OSPF and
still come up with the exact same result. This is one reason why OSPF is
more fun and challenging than other routing protocols---it gives us all
a lot more ways to screw things up, which automatically provides
\protect\hypertarget{c18.xhtmlux5cux23Page_758}{}{}a troubleshooting
opportunity! We'll use our network as shown in
\protect\hyperlink{c18.xhtmlux5cux23figure18-4}{Figure 18.4} to
configure OSPF, and by the way, notice I added a new router!

\begin{figure}
\centering
\includegraphics{images/c18f004.jpg}
\caption{{\protect\hyperlink{c18.xhtmlux5cux23figureanchor18-4}{\textbf{FIGURE
18.4}} Our new network layout}}
\end{figure}

\paragraph{Corp}

Here's the Corp router's configuration:

\begin{verbatim}
Corp#sh ip int brief
Interface        IP-Address      OK? Method Status                Protocol
FastEthernet0/0  10.10.10.1      YES manual up                    up
Serial0/0        172.16.10.1     YES manual up                    up
FastEthernet0/1  unassigned      YES unset  administratively down down
Serial0/1        172.16.10.5     YES manual up                    up
Corp#config t
Corp(config)#no router rip
Corp(config)#router ospf 132
Corp(config-router)#network 10.10.10.1 0.0.0.0 area 0
Corp(config-router)#network 172.16.10.1 0.0.0.0 area 0
Corp(config-router)#network 172.16.10.5 0.0.0.0 area 0
\end{verbatim}

Alright---it looks like we have a few things to talk about here. First,
I removed RIP and then added OSPF. Why did I use OSPF 132? It really
doesn't matter---the number is irrelevant. I guess it just felt good to
use 132. But notice that I started with the
\texttt{show\ ip\ int\ brief} command, just like when I was configuring
RIP. I did this because it's always
\protect\hypertarget{c18.xhtmlux5cux23Page_759}{}{}important to verify
exactly what you are directly connected to. Doing this really helps
prevent typos!

The network commands are pretty straightforward. I typed in the IP
address of each interface and used the wildcard mask of 0.0.0.0, which
means that the IP address must precisely match each octet. This is
actually one of those times where easier is better, so just do this:

\begin{verbatim}
Corp(config)#router ospf 132
Corp(config-router)#network 172.16.10.0 0.0.0.255 area 0
\end{verbatim}

Nice---there's only one line instead of two for the 172.16.10.0 network!
I really want you to understand that OSPF will work the same here no
matter which way you configure the network statement. Now, let's move on
to SF. To simplify things, we're going to use our same sample
configuration.

\paragraph{SF}

The SF router has two directly connected networks. I'll use the IP
addresses on each interface to configure this router.

\begin{verbatim}
SF#sh ip int brief
Interface    IP-Address      OK? Method Status                Protocol
FastEthernet0/0  192.168.10.1    YES manual up                    up
FastEthernet0/1  unassigned      YES unset  administratively down down
Serial0/0/0      172.16.10.2     YES manual up                    up
Serial0/0/1      unassigned      YES unset  administratively down down
SF#config t
SF(config)#no router rip
SF(config)#router ospf 300
SF(config-router)#network 192.168.10.1 0.0.0.0 area 0
SF(config-router)#network 172.16.10.2 0.0.0.0 area 0
*Apr 30 00:25:43.810: %OSPF-5-ADJCHG: Process 300, Nbr 172.16.10.5 on Serial0/0/0 from LOADING to FULL, Loading Done
\end{verbatim}

Here, all I did was to first disable RIP, turn on OSPF routing process
300, and then I added my two directly connected networks. Now let's move
on to LA!

\paragraph{LA}

We're going to give some attention to the LA router that's directly
connected to two networks:

\begin{verbatim}
LA#sh ip int brief
Interface    IP-Address      OK? Method Status                Protocol
FastEthernet0/0 192.168.20.1    YES manual up                    up
FastEthernet0/1 unassigned      YES unset  administratively down down
Serial0/0/0     unassigned      YES unset  administratively down down
Serial0/0/1     172.16.10.6     YES manual up                    up
LA#config t
LA(config)#router ospf 100
LA(config-router)#network 192.168.20.0 0.0.0.255 area 0
LA(config-router)#network 172.16.0.0 0.0.255.255 area 0
*Apr 30 00:56:37.090: %OSPF-5-ADJCHG: Process 100, Nbr 172.16.10.5 on Serial0/0/1 from LOADING to FULL, Loading Done
\end{verbatim}

Remember that when you're configuring dynamic routing, using the
\texttt{show\ ip\ int\ brief} command first will make it all so much
easier!

And don't forget, I can use any process ID I want, as long as it's a
value from 1 to 65,535, because it doesn't matter if all routers use the
same process ID. Also, notice that I used different wildcards in this
example. Doing this works really well too.

Okay, I want you to think about something for a second before we move
onto more advanced OSPF topics: What if the Fa0/1 interface of the LA
router was connected to a link that we didn't need OSPF running on, as
shown in \protect\hyperlink{c18.xhtmlux5cux23figure18-5}{Figure 18.5}?

\begin{figure}
\centering
\includegraphics{images/c18f005.jpg}
\caption{{\protect\hyperlink{c18.xhtmlux5cux23figureanchor18-5}{\textbf{FIGURE
18.5}} Adding a non-OSPF network to the LA router}}
\end{figure}

You've seen this before because I demonstrated this already back in the
RIP section. We can use the same command that we did under that routing
process here as well! Take a look:

\begin{verbatim}
LA(config)#router ospf 100
LA(config-router)#passive-interface fastEthernet 0/1
\end{verbatim}

Even though this is pretty simple, you've really got to be careful
before you configure this command on your router! I added this command
as an example on interface Fa0/1, which happens to be an interface we're
not using in this network because I want OSPF to work on my other
router's interfaces.

\protect\hypertarget{c18.xhtmlux5cux23Page_761}{}{}Now it's time to
configure our Corp router to advertise a default route to the SF and LA
routers because doing so will make our lives a lot easier. Instead of
having to configure all our routers with a default route, we'll only
configure one router and then advertise that this router is the one that
holds the default route---elegant!

In \protect\hyperlink{c18.xhtmlux5cux23figure18-4}{Figure 18.4}, keep in
mind that, for now, the corporate router is connected to the Internet
off of Fa0/0. We'll create a default route toward this imaginary
Internet and then tell the other routers that this is the route they'll
use to get to the Internet. Here is the configuration:

\begin{verbatim}
Corp#config t
Corp(config)#ip route 0.0.0.0 0.0.0.0 Fa0/0
Corp(config)#router ospf 1
Corp(config-router)#default-information originate
\end{verbatim}

Now, let's check and see if our other routers have received this default
route from the Corp router:

\begin{verbatim}
SF#show ip route
[output cut]
E1 - OSPF external type 1, E2 - OSPF external type 2
[output cut]
O*E2 0.0.0.0/0 [110/1] via 172.16.10.1, 00:01:54, Serial0/0/0
SF#
\end{verbatim}

Sure enough---the last line in the SF router shows that it received the
advertisement from the Corp router regarding the fact that the corporate
router is the one holding the default route out of the AS.

But hold on a second! I need to configure our new router into my lab to
create the example network we'll use from here on. Here's the
configuration of the new router that I connected to the same network
that the Corp router is connected to via the Fa0/0 interface:

\begin{verbatim}
Router#config t
Router(config)#hostname Boulder
Boulder(config)#int f0/0
Boulder(config-if)#ip address 10.10.10.2 255.255.255.0
Boulder(config-if)#no shut
*Apr  6 18:01:38.007: %LINEPROTO-5-UPDOWN: Line protocol on Interface FastEthernet0/0, changed state to up
Boulder(config-if)#router ospf 2
Boulder(config-router)#network 10.0.0.0 0.255.255.255 area 0
*Apr  6 18:03:27.267: %OSPF-5-ADJCHG: Process 2, Nbr 223.255.255.254 on FastEthernet0/0 from LOADING to FULL, Loading Done
\end{verbatim}

This is all good, but I need to make sure that you don't follow my
example to a tee because here, I just quickly brought a router up
without setting my passwords first. I can
\protect\hypertarget{c18.xhtmlux5cux23Page_762}{}{}get away with this
only because I am in a nonproduction network, so don't do this in the
real world where security is key!

Anyway, now that I have my new router nicely connected with a basic
configuration, we're going to move on to cover loopback interfaces, how
to set the router ID (RID) used with OSPF, and finally, how to verify
OSPF.

\subsection[OSPF and Loopback
Interfaces]{\texorpdfstring{\protect\hypertarget{c18.xhtmlux5cux23c18-sec-9}{}{}OSPF
and Loopback Interfaces}{OSPF and Loopback Interfaces}}

It's really vital to configure loopback interfaces when using OSPF. In
fact, Cisco suggests using them whenever you configure OSPF on a router
for stability purposes.

\emph{Loopback interfaces} are logical interfaces, which means they're
virtual, software-only interfaces, not actual, physical router
interfaces. A big reason we use loopback interfaces with OSPF
configurations is because they ensure that an interface is always active
and available for OSPF processes.

Loopback interfaces also come in very handy for diagnostic purposes as
well as for OSPF configuration. Understand that if you don't configure a
loopback interface on a router, the highest active IP address on a
router will become that router's RID during bootup!
\protect\hyperlink{c18.xhtmlux5cux23figure18-6}{Figure 18.6} illustrates
how routers know each other by their router ID.

\begin{figure}
\centering
\includegraphics{images/c18f006.jpg}
\caption{{\protect\hyperlink{c18.xhtmlux5cux23figureanchor18-6}{\textbf{FIGURE
18.6}} OSPF router ID (RID)}}
\end{figure}

The RID is not only used to advertise routes, it's also used to elect
the designated router (DR) and the backup designated router (BDR). These
designated routers create adjacencies when a new router comes up and
exchanges LSAs to build topological databases.

\begin{center}\rule{0.5\linewidth}{0.5pt}\end{center}

\includegraphics{images/note.png}By default, OSPF uses the highest IP
address on any active interface at the moment OSPF starts up to
determine the RID of the router. But this behavior can be overridden via
a logical interface. Remember---the highest IP address of any logical
interface will always become a router's RID!

\begin{center}\rule{0.5\linewidth}{0.5pt}\end{center}

\protect\hypertarget{c18.xhtmlux5cux23Page_763}{}{}Now it's time to show
you how to configure these logical loopback interfaces and how to verify
them, as well as verify RIDs.

\subsubsection[Configuring Loopback
Interfaces]{\texorpdfstring{\protect\hypertarget{c18.xhtmlux5cux23c18-sec-10}{}{}Configuring
Loopback Interfaces}{Configuring Loopback Interfaces}}

Configuring loopback interfaces rocks mostly because it's the easiest
part of OSPF configuration, and we all need a break about now---right?
So hang on---we're in the home stretch!

First, let's see what the RID is on the Corp router with the
\texttt{show\ ip\ ospf} command:

\begin{verbatim}
Corp#sh ip ospf
 Routing Process "ospf 1" with ID 172.16.10.5
[output cut]
\end{verbatim}

Okay, we can see that the RID is 172.16.10.5---the Serial0/1 interface
of the router. So let's configure a loopback interface using a
completely different IP addressing scheme:

\begin{verbatim}
Corp(config)#int loopback 0
*Mar 22 01:23:14.206: %LINEPROTO-5-UPDOWN: Line protocol on Interface
   Loopback0, changed state to up
Corp(config-if)#ip address 172.31.1.1 255.255.255.255
\end{verbatim}

The IP scheme really doesn't matter here, but each one being in a
separate subnet does! By using the /32 mask, we can use any IP address
we want as long as the addresses are never the same on any two routers.

Let's configure the other routers now:

\begin{verbatim}
SF#config t
SF(config)#int loopback 0
*Mar 22 01:25:11.206: %LINEPROTO-5-UPDOWN: Line protocol on Interface
   Loopback0, changed state to up
SF(config-if)#ip address 172.31.1.2 255.255.255.255
\end{verbatim}

Here's the configuration of the loopback interface on LA:

\begin{verbatim}
LA#config t
LA(config)#int loopback 0
*Mar 22 02:21:59.686: %LINEPROTO-5-UPDOWN: Line protocol on Interface
   Loopback0, changed state to up
LA(config-if)#ip address 172.31.1.3 255.255.255.255
\end{verbatim}

I'm pretty sure you're wondering what the IP address mask of
255.255.255.255 (/32) means and why we don't just use 255.255.255.0
instead. While it's true that either mask works, the /32 mask is called
a host mask and works fine for loopback interfaces. It also allows us to
save subnets. Notice how I was able to use 172.31.1.1, .2, .3, and .4?
If I didn't use the /32, I'd have to use a separate subnet for each and
every router---not good!

\protect\hypertarget{c18.xhtmlux5cux23Page_764}{}{}One important
question to answer before we move on is did we actually change the RIDs
of our router by setting the loopback interfaces? Let's find out by
taking a look at the Corp's RID:

\begin{verbatim}
Corp#sh ip ospf
 Routing Process "ospf 1" with ID 172.16.10.5
\end{verbatim}

What happened here? You would think that because we set logical
interfaces, the IP addresses under them would automatically become the
RID of the router, right? Well, sort of, but only if you do one of two
things: either reboot the router or delete OSPF and re-create the
database on your router. Neither is all that great an option, so try to
remember to create your logical interfaces before you start OSPF
routing. That way, the loopback interface would always become your RID
straight away!

With all this in mind, I'm going with rebooting the Corp router because
it's the easier of the two options I have right now.

Now let's look and see what our RID is:

\begin{verbatim}
Corp#sh ip ospf
 Routing Process "ospf 1" with ID 172.31.1.1
\end{verbatim}

That did the trick! The Corp router now has a new RID, so I guess I'll
just go ahead and reboot all my routers to get their RIDs reset to our
logical addresses. But should I really do that?

Maybe not because there is \emph{one} other way. What do you think about
adding a new RID for the router right under the \texttt{router\ ospf}
\texttt{process-id} command instead? Sounds good, so I'd say let's give
that a shot! Here's an example of doing that on the Corp router:

\begin{verbatim}
Corp#config t
Corp(config)#router ospf 1
Corp(config-router)#router-id 223.255.255.254
Reload or use "clear ip ospf process" command, for this to take effect
Corp(config-router)#do clear ip ospf process
Reset ALL OSPF processes? [no]: yes
*Jan 16 14:20:36.906: %OSPF-5-ADJCHG: Process 1, Nbr 192.168.20.1
on Serial0/1 from FULL to DOWN, Neighbor Down: Interface down
or detached
*Jan 16 14:20:36.906: %OSPF-5-ADJCHG: Process 1, Nbr 192.168.10.1
on Serial0/0 from FULL to DOWN, Neighbor Down: Interface down
or detached
*Jan 16 14:20:36.982: %OSPF-5-ADJCHG: Process 1, Nbr 192.168.20.1
on Serial0/1 from LOADING to FULL, Loading Done
*Jan 16 14:20:36.982: %OSPF-5-ADJCHG: Process 1, Nbr 192.168.10.1
on Serial0/0 from LOADING to FULL, Loading Done
Corp(config-router)#do sh ip ospf
 Routing Process "ospf 1" with ID 223.255.255.254
\end{verbatim}

\protect\hypertarget{c18.xhtmlux5cux23Page_765}{}{}Now look at that---it
worked! We changed the RID without reloading the router! But
wait---remember, we set a logical loopback interface earlier. Does that
mean the loopback interface will win over the \texttt{router-id}
command? Well, we can see our answer\ldots{} A loopback interface will
\emph{not} override the \texttt{router-id} command, and we don't have to
reboot the router to make it take effect as the RID!

So this process follows this hierarchy:

\begin{enumerate}
\tightlist
\item
  Highest active interface by default.
\item
  Highest logical interface overrides a physical interface.
\item
  The \texttt{router-id} overrides the interface and loopback interface.
\end{enumerate}

The only thing left now is to decide whether you want to advertise the
loopback interfaces under OSPF. There are pros and cons to using an
address that won't be advertised versus using an address that will be.
Using an unadvertised address saves on real IP address space, but the
address won't appear in the OSPF table, which means you can't ping it.

So basically, what you're faced with here is a choice that equals a
trade-off between the ease of debugging the network and conservation of
address space---what to do? A really tight strategy is to use a private
IP address scheme as I did. Do this and all will be well!

Now that we've configured all the routers with OSPF, what's next? Miller
time? Nope---not yet. It's that verification thing again. We still have
to make sure that OSPF is really working, and that's exactly what we're
going to do next.

\subsection[Verifying OSPF
Configuration]{\texorpdfstring{\protect\hypertarget{c18.xhtmlux5cux23c18-sec-11}{}{}Verifying
OSPF Configuration}{Verifying OSPF Configuration}}

There are several ways to verify proper OSPF configuration and
operation, so next, I'm going to demonstrate the various OSPF
\texttt{show} commands you need to know in order to achieve this. We're
going to start by taking a quick look at the routing table of the Corp
router.

First, let's issue a \texttt{show\ ip\ route} command on the Corp
router:

\begin{verbatim}
O    192.168.10.0/24 [110/65] via 172.16.10.2, 1d17h, Serial0/0
     172.131.0.0/32 is subnetted, 1 subnets
    172.131.0.0/32 is subnetted, 1 subnets
C        172.131.1.1 is directly connected, Loopback0
     172.16.0.0/30 is subnetted, 4 subnets
C       172.16.10.4 is directly connected, Serial0/1
L       172.16.10.5/32 is directly connected, Serial0/1
C       172.16.10.0 is directly connected, Serial0/0
L       172.16.10.1/32 is directly connected, Serial0/0
O    192.168.20.0/24 [110/65] via 172.16.10.6, 1d17h, Serial0/1
     10.0.0.0/24 is subnetted, 2 subnets
C       10.10.10.0 is directly connected, FastEthernet0/0
L       10.10.10.1/32 is directly connected, FastEthernet0/0
\end{verbatim}

\protect\hypertarget{c18.xhtmlux5cux23Page_766}{}{}The Corp router shows
only two dynamic routes for the internetwork, with the \emph{O}
representing OSPF internal routes. The \emph{C}s are clearly our
directly connected networks, and our two remote networks are showing up
too---nice! Notice the 110/65, which is our administrative
distance/metric.

Now that's a really sweet-looking OSPF routing table! It's important to
make it easier to troubleshoot and fix an OSPF network, which is why I
always use the \texttt{show\ ip\ int\ brief} command when configuring my
routing protocols. It's very easy to make little mistakes with OSPF, so
keep your eyes on the details!

It's time to show you all the OSPF verification commands that you need
in your toolbox for now.

\subsubsection[The \emph{show ip ospf}
Command]{\texorpdfstring{\protect\hypertarget{c18.xhtmlux5cux23c18-sec-12}{}{}The
\emph{show ip ospf} Command}{The show ip ospf Command}}

The \texttt{show\ ip\ ospf} command is what you'll need to display OSPF
information for one or all OSPF processes running on the router.
Information contained therein includes the router ID, area information,
SPF statistics, and LSA timer information. Let's check out the output
from the Corp router:

\begin{verbatim}
Corp#sh ip ospf
 Routing Process "ospf 1" with ID 223.255.255.254
 Start time: 00:08:41.724, Time elapsed: 2d16h
 Supports only single TOS(TOS0) routes
 Supports opaque LSA
 Supports Link-local Signaling (LLS)
 Supports area transit capability
 Router is not originating router-LSAs with maximum metric
 Initial SPF schedule delay 5000 msecs
 Minimum hold time between two consecutive SPFs 10000 msecs
 Maximum wait time between two consecutive SPFs 10000 msecs
 Incremental-SPF disabled
 Minimum LSA interval 5 secs
 Minimum LSA arrival 1000 msecs
 LSA group pacing timer 240 secs
 Interface flood pacing timer 33 msecs
 Retransmission pacing timer 66 msecs
 Number of external LSA 0. Checksum Sum 0x000000
 Number of opaque AS LSA 0. Checksum Sum 0x000000
 Number of DCbitless external and opaque AS LSA 0
 Number of DoNotAge external and opaque AS LSA 0
 Number of areas in this router is 1. 1 normal 0 stub 0 nssa
 Number of areas transit capable is 0
 External flood list length 0
 IETF NSF helper support enabled
 Cisco NSF helper support enabled
    Area BACKBONE(0)
        Number of interfaces in this area is 3
        Area has no authentication
        SPF algorithm last executed 00:11:08.760 ago
        SPF algorithm executed 5 times
        Area ranges are
        Number of LSA 6. Checksum Sum 0x03B054
        Number of opaque link LSA 0. Checksum Sum 0x000000
        Number of DCbitless LSA 0
        Number of indication LSA 0
        Number of DoNotAge LSA 0
        Flood list length 0
\end{verbatim}

Notice the router ID (RID) of 223.255.255.254, which is the highest IP
address configured on the router. Hopefully, you also noticed that I set
the RID of the corporate router to the highest IP address available with
IPv4.

\subsubsection[The \emph{show ip ospf database}
Command]{\texorpdfstring{\protect\hypertarget{c18.xhtmlux5cux23c18-sec-13}{}{}The
\emph{show ip ospf database}
Command}{The show ip ospf database Command}}

Using the \texttt{show\ ip\ ospf\ database} command will give you
information about the number of routers in the internetwork (AS) plus
the neighboring router's ID---the topology database I mentioned earlier.
Unlike the \texttt{show\ ip\ eigrp\ topology} command, this command
reveals the OSPF routers, but not each and every link in the AS like
EIGRP does.

The output is broken down by area. Here's a sample output, again from
Corp:

\begin{verbatim}
Corp#sh ip ospf database
 
                OSPF Router with ID (223.255.255.254) (Process ID 1)
Router Link States (Area 0)
 
Link ID         ADV Router      Age         Seq#       Checksum Link count
10.10.10.2      10.10.10.2      966         0x80000001 0x007162 1
172.31.1.4      172.31.1.4      885         0x80000002 0x00D27E 1
192.168.10.1    192.168.10.1    886         0x8000007A 0x00BC95 3
192.168.20.1    192.168.20.1    1133        0x8000007A 0x00E348 3
223.255.255.254 223.255.255.254 925         0x8000004D 0x000B90 5
 
                Net Link States (Area 0)
 
Link ID         ADV Router      Age         Seq#       Checksum
10.10.10.1      223.255.255.254 884         0x80000002 0x008CFE
\end{verbatim}

\protect\hypertarget{c18.xhtmlux5cux23Page_768}{}{}You can see all the
routers and the RID of each router---the highest IP address on each of
them. For example, the link ID and ADV router of my new Boulder router
shows up twice: once with the directly connected IP address (10.10.10.2)
and as the RID that I set under the OSPF process (172.31.1.4).

The router output shows the link ID---remember that an interface is also
a link---and the RID of the router on that link under the ADV router, or
advertising router.

\subsubsection[The \emph{show ip ospf interface}
Command]{\texorpdfstring{\protect\hypertarget{c18.xhtmlux5cux23c18-sec-14}{}{}The
\emph{show ip ospf interface}
Command}{The show ip ospf interface Command}}

The \texttt{show\ ip\ ospf\ interface} command reveals all
interface-related OSPF information. Data is displayed about OSPF
information for all OSPF-enabled interfaces or for specified interfaces.
I'll highlight some of the more important factors for you. Check it out:

\begin{verbatim}
Corp#sh ip ospf int f0/0
FastEthernet0/0 is up, line protocol is up
  Internet Address 10.10.10.1/24, Area 0
  Process ID 1, Router ID 223.255.255.254, Network Type BROADCAST, Cost: 1
  Transmit Delay is 1 sec, State DR, Priority 1
  Designated Router (ID) 223.255.255.254, Interface address 10.10.10.1
  Backup Designated router (ID) 172.31.1.4, Interface address 10.10.10.2
  Timer intervals configured, Hello 10, Dead 40, Wait 40, Retransmit 5
    oob-resync timeout 40
    Hello due in 00:00:08
  Supports Link-local Signaling (LLS)
  Cisco NSF helper support enabled
  IETF NSF helper support enabled
  Index 3/3, flood queue length 0
  Next 0x0(0)/0x0(0)
  Last flood scan length is 1, maximum is 1
  Last flood scan time is 0 msec, maximum is 0 msec
  Neighbor Count is 1, Adjacent neighbor count is 1
    Adjacent with neighbor 172.31.1.  Suppress hello for 0 neighbor(s)
\end{verbatim}

So this command has given us the following information:

\begin{enumerate}
\tightlist
\item
  Interface IP address
\item
  Area assignment
\item
  Process ID
\item
  Router ID
\item
  Network type
\item
  Cost
\item
  Priority
\item
  \protect\hypertarget{c18.xhtmlux5cux23Page_769}{}{}DR/BDR election
  information (if applicable)
\item
  Hello and Dead timer intervals
\item
  Adjacent neighbor information
\end{enumerate}

The reason I used the \texttt{show\ ip\ ospf\ interface\ f0/0} command
is because I knew that there would be a designated router elected on the
FastEthernet broadcast multi-access network between our Corp and Boulder
routers. The information that I highlighted is all very important, so
make sure you've noted it! A good question to ask you here is what are
the Hello and Dead timers set to by default?

What if you type in the \texttt{show\ ip\ ospf\ interface} command and
receive this response:

\begin{verbatim}
Corp#sh ip ospf int f0/0
%OSPF: OSPF not enabled on FastEthernet0/0
\end{verbatim}

This error occurs when OSPF is enabled on the router, but not the
interface. When this happens, you need to check your network statements
because it means that the interface you're trying to verify is not in
your OSPF process!

\subsubsection[The \emph{show ip ospf neighbor}
Command]{\texorpdfstring{\protect\hypertarget{c18.xhtmlux5cux23c18-sec-15}{}{}The
\emph{show ip ospf neighbor}
Command}{The show ip ospf neighbor Command}}

The \texttt{show\ ip\ ospf\ neighbor} command is super-useful because it
summarizes the pertinent OSPF information regarding neighbors and the
adjacency state. If a DR or BDR exists, that information will also be
displayed. Here's a sample:

\begin{verbatim}
Corp#sh ip ospf neighbor
 
Neighbor ID     Pri   State      Dead Time   Address         Interface
172.31.1.4        1   FULL/BDR   00:00:34    10.10.10.2     FastEthernet0/0
192.168.20.1      0   FULL/  -   00:00:31    172.16.10.6     Serial0/1
192.168.10.1      0   FULL/  -   00:00:32    172.16.10.2     Serial0/0
\end{verbatim}

\begin{center}\rule{0.5\linewidth}{0.5pt}\end{center}

\subsubsection[\hfill\break
An Admin Connects Two Disparate Routers Together with OSPF and the Link
between them Never Comes
Up]{\texorpdfstring{\protect\includegraphics{images/c18inline02.png}\\
An Admin Connects Two Disparate Routers Together with OSPF and the Link
between them Never Comes
Up}{ An Admin Connects Two Disparate Routers Together with OSPF and the Link between them Never Comes Up}}

Quite a few years ago, an admin called me in a panic because he couldn't
get OSPF working between two routers, one of which was an older router
that they needed to use while they were waiting for their new router to
be shipped to them.

OSPF can be used in a multi-vendor network, so he was confused as to why
this wasn't working. He turned on RIP and it worked, so he was super
confused with why OSPF was
\protect\hypertarget{c18.xhtmlux5cux23Page_770}{}{}not creating
adjacencies. I had him use the \texttt{show\ ip\ ospf\ interface}
command to look at the link between the two routers and sure enough, the
hello and dead timers didn't match. I had him configure the mismatched
parameters so they would match, but it still wouldn't create an
adjacency. Looking more closely at the
\texttt{show\ ip\ ospf\ interface} command, I noticed the cost did not
match! Cisco calculated the bandwidth differently than the other vendor.
Once I had him configure both as the same value, the link came up!
Always remember, just because OSPF can be used in a multi-vendor network
does not mean it will work out of the box!

\begin{center}\rule{0.5\linewidth}{0.5pt}\end{center}

This is a critical command to understand because it's extremely useful
in production networks. Let's take a look at the Boulder router output:

\begin{verbatim}
Boulder>sh ip ospf neighbor
 
Neighbor ID     Pri   State    Dead Time   Address         Interface
223.255.255.254   1   FULL/DR  00:00:31    10.10.10.1      FastEthernet0/0
\end{verbatim}

Here we can see that since there's an Ethernet link (broadcast
multi-access) on the link between the Boulder and the Corp router,
there's going to be an election to determine who will be the designated
router (DR) and who will be the backup designated router (BDR). We can
see that the Corp became the designated router, and it won because it
had the highest IP address on the network---the highest RID.

Now the reason that the Corp connections to SF and LA don't have a DR or
BDR listed in the output is that by default, elections don't happen on
point-to-point links and they show \texttt{FULL/\ -} . But we can still
determine that the Corp router is fully adjacent to all three routers
from its output.

\subsubsection[The \emph{show ip protocols}
Command]{\texorpdfstring{\protect\hypertarget{c18.xhtmlux5cux23c18-sec-16}{}{}The
\emph{show ip protocols} Command}{The show ip protocols Command}}

The \texttt{show\ ip\ protocols} command is also highly useful, whether
you're running OSPF, EIGRP, RIP, BGP, IS-IS, or any other routing
protocol that can be configured on your router. It provides an excellent
overview of the actual operation of all currently running protocols!

Check out the output from the Corp router:

\begin{verbatim}
Corp#sh ip protocols
Routing Protocol is "ospf 1"
  Outgoing update filter list for all interfaces is not set
  Incoming update filter list for all interfaces is not set
  Router ID 223.255.255.254
  Number of areas in this router is 1. 1 normal 0 stub 0 nssa
  Maximum path: 4
  Routing for Networks:
    10.10.10.1 0.0.0.0 area 0
    172.16.10.1 0.0.0.0 area 0
    172.16.10.5 0.0.0.0 area 0
 Reference bandwidth unit is 100 mbps
  Routing Information Sources:
    Gateway         Distance      Last Update
    192.168.10.1         110      00:21:53
    192.168.20.1         110      00:21:53
  Distance: (default is 110) Distance: (default is 110)
\end{verbatim}

From looking at this output, you can determine the OSPF process ID, OSPF
router ID, type of OSPF area, networks and areas configured for OSPF,
and the OSPF router IDs of neighbors---that's a lot. It's
super-efficient!

\subsection[Summary]{\texorpdfstring{\protect\hypertarget{c18.xhtmlux5cux23c18-sec-17}{}{}Summary}{Summary}}

This chapter gave you a great deal of information about OSPF. It's
really difficult to include everything about OSPF because so much of it
falls outside the scope of this chapter and book, but I've given you a
few tips here and there, so you're good to go---as long as you make sure
you've got what I presented to you dialed in, that is!

I talked about a lot of OSPF topics, including terminology, operations,
and configuration as well as verification and monitoring.

Each of these topics encompasses quite a bit of information---the
terminology section just scratched the surface of OSPF. But you've got
the goods you really need for your studies. Finally, I gave you a tight
survey of commands highly useful for observing the operation of OSPF so
you can verify that things are moving along as they should. So eat it
all up, and you're set!

\subsection[Exam
Essentials]{\texorpdfstring{\protect\hypertarget{c18.xhtmlux5cux23c18-sec-18}{}{}Exam
Essentials}{Exam Essentials}}

\textbf{Compare OSPF and RIPv1.} OSPF is a link-state protocol that
supports VLSM and classless routing; RIPv1 is a distance-vector protocol
that does not support VLSM and supports only classful routing.

\textbf{Know how OSPF routers become neighbors and/or adjacent.} OSPF
routers become neighbors when each router sees the other's Hello packets
and the timers match between routers.

\protect\hypertarget{c18.xhtmlux5cux23Page_772}{}{}\textbf{Be able to
configure single-area OSPF.} A minimal single-area configuration
involves only two commands: \texttt{router\ ospf} \texttt{process-id}
and \texttt{network} \texttt{x.x.x.x\ y.y.y.y\ area\ Z}.

\textbf{Be able to verify the operation of OSPF.} There are many
\texttt{show} commands that provideuseful details on OSPF, and it is
useful to be completely familiar with the output of each:
\texttt{show\ ip\ ospf}, \texttt{show\ ip\ ospf\ database},
\texttt{show\ ip\ ospf\ interface}, \texttt{show\ ip\ ospf}
\texttt{neighbor}, and \texttt{show\ ip\ protocols}.

\subsection[Written Lab
18]{\texorpdfstring{\protect\hypertarget{c18.xhtmlux5cux23c18-sec-19}{}{}Written
Lab 18}{Written Lab 18}}

You can find the answers to this lab in Appendix A, ``Answers to Written
Labs.''

\begin{enumerate}
\tightlist
\item
  Write the command that will enable the OSPF process 101 on a router.
\item
  Write the command that will display details of all OSPF routing
  processes enabled on a router.
\item
  Write the command that will display interface-specific OSPF
  information.
\item
  Write the command that will display all OSPF neighbors.
\item
  Write the command that will display all different OSPF route types
  that are currently known by the router.
\item
  Which parameter or parameters are used to calculate OSPF cost in Cisco
  routers?
\item
  Two routers are not forming an adjacency. What are all the reasons
  that OSPF will not form this adjacency with the neighbor router?
\item
  Which command is used to display the collection of OSPF link states?
\item
  What is the default administrative distance of OSPF?
\item
  What is the default to which hello and dead timers are set?
\end{enumerate}

\subsection[Hands-on
Labs]{\texorpdfstring{\protect\hypertarget{c18.xhtmlux5cux23c18-sec-20}{}{}Hands-on
Labs}{Hands-on Labs}}

In this section, you will use the following network and add OSPF
routing.

\begin{figure}
\centering
\includegraphics{images/c18f007.jpg}
\caption{}
\end{figure}

\protect\hypertarget{c18.xhtmlux5cux23Page_773}{}{}The first lab (Lab
18.1) requires you to configure three routers for OSPF and then view the
configuration. Note that the labs in this chapter were written to be
used with real equipment---but they can be used with any router
simulator. You can replace the WAN links with Ethernet links if you want
to.

The labs in this chapter are as follows:

\begin{enumerate}
\tightlist
\item
  Lab 18.1: Enabling the OSPF Process
\item
  Lab 18.2: Configuring OSPF Interfaces
\item
  Lab 18.3: Verifying OSPF Operation
\end{enumerate}

\protect\hyperlink{c18.xhtmlux5cux23table18-3}{Table 18.3} shows our IP
addresses for each router (each interface uses a /24 mask).

{\protect\hyperlink{c18.xhtmlux5cux23tableanchor18-3}{\textbf{TABLE~18.3}}
Our IP addresses}

\begin{longtable}[]{@{}lll@{}}
\toprule
Router & Interface & IP address\tabularnewline
\midrule
\endhead
Lab\_A & Fa0/0 & 172.16.10.1\tabularnewline
Lab\_A & S0/0 & 172.16.20.1\tabularnewline
Lab\_B & S0/0 & 172.16.20.2\tabularnewline
Lab\_B & S0/1 & 172.16.30.1\tabularnewline
Lab\_C & S0/0 & 172.16.30.2\tabularnewline
Lab\_C & Fa0/0 & 172.16.40.1\tabularnewline
\bottomrule
\end{longtable}

\subsubsection[Hands-on Lab 18.1: Enabling the OSPF
Process]{\texorpdfstring{\protect\hypertarget{c18.xhtmlux5cux23c18-sec-21}{}{}Hands-on
Lab 18.1: Enabling the OSPF
Process}{Hands-on Lab 18.1: Enabling the OSPF Process}}

This is the first mandatory step in OSPF configuration.

\begin{enumerate}
\item
  Enable OSPF process 100 on Lab\_A:

\begin{verbatim}
Lab_A#conf t
Enter configuration commands, one per line.
  End with CNTL/Z.
Lab_A (config)#router ospf 100
Lab_A (config-router)#^Z
\end{verbatim}
\item
  Enable OSPF process 101 on Lab\_B:

\begin{verbatim}
Lab_B#conf t
Enter configuration commands, one per line.
  End with CNTL/Z.
Lab_B (config)#router ospf 101
Lab_B (config-router)#^Z
\end{verbatim}
\item
  Enable OSPF process 102 on Lab\_C:

\begin{verbatim}
Lab_C#conf t
Enter configuration commands, one per line.
  End with CNTL/Z.
Lab_C (config)#router ospf 102
Lab_C (config-router)#^Z
\end{verbatim}
\end{enumerate}

\subsubsection[Hands-on Lab 18.2: Configuring OSPF
Interfaces]{\texorpdfstring{\protect\hypertarget{c18.xhtmlux5cux23c18-sec-22}{}{}Hands-on
Lab 18.2: Configuring OSPF
Interfaces}{Hands-on Lab 18.2: Configuring OSPF Interfaces}}

The second mandatory step in OSPF is adding your network statements.

\begin{enumerate}
\item
  Configure the LAN and the network between Lab\_A and Lab\_B. Assign it
  to area 0.

\begin{verbatim}
Lab_A#conf t
Enter configuration commands, one per line.
  End with CNTL/Z.
Lab_A (config)#router ospf 100
Lab_A (config-router)#network 172.16.10.1 0.0.0.0 area 0
Lab_A (config-router)#network 172.16.20.1 0.0.0.0 area 0
Lab_A (config-router)#^Z
Lab_A #
\end{verbatim}
\item
  Configure the networks on the Lab\_B router. Assign them to area 0.

\begin{verbatim}
Lab_B#conf t
Enter configuration commands, one per line.
  End with CNTL/Z.
Lab_B(config)#router ospf 101
Lab_B(config-router)#network 172.16.20.2 0.0.0.0 area 0
Lab_B(config-router)#network 172.16.30.1 0.0.0.0 area 0
Lab_B(config-router)#^Z
Lab_B #
\end{verbatim}
\item
  Configure the networks on the Lab\_C router. Assign them to area 0.

\begin{verbatim}
Lab_C#conf t
Enter configuration commands, one per line.
  End with CNTL/Z.
Lab_C(config)#router ospf 102
Lab_C(config-router)#network 172.16.30.2 0.0.0.0 area 0
Lab_C(config-router)#network 172.16.40.1 0.0.0.0 area 0
Lab_C(config-router)#^Z
Lab_C#
\end{verbatim}
\end{enumerate}

\subsubsection[Hands-on Lab 18.3: Verifying OSPF
Operation]{\texorpdfstring{\protect\hypertarget{c18.xhtmlux5cux23c18-sec-23}{}{}Hands-on
Lab 18.3: Verifying OSPF
Operation}{Hands-on Lab 18.3: Verifying OSPF Operation}}

You need to be able to verify what you configure.

\begin{enumerate}
\item
  Execute a \texttt{show\ ip\ ospf\ neighbors} command from the Lab\_A
  router and view the results.

\begin{verbatim}
Lab_A#sho ip ospf neighbors
\end{verbatim}
\item
  Execute a \texttt{show\ ip\ route} command to verify that all other
  routers are learning all routes.

\begin{verbatim}
Lab_A#sho ip route
\end{verbatim}
\item
  Execute a \texttt{show\ ip\ protocols} command to verify OSPF
  information.

\begin{verbatim}
Lab_A#sho ip protocols
\end{verbatim}
\item
  Execute a \texttt{show\ ip\ OSPF} command to verify your RID.

\begin{verbatim}
Lab_A#sho ip ospf
\end{verbatim}
\item
  Execute a \texttt{show\ ip\ ospf\ interface\ f0/0} command to verify
  your timers.

\begin{verbatim}
Lab_A#sho ip ospf int f0/0
\end{verbatim}
\end{enumerate}

\subsection[Review
Questions]{\texorpdfstring{\protect\hypertarget{c18.xhtmlux5cux23c18-sec-24}{}{}\protect\hypertarget{c18.xhtmlux5cux23Page_776}{}{}Review
Questions}{Review Questions}}

\begin{center}\rule{0.5\linewidth}{0.5pt}\end{center}

\includegraphics{images/note.png}The following questions are designed to
test your understanding of this chapter's material. For more information
on how to get additional questions, please see
\href{http://www.lammle.com/ccna}{www.lammle.com/ccna}.

\begin{center}\rule{0.5\linewidth}{0.5pt}\end{center}

You can find the answers to these questions in Appendix B, ``Answers to
Review Questions.''

\begin{enumerate}
\item
  There are three possible routes for a router to reach a destination
  network. The first route is from OSPF with a metric of 782. The second
  route is from RIPv2 with a metric of 4. The third is from EIGRP with a
  composite metric of 20514560. Which route will be installed by the
  router in its routing table?

  \begin{enumerate}
  \def\labelenumii{\Alph{enumii}.}
  \tightlist
  \item
    RIPv2
  \item
    EIGRP
  \item
    OSPF
  \item
    All three
  \end{enumerate}
\item
  In the accompanying diagram, which of the routers must be ABRs?
  (Choose all that apply.)

  \begin{figure}
  \centering
  \includegraphics{images/c18f008.jpg}
  \caption{}
  \end{figure}

  \begin{enumerate}
  \def\labelenumii{\Alph{enumii}.}
  \tightlist
  \item
    C
  \item
    D
  \item
    E
  \item
    F
  \item
    G
  \item
    H
  \end{enumerate}
\item
  \protect\hypertarget{c18.xhtmlux5cux23Page_777}{}{}Which of the
  following describe the process identifier that is used to run OSPF on
  a router? (Choose two.)

  \begin{enumerate}
  \def\labelenumii{\Alph{enumii}.}
  \tightlist
  \item
    It is locally significant.
  \item
    It is globally significant.
  \item
    It is needed to identify a unique instance of an OSPF database.
  \item
    It is an optional parameter required only if multiple OSPF processes
    are running on the router.
  \item
    All routes in the same OSPF area must have the same process ID if
    they are to exchange routing information.
  \end{enumerate}
\item
  All of the following must match for two OSPF routers to become
  neighbors except which?

  \begin{enumerate}
  \def\labelenumii{\Alph{enumii}.}
  \tightlist
  \item
    Area ID
  \item
    Router ID
  \item
    Stub area flag
  \item
    Authentication password if using one
  \end{enumerate}
\item
  In the diagram, by default what will be the router ID of Lab\_B?

  \begin{figure}
  \centering
  \includegraphics{images/c18f009.jpg}
  \caption{}
  \end{figure}

  \begin{enumerate}
  \def\labelenumii{\Alph{enumii}.}
  \tightlist
  \item
    10.255.255.82
  \item
    10.255.255.9
  \item
    192.168.10.49
  \item
    10.255.255.81
  \end{enumerate}
\item
  You get a call from a network administrator who tells you that he
  typed the following into his router:

\begin{verbatim}
Router(config)#router ospf 1
Router(config-router)#network 10.0.0.0 255.0.0.0 area 0
\end{verbatim}

  He tells you he still can't see any routes in the routing table. What
  configuration error did the administrator make?

  \begin{enumerate}
  \def\labelenumii{\Alph{enumii}.}
  \tightlist
  \item
    The wildcard mask is incorrect.
  \item
    The OSPF area is wrong.
  \item
    \protect\hypertarget{c18.xhtmlux5cux23Page_778}{}{}The OSPF process
    ID is incorrect.
  \item
    The AS configuration is wrong.
  \end{enumerate}
\item
  Which of the following statements is true with regard to the output
  shown?

\begin{verbatim}
Corp#sh ip ospf neighbor
Neighbor ID     Pri   State      Dead Time   Address         Interface
172.31.1.4        1   FULL/BDR   00:00:34    10.10.10.2     FastEthernet0/0
192.168.20.1      0   FULL/  -   00:00:31    172.16.10.6     Serial0/1
192.168.10.1      0   FULL/  -   00:00:32    172.16.10.2     Serial0/0
\end{verbatim}

  \begin{enumerate}
  \def\labelenumii{\Alph{enumii}.}
  \tightlist
  \item
    There is no DR on the link to 192.168.20.1.
  \item
    The Corp router is the BDR on the link to 172.31.1.4.
  \item
    The Corp router is the DR on the link to 192.168.20.1.
  \item
    The link to 192.168.10.1 is Active.
  \end{enumerate}
\item
  What is the administrative distance of OSPF?

  \begin{enumerate}
  \def\labelenumii{\Alph{enumii}.}
  \tightlist
  \item
    90
  \item
    100
  \item
    120
  \item
    110
  \end{enumerate}
\item
  In OSPF, Hellos are sent to what IP address?

  \begin{enumerate}
  \def\labelenumii{\Alph{enumii}.}
  \tightlist
  \item
    224.0.0.5
  \item
    224.0.0.9
  \item
    224.0.0.10
  \item
    224.0.0.1
  \end{enumerate}
\item
  What command generated the following output?

\begin{verbatim}
172.31.1.4        1   FULL/BDR   00:00:34    10.10.10.2     FastEthernet0/0
192.168.20.1      0   FULL/  -   00:00:31    172.16.10.6     Serial0/1
192.168.10.1      0   FULL/  -   00:00:32    172.16.10.2     Serial0/0
\end{verbatim}

  \begin{enumerate}
  \def\labelenumii{\Alph{enumii}.}
  \tightlist
  \item
    \texttt{show\ ip\ ospf\ neighbor}
  \item
    \texttt{show\ ip\ ospf\ database}
  \item
    \texttt{show\ ip\ route}
  \item
    \texttt{show\ ip\ ospf\ interface}
  \end{enumerate}
\item
  Updates addressed to 224.0.0.6 are destined for which type of OSPF
  router?

  \begin{enumerate}
  \def\labelenumii{\Alph{enumii}.}
  \tightlist
  \item
    DR
  \item
    ASBR
  \item
    \protect\hypertarget{c18.xhtmlux5cux23Page_779}{}{}ABR
  \item
    All OSPF routers
  \end{enumerate}
\item
  For some reason, you cannot establish an adjacency relationship on a
  common Ethernet link between two routers. Looking at this output, what
  is the cause of the problem?

\begin{verbatim}
RouterA#
Ethernet0/0 is up, line protocol is up
  Internet Address 172.16.1.2/16, Area 0
  Process ID 2, Router ID 172.126.1.2, Network Type BROADCAST, Cost: 10
  Transmit Delay is 1 sec, State DR, Priority 1
  Designated Router (ID) 172.16.1.2, interface address 172.16.1.1
  No backup designated router on this network
  Timer intervals configured, Hello 5, Dead 20, Wait 20, Retransmit 5
 
RouterB#
Ethernet0/0 is up, line protocol is up
  Internet Address 172.16.1.1/16, Area 0
  Process ID 2, Router ID 172.126.1.1, Network Type BROADCAST, Cost: 10
  Transmit Delay is 1 sec, State DR, Priority 1
  Designated Router (ID) 172.16.1.1, interface address 172.16.1.2
  No backup designated router on this network
  Timer intervals configured, Hello 10, Dead 40, Wait 40, Retransmit 5
\end{verbatim}

  \begin{enumerate}
  \def\labelenumii{\Alph{enumii}.}
  \tightlist
  \item
    The OSPF area is not configured properly.
  \item
    The priority on RouterA should be set higher.
  \item
    The cost on RouterA should be set higher.
  \item
    The Hello and Dead timers are not configured properly.
  \item
    A backup designated router needs to be added to the network.
  \item
    The OSPF process ID numbers must match.
  \end{enumerate}
\item
  In the work area, match each OSPF term (by line) to its definition.

  \begin{figure}
  \centering
  \includegraphics{images/c18f010.jpg}
  \caption{}
  \end{figure}
\item
  Type the command that will disable OSPF on the Fa0/1 interface under
  the routing process. Write only the command and not the prompt.
\item
  \protect\hypertarget{c18.xhtmlux5cux23Page_780}{}{}Which two of the
  following commands will place network 10.2.3.0/24 into area 0? (Choose
  two.)

  \begin{enumerate}
  \def\labelenumii{\Alph{enumii}.}
  \tightlist
  \item
    \texttt{router\ eigrp\ 10}
  \item
    \texttt{router\ ospf\ 10}
  \item
    \texttt{router\ rip}
  \item
    \texttt{network\ 10.0.0.0}
  \item
    \texttt{network\ 10.2.3.0\ 255.255.255.0\ area\ 0}
  \item
    \texttt{network\ 10.2.3.0\ 0.0.0.255\ area0}
  \item
    \texttt{network\ 10.2.3.0\ 0.0.0.255\ area\ 0}
  \end{enumerate}
\item
  Given the following output, which statement or statements can be
  determined to be true? (Choose all that apply.)

\begin{verbatim}
RouterA2# show ip ospf neighbor
 
Neighbor ID Pri State Dead Time Address Interface
192.168.23.2 1 FULL/BDR 00:00:29 10.24.4.2 FastEthernet1/0
192.168.45.2 2 FULL/BDR 00:00:24 10.1.0.5 FastEthernet0/0
192.168.85.1 1 FULL/- 00:00:33 10.6.4.10 Serial0/1
192.168.90.3 1 FULL/DR 00:00:32 10.5.5.2 FastEthernet0/1
192.168.67.3 1 FULL/DR 00:00:20 10.4.9.20 FastEthernet0/2
192.168.90.1 1 FULL/BDR 00:00:23 10.5.5.4 FastEthernet0/1
<<output omitted>>
\end{verbatim}

  \begin{enumerate}
  \def\labelenumii{\Alph{enumii}.}
  \tightlist
  \item
    The DR for the network connected to Fa0/0 has an interface priority
    higher than 2.
  \item
    This router (A2) is the BDR for subnet 10.1.0.0.
  \item
    The DR for the network connected to Fa0/1 has a router ID of
    10.5.5.2.
  \item
    The DR for the serial subnet is 192.168.85.1.
  \end{enumerate}
\item
  What are three reasons for creating OSPF in a hierarchical design?
  (Choose three.)

  \begin{enumerate}
  \def\labelenumii{\Alph{enumii}.}
  \tightlist
  \item
    To decrease routing overhead
  \item
    To speed up convergence
  \item
    To confine network instability to single areas of the network
  \item
    To make configuring OSPF easier
  \end{enumerate}
\item
  Type the command that produced the following output. Write only the
  command and not the prompt.

\begin{verbatim}
FastEthernet0/0 is up, line protocol is up
  Internet Address 10.10.10.1/24, Area 0
  Process ID 1, Router ID 223.255.255.254, Network Type BROADCAST, Cost: 
1 Transmit Delay is 1 sec, State DR, Priority 1
  Designated Router (ID) 223.255.255.254, Interface address 10.10.10.1
Backup Designated router (ID) 172.31.1.4, Interface address 10.10.10.2
Timer intervals configured, Hello 10, Dead 40, Wait 40, Retransmit 5
    oob-resync timeout 40
    Hello due in 00:00:08
  Supports Link-local Signaling (LLS)
  Cisco NSF helper support enabled
  IETF NSF helper support enabled
  Index 3/3, flood queue length 0
  Next 0x0(0)/0x0(0)
  Last flood scan length is 1, maximum is 1
  Last flood scan time is 0 msec, maximum is 0 msec
  Neighbor Count is 1, Adjacent neighbor count is 1
    Adjacent with neighbor 172.31.1.  Suppress hello for 0 neighbor(s)
\end{verbatim}
\item
  A(n) \_\_\_\_\_\_\_\_\_\_ is an OSPF data packet containing link-state
  and routing information that is shared among OSPF routers.

  \begin{enumerate}
  \def\labelenumii{\Alph{enumii}.}
  \tightlist
  \item
    LSA
  \item
    TSA
  \item
    Hello
  \item
    SPF
  \end{enumerate}
\item
  If routers in a single area are configured with the same priority
  value, what value does a router use for the OSPF router ID in the
  absence of a loopback interface?

  \begin{enumerate}
  \def\labelenumii{\Alph{enumii}.}
  \tightlist
  \item
    The lowest IP address of any physical interface
  \item
    The highest IP address of any physical interface
  \item
    The lowest IP address of any logical interface
  \item
    The highest IP address of any logical interface
  \end{enumerate}
\end{enumerate}

\protect\hypertarget{c19.xhtml}{}{}

\section[{Chapter 19}\\
{Multi-Area
OSPF}]{\texorpdfstring{\protect\hypertarget{c19.xhtmlux5cux23c19}{}{}\protect\hypertarget{c19.xhtmlux5cux23Page_783}{}{}{Chapter
19}\\
{Multi-Area OSPF}}{Chapter 19 Multi-Area OSPF}}

\begin{center}\rule{0.5\linewidth}{0.5pt}\end{center}

\subsection{THE FOLLOWING ICND2 EXAM TOPICS ARE COVERED IN THIS
CHAPTER:}

\begin{enumerate}
\tightlist
\item
  \includegraphics{images/rarr.png}\textbf{2.0 Routing Technologies}
\item
  \includegraphics{images/rarr.png}\textbf{2.2 Compare and contrast
  distance vector and link-state routing protocols}
\item
  \includegraphics{images/rarr.png}\textbf{2.3 Compare and contrast
  interior and exterior routing protocols}
\item
  \includegraphics{images/rarr.png}\textbf{2.4 Configure, verify, and
  troubleshoot single area and multiarea OSPFv2 for IPv4 (excluding
  authentication, filtering, manual summarization, redistribution, stub,
  virtual-link, and LSAs)}
\item
  \includegraphics{images/rarr.png}\textbf{2.5 Configure, verify, and
  troubleshoot single area and multiarea OSPFv3 for IPv6 (excluding
  authentication, filtering, manual summarization, redistribution, stub,
  virtual-link, and LSAs)}
\end{enumerate}

\protect\hypertarget{c19.xhtmlux5cux23Page_784}{}{}
\includegraphics{images/intro.png}We'll begin this chapter by focusing
on the scalability constraints of an Open Shortest Path First (OSPF)
network with a single area and move on from there to explore the concept
of multi-area OSPF as a solution to these scalability limitations.

I'll also identify and introduce you to the various categories of
routers used in multi-area configurations, including backbone routers,
internal routers, area border routers (ABRs), and autonomous system
boundary routers (ASBRs).

The functions of different OSPF Link-State Advertisements (LSAs) are
absolutely crucial for you to understand for success in taking the Cisco
exam, so I'll go into detail about the types of LSAs used by OSPF as
well as the Hello protocol and different neighbor states when an
adjacency is taking place.

And because troubleshooting is always a vital skill to have, I'll guide
you through the process with a collection of \texttt{show} commands that
can be effectively used to monitor and troubleshoot a multi-area OSPF
implementation. Finally, I'll end the chapter with the easiest part:
configuring and verifying OSPFv3.

\begin{center}\rule{0.5\linewidth}{0.5pt}\end{center}

\includegraphics{images/note.png}To find up-to-the-minute updates for
this chapter, please see
\href{http://www.lammle.com/ccna}{www.lammle.com/ccna} or the book's web
page at \href{http://www.sybex.com/go/ccna}{www.sybex.com/go/ccna}.

\begin{center}\rule{0.5\linewidth}{0.5pt}\end{center}

\subsection[OSPF
Scalability]{\texorpdfstring{\protect\hypertarget{c19.xhtmlux5cux23c19-sec-1}{}{}OSPF
Scalability}{OSPF Scalability}}

At this point, and before you read this chapter, be sure that you have
the foundation of single-area OSPF down pat. I'm sure you remember
OSPF's significant advantage over distance-vector protocols like RIP,
due to OSPF's ability to represent an entire network within its
link-state database, which dramatically reduces the time required for
convergence!

But what does a router actually go through to give us this great
performance? Each router recalculates its database every time there's a
topology change. If you have numerous routers in an area, they'll
clearly have lots of links. Every time a link goes up or down, an LSA
Type 1 packet is advertised, forcing all of the routers in the same area
to recalculate their shortest path first (SPF) tree. Predictably, this
kind of heavy lifting requires a ton of CPU overhead. On top of that,
each router must hold the entire link-state database that represents the
topology of the entire network, which results in considerable memory
overhead. As if all that weren't enough, each router also holds a
complete copy of the routing table, adding more to the already heavy
overhead burden on memory. And keep in mind that the
\protect\hypertarget{c19.xhtmlux5cux23Page_785}{}{}number of entries in
the routing table can be much greater than the number of networks in the
routing table because there are typically multiple routes to the same
remote networks!

Considering these OSPF factors, it's easy to imagine that in a really
large network, single-area OSPF presents some serious scalability
challenges, as shown in
\protect\hyperlink{c19.xhtmlux5cux23figure19-1}{Figure 19.1}. We'll move
on in a bit to compare the single-area OSPF network in that illustration
to our multi-area networks.

\begin{figure}
\centering
\includegraphics{images/c19f001.jpg}
\caption{{\protect\hyperlink{c19.xhtmlux5cux23figureanchor19-1}{\textbf{Figure
19.1}} OSPF single-area network: All routers flood the network with
link-state information to all other routers within the same area.}}
\end{figure}

Single-area OSPF design places all routers into a single OSPF area,
which results in many LSAs being processed on every router.

Fortunately, OSPF allows us to take a large OSPF topology and break it
down into multiple, more manageable areas, as illustrated in
\protect\hyperlink{c19.xhtmlux5cux23figure19-2}{Figure 19.2}.

\begin{figure}
\centering
\includegraphics{images/c19f002.jpg}
\caption{{\protect\hyperlink{c19.xhtmlux5cux23figureanchor19-2}{\textbf{Figure
19.2}} OSPF multi-area network: All routers flood the network only
within their area.}}
\end{figure}

\protect\hypertarget{c19.xhtmlux5cux23Page_786}{}{}Just take a minute to
think about the advantages of this hierarchical approach. First, routers
that are internal to a defined area don't need to worry about having a
link-state database for the entire network because they need one for
only their own areas. This factor seriously reduces memory overhead!
Second, routers that are internal to a defined area now have to
recalculate their link-state database only when there's a topology
change within their given area. Topology changes in one area won't cause
global OSPF recalculations, further reducing processor overhead.
Finally, because routes can be summarized at area boundaries, the
routing tables on each router just don't need to be nearly as huge as
they would be in a single-area environment!

But of course there's a catch: As you start subdividing your OSPF
topology into multiple areas, the configuration gets more complex, so
we'll explore some strategic ways to finesse the configuration plus look
at some cool tricks for effectively troubleshooting multi-area OSPF
networks.

\subsection[Categories of Multi-area
Components]{\texorpdfstring{\protect\hypertarget{c19.xhtmlux5cux23c19-sec-2}{}{}Categories
of Multi-area Components}{Categories of Multi-area Components}}

In the following sections, I'm going to cover the various roles that
routers play in a multi-area OSPF network. You'll find routers serving
as backbone routers, internal routers, area border routers, and
autonomous system boundary routers. I'll also introduce you to the
different types of advertisements used in an OSPF network.

Link-State Advertisements (LSAs) describe a router and the networks that
are connected to it by sending the LSAs to neighbor routers. Routers
exchange LSAs and learn the complete topology of the network until all
routers have the exact same topology database. After the topology
database is built, OSPF uses the Dijkstra algorithm to find the best
path to each remote network and places only the best routes into the
routing table.

\subsubsection[Adjacency
Requirements]{\texorpdfstring{\protect\hypertarget{c19.xhtmlux5cux23c19-sec-3}{}{}Adjacency
Requirements}{Adjacency Requirements}}

Once neighbors have been identified, adjacencies must be established so
that routing (LSA) information can be exchanged. There are two steps
required to change a neighboring OSPF router into an adjacent OSPF
router:

\begin{enumerate}
\tightlist
\item
  Two-way communication (achieved via the Hello protocol)
\item
  Database synchronization, which consists of three packet types being
  exchanged between routers:

  \begin{enumerate}
  \tightlist
  \item
    Database Description (DD) packets
  \item
    Link-State Request (LSR) packets
  \item
    Link-State Update (LSU) packets
  \end{enumerate}
\end{enumerate}

Once database synchronization is complete, the two routers are
considered adjacent. This is how adjacency is achieved, but you need to
know when an adjacency will occur.

\protect\hypertarget{c19.xhtmlux5cux23Page_787}{}{}It's important to
remember that neighbors will not form an adjacency if the following do
not match:

\begin{enumerate}
\tightlist
\item
  Area ID
\item
  Subnet
\item
  Hello and dead timers
\item
  Authentication (if configured)
\end{enumerate}

When adjacencies form depends on the network type. If the link is
point-to-point, the two neighbors will become adjacent if the Hello
packet information for both routers is configured properly. On broadcast
multi-access networks, adjacencies are formed only between the OSPF
routers on the network and the DR and BDR.

\subsubsection[OSPF Router
Roles]{\texorpdfstring{\protect\hypertarget{c19.xhtmlux5cux23c19-sec-4}{}{}OSPF
Router Roles}{OSPF Router Roles}}

Routers within a multi-area OSPF network fall into different categories.
Check out \protect\hyperlink{c19.xhtmlux5cux23figure19-3}{Figure 19.3}
to see the various roles that routers can play.

\begin{figure}
\centering
\includegraphics{images/c19f003.jpg}
\caption{{\protect\hyperlink{c19.xhtmlux5cux23figureanchor19-3}{\textbf{Figure
19.3}} Router roles: Routers within an area are called internal
routers.}}
\end{figure}

Notice that there are four routers that are part of area 0: the Corp
router, SF and NY, and the autonomous system border router (ASBR). When
configuring multi-area OSPF, one area must be called area 0, referred to
as the \emph{backbone area}. All other areas must connect to area 0. The
four routers are referred to as the backbone routers, which are any
routers that exist either partially or completely in OSPF area 0.

Another key distinction about the SF and NY routers connecting to other
areas is that they have interfaces in more than one area. This makes
them \emph{area border routers (ABRs)} because in addition to having an
interface in area 0, SF has an interface in area 1 and NY has an
interface in area 2.

\protect\hypertarget{c19.xhtmlux5cux23Page_788}{}{}An ABR is a router
that belongs to more than one OSPF area. It maintains information from
all directly connected areas in its topology table but doesn't share the
topological details from one area with the other. But it will forward
routing information from one area to the other. The key concept here is
that an ABR separates the LSA flooding zone, is a primary point for area
address summarization, and typically has the source default route, all
while maintaining the link-state database (LSDB) for each area it's
connected to.

\begin{center}\rule{0.5\linewidth}{0.5pt}\end{center}

\includegraphics{images/c19inline02.png}Remember that a router can play
more than one role. In
\protect\hyperlink{c19.xhtmlux5cux23figure19-3}{Figure 19.3}, SF and NY
are both backbone routers and area border routers.

\begin{center}\rule{0.5\linewidth}{0.5pt}\end{center}

Let's turn our focus to the San Jose and Oakland routers. You can see
that all interfaces on both of these routers reside only in area 1.
Because all of San Jose's and Oakland's interfaces are internal to a
single area, they're called internal routers. An \emph{internal router}
is any router with all of its interfaces included as members of the same
area. This also applies to the Boston and Norfolk routers and their
relationship to area 2. The Corp router is internal to area 0.

Finally, the ASBR is unique among all routers in our example because of
its connection to an external \emph{autonomous system (AS)}. When an
OSPF network is connected to an EIGRP network, a \emph{Border Gateway
Protocol (BGP)} network, or a network running any other external routing
process, it's referred to as an AS.

An \emph{autonomous system boundary router (ASBR)} is an OSPF router
with at least one interface connected to an external network or
different AS. A network is considered external if a route received is
from a routing protocol other than OSPF. An ASBR is responsible for
injecting route information learned via the external network into OSPF.

I want to point out that an ASBR doesn't automatically exchange routing
information between its OSPF routing process and the external routing
process that it's connected to. These routes are exchanged through a
method called \emph{route redistribution}, which is beyond the scope of
this book.

\subsubsection[Link-State
Advertisements]{\texorpdfstring{\protect\hypertarget{c19.xhtmlux5cux23c19-sec-5}{}{}Link-State
Advertisements}{Link-State Advertisements}}

You know that a router's link-state database is made up of
\emph{Link-State Advertisements (LSAs)}. But just as there are several
OSPF router categories to remember, there are also various types of LSAs
to keep in mind---five of them, to be exact. These LSA classifications
may not seem important at first, but you'll see why they are when we
cover how the various types of OSPF areas operate. Let's start by
exploring the different types of LSAs that Cisco uses:

\textbf{Type 1 LSA} Referred to as a \emph{router link advertisement
(RLA)}, or just router LSA, a \emph{Type 1 LSA} is sent by every router
to other routers in its area. This advertisement contains the status of
a router's link in the area to which it is connected. If a router is
connected to multiple areas, then it will send separate Type 1 LSAs for
each of the areas it's connected to. Type 1 LSAs contain the router ID
(RID), interfaces, IP information, and current
\protect\hypertarget{c19.xhtmlux5cux23Page_789}{}{}interface state. For
example, in the network in
\protect\hyperlink{c19.xhtmlux5cux23figure19-4}{Figure 19.4}, router SF
will send an LSA Type 1 advertisement for its interface into area 0 and
a separate LSA Type 1 advertisement for its interfaces into area 1
describing the state of its links. The same will happen with the other
routers in \protect\hyperlink{c19.xhtmlux5cux23figure19-4}{Figure 19.4}.

\begin{figure}
\centering
\includegraphics{images/c19f004.jpg}
\caption{{\protect\hyperlink{c19.xhtmlux5cux23figureanchor19-4}{\textbf{Figure
19.4}} Type 1 Link-State Advertisements}}
\end{figure}

\textbf{Type 2 LSA} Referred to as a \emph{network link advertisement
(NLA)}, a \emph{Type 2 LSA} is generated by designated routers (DRs).
Remember that a designated router is elected to represent other routers
in its network, and it establishes adjacencies with them. The DR uses a
Type~2 LSA to send out information about the state of other routers that
are part of the same ­network. Note that the Type 2 LSA is flooded to
all routers that are in the same area as the one containing the specific
network but not to any outside of that area. These updates contain the
DR and BDR IP information.

\textbf{Type 3 LSA} Referred to as a \emph{summary link advertisement
(SLA)}, a \emph{Type 3 LSA} is generated by area border routers. These
ABRs send Type 3 LSAs toward the area external to the one where they
were generated. The Type 3 LSA advertises networks, and these LSAs
advertise \emph{inter-area routes} to the backbone area (area 0).
Advertisements contain the IP information and RID of the ABR that is
advertising an LSA Type 3.

\begin{center}\rule{0.5\linewidth}{0.5pt}\end{center}

\includegraphics{images/note.png}The word \emph{summary} often invokes
images of a summarized network address that hides the details of many
small subnets within the advertisement of a single large one. But in
OSPF, summary link advertisements don't necessarily contain network
summaries. Unless the administrator manually creates a summary, the full
list of individual networks available within an area will be advertised
by the SLAs.

\begin{center}\rule{0.5\linewidth}{0.5pt}\end{center}

\protect\hypertarget{c19.xhtmlux5cux23Page_790}{}{}\textbf{Type 4 LSA}
\emph{Type 4 LSAs} are generated by area border routers. These ABRs send
a Type 4 LSA toward the area external to the one in which they were
generated. These are also summary LSAs like Type 3, but Type 4 are
specifically used to inform the rest of the OSPF areas how to get to the
ASBR.

\textbf{Type 5 LSA} Referred to as \emph{AS external link
advertisements}, a \emph{Type 5 LSA} is sent by autonomous system
boundary routers to advertise routes that are external to the OSPF
autonomous system and are flooded everywhere. A Type 5 LSA is generated
for each individual external network advertised by the ASBR.

\protect\hyperlink{c19.xhtmlux5cux23figure19-5}{Figure 19.5} shows how
each LSA type would be used in a multi-area OSPF network.

\begin{figure}
\centering
\includegraphics{images/c19f005.jpg}
\caption{{\protect\hyperlink{c19.xhtmlux5cux23figureanchor19-5}{\textbf{Figure
19.5}} Basic LSA types}}
\end{figure}

It's important to understand the different LSA types and how they work.
Looking at \protect\hyperlink{c19.xhtmlux5cux23figure19-5}{Figure 19.5},
you can see that Type 1 and 2 are flooded between routers in their same
area. Type 3 LSAs from the Corp router (which is an ABR and maintains
the LSDB for each area it is connected to) will summarize information
learned from area 1 into area 0 and vice versa. The ASBR will flood Type
5 LSAs into area 1, and the Corp router will then flood Type 4 LSAs into
area 0, telling all routers how to get to the ASBR, basically becoming a
proxy ASBR.

\subsubsection[OSPF Hello
Protocol]{\texorpdfstring{\protect\hypertarget{c19.xhtmlux5cux23c19-sec-6}{}{}OSPF
Hello Protocol}{OSPF Hello Protocol}}

The Hello protocol provides a lot of information to neighbors. The
following is communicated between neighbors, by default, every 10
seconds:

\textbf{Router ID (RID)} This is the highest active IP address on the
router. The highest loopback IP addresses are used first. If no loopback
interfaces are configured, OSPF will choose from physical interfaces
instead.

\protect\hypertarget{c19.xhtmlux5cux23Page_791}{}{}\textbf{Hello/Dead
interval} The period between Hello packets is the Hello time, which is
10 seconds by default. The dead time is the length of time allotted for
a Hello packet to be received before a neighbor is considered
down---four times the Hello interval, unless otherwise configured.

\textbf{Neighbors} The information includes a list of the router IDs for
all the originating router's neighbors, neighbors being defined as
routers that are attached to a common IP subnet and use identical subnet
masks.

\textbf{Area ID} This represents the area that the originating router
interface belongs to.

\textbf{Router priority} The priority is an 8-bit value used to aid in
the election of the DR and BDR. This isn't set on point-to-point links!

\textbf{DR IP address} This is the router ID of the current DR.

\textbf{BDR IP address} This is the router ID of the current BDR.

\textbf{Authentication data} This is the authentication type and
corresponding information (if configured).

The mandatory information within the Hello update that must match
exactly are the hello and dead timer values intervals, area ID, OSPF
area type, subnet, and authentication data if used. If any of those
don't match perfectly, no adjacency will occur!

\subsubsection[Neighbor
States]{\texorpdfstring{\protect\hypertarget{c19.xhtmlux5cux23c19-sec-7}{}{}Neighbor
States}{Neighbor States}}

Before we move on to configuration, verification, and troubleshooting
OSPF, it's important for you to grasp how OSPF routers traverse
different states when adjacencies are being established.

When OSPF routers are initialized, they first start exchanging
information using the Hello protocol via the multicast address
224.0.0.5. After the neighbor relationship is established between
routers, the routers synchronize their link-state database (LSDB) by
reliably exchanging LSAs. They actually exchange quite a bit of vital
information when they start up.

The relationship that one router has with another consists of eight
possible states. All OSPF routers begin in the DOWN state, and if all is
well, they'll progress to either the 2WAY or FULL state with their
neighbors. \protect\hyperlink{c19.xhtmlux5cux23figure19-6}{Figure 19.6}
shows this neighbor state progression.

\begin{figure}
\centering
\includegraphics{images/c19f006.jpg}
\caption{{\protect\hyperlink{c19.xhtmlux5cux23figureanchor19-6}{\textbf{Figure
19.6}} OSPF neighbor states, part 1}}
\end{figure}

\protect\hypertarget{c19.xhtmlux5cux23Page_792}{}{}The process starts by
sending out Hello packets. Every listening router will then add the
originating router to the neighbor database. The responding routers will
reply with all of their Hello information so that the originating router
can add them to its own neighbor table. At this point, we will have
reached the 2WAY state---only certain routers will advance beyond this
to establish adjacencies.

Here's a definition of the eight possible relationship states:

\textbf{DOWN} In the \emph{DOWN state}, no Hello packets have been
received on the interface. Bear in mind that this does not imply that
the interface itself is physically down.

\textbf{ATTEMPT} In the \emph{ATTEMPT state}, neighbors must be
configured manually. It applies only to nonbroadcast multi-access (NBMA)
network connections.

\textbf{INIT} In the \emph{INIT state}, Hello packets have been received
from another router. Still, the absence of the router ID for the
receiving router in the Neighbor field indicates that bidirectional
communication hasn't been established yet.

\textbf{2WAY} In the \emph{2WAY state}, Hello packets that include their
own router ID in the Neighbor field have been received. Bidirectional
communication has been established. In broadcast multi-access networks,
an election can occur after this point.

After the DR and BDR have been selected, the routers will enter into the
EXSTART state and the routers are ready to discover the link-state
information about the internetwork and create their LSDB. This process
is illustrated in \protect\hyperlink{c19.xhtmlux5cux23figure19-7}{Figure
19.7}.

\begin{figure}
\centering
\includegraphics{images/c19f007.jpg}
\caption{{\protect\hyperlink{c19.xhtmlux5cux23figureanchor19-7}{\textbf{Figure
19.7}} OSPF router neighbor states, part 2}}
\end{figure}

\textbf{EXSTART} In the \emph{EXSTART state}, the DR and BDR establish
adjacencies with each router in the network. A master-slave relationship
is created between each router and its adjacent DR and DBR. The router
with the highest RID becomes the master, and the ­master-slave election
dictates which router will start the exchange. Once routers exchange DBD
packets, the routers will move into the EXCHANGE state.

\protect\hypertarget{c19.xhtmlux5cux23Page_793}{}{}

\begin{center}\rule{0.5\linewidth}{0.5pt}\end{center}

\includegraphics{images/note.png}One reason two neighbor routers won't
get past the EXSTART state is that they have different MTUs.

\begin{center}\rule{0.5\linewidth}{0.5pt}\end{center}

\textbf{EXCHANGE} In the \emph{EXCHANGE state}, routing information is
exchanged using Database Description (DBD or DD) packets, and Link-State
Request (LSR) and Link-State Update packets may also be sent. When
routers start sending LSRs, they're considered to be in the LOADING
state.

\textbf{LOADING} In the \emph{LOADING state}, Link-State Request (LSR)
packets are sent to neighbors to request any Link-State Advertisements
(LSAs) that may have been missed or corrupted while the routers were in
the EXCHANGE state. Neighbors respond with Link-State Update (LSU)
packets, which are in turn acknowledged with Link-State Acknowledgement
(LSAck) packets. When all LSRs have been satisfied for a given router,
the adjacent routers are considered synchronized and enter the FULL
state.

\textbf{FULL} In the \emph{FULL state}, all LSA information is
synchronized among neighbors and adjacency has been established. OSPF
routing can begin only after the FULL state has been reached!

It's important to understand that routers should be in the 2WAY and FULL
states and the others are considered transitory. Routers shouldn't
remain in any other state for extended period of times. Let's configure
OSPF now to see what we've covered so far in action.

\subsection[Basic Multi-area
Configuration]{\texorpdfstring{\protect\hypertarget{c19.xhtmlux5cux23c19-sec-8}{}{}Basic
Multi-area Configuration}{Basic Multi-area Configuration}}

Basic multi-area configuration isn't all that hard. Understanding your
design, layout, and types of LSAs and DRs and configuring the elections,
troubleshooting, and fully comprehending what's happening in the
background are really the most complicated aspects of OSPF.

As I was saying, configuring OSPF is pretty simple, and you'll see
toward the end of this chapter that configuring OSPFv3 is even easier!
After I show you the basic OSPF multi-area configuration in this
section, we'll work on the verification of OSPF and then go through a
detailed troubleshooting scenario just as we did with EIGRP. Let's get
the ball rolling with the multi-area configuration shown in
\protect\hyperlink{c19.xhtmlux5cux23figure19-8}{Figure 19.8}.

We'll use the same routers we've been working with throughout all the
chapters, but we're going to create three areas. The routers are still
configured with the IPv6 addresses from my last EIGRPv6 section in
Chapter 3, and I've also verified that the IPv4 addresses are on the
interfaces and working as well since then, so we're all set to rock the
configs for this chapter! Here's the Corp configuration:

\begin{verbatim}
Corp#config t
Corp(config)#router ospf 1
Corp(config-router)#router-id 1.1.1.1
Reload or use "clear ip ospf process" command, for this to
take effect
 
Corp(config-router)#network 10.10.0.0 0.0.255.255 area 0
Corp(config-router)#network 172.16.10.0 0.0.0.3 area 1
Corp(config-router)#network 172.16.10.4 0.0.0.3 area 2
\end{verbatim}

\begin{figure}
\centering
\includegraphics{images/c19f008.jpg}
\caption{{\protect\hyperlink{c19.xhtmlux5cux23figureanchor19-8}{\textbf{Figure
19.8}} Our internetwork}}
\end{figure}

Pretty straightforward, but let's talk about it anyway. First I started
the OSPF process with the router \texttt{ospf\ process-id} command,
using any number from 1--65,535 because they're only locally
significant, so they don't need to match my neighbor routers. I set the
RID of the router only to remind you that this can be configured under
the router process, but with our small network it wouldn't really be
necessary to mess with RIDs if this was an actual production network.
The one thing that you need to keep in mind here is that in OSPF, the
RID must be different on each router. With EIGRP, they can all be the
same because they are not as important in that process. Still, as I
showed you in the EIGRPv6 section, we still need them!

Anyway, at this point in the configurations I needed to choose my
network statements for the OSPF process to use, which allowed me to
place my four interfaces on the Corp router into three different areas.
In the first network statement, 10.10.0.0 0.0.255.255, I placed the g0/0
and g0/1 interfaces into area 0. The second and third statements needed
to be more exact since there are /30 networks. 172.16.10.0 0.0.0.3 tells
OSPF process 1 to go find an active interface that's configured with
172.16.10.1 or .2 and to place that interface into area 1. The last line
tells the OSPF process to go find any active interface configured
\protect\hypertarget{c19.xhtmlux5cux23Page_795}{}{}with 172.16.10.5 or
.6 and place that interface into area 2. The wildcard of 0.0.0.3 means
the first three octets can match any value, but the last octet is a
block size of 4.

The only thing different about these configurations from those in the
single-area OSPF is the different areas at the end of the
command---that's it!

Here is the configuration for the SF and NY routers:

\begin{verbatim}
SF(config)#router ospf 1
SF(config-router)#network 10.10.0.0 0.0.255.255 area 1
SF(config-router)#network 172.16.0.0 0.0.255.255 area 1
 
NY(config)#router ospf 1
NY(config-router)#network 0.0.0.0 255.255.255.255 area 2
00:01:07: %OSPF-5-ADJCHG: Process 1, Nbr 1.1.1.1 on Serial0/0/0 from LOADING to FULL,
Loading Done
\end{verbatim}

I configured each one slightly different from the Corp router, but since
they didn't have an interface in more than area 1, I had more leeway in
configuring them. For the NY router I just configured a network
statement (0.0.0.0 255.255.255.255) that says ``go find any active
interface and place it into area 2!'' I'm not recommending that you
configure your routers in such a broad manner; I just wanted to show you
your options.

Before we move onto verifying our network, let me show you another way
that the CCNA objectives configure OSPF. For the Corp router, we had
three network statements, which covered the four interfaces used. We
could have configured the OSPF process like this on the Corp router (or
all routers); it doesn't matter which way you choose:

\begin{verbatim}
Corp(config)#router ospf 1
Corp(config-router)#router-id 1.1.1.1
Corp(config-router)#int g0/0
Corp(config-if)#ip ospf 1 area 0
Corp(config-if)#int g0/1
Corp(config-if)#ip ospf 1 area 0
Corp(config-if)#int s0/0
Corp(config-if)#ip ospf 1 area 1
Corp(config-if)#int s0/1
Corp(config-if)#ip ospf 1 area 2
\end{verbatim}

First I chose my process ID, then set my RID (this absolutely must be
different on every router in your internetwork!), then I just went to
each interface and told it what area it was in. Easy! No network
commands to screw up! Nice. Again, you can configure it with the network
statement or the interface statement, it doesn't matter, but you need to
really remember this for the CCNA objectives!

Now that our three routers are configured, let's verify our
internetwork.

\subsection[Verifying and Troubleshooting ­Multi-area OSPF
Networks]{\texorpdfstring{\protect\hypertarget{c19.xhtmlux5cux23c19-sec-9}{}{}\protect\hypertarget{c19.xhtmlux5cux23Page_796}{}{}Verifying
and Troubleshooting ­Multi-area OSPF
Networks}{Verifying and Troubleshooting ­Multi-area OSPF Networks}}

Cisco's IOS has several \texttt{show} and \texttt{debug} commands that
can help you monitor and troubleshoot OSPF networks. A sampling of these
commands, which can be used to gain information about various OSPF
characteristics, is included in
\protect\hyperlink{c19.xhtmlux5cux23table19-1}{Table 19.1}.

{\protect\hyperlink{c19.xhtmlux5cux23tableanchor19-1}{\textbf{Table~19.1}}
OSPF verification commands}

\begin{longtable}[]{@{}ll@{}}
\toprule
Command & Provides the following\tabularnewline
\midrule
\endhead
\texttt{show\ ip\ ospf\ neighbor} & Verifies your OSPF-enabled
interfaces\tabularnewline
\texttt{show\ ip\ ospf\ interface} & Displays OSPF-related information
on an OSPF-enabled interface\tabularnewline
\texttt{show\ ip\ protocols} & Verifies the OSPF process ID and that
OSPF is enabled on the router\tabularnewline
\texttt{show\ ip\ route} & Verifies the routing table, and displays any
OSPF injected routes\tabularnewline
\texttt{show\ ip\ ospf\ database} & Lists a summary of the LSAs in the
database, with one line of output per LSA, organized by
type\tabularnewline
\bottomrule
\end{longtable}

Let's go through some verification commands---the same commands we used
to verify our single-area OSPF network---then we'll move onto the OSPF
troubleshooting scenario section.

Okay, once you've checked the link between your neighbors and can use
the Ping program, the best command when verifying a routing protocol is
to always check the status of your neighbor's connection first. The
\texttt{show\ ip\ ospf\ neighbor} command is super useful because it
summarizes the pertinent OSPF information regarding neighbors and their
adjacency state. If a DR or BDR exists, that information will also be
displayed. Here's a sample:

\begin{verbatim}
Corp#sh ip ospf neighbor
Neighbor ID     Pri   State           Dead Time   Address         Interface
172.16.10.2       0   FULL/  -        00:00:34    172.16.10.2     Serial0/0/0
172.16.10.6       0   FULL/  -        00:00:31    172.16.10.6     Serial0/0/1
 
SF#sh ip ospf neighbor
Neighbor ID     Pri   State           Dead Time   Address         Interface
1.1.1.1           0   FULL/  -        00:00:39    172.16.10.1     Serial0/0/0
 
NY#sh ip ospf neighbor
Neighbor ID     Pri   State           Dead Time   Address         Interface
1.1.1.1           0   FULL/  -        00:00:34    172.16.10.5     Serial0/0/0
\end{verbatim}

The reason that the Corp connections to SF and LA don't have a DR or BDR
listed in the output is that by default, elections don't happen on
point-to-point links and they show \texttt{FULL/-}. But we can see that
the Corp router is fully adjacent to all three routers from its output.

The output of this command shows the neighbor ID, which is the RID of
the router. Notice in the output of the Corp router that the RIDs for
the SF and NY routers were chosen based on highest IP address of any
active interface when I started the OSPF process on those routers. Both
the SF and NY routers see the Corp router RID as 1.1.1.1 because I set
that manually under the \texttt{router\ ospf} process command.

Next we see the \texttt{Pri} field, which is the priority field that's
set to 1 by default. Don't forget that on point-to-point links,
elections don't happen, so the interfaces are all set to 0 in this
example because none of these routers will have elections on these
interfaces with each other over this serial WAN network. The state field
shows \texttt{Full/-}, which means all routers are synchronized with
their LSDB, and the \texttt{/-} means there is no election on this type
of interface. The dead timer is counting down, and if the router does
not hear from this neighbor before this expires, the link will be
considered down. The Address field is the actual address of the
neighbor's interface connecting to the router.

\subsubsection[The \emph{show ip ospf}
Command]{\texorpdfstring{\protect\hypertarget{c19.xhtmlux5cux23c19-sec-10}{}{}The
\emph{show ip ospf} Command}{The show ip ospf Command}}

We use the \texttt{show\ ip\ ospf} command to display OSPF information
for one or all OSPF processes running on the router. Information
contained therein includes the router ID, area information, SPF
statistics, and LSA timer information. Let's check out the output from
the Corp router:

\begin{verbatim}
Corp#sh ip ospf
 Routing Process "ospf 1" with ID 1.1.1.1
 Supports only single TOS(TOS0) routes
 Supports opaque LSA
 It is an area border router
 SPF schedule delay 5 secs, Hold time between two SPFs 10 secs
 Minimum LSA interval 5 secs. Minimum LSA arrival 1 secs
 Number of external LSA 0. Checksum Sum 0x000000
 Number of opaque AS LSA 0. Checksum Sum 0x000000
 Number of DCbitless external and opaque AS LSA 0
 Number of DoNotAge external and opaque AS LSA 0
 Number of areas in this router is 3. 3 normal 0 stub 0 nssa
 External flood list length 0
    Area BACKBONE(0)
        Number of interfaces in this area is 2
        Area has no authentication
        SPF algorithm executed 19 times
        Area ranges are
        Number of LSA 7. Checksum Sum 0x0384d5
        Number of opaque link LSA 0. Checksum Sum 0x000000
        Number of DCbitless LSA 0
        Number of indication LSA 0
        Number of DoNotAge LSA 0
        Flood list length 0
    Area 1
        Number of interfaces in this area is 1
        Area has no authentication
        SPF algorithm executed 43 times
        Area ranges are
        Number of LSA 7. Checksum Sum 0x0435f8
        Number of opaque link LSA 0. Checksum Sum 0x000000
        Number of DCbitless LSA 0
        Number of indication LSA 0
        Number of DoNotAge LSA 0
        Flood list length 0
    Area 2
        Number of interfaces in this area is 1
        Area has no authentication
        SPF algorithm executed 38 times
        Area ranges are
        Number of LSA 7. Checksum Sum 0x0319ed
        Number of opaque link LSA 0. Checksum Sum 0x000000
        Number of DCbitless LSA 0
        Number of indication LSA 0
        Number of DoNotAge LSA 0
        Flood list length 0
\end{verbatim}

You'll notice that most of the preceding information wasn't displayed
with this command output in single-area OSPF. We have more displayed
here because it's providing information about each area we've configured
on this router.

\subsubsection[The \emph{show ip ospf interface}
Command]{\texorpdfstring{\protect\hypertarget{c19.xhtmlux5cux23c19-sec-11}{}{}The
\emph{show ip ospf interface}
Command}{The show ip ospf interface Command}}

The \texttt{show\ ip\ ospf\ interface} command displays all
interface-related OSPF information. Data is displayed for all
OSPF-enabled interfaces or for specified interfaces. I'll highlight some
important portions I want you to pay special attention to.

\begin{verbatim}
Corp#sh ip ospf interface gi0/0
GigabitEthernet0/0 is up, line protocol is up
  Internet address is 10.10.10.1/24, Area 0
  Process ID 1, Router ID 1.1.1.1, Network Type BROADCAST, Cost: 1
  Transmit Delay is 1 sec, State DR, Priority 1
  Designated Router (ID) 1.1.1.1, Interface address 10.10.10.1
  No backup designated router on this network
  Timer intervals configured, Hello 10, Dead 40, Wait 40, Retransmit 5
    Hello due in 00:00:05
  Index 1/1, flood queue length 0
  Next 0x0(0)/0x0(0)
  Last flood scan length is 1, maximum is 1
  Last flood scan time is 0 msec, maximum is 0 msec
  Neighbor Count is 0, Adjacent neighbor count is 0
  Suppress hello for 0 neighbor(s)
\end{verbatim}

Let's take a look at a serial interface so we can compare it to the
Gigabit Ethernet interface just shown. The Ethernet network is a
broadcast multi-access network by default, and the serial interface is a
point-to-point nonbroadcast multi-access network, so they will act
differently with OSPF:

\begin{verbatim}
Corp#sh ip ospf interface s0/0/0
Serial0/0/0 is up, line protocol is up
  Internet address is 172.16.10.1/30, Area 1
  Process ID 1, Router ID 1.1.1.1, Network Type POINT-TO-POINT, Cost: 64
  Transmit Delay is 1 sec, State POINT-TO-POINT, Priority 0
  No designated router on this network
  No backup designated router on this network
  Timer intervals configured, Hello 10, Dead 40, Wait 40, Retransmit 5
    Hello due in 00:00:02
  Index 3/3, flood queue length 0
  Next 0x0(0)/0x0(0)
  Last flood scan length is 1, maximum is 1
  Last flood scan time is 0 msec, maximum is 0 msec
  Neighbor Count is 1, Adjacent neighbor count is 1
    Adjacent with neighbor 172.16.10.2
  Suppress hello for 0 neighbor(s)
\end{verbatim}

The following information is displayed via this command:

\begin{enumerate}
\tightlist
\item
  Interface IP address
\item
  Area assignment
\item
  Process ID
\item
  Router ID
\item
  \protect\hypertarget{c19.xhtmlux5cux23Page_800}{}{}Network type
\item
  Cost
\item
  Priority
\item
  DR/BDR election information (if applicable)
\item
  Hello and dead timer intervals
\item
  Adjacent neighbor information
\end{enumerate}

I used the \texttt{show\ ip\ ospf\ interface\ gi0/0} command first
because I knew that there would be a designated router elected on the
Ethernet broadcast multi-access network on the Corp router, even though
it has no one to run against, which means the Corp router automatically
wins. The information that I bolded is all very important! What are the
hello and dead timers set to by default? Even though I haven't talked
much about the cost output on an interface, it can also be very
important. Two OSPF routers still could create an adjacency if the costs
don't match, but it could lead to certain links not being utilized.
We'll discuss this more at the end of the verification section.

\begin{center}\rule{0.5\linewidth}{0.5pt}\end{center}

\includegraphics{images/c16inline02.png}

\subsection{Neighbor Routers Don't Form an Adjacency}

I'd like to talk more about the adjacency issue and how the
\texttt{show\ ip\ ospf\ interface} command can help you solve problems,
especially in multi-vendor networks.

Years ago I was consulting with the folks at a large PC/laptop
manufacturer and was helping them build out their large internetwork.
They were using OSPF because their company was a worldwide company and
used many types of routers from all manufacturers.

I received a call from a remote branch informing me that they installed
a new router but it was not seeing the Cisco router off their Ethernet
interface. Of course it was an emergency because this new router was
holding some important WAN links to a new remote location that needed to
be up yesterday!

After calming down the person on the phone, I simply had the admin use
the \texttt{show\ ip\ ospf\ interface\ fa0/0} command and verify the
hello and dead timers and the area configured for that interface and
then had him verify that the IP addresses were correct between routers
and that there was no passive interface set.

Then I had him verify that same information on the neighbor, and sure
enough the neighbor's hello and dead timers didn't match. Quick and easy
fix on the interface of the Cisco router with the
\texttt{ip\ ospf\ dead\ 30} command, and they were up!

Always remember that OSPF can work with multi-vendor routers, but no one
ever said it works out of the box between various vendors!

\begin{center}\rule{0.5\linewidth}{0.5pt}\end{center}

\subsubsection[The \emph{show ip protocols}
Command]{\texorpdfstring{\protect\hypertarget{c19.xhtmlux5cux23c19-sec-12}{}{}\protect\hypertarget{c19.xhtmlux5cux23Page_801}{}{}The
\emph{show ip protocols} Command}{The show ip protocols Command}}

The \texttt{show\ ip\ protocols} command is also useful, whether you're
running OSPF, EIGRP, RIP, BGP, IS-IS, or any other routing protocol that
can be configured on your router. It provides an excellent overview of
the actual operation of all currently running protocols.

Check the output from the Corp router:

\begin{verbatim}
Corp#sh ip protocols
Routing Protocol is "ospf 1"
  Outgoing update filter list for all interfaces is not set
  Incoming update filter list for all interfaces is not set
  Router ID 1.1.1.1
  Number of areas in this router is 3. 3 normal 0 stub 0 nssa
  Maximum path: 4
  Routing for Networks:
    10.10.0.0 0.0.255.255 area 0
    172.16.10.0 0.0.0.3 area 1
    172.16.10.4 0.0.0.3 area 2
  Routing Information Sources:
    Gateway         Distance      Last Update
    1.1.1.1              110      00:17:42
    172.16.10.2          110      00:17:42
    172.16.10.6          110      00:17:42
  Distance: (default is 110)
\end{verbatim}

Here we can determine the OSPF process ID, OSPF router ID, type of OSPF
area, networks, and the three areas configured for OSPF as well as the
OSPF router IDs of neighbors---that's a lot. Read efficient!

\subsubsection[The \emph{show ip route}
Command]{\texorpdfstring{\protect\hypertarget{c19.xhtmlux5cux23c19-sec-13}{}{}The
\emph{show ip route} Command}{The show ip route Command}}

Now would be a great time to issue a \texttt{show\ ip\ route} command on
the Corp router. The Corp router shows only four dynamic routes for our
internetwork, with the \emph{O} representing OSPF internal routes. The
\emph{C}s clearly represent our directly connected networks, but our
four remote networks are also showing up---nice! Notice the 110/65,
which is the administrative distance/metric:

\begin{verbatim}
Corp#sh ip route
[output cut]
     10.0.0.0/8 is variably subnetted, 8 subnets, 2 masks
C       10.10.10.0/24 is directly connected, GigabitEthernet0/0
L       10.10.10.1/32 is directly connected, GigabitEthernet0/0
C       10.10.11.0/24 is directly connected, GigabitEthernet0/1
L       10.10.11.1/32 is directly connected, GigabitEthernet0/1
O       10.10.20.0/24 [110/65] via 172.16.10.2, 02:18:27, Serial0/0/0
O       10.10.30.0/24 [110/65] via 172.16.10.2, 02:18:27, Serial0/0/0
O       10.10.40.0/24 [110/65] via 172.16.10.6, 03:37:24, Serial0/0/1
O       10.10.50.0/24 [110/65] via 172.16.10.6, 03:37:24, Serial0/0/1
     172.16.0.0/16 is variably subnetted, 4 subnets, 2 masks
C       172.16.10.0/30 is directly connected, Serial0/0/0
L       172.16.10.1/32 is directly connected, Serial0/0/0
C       172.16.10.4/30 is directly connected, Serial0/0/1
L       172.16.10.5/32 is directly connected, Serial0/0/1
\end{verbatim}

In addition, you can use the \texttt{show\ ip\ route\ ospf} command to
get only OSPF-injected routes in your routing table. I can't stress
enough how useful this is when dealing with large networks!

\begin{verbatim}
Corp#sh ip route ospf
     10.0.0.0/8 is variably subnetted, 8 subnets, 2 masks
O       10.10.20.0 [110/65] via 172.16.10.2, 02:18:33, Serial0/0/0
O       10.10.30.0 [110/65] via 172.16.10.2, 02:18:33, Serial0/0/0
O       10.10.40.0 [110/65] via 172.16.10.6, 03:37:30, Serial0/0/1
O       10.10.50.0 [110/65] via 172.16.10.6, 03:37:30, Serial0/0/1
\end{verbatim}

Now that's a really nice-looking OSPF routing table! Troubleshooting and
fixing an OSPF network is as vital a skill to have as it is in any other
networking environment, which is why I always use the
\texttt{show\ ip\ int\ brief} command when configuring my routing
protocols. It's very easy to make little mistakes with OSPF, so pay very
close attention to the details---especially when troubleshooting!

\subsubsection[The \emph{show ip ospf database}
Command]{\texorpdfstring{\protect\hypertarget{c19.xhtmlux5cux23c19-sec-14}{}{}The
\emph{show ip ospf database}
Command}{The show ip ospf database Command}}

Using the \texttt{show\ ip\ ospf\ database} command will give you
information about the number of routers in the internetwork (AS), plus
the neighboring router's ID. This is the topology database I referred to
earlier.

The output is broken down by area. Here's a sample, again from Corp:

\begin{verbatim}
Corp#sh ip ospf database
            OSPF Router with ID (1.1.1.1) (Process ID 1)
 
                Router Link States (Area 0)
 
Link ID         ADV Router      Age         Seq#       Checksum Link count
1.1.1.1         1.1.1.1         196         0x8000001a 0x006d76 2
 
                Summary Net Link States (Area 0)
Link ID         ADV Router      Age         Seq#       Checksum
172.16.10.0     1.1.1.1         182         0x80000095 0x00be04
172.16.10.4     1.1.1.1         177         0x80000096 0x009429
10.10.40.0      1.1.1.1         1166        0x80000091 0x00222b
10.10.50.0      1.1.1.1         1166        0x80000092 0x00b190
10.10.20.0      1.1.1.1         1114        0x80000093 0x00fa64
10.10.30.0      1.1.1.1         1114        0x80000094 0x008ac9
 
                Router Link States (Area 1)
 
Link ID         ADV Router      Age         Seq#       Checksum Link count
1.1.1.1         1.1.1.1         1118        0x8000002a 0x00a59a 2
172.16.10.2     172.16.10.2     1119        0x80000031 0x00af47 4
 
                Summary Net Link States (Area 1)
Link ID         ADV Router      Age         Seq#       Checksum
10.10.10.0      1.1.1.1         178         0x80000076 0x0021a5
10.10.11.0      1.1.1.1         178         0x80000077 0x0014b0
172.16.10.4     1.1.1.1         173         0x80000078 0x00d00b
10.10.40.0      1.1.1.1         1164        0x80000074 0x005c0e
10.10.50.0      1.1.1.1         1164        0x80000075 0x00eb73
 
                Router Link States (Area 2)
 
Link ID         ADV Router      Age         Seq#       Checksum Link count
1.1.1.1         1.1.1.1         1119        0x8000002b 0x005cd6 2
172.16.10.6     172.16.10.6     1119        0x8000002d 0x0020a3 4
 
                Summary Net Link States (Area 2)
Link ID         ADV Router      Age         Seq#       Checksum
10.10.10.0      1.1.1.1         179         0x8000007a 0x0019a9
10.10.11.0      1.1.1.1         179         0x8000007b 0x000cb4
172.16.10.0     1.1.1.1         179         0x8000007c 0x00f0ea
10.10.20.0      1.1.1.1         1104        0x80000078 0x003149
10.10.30.0      1.1.1.1         1104        0x80000079 0x00c0ae
Corp#
\end{verbatim}

Considering we only have eight networks configured in our internetwork,
there's a huge amount of information in this database! You can see all
the routers and the RID of each---the highest IP address related to
individual routers. And each output under each area represents LSA Type
1, indicating the area they're connected to.

The router output also shows the link ID. Remember that an interface is
also a link, as is the RID of the router on that link under the ADV
router---the advertising router.

\protect\hypertarget{c19.xhtmlux5cux23Page_804}{}{}So far, this has been
a great chapter, brimming with detailed OSPF information, a whole lot
more than what was needed to meet past Cisco objectives, for sure! Next,
we'll use the same sample network that I built in Chapter 3 on EIGRP and
run through a troubleshooting scenario using multi-area OSPF.

\subsection[Troubleshooting OSPF
Scenario]{\texorpdfstring{\protect\hypertarget{c19.xhtmlux5cux23c19-sec-15}{}{}Troubleshooting
OSPF Scenario}{Troubleshooting OSPF Scenario}}

When you notice problems with your OSPF network, it's wise to first test
your layer 3 connectivity with Ping and the \texttt{traceroute} command
to see if your issue is a local one. If all looks good locally, then
follow these Cisco-provided guidelines:

\begin{enumerate}
\tightlist
\item
  Verify your adjacency with your neighbor routers using the
  \texttt{show\ ip\ ospf\ neighbors} command. If you are not seeing your
  neighbor adjacencies, then you need to verify that the interfaces are
  operational and enabled for OSPF. If all is well with the interfaces,
  verify the hello and dead timers next, and establish that the
  interfaces are in the same area and that you don't have a passive
  interface configured.
\item
  Once you've determined that your adjacencies to all neighbors are
  working, use the \texttt{show\ ip\ route} to verify your layer 3
  routes to all remote networks. If you see no OSPF routes in the
  routing table, you need to verify that you don't have another routing
  protocol running with a lower administrative distance. You can use
  \texttt{show\ ip\ protocols} to see all routing protocols running on
  your router. If no other protocols are running, then verify your
  \texttt{network} statements under the OSPF process. In a multi-area
  network, make sure all non--backbone area routers are directly
  connected to area 0 through an ABR or they won't be able to send and
  receive updates.
\item
  If you can see all the remote networks in the routing table, move on
  to verify the path for each network and that each path for each
  specific network is correct. If not, you need to verify the cost on
  your interfaces with the \texttt{show\ ip\ ospf\ interface} command.
  You may need to adjust the cost on an interface either higher or
  lower, depending on which path you want OSPF to use for sending
  packets to a remote network. Remember---the path with the lowest cost
  is the preferred path!
\end{enumerate}

Okay, with our marching orders for troubleshooting OSPF in hand, let's
take a look at \protect\hyperlink{c19.xhtmlux5cux23figure19-9}{Figure
19.9}, which we'll use to verify our network now.

\begin{figure}
\centering
\includegraphics{images/c19f009.jpg}
\caption{{\protect\hyperlink{c19.xhtmlux5cux23figureanchor19-9}{\textbf{Figure
19.9}} Our internetwork}}
\end{figure}

\protect\hypertarget{c19.xhtmlux5cux23Page_805}{}{}Here's the OSPF
configuration on the three routers:

\begin{verbatim}
Corp(config-if)#router ospf 1
Corp(config-router)#network 10.1.1.0 0.0.0.255 area 0
Corp(config-router)#network 192.168.1.0 0.0.0.3 area 1
 
Internal(config)#router ospf 3
Internal(config-router)#network 10.1.1.2 0.0.0.0 area 0
 
Branch(config-if)#router ospf 2
Branch(config-router)#network 192.168.1.2 0.0.0.0 area 1
Branch(config-router)#network 10.2.2.1 0.0.0.0 area 1
\end{verbatim}

Let's check out our network now, beginning by checking the layer 1 and
layer 2 status between routers:

\begin{verbatim}
Corp#sh ip int brief
Interface               IP-Address      OK? Method Status            Protocol
FastEthernet0/0         10.1.1.1        YES manual up                    up
Serial0/0               192.168.1.1     YES manual up                    up
\end{verbatim}

The IP addresses look correct and the layer 1 and 2 status is up/up, so
next we'll use the Ping program to check connectivity like this:

\begin{verbatim}
Corp#ping 192.168.1.2
Type escape sequence to abort.
Sending 5, 100-byte ICMP Echos to 192.168.1.2, timeout is 2 seconds:
!!!!!
Success rate is 100 percent (5/5), round-trip min/avg/max = 1/2/4 ms
Corp#ping 10.1.1.2
Type escape sequence to abort.
Sending 5, 100-byte ICMP Echos to 10.1.1.2, timeout is 2 seconds:
!!!!!
Success rate is 100 percent (5/5), round-trip min/avg/max = 1/2/4 ms
\end{verbatim}

Nice---I can ping both directly connected neighbors, so this means
layers 1, 2, and 3 are working between neighbor routers. This is a great
start, but it still doesn't mean OSPF is actually working yet. If any of
the preceding commands had failed, I first would've verified layers 1
and 2 to make sure my data link was working between neighbors and then
moved on to verify my layer 3 IP configuration.

Since our data link appears to be working between each neighbor, our
next move is to check the OSPF configuration and status of the routing
protocol. I'll start with the interfaces:

\begin{verbatim}
Corp#sh ip ospf interface s0/0
Serial0/0 is up, line protocol is up
  Internet Address 192.168.1.1/30, Area 1
  Process ID 1, Router ID 192.168.1.1, Network Type POINT_TO_POINT, Cost: 100
  Transmit Delay is 1 sec, State POINT_TO_POINT
  Timer intervals configured, Hello 10, Dead 40, Wait 40, Retransmit 5
    oob-resync timeout 40
    Hello due in 00:00:03
  Supports Link-local Signaling (LLS)
  Cisco NSF helper support enabled
  IETF NSF helper support enabled
  Index 1/2, flood queue length 0
  Next 0x0(0)/0x0(0)
  Last flood scan length is 1, maximum is 1
  Last flood scan time is 0 msec, maximum is 0 msec
  Neighbor Count is 1, Adjacent neighbor count is 1
    Adjacent with neighbor 192.168.1.2
  Suppress hello for 0 neighbor(s)
\end{verbatim}

I've highlighted the important statistics that you should always check
first on an OSPF interface. You need to verify that the interface is
configured in the same area as the neighbor and that the hello and dead
timers match. A cost mismatch won't stop an adjacency from forming, but
it could cause ugly routing issues. We'll explore that more in a minute.

For now let's take a look at the LAN interface that's connecting to the
Internal router:

\begin{verbatim}
Corp#sh ip ospf int f0/0
FastEthernet0/0 is up, line protocol is up
  Internet Address 10.1.1.1/24, Area 0
  Process ID 1, Router ID 192.168.1.1, Network Type BROADCAST, Cost: 1
  Transmit Delay is 1 sec, State DR, Priority 1
  Designated Router (ID) 192.168.1.1, Interface address 10.1.1.1
  Backup Designated router (ID) 10.1.1.2, Interface address 10.1.1.2
  Timer intervals configured, Hello 10, Dead 40, Wait 40, Retransmit 5
    oob-resync timeout 40
    Hello due in 00:00:00
  Supports Link-local Signaling (LLS)
  Cisco NSF helper support enabled
  IETF NSF helper support enabled
  Index 1/1, flood queue length 0
  Next 0x0(0)/0x0(0)
  Last flood scan length is 1, maximum is 1
  Last flood scan time is 0 msec, maximum is 0 msec
  Neighbor Count is 1, Adjacent neighbor count is 1
    Adjacent with neighbor 10.1.1.2  (Backup Designated Router)
  Suppress hello for 0 neighbor(s)
\end{verbatim}

\protect\hypertarget{c19.xhtmlux5cux23Page_807}{}{}We'll focus on the
same key factors on a LAN interface that we did on our serial interface:
the area ID and hello and dead timers. Notice that the cost is 1.
According to Cisco's method of calculating cost, anything 100 Mbps or
higher will always be a cost of 1 and serial links with the default
bandwidth are always 64. This can cause problems in a large network with
lots of high-bandwidth links. One thing to take special note of is that
there's a designated and backup designated router on a broadcast
multi-access network. DRs and BDRs won't cause a routing problem between
neighbors, but it's still a consideration when designing and configuring
in a really large internetwork environment. But we won't be focusing on
that for our purposes here. It's just something to keep in mind.

Staying with the troubleshooting step of checking our interfaces, look
at the error I received when I tried to verify OSPF on the fa0/1
interface of the Corp router (which we're not using):

\begin{verbatim}
Corp#sh ip ospf int fa0/1
%OSPF: OSPF not enabled on FastEthernet0/1
\end{verbatim}

I got this error because the network statements under the OSPF process
are not enabled for the network on the fa0/1 interface. If you receive
this error, immediately check your network statements!

Next, let's check out the networks we're routing for with the
\texttt{show\ ip\ protocols} command:

\begin{verbatim}
Corp#sh ip protocols
Routing Protocol is "ospf 1"
  Outgoing update filter list for all interfaces is not set
  Incoming update filter list for all interfaces is not set
  Router ID 192.168.1.1
  It is an area border router
  Number of areas in this router is 2. 2 normal 0 stub 0 nssa
  Maximum path: 4
  Routing for Networks:
    10.1.1.0 0.0.0.255 area 0
    192.168.1.0 0.0.0.3 area 1
 Reference bandwidth unit is 100 mbps
  Routing Information Sources:
    Gateway         Distance      Last Update
    192.168.1.2          110      00:28:40
  Distance: (default is 110)
\end{verbatim}

From this output we can check our process ID as well as reveal if we
have an ACL set on our routing protocol, just as we found when
troubleshooting EIGRP in Chapter 3. But this time, we'll first examine
the network statements and the area they're configured for---most
important, the specific areas that each interface is configured for.
This is key, because if your neighbor's interface isn't in the same
area, you won't be able to form an adjacency!
\protect\hypertarget{c19.xhtmlux5cux23Page_808}{}{}This command's output
provides a great view of what exactly we typed in for the network
statements under the OSPF process. Also, notice that the default
reference bandwidth is set to 100 Mbps. I'll talk about this factor at
the end of this section.

I want to point out that the neighbor IP address and administrative
distance is listed. OSPF uses 110 by default, so remember that if EIGRP
were running here, we wouldn't see OSPF routes in the routing table
because EIGRP has an AD of 90!

Next, we'll look at our neighbor table on the Corp router to find out if
OSPF has formed an adjacency with the Branch router:

\begin{verbatim}
Corp#sh ip ospf neighbor
Neighbor ID     Pri   State         Dead Time   Address       Interface
10.1.1.2          1   FULL/BDR      00:00:39    10.1.1.2      FastEthernet0/0
\end{verbatim}

Okay, we've finally zeroed in on our problem---the Corp router can see
the Internal router in area 0 but not the Branch router in area 1! What
now?

First, let's review what we know so far about the Corp and Branch
router. The data link is good, and we can use Ping successfully between
the routers. This shouts out that we have a routing protocol issue, so
we'll look further into the details of the OSPF configuration on each
router. Let's run a \texttt{show\ ip\ protocols} on the Branch router:

\begin{verbatim}
Branch#sh ip protocols
Routing Protocol is "eigrp 20"
  Outgoing update filter list for all interfaces is not set
  Incoming update filter list for all interfaces is not set
  Default networks flagged in outgoing updates
  Default networks accepted from incoming updates
  EIGRP metric weight K1=1, K2=0, K3=1, K4=0, K5=0
  EIGRP maximum hopcount 100
  EIGRP maximum metric variance 1
  Redistributing: eigrp 20
  EIGRP NSF-aware route hold timer is 240s
  Automatic network summarization is not in effect
  Maximum path: 4
  Routing for Networks:
    10.0.0.0
    192.168.1.0
  Routing Information Sources:
    Gateway         Distance      Last Update
    (this router)         90      3d22h
    192.168.1.1           90      00:00:07
  Distance: internal 90 external 170

Routing Protocol is "ospf 2"
  Outgoing update filter list for all interfaces is not set
  Incoming update filter list for all interfaces is not set
  Router ID 192.168.1.2
  Number of areas in this router is 1. 1 normal 0 stub 0 nssa
  Maximum path: 4
  Routing for Networks:
    10.2.2.1 0.0.0.0 area 1
    192.168.1.2 0.0.0.0 area 1
 Reference bandwidth unit is 100 mbps
  Passive Interface(s):
    Serial0/0/0
  Routing Information Sources:
    Gateway         Distance      Last Update
    192.168.1.1          110      03:29:07
  Distance: (default is 110)
\end{verbatim}

Do you see two routing protocols running on the Branch router? Both
EIGRP and OSPF are running, but that's not necessarily our problem. The
Corp router would need to be running EIGRP, and if so, we would have
only EIGRP routes in our routing table because EIGRPs have the lower AD
of 90 versus OSPF's AD of 110.

Let's check the routing table of the Branch router and see if the Corp
router is also running EIGRP. This will be easy to determine if we
discover EIGRP-injected routes in the table:

\begin{verbatim}
Branch#sh ip route
[output cut]
     10.0.0.0/24 is subnetted, 2 subnets
C       10.2.2.0 is directly connected, FastEthernet0/0
D       10.1.1.0 [90/2172416] via 192.168.1.1, 00:02:35, Serial0/0/0
     192.168.1.0/30 is subnetted, 1 subnets
C       192.168.1.0 is directly connected, Serial0/0/0
\end{verbatim}

Okay---so yes, the Corp router is clearly running EIGRP. This is a
leftover configuration from Chapter 3. All I need to do to fix this
issue is disable EIGRP on the Branch router. After that, we should see
OSPF in the routing table:

\begin{verbatim}
Branch#config t
Branch(config)#no router eigrp 20
Branch(config)#do sh ip route
[output cut]
     10.0.0.0/24 is subnetted, 1 subnets
C       10.2.2.0 is directly connected, FastEthernet0/0
     192.168.1.0/30 is subnetted, 1 subnets
C       192.168.1.0 is directly connected, Serial0/0/0
\end{verbatim}

That's not so good---I disabled the EIGRP protocol on the Branch router,
but we still didn't receive OSPF updates! Let investigate further using
the \texttt{show\ ip\ protocols} command on the Branch router:

\begin{verbatim}
Branch#sh ip protocols
Routing Protocol is "ospf 2"
  Outgoing update filter list for all interfaces is not set
  Incoming update filter list for all interfaces is not set
  Router ID 192.168.1.2
  Number of areas in this router is 1. 1 normal 0 stub 0 nssa
  Maximum path: 4
  Routing for Networks:
    10.2.2.1 0.0.0.0 area 1
    192.168.1.2 0.0.0.0 area 1
 Reference bandwidth unit is 100 mbps
  Passive Interface(s):
    Serial0/0/0
  Routing Information Sources:
    Gateway         Distance      Last Update
    192.168.1.1          110      03:34:19
  Distance: (default is 110)
\end{verbatim}

Do you see the problem? There's no ACL, the networks are configured
correctly, but see the passive interface for Serial0/0/0? That will
definitely prevent an adjacency from happening between the Corp and
Branch routers! Let's fix that:

\begin{verbatim}
Branach#show run
[output cut]
!
router ospf 2
 log-adjacency-changes
 passive-interface Serial0/0/0
 network 10.2.2.1 0.0.0.0 area 1
 network 192.168.1.2 0.0.0.0 area 1
!
[output cut]
Branch#config t
Branch(config)#router ospf 2
Branch(config-router)#no passive-interface serial 0/0/0
\end{verbatim}

\protect\hypertarget{c19.xhtmlux5cux23Page_811}{}{}Let's see what our
neighbor table and routing table look like now:

\begin{verbatim}
Branch#sh ip ospf neighbor
Neighbor ID     Pri   State           Dead Time   Address         Interface
192.168.1.1       0   FULL/  -        00:00:32    192.168.1.1     Serial0/0/0
 
Branch#sh ip route
     10.0.0.0/24 is subnetted, 2 subnets
C       10.2.2.0 is directly connected, FastEthernet0/0
O IA    10.1.1.0 [110/65] via 192.168.1.1, 00:01:21, Serial0/0/0
     192.168.1.0/30 is subnetted, 1 subnets
C       192.168.1.0 is directly connected, Serial0/0/0
\end{verbatim}

Awesome---our little internetwork is finally happy! That was actually
pretty fun and really not all that hard once you know what to look for.

But there's one more thing we need to cover before moving onto
OSPFv3---load balancing with OSPF. To explore that, we'll use
\protect\hyperlink{c19.xhtmlux5cux23figure19-10}{Figure 19.10}, wherein
I added another link between the Corp and Branch routers.

\begin{figure}
\centering
\includegraphics{images/c19f010.jpg}
\caption{{\protect\hyperlink{c19.xhtmlux5cux23figureanchor19-10}{\textbf{Figure
19.10}} Our internetwork with dual links}}
\end{figure}

First, it's clear that having a Gigabit Ethernet interface between our
two routers is way better than any serial link we could possibly have,
which means we want the routers to use the LAN link. We can either
disconnect the serial link or use it as a backup link.

Let's start by looking at the routing table and seeing what OSPF found:

\begin{verbatim}
Corp#sh ip route ospf
     10.0.0.0/8 is variably subnetted, 3 subnets, 2 masks
O       10.2.2.0 [110/2] via 192.168.1.6, 00:00:13, GigabitEthernet0/1
\end{verbatim}

Look at that! OSPF wisely went with the Gigabit Ethernet link because it
has the lowest cost. Although it's possible you'll have to mess with the
links to help OSPF choose the best paths, it's likely best to just leave
it alone at this point.

But that wouldn't be very much fun, now would it? Instead, let's
configure OSPF to fool it into thinking the links are equal so it will
use both of them by setting the cost on the interfaces to the same
value:

\begin{verbatim}
Corp#config t
Corp(config)#int g0/1
Corp(config-if)#ip ospf cost 10
Corp(config-if)#int s0/0/0
Corp(config-if)#ip ospf cost 10
\end{verbatim}

Obviously you need to deploy this configuration on both sides of the
link, and I've already configured the Branch router as well. Now that
both sides are configured with the same cost, let's check out our
routing table now:

\begin{verbatim}
Corp#sh ip route ospf
     10.0.0.0/8 is variably subnetted, 3 subnets, 2 masks
O       10.2.2.0 [110/11] via 192.168.1.2, 00:01:23, Serial0/0/0
                 [110/11] via 192.168.1.6, 00:01:23, GigabitEthernet0/1
\end{verbatim}

I'm not saying you should configure a serial link and Gigabit Ethernet
link as equal costs as I just demonstrated, but there are times when you
need to adjust the cost for OSPF. If you don't have multiple links to
any remote networks, you really don't need to worry about this, but with
regard to the objectives, you absolutely must understand the cost, how
it works, and how to set it so OSPF can choose a preferred path. And
there's still one more thing about cost I want to cover with you.

It's possible to change the reference bandwidth of the router, but you
need to make sure all the routers within the OSPF AS have the same
reference bandwidth. The default reference bandwidth is
10\textsuperscript{8}, which is 100,000,000, or the equivalent of the
bandwidth of Fast Ethernet, which is 100 Mbps, as demonstrated via
\texttt{show\ ip\ ospf} and the \texttt{show\ ip\ ­protocols} command:

\begin{verbatim}
Routing for Networks:
    10.2.2.1 0.0.0.0 area 1
    192.168.1.2 0.0.0.0 area 1
Reference bandwidth unit is 100 mbps
\end{verbatim}

This will basically make any interface running 100 Mbps or higher have a
cost of 1. The default is 100, and if you change it to 1,000, it will
increase the cost by a factor of 10. Again, if you do want to change
this, you must make sure to configure the change on all routers in your
AS! Here is how you would do that:

\begin{verbatim}
Corp(route)#router ospf 1
Corp(config-router)#auto-cost reference-bandwidth ?
  <1-4294967>  The reference bandwidth in terms of Mbits per second
\end{verbatim}

\subsubsection[Simple Troubleshooting OSPF for the
CCNA]{\texorpdfstring{\protect\hypertarget{c19.xhtmlux5cux23c19-sec-16}{}{}Simple
Troubleshooting OSPF for the
CCNA}{Simple Troubleshooting OSPF for the CCNA}}

Let's do a troubleshooting scenario. You have two routers not forming an
adjacency. What would you do first? Well, we went through a lot in this
chapter, but let me make it super easy for you when troubleshooting on
the CCNA exam.

\protect\hypertarget{c19.xhtmlux5cux23Page_813}{}{}All you need to do is
perform a \texttt{show\ running-config} on each router. That's it! You
can then fix anything regarding OSPF because all the problems will be
shown there if you know what to look for. Unlike with EIGRP, where you
don't need to see a neighbor router's configuration to verify the
protocol, with OSPF we need to compare directly connected interfaces to
make sure they match up.

Let's look at each router's configuration and determine what the
problems are.

Here is the first router's configuration:

\begin{verbatim}
R1#sh run
Building configuration...
!
interface Loopback0
 ip address 10.1.1.1 255.255.255.255
 ip ospf 3 area 0
!
int FastEthernet0/0
 Description **Connected to R2 F0/0**
 ip address 192.168.16.1 255.255.255.0
 ip ospf 3 area 0
 ip ospf hello-interval 25
!
router ospf 3
 router-id 192.168.3.3
\end{verbatim}

Here is the neighbor router's configuration:

\begin{verbatim}
R2#sh run
Building configuration...
!
interface Loopback0
 ip address 10.1.1.2 255.255.255.255
 ip ospf 6 area 0
!
 Description **Connected to R1 F0/0**
 int FastEthernet0/0
 ip address 192.168.17.2 255.255.255.0
 ip ospf 6 area 1
!
router ospf 6
 router-id 192.168.3.3
\end{verbatim}

\protect\hypertarget{c19.xhtmlux5cux23Page_814}{}{}Can you see the
problems? Pretty simple. Just use the \texttt{show\ running-config} on
each router and compare directly connected interfaces.

The loopbacks on each router are fine. I don't see a problem with their
configuration, and they don't connect to each other, so whatever
configuration they would have wouldn't matter.

However on the FastEthernet 0/0 interface of each router, where the
description tells us that R1 and R2 are directly connected with
interface f0/0, we can see a few problems. First, R1 is not using the
default hellointerval of 10 seconds as R2 is, so that will never work.
On R1 under the f0/0 interface, configure the
\texttt{no\ ip\ ospf\ hello-interval\ 25} command.

Let's dig deeper. Both routers have a different process ID, but that's
not a problem. However, the areas are configured differently on each
interface, and the IP addresses are not in the same subnet.

Last, they are both using the same RID under their process ID---this
will never work!

Now, finally, let's get to the easy section of the chapter!

\subsection[OSPFv3]{\texorpdfstring{\protect\hypertarget{c19.xhtmlux5cux23c19-sec-17}{}{}OSPFv3}{OSPFv3}}

The new version of OSPF continues the trend of routing protocols having
a lot in common with their IPv4 versions. The foundation of OSPF remains
the same---it's still a link-state routing protocol that divides an
entire internetwork or autonomous system into areas, establishing a
hierarchy.

In OSPF version 2, the router ID (RID) is determined by the highest IP
addresses assigned to the router. And as you now know, the RID can be
assigned. In version 3, nothing has really changed because you can still
assign the RID, area ID, and link-state ID, which remain 32-bit values.

Adjacencies and next-hop attributes now use link-local addresses, but
OSPFv3 still uses multi-cast traffic to send its updates and
acknowledgements. It uses the addresses FF02::5 for OSPF routers and
FF02::6 for OSPF-designated routers. These new addresses are the
replacements for 224.0.0.5 and 224.0.0.6, respectively.

Other, less flexible IPv4 protocols don't give us the ability that
OSPFv2 does to assign specific networks and interfaces into the OSPF
process, but this is still configured under the router configuration
process. And with OSPFv3, just as with the EIGRPv6 routing protocols
we've talked about, the interfaces and therefore the networks attached
to them are configured directly on the interface in interface
configuration mode.

The configuration of OSPFv3 is going to look like this: First,
optionally start by assigning the RID, but if you have IPv4 addresses
assigned to your interface, you can let OSPF pick the RID just as we did
with OSPFv2:

\begin{verbatim}
Router(config)#ipv6 router ospf 10
Router(config-rtr)#router-id 1.1.1.1
\end{verbatim}

You get to perform some other configurations from router configuration
mode, like summarization and redistribution, but again, we don't even
need to configure OSPFv3 from this prompt if we configure it from the
interface!

\protect\hypertarget{c19.xhtmlux5cux23Page_815}{}{}A simple interface
configuration looks like this:

\begin{verbatim}
Router(config-if)#ipv6 ospf 10 area 0.0.0.0
\end{verbatim}

So, if we just go to each interface and assign a process ID and
area---poof, we're done! See? Easy! As the configuration shows, I
configured the area as 0.0.0.0, which is the same thing as just typing
\texttt{area\ 0}. We'll use
\protect\hyperlink{c19.xhtmlux5cux23figure19-11}{Figure 19.11}, which is
the same network and IPv6-addressing we used in the EIGRPv6 section in
Chapter 3.

\begin{figure}
\centering
\includegraphics{images/c19f011.jpg}
\caption{{\protect\hyperlink{c19.xhtmlux5cux23figureanchor19-11}{\textbf{Figure
19.11}} Configuring OSPFv3}}
\end{figure}

Okay, so all we have to do to enable OSPF on the internetwork is go to
each interface that we want to run it on. Here's the Corp configuration:

\begin{verbatim}
Corp#config t
Corp(config)#int g0/0
Corp(config-if)#ipv6 ospf 1 area 0
Corp(config-if)#int g0/1
Corp(config-if)#ipv6 ospf 1 area 0
Corp(config-if)#int s0/0/0
Corp(config-if)#ipv6 ospf 1 area 0
Corp(config-if)#int s0/0/1
Corp(config-if)#ipv6 ospf 1 area 0
\end{verbatim}

That wasn't so bad---much easier than it was with IPv4! To configure
OSPFv3, you just need to establish the specific interfaces you'll be
using! Let's configure the other two routers now:

\begin{verbatim}
SF#config t
SF(config)#int g0/0
SF(config-if)#ipv6 ospf 1 area 1
SF(config-if)#int g0/1
SF(config-if)#ipv6 ospf 1 area 1
SF(config-if)#int s0/0/0
SF(config-if)#ipv6 ospf 1 area 0
01:03:55: %OSPFv3-5-ADJCHG: Process 1, Nbr 192.168.1.5 on Serial0/0/0 from LOADING to
FULL, Loading Done
\end{verbatim}

Sweet---the SF has become adjacent to the Corp router! One interesting
output line I want to point out is that the IPv4 RID is being used in
the OSPFv3 adjacent change. I didn't set the RIDs manually because I
knew I had interfaces with IPv4 addresses already on them, which the
OSPF process would use for a RID.

Now let's configure the NY router:

\begin{verbatim}
NY(config)#int g0/0
NY(config-if)#ipv6 ospf 1 area 2
%OSPFv3-4-NORTRID:OSPFv3 process 1 could not pick a router-id,please configure manually
NY(config-if)#ipv6 router ospf 1
NY(config-rtr)#router-id 1.1.1.1
NY(config-if)#int g0/0
NY(config-if)#ipv6 ospf 1 area 2
NY(config-if)#int g0/1
NY(config-if)#ipv6 ospf 1 area 2
NY(config-if)#int s0/0/0
NY(config-if)#ipv6 ospf 1 area 0
00:09:00: %OSPFv3-5-ADJCHG: Process 1, Nbr 192.168.1.5 on Serial0/0/0 from LOADING to
FULL, Loading Done
\end{verbatim}

Our adjacency popped up---this is great. But did you notice that I had
to set the RID? That's because there wasn't an IPv4 32-bit address
already on an interface for the router to use as the RID, so it was
mandatory to set the RID manually!

Without even verifying our network, it appears it's up and running. Even
so, it's always important to verify!

\subsubsection[Verifying
OSPFv3]{\texorpdfstring{\protect\hypertarget{c19.xhtmlux5cux23c19-sec-18}{}{}Verifying
OSPFv3}{Verifying OSPFv3}}

I'll start as usual with the \texttt{show\ ipv6\ route\ ospf} command:

\begin{verbatim}
Corp#sh ipv6 route ospf
OI  2001:DB8:3C4D:15::/64 [110/65]
     via FE80::201:C9FF:FED2:5E01, Serial0/0/1
OI  2001:DB8:3C4D:16::/64 [110/65]
     via FE80::201:C9FF:FED2:5E01, Serial0/0/1
O   2001:DB8:C34D:11::/64 [110/128]
     via FE80::2E0:F7FF:FE13:5E01, Serial0/0/0
OI  2001:DB8:C34D:17::/64 [110/65]
     via FE80::2E0:F7FF:FE13:5E01, Serial0/0/0
OI  2001:DB8:C34D:18::/64 [110/65]
     via FE80::2E0:F7FF:FE13:5E01, Serial0/0/0
\end{verbatim}

Perfect. I see all six subnets. Notice the \texttt{O} and \texttt{OI}?
The \texttt{O} is intra-area and the \texttt{OI} is inter-area, meaning
it's a route from a different area. You can't simply distinguish the
area by looking at the routing table though. Plus, don't forget that the
routers communicate with their neighbor via link-local addresses: via
FE80::2E0:F7FF:FE13:5E01, Serial0/0/0, for example.

Let's take a look at the \texttt{show\ ipv6\ protocols} command:

\begin{verbatim}
Corp#sh ipv6 protocols
IPv6 Routing Protocol is "connected"
IPv6 Routing Protocol is "static
IPv6 Routing Protocol is "ospf 1"
  Interfaces (Area 0)
    GigabitEthernet0/0
    GigabitEthernet0/1
    Serial0/0/0
    Serial0/0/1
\end{verbatim}

This just tells us which interfaces are part of OSPF process 1, area 0.
To configure OSPFv3, you absolutely have to know which interfaces are in
use. \texttt{Show\ ip\ int\ brief} can really help you if you're having
a problem finding your active interfaces.

Let's take a look at the Gigabit Ethernet OSPFv3 active interface on the
Corp router:

\begin{verbatim}
Corp#sh ipv6 ospf int g0/0
GigabitEthernet0/0 is up, line protocol is up
  Link Local Address FE80::2E0:F7FF:FE0A:3301, Interface ID 1
  Area 0, Process ID 1, Instance ID 0, Router ID 192.168.1.5
  Network Type BROADCAST, Cost: 1
  Transmit Delay is 1 sec, State DR, Priority 1
  Designated Router (ID) 192.168.1.5, local address FE80::2E0:F7FF:FE0A:3301
  No backup designated router on this network
  Timer intervals configured, Hello 10, Dead 40, Wait 40, Retransmit 5
    Hello due in 00:00:09
  Index 1/1, flood queue length 0
  Next 0x0(0)/0x0(0)
  Last flood scan length is 1, maximum is 1
  Last flood scan time is 0 msec, maximum is 0 msec
  Neighbor Count is 0, Adjacent neighbor count is 0
  Suppress hello for 0 neighbor(s)
\end{verbatim}

This is basically the same information we saw earlier in the
verification and troubleshooting section. Let's take a look at the
neighbor table on the Corp router via
\texttt{show\ ipv6\ ospf\ neighbor}:

\begin{verbatim}
Corp#sh ipv6 ospf neighbor
Neighbor ID     Pri   State           Dead Time   Interface ID    Interface
2.2.2.2           0   FULL/  -        00:00:36    4               Serial0/0/1
192.168.1.6       0   FULL/  -        00:00:39    4               Serial0/0/0
\end{verbatim}

Okay, we can see our two neighbors, and there's also a slight difference
in this version's command from OSPFv2. We still see the RID on the left
and that we're also fully adjacent with both our neighbors---the dash is
there because there are no elections on serial point-to-point links. But
we don't see the neighbor's IPv6 address listed as we did with OSPFv2's
IPv4 addresses, which were listed in the interface ID field.

There's one other command I want to finish with---the
\texttt{show\ ipv6\ ospf} command:

\begin{verbatim}
Corp#sh ipv6 ospf
 Routing Process "ospfv3 1" with ID 192.168.1.5
 SPF schedule delay 5 secs, Hold time between two SPFs 10 secs
 Minimum LSA interval 5 secs. Minimum LSA arrival 1 secs
 LSA group pacing timer 240 secs
 Interface flood pacing timer 33 msecs
 Retransmission pacing timer 66 msecs
 Number of external LSA 0. Checksum Sum 0x000000
 Number of areas in this router is 1. 1 normal 0 stub 0 nssa
 Reference bandwidth unit is 100 mbps
    Area BACKBONE(0)
        Number of interfaces in this area is 4
        SPF algorithm executed 10 times
        Number of LSA 10. Checksum Sum 0x05aebb
        Number of DCbitless LSA 0
        Number of indication LSA 0
        Number of DoNotAge LSA 0
        Flood list length 0
\end{verbatim}

This shows the process ID and RID, our reference bandwidth for this
interface, and how many interfaces we have in each area, which in our
example is only area 0.

\protect\hypertarget{c19.xhtmlux5cux23Page_819}{}{}Holy output! Now
that's what I call a fun chapter. The best thing you can do to get a
solid grasp of OSPF and OSPv3 multi-area networks is to gather up some
routers and spend some quality time with them, practicing everything
we've covered!

\subsection[Summary]{\texorpdfstring{\protect\hypertarget{c19.xhtmlux5cux23c19-sec-19}{}{}Summary}{Summary}}

In this chapter, you learned about the scalability constraints of a
single-area OSPF network, and you were introduced to the concept of
multi-area OSPF as a solution to these scalability limitations.

You're now able to identify the different categories of routers used in
multi-area configurations, including the backbone router, internal
router, area border router, and autonomous system boundary router.

I detailed the function of different OSPF Link-State Advertisements
(LSAs) and you discovered how these LSAs can be minimized through the
effective implementation of specific OSPF area types. I discussed the
Hello protocols and the different neighbor states experienced when an
adjacency is taking place.

Verification and troubleshooting are very large parts of the objectives,
and I covered everything you need to know in order to verify and
troubleshoot OSPFv2 and meet those requirements.

Finally, we ended the chapter with the easiest part: configuring and
verifying OSPFv3.

\subsection[Exam
Essentials]{\texorpdfstring{\protect\hypertarget{c19.xhtmlux5cux23c19-sec-20}{}{}Exam
Essentials}{Exam Essentials}}

\textbf{Know the scalability issues multi-area OSPF addresses}. The
primary problems in single-area OSPF networks are the large size of the
topology and routing tables as well as the excessive computation of the
SPF algorithm due to the large number of link-state updates that occur
in this single area.

\textbf{Know the different types of OSPF routers}. Backbone routers have
at least one interface in area 0. Area border routers (ABRs) belong to
two or more OSPF areas simultaneously. Internal routers have all of
their interfaces within the same area. Autonomous system boundary
routers (ASBRs) have at least one interface connected to an external
network.

\textbf{Know the different types of LSA packets.} There are seven
different types of LSA packets that Cisco uses, but here are the ones
you need to remember: Type 1 LSAs (router link advertisements), Type 2
LSAs (network link advertisements), Type 3 and 4 LSAs (summary LSAs),
and Type 5 LSAs (AS external link advertisements). Know how each
functions.

\textbf{Be able to monitor multi-area OSPF.} There are a number of
commands that provide information useful in a multi-area OSPF
environment: s\texttt{how\ ip\ route\ ospf},
\texttt{show\ ip\ ospf\ neighbor}, \texttt{show\ ip\ ospf}, and
\texttt{show\ ip\ ospf\ database}. It's important to understand what
each provides.

\protect\hypertarget{c19.xhtmlux5cux23Page_820}{}{}\textbf{Be able to
troubleshoot OSPF networks.} It's important that you can work your way
through the troubleshooting scenario that I presented in this chapter.
Be able to look for neighbor adjacencies, and if they are not there,
look for ACLs set on the routing protocol, passive interfaces, and wrong
network statements.

\textbf{Understand how to configure OSPFv3.} OSPFv3 uses the same basic
mechanisms that OSPFv2 uses, but OSPFv3 is more easily configured by
placing the configuring OSPFv3 on a per-interface basis with
\texttt{ipv6\ ospf\ process-ID\ area\ area}.

\subsection[Written Lab
19]{\texorpdfstring{\protect\hypertarget{c19.xhtmlux5cux23c19-sec-21}{}{}Written
Lab 19}{Written Lab 19}}

You can find the answers to this lab in Appendix A, ``Answers to Written
Labs.''

\begin{enumerate}
\tightlist
\item
  What type of LSAs are sent by an ASBR?
\item
  What state would a router adjacency be in after the INIT state has
  finished?
\item
  What LSA types are sent by ABR toward the area external to the one in
  which they were generated?
\item
  When would you see an adjacency show this: \texttt{FULL/-}?
\item
  True/False: OSPFv3 is configured per area, per interface.
\item
  Which OSPF state uses DBD packets and LSRs?
\item
  Which LSA type is referred to as a router link advertisement (RLA)?
\item
  What is the command to configure OSPFv3 on an interface with process
  ID 1 into area 0?
\item
  What must match exactly between two routers to form an adjacency when
  using OSPFv3?
\item
  How can you see all the routing protocols configured and running on
  your router from user mode?
\end{enumerate}

\subsection[Hands-on
Labs]{\texorpdfstring{\protect\hypertarget{c19.xhtmlux5cux23c19-sec-22}{}{}Hands-on
Labs}{Hands-on Labs}}

In this section, you will use the following network and add OSPF and
OSPFv3 routing.

\begin{figure}
\centering
\includegraphics{images/c19f012.jpg}
\caption{}
\end{figure}

\protect\hypertarget{c19.xhtmlux5cux23Page_821}{}{}The first lab
requires you to configure two routers with OSPF and then verify the
configuration. In the second, you will be asked to enable OSPFv3 routing
on the same network. Note that the labs in this chapter were written to
be used with real equipment---real cheap equipment, that is. As with the
chapter on EIGRP, I wrote these labs with the cheapest, oldest routers I
had lying around so you can see that you don't need expensive gear to
get through some of the hardest labs in this book. However, you can use
the free LammleSim IOS version simulator or Cisco's Packet Tracer to run
through these labs.

The labs in this chapter are as follows:

\begin{enumerate}
\tightlist
\item
  Lab 19.1: Configuring and Verifying Multi-Area OSPF
\item
  Lab 19.2: Configuring and Verifying OSPFv3
\end{enumerate}

\subsubsection[Hands-on Lab 19.1: Configuring and Verifying OSPF
Multi-Area]{\texorpdfstring{\protect\hypertarget{c19.xhtmlux5cux23c19-sec-23}{}{}Hands-on
Lab 19.1: Configuring and Verifying OSPF
Multi-Area}{Hands-on Lab 19.1: Configuring and Verifying OSPF Multi-Area}}

In this lab, you'll configure and verify multi-area OSPF:

\begin{enumerate}
\item
  Implement OSPFv2 on RouterA based on the information in the diagram.

\begin{verbatim}
RouterA#conf t
RouterA(config)#router ospf 10
RouterA(config-router)#network 10.0.0.0 0.255.255.255 area 0
RouterA(config-router)#network 192.168.1.0 0.0.0.255 area 0
\end{verbatim}
\item
  Implement OSPF on RouterB based on the diagram.

\begin{verbatim}
RouterB#conf t
RouterB(config)#router ospf 1
RouterB(config-router)#network 192.168.1.2 0.0.0.0 area 0
RouterB(config-router)#network 10.2.2.0 0.0.0.255 area 1
\end{verbatim}
\item
  Display all the LSAs received on RouterA.

\begin{verbatim}
RouterA#sh ip ospf database
 
            OSPF Router with ID (192.168.1.1) (Process ID 10)
 
                Router Link States (Area 0)
 
Link ID         ADV Router      Age         Seq#       Checksum Link count
10.1.1.2        10.1.1.2        380         0x80000035 0x0012AB 1
192.168.1.1     192.168.1.1     13          0x8000000A 0x00729F 3
192.168.1.2     192.168.1.2     10          0x80000002 0x0090F9 2
 
                Net Link States (Area 0)
 
Link ID         ADV Router      Age         Seq#       Checksum
10.1.1.2        10.1.1.2        381         0x80000001 0x003371
 
                Summary Net Link States (Area 0)
 
Link ID         ADV Router      Age         Seq#       Checksum
10.2.2.0        192.168.1.2     8           0x80000001 0x00C3FD
\end{verbatim}
\item
  Display the routing table for RouterA.

\begin{verbatim}
RouterA#sh ip route
Codes: C - connected, S - static, R - RIP, M - mobile, B - BGP
       D - EIGRP, EX - EIGRP external, O - OSPF, IA - OSPF inter area
       N1 - OSPF NSSA external type 1, N2 - OSPF NSSA external type 2
       E1 - OSPF external type 1, E2 - OSPF external type 2
       i - IS-IS, su - IS-IS summary, L1 - IS-IS level-1, L2 - IS-IS level-2
       ia -IS-IS inter area,* - candidate default,U - per-user static route
       o - ODR, P - periodic downloaded static route
 
Gateway of last resort is not set
 
     10.0.0.0/24 is subnetted, 2 subnets
O IA    10.2.2.0 [110/101] via 192.168.1.2, 00:00:29, Serial0/0
C       10.1.1.0 is directly connected, FastEthernet0/0
     192.168.1.0/30 is subnetted, 1 subnets
C       192.168.1.0 is directly connected, Serial0/0
\end{verbatim}
\item
  Display the neighbor table for RouterA.

\begin{verbatim}
RouterA#sh ip ospf neighbor
 
Neighbor ID     Pri   State        Dead Time   Address        Interface
192.168.1.2       0   FULL/  -     00:00:35    192.168.1.2    Serial0/0
10.1.1.2          1   FULL/DR      00:00:34    10.1.1.2       FastEthernet0/0
\end{verbatim}
\item
  Use the \texttt{show\ ip\ ospf} command on RouterB to see that it is
  an ABR.

\begin{verbatim}
RouterB#sh ip ospf
 Routing Process "ospf 1" with ID 192.168.1.2
 Start time: 1w4d, Time elapsed: 00:07:04.100
 Supports only single TOS(TOS0) routes
 Supports opaque LSA
 Supports Link-local Signaling (LLS)
 Supports area transit capability
 It is an area border router
 Router is not originating router-LSAs with maximum metric
 Initial SPF schedule delay 5000 msecs
 Minimum hold time between two consecutive SPFs 10000 msecs
 Maximum wait time between two consecutive SPFs 10000 msecs
 Incremental-SPF disabled
 Minimum LSA interval 5 secs
 Minimum LSA arrival 1000 msecs
 LSA group pacing timer 240 secs
 Interface flood pacing timer 33 msecs
 Retransmission pacing timer 66 msecs
 Number of external LSA 0. Checksum Sum 0x000000
 Number of opaque AS LSA 0. Checksum Sum 0x000000
 Number of DCbitless external and opaque AS LSA 0
 Number of DoNotAge external and opaque AS LSA 0
 Number of areas in this router is 2. 2 normal 0 stub 0 nssa
 Number of areas transit capable is 0
 External flood list length 0
    Area BACKBONE(0)
        Number of interfaces in this area is 1
        Area has no authentication
        SPF algorithm last executed 00:06:44.492 ago
        SPF algorithm executed 3 times
        Area ranges are
        Number of LSA 5. Checksum Sum 0x020DB1
        Number of opaque link LSA 0. Checksum Sum 0x000000
        Number of DCbitless LSA 0
        Number of indication LSA 0
        Number of DoNotAge LSA 0
        Flood list length 0
    Area 1
        Number of interfaces in this area is 1
        Area has no authentication
        SPF algorithm last executed 00:06:45.640 ago
        SPF algorithm executed 2 times
        Area ranges are
        Number of LSA 3. Checksum Sum 0x00F204
        Number of opaque link LSA 0. Checksum Sum 0x000000
        Number of DCbitless LSA 0
        Number of indication LSA 0
        Number of DoNotAge LSA 0
        Flood list length 0
\end{verbatim}
\end{enumerate}

\subsubsection[Hands-on Lab 19.2: Configuring and Verifying
OSPFv3]{\texorpdfstring{\protect\hypertarget{c19.xhtmlux5cux23c19-sec-24}{}{}\protect\hypertarget{c19.xhtmlux5cux23Page_824}{}{}Hands-on
Lab 19.2: Configuring and Verifying
OSPFv3}{Hands-on Lab 19.2: Configuring and Verifying OSPFv3}}

In this lab, you will configure and verify OSPFv3:

\begin{enumerate}
\item
  Implement OSPFv3 on RouterA. Since the routers have IPv4 addresses, we
  don't need to set the RID of the router.

\begin{verbatim}
RouterA#config t
RouterA(config)#int g0/0
RouterA(config-if)#ipv6 ospf 1 area 0
RouterA(config-if)#int s0/0
RouterA(config-if)#ipv6 ospf 1 area 0
\end{verbatim}
\item
  That's all there is to it! Nice.
\item
  Implement OSPFv3 on RouterB.

\begin{verbatim}
RouterB#config t
RouterB(config)#int s0/0/0
RouterB(config-if)#ipv6 ospf 1 area 0
RouterB(config-if)#int f0/0
RouterB(config-if)#ipv6 ospf 1 area 1
\end{verbatim}
\item
  Again, that's all there is to it!
\item
  Display the routing table for RouterA.

\begin{verbatim}
RouterA#sh ipv6 route ospf
IPv6 Routing Table - 11 entries
Codes: C - Connected, L - Local, S - Static, R - RIP, B - BGP
       U - Per-user Static route
       I1 - ISIS L1, I2 - ISIS L2, IA - ISIS interarea, IS - ISIS summary
       O - OSPF intra, OI - OSPF inter, OE1 - OSPF ext 1, OE2 - OSPF ext 2
       ON1 - OSPF NSSA ext 1, ON2 - OSPF NSSA ext 2
       D - EIGRP, EX - EIGRP external
OI  2001:DB8:3C4D:15::/64 [110/65]
     via FE80::21A:2FFF:FEE7:4398, Serial0/0
\end{verbatim}
\item
  Notice that the one route OSPFv3 found is an inter-area route, meaning
  the network is in another area.
\item
  Display the neighbor table for RouterA.

\begin{verbatim}
RouterA#sh ipv6 ospf neighbor
 
Neighbor ID     Pri   State           Dead Time   Interface ID    Interface
192.168.1.2       1   FULL/  -        00:00:32    6               Serial0/0
\end{verbatim}
\item
  \protect\hypertarget{c19.xhtmlux5cux23Page_825}{}{}Display the
  \texttt{show\ ipv6\ ospf} command on RouterB.

\begin{verbatim}
RouterB#sh ipv6 ospf
 Routing Process "ospfv3 1" with ID 192.168.1.2
 It is an area border router
 SPF schedule delay 5 secs, Hold time between two SPFs 10 secs
 Minimum LSA interval 5 secs. Minimum LSA arrival 1 secs
 LSA group pacing timer 240 secs
 Interface flood pacing timer 33 msecs
 Retransmission pacing timer 66 msecs
 Number of external LSA 0. Checksum Sum 0x000000
 Number of areas in this router is 2. 2 normal 0 stub 0 nssa
 Reference bandwidth unit is 100 mbps
    Area BACKBONE(0)
        Number of interfaces in this area is 1
        SPF algorithm executed 3 times
        Number of LSA 7. Checksum Sum 0x041C1B
        Number of DCbitless LSA 0
        Number of indication LSA 0
        Number of DoNotAge LSA 0
        Flood list length 0
    Area 1
        Number of interfaces in this area is 1
        SPF algorithm executed 2 times
        Number of LSA 5. Checksum Sum 0x02C608
        Number of DCbitless LSA 0
        Number of indication LSA 0
        Number of DoNotAge LSA 0
        Flood list length 0
\end{verbatim}
\end{enumerate}

\subsection[Review
Questions]{\texorpdfstring{\protect\hypertarget{c19.xhtmlux5cux23c19-sec-25}{}{}\protect\hypertarget{c19.xhtmlux5cux23Page_826}{}{}Review
Questions}{Review Questions}}

\begin{center}\rule{0.5\linewidth}{0.5pt}\end{center}

\includegraphics{images/note.png}The following questions are designed to
test your understanding of this chapter's material. For more information
on how to get additional questions, please see
\href{http://www.lammle.com/ccna}{www.lammle.com/ccna}.

\begin{center}\rule{0.5\linewidth}{0.5pt}\end{center}

You can find the answers to these questions in Appendix B, ``Answers to
Review Questions.''

\begin{enumerate}
\item
  Which of the following are scalability issues with single-area OSPF
  networks? (Choose all that apply.)

  \begin{enumerate}
  \def\labelenumii{\Alph{enumii}.}
  \tightlist
  \item
    Size of the routing table
  \item
    Size of the OSPF database
  \item
    Maximum hop-count limitation
  \item
    Recalculation of the OSPF database
  \end{enumerate}
\item
  Which of the following describes a router that connects to an external
  routing process (e.g., EIGRP)?

  \begin{enumerate}
  \def\labelenumii{\Alph{enumii}.}
  \tightlist
  \item
    ABR
  \item
    ASBR
  \item
    Type 2 LSA
  \item
    Stub router
  \end{enumerate}
\item
  Which of the following must match in order for an adjacency to occur
  between routers? (Choose three.)

  \begin{enumerate}
  \def\labelenumii{\Alph{enumii}.}
  \tightlist
  \item
    Process ID
  \item
    Hello and dead timers
  \item
    Link cost
  \item
    Area
  \item
    IP address/subnet mask
  \end{enumerate}
\item
  In which OSPF state do two routers forming an adjacency appear as in
  the \texttt{show\ ip\ ospf\ neighbor} output after adding neighbors
  into the table and exchanging hello information?

  \begin{enumerate}
  \def\labelenumii{\Alph{enumii}.}
  \tightlist
  \item
    ATTEMPT
  \item
    INIT
  \item
    2WAY
  \item
    EXSTART
  \item
    FULL
  \end{enumerate}
\item
  You need to set up a preferred link that OSPF will use to route
  information to a remote ­network. Which command will allow you to set
  the interface link as preferred over another?

  \begin{enumerate}
  \def\labelenumii{\Alph{enumii}.}
  \tightlist
  \item
    \texttt{ip\ ospf\ preferred\ 10}
  \item
    \texttt{ip\ ospf\ priority\ 10}
  \item
    \texttt{ospf\ bandwidth\ 10}
  \item
    \texttt{ip\ ospf\ cost\ 10}
  \end{enumerate}
\item
  When would a router's neighbor table show the FULL/DR state?

  \begin{enumerate}
  \def\labelenumii{\Alph{enumii}.}
  \tightlist
  \item
    After the first Hello packets are received by a neighbor
  \item
    When all information is synchronized among adjacent neighbors
  \item
    When the router's neighbor table is too full of information and is
    discarding neighbor information
  \item
    After the EXSTART state
  \end{enumerate}
\item
  Which is/are true regarding OSPFv3? (Choose all that apply.)

  \begin{enumerate}
  \def\labelenumii{\Alph{enumii}.}
  \tightlist
  \item
    You must add network statements under the OSPF process.
  \item
    There are no network statements in OSPFv3 configurations.
  \item
    OSPFv3 uses a 128-bit RID.
  \item
    If you have IPv4 configured on the router, it is not mandatory that
    you configure the RID.
  \item
    If you don't have IPv4 configured on the router, it is mandatory
    that you configure the RID.
  \item
    OSPFv3 doesn't use LSAs like OSPFv2 does.
  \end{enumerate}
\item
  When a router undergoes the exchange protocol within OSPF, in what
  order does it pass through each state?

  \begin{enumerate}
  \def\labelenumii{\Alph{enumii}.}
  \tightlist
  \item
    EXSTART state \textgreater{} LOADING state \textgreater{} EXCHANGE
    state \textgreater{} FULL state
  \item
    EXSTART state \textgreater{} EXCHANGE state \textgreater{} LOADING
    state \textgreater{} FULL state
  \item
    EXSTART state \textgreater{} FULL state \textgreater{} LOADING state
    \textgreater{} EXCHANGE state
  \item
    LOADING state \textgreater{} EXCHANGE state \textgreater{} FULL
    state \textgreater{} EXSTART state
  \end{enumerate}
\item
  Which type of LSA is generated by DRs and referred to as a network
  link advertisement (NLA)?

  \begin{enumerate}
  \def\labelenumii{\Alph{enumii}.}
  \tightlist
  \item
    Type 1
  \item
    Type 2
  \item
    Type 3
  \item
    Type 4
  \item
    Type 5
  \end{enumerate}
\item
  Which type of LSA is generated by ABRs and is referred to as a summary
  link advertisement (SLA)?

  \begin{enumerate}
  \def\labelenumii{\Alph{enumii}.}
  \tightlist
  \item
    Type 1
  \item
    Type 2
  \item
    Type 3
  \item
    Type 4
  \item
    Type 5
  \end{enumerate}
\item
  Which command will show all the LSAs known by a router?

  \begin{enumerate}
  \def\labelenumii{\Alph{enumii}.}
  \tightlist
  \item
    \texttt{show\ ip\ ospf}
  \item
    \texttt{show\ ip\ ospf\ neighbor}
  \item
    \texttt{show\ ip\ ospf\ interface}
  \item
    \texttt{show\ ip\ ospf\ database}
  \end{enumerate}
\item
  Using the following illustration, what is the cost from R1's routing
  table to reach the network with Server 1? Each Gigabit Ethernet link
  has a cost of 4, and each serial link has a cost of 15.

  \begin{figure}
  \centering
  \includegraphics{images/c19f013.jpg}
  \caption{}
  \end{figure}

  \begin{enumerate}
  \def\labelenumii{\Alph{enumii}.}
  \tightlist
  \item
    100
  \item
    23
  \item
    64
  \item
    19
  \item
    27
  \end{enumerate}
\item
  Using the following illustration, which of the following are true?
  (Choose all that apply.)

  \begin{figure}
  \centering
  \includegraphics{images/c19f014.jpg}
  \caption{}
  \end{figure}

  \begin{enumerate}
  \def\labelenumii{\Alph{enumii}.}
  \tightlist
  \item
    R1 is an internal router.
  \item
    R3 would see the networks connected to the R1 router as an
    inter-area route.
  \item
    R2 is an ASBR.
  \item
    R3 and R4 would receive information from R2 about the backbone area,
    and the same LSA information would be in both LSDBs.
  \item
    R4 is an ABR.
  \end{enumerate}
\item
  \protect\hypertarget{c19.xhtmlux5cux23Page_829}{}{}Which of the
  following could cause two routers to not form an adjacency? (Choose
  all that apply.)

  \begin{enumerate}
  \def\labelenumii{\Alph{enumii}.}
  \tightlist
  \item
    They are configured in different areas.
  \item
    Each router sees the directly connected link as different costs.
  \item
    Two different process IDs are configured.
  \item
    ACL is configured on the routing protocol.
  \item
    There is an IP address/mask mismatch.
  \item
    Passive interface is configured.
  \item
    They both have been configured with the same RID.
  \end{enumerate}
\item
  Which of the following IOS commands shows the state of an adjacency
  with directly connected routers?

  \begin{enumerate}
  \def\labelenumii{\Alph{enumii}.}
  \tightlist
  \item
    \texttt{debug\ ospf\ events}
  \item
    \texttt{show\ ip\ ospf\ border-routers}
  \item
    \texttt{show\ ip\ ospf\ neighbor}
  \item
    \texttt{show\ ip\ ospf\ database}
  \end{enumerate}
\item
  What command will show you the DR and DBR address of the area you are
  connected to directly with an interface?

  \begin{enumerate}
  \def\labelenumii{\Alph{enumii}.}
  \tightlist
  \item
    \texttt{show\ interface\ s0/0/0}
  \item
    \texttt{show\ interface\ fa0/0}
  \item
    \texttt{show\ ip\ ospf\ interface\ s0/0/0}
  \item
    \texttt{show\ ip\ ospf\ interface\ fa0/0}
  \end{enumerate}
\item
  Which of the following could be causing a problem with the Corp router
  not forming an adjacency with its neighbor router? (Choose all that
  apply.)

  \begin{figure}
  \centering
  \includegraphics{images/c19f015.jpg}
  \caption{}
  \end{figure}

  \begin{enumerate}
  \def\labelenumii{\Alph{enumii}.}
  \tightlist
  \item
    \protect\hypertarget{c19.xhtmlux5cux23Page_830}{}{}The routers are
    configured with the wrong network statements.
  \item
    They have different maximum paths configured.
  \item
    There is a passive interface configured.
  \item
    There is an ACL set stopping Hellos.
  \item
    The costs of the links between the routers are configured
    differently.
  \item
    They are in different areas.
  \end{enumerate}
\item
  Which of the following is/are true? (Choose all that apply.)

  \begin{enumerate}
  \def\labelenumii{\Alph{enumii}.}
  \tightlist
  \item
    The reference bandwidth for OSPF and OSPFv3 is 1.
  \item
    The reference bandwidth for OSPF and OSPFv3 is 100.
  \item
    You change the reference bandwidth from global config with the
    command \texttt{auto-cost\ reference\ bandwidth\ number}.
  \item
    You change the reference bandwidth under the OSPF router process
    with the command \texttt{auto-cost\ reference\ bandwidth\ number}.
  \item
    Only one router needs to set the reference bandwidth if it is
    changed from its default.
  \item
    All routers in a single area must set the reference bandwidth if it
    is changed from its default.
  \item
    All routers in the AS must set the reference bandwidth if it is
    changed from its default.
  \end{enumerate}
\item
  Which two statements about the OSPF router ID are true? (Choose two.)

  \begin{enumerate}
  \def\labelenumii{\Alph{enumii}.}
  \tightlist
  \item
    It identifies the source of a Type 1 LSA.
  \item
    It should be the same on all routers in an OSPF routing instance.
  \item
    By default, the lowest IP address on the router becomes the OSPF
    router ID.
  \item
    The router automatically chooses the IP address of a loopback as the
    OSPF router ID.
  \item
    It is created using the MAC address of the loopback interface.
  \end{enumerate}
\item
  What are two benefits of using a single OSPF area network design?
  (Choose two.)

  \begin{enumerate}
  \def\labelenumii{\Alph{enumii}.}
  \tightlist
  \item
    It is less CPU intensive for routers in the single area.
  \item
    It reduces the types of LSAs that are generated.
  \item
    It removes the need for virtual links.
  \item
    It increases LSA response times.
  \item
    It reduces the number of required OSPF neighbor adjacencies.
  \end{enumerate}
\end{enumerate}

\protect\hypertarget{c20.xhtml}{}{}

\section[{Chapter 20}\\
{Troubleshooting IP, IPv6, and
VLANs}]{\texorpdfstring{\protect\hypertarget{c20.xhtmlux5cux23c16}{}{}\protect\hypertarget{c20.xhtmlux5cux23Page_831}{}{}{Chapter
20}\\
{Troubleshooting IP, IPv6, and
VLANs}}{Chapter 20 Troubleshooting IP, IPv6, and VLANs}}

\begin{center}\rule{0.5\linewidth}{0.5pt}\end{center}

\subsection{THE FOLLOWING ICND2 EXAM TOPICS ARE COVERED IN THIS
CHAPTER:}

\begin{enumerate}
\tightlist
\item
  \includegraphics{images/rarr.png}\textbf{1.7 Describe common access
  layer threat mitigation techniques}

  \begin{enumerate}
  \tightlist
  \item
    \includegraphics{images/squ.png} 1.7.c Nondefault native VLAN
  \end{enumerate}
\item
  \includegraphics{images/rarr.png}\textbf{4.0 Infrastructure Services}
\item
  \includegraphics{images/rarr.png}\textbf{4.4 Configure, verify, and
  troubleshoot IPv4 and IPv6 access list for traffic filtering}

  \begin{enumerate}
  \tightlist
  \item
    \includegraphics{images/squ.png} 4.4.a Standard
  \item
    \includegraphics{images/squ.png} 4.4.b Extended
  \item
    \includegraphics{images/squ.png} 4.4.c Named
  \end{enumerate}
\item
  \includegraphics{images/rarr.png}\textbf{5.0 Infrastructure
  Maintenance}
\item
  \includegraphics{images/rarr.png}\textbf{5.2 Troubleshoot network
  connectivity issues using ICMP echo-based IP SLA}
\item
  \includegraphics{images/rarr.png}\textbf{5.3 Use local SPAN to
  troubleshoot and resolve problems}
\end{enumerate}

\protect\hypertarget{c20.xhtmlux5cux23Page_832}{}{}\includegraphics{images/intro.png}
In this chapter, especially at first, it's going to seem like we're
going over lot of the same ground and concepts already covered in other
chapters. The reason for this is that troubleshooting is such a major
focus of the Cisco ICND1 and ICND2 objectives that I've got to make sure
I've guided you through this vital topic in depth. If not, then I just
haven't done all I can to really set you up for success! So to make that
happen, we're going to thoroughly examine troubleshooting with IP, IPv6,
and \emph{virtual LANs (VLANs)} now. And I can't stress the point enough
that you absolutely must have a solid, fundamental understanding of IP
and IPv6 routing as well as a complete understanding of VLANs and
trunking nailed down tight if you're going to win at this!

To help you do that, I'll be using different scenarios to walk you
through the Cisco troubleshooting steps to correctly solve the problems
you're likely to be faced with. Although it's hard to tell exactly what
the ICND1 and ICND2 exams will throw at you, you can read and completely
understand the objectives so that no matter what, you'll be prepared,
equipped, and up to the challenge. The way to do this is by building
upon a really strong foundation, including being skilled at
troubleshooting. This chapter is precisely designed, and exactly what
you need, to seriously help solidify your troubleshooting foundation.

The previous chapters on EIGRP and OSPF each had their own
troubleshooting section. Troubleshooting WAN protocols will be
thoroughly covered in Chapter 21. In this chapter we'll concentrate
solely on IP, IPv6, and VLAN troubleshooting.

\begin{center}\rule{0.5\linewidth}{0.5pt}\end{center}

\includegraphics{images/note.png}To find up-to-the-minute updates for
this chapter, please see
\href{http://www.lammle.com/ccna}{www.lammle.com/ccna} or the book's web
page at \href{http://www.sybex.com/go/ccna}{www.sybex.com/go/ccna}.

\begin{center}\rule{0.5\linewidth}{0.5pt}\end{center}

\subsection[Troubleshooting IP Network
Connectivity]{\texorpdfstring{\protect\hypertarget{c20.xhtmlux5cux23c20-sec-1}{}{}Troubleshooting
IP Network Connectivity}{Troubleshooting IP Network Connectivity}}

Let's start out by taking a moment for a short and sweet review of IP
routing. Always remember that when a host wants to transmit a packet, IP
looks at the destination address and determines if it's a local or
remote request. If it's determined to be a local request, IP just
broadcasts a frame out on the local network looking for the local host
using an ARP request. If it's a remote request, the host sends an ARP
request to the default gateway to discover the MAC address of the
router.

\protect\hypertarget{c20.xhtmlux5cux23Page_833}{}{}Once the hosts have
the default gateway address, they'll send each packet that needs to be
transmitted to the Data Link layer for framing, and newly framed packets
are then sent out on the local collision domain. The router will receive
the frame and remove the packet from the frame, and IP will then parse
the routing table looking for the exit interface on the router. If the
destination is found in the routing table, it will packet-switch the
packet to the exit interface. At this point, the packet will be framed
with new source and destination MAC addresses.

Okay, with that short review in mind, what would you say to someone who
called you saying they weren't able to get to a server on a remote
network? What's the first thing you would have this user do (besides
reboot Windows) or that you would do yourself to test network
connectivity? If you came up with using the Ping program, that's a great
place to start. The Ping program is a great tool for finding out if a
host is alive on the network with a simple ICMP echo request and echo
reply. But being able to ping the host as well as the server doesn't
guarantee that all is well in the network! Keep in mind that there's
more to the Ping program than just being used as a quick and simple
testing protocol.

To be prepared for the exam objectives, it's a great idea to get used to
connecting to various routers and pinging from them. Of course, pinging
from a router is not as good as pinging from the host reporting the
problem, but that doesn't mean we can't isolate some problems from the
routers themselves.

Let's use \protect\hyperlink{c20.xhtmlux5cux23figure20-1}{Figure 20.1}
as a basis to run through some troubleshooting scenarios.

\begin{figure}
\centering
\includegraphics{images/c20f001.jpg}
\caption{{\protect\hyperlink{c20.xhtmlux5cux23figureanchor20-1}{\textbf{Figure
20.1}} Troubleshooting scenario}}
\end{figure}

In this first scenario, a manager calls you and says that he cannot log
in to Server1 from PC1. Your job is to find out why and fix it. The
Cisco objectives are clear on the troubleshooting steps you need to take
when a problem has been reported, and here they are:

\begin{enumerate}
\tightlist
\item
  Check the cables to find out if there's a faulty cable or interface in
  the mix and verify the interface's statistics.
\item
  \protect\hypertarget{c20.xhtmlux5cux23Page_834}{}{}Make sure that
  devices are determining the correct path from the source to the
  destination. Manipulate the routing information if needed.
\item
  Verify that the default gateway is correct.
\item
  Verify that name resolution settings are correct.
\item
  Verify that there are no \emph{access control lists (ACLs)} blocking
  traffic.
\end{enumerate}

In order to effectively troubleshoot this problem, we'll narrow down the
possibilities by process of elimination. We'll start with PC1 and verify
that it's configured correctly and also that IP is working correctly.

There are four steps for checking the PC1 configuration:

\begin{enumerate}
\tightlist
\item
  Test that the local IP stack is working by pinging the loopback
  address.
\item
  Test that the local IP stack is talking to the Data Link layer (LAN
  driver) by pinging the local IP address.
\item
  Test that the host is working on the LAN by pinging the default
  gateway.
\item
  Test that the host can get to remote networks by pinging remote
  Server1.
\end{enumerate}

Let's check out the PC1 configuration by using the \texttt{ipconfig}
command, or \texttt{ifconfig} on a Mac:

\begin{verbatim}
C:\Users\Todd Lammle>ipconfig
 
Windows IP Configuration
 
Ethernet adapter Local Area Connection:
 
   Connection-specific DNS Suffix  . : localdomain
   Link-local IPv6 Address . . . . . : fe80::64e3:76a2:541f:ebcb%11
   IPv4 Address. . . . . . . . . . . : 10.1.1.10
   Subnet Mask . . . . . . . . . . . : 255.255.255.0
   Default Gateway . . . . . . . . . : 10.1.1.1
\end{verbatim}

We can also check the route table on the host with the
\texttt{route\ print} command to see if it truly does know the default
gateway:

\begin{verbatim}
C:\Users\Todd Lammle>route print
[output cut]
IPv4 Route Table
=======================================================================
Active Routes:
Network Destination      Netmask        Gateway       Interface  Metric
          0.0.0.0        0.0.0.0        10.1.1.10     10.1.1.1   10
[output cut]
\end{verbatim}

Between the output of the \texttt{ipconfig} command and the
\texttt{route\ print} command, we can be assured that the hosts are
aware of the correct default gateway.

\protect\hypertarget{c20.xhtmlux5cux23Page_835}{}{}

\begin{center}\rule{0.5\linewidth}{0.5pt}\end{center}

\includegraphics{images/note.png}For the Cisco objectives, it's
extremely important to be able to check and verify the default gateway
on a host and also that this address matches the router's interface!

\begin{center}\rule{0.5\linewidth}{0.5pt}\end{center}

So, let's verify that the local IP stack is initialized by pinging the
loopback address now:

\begin{verbatim}
C:\Users\Todd Lammle>ping 127.0.0.1
 
Pinging 127.0.0.1 with 32 bytes of data:
Reply from 127.0.0.1: bytes=32 time<1ms TTL=128
Reply from 127.0.0.1: bytes=32 time<1ms TTL=128
Reply from 127.0.0.1: bytes=32 time<1ms TTL=128
Reply from 127.0.0.1: bytes=32 time<1ms TTL=128
 
Ping statistics for 127.0.0.1:
    Packets: Sent = 4, Received = 4, Lost = 0 (0% loss),
Approximate round trip times in milli-seconds:
    Minimum = 0ms, Maximum = 0ms, Average = 0ms
\end{verbatim}

This first output confirms the IP address and configured default gateway
of the host, and then I verified the fact that the local IP stack is
working. Our next move is to verify that the IP stack is talking to the
LAN driver by pinging the local IP address.

\begin{verbatim}
C:\Users\Todd Lammle>ping 10.1.1.10
 
Pinging 10.1.1.10 with 32 bytes of data:
Reply from 10.1.1.10: bytes=32 time<1ms TTL=128
Reply from 10.1.1.10: bytes=32 time<1ms TTL=128
Reply from 10.1.1.10: bytes=32 time<1ms TTL=128
Reply from 10.1.1.10: bytes=32 time<1ms TTL=128
 
Ping statistics for 10.1.1.10:
    Packets: Sent = 4, Received = 4, Lost = 0 (0% loss),
Approximate round trip times in milli-seconds:
    Minimum = 0ms, Maximum = 0ms, Average = 0ms
\end{verbatim}

And now that we know the local stack is solid and the IP stack is
communicating to the LAN driver, it's time to check our local LAN
connectivity by pinging the default gateway:

\begin{verbatim}
C:\Users\Todd Lammle>ping 10.1.1.1
 
Pinging 10.1.1.1 with 32 bytes of data:
Reply from 10.1.1.1: bytes=32 time<1ms TTL=128
Reply from 10.1.1.1: bytes=32 time<1ms TTL=128
Reply from 10.1.1.1: bytes=32 time<1ms TTL=128
Reply from 10.1.1.1: bytes=32 time<1ms TTL=128
 
Ping statistics for 10.1.1.1:
    Packets: Sent = 4, Received = 4, Lost = 0 (0% loss),
Approximate round trip times in milli-seconds:
    Minimum = 0ms, Maximum = 0ms, Average = 0ms
\end{verbatim}

Looking good! I'd say our host is in good shape. Let's try to ping the
remote server next to see if our host is actually getting off the local
LAN to communicate remotely:

\begin{verbatim}
C:\Users\Todd Lammle>ping 172.16.20.254
 
Pinging 172.16.20.254 with 32 bytes of data:
Request timed out.
Request timed out.
Request timed out.
Request timed out.
 
Ping statistics for 172.16.20.254:
    Packets: Sent = 4, Received = 0, Lost = 4 (100% loss),
\end{verbatim}

Well, looks like we've confirmed local connectivity but not remote
connectivity, so we're going to have to dig deeper to isolate our
problem. But first, and just as important, it's key to make note of what
we can rule out at this point:

\begin{enumerate}
\tightlist
\item
  The PC is configured with the correct IP address and the local IP
  stack is working.
\item
  The default gateway is configured correctly and the PC's default
  gateway configuration matches the router interface IP address.
\item
  The local switch is working because we can ping through the switch to
  the router.
\item
  We don't have a local LAN issue, meaning our Physical layer is good
  because we can ping the router. If we couldn't ping the router, we
  would need to verify our physical cables and interfaces.
\end{enumerate}

Let's see if we can narrow the problem down further using the
\texttt{traceroute} command:

\begin{verbatim}
C:\Users\Todd Lammle>tracert 172.16.20.254
 
Tracing route to 172.16.20.254 over a maximum of 30 hops
 
  1     1 ms     1 ms    <1 ms  10.1.1.1
  2     *        *        *     Request timed out.
  3     *        *        *     Request timed out.
\end{verbatim}

Well, we didn't get beyond our default gateway, so let's go over to R2
and see if we can talk locally to the server:

\begin{verbatim}
R2#ping 172.16.20.254
 
Pinging 172.16.20.254 with 32 bytes of data:
Reply from 172.16.20.254: bytes=32 time<1ms TTL=128
Reply from 172.16.20.254: bytes=32 time<1ms TTL=128
Reply from 172.16.20.254: bytes=32 time<1ms TTL=128
Reply from 172.16.20.254: bytes=32 time<1ms TTL=128
 
Ping statistics for 172.16.20.254:
    Packets: Sent = 4, Received = 0, Lost = 4 (100% loss),
\end{verbatim}

Okay, we just eliminated a local LAN problem by connecting to Server1
from the R2 router, so we're good there. Let's summarize what we know so
far:

\begin{enumerate}
\tightlist
\item
  PC1 is configured correctly.
\item
  The switch located on the 10.1.1.0 LAN is working.
\item
  PC1's default gateway is configured correctly.
\item
  R2 can communicate to Server1, so we don't have a remote LAN issue.
\end{enumerate}

But something is still clearly wrong, so what should we check now? Now
would be a great time to verify the Server1 IP configuration and make
sure the default gateway is configured correctly. Let's take a look:

\begin{verbatim}
C:\Users\Server1>ipconfig
 
Windows IP Configuration
 
Ethernet adapter Local Area Connection:
 
   Connection-specific DNS Suffix  . : localdomain
   Link-local IPv6 Address . . . . . : fe80::7723:76a2:e73c:2acb%11
   IPv4 Address. . . . . . . . . . . : 172.16.20.254
   Subnet Mask . . . . . . . . . . . : 255.255.255.0
   Default Gateway . . . . . . . . . : 172.16.20.1
\end{verbatim}

Okay---the Server1 configuration looks good and the R2 router can ping
the server, so it seems that the server's local LAN is solid, the local
switch is working, and there are no cable or interface issues. But let's
zoom in on interface Fa0/0 on R2 and talk about what to expect if there
were errors on this interface:

\begin{verbatim}
R2#sh int fa0/0
FastEthernet0/0 is up, line protocol is up
[output cut]
  Full-duplex, 100Mb/s, 100BaseTX/FX
  ARP type: ARPA, ARP Timeout 04:00:00
  Last input 00:00:05, output 00:00:01, output hang never
  Last clearing of "show interface" counters never
  Input queue: 0/75/0/0 (size/max/drops/flushes); Total output drops: 0
  Queueing strategy: fifo
  Output queue: 0/40 (size/max)
  5 minute input rate 0 bits/sec, 0 packets/sec
  5 minute output rate 0 bits/sec, 0 packets/sec
     1325 packets input, 157823 bytes
     Received 1157 broadcasts (0 IP multicasts)
     0 runts, 0 giants, 0 throttles
     0 input errors, 0 CRC, 0 frame, 0 overrun, 0 ignored
     0 watchdog
     0 input packets with dribble condition detected
     2294 packets output, 244630 bytes, 0 underruns
     0 output errors, 0 collisions, 3 interface resets
     347 unknown protocol drops
     0 babbles, 0 late collision, 0 deferred
     4 lost carrier, 0 no carrier
     0 output buffer failures, 0 output buffers swapped out
\end{verbatim}

You've got to be able to analyze interface statistics to find problems
there if they exist, so let's pick out the important factors relevant to
meeting that challenge effectively now.

\textbf{Speed and duplex settings} Good to know that the most common
cause of interface errors is a mismatched duplex mode between two ends
of an Ethernet link. This is why it's so important to make sure that the
switch and its hosts (PCs, router interfaces, etc.) have the same speed
setting. If not, they just won't connect. And if they have mismatched
duplex settings, you'll receive a legion of errors, which cause nasty
performance issues, intermittent connectivity---even total loss of
communication!

Using autonegotiation for speed and duplex is a very common practice,
and it's enabled by default. But if this fails for some reason, you'll
have to set the configuration manually like this:

\begin{verbatim}
Switch(config)#int gi0/1
Switch(config-if)#speed ?
  10    Force 10 Mbps operation
  100   Force 100 Mbps operation
  1000  Force 1000 Mbps operation
  auto  Enable AUTO speed configuration
Switch(config-if)#speed 1000
Switch(config-if)#duplex  ?
  auto  Enable AUTO duplex configuration
  full  Force full duplex operation
  half  Force half-duplex operation
Switch(config-if)#duplex  full
\end{verbatim}

\begin{center}\rule{0.5\linewidth}{0.5pt}\end{center}

\includegraphics{images/c19inline02.png}If you have a duplex mismatch, a
telling sign is that the late collision counter will increment.

\begin{center}\rule{0.5\linewidth}{0.5pt}\end{center}

\protect\hypertarget{c20.xhtmlux5cux23Page_839}{}{}\textbf{Input queue
drops} If the input queue drops counter increments, this signifies that
more traffic is being delivered to the router that it can process. If
this is consistently high, try to determine exactly when these counters
are increasing and how the events relate to CPU usage. You'll see the
ignored and throttle counters increment as well.

\textbf{Output queue drops} This counter indicates that packets were
dropped due to interface congestion, leading to queuing delays. When
this occurs, applications like VoIP will experience performance issues.
If you observe this constantly incrementing, consider QoS.

\textbf{Input errors} Input errors often indicate high errors such as
CRCs. This can point to cabling problems, hardware issues, or duplex
mismatches.

\textbf{Output errors} This is the total number of frames that the port
tried to transmit when an issue such as a collision occurred.

We're going to move on in our troubleshooting process of elimination by
analyzing the routers' actual configurations. Here's R1's routing table:

\begin{verbatim}
R1>sh ip route
[output cut]
Gateway of last resort is 192.168.10.254 to network 0.0.0.0
 
S*    0.0.0.0/0 [1/0] via 192.168.10.254
      10.0.0.0/8 is variably subnetted, 2 subnets, 2 masks
C        10.1.1.0/24 is directly connected, FastEthernet0/0
L        10.1.1.1/32 is directly connected, FastEthernet0/0
      192.168.10.0/24 is variably subnetted, 2 subnets, 2 masks
C        192.168.10.0/24 is directly connected, FastEthernet0/1
L        192.168.10.1/32 is directly connected, FastEthernet0/1
\end{verbatim}

This actually looks pretty good! Both of our directly connected networks
are in the table and we can confirm that we have a default route going
to the R2 router. So now let's verify the connectivity to R2 from R1:

\begin{verbatim}
R1>sh ip int brief
Interface               IP-Address    OK? Method Status              Protocol
FastEthernet0/0         10.1.1.1      YES manual up                    up
FastEthernet0/1         192.168.10.1  YES manual up                    up
Serial0/0/0             unassigned    YES unset  administratively down down
Serial0/1/0             unassigned    YES unset  administratively down down
R1>ping 192.168.10.254
Type escape sequence to abort.
Sending 5, 100-byte ICMP Echos to 192.168.10.254, timeout is 2 seconds:
!!!!!
Success rate is 100 percent (5/5), round-trip min/avg/max = 1/2/4 ms
\end{verbatim}

\protect\hypertarget{c20.xhtmlux5cux23Page_840}{}{}This looks great too!
Our interfaces are correctly configured with the right IP address and
the Physical and Data Link layers are up. By the way, I also tested
layer 3 connectivity by pinging the R2 Fa0/1 interface.

Since everything looks good so far, our next step is to check into the
status of R2's interfaces:

\begin{verbatim}
R2>sh ip int brief
Interface               IP-Address      OK? Method Status              Protocol
FastEthernet0/0         172.16.20.1     YES manual up                    up
FastEthernet0/1         192.168.10.254  YES manual up                    up
R2>ping 192.168.10.1
Type escape sequence to abort.
Sending 5, 100-byte ICMP Echos to 192.168.10.1, timeout is 2 seconds:
!!!!!
Success rate is 100 percent (5/5), round-trip min/avg/max = 1/2/4 ms
\end{verbatim}

Well, everything still checks out at this point. The IP addresses are
correct and the Physical and Data Link layers are up. I also tested the
layer 3 connectivity with a ping to R1, so we're all good so far. We'll
examine the routing table next:

\begin{verbatim}
R2>sh ip route
[output cut]
Gateway of last resort is not set
 
      10.0.0.0/24 is subnetted, 1 subnets
S        10.1.1.0 is directly connected, FastEthernet0/0
      172.16.0.0/16 is variably subnetted, 2 subnets, 2 masks
C        172.16.20.0/24 is directly connected, FastEthernet0/0
L        172.16.20.1/32 is directly connected, FastEthernet0/0
      192.168.10.0/24 is variably subnetted, 2 subnets, 2 masks
C        192.168.10.0/24 is directly connected, FastEthernet0/1
L        192.168.10.254/32 is directly connected, FastEthernet0/1
\end{verbatim}

Okay---we can see that all our local interfaces are in the table, as
well as a static route to the 10.1.1.0 network. But do you see the
problem? Look closely at the static route. The route was entered with an
exit interface of Fa0/0, and the path to the 10.1.1.0 network is out
Fa0/1! Aha! We've found our problem! Let's fix R2:

\begin{verbatim}
R2#config t
R2(config)#no ip route 10.1.1.0 255.255.255.0 fa0/0
R2(config)#ip route 10.1.1.0 255.255.255.0 192.168.10.1
\end{verbatim}

That should do it. Let's verify from PC1:

\begin{verbatim}
C:\Users\Todd Lammle>ping 172.16.20.254
 
Pinging 172.16.20.254 with 32 bytes of data:
Reply from 172.16.20.254: bytes=32 time<1ms TTL=128
Reply from 172.16.20.254: bytes=32 time<1ms TTL=128
Reply from 172.16.20.254: bytes=32 time<1ms TTL=128
Reply from 172.16.20.254: bytes=32 time<1ms TTL=128
 
Ping statistics for 172.16.20.254
    Packets: Sent = 4, Received = 4, Lost = 0 (0% loss),
Approximate round trip times in milli-seconds:
    Minimum = 0ms, Maximum = 0ms, Average = 0ms
\end{verbatim}

Our snag appears to be solved, but just to make sure, we really need to
verify with a higher-level protocol like Telnet:

\begin{verbatim}
C:\Users\Todd Lammle>telnet 172.16.20.254
Connecting To 172.16.20.254...Could not open connection to the host, on
port 23: Connect failed
\end{verbatim}

Okay, that's not good! We can ping to Server1, but we can't telnet to
it. In the past, I've verified that telnetting to this server worked,
but it's still possible that we have a failure on the server side. To
find out, let's verify our network first, starting at R1:

\begin{verbatim}
R1>ping 172.16.20.254
Type escape sequence to abort.
Sending 5, 100-byte ICMP Echos to 172.16.20.254, timeout is 2 seconds:
!!!!!
Success rate is 100 percent (5/5), round-trip min/avg/max = 1/1/4 ms
R1>telnet 172.16.20.254
Trying 172.16.20.254 ...
% Destination unreachable; gateway or host down
\end{verbatim}

This is some pretty ominous output! Let's try from R2 and see what
happens:

\begin{verbatim}
R2#telnet 172.16.20.254
Trying 172.16.20.254 ... Open
 
User Access Verification
 
Password:
\end{verbatim}

Oh my---I can ping the server from a remote network, but I can't telnet
to it; however, the local router R2 can! These factors eliminate the
server being a problem since I can telnet to the server when I'm on the
local LAN.

And we know we don't have a routing problem because we fixed that
already. So what's next? Let's check to see if there's an ACL on R2:

\begin{verbatim}
R2>sh access-lists
Extended IP access list 110
    10 permit icmp any any (25 matches)
\end{verbatim}

Seriously? What a loopy access list to have on a router! This ridiculous
list permits ICMP, but that's it. It denies everything except ICMP due
to the implicit\texttt{deny\ ip\ any\ any} at the end of every ACL. But
before we uncork the champagne, we need to see if this foolish list has
been applied to our interfaces on R2 to confirm that this is really our
problem:

\begin{verbatim}
R2>sh ip int fa0/0
FastEthernet0/0 is up, line protocol is up
  Internet address is 172.16.20.1/24
  Broadcast address is 255.255.255.255
  Address determined by setup command
  MTU is 1500 bytes
  Helper address is not set
  Directed broadcast forwarding is disabled
  Outgoing access list is 110
  Inbound  access list is not set
\end{verbatim}

There it is---that's our problem all right! In case you're wondering why
R2 could telnet to Server1, it's because an ACL filters only packets
trying to go through the router---not packets generated at the router.
Let's get to work and fix this:

\begin{verbatim}
R2#config t
R2(config)#no access-list 110
\end{verbatim}

I just verified that I can telnet from PC1 to Server1, but let's try
telnetting from R1 again:

\begin{verbatim}
R1#telnet 172.16.20.254
Trying 172.16.20.254 ... Open
 
User Access Verification
 
Password:
\end{verbatim}

Nice---looks like we're set, but what about using the name?

\begin{verbatim}
R1#telnet Server1
Translating "Server1"...domain server (255.255.255.255)
 
% Bad IP address or host name
\end{verbatim}

Well, we're not all set just yet. Let's fix R1 so that it can provide
name resolution:

\begin{verbatim}
R1(config)#ip host Server1 172.16.20.254
R1#telnet Server1
Trying Server1 (172.16.20.254)... Open
 
User Access Verification
 
Password:
\end{verbatim}

Great---things are looking good from the router, but if the customer
can't telnet to the remote host using the name, we've got to check the
DNS server to confirm connectivity and for the correct entry to the
server. Another option would be to configure the local host table
manually on PC1.

The last thing to do is to check the server to see if it's responding to
HTTP requests via the \texttt{telnet} command, believe it or not! Here's
an example:

\begin{verbatim}
R1#telnet 172.16.20.254 80
Trying 172.16.20.254, 80 ... Open
\end{verbatim}

Yes---finally! Server1 is responding to requests on port 80, so we're in
the clear.

\subsubsection[Using IP SLA for
Troubleshooting]{\texorpdfstring{\protect\hypertarget{c20.xhtmlux5cux23c20-sec-2}{}{}Using
IP SLA for Troubleshooting}{Using IP SLA for Troubleshooting}}

I want to mention one more thing that can help you troubleshoot your IP
network, and this is using IP service-level agreements (SLAs), which
will allow us to use IP SLA ICMP echo to test far-end devices instead of
pinging manually.

There are several reasons to use the IP SLA measurements:

\begin{enumerate}
\tightlist
\item
  Edge-to-edge network availability monitoring

  \begin{enumerate}
  \tightlist
  \item
    For example, packet loss statistics
  \end{enumerate}
\item
  Network performance monitoring and network performance visibility

  \begin{enumerate}
  \tightlist
  \item
    For example, network latency and response time
  \end{enumerate}
\item
  Troubleshooting basic network operation

  \begin{enumerate}
  \tightlist
  \item
    For example, end-to-end network connectivity
  \end{enumerate}
\end{enumerate}

Step 1: Enable an IP SLA operation that enters the IP SLA configuration
mode. Chose any number from 1 to 2.1 billion as an operation number.

\begin{verbatim}
R1(config)#ip sla 1
\end{verbatim}

Step 2: Configure the IP SLA ICMP echo test and destination.

\begin{verbatim}
R1(config-ip-sla)#icmp?
icmp-echo  icmp-jitter
 
R1(config-ip-sla)#icmp-echo ?
  Hostname or X:X:X:X::X
  Hostname or A.B.C.D  Destination IPv6/IP address or hostname
 
R1(config-ip-sla)#icmp-echo 172.16.20.254
\end{verbatim}

\protect\hypertarget{c20.xhtmlux5cux23Page_844}{}{}Step 3: Set the test
frequency.

\begin{verbatim}
R1(config-ip-sla-echo)#frequency ?
  <1-604800>  Frequency in seconds (default 60)
 
R1(config-ip-sla-echo)#frequency 10
\end{verbatim}

Step 4: Schedule your IP SLA test.

\begin{verbatim}
R1(config-ip-sla-echo)#exit
R1(config)#ip sla schedule ?
  <1-2147483647>  Entry number
 
R1(config)#ip sla schedule 1 life ?
  <0-2147483647>  Life seconds (default 3600)
  forever         continue running forever
 
R1(config)#ip sla schedule 1 life forever start-time ?
  after     Start after a certain amount of time from now
  hh:mm     Start time (hh:mm)
  hh:mm:ss  Start time (hh:mm:ss)
  now       Start now
  pending   Start pending
 
R1(config)#ip sla schedule 1 life forever start-time now
 
\end{verbatim}

Step 5: Verify the IP SLA operation. Use the following commands:

\begin{verbatim}
Show ip sla configuration
Show ip sla statistics
\end{verbatim}

R1 should have an ICMP Echo test configured to the remote server
address, and the test should run every 10 seconds and be scheduled to
run forever.

\begin{verbatim}
R1#show ip sla configuration
IP SLAs Infrastructure Engine-II
Entry number: 1
Owner:
Tag:
Type of operation to perform: icmp-echo
Target address/Source address: 172.16.20.254/0.0.0.0
Type Of Service parameter: 0x0
Request size (ARR data portion): 28
Operation timeout (milliseconds): 5000
Verify data: No
Vrf Name:
Schedule:
   Operation frequency (seconds): 10  (not considered if randomly scheduled)
   Next Scheduled Start Time: Start Time already passed
   Group Scheduled : FALSE
   Randomly Scheduled : FALSE
   Life (seconds): Forever
   Entry Ageout (seconds): never
   Recurring (Starting Everyday): FALSE
   Status of entry (SNMP RowStatus): Active
[output cut]
 
R1#sh ip sla statistics
IPSLAs Latest Operation Statistics
 
IPSLA operation id: 1
Type of operation: icmp-echo
        Latest RTT: 1 milliseconds
Latest operation start time: *15:27:51.365 UTC Mon Jun 6 2016
Latest operation return code: OK
Number of successes: 38
Number of failures: 0
Operation time to live: Forever
\end{verbatim}

The IP SLA 1 test on R1 has been successfully performed 38 times and the
test never failed.

\subsubsection[Using SPAN for
Troubleshooting]{\texorpdfstring{\protect\hypertarget{c20.xhtmlux5cux23c20-sec-3}{}{}Using
SPAN for Troubleshooting}{Using SPAN for Troubleshooting}}

A traffic sniffer can be a valuable tool for monitoring and
troubleshooting your network. However, since the inception of switches
into our networks more than 20 years ago, troubleshooting has become
more difficult because we can't just plug an analyzer into a switch port
and be able to read all the network traffic. Before we had switches, we
used hubs, and when a hub received a digital signal on one port, the hub
sent that digital signal out on all ports except the port it was
received on. This allows a traffic sniffer that is connected to a hub
port to receive all traffic in the network.

Modern local networks are essentially switched networks. After a switch
boots, it starts to build up a layer 2 forwarding table based on the
source MAC addresses of the different packets that the switch receives.
After the switch builds this forwarding table, it forwards traffic that
is destined for a MAC address directly to the exit port. By default,
this prevents a traffic sniffer that is connected to another port from
receiving the unicast traffic. The SPAN feature was therefore introduced
on switches to help solve this problem (see
\protect\hyperlink{c20.xhtmlux5cux23figure20-2}{Figure 20.2}).

\protect\hypertarget{c20.xhtmlux5cux23Page_846}{}{}

\begin{figure}
\centering
\includegraphics{images/c20f002.jpg}
\caption{{\protect\hyperlink{c20.xhtmlux5cux23figureanchor20-2}{\textbf{Figure
20.2}} Using SPAN for troubleshooting}}
\end{figure}

The SPAN feature allows you to analyze network traffic passing through
the port and send a copy of the traffic to another port on the switch
that has been connected to a network analyzer or other monitoring
device. SPAN copies the traffic that the device receives and/or sends on
source ports to a destination port for analysis.

For example, if you would like to analyze the traffic flowing from PC1
to PC2, shown in \protect\hyperlink{c20.xhtmlux5cux23figure20-2}{Figure
20.2}, you need to specify a source port where you want to capture the
data. You can either configure the interface Fa0/1 to capture the
ingress traffic or configure the interface Fa0/3 to capture the egress
traffic---your choice! Next, specify the destination port interface
where the sniffer is connected and will capture the data, in this
example Fa0/2. The traffic flowing from PC1 to PC2 will then be copied
to that interface, and you will be able to analyze it with a traffic
sniffer.

Step 1: Associate a SPAN session number with the source port of what you
want to monitor.

\begin{verbatim}
S1(config)#monitor session 1 source interface f0/1
\end{verbatim}

Step 2: Associate a SPAN session number of the sniffer with the
destination interface.

\begin{verbatim}
S1(config)#monitor session 1 dest interface f0/2
\end{verbatim}

Step 3: Verify that the SPAN session has been configured correctly.

\begin{verbatim}
S1(config)#do sh monitor
Session 1
---------
Type                   : Local Session
Source Ports           :
    Both               : Fa0/1
Destination Ports      : Fa0/2
    Encapsulation      : Native
          Ingress      : Disabled
\end{verbatim}

\protect\hypertarget{c20.xhtmlux5cux23Page_847}{}{}Now connect up your
network analyzer into port F0/2 and enjoy!

\subsubsection[Configuring and Verifying Extended Access
Lists]{\texorpdfstring{\protect\hypertarget{c20.xhtmlux5cux23c20-sec-4}{}{}Configuring
and Verifying Extended Access
Lists}{Configuring and Verifying Extended Access Lists}}

Even though I went through some very basic troubleshooting with ACLs
earlier in this chapter, let's dig a little deeper to make sure we
really understand extended named ACLs before hitting IPv6.

First off, you should be familiar with ACLs from your ICND1 studies; if
not, head back and read that chapter, including the standard and
extended ACLs section. I'm going to focus solely on extended named ACLs,
since that is what the ICND2 objectives are all about.

As you know, standard access lists focus only on IP or IPv6 source
addresses. Extended ACLs, however, filter based on the source \emph{and}
destination layer 3 addresses at a minimum, but in addition can filter
using the protocol field in the IP header (Next Header field in IPv6),
as well as the source and destination port numbers at layer 4, all shown
in \protect\hyperlink{c20.xhtmlux5cux23figure20-3}{Figure 20.3}

\begin{figure}
\centering
\includegraphics{images/c20f003.jpg}
\caption{{\protect\hyperlink{c20.xhtmlux5cux23figureanchor20-3}{\textbf{Figure
20.3}} Extended ACLs}}
\end{figure}

Using the network layout in
\protect\hyperlink{c20.xhtmlux5cux23figure20-4}{Figure 20.1}, let's
create an extended named ACL that blocks Telnet to the 172.16.20.254
server from 10.1.1.10. It's an extended list, so we'll place it closest
to the source address as possible.

Step 1: Test that you can telnet to the remote host.

\begin{verbatim}
R1#telnet 172.16.20.254
Trying 172.16.20.254 ... Open
Server1>
\end{verbatim}

\protect\hypertarget{c20.xhtmlux5cux23Page_848}{}{}Okay, great!

Step 2: Create an ACL on R1 that stops telnetting to the remote host of
172.16.20.254. Using a named ACL, start with the protocol (IP or IPv6),
choose either a standard or extended list, and then name it. The name is
absolutely case sensitive when applying to an interface.

\begin{verbatim}
 
R1(config)#ip access-list extended Block_Telnet
R1(config-ext-nacl)#
\end{verbatim}

Step 3: Once you have created the named list, add your test parameters.

\begin{verbatim}
R1(config-ext-nacl)#deny tcp host 10.1.1.1 host 172.16.20.254 eq 23
R1(config-ext-nacl)#permit ip any any
\end{verbatim}

Step 4: Verify your access list.

\begin{verbatim}
 
R1(config-ext-nacl)#do sh access-list
Extended IP access list Block_Telnet
    10 deny tcp host 10.1.1.1 host 172.16.20.254 eq telnet
    20 permit ip any any
\end{verbatim}

Notice the numbers 10 and 20 on the left side for each test statement.
These are called sequence numbers. We can use these number to then edit
a single line, delete it, or even add a new line in between two sequence
numbers. Named ACLs can be edited; numbered ACLs cannot.

Step 5: Configure your ACL on your router interface.

Since we're adding this to the R1 router in
\protect\hyperlink{c20.xhtmlux5cux23figure20-3}{Figure 20.3}, we'll add
it inbound to interface FastEthernet 0/0, stopping traffic closest to
the source.

\begin{verbatim}
R1(config)#int fa0/0
R1(config-if)#ip access-group Block_Telnet in
\end{verbatim}

Step 6: Test your access list.

\begin{verbatim}
R1#telnet 172.16.20.254
Trying 172.16.20.254 ... Open
Server1>
\end{verbatim}

Hmm\ldots okay, that didn't work because I'm still able to telnet to the
remote host. Let's take a look at our list, verify our interface, and
then fix the problem.

\begin{verbatim}
R1#sh access-list
Extended IP access list Block_Telnet
    10 deny tcp host 10.1.1.1 host 172.16.20.254 eq telnet
    20 permit ip any any
\end{verbatim}

\protect\hypertarget{c20.xhtmlux5cux23Page_849}{}{}By verifying the IP
addresses in the deny statement in line sequence 10, you can see that my
source address is 10.1.1.1 and instead should have been 10.1.1.10.

Step 7: Fix and/or edit your access list. Delete the bad line and
reconfigure the ACL to the correct IP.

\begin{verbatim}
R1(config)#ip access-list extended Block_Telnet
R1(config-ext-nacl)#no 10
R1(config-ext-nacl)#10 deny tcp host 10.1.1.10 host 172.16.20.254 eq 80
\end{verbatim}

Verify that your list is working.

\begin{verbatim}
R1#telnet 172.16.20.254
Trying 172.16.20.254 ...
% Destination unreachable; gateway or host down
\end{verbatim}

Step 8: Display the ACL again and observe the updated hit counters with
each line, and also verify that the interface is set with the ACL.

\begin{verbatim}
R1#sh access-list
Extended IP access list Block_Telnet
    10 deny tcp host 10.1.1.10 host 172.16.20.254 eq telnet (58 matches)
    20 permit ip any any (86 matches)
 
 
R1#sh ip int f0/0
FastEthernet0/0 is up, line protocol is up
  Internet address is 10.10.10.1/24
  Broadcast address is 255.255.255.255
  Address determined by non-volatile memory
  MTU is 1500 bytes
  Helper address is not set
  Directed broadcast forwarding is disabled
  Multicast reserved groups joined: 224.0.0.10
  Outgoing access list is not set
  Inbound  access list is Block_Telnet
  Proxy ARP is enabled
[output cut]
\end{verbatim}

The interface was up and working, so verifying at this point was a
little overkill, but you must be able to look at an interface and
troubleshoot issues, such as ACLs set on an interface. So be sure to
remember the \texttt{show\ ip\ interface} command.

Now, let's mix things up a little by adding IPv6 to our network and work
through the same troubleshooting steps.

\subsection[Troubleshooting IPv6 Network
Connectivity]{\texorpdfstring{\protect\hypertarget{c20.xhtmlux5cux23c20-sec-5}{}{}\protect\hypertarget{c20.xhtmlux5cux23Page_850}{}{}Troubleshooting
IPv6 Network Connectivity}{Troubleshooting IPv6 Network Connectivity}}

I'm going to be straight with you: there isn't a lot that's going to be
much different between this section and the process you just went
through with the IPv4 troubleshooting steps. Except regarding the
addressing of course! So other than that key factor, we'll take the same
approach, using \protect\hyperlink{c20.xhtmlux5cux23figure20-4}{Figure
20.4}, specifically because I really want to highlight the differences
associated with IPv6. So the problem scenario I'm going to use will also
stay the same: PC1 cannot communicate to Server1.

\begin{figure}
\centering
\includegraphics{images/c20f004.jpg}
\caption{{\protect\hyperlink{c20.xhtmlux5cux23figureanchor20-4}{\textbf{Figure
20.4}} IPv6 troubleshooting scenario}}
\end{figure}

I want to point out that this is not an ``introduction to IPv6''
chapter, so I'm assuming you've got some IPv6 fundamentals down.

Notice that I documented both the \emph{link-local} and \emph{global
addresses} assigned to each router interface in
\protect\hyperlink{c20.xhtmlux5cux23figure20-4}{Figure 20.4}. We need
both in order to troubleshoot, so right away, you can see that things
get a bit more complicated because of the longer addresses and the fact
that there are multiple addresses per interface involved!

But \emph{before} we start troubleshooting the IPv6 network in
\protect\hyperlink{c20.xhtmlux5cux23figure20-4}{Figure 20.4}, I want to
refresh your memory on the ICMPv6 protocol, which is an important
protocol in our troubleshooting arsenal.

\subsubsection[ICMPv6]{\texorpdfstring{\protect\hypertarget{c20.xhtmlux5cux23c20-sec-6}{}{}ICMPv6}{ICMPv6}}

IPv4 used the ICMP workhorse for lots of tasks, including error messages
like destination unreachable and troubleshooting functions like Ping and
Traceroute. ICMPv6 still does those things for us, but unlike its
predecessor, the v6 flavor isn't implemented as a separate layer 3
protocol. Instead, it's an integrated part of IPv6 and is carried after
the basic IPv6 header information as an extension header.

ICMPv6 is used for router solicitation and advertisement, for neighbor
solicitation and advertisement (i.e., finding the MAC addresses for IPv6
neighbors), and for redirecting the host to the best router (default
gateway).

\subsubsection[Neighbor Discovery
(NDP)]{\texorpdfstring{\protect\hypertarget{c20.xhtmlux5cux23c20-sec-7}{}{}\protect\hypertarget{c20.xhtmlux5cux23Page_851}{}{}Neighbor
Discovery (NDP)}{Neighbor Discovery (NDP)}}

ICMPv6 also takes over the task of finding the address of other devices
on the local link. The Address Resolution Protocol is used to perform
this function for IPv4, but that's been renamed Neighbor Discovery (ND
or NDP) in ICMPv6. This process is now achieved via a multicast address
called the solicited node address because all hosts join this multicast
group upon connecting to the network.

Neighbor discovery enables these functions:

\begin{enumerate}
\tightlist
\item
  Determining the MAC address of neighbors
\item
  Router solicitation (RS) FF02::2
\item
  Router advertisements (RA) FF02::1
\item
  Neighbor solicitation (NS)
\item
  Neighbor advertisement (NA)
\item
  Duplicate address detection (DAD)
\end{enumerate}

The part of the IPv6 address designated by the 24 bits farthest to the
right is added to the end of the multicast address
FF02:0:0:0:0:1:FF/104. When this address is queried, the corresponding
host will send back its layer 2 address. Devices can find and keep track
of other neighbor devices on the network in pretty much the same way.
When I talked about RA and RS messages earlier in the CCENT chapters,
and told you that they use multicast traffic to request and send address
information, that too is actually a function of ICMPv6---specifically,
neighbor discovery.

In IPv4, the protocol IGMP was used to allow a host device to tell its
local router that it was joining a multicast group and would like to
receive the traffic for that group. This IGMP function has been replaced
by ICMPv6, and the process has been renamed multicast listener
discovery.

With IPv4, our hosts could have only one default gateway configured, and
if that router went down we had to fix the router, change the default
gateway, or run some type of virtual default gateway with other
protocols created as a solution for this inadequacy in IPv4.
\protect\hyperlink{c20.xhtmlux5cux23figure20-5}{Figure 20.5} shows how
IPv6 devices find their default gateways using neighbor discovery.

\begin{figure}
\centering
\includegraphics{images/c20f005.jpg}
\caption{{\protect\hyperlink{c20.xhtmlux5cux23figureanchor20-5}{\textbf{Figure
20.5}} Router solicitation (RS) and router advertisement (RA)}}
\end{figure}

IPv6 hosts send a router solicitation (RS) onto their data link asking
for all routers to respond, and they use the multicast address FF02::2
to achieve this. Routers on the same link respond with a unicast to the
requesting host, or with a router advertisement (RA) using FF02::1.

\protect\hypertarget{c20.xhtmlux5cux23Page_852}{}{}But that's not all!
Hosts also can send solicitations and advertisements between themselves
using a neighbor solicitation (NS) and neighbor advertisement (NA), as
shown in \protect\hyperlink{c20.xhtmlux5cux23figure20-6}{Figure 20.6}.

\begin{figure}
\centering
\includegraphics{images/c20f006.jpg}
\caption{{\protect\hyperlink{c20.xhtmlux5cux23figureanchor20-6}{\textbf{Figure
20.6}} Neighbor solicitation (NS) and neighbor advertisement (NA)}}
\end{figure}

Remember that RA and RS gather or provide information about routers and
NS and NA gather information about hosts. Also, remember that a
``neighbor'' is a host on the same data link or VLAN.

With that foundation review in mind, here are the troubleshooting steps
we'll progress through in our investigation:

\begin{enumerate}
\tightlist
\item
  Check the cables because there might be a faulty cable or interface.
  Verify interface ­statistics.
\item
  Make sure that devices are determining the correct path from the
  source to the destination. Manipulate the routing information if
  needed.
\item
  Verify that the default gateway is correct.
\item
  Verify that name resolution settings are correct, and especially for
  IPv6, make sure the DNS server is reachable via IPv4 and IPv6.
\item
  Verify that there are no ACLs that are blocking traffic.
\end{enumerate}

In order to troubleshoot this problem, we'll use the same process of
elimination, beginning with PC1. We must verify that it's configured
correctly and that IP is working properly. Let's start by pinging the
loopback address to verify the IPv6 stack:

\begin{verbatim}
C:\Users\Todd Lammle>ping ::1
 
Pinging ::1 with 32 bytes of data:
Reply from ::1: time<1ms
Reply from ::1: time<1ms
Reply from ::1: time<1ms
Reply from ::1: time<1ms
\end{verbatim}

Well, the IPv6 stack checks out, so let's ping the Fa0/0 of R1, which
PC1 is directly connected to on the same LAN, starting with the
link-local address:

\begin{verbatim}
C:\Users\Todd Lammle>ping fe80::21a:6dff:fe37:a44e
 
Pinging fe80:21a:6dff:fe37:a44e with 32 bytes of data:
Reply from fe80::21a:6dff:fe37:a44e: time<1ms
Reply from fe80::21a:6dff:fe37:a44e: time<1ms
Reply from fe80::21a:6dff:fe37:a44e: time<1ms
Reply from fe80::21a:6dff:fe37:a44e: time<1ms
\end{verbatim}

Next, we'll ping the global address on Fa0/0:

\begin{verbatim}
C:\Users\Todd Lammle>ping 2001:db8:3c4d:3:21a:6dff:fe37:a44e
 
Pinging 2001:db8:3c4d:3:21a:6dff:fe37:a44e with 32 bytes of data:
Reply from 2001:db8:3c4d:3:21a:6dff:fe37:a44e: time<1ms
Reply from 2001:db8:3c4d:3:21a:6dff:fe37:a44e: time<1ms
Reply from 2001:db8:3c4d:3:21a:6dff:fe37:a44e: time<1ms
Reply from 2001:db8:3c4d:3:21a:6dff:fe37:a44e: time<1ms
\end{verbatim}

Okay---looks like PC1 is configured and working on the local LAN to the
R1 router, so we've confirmed the Physical, Data Link, and Network
layers between the PC1 and the R1 router Fa0/0 interface.

Our next move is to check the local connection on Server1 to the R2
router to verify that LAN. First we'll ping the link-local address of
the router from Server1:

\begin{verbatim}
C:\Users\Server1>ping fe80::21a:6dff:fe64:9b2
 
Pinging fe80::21a:6dff:fe64:9b2  with 32 bytes of data:
Reply from fe80::21a:6dff:fe64:9b2: time<1ms
Reply from fe80::21a:6dff:fe64:9b2: time<1ms
Reply from fe80::21a:6dff:fe64:9b2: time<1ms
Reply from fe80::21a:6dff:fe64:9b2: time<1ms
\end{verbatim}

And next, we'll ping the global address of Fa0/0 on R2:

\begin{verbatim}
C:\Users\Server1>ping 2001:db8:3c4d:1:21a:6dff:fe37:a443
 
Pinging 2001:db8:3c4d:1:21a:6dff:fe37:a443 with 32 bytes of data:
Reply from 2001:db8:3c4d:1:21a:6dff:fe37:a443: time<1ms
Reply from 2001:db8:3c4d:1:21a:6dff:fe37:a443: time<1ms
Reply from 2001:db8:3c4d:1:21a:6dff:fe37:a443: time<1ms
Reply from 2001:db8:3c4d:1:21a:6dff:fe37:a443: time<1ms
\end{verbatim}

Let's quickly summarize what we know at this point:

\begin{enumerate}
\tightlist
\item
  By using the \texttt{ipconfig\ /all} command on PC1 and Server1, I was
  able to document their global and link-local IPv6 addresses.
\item
  We know the IPv6 link-local addresses of each router interface.
\item
  We know the IPv6 global address of each router interface.
\item
  \protect\hypertarget{c20.xhtmlux5cux23Page_854}{}{}We can ping from
  PC1 to router R1's Fa0/0 interface.
\item
  We can ping from Server1 to router R2's Fa0/0 interface.
\item
  We can eliminate a local problem on both LANs.
\end{enumerate}

From here, we'll go to PC1 and see if we can route to Server1:

\begin{verbatim}
C:\Users\Todd Lammle>tracert 2001:db8:3c4d:1:a14c:8c33:2d1:be3d
 
Tracing route to 2001:db8:3c4d:1:a14c:8c33:2d1:be3d over a maximum of 30 hops
 
  1     Destination host unreachable.
\end{verbatim}

Okay, now that's not good. Looks like we might have a routing problem.
And on this little network, we're doing static IPv6 routing, so getting
to the bottom of things will definitely take a little effort! But before
we start looking into our potential routing issue, let's check the link
between R1 and R2. We'll ping R2 from R1 to test the directly connected
link.

The first thing you need to do before attempting to ping between routers
is verify your addresses---yes, verify them again! Let's check out both
routers, then try pinging from R1 to R2:

\begin{verbatim}
R1#sh ipv6 int brief
FastEthernet0/0            [up/up]
    FE80::21A:6DFF:FE37:A44E
    2001:DB8:3C4D:3:21A:6DFF:FE37:A44E
FastEthernet0/1            [up/up]
    FE80::21A:6DFF:FE37:A44F
    2001:DB8:3C4D:2:21A:6DFF:FE37:A44F
 
R2#sh ipv6 int brief
FastEthernet0/0            [up/up]
    FE80::21A:6DFF:FE64:9B2
    2001:DB8:3C4D:1:21A:6DFF:FE37:A443
FastEthernet0/1            [up/up]
    FE80::21A:6DFF:FE64:9B3
    2001:DB8:3C4D:2:21A:6DFF:FE64:9B3
 
R1#ping 2001:DB8:3C4D:2:21A:6DFF:FE64:9B3
Type escape sequence to abort.
Sending 5, 100-byte ICMP Echos to ping 2001:DB8:3C4D:2:21A:6DFF:FE64:9B3, timeout
is 2 seconds:
!!!!!
Success rate is 100 percent (5/5), round-trip min/avg/max = 0/2/8 ms
\end{verbatim}

\protect\hypertarget{c20.xhtmlux5cux23Page_855}{}{}In the preceding
output, you can see that I now have the IPv6 addresses for both the R1
and R2 directly connected interfaces. The output also shows that I used
the Ping program to verify layer 3 connectivity. Just as with IPv4, we
need to resolve the logical (IPv6) address to a MAC address in order to
communicate on the local LAN. But unlike IPv4, IPv6 doesn't use ARP---it
uses ICMPv6 neighbor solicitations instead---so after the successful
ping, we can now see the neighbor resolution table on R1:

\begin{verbatim}
R1#sh ipv6 neighbors
IPv6 Address                          Age Link-layer Addr State Interface
FE80::21A:6DFF:FE64:9B3                 0 001a.6c46.9b09  DELAY Fa0/1
2001:DB8:3C4D:2:21A:6DFF:FE64:9B3       0 001a.6c46.9b09  REACH Fa0/1
\end{verbatim}

Let's take a minute to talk about the possible states that a resolved
address shows us:

\textbf{INCMP (incomplete)} Address resolution is being performed on the
entry. A neighbor solicitation message has been sent, but the neighbor
message has not yet been received.

\textbf{REACH (reachable)} Positive confirmation has been received
confirming that the path to the neighbor is functioning correctly. REACH
is good!

\textbf{STALE} The state is STALE when the interface has not
communicated within the neighbor reachable time frame. The next time the
neighbor communicates, the state will change back to REACH.

\textbf{DELAY} Occurs after the STALE state, when no reachability
confirmation has been received within what's known as the
DELAY\_FIRST\_PROBE\_TIME. This means that the path was functioning but
it hasn't had communication within the neighbor reachable time frame.

\textbf{PROBE} When in PROBE state, the configured interface is
resending a neighbor solicitation and waiting for a reachability
confirmation from a neighbor.

We can verify our default gateway with IPv6 with the \texttt{ipconfig}
command like this:

\begin{verbatim}
C:\Users\Todd Lammle>ipconfig
   Connection-specific DNS Suffix  . : localdomain
   IPv6 Address. . . . . . . . . . . : 2001:db8:3c4d:3:ac3b:2ef:1823:8938
   Temporary IPv6 Address. . . . . . : 2001:db8:3c4d:3:2f33:44dd:211:1c3d
   Link-local IPv6 Address . . . . . : fe80::ac3b:2ef:1823:8938%11
   IPv4 Address. . . . . . . . . . . : 10.1.1.10
   Subnet Mask . . . . . . . . . . . : 255.255.255.0
   Default Gateway . . . . . . . . . : Fe80::21a:6dff:fe37:a44e%11
      10.1.1.1
\end{verbatim}

It's important to understand that the default gateway will be the
link-local address of the router, and in this case, we can see that the
address the host learned is truly the link-local address of the Fa0/0
interface of R1. The \texttt{\%11} is just used to identify an interface
and isn't used as part of the IPv6 address.

\begin{center}\rule{0.5\linewidth}{0.5pt}\end{center}

\subsection[Temporary IPv6
Addresses]{\texorpdfstring{\protect\hypertarget{c20.xhtmlux5cux23Page_856}{}{}Temporary
IPv6 Addresses}{Temporary IPv6 Addresses}}

The temporary IPv6 address, listed under the unicast IPv6 address as
2001:db8:3c4d:3:2f33:44dd:211:1c3d, was created by Windows to provide
privacy from the EUI-64 format. This creates a global address for your
host without using your MAC address by generating a random number for
the interface and hashing it; the result is then appended to the /64
prefix from the router. You can disable this feature with the following
commands:

\begin{verbatim}
netsh interface ipv6 set global randomizeidentifiers=disabled
netsh interface ipv6 set privacy state-disabled
\end{verbatim}

\begin{center}\rule{0.5\linewidth}{0.5pt}\end{center}

In addition to the \texttt{ipconfig} command, we can use the command
\texttt{netsh\ interface\ ipv6\ show\ neighbor} to verify our default
gateway address:

\begin{verbatim}
C:\Users\Todd Lammle>netsh interface ipv6 show neighbor
[output cut]
 
Interface 11: Local Area Connection
 
Internet Address                              Physical Address   Type
--------------------------------------------  -----------------  -----------
2001:db8:3c4d:3:21a:6dff:fe37:a44e            00-1a-6d-37-a4-4e  (Router)
Fe80::21a:6dff:fe37:a44e                      00-1a-6d-37-a4-4e  (Router)
ff02::1                                       33-33-00-00-00-01  Permanent
ff02::2                                       33-33-00-00-00-02  Permanent
ff02::c                                       33-33-00-00-00-0c  Permanent
ff02::16                                      33-33-00-00-00-16  Permanent
ff02::fb                                      33-33-00-00-00-fb  Permanent
ff02::1:2                                     33-33-00-01-00-02  Permanent
ff02::1:3                                     33-33-00-01-00-03  Permanent
ff02::1:ff1f:ebcb                             33-33-ff-1f-eb-cb  Permanent
\end{verbatim}

\begin{center}\rule{0.5\linewidth}{0.5pt}\end{center}

\includegraphics{images/note.png}I've checked the default gateway
addresses on Server1 and they are correct. They should be, because this
is provided directly from the router with an ICMPv6 RA (router
advertisement) message. The output for that verification is not shown.

\begin{center}\rule{0.5\linewidth}{0.5pt}\end{center}

Let's establish the information we have right now:

\begin{enumerate}
\tightlist
\item
  Our PC1 and Server1 configurations are working and have been verified.
\item
  The LANs are working and verified, so there is no Physical layer
  issue.
\item
  \protect\hypertarget{c20.xhtmlux5cux23Page_857}{}{}The default
  gateways are correct.
\item
  The link between the R1 and R2 routers is working and verified.
\end{enumerate}

So all this tells us is that it's now time to check our routing tables!
We'll start with the R1 router:

\begin{verbatim}
R1#sh ipv6 route
C   2001:DB8:3C4D:2::/64 [0/0]
     via FastEthernet0/1, directly connected
L   2001:DB8:3C4D:2:21A:6DFF:FE37:A44F/128 [0/0]
     via FastEthernet0/1, receive
C   2001:DB8:3C4D:3::/64 [0/0]
     via FastEthernet0/0, directly connected
L   2001:DB8:3C4D:3:21A:6DFF:FE37:A44E/128 [0/0]
     via FastEthernet0/0, receive
L   FF00::/8 [0/0]
     via Null0, receive
\end{verbatim}

All we can see in the output is the two directly connected interfaces
configured on the router, and that won't help us send IPv6 packets to
the 2001:db8:3c4d:1::/64 subnet off of Fa0/0 on R2. So let's find out
what R2 can tell us:

\begin{verbatim}
R2#sh ipv6 route
C   2001:DB8:3C4D:1::/64 [0/0]
     via FastEthernet0/0, directly connected
L   2001:DB8:3C4D:1:21A:6DFF:FE37:A443/128 [0/0]
     via FastEthernet0/0, receive
C   2001:DB8:3C4D:2::/64 [0/0]
     via FastEthernet0/1, directly connected
L   2001:DB8:3C4D:2:21A:6DFF:FE64:9B3/128 [0/0]
     via FastEthernet0/1, receive
S   2001:DB8:3C4D:3::/64 [1/0]
     via 2001:DB8:3C4D:2:21B:D4FF:FE0A:539
L   FF00::/8 [0/0]
     via Null0, receive
\end{verbatim}

Now we're talking---that tells us a lot more than R1's table did! We
have both of our directly connected configured LANs, Fa0/0 and Fa0/1,
right there in the routing table, as well as a static route to
2001:DB8:3C4D:3::/64, which is the remote LAN Fa0/0 off of R1, which is
good. Now let's fix the route problem on R1 by adding a route that gives
us access to the Server1 network.

\begin{verbatim}
R1(config)#ipv6 route ::/0 fastethernet 0/1 FE80::21A:6DFF:FE64:9B3
\end{verbatim}

\protect\hypertarget{c20.xhtmlux5cux23Page_858}{}{}I want to point out
that I didn't need to make the default route as difficult as I did. I
entered both the exit interface and next-hop link-local address when
just the exit interface or next-hop global addresses would be mandatory,
but not the link-local. So it could have simply just been this:

\begin{verbatim}
R1(config)#ipv6 route ::/0 fa0/1
\end{verbatim}

Next, we'll verify that we can now ping from PC1 to Server1:

\begin{verbatim}
C:\Users\Todd Lammle>ping 2001:db8:3c4d:1:a14c:8c33:2d1:be3d
 
Pinging 2001:db8:3c4d:1:a14c:8c33:2d1:be3d with 32 bytes of data:
Reply from 2001:db8:3c4d:1:a14c:8c33:2d1:be3d: time<1ms
Reply from 2001:db8:3c4d:1:a14c:8c33:2d1:be3d: time<1ms
Reply from 2001:db8:3c4d:1:a14c:8c33:2d1:be3d: time<1ms
Reply from 2001:db8:3c4d:1:a14c:8c33:2d1:be3d: time<1ms
\end{verbatim}

Sweet---we're looking golden with this particular scenario! But know
that it is still possible to have name resolution issues. If that were
the case, you would just need to check your DNS server or local host
table.

Moving on in the same way we did in the IPv4 troubleshooting section,
it's a good time to check into your ACLs, especially if you're still
having a problem after troubleshooting all your local LANs and all other
potential routing issues.

\subsubsection[Troubleshooting IPv6 Extended Access
Lists]{\texorpdfstring{\protect\hypertarget{c20.xhtmlux5cux23c20-sec-8}{}{}Troubleshooting
IPv6 Extended Access Lists}{Troubleshooting IPv6 Extended Access Lists}}

Let's create an extended IPv6 ACL on R2, pretty much just like we did in
the IPv4 troubleshooting section.

First, understand that you can only create named extended IPv6 ACLs, so
you don't need to specify standard or extended in your named list, and
although you won't see any sequence numbers, you can still somewhat edit
a named IPv6 ACL, meaning you can delete a single line but there is no
way to insert a line other than at the end of the ACL.

In addition, every IPv4 access list has an implicit
\texttt{deny\ ip\ any\ any} at the bottom; however, IPv6 access lists
actually have \emph{three} implied statements at the bottom:

\begin{enumerate}
\tightlist
\item
  \texttt{permit\ icmp\ any\ any\ nd-na}
\item
  \texttt{permit\ icmp\ any\ any\ nd-ns}
\item
  \texttt{deny\ ipv6\ any\ any}
\end{enumerate}

The two permit statements are required for neighbor discovery, which is
an important protocol in IPv6, because it's the replacement for ARP.

Using the network layout and IPv6 addresses in
\protect\hyperlink{c20.xhtmlux5cux23figure20-4}{Figure 20.4}, let's
create an IPv6 extended named ACL that blocks Telnet to Server1 (with an
IPv6 address of 2001:db8:3c4d:1:a14c:8c33:2d1:be3d) from PC1 (with a
destination IPv6 address of 2001:db8:3c4d:3:2ef:1823:8938). Since it's
an IPv6 extended named ACL (always), we'll place it closest to the
source address if possible.

\protect\hypertarget{c20.xhtmlux5cux23Page_859}{}{}Step 1: Test that you
can telnet to the remote host.

\begin{verbatim}
R1#telnet 2001:db8:3c4d:1:a14c:8c33:2d1:be3d
Trying 2001:db8:3c4d:1:a14c:8c33:2d1:be3d... Open
 
Server1>
\end{verbatim}

Okay, great---but that was way too much effort! Let's create an entry
into the hosts table of R1 so we don't have to type an IPv6 address when
trying to access that host.

\begin{verbatim}
R1(config)#ipv6 host Server1 2001:db8:3c4d:1:a14c:8c33:2d1:be3d
R1(config)#do sh host
[output cut]
 
Host Port Flags Age Type Address(es)
Server1 None (perm, OK) 0 IPV6 2001:DB8:3C4D:1:A14C:8C33:2D1:BE3D
 
\end{verbatim}

Now we can just type this from now on (the name is case sensitive).

\begin{verbatim}
R1#telnet Server1
Trying 2001:DB8:3C4D:1:A14C:8C33:2D1:BE3D... Open
 
Server1>
\end{verbatim}

Or better yet, just the name (Telnet is the default).

\begin{verbatim}
R1#Server1
Trying 2001:DB8:3C4D:1:A14C:8C33:2D1:BE3D... Open
 
Server1>exit
 
\end{verbatim}

Also, ping using the name.

\begin{verbatim}
R1#ping Server1
Type escape sequence to abort.
Sending 5, 100-byte ICMP Echos to 2001:DB8:3C4D:1:A14C:8C33:2D1:BE3D, timeout is 2 seconds:
!!!!!
Success rate is 100 percent (5/5), round-trip min/avg/max = 0/0/1 ms
\end{verbatim}

Step 2: Create an ACL on R2 that stops Telnet to the remote host Server1
(2001:db8:3c4d:1:a14c:8c33:2d1:be3d). The name is absolutely case
sensitive when applying to an interface.

\begin{verbatim}
 
R2(config)#ipv6 access-list Block_Telnet
R2(config-ipv6-acl)#
\end{verbatim}

Step 3: Once you have created the named list, add your test parameters.

\begin{verbatim}
R2(config-ipv6-acl)#deny tcp host 2001:DB8:3C4D:2:21A:6DFF:FE37:A44F host 2001:DB8:3C4D:1:A14C:8C33:2D1:BE3D eq telnet
R2(config-ipv6-acl)#permit ipv6 any any
\end{verbatim}

Step 4: Configure your ACL on your router interface.

Since we're adding this to the R2 router in
\protect\hyperlink{c20.xhtmlux5cux23figure20-4}{Figure 20.4}, we'll add
it to interface FastEthernet 0/1, stopping traffic closest to the
source, and use the command \texttt{ipv6\ traffic-filter}.

\begin{verbatim}
R2(config)#int fa0/1
R2(config-if)#ipv6 traffic-filter Block_Telnet out
\end{verbatim}

Step 5: Test your access list by telnetting from Server1 on the R1
router.

\begin{verbatim}
R1#Server1
Trying 2001:DB8:3C4D:1:A14C:8C33:2D1:BE3D ...Open
 
 
Server1>
\end{verbatim}

Hmm\ldots{} and I tried really hard not to make a typo! Let's take a
look.

\begin{verbatim}
R2#sh access-lists
IPv6 access list Block_Telnet
      deny tcp host 2001:DB8:3C4D:2:21A:6DFF:FE37:A44F host 2001:DB8:3C4D:1:A14C:8C33:2D1:BE3D eq telnet (96 match(es))
      permit ipv6 any any (181 match(es))
\end{verbatim}

By verifying the IPv6 addresses with the interfaces of the routers, this
list looks correct. It's important to verify your addresses with a
\texttt{show\ ipv6\ interface\ brief} command. Let's take a look.

\begin{verbatim}
R1#sh ipv6 int brief
FastEthernet0/0 [up/up]
      FE80::2E0:B0FF:FED2:B701
      2001:DB8:3C4D:3:21A:6DFF:FE37:A44E
FastEthernet0/1 [up/up]
      FE80::2E0:B0FF:FED2:B702
      2001:DB8:3C4D:2:21A:6DFF:FE37:A44F
\end{verbatim}

Since R1 Fa0/1 is my source address, we can see that this address is
correct in my ACL. Let's take a look at the destination device.

\begin{verbatim}
Server1#sh ipv6 int br
FastEthernet0/0 [up/up]
      FE80::260:70FF:FED8:DD01
      2001:DB8:3C4D:1:A14C:8C33:2D1:BE3D
\end{verbatim}

\protect\hypertarget{c20.xhtmlux5cux23Page_861}{}{}Yup, this one is
correct too! My IPv6 ACL is correct, so now we need to check our
interface.

Step 6: Fix and/or edit your access list and/or interfaces.

\begin{verbatim}
R2#show running-config
[output cut]
!
interface FastEthernet0/0
      no ip address
      duplex auto
      speed auto
      ipv6 address 2001:DB8:3C4D:1:21A:6DFF:FE37:A443/64
      ipv6 rip 1 enable
!
interface FastEthernet0/1
      no ip address
      ipv6 traffic-filter Block_Telnet out
      duplex auto
      speed auto
      ipv6 address 2001:DB8:3C4D:2:21A:6DFF:FE64:9B3/64
      ipv6 rip 1 enable
!
\end{verbatim}

Unlike IPv4, where we can use the \texttt{show\ ip\ interface} command
to see if an ACL is set, we can only use the
\texttt{show\ running-config} command to verify if an IPv6 ACL is set on
an interface. In the above output, we can see that I certainly did set
the ACL to the interface Fa0/1, but I configured it to \texttt{out}
instead of \texttt{in} on the interface. Let's fix that.

\begin{verbatim}
R2#config t
R2(config)#int fa0/1
R2(config-if)#no ipv6 traffic-filter Block_Telnet out
R2(config-if)#ipv6 traffic-filter Block_Telnet in
\end{verbatim}

Step 7: Retest your ACL.

\begin{verbatim}
R1#Server1
 
Trying 2001:DB8:3C4D:1:A14C:8C33:2D1:BE3D ...% Connection timed out; remote host not responding
R1#
 
\end{verbatim}

Looks good! Although I don't recommend using this method to block Telnet
to a router, it was a great way to test our IPv6 ACLs.

\subsection[Troubleshooting VLAN
Connectivity]{\texorpdfstring{\protect\hypertarget{c20.xhtmlux5cux23c20-sec-9}{}{}\protect\hypertarget{c20.xhtmlux5cux23Page_862}{}{}Troubleshooting
VLAN Connectivity}{Troubleshooting VLAN Connectivity}}

You know by now that VLANs are used to break up broadcast domains in a
layer 2 switched network. You've also learned that we assign ports on a
switch into a VLAN broadcast domain by using the
\texttt{switchport\ access\ vlan} command.

The access port carries traffic for a single VLAN that the port is a
member of. If members of one VLAN want to communicate to members in the
same VLAN that are located on a different switch, then a port between
the two switches needs to be either configured to be a member of this
single VLAN or configured as a trunk link, which passes information on
all VLANs by default.

We're going to use
\protect\hyperlink{c20.xhtmlux5cux23figure20-7}{Figure 20.7} to
reference as we go through the procedures for troubleshooting VLAN and
trunking.

\begin{figure}
\centering
\includegraphics{images/c20f007.jpg}
\caption{{\protect\hyperlink{c20.xhtmlux5cux23figureanchor20-7}{\textbf{Figure
20.7}} VLAN connectivity}}
\end{figure}

I'm going to begin with VLAN troubleshooting and then move on to trunk
troubleshooting.

\subsubsection[VLAN
Troubleshooting]{\texorpdfstring{\protect\hypertarget{c20.xhtmlux5cux23c20-sec-10}{}{}VLAN
Troubleshooting}{VLAN Troubleshooting}}

A couple of key times to troubleshoot VLANs are when and if you lose
connectivity between hosts and when you're configuring new hosts into a
VLAN but they're not working.

Here are the steps we'll follow to troubleshoot VLANs:

\begin{enumerate}
\tightlist
\item
  Verify the VLAN database on all your switches.
\item
  Verify your \emph{Content Addressable Memory (CAM)} table.
\item
  Verify that your port VLAN assignments are configured correctly.
\end{enumerate}

And here's a list of the commands we'll be using in the following
sections:

\begin{verbatim}
Show vlan
Show mac address-table
Show interfaces interface switchport
switchport access vlan vlan
\end{verbatim}

\paragraph{VLAN Troubleshooting Scenario}

A manager calls and says they can't communicate to the new sales team
member that just connected to the network. How would you proceed to
solve this issue? Well, because the
\protect\hypertarget{c20.xhtmlux5cux23Page_863}{}{}sales hosts are in
VLAN 10, we'll begin with step 1 and verify that our databases on both
switches are correct.

First, I'll use the \texttt{show\ vlan} or \texttt{show\ vlan\ brief}
command to check if the expected VLAN is actually in the database.
Here's a look at the VLAN database on S1:

\begin{verbatim}
S1#sh vlan
 
VLAN Name                             Status    Ports
---- -------------------------------- --------- -------------------------------
1    default                          active    Gi0/3, Gi0/4, Gi0/5, Gi0/6
                                                Gi0/7, Gi0/8, Gi0/9, Gi0/10
                                                Gi0/11, Gi0/12, Gi0/13, Gi0/14
                                                Gi0/15, Gi0/16, Gi0/17, Gi0/18
                                                Gi0/19, Gi0/20, Gi0/21, Gi0/22
                                                Gi0/23, Gi0/24, Gi0/25, Gi0/26
                                                Gi0/27, Gi0/28
10   Sales                            active    Gi0/1, Gi0/2
20   Accounting                       active
26   Automation10                     active
27   VLAN0027                         active
30   Engineering                      active
170  VLAN0170                         active
501  Private501                       active
502  Private500                       active
[output cut]
\end{verbatim}

This output shows that VLAN 10 is in the local database and that Gi0/1
and Gi0/2 are associated to VLAN 10.

So next, we'll go to step 2 and verify the CAM with the
\texttt{show\ mac\ address-table} command:

\begin{verbatim}
S1#sh mac address-table
          Mac Address Table
-------------------------------------------
 
Vlan    Mac Address       Type        Ports
----    -----------       --------    -----
 All    0100.0ccc.cccc    STATIC      CPU
[output cut]
   1    000d.2830.2f00    DYNAMIC     Gi0/24
   1    0021.1c91.0d8d    DYNAMIC     Gi0/13
   1    0021.1c91.0d8e    DYNAMIC     Gi0/14
   1    b414.89d9.1882    DYNAMIC     Gi0/17
   1    b414.89d9.1883    DYNAMIC     Gi0/18
   1    ecc8.8202.8282    DYNAMIC     Gi0/15
   1    ecc8.8202.8283    DYNAMIC     Gi0/16
  10    001a.2f55.c9e8    DYNAMIC     Gi0/1
  10    001b.d40a.0538    DYNAMIC     Gi0/2
Total Mac Addresses for this criterion: 29
\end{verbatim}

Okay---know that your switch will show quite a few MAC addresses
assigned to the CPU at the top of the output; those MAC addresses are
used by the switch to manage the ports. The very first MAC address
listed is the base MAC address of the switch and used by STP in the
bridge ID. In the preceding output, we can see that there are two MAC
addresses associated with VLAN 10 and that it was dynamically learned.
We can also establish that this MAC address is associated to Gi0/1. S1
looks really good!

Let's take a look at S2 now. First, let's confirm that port PC3 is
connected and check its configuration. I'll use the
\texttt{show\ interfaces\ interface\ switchport} command to do that:

\begin{verbatim}
S2#sh interfaces gi0/3 switchport
Name: Gi0/3
Switchport: Enabled
Administrative Mode: dynamic desirable
Operational Mode: static access
Administrative Trunking Encapsulation: negotiate
Operational Trunking Encapsulation: native
Negotiation of Trunking: On
Access Mode VLAN: 10 (Inactive)
Trunking Native Mode VLAN: 1 (default)
[output cut]
\end{verbatim}

Okay---we can see that the port is enabled and that it's set to dynamic
desirable. This means that if it connects to another Cisco switch, it
will desire to trunk on that link. But keep in mind that we're using it
as an access port, which is confirmed by the operational mode of static
access. At the end of the output, the text shows
\texttt{Access\ Mode\ VLAN:\ 10\ (Inactive)}. This is not a good thing!
Let's examine S2's CAM and see what we find out:

\begin{verbatim}
S2#sh mac address-table
          Mac Address Table
-------------------------------------------
 
Vlan    Mac Address       Type        Ports
----    -----------       --------    -----
 All    0100.0ccc.cccc    STATIC      CPU
 [output cut]
   1    001b.d40a.0538    DYNAMIC     Gi0/13
   1    0021.1bee.a70d    DYNAMIC     Gi0/13
   1    b414.89d9.1884    DYNAMIC     Gi0/17
   1    b414.89d9.1885    DYNAMIC     Gi0/18
   1    ecc8.8202.8285    DYNAMIC     Gi0/16
Total Mac Addresses for this criterion: 26
\end{verbatim}

Referring back to \protect\hyperlink{c20.xhtmlux5cux23figure20-7}{Figure
20.7}, we can see that the host is connected to Gi0/3. The problem here
is that we don't see a MAC address dynamically associated to Gi0/3 in
the MAC address table. So what do we know so far that can help us? Well
first, we can see that Gi0/3 is configured into VLAN 10, but that VLAN
is inactive. Second, the host off of Gi0/3 doesn't appear in the CAM
table. Now would be a good time to take a look at the VLAN database like
this:

\begin{verbatim}
S2#sh vlan brief
 
VLAN Name                             Status    Ports
---- -------------------------------- --------- -------------------------------
1    default                          active    Gi0/1, Gi0/2, Gi0/4, Gi0/5
                                                Gi0/6, Gi0/7, Gi0/8, Gi0/9
                                                Gi0/10, Gi0/11, Gi0/12, Gi0/13
                                                Gi0/14, Gi0/15, Gi0/16, Gi0/17
                                                Gi0/18, Gi0/19, Gi0/20, Gi0/21
                                                Gi0/22, Gi0/23, Gi0/24, Gi0/25
                                                Gi0/26, Gi0/27, Gi0/28
26   Automation10                     active
27   VLAN0027                         active
30   Engineering                      active
170  VLAN0170                         active
[output cut]
\end{verbatim}

Look at that: there is no VLAN 10 in the database! Clearly the problem,
but also an easy one to fix by simply creating the VLAN in the database:

\begin{verbatim}
S2#config t
S2(config)#vlan 10
S2(config-vlan)#name Sales
\end{verbatim}

That's all there is to it. Now let's check the CAM again:

\begin{verbatim}
S2#sh mac address-table
          Mac Address Table
-------------------------------------------
 
Vlan    Mac Address       Type        Ports
----    -----------       --------    -----
 All    0100.0ccc.cccc    STATIC      CPU
[output cut]
   1    0021.1bee.a70d    DYNAMIC     Gi0/13
  10    001a.6c46.9b09    DYNAMIC     Gi0/3
Total Mac Addresses for this criterion: 22
\end{verbatim}

We're good to go---the MAC address off of Gi0/3 shows in the MAC address
table configured into VLAN 10.

That was pretty straightforward, but if the port had been assigned to
the wrong VLAN, I would have used the \texttt{switch\ access\ vlan}
command to correct the VLAN membership. Here's an example of how to do
that:

\begin{verbatim}
S2#config t
S2(config)#int gi0/3
S2(config-if)#switchport access vlan 10
S2(config-if)#do sh vlan
 
VLAN Name                             Status    Ports
---- -------------------------------- --------- -------------------------------
1    default                          active    Gi0/1, Gi0/2, Gi0/4, Gi0/5
                                                Gi0/6, Gi0/7, Gi0/8, Gi0/9
                                                Gi0/10, Gi0/11, Gi0/12, Gi0/13
                                                Gi0/14, Gi0/15, Gi0/16, Gi0/17
                                                Gi0/18, Gi0/19, Gi0/20, Gi0/21
                                                Gi0/22, Gi0/23, Gi0/24, Gi0/25
                                                Gi0/26, Gi0/27, Gi0/28
10   Sales                            active    Gi0/3
\end{verbatim}

Okay, great---we can see that our port Gi0/3 is in the VLAN 10
membership. Now let's try to ping from PC1 to PC3:

\begin{verbatim}
PC1#ping 192.168.10.3
Type escape sequence to abort.
Sending 5, 100-byte ICMP Echos to 192.168.10.3, timeout is 2 seconds:
.....
Success rate is 0 percent (0/5)
\end{verbatim}

No luck, so let's see if PC1 can ping PC2:

\begin{verbatim}
PC1#ping 192.168.10.2
Type escape sequence to abort.
Sending 5, 100-byte ICMP Echos to 192.168.10.2, timeout is 2 seconds:
!!!!!
Success rate is 100 percent (5/5), round-trip min/avg/max = 1/2/4 ms
PC1#
\end{verbatim}

\protect\hypertarget{c20.xhtmlux5cux23Page_867}{}{}That worked! I can
ping a host that's a member of the same VLAN connected to the same
switch, but I can't ping to a host on another switch that's a member of
the same VLAN, which is VLAN 10. To get to the bottom of this, let's
quickly summarize what we've learned so far:

\begin{enumerate}
\tightlist
\item
  We know that the VLAN database is now correct on each switch.
\item
  The MAC address table shows the ARP entries for each host as well as a
  connection to each switch.
\item
  We've verified that our VLAN memberships are now correct on all the
  ports we're using.
\end{enumerate}

But since we still can't ping to a host on another switch, we need to
start checking out the connections between our switches.

\subsubsection[Trunk
Troubleshooting]{\texorpdfstring{\protect\hypertarget{c20.xhtmlux5cux23c20-sec-11}{}{}Trunk
Troubleshooting}{Trunk Troubleshooting}}

You'll need to troubleshoot trunk links when you lose connectivity
between hosts that are in the same VLAN but are located on different
switches. Cisco refers to this as ``VLAN leaking.'' Seems to me we are
leaking VLAN 10 between switches somehow.

These are the steps we'll take to troubleshoot VLANs:

\begin{enumerate}
\tightlist
\item
  Verify that the interface configuration is set to the correct trunk
  parameters.
\item
  Verify that the ports are configured correctly.
\item
  Verify the native VLAN on each switch.
\end{enumerate}

And here are the commands we'll use to perform trunk troubleshooting:

\begin{verbatim}
Show interfaces trunk
Show vlan
Show interfaces interface trunk
Show interfaces interface switchport
Show dtp interface interface
switchport mode
switchport mode dynamic
switchport trunk native vlan vlan
\end{verbatim}

Okay, let's get started by checking ports Gi0/13 and Gi0/14 on each
switch because these are what the figure is showing as forming the
connection between our switches. We'll start with the
\texttt{show\ interfaces\ trunk} command:

\begin{verbatim}
S1>sh interfaces trunk
 
S2>sh interfaces trunk
 
\end{verbatim}

Not a scrap of output---that's definitely a bad sign! Let's take another
look at the \texttt{show\ vlan} output on S1 and see what we can find
out:

\begin{verbatim}
S1>sh vlan brief
 
VLAN Name                             Status    Ports
---- -------------------------------- --------- -------------------------------
1    default                          active    Gi0/3, Gi0/4, Gi0/5, Gi0/6
                                                Gi0/7, Gi0/8, Gi0/9, Gi0/10
                                                Gi0/11, Gi0/12, Gi0/13, Gi0/14
                                                Gi0/15, Gi0/16, Gi0/17, Gi0/18
                                                Gi0/19, Gi0/20, Gi0/21, Gi0/22
                                                Gi0/23, Gi0/24, Gi0/25, Gi0/26
                                                Gi0/27, Gi0/28
10   Sales                            active    Gi0/1, Gi0/2
20   Accounting                       active
[output cut]
\end{verbatim}

Nothing new from when we checked it a few minutes ago, but look there
under VLAN 1---we can see interfaces Gi0/13 and Gi0/14. This means that
our ports between switches are members of VLAN 1 and will pass only VLAN
1 frames!

Typically I'll tell my students that if you type the \texttt{show\ vlan}
command, you're really typing the nonexistent ``show access ports''
command since this output shows interfaces in access mode but doesn't
show the trunk interfaces. This means that our ports between switches
are access ports instead of trunk ports, so they'll pass information
about only VLAN 1.

Let's go back over to the S2 switch to verify and see which port
interfaces Gi0/13 and Gi0/14 are members of:

\begin{verbatim}
S2>sh vlan brief
 
VLAN Name                             Status    Ports
---- -------------------------------- --------- -----------------------------
1    default                          active    Gi0/1, Gi0/2, Gi0/4, Gi0/5
                                                Gi0/6, Gi0/7, Gi0/8, Gi0/9
                                                Gi0/10, Gi0/11, Gi0/12, Gi0/13
                                                Gi0/14, Gi0/15, Gi0/16, Gi0/17
                                                Gi0/18, Gi0/19, Gi0/20, Gi0/21
                                                Gi0/22, Gi0/23, Gi0/24, Gi0/25
                                                Gi0/26, Gi0/27, Gi0/28
10   Sales                            active    Gi0/3
\end{verbatim}

Again, as with S1, the links between switches are showing in the output
of the \texttt{show\ vlan} command, which means that they are not trunk
ports. We can use the \texttt{show} \texttt{interfaces} \emph{interface}
\texttt{\ switchport} command to verify this as well:

\begin{verbatim}
S1#sho interfaces gi0/13 switchport
Name: Gi0/13
Switchport: Enabled
Administrative Mode: dynamic auto
Operational Mode: static access
Administrative Trunking Encapsulation: negotiate
Operational Trunking Encapsulation: native
Negotiation of Trunking: On
Access Mode VLAN: 1 (default)
Trunking Native Mode VLAN: 1 (default)
\end{verbatim}

This output tells us that interface Gi0/13 is in dynamic auto mode. But
its operational mode is static access, meaning it's not a trunk port. We
can look closer at its trunking capabilities with the
\texttt{show\ interfaces} \emph{interface} \texttt{\ trunk} command:

\begin{verbatim}
S1#sh interfaces gi0/1 trunk
 
Port        Mode         Encapsulation  Status        Native vlan
Gi0/1       auto         negotiate      not-trunking  1
[output cut]
\end{verbatim}

Sure enough---the port is not trunking, but we already knew that. Now we
know it again. Notice that we can see that the native VLAN is set to
VLAN 1, which is the default native VLAN. This means that VLAN 1 is the
default VLAN for untagged traffic.

Now, before we check the native VLAN on S2 to verify that there isn't a
mismatch, I want to point out a key fact about trunking and how we would
get these ports between switches to do that.

Many Cisco switches support the Cisco proprietary \emph{Dynamic Trunking
Protocol (DTP)}, which is used to manage automatic trunk negotiation
between switches. Cisco recommends that you don't allow this and to
configure your switch ports manually instead. I agree with them!

Okay, with that in mind, let's check out our switch port Gi0/13 on S1
and view its DTP status. I'll use the
\texttt{show\ dtp\ interface\ interface} command to view the DTP
statistics:

\begin{verbatim}
S1#sh dtp interface gi0/13
DTP information for GigabitEthernet0/13:
  TOS/TAS/TNS:                              ACCESS/AUTO/ACCESS
  TOT/TAT/TNT:                              NATIVE/NEGOTIATE/NATIVE
  Neighbor address 1:                       00211C910D8D
  Neighbor address 2:                       000000000000
  Hello timer expiration (sec/state):       12/RUNNING
  Access timer expiration (sec/state):      never/STOPPED
\end{verbatim}

Did you notice that our port GI0/13 from S1 to S2 is an access port
configured to autonegotiate using DTP? That's interesting, and I want to
delve a bit deeper into the different port configurations and how they
affect trunking capabilities to clarify why.

\protect\hypertarget{c20.xhtmlux5cux23Page_870}{}{}\textbf{Access}
Trunking is not allowed on a port set to access mode.

\textbf{Auto} Will trunk to neighbor switch only if the remote port is
set to on or to desirable mode. This creates the trunk based on the DTP
request from the neighboring switch.

\textbf{Desirable} This will trunk with all port modes except access.
Ports set to dynamic desirable will communicate via DTP that the
interface is attempting to become a trunk if the neighboring switch
interface is able to become a trunk.

\textbf{Nonegotiate} No DTP frames are generated from the interface. Can
only be used if the neighbor interface is manually set as trunk or
access.

\textbf{Trunk (on)} Trunks with all switch port modes except access.
Automatically enables trunking regardless of the state of the
neighboring switch and regardless of any DTP requests.

Let's check out the different options available on the S1 switch with
the \texttt{switchport\ mode\ dynamic} command:

\begin{verbatim}
S1(config-if)#switchport mode ?
  access        Set trunking mode to ACCESS unconditionally
  dot1q-tunnel  set trunking mode to TUNNEL unconditionally
  dynamic       Set trunking mode to dynamically negotiate access or trunk mode
  private-vlan  Set private-vlan mode
  trunk         Set trunking mode to TRUNK unconditionally
 
S1(config-if)#switchport mode dynamic ?
  auto       Set trunking mode dynamic negotiation parameter to AUTO
  desirable  Set trunking mode dynamic negotiation parameter to DESIRABLE
\end{verbatim}

From interface mode, use the \texttt{switch\ mode\ trunk} command to
turn trunking on. You can also use the \texttt{switch\ mode\ dynamic}
command to set the port to auto or desirable trunking modes. To turn off
DTP and any type of negotiation, use the \texttt{switchport}
\texttt{nonegotiate} command.

Let's take a look at S2 and see if we can figure out why our two
switches didn't create a trunk:

\begin{verbatim}
S2#sh int gi0/13 switchport
Name: Gi0/13
Switchport: Enabled
Administrative Mode: dynamic auto
Operational Mode: static access
Administrative Trunking Encapsulation: negotiate
Operational Trunking Encapsulation: native
Negotiation of Trunking: On
\end{verbatim}

Okay---we can see that the port is in dynamic auto and that it's
operating as an access port. Let's look into this further:

\begin{verbatim}
S2#sh dtp interface gi0/13
DTP information for GigabitEthernet0/3:
  DTP information for GigabitEthernet0/13:
  TOS/TAS/TNS:                              ACCESS/AUTO/ACCESS
  TOT/TAT/TNT:                              NATIVE/NEGOTIATE/NATIVE
  Neighbor address 1:                       000000000000
  Neighbor address 2:                       000000000000
  Hello timer expiration (sec/state):       17/RUNNING
  Access timer expiration (sec/state):      never/STOPPED
\end{verbatim}

Do you see the problem? Don't be fooled---it's not that they're running
in access mode; it's because two ports in dynamic auto will not form a
trunk! This is a really common problem to look for since most Cisco
switches ship in dynamic auto. The other issue you need to be aware of,
as well as check for, is the frame-tagging method. Some switches run
802.1q, some run both 802.1q and \emph{Inter-Switch Link (ISL) routing},
so be sure the tagging method is compatible between all of your
switches!

It's time to fix our problem on the trunk ports between S1 and S2. All
we need to do is to just fix one side of each link since dynamic auto
will trunk with a port set to desirable or on:

\begin{verbatim}
S2(config)#int gi0/13
S2(config-if)#switchport mode dynamic desirable
23:11:37:%LINEPROTO-5-UPDOWN:Line protocol on Interface GigabitEthernet0/13, changed state to down
23:11:37:%LINEPROTO-5-UPDOWN:Line protocol on Interface Vlan1, changed state to down
23:11:40:%LINEPROTO-5-UPDOWN:Line protocol on Interface GigabitEthernet0/13, changed state to up
23:12:10:%LINEPROTO-5-UPDOWN:Line protocol on Interface Vlan1, changed state to up
S2(config-if)#do show int trunk
 
Port        Mode         Encapsulation  Status        Native vlan
Gi0/13      desirable    n-isl          trunking      1
[output cut]
\end{verbatim}

Nice---it worked! With one side in \texttt{Auto} and the other now in
\texttt{Desirable}, DTPs will be exchanged and they will trunk. Notice
in the preceding output that the mode of S2's Gi0/13 link is desirable
and that the switches actually negotiated ISL as a trunk
encapsulation---go figure! But don't forget to notice the native VLAN.
We'll work on the frame-tagging method and native VLAN in a minute, but
first, let's configure our other link:

\begin{verbatim}
S2(config-if)#int gi0/14
S2(config-if)#switchport mode dynamic desirable
23:12:%LINEPROTO-5-UPDOWN:Line protocol on Interface GigabitEthernet0/14, changed state to down
23:12:%LINEPROTO-5-UPDOWN:Line protocol on Interface GigabitEthernet0/14, changed state to up
S2(config-if)#do show int trunk
 
Port        Mode         Encapsulation  Status        Native vlan
Gi0/13      desirable    n-isl          trunking      1
Gi0/14      desirable    n-isl          trunking      1
 
Port        Vlans allowed on trunk
Gi0/13      1-4094
Gi0/14      1-4094
[output cut]
\end{verbatim}

Great, we now have two trunked links between switches. But I've got to
say, I really don't like the ISL method of frame tagging since it can't
send untagged frames across the link. So let's change our native VLAN
from the default of 1 to 392. The number 392 just randomly sounded good
at the moment. Here's what I entered on S1:

\begin{verbatim}
S1(config-if)#switchport trunk native vlan 392
S1(config-if)#
23:17:40: Port is not 802.1Q trunk, no action
\end{verbatim}

See what I mean? I tried to change the native VLAN and ISL basically
responded with, ``What's a native VLAN?'' Very annoying, so I'm going to
take care of that now!

\begin{verbatim}
S1(config-if)#int range gi0/13 - 14
S1(config-if-range)#switchport trunk encapsulation ?
  dot1q      Interface uses only 802.1q trunking encapsulation when trunking
  isl        Interface uses only ISL trunking encapsulation when trunking
  negotiate  Device will negotiate trunking encapsulation with peer on
             interface
 
S1(config-if-range)#switchport trunk encapsulation dot1q
23:23:%LINEPROTO-5-UPDOWN:Line protocol on Interface GigabitEthernet0/13, changed state to down
23:23:%LINEPROTO-5-UPDOWN: Line protocol on Interface GigabitEthernet0/14, changed state to down
23:23:%CDP-4-NATIVE_VLAN_MISMATCH: Native VLAN mismatch discovered on GigabitEthernet0/13 (392), with S2 GigabitEthernet0/13 (1).
23:23:%LINEPROTO-5-UPDOWN: Line protocol on Interface GigabitEthernet0/14, changed state to up
23:23:%LINEPROTO-5-UPDOWN: Line protocol on Interface GigabitEthernet0/13, changed state to up
23:23:%CDP-4-NATIVE_VLAN_MISMATCH: Native VLAN mismatch discovered on GigabitEthernet0/13 (392), with S2 GigabitEthernet0/13 (1).
\end{verbatim}

\protect\hypertarget{c20.xhtmlux5cux23Page_873}{}{}Okay, that's more
like it! As soon as I changed the encapsulation type on S1, DTP frames
changed the frame-tagging method between S2 to 802.1q. Since I had
already changed the native VLAN on port Gi0/13 on S1, the switch lets us
know, via CDP, that we now have a native VLAN mismatch. Let's proceed to
deal with this by verifying our interfaces with the
\texttt{show\ interface\ trunk} command:

\begin{verbatim}
S1#sh int trunk
Port        Mode         Encapsulation  Status        Native vlan
Gi0/13      auto         802.1q         trunking      392
Gi0/14      auto         802.1q         trunking      1
 
S2#sh int trunk
 
Port        Mode         Encapsulation  Status        Native vlan
Gi0/13      desirable    n-802.1q       trunking      1
Gi0/14      desirable    n-802.1q       trunking      1
\end{verbatim}

Now notice that both links are running 802.1q and that S1 is in auto
mode and S2 is in desirable mode. And we can see a native VLAN mismatch
on port Gi0/13. We can also see the mismatched native VLAN with the
\texttt{show\ interfaces\ interface\ switchport} command by looking at
both sides of the link like this:

\begin{verbatim}
S2#sh interfaces gi0/13 switchport
Name: Gi0/13
Switchport: Enabled
Administrative Mode: dynamic desirable
Operational Mode: trunk
Administrative Trunking Encapsulation: negotiate
Operational Trunking Encapsulation: dot1q
Negotiation of Trunking: On
Access Mode VLAN: 1 (default)
Trunking Native Mode VLAN: 1 (default)
 
S1#sh int gi0/13 switchport
Name: Gi0/13
Switchport: Enabled
Administrative Mode: dynamic auto
Operational Mode: trunk
Administrative Trunking Encapsulation: dot1q
Operational Trunking Encapsulation: dot1q
Negotiation of Trunking: On
Access Mode VLAN: 1 (default)
Trunking Native Mode VLAN: 392 (Inactive)
\end{verbatim}

\protect\hypertarget{c20.xhtmlux5cux23Page_874}{}{}So this has got to be
bad, right? I mean really---are we sending any frames down that link or
not? Let's see if we solved our little problem of not being able to ping
to hosts from S1 to S2 and find out:

\begin{verbatim}
PC1#ping 192.168.10.3
Type escape sequence to abort.
Sending 5, 100-byte ICMP Echos to 192.168.10.3, timeout is 2 seconds:
!!!!!
Success rate is 100 percent (5/5), round-trip min/avg/max = 1/1/4 ms
\end{verbatim}

Yes, it works! Not so bad after all. We've solved our problem, or at
least most of it. Having a native VLAN mismatch only means you can't
send untagged frames down the link, which are essentially management
frames like CDP, for example. So although it's not the end of the world,
it will prevent us from being able to remotely manage the switch, or
even sending any other types of traffic down just that one VLAN.

So am I saying you can just leave this issue the way it is? Well, you
could, but you won't. No, you'll fix it because if you don't, CDP will
send you a message every minute telling you that there's a mismatch,
which will drive you mad! So, this is how we'll stop that from
happening:

\begin{verbatim}
S2(config)#int gi0/13
S2(config-if)#switchport trunk native vlan 392
S2(config-if)#^Z
S2#sh int trunk
 
Port        Mode         Encapsulation  Status     Native vlan
Gi0/13      desirable    n-802.1q       trunking      392
Gi0/14      desirable    n-802.1q       trunking      1
[output cut]
\end{verbatim}

All better! Both sides of the same link between switches are now using
native VLAN 392 on Gigabit Ethernet 0/13. I want you to know that it's
fine to have different native VLANs for each link if that's what works
best for you. Each network is different and you have to make choices
between options that will end up meeting your particular business
requirements the most optimal way.

\subsection[Summary]{\texorpdfstring{\protect\hypertarget{c20.xhtmlux5cux23c20-sec-12}{}{}Summary}{Summary}}

This chapter covered troubleshooting techniques from basic to advanced.
Although most chapters in this book cover troubleshooting, this chapter
focused purely on IPv4, IPv6, and VLAN/trunk troubleshooting.

You learned how to troubleshoot step-by-step from a host to a remote
device. Starting with IPv4, you learned the steps to test the host and
the local connectivity and then how to troubleshoot remote connectivity.

\protect\hypertarget{c20.xhtmlux5cux23Page_875}{}{}We then moved on to
IPv6 and proceeded to troubleshoot using the same techniques that you
learned with IPv4. It's important that you can use the verification
commands that I used in each step of this chapter.

Last, I covered VLAN and trunk troubleshooting and how to go
step-by-step through a switched network using verification commands and
narrowing down the problem.

\subsection[Exam
Essentials]{\texorpdfstring{\protect\hypertarget{c20.xhtmlux5cux23c20-sec-13}{}{}Exam
Essentials}{Exam Essentials}}

\textbf{Remember the Cisco steps in troubleshooting an IPv4 and IPv6
network.}

\begin{enumerate}
\tightlist
\item
  Check the cables to find out if there's a faulty cable or interface in
  the mix and verify the interface's statistics.
\item
  Make sure that devices are determining the correct path from the
  source to the destination. Manipulate the routing information if
  needed.
\item
  Verify that the default gateway is correct.
\item
  Verify that name resolution settings are correct.
\item
  Verify that there are no ACLs blocking traffic.
\end{enumerate}

\textbf{Remember the commands to verify and troubleshoot IPv4 and IPv6.}
You need to remember and practice the commands used in this chapter,
especially \texttt{ping} and\texttt{\ traceroute} (\texttt{tracert} on
Windows). But we also used the Windows commands \texttt{ipconfig} and
\texttt{route\ print} and Cisco's commands
\texttt{show\ ip\ int\ brief}, \texttt{show\ interface}, and
\texttt{show\ route}.

\textbf{Remember how to verify an ARP cache with IPv6.} The command
\texttt{show\ ipv6\ neighbors} shows the IP-to-MAC-address resolution
table on a Cisco router.

\textbf{Remember to look at the statistics on a router and switch
interface to determine problems.} You've got to be able to analyze
interface statistics to find problems if they exist, and this includes
speed and duplex settings, input queue drops, output queue drops, and
input and output errors.

\textbf{Understand what a native VLAN is and how to change it.} A native
VLAN works with only 802.1q trunks and allows untagged traffic to
traverse the trunk link. This is VLAN 1 by default on all Cisco
switches, but it can be changed for security reasons with the
\texttt{switchport\ native\ vlan\ vlan} command.

\subsection[Written Lab
20]{\texorpdfstring{\protect\hypertarget{c20.xhtmlux5cux23c20-sec-14}{}{}Written
Lab 20}{Written Lab 20}}

You can find the answers to this lab in Appendix A, ``Answers to Written
Labs.''

Write the answers to the following questions:

\begin{enumerate}
\tightlist
\item
  If your IPv6 ARP cache shows an entry of INCMP, what does this mean?
\item
  \protect\hypertarget{c20.xhtmlux5cux23Page_876}{}{}You want traffic
  from VLAN 66 to traverse a trunked link untagged. Which command will
  you use?
\item
  What are the five modes you can set a switch port to?
\item
  You are having a network problem and have checked the cables to find
  out if there's a faulty cable or interface in the mix and also
  verified the interface's statistics, made sure that devices are
  determining the correct path from the source to the destination, and
  verified that you don't need to manipulate the routing. What are your
  next troubleshooting steps?
\item
  You need to find out if the local IPv6 stack is working on a host.
  What command will you use?~
\end{enumerate}

\begin{center}\rule{0.5\linewidth}{0.5pt}\end{center}

\subsection{Hands-on Labs for Troubleshooting}

Please check \href{http://www.lammle.com/ccna}{www.lammle.com/ccna} for
the latest information and downloads available for studying when using
my books. Preconfigured hands-on troubleshooting labs are available for
download, with the answers to the troubleshooting problems found on my
forum.

\begin{center}\rule{0.5\linewidth}{0.5pt}\end{center}

\subsection[Review
Questions]{\texorpdfstring{\protect\hypertarget{c20.xhtmlux5cux23c20-sec-15}{}{}\protect\hypertarget{c20.xhtmlux5cux23Page_877}{}{}Review
Questions}{Review Questions}}

\begin{center}\rule{0.5\linewidth}{0.5pt}\end{center}

\includegraphics{images/note.png}The following questions are designed to
test your understanding of this chapter's material. For more information
on how to get additional questions, please
\href{http://www.lammle.com/ccna}{www.lammle.com/ccna}.

\begin{center}\rule{0.5\linewidth}{0.5pt}\end{center}

You can find the answers to these questions in Appendix B, ``Answers to
Review Questions.''

\begin{enumerate}
\item
  You need to verify the IPv6 ARP cache on a router and see that the
  state of an entry is REACH. What does REACH mean?

  \begin{enumerate}
  \def\labelenumii{\Alph{enumii}.}
  \tightlist
  \item
    The router is reaching out to get the address.
  \item
    The entry is incomplete.
  \item
    The entry has reached the end of life and will be discarded from the
    table.
  \item
    A positive confirmation has been received by the neighbor and the
    path to it is functioning correctly.
  \end{enumerate}
\item
  What is the most common cause of interface errors?

  \begin{enumerate}
  \def\labelenumii{\Alph{enumii}.}
  \tightlist
  \item
    Speed mismatch
  \item
    Duplex mismatch
  \item
    Buffer overflows
  \item
    Collisions between a dedicated switch port and an NIC
  \end{enumerate}
\item
  Which command will verify the DTP status on a switch interface?

  \begin{enumerate}
  \def\labelenumii{\Alph{enumii}.}
  \tightlist
  \item
    \texttt{sh\ dtp\ status}
  \item
    \texttt{sh\ dtp\ status\ interface\ interface}
  \item
    \texttt{sh\ interface\ interface\ dtp}
  \item
    \texttt{sh\ dtp\ interface\ interface}
  \end{enumerate}
\item
  What mode will not allow DTP frames generated from a switch port?

  \begin{enumerate}
  \def\labelenumii{\Alph{enumii}.}
  \tightlist
  \item
    Nonegotiate
  \item
    Trunk
  \item
    Access
  \item
    Auto
  \end{enumerate}
\item
  The following output was generated by which command?

\begin{verbatim}
IPv6 Address                          Age Link-layer Addr State Interface
FE80::21A:6DFF:FE64:9B3                  0 001a.6c46.9b09  DELAY Fa0/1
2001:DB8:3C4D:2:21A:6DFF:FE64:9B3        0 001a.6c46.9b09  REACH Fa0/1
\end{verbatim}

  \begin{enumerate}
  \def\labelenumii{\Alph{enumii}.}
  \tightlist
  \item
    \protect\hypertarget{c20.xhtmlux5cux23Page_878}{}{}\texttt{show\ ip\ arp}
  \item
    \texttt{show\ ipv6\ arp}
  \item
    \texttt{show\ ip\ neighbors}
  \item
    \texttt{show\ ipv6\ neighbors}
  \end{enumerate}
\item
  Which of the following states tells you that an interface has not
  communicated within the neighbor-reachable time frame?

  \begin{enumerate}
  \def\labelenumii{\Alph{enumii}.}
  \tightlist
  \item
    REACH
  \item
    STALE
  \item
    TIMEOUT
  \item
    CLEARED
  \end{enumerate}
\item
  You receive a call from a user who says that they cannot log in to a
  remote server, which only runs IPv6. Based on the output, what could
  the problem be?

  \begin{figure}
  \centering
  \includegraphics{images/c20f008.jpg}
  \caption{}
  \end{figure}

  \begin{enumerate}
  \def\labelenumii{\Alph{enumii}.}
  \tightlist
  \item
    The global address is in the wrong subnet.
  \item
    The IPv6 default gateway has not been configured or received from
    the router.
  \item
    The link-local address has not been resolved, so the host cannot
    communicate to the router.
  \item
    There are two IPv6 global addresses configured. One must be removed
    from the configuration.
  \end{enumerate}
\item
  Your host cannot reach remote networks. Based on the output, what is
  the problem?

  \begin{figure}
  \centering
  \includegraphics{images/c20f009.jpg}
  \caption{}
  \end{figure}

  \begin{enumerate}
  \def\labelenumii{\Alph{enumii}.}
  \tightlist
  \item
    The link-local IPv6 address is wrong.
  \item
    The IPv6 global address is missing.
  \item
    \protect\hypertarget{c20.xhtmlux5cux23Page_879}{}{}There is no DNS
    server configuration.
  \item
    The IPv4 default gateway address is misconfigured.
  \end{enumerate}
\item
  Which two commands will show you if you have a native VLAN mismatch?

  \begin{enumerate}
  \def\labelenumii{\Alph{enumii}.}
  \tightlist
  \item
    \texttt{show\ interface\ native\ vlan}
  \item
    \texttt{show\ interface\ trunk}
  \item
    \texttt{show\ interface\ interface\ switchport}
  \item
    \texttt{show\ switchport\ interface}
  \end{enumerate}
\item
  You connect two new Cisco 3560 switches together and expect them to
  use DTP and create a trunk. However, when you check statistics, you
  find that they are access ports and didn't negotiate. Why didn't DTP
  work on these Cisco switches?

  \begin{enumerate}
  \def\labelenumii{\Alph{enumii}.}
  \tightlist
  \item
    The ports on each side of the link are set to auto trunking.
  \item
    The ports on each side of the link are set to on.
  \item
    The ports on each side of the link are set to dynamic.
  \item
    The ports on each side of the link are set to desirable.
  \end{enumerate}
\end{enumerate}

\protect\hypertarget{c21.xhtml}{}{}

\section[{Chapter 21}\\
{Wide Area
Networks}]{\texorpdfstring{\protect\hypertarget{c21.xhtmlux5cux23c21}{}{}\protect\hypertarget{c21.xhtmlux5cux23Page_881}{}{}{Chapter
21}\\
{Wide Area Networks}}{Chapter 21 Wide Area Networks}}

\subsection{THE FOLLOWING ICND2 EXAM TOPICS ARE COVERED IN THIS
CHAPTER:}

\begin{enumerate}
\tightlist
\item
  \includegraphics{images/tick.png} \textbf{3.0 WAN Technologies}
\item
  \includegraphics{images/tick.png} \textbf{3.1 Configure and verify PPP
  and MLPPP on WAN interfaces using local authentication}
\item
  \includegraphics{images/tick.png} \textbf{3.2 Configure, verify, and
  troubleshoot PPPoE client-side interfaces using local authentication}
\item
  \includegraphics{images/tick.png} \textbf{3.3 Configure, verify, and
  troubleshoot GRE tunnel connectivity}
\item
  \includegraphics{images/tick.png} \textbf{3.4 Describe WAN topology
  options}

  \begin{enumerate}
  \tightlist
  \item
    \includegraphics{images/square1.png} 3.4.a Point-to-point
  \item
    \includegraphics{images/square1.png} 3.4.b Hub and spoke
  \item
    \includegraphics{images/square1.png} 3.4.c Full mesh
  \item
    \includegraphics{images/square1.png} 3.4.d Single vs dual-homed
  \end{enumerate}
\item
  \includegraphics{images/tick.png} \textbf{3.5 Describe WAN access
  connectivity options}

  \begin{enumerate}
  \tightlist
  \item
    \includegraphics{images/square1.png} 3.5.a MPLS
  \item
    \includegraphics{images/square1.png} 3.5.b MetroEthernet
  \item
    \includegraphics{images/square1.png} 3.5.c Broadband PPPoE
  \item
    \includegraphics{images/square1.png} 3.5.d Internet VPN (DMVPN,
    site-to-site VPN, client VPN)
  \end{enumerate}
\item
  \includegraphics{images/tick.png} \textbf{3.6 Configure and verify
  single-homed branch connectivity using eBGP IPv4 (limited to peering
  and route advertisement using Network command only)}
\end{enumerate}

\protect\hypertarget{c21.xhtmlux5cux23Page_882}{}{}\includegraphics{images/intro.png}
The Cisco IOS supports a ton of different wide area network (WAN)
protocols that help you extend your local LANs to other LANs at remote
sites. And I don't think I have to tell you how essential information
exchange between disparate sites is these days---it's absolutely vital!
But even so, it wouldn't exactly be cost effective or efficient to
install your own cable and connect all of your company's remote
locations yourself, would it? A much better way to get this done is to
just lease the existing installations that service providers already
have in place.

This is exactly why I'm going to devote most of this chapter to covering
the various types of connections, technologies, and devices used in
today's WANs.

We'll also delve into how to implement and configure High-Level
Data-Link Control (HDLC), and Point-to-Point Protocol (PPP). I'll
describe Point-to-Point Protocol over Ethernet (PPPoE), cable, digital
subscriber line (DSL), MultiProtocol Label Switching (MPLS), and Metro
Ethernet plus last mile and long-range WAN technologies. I'll also
introduce you to WAN security concepts, tunneling, virtual private
networks (VPNs) and how to create a tunnel using GRE (Generic Routing
Encapsulation). Finally, I'll close the chapter with a discussion on
Border Gateway Protocol (BGP) and how to configure External BGP.

\begin{center}\rule{0.5\linewidth}{0.5pt}\end{center}

\includegraphics{images/note.png}To find up-to-the-minute updates for
this chapter, please see
\href{http://www.lammle.com/ccna}{www.lammle.com/ccna} or the book's web
page at \href{http://www.sybex.com/go/ccna}{www.sybex.com/go/ccna}.

\begin{center}\rule{0.5\linewidth}{0.5pt}\end{center}

\subsection[Introduction to Wide Area
Networks]{\texorpdfstring{\protect\hypertarget{c21.xhtmlux5cux23c21-sec-1}{}{}Introduction
to Wide Area Networks}{Introduction to Wide Area Networks}}

Let's begin exploring WAN basics by asking, what's the difference
between a \emph{wide area network (WAN)} and a local area network (LAN)?
Clearly there's the distance factor, but modern wireless LANs can cover
some serious turf, so there's more to it than that. What about
bandwidth? Here again, some really big pipes can be had for a price in
many places, so that's not it either. What's the answer we're looking
for?

A major distinction between a WAN and a LAN is that while you generally
own a LAN infrastructure, you usually lease a WAN infrastructure from a
service provider. And to be honest, modern technologies even blur this
characteristic somewhat, but it still fits neatly into the context of
Cisco's exam objectives!

I've already talked about the data link that you usually own back when
we covered Ethernet, so now I'm going to focus on the type you usually
don't own---the kind you typically lease from a service provider.

There are several reasons why WANs are necessary in corporate
environments today.

\protect\hypertarget{c21.xhtmlux5cux23Page_883}{}{}LAN technologies
provide amazing speeds (10/40/100 Gbps is now common) and at a great
bang for your buck! But these type of solutions can only work well in
relatively small geographic areas. You still need WANs in a
communications environment because some business needs require
connections to remote sites for many reasons, including the following:

\begin{enumerate}
\tightlist
\item
  People in the regional or branch offices of an organization need to be
  able to communicate and share data.
\item
  Organizations often want to share information with other organizations
  across large distances.
\item
  Employees who travel on company business frequently need to access
  information that resides on their corporate networks.
\end{enumerate}

Here are three major characteristics of WANs:

\begin{enumerate}
\tightlist
\item
  WANs generally connect devices that are separated by a broader
  geographic area than a LAN can serve.
\item
  WANs use the services of carriers such as telcos, cable companies,
  satellite systems, and network providers.
\item
  WANs use serial connections of various types to provide access to
  bandwidth over large geographic areas.
\end{enumerate}

The first key to understanding WAN technologies is to be familiar with
the different WAN topologies, terms, and connection types commonly used
by service providers to join your LAN networks together. We'll begin
covering these topics now.

\subsubsection[WAN Topology
Options]{\texorpdfstring{\protect\hypertarget{c21.xhtmlux5cux23c21-sec-2}{}{}WAN
Topology Options}{WAN Topology Options}}

A physical topology describes the physical layout of the network, in
contrast to logical topologies, which describe the path a signal takes
through the physical topology. There are three basic topologies for a
WAN design.

\textbf{Star or hub-and-spoke topology} This topology features a single
hub (central router) that provides access from remote networks to a core
router. \protect\hyperlink{c21.xhtmlux5cux23figure21-1}{Figure 21.1}
illustrates a hub-and-spoke topology:

\begin{figure}
\centering
\includegraphics{images/c21f001.jpg}
\caption{{\protect\hyperlink{c21.xhtmlux5cux23figureanchor21-1}{\textbf{FIGURE
21.1}} Hub-and-spoke}}
\end{figure}

\protect\hypertarget{c21.xhtmlux5cux23Page_884}{}{}All communication
among the networks travels through the core router. The advantages of a
star physical topology are less cost and easier administration, but the
disadvantages can be significant:

\begin{enumerate}
\tightlist
\item
  The central router (hub) represents a single point of failure.
\item
  The central router limits the overall performance for access to
  centralized resources. It is a single pipe that manages all traffic
  intended either for the centralized resources or for the other
  regional routers.
\end{enumerate}

\textbf{Fully meshed topology} In this topology, each routing node on
the edge of a given packet-switching network has a direct path to every
other node on the cloud.
\protect\hyperlink{c21.xhtmlux5cux23figure21-2}{Figure 21.2} shows a
fully meshed topology.

\begin{figure}
\centering
\includegraphics{images/c21f002.jpg}
\caption{{\protect\hyperlink{c21.xhtmlux5cux23figureanchor21-2}{\textbf{FIGURE
21.2}} Fully meshed topology}}
\end{figure}

This configuration clearly provides a high level of redundancy, but the
costs are the highest. So a fully meshed topology really isn't viable in
large packet-switched networks. Here are some issues you'll contend with
using a fully meshed topology:

\begin{enumerate}
\tightlist
\item
  Many virtual circuits are required---one for every connection between
  routers, which brings up the cost.
\item
  Configuration is more complex for routers without multicast support in
  non-broadcast environments.
\end{enumerate}

\textbf{Partially meshed topology} This type of topology reduces the
number of routers within a network that have direct connections to all
other routers in the topology.
\protect\hyperlink{c21.xhtmlux5cux23figure21-3}{Figure 21.3} depicts a
partially meshed topology.

\protect\hypertarget{c21.xhtmlux5cux23Page_885}{}{}

\begin{figure}
\centering
\includegraphics{images/c21f003.jpg}
\caption{{\protect\hyperlink{c21.xhtmlux5cux23figureanchor21-3}{\textbf{FIGURE
21.3}} Partially meshed topology}}
\end{figure}

Unlike in the full mesh network, all routers are not connected to all
other routers, but it still provides more redundancy than a typical
hub-and-spoke design will. This is actually considered the most balanced
design because it provides more virtual circuits, plus redundancy and
performance.

\subsubsection[Defining WAN
Terms]{\texorpdfstring{\protect\hypertarget{c21.xhtmlux5cux23c21-sec-3}{}{}Defining
WAN Terms}{Defining WAN Terms}}

Before you run out and order a WAN service type from a provider, you
really need to understand the following terms that service providers
typically use. Take a look at these in
\protect\hyperlink{c21.xhtmlux5cux23figure21-4}{Figure 21.4}:

\begin{figure}
\centering
\includegraphics{images/c21f004.jpg}
\caption{{\protect\hyperlink{c21.xhtmlux5cux23figureanchor21-4}{\textbf{FIGURE
21.4}} WAN terms}}
\end{figure}

\textbf{Customer premises equipment (CPE)} \emph{Customer premises
equipment (CPE)} is equipment that's typically owned by the subscriber
and located on the subscriber's premises.

\protect\hypertarget{c21.xhtmlux5cux23Page_886}{}{}\textbf{CSU/DSU} A
channel service unit/data service unit (CSU/DSU) is a device that is
used to connect data terminal equipment (DTE) to a digital circuit, such
as a T1/T3 line. A device is considered DTE if it is either a source or
destination for digital data---for example, PCs, servers, and routers.
In \protect\hyperlink{c21.xhtmlux5cux23figure21-4}{Figure 21.4}, the
router is considered DTE because it is passing data to the CSU/DSU,
which will forward the data to the service provider. Although the
CSU/DSU connects to the service provider infrastructure using a
telephone or coaxial cable, such as a T1 or E1 line, it connects to the
router with a serial cable. The most important aspect to remember for
the CCNA objectives is that the CSU/DSU provides clocking of the line to
the router. You really need to understand this completely, which is why
I'll cover it in depth later in the cabling the serial WAN interface
configuration section!

\textbf{Demarcation point} The \emph{demarcation point} (demarc for
short) is the precise spot where the service provider's responsibility
ends and the CPE begins. It's generally a device in a telecommunications
closet owned and installed by the telecommunications company (telco).
It's your responsibility to cable (extended demarc) from this box to the
CPE, which is usually a connection to a CSU/DSU, although more recently
we see the provider giving us an Ethernet connection. Nice!

\textbf{Local loop} The \emph{local loop} connects the demarc to the
closest switching office, referred to as the central office.

\textbf{Central office (CO)} This point connects the customer's network
to the provider's switching network. Make a mental note that a
\emph{central office (CO)} is sometimes also referred to as a
\emph{point of presence (POP)}.

\textbf{Toll network} The \emph{toll network} is a trunk line inside a
WAN provider's network. This network is a collection of switches and
facilities owned by the Internet service provider (ISP).

\textbf{Optical fiber converters} Even though I'm not employing this
device in \protect\hyperlink{c21.xhtmlux5cux23figure21-4}{Figure 21.4},
optical fiber converters are used where a fiber-optic link terminates to
convert optical signals into electrical signals and vice versa. You can
also implement the converter as a router or switch module.

Definitely familiarize yourself with these terms, what they represent,
and where they're located, as shown in
\protect\hyperlink{c21.xhtmlux5cux23figure21-4}{Figure 21.4}, because
they're key to understanding WAN technologies.

\subsubsection[WAN Connection
Bandwidth]{\texorpdfstring{\protect\hypertarget{c21.xhtmlux5cux23c21-sec-4}{}{}WAN
Connection Bandwidth}{WAN Connection Bandwidth}}

Next, I want you to know these basic but very important bandwidth terms
used when referring to WAN connections:

\textbf{Digital Signal 0 (DS0)} This is the basic digital signaling rate
of 64 Kbps, equivalent to one channel. Europe uses the E0 and Japan uses
the J0 to reference the same channel speed. Typical to T-carrier
transmission, this is the generic term used by several multiplexed
digital carrier systems and is also the smallest-capacity digital
circuit. One DS0 = One voice/data line.

\textbf{T1} Also referred to as a DS1, a T1 comprises 24 DS0 circuits
bundled together for a total bandwidth of 1.544 Mbps.

\protect\hypertarget{c21.xhtmlux5cux23Page_887}{}{}\textbf{E1} This is
the European equivalent of a T1 and comprises 30 DS0 circuits bundled
together for a bandwidth of 2.048 Mbps.

\textbf{T3} Referred to as a DS3, a T3 comprises 28 DS1s bundled
together, or 672 DS0s, for a bandwidth of 44.736 Mbps.

\textbf{OC-3} Optical Carrier (OC) 3 uses fiber and is made up of three
DS3s bundled together. It's made up of 2,016 DS0s and avails a total
bandwidth of 155.52 Mbps.

\textbf{OC-12} Optical Carrier 12 is made up of four OC-3s bundled
together and contains 8,064 DS0s for a total bandwidth of 622.08 Mbps.

\textbf{OC-48} Optical Carrier 48 is made up of four OC-12s bundled
together and contains 32,256 DS0s for a total bandwidth of 2488.32 Mbps.

\textbf{OC-192} Optical Carrier 192 is four OC-48s and contains 129,024
DS0s for a total bandwidth of 9953.28 Mbps.

\subsubsection[WAN Connection
Types]{\texorpdfstring{\protect\hypertarget{c21.xhtmlux5cux23c21-sec-5}{}{}WAN
Connection Types}{WAN Connection Types}}

You're probably aware that a WAN can use a number of different
connection types available on the market today.
\protect\hyperlink{c21.xhtmlux5cux23figure21-5}{Figure 21.5} shows the
different WAN connection types that can be used to connect your LANs
(made up of data terminal equipment, or DTE) together over the data
communication equipment (DCE) network.

\begin{figure}
\centering
\includegraphics{images/c21f005.jpg}
\caption{{\protect\hyperlink{c21.xhtmlux5cux23figureanchor21-5}{\textbf{FIGURE
21.5}} WAN connection types}}
\end{figure}

Let me explain the different WAN connection types in detail now:

\textbf{Dedicated (leased lines)} These are usually referred to as a
\emph{point-to-point} or dedicated connections. A \emph{leased line} is
a pre-established WAN communications path that goes from
\protect\hypertarget{c21.xhtmlux5cux23Page_888}{}{}the CPE through the
DCE switch, then over to the CPE of the remote site. The CPE enables DTE
networks to communicate at any time with no cumbersome setup procedures
to muddle through before transmitting data. When you've got plenty of
cash, this is definitely the way to go because it uses synchronous
serial lines up to 45 Mbps. HDLC and PPP encapsulations are frequently
used on leased lines, and I'll go over these with you soon.

\textbf{Circuit switching} When you hear the term \emph{circuit
switching}, think phone call. The big advantage is cost; most plain old
telephone service (POTS) and ISDN dial-up connections are not flat rate,
which is their advantage over dedicated lines because you pay only for
what you use, and you pay only when the call is established. No data can
transfer before an end-to-end connection is established. Circuit
switching uses dial-up modems or ISDN and is used for low-bandwidth data
transfers. Okay, I know what you're thinking, ``Modems? Did he say
modems? Aren't those found only in museums now?'' After all, with all
the wireless technologies available, who would use a modem these days?
Well, some people do have ISDN; it's still viable and there are a few
who still use a modem now and then. And circuit switching can be used in
some of the newer WAN technologies as well.

\textbf{Packet switching} This is a WAN switching method that allows you
to share bandwidth with other companies to save money, just like a super
old party line, where homes shared the same phone number and line to
save money. \emph{Packet switching} can be thought of as a network
that's designed to look like a leased line yet it charges you less, like
circuit switching does. As usual, you get what you pay for, and there's
definitely a serious downside to this technology. If you need to
transfer data constantly, well, just forget about this option and get a
leased line instead! Packet switching will only really work for you if
your data transfers are bursty, not continuous; think of a highway,
where you can only go as fast as the traffic---packet switching is the
same thing. Frame Relay and X.25 are packet-switching technologies with
speeds that can range from 56 Kbps up to T3 (45 Mbps).

\begin{center}\rule{0.5\linewidth}{0.5pt}\end{center}

\includegraphics{images/note.png}MultiProtocol Label Switching (MPLS)
uses a combination of both circuit switching and packet switching.

\begin{center}\rule{0.5\linewidth}{0.5pt}\end{center}

\subsubsection[WAN
Support]{\texorpdfstring{\protect\hypertarget{c21.xhtmlux5cux23c21-sec-6}{}{}WAN
Support}{WAN Support}}

Cisco supports many layer 2 WAN encapsulations on its serial interfaces,
including HDLC, PPP, and Frame Relay, which map to the Cisco exam
objectives. You can view them via the \texttt{encapsulation\ ?} command
from any serial interface, but understand that the output you'll get can
vary based upon the specific IOS version you're running:

\begin{verbatim}
Corp#config t
Corp(config)#int s0/0/0
Corp(config-if)#encapsulation ?
  atm-dxi      ATM-DXI encapsulation
 frame-relay  Frame Relay networks
 hdlc         Serial HDLC synchronous
  lapb         LAPB (X.25 Level 2)
 ppp          Point-to-Point protocol
  smds         Switched Megabit Data Service (SMDS)
  x25          X.25
\end{verbatim}

I also want to point out that if I had other types of interfaces on my
router, I would have a different set of encapsulation options. And never
forget that you can't configure an Ethernet encapsulation on a serial
interface or vice versa!

Next, I'm going to define the most prominently known WAN protocols used
in the latest Cisco exam objectives: Frame Relay, ISDN, HDLC, PPP,
PPPoE, cable, DSL, MPLS, ATM, Cellular 3G/4G, VSAT, and Metro Ethernet.
Just so you know, the only WAN protocols you'll usually find configured
on a serial interface are HDLC, PPP, and Frame Relay, but who said
you're stuck with using only serial interfaces for wide area
connections? Actually, we're beginning to see fewer and fewer serial
connections because they're not as scalable or cost effective as a Fast
Ethernet connection to your ISP.

\textbf{Frame Relay} A packet-switched technology that made its debut in
the early 1990s, \emph{Frame Relay} is a high-performance Data Link and
Physical layer specification. It's pretty much a successor to X.25,
except that much of the technology in X.25 that was used to compensate
for physical errors like noisy lines has been eliminated. An upside to
Frame Relay is that it can be more cost effective than point-to-point
links, plus it typically runs at speeds of 64 Kbps up to 45 Mbps (T3).
Another Frame Relay benefit is that it provides features for dynamic
bandwidth allocation and congestion control.

\textbf{ISDN} \emph{Integrated Services Digital Network (ISDN)} is a set
of digital services that transmit voice and data over existing phone
lines. ISDN offers a cost-effective solution for remote users who need a
higher-speed connection than analog POTS dial-up links can give them,
and it's also a good choice to use as a backup link for other types of
links, such as Frame Relay or T1 connections.

\textbf{HDLC} \emph{High-Level Data-Link Control (HDLC)} was derived
from Synchronous Data Link Control (SDLC), which was created by IBM as a
Data Link connection protocol. HDLC works at the Data Link layer and
creates very little overhead compared to Link Access Procedure, Balanced
(LAPB).

Generic HDLC wasn't intended to encapsulate multiple Network layer
protocols across the same link---the HDLC header doesn't contain any
identification about the type of protocol being carried inside the HDLC
encapsulation. Because of this, each vendor that uses HDLC has its own
way of identifying the Network layer protocol, meaning each vendor's
HDLC is proprietary with regard to its specific equipment.

\textbf{PPP} \emph{Point-to-Point Protocol (PPP)} is a pretty famous,
industry-standard protocol. Because all multiprotocol versions of HDLC
are proprietary, PPP can be used to create point-to-point links between
different vendors' equipment. It uses a Network Control Protocol field
in the Data Link header to identify the Network layer protocol being
carried and allows authentication and multilink connections to be run
over asynchronous and synchronous links.

\protect\hypertarget{c21.xhtmlux5cux23Page_890}{}{}\textbf{PPPoE}
\emph{Point-to-Point Protocol over Ethernet} encapsulates PPP frames in
Ethernet frames and is usually used in conjunction with xDSL services.
It gives you a lot of the familiar PPP features like authentication,
encryption, and compression, but there's a downside---it has a lower
maximum transmission unit (MTU) than standard Ethernet does. If your
firewall isn't solidly configured, this little factor can really give
you some grief!

Still somewhat popular in the United States, PPPoE's main feature is
that it adds a direct connection to Ethernet interfaces while also
providing DSL support. It's often used by many hosts on a shared
Ethernet interface for opening PPP sessions to various destinations via
at least one bridging modem.

\textbf{Cable} In a modern \emph{hybrid fiber-coaxial (HFC)} network,
typically 500 to 2,000 active data subscribers are connected to a
certain cable network segment, all sharing the upstream and downstream
bandwidth. HFC is a telecommunications industry term for a network that
incorporates both optical fiber and coaxial cables to create a broadband
network. The actual bandwidth for Internet service over a cable TV
(CATV) line can be up to about 27 Mbps on the download path to the
subscriber, with about 2.5 Mbps of bandwidth on the upload path.
Typically users get an access speed from 256 Kbps to 6 Mbps. This data
rate varies greatly throughout the United States and can be much, much
higher today.

\textbf{DSL} Digital subscriber line is a technology used by traditional
telephone companies to deliver advanced services such as high-speed data
and sometimes video over twisted-pair copper telephone wires. It
typically has lower data-carrying capacity than HFC networks, and data
speeds can be limited in range by line lengths and quality. Digital
subscriber line is not a complete end-to-end solution but rather a
Physical layer transmission technology like dial-up, cable, or wireless.
DSL connections are deployed in the last mile of a local telephone
network---the local loop. The connection is set up between a pair of DSL
modems on either end of a copper wire located between the customer
premises equipment (CPE) and the Digital Subscriber Line Access
Multiplexer (DSLAM). A DSLAM is the device that is located at the
provider's central office (CO) and concentrates connections from
multiple DSL subscribers.

\textbf{MPLS} \emph{MultiProtocol Label Switching (MPLS)} is a
data-carrying mechanism that emulates some properties of a
circuit-switched network over a packet-switched network. MPLS is a
switching mechanism that imposes labels (numbers) to packets and then
uses them to forward the packets. The labels are assigned on the edge of
the MPLS network, and forwarding inside the MPLS network is carried out
solely based on the labels. The labels usually correspond to a path to
layer 3 destination addresses, which is on par with IP destination-based
routing. MPLS was designed to support the forwarding of protocols other
than TCP/IP. Because of this, label switching within the network is
achieved the same way irrespective of the layer 3 protocol. In larger
networks, the result of MPLS labeling is that only the edge routers
perform a routing lookup. All the core routers forward packets based on
the labels, which makes forwarding the packets through the service
provider network
\protect\hypertarget{c21.xhtmlux5cux23Page_891}{}{}faster. This is a big
reason most companies have replaced their Frame Relay networks with MPLS
service today. Last, you can use Ethernet with MPLS to connect a WAN,
and this is called Ethernet over MPLS, or EoMPLS.

\textbf{ATM} Asynchronous Transfer Mode (ATM) was created for
time-sensitive traffic, providing simultaneous transmission of voice,
video, and data. ATM uses cells that are a fixed 53 bytes long instead
of packets. It also can use isochronous clocking (external clocking) to
help the data move faster. Typically, if you're running Frame Relay
today, you will be running Frame Relay over ATM.

\textbf{Cellular 3G/4G} Having a wireless hot spot in your pocket is
pretty normal these days. If you have a pretty current cellular phone,
then you can probably can gain access through your phone to the
Internet. You can even get a 3G/4G card for an ISR router that's useful
for a small remote office that's in the coverage area.

\textbf{VSAT} Very Small Aperture Terminal (VSAT) can be used if you
have many locations geographically spread out in a large area. VSAT uses
a two-way satellite ground station with dishes available through many
companies like Dish Network or Hughes and connects to satellites in
geosynchronous orbit. A good example of where VSATs are a useful,
cost-effective solution would be companies that use satellite
communications to VSATs, like gasoline stations that have hundreds or
thousands of locations spread out over the entire country. How could you
connect them otherwise? Using leased lines would be cost prohibitive and
dial-ups would be way too slow and hard to manage. Instead, the signal
from the satellite connects to many remote locations at once, which is
much more cost effective and efficient! It's a lot faster than a modem
(about 10x faster), but the upload speeds only come in at about 10
percent of their download speeds.

\textbf{Metro Ethernet} Metropolitan-area Ethernet is a metropolitan
area network (MAN) that's based on Ethernet standards and can connect a
customer to a larger network and the Internet. If available, businesses
can use Metro Ethernet to connect their own offices together, which is
another very cost-effective connection option. MPLS-based Metro Ethernet
networks use MPLS in the ISP by providing an Ethernet or fiber cable to
the customer as a connection. From the customer, it leaves the Ethernet
cable, jumps onto MPLS, and then Ethernet again on the remote side. This
is a smart and thrifty solution that's very popular if you can get it in
your area.

\subsubsection[Cisco Intelligent WAN
(IWAN)]{\texorpdfstring{\protect\hypertarget{c21.xhtmlux5cux23c21-sec-7}{}{}Cisco
Intelligent WAN (IWAN)}{Cisco Intelligent WAN (IWAN)}}

Bottom line, WANs are expensive, and if they're not deployed correctly,
they can be a very costly mistake! Here's a short list of how most
companies are currently deployingWANs:

\begin{enumerate}
\tightlist
\item
  Using MPLS links for the branch and remote locations to headquarters.
\item
  Leveraging low-cost, high-bandwidth Internet links as backup for the
  MPLS links.
\end{enumerate}

\protect\hypertarget{c21.xhtmlux5cux23Page_892}{}{}Still, even these
solutions are no longer good enough. The pressure on today's WANs has
increased dramatically to make them handle more and more, as shown in
\protect\hyperlink{c21.xhtmlux5cux23figure21-6}{Figure 21.6}:

\begin{enumerate}
\tightlist
\item
  Steadily increasing cloud traffic like Google Docs, Office365, etc.
\item
  Unprecedented proliferation of mobile devices.
\item
  A legion of high-bandwidth applications like Video---lots of video!
\end{enumerate}

\begin{figure}
\centering
\includegraphics{images/c21f006.jpg}
\caption{{\protect\hyperlink{c21.xhtmlux5cux23figureanchor21-6}{\textbf{FIGURE
21.6}} Branch WAN challenges}}
\end{figure}

Here are two new winning strategies that many businesses in today's fast
paced market utilize:

\begin{enumerate}
\tightlist
\item
  Using a low-cost Internet link set in an active/active mode rather
  than just sitting there idly most of the time in active/standby mode.
\item
  Leveraging the Internet link for remote employees accessing public
  clouds or the Internet as well as for guest users' access.
\end{enumerate}

These new strategies offer cost reductions for your company plus
increased WAN capacity. They result in improved performance and
scalability from the end user point of view and pave the way for
implementing cloud, mobility, and BYOD effectively.

So what does the Cisco Intelligent WAN (IWAN) have to do with all this?
The Cisco IWAN enables application service-level agreements (SLAs),
endpoint type, and network conditions so that Cisco IWAN traffic is
dynamically routed to deliver the best-quality experience. The savings
over traditional WANs not only allows companies to pay for the
infrastructure upgrades, they can also free up resources for new
business innovation.

IT organizations can now provide more bandwidth to their branch office
connections by using less-expensive WAN transport options, all without
affecting performance, security, or reliability, as pictured in
\protect\hyperlink{c21.xhtmlux5cux23figure21-7}{Figure 21.7}:

\protect\hypertarget{c21.xhtmlux5cux23Page_893}{}{}

\begin{figure}
\centering
\includegraphics{images/c21f007.jpg}
\caption{{\protect\hyperlink{c21.xhtmlux5cux23figureanchor21-7}{\textbf{FIGURE
21.7}} Intelligent WAN}}
\end{figure}

Cisco's IWAN solution is based upon four technology pillars demonstrated
in \protect\hyperlink{c21.xhtmlux5cux23figure21-8}{Figure 21.8}:

\begin{figure}
\centering
\includegraphics{images/c21f008.jpg}
\caption{{\protect\hyperlink{c21.xhtmlux5cux23figureanchor21-8}{\textbf{FIGURE
21.8}} IWAN four technology pillars}}
\end{figure}

\textbf{Transport Independent Connectivity} IWAN should provide
consistent connectivity over the entire access network while also
providing simplicity, scalability, and modularity. It also allows for a
simple, efficient migration strategy. Keep in mind that you don't
typically use a design based on DMVPN to achieve this.

\textbf{Intelligent Path Control} This solution is intended to help
utilize full WAN links without oversubscribing lines. Through
Intelligent Path Control (Path Selection), routing decisions are made
dynamically by looking at application type, policies, and path status.
This allows new cloud traffic and guest services as well as video
services to easily load-balance across multiple links!

\protect\hypertarget{c21.xhtmlux5cux23Page_894}{}{}\textbf{Application
Optimization} This solution provides an application-aware network for
optimized performance, providing full visibility and control at the
Application layer (layer 7). It uses technologies like AVC and NBAR2,
NetFlow, QoS, and more to reach optimization goals.

\textbf{High Secure Connectivity} US government FIPS 140-2 certified
IPsec solutions provide security, privacy, and dynamic site-to-site IP
Security (IPsec) tunnels with DMVPN. Threat defense with zone-based
firewalls permit access to the Internet using the Cloud Web Security
(CWS) Connector.

IWAN also gives us more useful technologies, like the following:

\textbf{Intelligent Virtualization} Cisco's IWAN offers a virtual WAN
overlay over any transport type, which doesn't compromise application
performance, availability, or security.

\textbf{Automation} This Cisco IWAN technology provides network services
by provisioning security and application policies.

\textbf{Cloud Integration} An innovation that permits private cloud
integration through APIC and public cloud application optimization with
security.

\textbf{Service Virtualization} Cisco's IWAN delivers virtual services
on specialized router platforms as well as virtual routers and services
on x86 server platforms.

\textbf{Self-Learning Networks} Using policies, this technology
leverages network analytics to proactively optimize the infrastructure.

We're going to pause a second and head back in time to cover some good
old serial connections, which still actually happen to be a valid form
of connecting via a WAN.

\subsection[Cabling the Serial Wide Area
Network]{\texorpdfstring{\protect\hypertarget{c21.xhtmlux5cux23c21-sec-8}{}{}Cabling
the Serial Wide Area Network}{Cabling the Serial Wide Area Network}}

As you can imagine, there are a few things that you need to know before
connecting your WAN to ensure that everything goes well. For starters,
you have to understand the kind of WAN Physical layer implementation
that Cisco provides and be familiar with the various types of WAN serial
connectors involved.

The good news is that Cisco serial connections support almost any type
of WAN service. Your typical WAN connection is a dedicated leased line
using HDLC or PPP, with speeds that can kick it up to 45 Mbps (T3).

HDLC, PPP, and Frame Relay can use the same Physical layer
specifications. I'll go over the various types of connections and then
move on to tell you all about the WAN protocols specified in the ICND2
and CCNA R/S objectives.

\subsubsection[Serial
Transmission]{\texorpdfstring{\protect\hypertarget{c21.xhtmlux5cux23c21-sec-9}{}{}Serial
Transmission}{Serial Transmission}}

WAN serial connectors use \emph{serial transmission}, something that
takes place 1 bit at a time over a single channel.

Older Cisco routers have used a proprietary 60-pin serial connector that
you have to get from Cisco or a provider of Cisco equipment. Cisco also
has a new, smaller proprietary serial connection that's about one-tenth
the size of the 60-pin basic serial cable called the
\protect\hypertarget{c21.xhtmlux5cux23Page_895}{}{}\emph{smart-serial}.
You have to verify that you have the right type of interface in your
router before using this cable connector.

The type of connector you have on the other end of the cable depends on
your service provider and its particular end-device requirements. There
are several different types of ends you'll run into:

\begin{enumerate}
\tightlist
\item
  EIA/TIA-232---Allowed speed up to 64 Kbps on 24-pin connector
\item
  EIA/TIA-449
\item
  V.35---Standard used to connect to a CSU/DSU, with speeds up to 2.048
  Mbps using a 34-pin rectangular connector
\item
  EIA-530
\end{enumerate}

Make sure you're clear on these things: Serial links are described in
frequency, or cycles per second (hertz). The amount of data that can be
carried within these frequencies is called \emph{bandwidth}. Bandwidth
is the amount of data in bits per second that the serial channel can
carry.

\subsubsection[Data Terminal Equipment and Data Communication
Equipment]{\texorpdfstring{\protect\hypertarget{c21.xhtmlux5cux23c21-sec-10}{}{}Data
Terminal Equipment and Data Communication
Equipment}{Data Terminal Equipment and Data Communication Equipment}}

By default, router interfaces are typically \emph{data terminal
equipment (DTE)}, and they connect into \emph{data communication
equipment (DCE)} like a \emph{channel service unit/data service unit
(CSU/DSU)} using a V.35 connector. CSU/DSU then plugs into a demarcation
location (demarc) and is the service provider's last responsibility.
Most of the time, the demarc is a jack that has an RJ45 (8-pin modular)
female connector located in a telecommunications closet.

Actually, you may already have heard of demarcs. If you've ever had the
glorious experience of reporting a problem to your service provider,
they'll usually tell you everything tests out fine up to the demarc, so
the problem must be the CPE, or customer premises equipment. In other
words, it's your problem, not theirs!

\protect\hyperlink{c21.xhtmlux5cux23figure21-9}{Figure 21.9} shows a
typical DTE-DCE-DTE connection and the devices used in the network.

\begin{figure}
\centering
\includegraphics{images/c21f009.jpg}
\caption{{\protect\hyperlink{c21.xhtmlux5cux23figureanchor21-9}{\textbf{FIGURE
21.9}} DTE-DCE-DTE WAN connection: Clocking is typically provided by the
DCE network to routers. In nonproduction environments, a DCE network is
not always present.}}
\end{figure}

\protect\hypertarget{c21.xhtmlux5cux23Page_896}{}{}The idea behind a WAN
is to be able to connect two DTE networks through a DCE network. The DCE
network includes the CSU/DSU, through the provider's wiring and
switches, all the way to the CSU/DSU at the other end. The network's DCE
device (CSU/DSU) provides clocking to the DTE-connected interface (the
router's serial interface).

As mentioned, the DCE network provides clocking to the router; this is
the CSU/DSU. If you have a nonproduction network and you're using a WAN
crossover type of cable and do not have a CSU/DSU, then you need to
provide clocking on the DCE end of the cable by using the
\texttt{clock\ rate} command. To find out which interface needs the
\texttt{clock\ rate} command, use the \texttt{show\ controllers\ int}
command:

\begin{verbatim}
Corp#sh controllers s0/0/0
Interface Serial0/0/0
Hardware is PowerQUICC MPC860
DCE V.35, clock rate 2000000
\end{verbatim}

The preceding output shows a DCE interface that has the clock rate set
to \texttt{2000000}, which is the default for ISR routers. This next
output shows a DTE connector, so you don't need enter the
\texttt{clock\ rate} command on this interface:

\begin{verbatim}
SF#sh controllers s0/0/0
Interface Serial0/0/0
Hardware is PowerQUICC MPC860
DTE V.35 TX and RX clocks detected
\end{verbatim}

\begin{center}\rule{0.5\linewidth}{0.5pt}\end{center}

\includegraphics{images/note.png}Terms such as \emph{EIA/TIA-232, V.35,
X.21}, and \emph{HSSI (High-Speed Serial Interface)} describe the
Physical layer between the DTE (router) and DCE device (CSU/DSU).

\begin{center}\rule{0.5\linewidth}{0.5pt}\end{center}

\subsection[High-Level Data-Link Control (HDLC)
Protocol]{\texorpdfstring{\protect\hypertarget{c21.xhtmlux5cux23c21-sec-11}{}{}High-Level
Data-Link Control (HDLC)
Protocol}{High-Level Data-Link Control (HDLC) Protocol}}

The High-Level Data-Link Control (HDLC) protocol is a popular
ISO-standard, bit-oriented, Data Link layer protocol. It specifies an
encapsulation method for data on synchronous serial data links using
frame characters and checksums. HDLC is a point-to-point protocol used
on leased lines. No authentication is provided by HDLC.

In byte-oriented protocols, control information is encoded using entire
bytes. On the other hand, bit-oriented protocols use single bits to
represent the control information. Some common bit-oriented protocols
are SDLC and HDLC; TCP and IP are byte-oriented protocols.

HDLC is the default encapsulation used by Cisco routers over synchronous
serial links. And Cisco's HDLC is proprietary, meaning it won't
communicate with any other vendor's
\protect\hypertarget{c21.xhtmlux5cux23Page_897}{}{}HDLC implementation.
But don't give Cisco grief for it---\emph{everyone's} HDLC
implementation is proprietary.
\protect\hyperlink{c21.xhtmlux5cux23figure21-10}{Figure 21.10} shows the
Cisco HDLC format.

\begin{figure}
\centering
\includegraphics{images/c21f010.jpg}
\caption{{\protect\hyperlink{c21.xhtmlux5cux23figureanchor21-10}{\textbf{FIGURE
21.10}} Cisco's HDLC frame format: Each vendor's HDLC has a proprietary
data field to support multiprotocol environments.}}
\end{figure}

The reason every vendor has a proprietary HDLC encapsulation method is
that each vendor has a different way for the HDLC protocol to
encapsulate multiple Network layer protocols. If the vendors didn't have
a way for HDLC to communicate the different layer 3 protocols, then HDLC
would be able to operate in only a single layer 3 protocol environment.
This proprietary header is placed in the data field of the HDLC
encapsulation.

It's pretty simple to configure a serial interface if you're just going
to connect two Cisco routers across a T1, for example.
\protect\hyperlink{c21.xhtmlux5cux23figure21-11}{Figure 21.11} shows a
point-to-point connection between two cities.

\begin{figure}
\centering
\includegraphics{images/c21f011.jpg}
\caption{{\protect\hyperlink{c21.xhtmlux5cux23figureanchor21-11}{\textbf{FIGURE
21.11}} Configuring Cisco's HDLC proprietary WAN encapsulation}}
\end{figure}

We can easily configure the routers with a basic IP address and then
enable the interface. Assuming the link to the ISP is up, the routers
will start communicating using the default HDLC encapsulation. Let's
take a look at the Corp router configuration so you can see just how
easy this can be:

\begin{verbatim}
Corp(config)#int s0/0
Corp(config-if)#ip address 172.16.10.1 255.255.255.252
Corp(config-if)#no shut
\end{verbatim}

\begin{verbatim}
Corp#sh int s0/0
Serial0/0 is up, line protocol is up
  Hardware is PowerQUICC Serial
  Internet address is 172.16.10.1/30
  MTU 1500 bytes, BW 1544 Kbit, DLY 20000 usec,
     reliability 255/255, txload 1/255, rxload 1/255
 Encapsulation HDLC, loopback not set
  Keepalive set (10 sec)
\end{verbatim}

\begin{verbatim}
Corp#sh run | begin interface Serial0/0
interface Serial0/0
 ip address 172.16.10.1 255.255.255.252
\end{verbatim}

Note that all I did was add an IP address before I then enabled the
interface---pretty simple! Now, as long as the SF router is running the
default serial encapsulation, this link will come up. Notice in the
preceding output that the \texttt{show\ interface} command does show the
encapsulation type of HDLC, but the output of
\texttt{show\ running-config} does not. This is important---remember
that if you don't see an encapsulation type listed under a serial
interface in the active configuration file, you know it's running the
default encapsulation of HDLC.

So let's say you have only one Cisco router and you need to connect to a
non-Cisco router because your other Cisco router is on order or
something. What would you do? You couldn't use the default HDLC serial
encapsulation because it wouldn't work. Instead, you would need to go
with an option like PPP, an ISO-standard way of identifying the
upper-layer protocols. Now is a great time to get into more detail about
PPP as well as how to connect to routers using the PPP encapsulation.
You can check out RFC 1661 for more information on the origins and
standards of PPP.

\subsection[Point-to-Point Protocol
(PPP)]{\texorpdfstring{\protect\hypertarget{c21.xhtmlux5cux23c21-sec-12}{}{}Point-to-Point
Protocol (PPP)}{Point-to-Point Protocol (PPP)}}

Point-to-Point Protocol (PPP) is a Data Link layer protocol that can be
used over either asynchronous serial (dial-up) or synchronous serial
media. It relies on Link Control Protocol (LCP) to build and maintain
data-link connections. Network Control Protocol (NCP) enables multiple
Network layer protocols (routed protocols) to be used on a
point-to-point connection.

Because HDLC is the default serial encapsulation on Cisco serial links
and it works great, why in the world would you choose to use PPP? Well,
the basic purpose of PPP is to transport layer 3 packets across a Data
Link layer point-to-point link, and it's nonproprietary. So unless you
have all Cisco routers, you need PPP on your serial interfaces because
the HDLC encapsulation is Cisco proprietary, remember? Plus, since PPP
can encapsulate several layer 3 routed protocols and provide
authentication, dynamic addressing, and callback, PPP could actually be
the best encapsulation solution for you over HDLC anyway.

\protect\hyperlink{c21.xhtmlux5cux23figure21-12}{Figure 21.12} shows the
PPP protocol stack compared to the OSI reference model.

\protect\hypertarget{c21.xhtmlux5cux23Page_899}{}{}

\begin{figure}
\centering
\includegraphics{images/c21f012.jpg}
\caption{{\protect\hyperlink{c21.xhtmlux5cux23figureanchor21-12}{\textbf{FIGURE
21.12}} Point-to-Point Protocol stack}}
\end{figure}

PPP contains four main components:

\textbf{EIA/TIA-232-C, V.24, V.35, and ISDN} A Physical layer
international standard for serial communication.

\textbf{HDLC} A method for encapsulating datagrams over serial links.

\textbf{LCP} A method of establishing, configuring, maintaining, and
terminating the point-to-point connection. It also provides features
such as authentication. I'll give you a complete list of these features
in the next section.

\textbf{NCP} NCP is a method of establishing and configuring different
Network layer protocols for transport across the PPP link. NCP is
designed to allow the simultaneous use of multiple Network layer
protocols. Two examples of protocols here are Internet Protocol Control
Protocol (IPCP) and Cisco Discovery Protocol Control Protocol (CDPCP).

Burn it into your mind that the PPP protocol stack is specified at the
Physical and Data Link layers only. NCP is used to allow communication
of multiple Network layer protocols by identifying and encapsulating the
protocols across a PPP data link.

\begin{center}\rule{0.5\linewidth}{0.5pt}\end{center}

\includegraphics{images/tip.png}Remember that if you have a Cisco router
and a non-Cisco router connected with a serial connection, you must
configure PPP or another encapsulation method like Frame Relay because
the HDLC default just won't work!

\begin{center}\rule{0.5\linewidth}{0.5pt}\end{center}

Next, we'll cover the options for LCP and PPP session establishment.

\subsubsection[Link Control Protocol (LCP) Configuration
Options]{\texorpdfstring{\protect\hypertarget{c21.xhtmlux5cux23c21-sec-13}{}{}Link
Control Protocol (LCP) Configuration
Options}{Link Control Protocol (LCP) Configuration Options}}

\emph{Link Control Protocol (LCP)} offers different PPP encapsulation
options, including the following:

\textbf{Authentication} This option tells the calling side of the link
to send information that can identify the user. The two methods for this
task are PAP and CHAP.

\protect\hypertarget{c21.xhtmlux5cux23Page_900}{}{}\textbf{Compression}
This is used to increase the throughput of PPP connections by
compressing the data or payload prior to transmission. PPP decompresses
the data frame on the receiving end.

\textbf{Error detection} PPP uses Quality and Magic Number options to
ensure a reliable, loop-free data link.

\textbf{Multilink PPP (MLP)} Starting with IOS version 11.1, multilink
is supported on PPP links with Cisco routers. This option makes several
separate physical paths appear to be one logical path at layer 3. For
example, two T1s running multilink PPP would show up as a single 3 Mbps
path to a layer 3 routing protocol.

\textbf{PPP callback} On a dial-up connection, PPP can be configured to
call back after successful authentication. \emph{PPP callback} can be a
very good thing because it allows us to keep track of usage based upon
access charges for accounting records and a bunch of other reasons. With
callback enabled, a calling router (client) will contact a remote router
(server) and authenticate. Predictably, both routers have to be
configured for the callback feature for this to work. Once
authentication is completed, the remote router will terminate the
connection and then reinitiate a connection to the calling router from
the remote router.

\subsubsection[PPP Session
Establishment]{\texorpdfstring{\protect\hypertarget{c21.xhtmlux5cux23c21-sec-14}{}{}PPP
Session Establishment}{PPP Session Establishment}}

When PPP connections are started, the links go through three phases of
session establishment, as shown in
\protect\hyperlink{c21.xhtmlux5cux23figure21-13}{Figure 21.13}:

\begin{figure}
\centering
\includegraphics{images/c21f013.jpg}
\caption{{\protect\hyperlink{c21.xhtmlux5cux23figureanchor21-13}{\textbf{FIGURE
21.13}} PPP session establishment}}
\end{figure}

\textbf{Link-establishment phase} LCP packets are sent by each PPP
device to configure and test the link. These packets contain a field
called Configuration Option that allows each device to see the size of
the data, the compression, and authentication. If no Configuration
Option field is present, then the default configurations will be used.

\textbf{Authentication phase} If required, either CHAP or PAP can be
used to authenticate a link. Authentication takes place before Network
layer protocol information is read, and it's also possible that
link-quality determination will occur simultaneously.

\textbf{Network layer protocol phase} PPP uses the \emph{Network Control
Protocol (NCP)} to allow multiple Network layer protocols to be
encapsulated and sent over a PPP data link. Each Network layer protocol
(e.g., IP, IPv6, which are routed protocols) establishes a service with
NCP.

\protect\hypertarget{c21.xhtmlux5cux23Page_901}{}{}

\subsubsection[PPP Authentication
Methods]{\texorpdfstring{\protect\hypertarget{c21.xhtmlux5cux23c21-sec-15}{}{}PPP
Authentication Methods}{PPP Authentication Methods}}

There are two methods of authentication that can be used with PPP links:

\textbf{Password Authentication Protocol (PAP)} The \emph{Password
Authentication Protocol (PAP)} is the less secure of the two methods.
Passwords are sent in clear text and PAP is performed only upon the
initial link establishment. When the PPP link is first established, the
remote node sends the username and password back to the originating
target router until authentication is acknowledged. Not exactly Fort
Knox!

\textbf{Challenge Handshake Authentication Protocol (CHAP)} The
\emph{Challenge Handshake Authentication Protocol (CHAP)} is used at the
initial startup of a link and at periodic checkups on the link to ensure
that the router is still communicating with the same host.

After PPP finishes its initial link-establishment phase, the local
router sends a challenge request to the remote device. The remote device
sends a value calculated using a one-way hash function called MD5. The
local router checks this hash value to make sure it matches. If the
values don't match, the link is immediately terminated.

\begin{center}\rule{0.5\linewidth}{0.5pt}\end{center}

\includegraphics{images/note.png}CHAP authenticates at the beginning of
the session and periodically throughout the session.

\begin{center}\rule{0.5\linewidth}{0.5pt}\end{center}

\subsubsection[Configuring PPP on Cisco
Routers]{\texorpdfstring{\protect\hypertarget{c21.xhtmlux5cux23c21-sec-16}{}{}Configuring
PPP on Cisco Routers}{Configuring PPP on Cisco Routers}}

Configuring PPP encapsulation on an interface is really pretty
straightforward. To configure it from the CLI, use these simple router
commands:

\begin{verbatim}
Router#config t
Router(config)#int s0
Router(config-if)#encapsulation ppp
Router(config-if)#^Z
\end{verbatim}

Of course, PPP encapsulation has to be enabled on both interfaces
connected to a serial line in order to work, and there are several
additional configuration options available to you via the
\texttt{ppp\ ?} command.

\subsubsection[Configuring PPP
Authentication]{\texorpdfstring{\protect\hypertarget{c21.xhtmlux5cux23c21-sec-17}{}{}Configuring
PPP Authentication}{Configuring PPP Authentication}}

After you configure your serial interface to support PPP encapsulation,
you can then configure authentication using PPP between routers. But
first, you must set the hostname of the router if it hasn't been set
already. After that, you set the username and password for the remote
router that will be connecting to your router, like this:

\begin{verbatim}
Router#config t
Router(config)#hostname RouterA
RouterA(config)#username RouterB password cisco
\end{verbatim}

\protect\hypertarget{c21.xhtmlux5cux23Page_902}{}{}When using the
\texttt{username} command, remember that the username is the hostname of
the remote router that's connecting to your router. And it's case
sensitive too. Also, the password on both routers must be the same. It's
a plain-text password that you can see with a \texttt{show\ run}
command, and you can encrypt the password by using the command
\texttt{service}\texttt{\ password-encryption}. You must have a username
and password configured for each remote system you plan to connect to.
The remote routers must also be similarly configured with usernames and
passwords.

Now, after you've set the hostname, usernames, and passwords, choose
either CHAP or PAP as the authentication method:

\begin{verbatim}
RouterA#config t
RouterA(config)#int s0
RouterA(config-if)#ppp authentication chap pap
RouterA(config-if)#^Z
\end{verbatim}

If both methods are configured on the same line as I've demonstrated
here, then only the first method will be used during link negotiation.
The second acts as a backup just in case the first method fails.

There is yet another command you can use if you're using PAP
authentication for some reason. The
\texttt{ppp\ pap\ sent-username\ \textless{}username\textgreater{}\ password\ \textless{}password\textgreater{}}
command enables outbound PAP authentication. The local router uses the
username and password that the \texttt{ppp\ pap\ sent-username} command
specifies to authenticate itself to a remote device. The other router
must have this same username/password configured as well.

\subsubsection[Verifying and Troubleshooting Serial
Links]{\texorpdfstring{\protect\hypertarget{c21.xhtmlux5cux23c21-sec-18}{}{}Verifying
and Troubleshooting Serial
Links}{Verifying and Troubleshooting Serial Links}}

Now that PPP encapsulation is enabled, you need to verify that it's up
and running. First, let's take a look at a figure of a sample
nonproduction network serial link.
\protect\hyperlink{c21.xhtmlux5cux23figure21-14}{Figure 21.14} shows two
routers connected with a point-to-point serial connection, with the DCE
side on the Pod1R1 router.

\begin{figure}
\centering
\includegraphics{images/c21f014.jpg}
\caption{{\protect\hyperlink{c21.xhtmlux5cux23figureanchor21-14}{\textbf{FIGURE
21.14}} PPP authentication example}}
\end{figure}

\protect\hypertarget{c21.xhtmlux5cux23Page_903}{}{}You can start
verifying the configuration with the \texttt{show\ interface} command
like this:

\begin{verbatim}
Pod1R1#sh int s0/0
Serial0/0 is up, line protocol is up
  Hardware is PowerQUICC Serial
  Internet address is 10.0.1.1/24
  MTU 1500 bytes, BW 1544 Kbit, DLY 20000 usec,
     reliability 239/255, txload 1/255, rxload 1/255
 Encapsulation PPP
  loopback not set
  Keepalive set (10 sec)
 LCP Open
 Open: IPCP, CDPCP
[output cut]
\end{verbatim}

The first line of output is important because it tells us that Serial
0/0 is up/up. Notice that the interface encapsulation is PPP and that
LCP is open. This means that it has negotiated the session establishment
and all is well. The last line tells us that NCP is listening for the
protocols IP and CDP, shown with the NCP headers IPCP and CDPCP.

But what would you see if everything isn't so perfect? I'm going to type
in the configuration shown in
\protect\hyperlink{c21.xhtmlux5cux23figure21-15}{Figure 21.15} to find
out.

\begin{figure}
\centering
\includegraphics{images/c21f015.jpg}
\caption{{\protect\hyperlink{c21.xhtmlux5cux23figureanchor21-15}{\textbf{FIGURE
21.15}} Failed PPP authentication}}
\end{figure}

What's wrong here? Take a look at the usernames and passwords. Do you
see the problem now? That's right, the \emph{C} is capitalized on the
Pod1R2 username command found in the configuration of router Pod1R1.
This is wrong because the usernames and passwords are case sensitive.
Now let's take a look at the \texttt{show\ interface} command and see
what happens:

\begin{verbatim}
Pod1R1#sh int s0/0
Serial0/0 is up, line protocol is down
  Hardware is PowerQUICC Serial
  Internet address is 10.0.1.1/24
  MTU 1500 bytes, BW 1544 Kbit, DLY 20000 usec,
     reliability 243/255, txload 1/255, rxload 1/255
  Encapsulation PPP, loopback not set
  Keepalive set (10 sec)
 LCP Closed
 Closed: IPCP, CDPCP
\end{verbatim}

First, notice that the first line of output shows us that
\texttt{Serial0/0\ is\ up} and \texttt{line\ protocol\ is\ down}. This
is because there are no keepalives coming from the remote router. The
next thing I want you to notice is that the LCP and NCP are closed
because the authentication failed.

\paragraph{Debugging PPP Authentication}

To display the CHAP authentication process as it occurs between two
routers in the network, just use the command
\texttt{debug\ ppp\ authentication}.

If your PPP encapsulation and authentication are set up correctly on
both routers and your usernames and passwords are all good, then the
\texttt{debug\ ppp\ authentication} command will display an output that
looks like the following output, which is called the three-way
handshake:

\begin{verbatim}
d16h: Se0/0 PPP: Using default call direction
1d16h: Se0/0 PPP: Treating connection as a dedicated line
1d16h: Se0/0 CHAP: O CHALLENGE id 219 len 27 from "Pod1R1"
1d16h: Se0/0 CHAP: I CHALLENGE id 208 len 27 from "Pod1R2"
1d16h: Se0/0 CHAP: O RESPONSE id 208 len 27 from "Pod1R1"
1d16h: Se0/0 CHAP: I RESPONSE id 219 len 27 from "Pod1R2"
1d16h: Se0/0 CHAP: O SUCCESS id 219 len 4
1d16h: Se0/0 CHAP: I SUCCESS id 208 len 4
\end{verbatim}

But if you have the password wrong as they were previously in the PPP
authentication failure example back in
\protect\hyperlink{c21.xhtmlux5cux23figure21-15}{Figure 21.15}, the
output would look something like this:

\begin{verbatim}
1d16h: Se0/0 PPP: Using default call direction
1d16h: Se0/0 PPP: Treating connection as a dedicated line
1d16h: %SYS-5-CONFIG_I: Configured from console by console
1d16h: Se0/0 CHAP: O CHALLENGE id 220 len 27 from "Pod1R1"
1d16h: Se0/0 CHAP: I CHALLENGE id 209 len 27 from "Pod1R2"
1d16h: Se0/0 CHAP: O RESPONSE id 209 len 27 from "Pod1R1"
1d16h: Se0/0 CHAP: I RESPONSE id 220 len 27 from "Pod1R2"
1d16h: Se0/0 CHAP: O FAILURE id 220 len 25 msg is "MD/DES compare failed"
\end{verbatim}

PPP with CHAP authentication is a three-way authentication, and if the
username and passwords aren't configured exactly the way they should be,
then the authentication will fail and the link will go down.

\protect\hypertarget{c21.xhtmlux5cux23Page_905}{}{}

\paragraph{Mismatched WAN Encapsulations}

If you have a point-to-point link but the encapsulations aren't the
same, the link will never come up.
\protect\hyperlink{c21.xhtmlux5cux23figure21-6}{Figure 21.16} shows one
link with PPP and one with HDLC.

\begin{figure}
\centering
\includegraphics{images/c21f016.jpg}
\caption{{\protect\hyperlink{c21.xhtmlux5cux23figureanchor21-16}{\textbf{FIGURE
21.16}} Mismatched WAN encapsulations}}
\end{figure}

Look at router Pod1R1 in this output:

\begin{verbatim}
Pod1R1#sh int s0/0
Serial0/0 is up, line protocol is down
  Hardware is PowerQUICC Serial
  Internet address is 10.0.1.1/24
  MTU 1500 bytes, BW 1544 Kbit, DLY 20000 usec,
     reliability 254/255, txload 1/255, rxload 1/255
 Encapsulation PPP, loopback not set
  Keepalive set (10 sec)
 LCP REQsent
Closed: IPCP, CDPCP
\end{verbatim}

The serial interface is up/down and LCP is sending requests but will
never receive any responses because router Pod1R2 is using the HDLC
encapsulation. To fix this problem, you would have to go to router
Pod1R2 and configure the PPP encapsulation on the serial interface. One
more thing: Even though the usernames are configured incorrectly, it
doesn't matter because the command \texttt{ppp\ authentication\ chap}
isn't used under the serial interface configuration. This means that the
username command isn't relevant in this example.

You can set a Cisco serial interface back to the default of HDLC with
the \texttt{no\ encapsulation} command like this:

\begin{verbatim}
Router(config)#int s0/0
Router(config-if)#no encapsulation
*Feb 7 16:00:18.678:%LINEPROTO-5-UPDOWN: Line protocol on Interface Serial0/0, changed state to up
\end{verbatim}

\protect\hypertarget{c21.xhtmlux5cux23Page_906}{}{}Notice the link came
up because it now matches the encapsulation on the other end of the
link!

\begin{center}\rule{0.5\linewidth}{0.5pt}\end{center}

\includegraphics{images/note.png}Always remember that you just can't
have PPP on one side and HDLC on the other---they don't get along!

\begin{center}\rule{0.5\linewidth}{0.5pt}\end{center}

\paragraph{Mismatched IP Addresses}

A tricky problem to spot is if you have HDLC or PPP configured on your
serial interface but your IP addresses are wrong. Things seem to be just
fine because the interfaces will show that they are up. Take a look at
\protect\hyperlink{c21.xhtmlux5cux23figure21-17}{Figure 21.17} and see
if you can see what I mean---the two routers are connected with
different subnets---router Pod1R1 with 10.0.1.1/24 and router Pod1R2
with 10.2.1.2/24.

\begin{figure}
\centering
\includegraphics{images/c21f017.jpg}
\caption{{\protect\hyperlink{c21.xhtmlux5cux23figureanchor21-17}{\textbf{FIGURE
21.17}} Mismatched IP addresses}}
\end{figure}

This will never work. Let's take a look at the output:

\begin{verbatim}
Pod1R1#sh int s0/0
Serial0/0 is up, line protocol is up
  Hardware is PowerQUICC Serial
  Internet address is 10.0.1.1/24
  MTU 1500 bytes, BW 1544 Kbit, DLY 20000 usec,
     reliability 255/255, txload 1/255, rxload 1/255
  Encapsulation PPP, loopback not set
  Keepalive set (10 sec)
  LCP Open
  Open: IPCP, CDPCP
\end{verbatim}

See that? The IP addresses between the routers are wrong but the link
appears to be working just fine. This is because PPP, like HDLC and
Frame Relay, is a layer 2 WAN encapsulation, so it doesn't care about
layer three addressing at all. So yes, the link is up, but you
\protect\hypertarget{c21.xhtmlux5cux23Page_907}{}{}can't use IP across
this link since it's misconfigured, or can you? Well, yes and no. If you
try to ping, you'll see that this actually works! This is a feature of
PPP, but not HDLC or Frame Relay. But just because you can ping to an IP
address that's not in the same subnet doesn't mean your network traffic
and routing protocols will work. So be careful with this issue,
especially when troubleshooting PPP links!

Take a look at the routing table of Pod1R1 and see if you can find the
mismatched IP address problem:

\begin{verbatim}
[output cut]
  10.0.0.0/8 is variably subnetted, 2 subnets, 2 masks
C       10.2.1.2/32 is directly connected, Serial0/0
C       10.0.1.0/24 is directly connected, Serial0/0
\end{verbatim}

Interesting! We can see our serial interface S0/0 address of
10.0.1.0/24, but what is that other address on interface
S0/0---10.2.1.2/32? That's our remote router's interface IP address! PPP
determines and places the neighbor's IP address in the routing table as
a connected interface, which then allows you to ping it even though it's
actually configured on a separate IP subnet.

\begin{center}\rule{0.5\linewidth}{0.5pt}\end{center}

\includegraphics{images/tip.png}For the Cisco objectives, you need to be
able to troubleshoot PPP from the routing table as I just described.

\begin{center}\rule{0.5\linewidth}{0.5pt}\end{center}

To find and fix this problem, you can also use the
\texttt{show\ running-config}, \texttt{show\ interfaces}, or
\texttt{show\ ip\ interfaces\ brief} command on each router, or you can
use the \texttt{show\ cdp\ neighbors\ detail} command:

\begin{verbatim}
Pod1R1#sh cdp neighbors detail
-------------------------
Device ID: Pod1R2
Entry address(es):
  IP address: 10.2.1.2
\end{verbatim}

Since the layer 1 Physical and layer 2 Data Link is up/up, you can view
and verify the directly connected neighbor's IP address and then solve
your problem.

\subsubsection[Multilink PPP
(MLP)]{\texorpdfstring{\protect\hypertarget{c21.xhtmlux5cux23c21-sec-19}{}{}Multilink
PPP (MLP)}{Multilink PPP (MLP)}}

There are many load-balancing mechanisms available, but this one is free
for use on serial WAN links! It provides multivendor support and is
specified in RFC 1990, which details the fragmentation and packet
sequencing specifications.

You can use MLP to connect your home network to an Internet service
provider using two traditional modems or to connect a company via two
leased lines.

The MLP feature provides a load-balancing functionality over multiple
WAN links while allowing for multivendor interoperability. It offers
support for packet fragmentation, proper sequencing, and load
calculation on both inbound and outbound traffic.

\protect\hypertarget{c21.xhtmlux5cux23Page_908}{}{}MLP allows packets to
be fragmented and then sent simultaneously over multiple point-to-point
links to the same remote address. It can work over synchronous and
asynchronous serial types.

MLP combines multiple physical links into a logical link called an MLP
bundle, which is essentially a single, virtual interface that connects
to the remote router. None of the links inside the bundle have any
knowledge about the traffic on the other links.

The MLP over serial interfaces feature provides us with the following
benefits:

\textbf{Load balancing} MLP provides bandwidth on demand, utilizing load
balancing on up to 10 links and can even calculate the load on traffic
between specific sites. You don't actually need to make all links the
same bandwidth, but doing so is recommended. Another key MLP advantage
is that it splits packets and fragments across all links, which reduces
latency across the WAN.

\textbf{Increased redundancy} This one is pretty straightforward\ldots{}
If a link fails, the others will still transmit and receive.

\textbf{Link fragmentation and interleaving} The fragmentation mechanism
in MLP works by fragmenting large packets, then sending the packet
fragments over the multiple point-to-point links. Smaller real-time
packets are not fragmented. So interleaving basically means real-time
packets can be sent in between sending the fragmented, non-real-time
packets, which helps reduce delay on the lines. So let's configure MLP
now to get a good feel for how it actually works now.

\paragraph{Configuring MLP}

We're going to use
\protect\hyperlink{c21.xhtmlux5cux23figure21-18}{Figure 21.18} to
demonstrate how to configure MLP between two routers.

\begin{figure}
\centering
\includegraphics{images/c21f018.jpg}
\caption{{\protect\hyperlink{c21.xhtmlux5cux23figureanchor21-18}{\textbf{FIGURE
21.18}} MLP between Corp and SF routers}}
\end{figure}

But first, I want you to study the configuration of the two serial
interfaces on the Corp router that we're going to use for making our
bundle:

\begin{verbatim}
Corp#show interfaces Serial0/0
Serial0/0 is up, line protocol is up
  Hardware is M4T
  Internet address is 172.16.10.1/30
  MTU 1500 bytes, BW 1544 Kbit/sec, DLY 20000 usec,
     reliability 255/255, txload 1/255, rxload 1/255
  Encapsulation PPP, LCP Open
  Open: IPCP, CDPCP, crc 16, loopback not set
\end{verbatim}

\begin{verbatim}
Corp#show interfaces Serial1/1
Serial1/1 is up, line protocol is up
  Hardware is M4T
  Internet address is 172.16.10.9/30
  MTU 1500 bytes, BW 1544 Kbit/sec, DLY 20000 usec,
     reliability 255/255, txload 1/255, rxload 1/255
  Encapsulation PPP, LCP Open
  Open: IPCP, CDPCP, crc 16, loopback not set
\end{verbatim}

Did you notice that each serial connection is on a different subnet
(they have to be) and that the encapsulation is PPP?

When you configure MLP, you must first remove your IP addresses off your
physical interface. Then, you configure a multilink bundle by creating a
multilink interface on both sides of the link. After that, you assign an
IP address to this multilink interface, which effectively restricts a
physical link so that it can only join the designated multilink group
interface.

So first I'm going to remove the IP addresses from the physical
interfaces that I'm going to include in my PPP bundle.

\begin{verbatim}
Corp#config t
Corp(config)#int Serial0/0
Corp(config-if)#no ip address
Corp(config-if)#int Serial1/1
Corp(config-if)#no ip address
Corp(config-if)#end
Corp#
\end{verbatim}

\begin{verbatim}
SF#config t
SF(config)#int Serial0/0
SF(config-if)#no ip address
SF(config-if)#int Serial0/1
SF(config-if)#no ip address
SF(config-if)#end
SF#
\end{verbatim}

Now we create the multilink interface on each side of the link and the
MLP commands to enable the bundle.

\begin{verbatim}
Corp#config t
Corp(config)#interface Multilink1
Corp(config-if)#ip address 10.1.1.1 255.255.255.0
Corp(config-if)#ppp multilink
Corp(config-if)#ppp multilink group 1
Corp(config-if)#end
\end{verbatim}

\begin{verbatim}
SF#config t
SF(config)#interface Multilink1
SF(config-if)#ip address 10.1.1.2 255.255.255.0
SF(config-if)#ppp multilink
SF(config-if)#ppp multilink group 1
SF(config-if)#exit
\end{verbatim}

We can see that a link joins an MLP bundle only if it negotiates to use
the bundle when a connection is established and the identification
information that has been exchanged matches the info for an existing
bundle.

When you configure the \texttt{ppp\ multilink\ group} command on a link,
that link won't be allowed to join any bundle other than the indicated
group interface.

\paragraph{Verifying MLP}

To verify that your bundle is up and running, just use the
\texttt{show\ ppp\ multilink} and \texttt{show\ interfaces\ multilink1}
commands:

\begin{verbatim}
Corp#show ppp multilink
\end{verbatim}

\begin{verbatim}
Multilink1
  Bundle name: Corp
  Remote Endpoint Discriminator: [1] SF
  Local Endpoint Discriminator: [1] Corp
  Bundle up for 02:12:05, total bandwidth 4188, load 1/255
  Receive buffer limit 24000 bytes, frag timeout 1000 ms
    0/0 fragments/bytes in reassembly list
    0 lost fragments, 53 reordered
    0/0 discarded fragments/bytes, 0 lost received
    0x56E received sequence, 0x572 sent sequence
  Member links: 2 active, 0 inactive (max 255, min not set)
    Se0/1, since 01:32:05
    Se1/2, since 01:31:31
No inactive multilink interfaces
\end{verbatim}

We can see that the physical interfaces, Se0/1 and Se1/1, are members of
the logical interface bundle Multilink 1. So now we'll verify the status
of the interface Multilink1 on the Corp router:

\begin{verbatim}
Corp#show int Multilink1
Multilink1 is up, line protocol is up
  Hardware is multilink group interface
  Internet address is 10.1.1.1/24
  MTU 1500 bytes, BW 1544 Kbit/sec, DLY 20000 usec,
     reliability 255/255, txload 1/255, rxload 1/255
Encapsulation PPP, LCP Open, multilink Open
Open: IPCP, CDPCP, loopback not set
 Keepalive set (10 sec)
[output cut]
\end{verbatim}

Let's move on to configure a PPPoE client on a Cisco router.

\subsubsection[PPP Client
(PPPoE)]{\texorpdfstring{\protect\hypertarget{c21.xhtmlux5cux23c21-sec-20}{}{}PPP
Client (PPPoE)}{PPP Client (PPPoE)}}

Used with ADSL services, PPPoE (Point-to-Point Protocol over Ethernet)
encapsulates PPP frames in Ethernet frames and uses common PPP features
like authentication, encryption, and compression. But as I said earlier,
it can be trouble. This is especially true if you've got a badly
configured firewall!

Basically, PPPoE is a tunneling protocol that layers IP and other
protocols running over PPP with the attributes of a PPP link. This is
done so protocols can then be used to contact other Ethernet devices and
initiate a point-to-point connection to transport IP packets.

\protect\hyperlink{c21.xhtmlux5cux23figure21-19}{Figure 21.19} displays
typical usage of PPPoE over ADSL. As you can see, a PPP session is
connected from the PC of the end user to the router. Subsequently, the
subscriber PC IP address is assigned by the router via IPCP.

\begin{figure}
\centering
\includegraphics{images/c21f019.jpg}
\caption{{\protect\hyperlink{c21.xhtmlux5cux23figureanchor21-19}{\textbf{FIGURE
21.19}} PPPoE with ADSL}}
\end{figure}

Your ISP will typically provide you with a DSL line and this will act as
a bridge if your line doesn't provide enhanced features. This means only
one host will connect using PPPoE. By using a Cisco router, you can run
the PPPoE client IOS feature which will connect multiple PCs on the
Ethernet segment that is connected to the router.

\protect\hypertarget{c21.xhtmlux5cux23Page_912}{}{}

\subsubsection[Configuring a PPPoE
Client]{\texorpdfstring{\protect\hypertarget{c21.xhtmlux5cux23c21-sec-21}{}{}Configuring
a PPPoE Client}{Configuring a PPPoE Client}}

The PPPoE client configuration is simple and straightforward. First, you
need to create a dialer interface and then tie it to a physical
interface.

Here are the easy steps:

\begin{enumerate}
\tightlist
\item
  Create a dialer interface using the \texttt{interface\ dialer\ number}
  command.
\item
  Instruct the client to use an IP address provided by the PPPoE server
  with the \texttt{ip\ address\ negotiated} command.
\item
  Set the encapsulation type to PPP.
\item
  Configure the dialer pool and number.
\item
  Under the physical interface, use the
  \texttt{pppoe-client\ dial-pool\ number\ number} command.
\end{enumerate}

On your PPPoE client router, enter the following commands:

\begin{verbatim}
R1#conf t
R1(config)#int dialer1
R1(config-if)#ip address negotiated
R1(config-if)#encapsulation ppp
R1(config-if)#dialer pool 1
R1(config-if)#interface f0/1
R1(config-if)#no ip address
R1(config-if)#pppoe-client dial-pool-number 1
*May 1 1:09:07.540: %DIALER-6-BIND: Interface Vi2 bound to profile Di1
*May 1 1:09:07.541: %LINK-3-UPDOWN: Interface Virtual-Access2, changed state to up
\end{verbatim}

That's it! Now let's verify the interface with
the\texttt{\ show\ ip\ interface\ brief} and the
\texttt{show\ pppoe\ session} commands:

\begin{verbatim}
R1#show ip int brief
Interface                 IP-Address     OK? Method Status            Protocol
FastEthernet0/1           unassigned     YES manual up                    up
<output cut>
Dialer1                    10.10.10.3    YES IPCP   up                    up
Loopback0                  192.168.1.1   YES NVRAM  up                    up
Loopback1                  172.16.1.1    YES NVRAM  up                    up
Virtual-Access1            unassigned    YES unset  up                    up
Virtual-Access2            unassigned    YES unset  up                    up
\end{verbatim}

\begin{verbatim}
R1#show pppoe session
     1 client session
\end{verbatim}

\begin{verbatim}
Uniq ID  PPPoE  RemMAC          Port                    VT  VA         State
           SID  LocMAC                                      VA-st      Type
    N/A      4  aacb.cc00.1419  FEt0/1                   Di1 Vi2        UP
                aacb.cc00.1f01                              UP
\end{verbatim}

Our connection using a PPPoE client is up and running. Now let's take a
look at VPNs.

\subsection[Virtual Private
Networks]{\texorpdfstring{\protect\hypertarget{c21.xhtmlux5cux23c21-sec-22}{}{}Virtual
Private Networks}{Virtual Private Networks}}

I'd be pretty willing to bet you've heard the term \emph{VPN} more than
once before. Maybe you even know what one is, but just in case, a
\emph{virtual private network (VPN)} allows the creation of private
networks across the Internet, enabling privacy and tunneling of IP and
non-TCP/IP protocols. VPNs are used daily to give remote users and
disjointed networks connectivity over a public medium like the Internet
instead of using more expensive permanent means.

No worries---VPNs aren't really that hard to understand. A VPN fits
somewhere between a LAN and WAN, with the WAN often simulating a LAN
link because your computer, on one LAN, connects to a different, remote
LAN and uses its resources remotely. The key drawback to using VPNs is a
big one---security! So the definition of connecting a LAN (or VLAN) to a
WAN may sound the same as using a VPN, but a VPN is actually much more.

Here's the difference: A typical WAN connects two or more remote LANs
together using a router and someone else's network, like, say, your
Internet service provider's. Your local host and router see these
networks as remote networks and not as local networks or local
resources. This would be a WAN in its most general definition. A VPN
actually makes your local host part of the remote network by using the
WAN link that connects you to the remote LAN. The VPN will make your
host appear as though it's actually local on the remote network. This
means that we now have access to the remote LAN's resources, and that
access is also very secure!

This may sound a lot like a VLAN definition, and really, the concept is
the same: ``Take my host and make it appear local to the remote
resources.'' Just remember this key distinction: For networks that are
physically local, using VLANs is a good solution, but for physically
remote networks that span a WAN, opt for using VPNs instead.

For a simple VPN example, let's use my home office in Boulder, Colorado.
Here, I have my personal host, but I want it to appear as if it's on a
LAN in my corporate office in Dallas, Texas, so I can get to my remote
servers. VPN is the solution I would opt for to achieve my goal.

\protect\hyperlink{c21.xhtmlux5cux23figure21-20}{Figure 21.20} shows
this example of my host using a VPN connection from Boulder to Dallas,
which allows me to access the remote network services and servers as if
my host were right there on the same VLAN as my servers.

\protect\hypertarget{c21.xhtmlux5cux23Page_914}{}{}

\begin{figure}
\centering
\includegraphics{images/c21f020.jpg}
\caption{{\protect\hyperlink{c21.xhtmlux5cux23figureanchor21-20}{\textbf{FIGURE
21.20}} Example of using a VPN}}
\end{figure}

Why is this so important? If you answered, ``Because my servers in
Dallas are secure, and only the hosts on the same VLAN are allowed to
connect to them and use the resources of these servers,'' you nailed it!
A VPN allows me to connect to these resources by locally attaching to
the VLAN through a VPN across the WAN. The other option is to open up my
network and servers to everyone on the Internet or another WAN service,
in which case my security goes ``poof.'' So clearly, it's imperative I
have a VPN!

\subsubsection[Benefits of
VPNs]{\texorpdfstring{\protect\hypertarget{c21.xhtmlux5cux23c21-sec-23}{}{}Benefits
of VPNs}{Benefits of VPNs}}

There are many benefits to using VPNs on your corporate and even home
network. The benefits covered in the CCNA R/S objectives are as follows:

\textbf{Security} VPNs can provide very good security by using advanced
encryption and authentication protocols, which will help protect your
network from unauthorized access. IPsec and SSL fall into this category.
Secure Sockets Layer (SSL) is an encryption technology used with web
browsers, which has native SSL encryption, and is known as Web VPN. You
can also use the Cisco AnyConnect SSL VPN client installed on your PC to
provide an SSL VPN solution, as well as the Clientless Cisco SSL VPN.

\textbf{Cost savings} By connecting the corporate remote offices to
their closest Internet provider, and then creating a VPN tunnel with
encryption and authentication, I gain a huge savings over opting for
traditional leased point-to-point lines. This also permits higher
bandwidth links and security, all for far less money than traditional
connections.

\textbf{Scalability} VPNs scale very well to quickly bring up new
offices or have mobile users connect securely while traveling or when
connecting from home.

\textbf{Compatibility with broadband technology} For remote and
traveling users and remote offices, any Internet access can provide a
connection to the corporate VPN. This allows users to take advantage of
the high-speed Internet access of DSL or cable modems.

\protect\hypertarget{c21.xhtmlux5cux23Page_915}{}{}

\subsubsection[Enterprise- and Provider-Managed
VPNs]{\texorpdfstring{\protect\hypertarget{c21.xhtmlux5cux23c21-sec-24}{}{}Enterprise-
and Provider-Managed VPNs}{Enterprise- and Provider-Managed VPNs}}

VPNs are categorized based upon the role they play in a business---for
example, enterprise-managed VPNs and provider-managed VPNs.

You'd use an enterprise-managed VPNs if your company manages its own
VPNs, which happens to be a very popular way of providing this service.
To get a picture of this, check out
\protect\hyperlink{c21.xhtmlux5cux23figure21-21}{Figure 21.21}.

\begin{figure}
\centering
\includegraphics{images/c21f021.jpg}
\caption{{\protect\hyperlink{c21.xhtmlux5cux23figureanchor21-21}{\textbf{FIGURE
21.21}} Enterprise-managed VPNs}}
\end{figure}

There are three different categories of enterprise-managed VPNs:

\begin{enumerate}
\tightlist
\item
  \emph{Remote-access VPNs} allow remote users such as telecommuters to
  securely access the corporate network wherever and whenever they need
  to.
\item
  \emph{Site-to-site VPNs}, or intranet VPNs, allow a company to connect
  its remote sites to the corporate backbone securely over a public
  medium like the Internet instead of requiring more expensive WAN
  connections like Frame Relay.
\item
  \emph{Extranet VPNs} allow an organization's suppliers, partners, and
  customers to be connected to the corporate network in a limited way
  for business-to-business (B2B) communications.
\end{enumerate}

Provider-managed VPNs are pictured in
\protect\hyperlink{c21.xhtmlux5cux23figure21-22}{Figure 21.22}.

\protect\hypertarget{c21.xhtmlux5cux23Page_916}{}{}

\begin{figure}
\centering
\includegraphics{images/c21f022.jpg}
\caption{{\protect\hyperlink{c21.xhtmlux5cux23figureanchor21-22}{\textbf{FIGURE
21.22}} Provider-managed VPNs}}
\end{figure}

And you need to be familiar with the two different categories of
provider-managed VPNs:

\textbf{Layer 2 MPLS VPN} Layer 2 VPNs are a type of virtual private
network (VPN) that use MPLS labels to transport data. The communication
occurs between routers known as provider edge routers (PEs) because they
sit on the edge of the provider's network, next to the customer's
network.

Internet providers who have an existing layer 2 network can opt to use
these VPNs instead of the other common layer 3 MPLS VPNs.

There are two typical technologies of layer 2 MPLS VPNs:

\begin{enumerate}
\tightlist
\item
  \textbf{Virtual private wire service (VPWS)} VPWS is the simplest form
  for enabling Ethernet services over MPLS. It is also known as ETHoMPLS
  (Ethernet over MPLS), or VLL (Virtual Leased Line). VPWS has the
  characteristics of a fixed relationship between an attachment-virtual
  circuit and an emulated virtual circuit. VPWS-based services are
  point-to-point, such as, for example, Frame-Relay/ATM/Ethernet
  services over IP/MPLS.
\item
  \textbf{Virtual private LAN switching service (VPLS)} This is an
  end-to-end service and is virtual because multiple instances of this
  service share the same Ethernet broadcast domain virtually. However,
  each connection is independent and isolated from the others in the
  network. A dynamic, ``learned'' relationship exists between an
  attachment-virtual circuit and emulated virtual circuits, which is
  determined by the customer's MAC address.
\item
  In this type of network, the customer manages its own routing
  protocols. One advantage that a layer 2 VPN has over its layer 3
  counterpart is that some applications just won't work if nodes are not
  in the same layer 2 network.
\end{enumerate}

\protect\hypertarget{c21.xhtmlux5cux23Page_917}{}{}\textbf{Layer 3 MPLS
VPN} Layer 3 MPLS VPN provides a layer 3 service across the backbone. A
different IP subnet connects each site. Since you would typically deploy
a routing protocol over this VPN, you must communicate with the service
provider to participate in the exchange of routes. Neighbor adjacency is
established between your router, called CE, and the provider router
that's called PE. The service provider network has many core routers
called P routers and it's the P routers' job to provide connectivity
between the PE routers.

If you really want to totally outsource your layer 3 VPN, then this
service is for you. Your service provider will maintain and manage
routing for all your sites. From your perspective as a customer who's
outsourced your VPNs, this will appear to you that your service provider
network is one big virtual switch.

Now you're interested in VPNs, huh? And since VPNs are inexpensive and
secure, I'm guessing you just can't wait to find out how to create VPNs
now! There's more than one way to bring a VPN into being. The first
approach uses IPsec to create authentication and encryption services
between endpoints on an IP network. The second way is via tunneling
protocols, which allow you to establish a tunnel between endpoints on a
network. And understand that the tunnel itself is a means for data or
protocols to be encapsulated inside another protocol---pretty clean!

I'm going to go over IPsec in a minute, but first I really want to
describe four of the most common tunneling protocols in use today:

\begin{enumerate}
\tightlist
\item
  \emph{Layer 2 Forwarding (L2F)} is a Cisco-proprietary tunneling
  protocol, and it was Cisco's first tunneling protocol created for
  virtual private dial-up networks (VPDNs). A VPDN allows a device to
  use a dial-up connection to create a secure connection to a corporate
  network. L2F was later replaced by L2TP, which is backward compatible
  with L2F.
\item
  \emph{Point-to-Point Tunneling Protocol (PPTP)} was created by
  Microsoft and others to allow the secure transfer of data from remote
  networks to the corporate network.
\item
  \emph{Layer 2 Tunneling Protocol (L2TP)} was created by Cisco and
  Microsoft to replace L2F and PPTP. L2TP merged the capabilities of
  both L2F and PPTP into one tunneling protocol.
\item
  \emph{Generic Routing Encapsulation (GRE)} is another
  Cisco-proprietary tunneling protocol. It forms virtual point-to-point
  links, allowing for a variety of protocols to be encapsulated in IP
  tunnels. I'll cover GRE in more detail, including how to configure it,
  at the end of this chapter.
\end{enumerate}

Now that you're clear on both exactly what a VPN is and the various
types of VPNs available, it's time to dive into IPsec.

\subsubsection[Introduction to Cisco IOS
IPsec]{\texorpdfstring{\protect\hypertarget{c21.xhtmlux5cux23c21-sec-25}{}{}Introduction
to Cisco IOS IPsec}{Introduction to Cisco IOS IPsec}}

Simply put, IPsec is an industry-wide standard framework of protocols
and algorithms that allows for secure data transmission over an IP-based
network and functions at the layer 3 Network layer of the OSI model.

Did you notice I said IP-based network? That's really important because
by itself, IPsec can't be used to encrypt non-IP traffic. This means
that if you run into a situation where
\protect\hypertarget{c21.xhtmlux5cux23Page_918}{}{}you have to encrypt
non-IP traffic, you'll need to create a Generic Routing Encapsulation
(GRE) tunnel for it (which I explain later) and then use IPsec to
encrypt that tunnel!

\subsubsection[IPsec
Transforms]{\texorpdfstring{\protect\hypertarget{c21.xhtmlux5cux23c21-sec-26}{}{}IPsec
Transforms}{IPsec Transforms}}

An \emph{IPsec transform} specifies a single security protocol with its
corresponding security algorithm; without these transforms, IPsec
wouldn't be able to give us its glory. It's important to be familiar
with these technologies, so let me take a second to define the security
protocols and briefly introduce the supporting encryption and hashing
algorithms that IPsec relies upon.

\paragraph{Security Protocols}

The two primary security protocols used by IPsec are
\emph{Authentication Header (AH)} and \emph{Encapsulating Security
Payload (ESP)}.

\subparagraph{Authentication Header (AH)}

The AH protocol provides authentication for the data and the IP header
of a packet using a one-way hash for packet authentication. It works
like this: The sender generates a one-way hash; then the receiver
generates the same one-way hash. If the packet has changed in any way,
it won't be authenticated and will be dropped since the hash values
won't match. So basically, IPsec relies upon AH to guarantee
authenticity. AH checks the entire packet, but it doesn't offer any
encryption services.

This is unlike ESP, which only provides an integrity check on the data
of a packet.

\subparagraph{Encapsulating Security Payload (ESP)}

It won't tell you when or how the NASDAQ's gonna bounce up and down like
a superball, but ESP will provide confidentiality, data origin
authentication, connectionless integrity, anti-replay service, and
limited traffic-flow confidentiality by defeating traffic flow
analysis---which is almost as good! Anyway, there are five components of
ESP:

\textbf{Confidentiality (encryption)} This allows the sending device to
encrypt the packets before transmitting in order to prevent
eavesdropping. Confidentiality is provided through the use of symmetric
encryption algorithms like DES or 3DES. Confidentiality can be selected
separately from all other services, but the confidentiality selected
must be the same on both endpoints of your VPN.

\textbf{Data integrity} Data integrity allows the receiver to verify
that the data received was not altered in any way along the way. IPsec
uses checksums as a simple check of the data.

\textbf{Authentication} Authentication ensures that the connection is
made with the correct partner. The receiver can authenticate the source
of the packet by guaranteeing and certifying the source of the
information.

\textbf{Anti-replay service} Anti-replay election is based upon the
receiver, meaning the service is effective only if the receiver checks
the sequence number. In case you were wondering, a replay attack is when
a hacker nicks a copy of an authenticated packet and later transmits
\protect\hypertarget{c21.xhtmlux5cux23Page_919}{}{}it to the intended
destination. When the duplicate, authenticated IP packet gets to the
destination, it can disrupt services and generally wreak havoc. The
\emph{Sequence Number} field is designed to foil this type of attack.

\textbf{Traffic flow} For traffic flow confidentiality to work, you have
to have at least tunnel mode selected. It's most effective if it's
implemented at a security gateway where tons of traffic amasses because
it's precisely the kind of environment that can mask the true
source-destination patterns to bad guys who are trying to breach your
network's security.

\paragraph{Encryption}

VPNs create a private network over a public network infrastructure, but
to maintain confidentiality and security, we really need to use IPsec
with our VPNs. IPsec uses various types of protocols to perform
encryption. The types of encryption algorithms used today are as
follows:

\textbf{Symmetric encryption} This encryption requires a shared secret
to encrypt and decrypt. Each computer encrypts the data before sending
info across the network, with this same key being used to both encrypt
and decrypt the data. Examples of symmetric key encryption are Data
Encryption Standard (DES), Triple DES (3DES), and Advanced Encryption
Standard (AES).

\textbf{Asymmetric encryption} Devices that use asymmetric encryption
use different keys for encryption than they do for decryption. These
keys are called private and public keys.

Private keys encrypt a hash from the message to create a digital
signature, which is then verified via decryption using the public key.
Public keys encrypt a symmetric key for secure distribution to the
receiving host, which then decrypts that symmetric key using its
exclusively held private key. It's not possible to encrypt and decrypt
using the same key. This is a variant of public key encryption that uses
a combination of both a public and private keys. An example of an
asymmetric encryption is Rivest, Shamir, and Adleman (RSA).

As you can see from the amount of information I've thrown at you so far,
establishing a VPN connection between two sites takes study, time, and
practice. And I am just scratching the surface here! I know it can be
difficult at times, and it can take quite a bit of patience. Cisco does
have some GUI interfaces to help with this process, and they can be very
helpful for configuring VPNs with IPsec. Though highly useful and very
interesting, they are just beyond the scope of this book, so I'm not
going to delve further into this topic here.

\subsection[GRE
Tunnels]{\texorpdfstring{\protect\hypertarget{c21.xhtmlux5cux23c21-sec-27}{}{}GRE
Tunnels}{GRE Tunnels}}

Generic Routing Encapsulation (GRE) is a tunneling protocol that can
encapsulate many protocols inside IP tunnels. Some examples would be
routing protocols such as EIGRP and OSPF and the routed protocol IPv6.
\protect\hyperlink{c21.xhtmlux5cux23figure21-23}{Figure 21.23} shows the
different pieces of a GRE header.

\protect\hypertarget{c21.xhtmlux5cux23Page_920}{}{}

\begin{figure}
\centering
\includegraphics{images/c21f023.jpg}
\caption{{\protect\hyperlink{c21.xhtmlux5cux23figureanchor21-23}{\textbf{FIGURE
21.23}} Generic Routing Encapsulation (GRE) tunnel structure}}
\end{figure}

A GRE tunnel interface supports a header for each of the following:

\begin{enumerate}
\tightlist
\item
  A passenger protocol or encapsulated protocols like IP or IPv6, which
  is the protocol being encapsulated by GRE
\item
  GRE encapsulation protocol
\item
  A transport delivery protocol, typically IP
\end{enumerate}

GRE tunnels have the following characteristics:

\begin{enumerate}
\tightlist
\item
  GRE uses a protocol-type field in the GRE header so any layer 3
  protocol can be used through the tunnel.
\item
  GRE is stateless and has no flow control.
\item
  GRE offers no security.
\item
  GRE creates additional overhead for tunneled packets---at least 24
  bytes.
\end{enumerate}

\subsubsection[GRE over
IPsec]{\texorpdfstring{\protect\hypertarget{c21.xhtmlux5cux23c21-sec-28}{}{}GRE
over IPsec}{GRE over IPsec}}

As I just mentioned, GRE by itself provides no security---no form of
payload confidentiality or encryption. If the packets are sniffed over
the public networks, their contents are in plain-text, and although
IPsec provides a secure method for tunneling data across an IP network,
it has limitations.

IPsec does not support IP broadcast or IP multicast, preventing the use
of protocols that need them, like routing protocols. IPsec also does not
support the use of the multiprotocol traffic. GRE is a protocol that can
be used to ``carry'' other passenger protocols like IP broadcast or IP
multicast, as well as non-IP protocols. So using GRE tunnels with IPsec
allows you to run a routing protocol, IP multicast, as well as
multiprotocol traffic across your network.

With a generic hub-and-spoke topology (corp to branch, for example), you
can implement static tunnels, typically GRE over IPsec, between the
corporate office and branch offices. When you want to add a new spoke to
the network, all you need to do is configure it on the hub router. The
traffic between spokes has to traverse the hub, where it
\protect\hypertarget{c21.xhtmlux5cux23Page_921}{}{}must exit one tunnel
and enter another. Static tunnels can be an appropriate solution for
small networks, but this solution actually becomes an unacceptable
problem as the number of spokes grows larger and larger!

\paragraph{Cisco DMVPN (Cisco Proprietary)}

The Cisco Dynamic Multipoint Virtual Private Network (DMVPN) feature
enables you to easily scale large and small IPsec VPNs. The Cisco DMVPN
is Cisco's answer to allow a corporate office to connect to branch
offices with low cost, easy configuration, and flexibility. DMVPN has
one central router, such as a corporate router, which is referred to as
the hub, and the branches are called spokes. So the corporate to branch
connection is referred to as the hub-and-spoke interconnection. Also
supported is the spoke-to-spoke design used for branch-to-branch
interconnections. If you're thinking this design sounds eerily similar
to your old Frame Relay network, you're right! The DMPVN features
enables you to configure a single GRE tunnel interface and a single
IPsec profile on the hub router to manage all spoke routers, which keeps
the size of the configuration on the hub router basically the same even
if you add more spoke routers to the network. DMVPN also allows spoke
router to dynamically create VPN tunnels between them as network data
travels from one spoke to another.

\paragraph{Cisco IPsec VTI (Cisco Proprietary)}

The IPsec Virtual Tunnel Interface (VTI) mode of an IPsec configuration
can greatly simplify a VPN configuration when protection is needed for
remote access. And it's a simpler option to GRE or L2TP for
encapsulation and crypto maps used with IPsec. Like GRE, it sends
routing protocol and multicast traffic, but you don't need the GRE
protocol and all the overhead that brings. A nice simple configuration
and routing adjacency directly over the VTI offers many benefits.
Understand that all traffic is encrypted and that it supports only one
protocol---either IPv4 or IPv6, just like standard IPsec.

Now let's take a look at how to configure a GRE tunnel. It's actually
pretty simple.

\subsubsection[Configuring GRE
Tunnels]{\texorpdfstring{\protect\hypertarget{c21.xhtmlux5cux23c21-sec-29}{}{}Configuring
GRE Tunnels}{Configuring GRE Tunnels}}

Before you attempt to configure a GRE tunnel, you need to create an
implementation plan. Here's a checklist for what you need to configure
and implement a GRE:

\begin{enumerate}
\tightlist
\item
  Use IP addressing.
\item
  Create the logical tunnel interfaces.
\item
  Specify that you're using GRE tunnel mode under the tunnel interface
  (this is optional since this is the default tunnel mode).
\item
  Specify the tunnel source and destination IP addresses.
\item
  Configure an IP address for the tunnel interface.
\end{enumerate}

Let's take a look at how to bring up a simple GRE tunnel.
\protect\hyperlink{c21.xhtmlux5cux23figure21-24}{Figure 21.24} shows the
network with two routers.

\protect\hypertarget{c21.xhtmlux5cux23Page_922}{}{}

\begin{figure}
\centering
\includegraphics{images/c21f024.jpg}
\caption{{\protect\hyperlink{c21.xhtmlux5cux23figureanchor21-24}{\textbf{FIGURE
21.24}} Example of GRE configuration}}
\end{figure}

First, we need to make the logical tunnel with the
\texttt{interface\ tunnel\ number} command. We can use any number up to
2.14 billion.

\begin{verbatim}
Corp(config)#int s0/0/0
Corp(config-if)#ip address 63.1.1.1 255.255.255.252
Corp(config)#int tunnel ?
  <0-2147483647>  Tunnel interface number
Corp(config)#int tunnel 0
*Jan 5 16:58:22.719:%LINEPROTO-5-UPDOWN: Line protocol on Interface Tunnel0, changed state to down
\end{verbatim}

Once we have configured our interface and created the logical tunnel, we
need to configure the mode and then the transport protocol.

\begin{verbatim}
Corp(config-if)#tunnel mode ?
  aurp    AURP TunnelTalk AppleTalk encapsulation
  cayman  Cayman TunnelTalk AppleTalk encapsulation
  dvmrp   DVMRP multicast tunnel
  eon     EON compatible CLNS tunnel
  gre     generic route encapsulation protocol
  ipip    IP over IP encapsulation
  ipsec   IPSec tunnel encapsulation
  iptalk  Apple IPTalk encapsulation
  ipv6    Generic packet tunneling in IPv6
  ipv6ip  IPv6 over IP encapsulation
  nos     IP over IP encapsulation (KA9Q/NOS compatible)
  rbscp   RBSCP in IP tunnel
Corp(config-if)#tunnel mode gre ?
  ip          over IP
  ipv6        over IPv6
  multipoint  over IP (multipoint)
\end{verbatim}

\begin{verbatim}
Corp(config-if)#tunnel mode gre ip
\end{verbatim}

Now that we've created the tunnel interface, the type, and the transport
protocol, we must configure our IP addresses for use inside of the
tunnel. Of course, you need to use
\protect\hypertarget{c21.xhtmlux5cux23Page_923}{}{}your actual physical
interface IP for the tunnel to send traffic across the Internet, but you
also need to configure the tunnel source and tunnel destination
addresses.

\begin{verbatim}
Corp(config-if)#ip address 192.168.10.1 255.255.255.0
Corp(config-if)#tunnel source 63.1.1.1
Corp(config-if)#tunnel destination 63.1.1.2
\end{verbatim}

\begin{verbatim}
Corp#sho run interface tunnel 0
Building configuration...
\end{verbatim}

\begin{verbatim}
Current configuration : 117 bytes
!
interface Tunnel0
 ip address 192.168.10.1 255.255.255.0
 tunnel source 63.1.1.1
 tunnel destination 63.1.1.2
end
\end{verbatim}

Now let's configure the other end of the serial link and watch the
tunnel pop up!

\begin{verbatim}
SF(config)#int s0/0/0
SF(config-if)#ip address 63.1.1.2 255.255.255.252
SF(config-if)#int t0
SF(config-if)#ip address 192.168.10.2 255.255.255.0
SF(config-if)#tunnel source 63.1.1.2
SF(config-if)#tun destination 63.1.1.1
*May 19 22:46:37.099: %LINEPROTO-5-UPDOWN: Line protocol on Interface Tunnel0, changed state to up
\end{verbatim}

Oops---did I forget to set my tunnel mode and transport to GRE and IP on
the SF router? No, I didn't need to because it's the default tunnel mode
on Cisco IOS. Nice! So, first I set the physical interface IP address
(which used a global address even though I didn't have to), then I
created the tunnel interface and set the IP address of the tunnel
interface. It's really important that you remember to configure the
tunnel interface with the actual source and destination IP addresses to
use or the tunnel won't come up. In my example, the 63.1.1.2 was the
source and 63.1.1.1 was the destination.

\subsubsection[Verifying GRP
Tunnels]{\texorpdfstring{\protect\hypertarget{c21.xhtmlux5cux23c21-sec-30}{}{}Verifying
GRP Tunnels}{Verifying GRP Tunnels}}

As usual I'll start with my favorite troubleshooting command,
\texttt{show\ ip\ interface\ brief}.

\begin{verbatim}
Corp#sh ip int brief
Interface        IP-Address      OK? Method Status                Protocol
FastEthernet0/0  10.10.10.5      YES manual up                    up
Serial0/0        63.1.1.1        YES manual up                    up
FastEthernet0/1  unassigned      YES unset  administratively down down
Serial0/1        unassigned      YES unset  administratively down down
Tunnel0          192.168.10.1    YES manual up                    up
\end{verbatim}

In this output, you can see that the tunnel interface is now showing as
an interface on my router. You can see the IP address of the tunnel
interface, and the Physical and Data Link status show as up/up. So far
so good. Let's take a look at the interface with the
\texttt{show\ interfaces\ tunnel\ 0} command.

\begin{verbatim}
Corp#sh int tun 0
Tunnel0 is up, line protocol is up
  Hardware is Tunnel
  Internet address is 192.168.10.1/24
  MTU 1514 bytes, BW 9 Kbit, DLY 500000 usec,
     reliability 255/255, txload 1/255, rxload 1/255
 Encapsulation TUNNEL, loopback not set
  Keepalive not set
 Tunnel source 63.1.1.1, destination 63.1.1.2
 Tunnel protocol/transport GRE/IP
    Key disabled, sequencing disabled
    Checksumming of packets disabled
  Tunnel TTL 255
  Fast tunneling enabled
  Tunnel transmit bandwidth 8000 (kbps)
  Tunnel receive bandwidth 8000 (kbps)
\end{verbatim}

The \texttt{show\ interfaces} command shows the configuration settings
and the interface status as well as the IP address, tunnel source, and
destination address. The output also shows the tunnel protocol, which is
GRE/IP. Last, let's take a look at the routing table with the
\texttt{show\ ip\ route} command.

\begin{verbatim}
Corp#sh ip route
[output cut]
     192.168.10.0/24 is subnetted, 2 subnets
C      192.168.10.0/24 is directly connected, Tunnel0
L      192.168.10.1/32 is directly connected, Tunnel0
     63.0.0.0/30 is subnetted, 2 subnets
C      63.1.1.0 is directly connected, Serial0/0
L      63.1.1.1/32 is directly connected, Serial0/0
\end{verbatim}

The \texttt{tunnel0} interface shows up as a directly connected
interface, and although it's a logical interface, the router treats it
as a physical interface, just like serial 0/0 in the routing table.

\begin{verbatim}
Corp#ping 192.168.10.2
\end{verbatim}

\begin{verbatim}
Type escape sequence to abort.
Sending 5, 100-byte ICMP Echos to 192.168.10.2, timeout is 2 seconds:
!!!!!
Success rate is 100 percent (5/5)
\end{verbatim}

Did you notice that I just pinged 192.168.10.2 across the Internet? I
hope so! Anyway, there's one last thing I want to cover before we move
on to EBGP, and that's troubleshooting an output, which is showing a
tunnel routing error. If you configure your GRE tunnel and receive this
GRE flapping message

\begin{verbatim}
          Line protocol on Interface Tunnel0, changed state to up
07:11:55: %TUN-5-RECURDOWN:
          Tunnel0 temporarily disabled due to recursive routing
07:11:59: %LINEPROTO-5-UPDOWN:
          Line protocol on Interface Tunnel0, changed state to down
07:12:59: %LINEPROTO-5-UPDOWN:
\end{verbatim}

it means that you've misconfigured your tunnel, which will cause your
router to try and route to the tunnel destination address using the
tunnel interface itself!

\subsection[Single-Homed
EBGP]{\texorpdfstring{\protect\hypertarget{c21.xhtmlux5cux23c21-sec-31}{}{}Single-Homed
EBGP}{Single-Homed EBGP}}

The \emph{Border Gateway Protocol (BGP)} is perhaps one of the most
well-known routing protocols in the world of networking. This is
understandable because BGP is the routing protocol that powers the
Internet and makes possible what we take for granted: connecting to
remote systems on the other side of the country or planet. Because of
its pervasive use, it's likely that each of us will have to deal with it
at some point in our careers. So it's appropriate that we spend some
time learning about BGP.

BGP version 4 has a long and storied history. Although the most recent
definition was published in 1995 as RFC 1771 by Rekhter and Li, BGP's
roots can be traced back to RFCs 827 and 904, which specified a protocol
called the exterior gateway protocol (EGP). These earlier specifications
date from 1982 and 1984, respectively---ages ago! Although BGP obsoletes
EGP, it uses many of the techniques first defined by EGP and draws upon
the many lessons learned from its use.

Way back in 1982, many organizations were connected to the ARPAnet, the
noncommercial predecessor of the Internet. When a new network was added
to ARPAnet, it would typically be added in a relatively unstructured way
and would begin participating in a common routing protocol, the Gateway
to Gateway Protocol (GGP). As you might expect, this solution did not
scale well. GGP suffered from excessive overhead in managing large
routing tables and from the difficulty of troubleshooting in an
environment in which there was no central administrative control.

\protect\hypertarget{c21.xhtmlux5cux23Page_926}{}{}To address these
deficiencies, EGP was developed, and with it the concept of
\emph{autonomous systems (ASs)}. RFC 827 was very clear in laying out
the problems with GGP and in pointing out that a new type of routing
protocol was required, an \emph{exterior gateway protocol (EGP)}. The
purpose of this new protocol was to facilitate the flow of traffic among
a series of autonomous systems by exchanging information about routes
contained in each system. The complexities of this network of networks,
the Internet, would be hidden from the end user who simply views the
Internet as a single address space through which they travel, unaware of
the exact path they take.

The BGP that we know today flows directly from this work on EGP and
builds upon it. So that you can get a better understanding of BGP, I
will provide an overview of its features next.

\subsubsection[Protocol Comparison and
Overview]{\texorpdfstring{\protect\hypertarget{c21.xhtmlux5cux23c21-sec-32}{}{}Protocol
Comparison and Overview}{Protocol Comparison and Overview}}

Because BGP is the first exterior gateway protocol we've encountered,
I'll briefly compare it to a more familiar interior gateway protocol,
like OSPF, so that we can put its features into context. After that
comes a brief overview of BGP so that you can quickly become aware of
its main capabilities.

Just to clarify things, with this comparison, I'm not implying that OSPF
could be a substitute for BGP. In fact, there are a number of reasons
that BGP is far better suited as an exterior gateway protocol than OSPF.
For example, the requirement that all OSPF areas be connected to area 0
simply doesn't allow OSPF to scale to the size required by the Internet.
Thousands of areas would have to connect to area 0, overwhelming it with
route updates. In addition, OSPF uses a metric based on bandwidth, but
in the context of the Internet, routing decisions are also based on
political and business issues. OSPF does not have any mechanism to
modify path selection based upon factors such as interconnection
agreements between Internet service providers.

Although BGP can be thought of as just another routing protocol, the
differences between it and protocols like OSPF are significant enough to
catapult it into an entirely different category.
\protect\hyperlink{c21.xhtmlux5cux23table21-1}{Table 21.1} contains a
comparison of BGP and OSPF.

{\protect\hyperlink{c21.xhtmlux5cux23tableanchor21-1}{\textbf{Table
21.1}} Comparison of BGP and OSPF}

\begin{longtable}[]{@{}lll@{}}
\toprule
Characteristic & BGP & OSPF\tabularnewline
\midrule
\endhead
Routing algorithm & Distance vector & Link state\tabularnewline
Classless support & Yes & Yes\tabularnewline
VLSM support & Yes & Yes\tabularnewline
Summarization & Any BGP router & ASBR/ABR\tabularnewline
\protect\hypertarget{c21.xhtmlux5cux23Page_927}{}{}Metric & Various &
Bandwidth\tabularnewline
Hierarchy & No & Yes\tabularnewline
Building blocks & Autonomous systems & Areas\tabularnewline
Base protocol & TCP port 179 & Protocol value 89\tabularnewline
Traffic type & Unicast & Multicast\tabularnewline
Neighbors & Specifically configured &
Discovered/configured\tabularnewline
Route exchange & Only with neighbors & Only with adjacent
neighbors\tabularnewline
Initial update & Synchronize database & Synchronize
database\tabularnewline
Update frequency & Incremental & Incremental with 60-minute
timer\tabularnewline
Hello timer & 60 seconds & 10 or 30 seconds\tabularnewline
Hold timer & 180 seconds & 40 or 120 seconds\tabularnewline
Internal route exchange & Internal BGP sessions & LSA Types 1 and
2\tabularnewline
External route exchange & External BGP sessions & LSA Types 3, 4, and
5\tabularnewline
Route updates & Contain network, attributes, AS path & Contain network,
metric (Types 3 and 4 LSAs)\tabularnewline
Network statement & Advertises network & Activates OSPF on
interface\tabularnewline
Special features & Route reflectors & Stub, totally stubby, NSSA
areas\tabularnewline
\bottomrule
\end{longtable}

It's easy to get lost in the specific details of BGP, so at the risk of
a small amount of repetition later on, the following is a high-level
overview of the BGP protocol and its main characteristics as listed in
\protect\hyperlink{c21.xhtmlux5cux23table21-1}{Table 21.1}.

BGP is a distance-vector protocol, which means that it advertises all or
a portion of its route table to its neighbors. The advertised routes
include the network being advertised, a list of attributes that
influence the selection of the best path, the next-hop address through
which the network can be reached, and a list of autonomous systems (ASs)
through which the route update has passed. BGP routers use the list of
autonomous systems to ensure a
\protect\hypertarget{c21.xhtmlux5cux23Page_928}{}{}loop-free path by
enforcing the rule that no AS path list is allowed to contain the same
AS number twice.

BGP supports classless networks, the use of variable length subnet masks
(VLSMs), and summarization. These characteristics allow BGP to work with
networks that are not organized on purely classful boundaries and to
create summaries of networks to reduce the size of the routing tables.

\begin{enumerate}
\tightlist
\item
  BGP uses a rich variety of metrics called \emph{attributes} to
  influence the selection of the best path to remote networks in the
  event that there are multiple advertised paths. Network administrators
  far removed from the initial origin of the advertised networks can
  manipulate these attributes. Paths can be chosen simply because a
  neighbor AS is preferred for political or economic reasons, thus
  overriding more traditional measures such as the distance to the
  advertised route.
\item
  BGP supports a nonhierarchical network structure and allows a complex
  combination of interconnections among neighbors. There is no
  counterpart in BGP to the OSPF concept of area 0, which is the single
  area through which interarea traffic passes. Traffic between different
  BGP autonomous systems may follow a variety of different paths.
\item
  BGP uses the concept of autonomous systems to define the boundaries of
  networks and treats communications among neighbors differently
  depending on whether the neighbors belong to the same autonomous
  system or not. An autonomous system is a collection of routers that
  are under a common administrative control and that present a common
  route policy to the outside world. It is not necessary that all of the
  routers run the same routing protocol, just that they all be
  controlled and coordinated by the same administrative authority.
\item
  An AS uses BGP to advertise routes that are in its network and need to
  be visible outside of the network; it also uses BGP to learn about the
  reachability of routes by listening to advertisement announcements
  from other autonomous systems. Each AS can have a specific policy
  regarding the routes it wishes to advertise externally. These policies
  can be different for every point in which the AS attaches to the
  outside world.
\item
  Inside autonomous networks, \emph{interior gateway protocols (IGPs)}
  are used to discover the connectivity among a set of IP subnets. IGPs
  are well-known protocols such as the Routing Information Protocol
  (RIP), Interior Gateway Routing Protocol (IGRP), Open Shortest Path
  First (OSPF), and Enhanced Interior Gateway Routing Protocol (EIGRP).
\end{enumerate}

BGP relies upon TCP for connection-oriented, acknowledged communications
using port 179. BGP routers are specifically configured as neighbors of
one another and use unicast packets to exchange route information,
keepalives, and a variety of other messages. BGP routers go through a
variety of stages as they establish communications with their configured
neighbors, verify consistent parameter configuration, and begin the
initial synchronization of their route information. After the initial
synchronization is complete, BGP neighbors exchange updates on a
triggered basis and monitor their connection state via periodic
keepalives.

BGP neighbors either live in the same AS, in which case they are
referred to as \emph{internal BGP (iBGP) neighbors}, or live in
different ASs, in which case they are referred to as \emph{external BGP}
\protect\hypertarget{c21.xhtmlux5cux23Page_929}{}{}\emph{(eBGP)
neighbors}. Internal BGP neighbors do not need to share a common network
and can be separated by many other routers that don't need to run BGP.
However, every iBGP router must be configured as a neighbor to every
other iBGP router in the same area. External BGP neighbors are normally
required to share a common network and are directly accessible to each
other. There is no requirement that every eBGP router be a neighbor to
every other eBGP router.

BGP can advertise networks that are learned dynamically, statically, or
through redistribution. The network command, which is used in most other
protocols to cause the router's interface(s) to begin listening for and
sending route updates, is used in BGP routers to advertise specific
networks. There are a number of rules, like the synchronization rule,
that govern BGP's interaction with interior gateway protocols and the
routes that are advertised as a result of this interaction.

Finally, BGP can implement a variety of mechanisms to improve
scalability. Summarization is certainly one of these mechanisms, as is
the use of route reflectors. Route reflectors provide a means to
eliminate the requirement for a full mesh of neighbor relationships
among iBGP neighbors, thus permitting larger BGP environments with less
traffic.

\subsubsection[Configuring and Verifying
EBGP]{\texorpdfstring{\protect\hypertarget{c21.xhtmlux5cux23c21-sec-33}{}{}Configuring
and Verifying EBGP}{Configuring and Verifying EBGP}}

If you're configuring BGP between a customer network and an ISP, this
process is called external BGP (EBGP). If you're configuring BGP peers
between two routers in the same AS, it's not considered EBGP.

You must have the basic information to configure EBGP:

\begin{enumerate}
\tightlist
\item
  AS numbers (your own, and all remote AS numbers, which must be
  different)
\item
  All the neighbors (peers) that are involved in BGP, and IP addressing
  that is used among the BGP neighbors
\item
  Networks that need to be advertised into BGP
\end{enumerate}

For an example of configuring EBGP, here's
\protect\hyperlink{c21.xhtmlux5cux23figure21-25}{Figure 21.25}.

\begin{figure}
\centering
\includegraphics{images/c21f025.jpg}
\caption{{\protect\hyperlink{c21.xhtmlux5cux23figureanchor21-25}{\textbf{FIGURE
21.25}} Example of EBGP lay layout}}
\end{figure}

\protect\hypertarget{c21.xhtmlux5cux23Page_930}{}{}There are three main
steps to configure basic BGP:

\begin{enumerate}
\tightlist
\item
  Define the BGP process.
\item
  Establish one or more neighbor relationships.
\item
  Advertise the local networks into BGP.
\end{enumerate}

\paragraph{Define the BGP Process}

To start the BGP process on a router, use the \texttt{router\ bgp\ AS}
command. Each process must be assigned a local AS number. There can only
be one BGP process in a router, which means that each router can only be
in one AS at any given time.

Here is an example:

\begin{verbatim}
ISP#config t
ISP(config)#router bgp ?
<1-65535> Autonomous system number
ISP(config)#router bgp 1
\end{verbatim}

Notice the AS number can be from 1 to 65,535.

\paragraph{Establish One or More Neighbor Relationships}

Since BGP does not automatically discover neighbors like other routing
protocols do, you have to explicitly configure them using the
\texttt{neighbor\ peer-ip-address\ remote-as\ peer-as-number} command.
Here is an example of configuring the ISP router in
\protect\hyperlink{c21.xhtmlux5cux23figure21-25}{Figure 21.25}:

\begin{verbatim}
ISP(config-router)#neighbor 192.168.1.2 remote-as 100
ISP(config-router)#neighbor 192.168.2.2 remote-as 200
\end{verbatim}

Be sure to understand that the above command is the neighbor's IP
address and neighbor's AS number.

\paragraph{Advertise the Local Networks Into BGP}

To specify your local networks and advertise them into BGP, you use the
\texttt{network} command with the \texttt{mask} keyword and then the
subnet mask:

\begin{verbatim}
ISP(config-router)#network 10.0.0.0 mask 255.255.255.0
\end{verbatim}

These network numbers must match what is found on the local router's
forwarding table exactly, which can be seen with the
\texttt{show\ ip\ route} or \texttt{show\ ip\ int\ brief} command. For
other routing protocols, the network command has a different meaning.
For OSPF and EIGRP, for example, the network command indicates the
interfaces for which the routing protocol will send and receive route
updates. In BGP, the network command indicates which routes should be
injected into the BGP table on the local router.

\protect\hyperlink{c21.xhtmlux5cux23figure21-25}{Figure 21.25} displays
the BGP routing configuration for the R1 and R2 routers:

\begin{verbatim}
R1#config t
R1(config)#router bgp 100
R1(config-router)#neighbor 192.168.1.1 remote-as 1
R1(config-router)#network 10.0.1.0 mask 255.255.255.0
\end{verbatim}

\begin{verbatim}
R2#config t
R2(config)#router bgp 200
R2(config-router)#neighbor 192.168.2.1 remote-as 1
R2(config-router)#network 10.0.2.0 mask 255.255.255.0
\end{verbatim}

That's it! Pretty simple. Now let's verify our configuration.

\subsubsection[Verifying
EBGP]{\texorpdfstring{\protect\hypertarget{c21.xhtmlux5cux23c21-sec-34}{}{}Verifying
EBGP}{Verifying EBGP}}

We'll use the following commands to verify our little EBGP network.

\begin{enumerate}
\tightlist
\item
  \texttt{show\ ip\ bgp\ summary}
\item
  \texttt{show\ ip\ bgp}
\item
  \texttt{show\ ip\ bgp\ neighbors}
\end{enumerate}

\paragraph{\texorpdfstring{The \texttt{show\ ip\ bgp\ summary}
Command}{The show ip bgp summary Command}}

The \texttt{show\ ip\ bgp\ summary} command gives you an overview of the
BGP status. Each configured neighbor is listed in the output of the
command. The output will display the IP address and AS number of the
neighbor, along with the status of the session. You can use this
information to verify that BGP sessions are up and established, or to
verify the IP address and AS number of the configured BGP neighbor.

\begin{verbatim}
ISP#sh ip bgp summary
BGP router identifier 10.0.0.1, local AS number 1
BGP table version is 4, main routing table version 6
3 network entries using 396 bytes of memory
3 path entries using 156 bytes of memory
2/2 BGP path/bestpath attribute entries using 368 bytes of memory
3 BGP AS-PATH entries using 72 bytes of memory
0 BGP route-map cache entries using 0 bytes of memory
0 BGP filter-list cache entries using 0 bytes of memory
Bitfield cache entries: current 1 (at peak 1) using 32 bytes of memory
BGP using 1024 total bytes of memory
BGP activity 3/0 prefixes, 3/0 paths, scan interval 60 secs
\end{verbatim}

\begin{verbatim}
Neighbor        V    AS MsgRcvd MsgSent   TblVer  InQ OutQ Up/Down State/PfxRcd
192.168.1.2     4   100      56      55        4    0    0 00:53:33        4
192.168.2.2     4   200      47      46        4    0    0 00:44:53        4
\end{verbatim}

\protect\hypertarget{c21.xhtmlux5cux23Page_932}{}{}The first section of
the \texttt{show\ ip\ bgp\ summary} command output describes the BGP
table and its content:

\begin{enumerate}
\tightlist
\item
  The router ID of the router and local AS number.
\item
  The BGP table version is the version number of the local BGP table.
  This number is increased every time the table is changed.
\end{enumerate}

The second section of the \texttt{show\ ip\ bgp\ summary} command output
is a table in which the current neighbor statuses are shown. Here's
information about what you see displayed in the output of this command:

\begin{enumerate}
\tightlist
\item
  IP address of the neighbor.
\item
  BGP version number that is used by the router when communicating with
  the neighbor (v4).
\item
  AS number of the remote neighbor.
\item
  Number of messages and updates that have been received from the
  neighbor since the session was established.
\item
  Number of messages and updates that have been sent to the neighbor
  since the session was established.
\item
  Version number of the local BGP table that has been included in the
  most recent update to the neighbor.
\item
  Number of messages that are waiting to be processed in the incoming
  queue from this neighbor.
\item
  Number of messages that are waiting in the outgoing queue for
  transmission to the neighbor.
\item
  How long the neighbor has been in the current state and the name of
  the current state. Interestingly, notice there is no state listed,
  which is actually what you want because that means the peers are
  established.
\item
  Number of received prefixes from the neighbor.
\item
  ISP1 has two established sessions with the following neighbors:

  \begin{enumerate}
  \tightlist
  \item
    192.168.1.2, which is the IP address of R1 router and is in AS 100
  \item
    192.168.2.2, which is the IP address of R2 router and is in AS 200
  \end{enumerate}
\item
  From each of the neighbors, ISP1 has received one prefix (one
  network).
\end{enumerate}

Now, for the CCNA objectives, remember that if you see this type of
output at the end of the \texttt{show\ ip\ bgp\ summary} command, that
the BGP session is not established between peers:

\begin{verbatim}
Neighbor       V    AS MsgRcvd MsgSent   TblVer  InQ OutQ Up/Down State/PfxRcd
192.168.1.2    4   64       0       0        0    0    0  never    Active
\end{verbatim}

Notice the state of Active. Remember, seeing no state output is good!
Active means we're actively trying to establish with the peer.

\protect\hypertarget{c21.xhtmlux5cux23Page_933}{}{}

\paragraph{\texorpdfstring{The \texttt{show\ ip\ bgp}
Command}{The show ip bgp Command}}

With the \texttt{show\ ip\ bgp} command, the entire BGP table is
displayed. A list of information about each route is displayed, so this
is a nice command to get quick information on your BGP routes.

\begin{verbatim}
ISP#sh ip bgp
BGP table version is 4, local router ID is 10.0.0.1
Status codes: s suppressed, d damped, h history, * valid, > best, i - internal,
r RIB-failure, S Stale
Origin codes: i - IGP, e - EGP, ? - incomplete
\end{verbatim}

\begin{verbatim}
Network Next Hop Metric LocPrf Weight Path
*> 10.0.0.0/24 0.0.0.0 0 0 32768 i
*> 10.0.1.0/24 192.168.1.2 0 0 0 100 i
*> 10.0.2.0/24 192.168.2.2 0 0 0 200 i
\end{verbatim}

The output is sorted in network number order, and if the BGP table
contains more than one route to the same network, the backup routes are
displayed on separate lines. We don't have multiple routes, so none are
shown.

The BGP path selection process selects one of the available routes to
each of the networks as the best. This route is pointed out by the
\textgreater{} character in the left column.

ISP1 has the following networks in the BGP table:

\begin{enumerate}
\tightlist
\item
  10.0.0.0/24, which is locally originated via the network command in
  BGP on the ISP router
\item
  10.0.1.0/24, which has been advertised from 192.168.1.2 (R1) neighbor
\item
  10.0.2.0/24, which has been advertised from 192.168.2.2 (R2) neighbor
\end{enumerate}

Since the command displays all routing information, note that network
10.0.0.0/24, with the next-hop attribute set to 0.0.0.0, is also
displayed. The next-hop attribute is set to 0.0.0.0 when you view the
BGP table on the router that originates the route in BGP. The
10.0.0.0/24 network is the network that I locally configured on ISP1
into BGP.

\paragraph{\texorpdfstring{The \texttt{show\ ip\ bgp\ neighbors}
Command}{The show ip bgp neighbors Command}}

The \texttt{show\ ip\ bgp\ neighbors} command provides more information
about BGP connections to neighbors than the \texttt{show\ ip\ bgp}
command does. This command can be used to get information about the TCP
sessions and the BGP parameters of the session, as well as the showing
the TCP timers and counters, and it's a long output! I'll just give you
the top part of the command here:

\begin{verbatim}
ISP#sh ip bgp neighbors
BGP neighbor is 192.168.1.2, remote AS 100, external link
BGP version 4, remote router ID 10.0.1.1
BGP state = Established, up for 00:10:55
Last read 00:10:55, last write 00:10:55,hold time is 180, keepalive interval is 60 seconds
Neighbor capabilities:
Route refresh: advertised and received(new)
Address family IPv4 Unicast: advertised and received
Message statistics:
InQ depth is 0
OutQ depth is 0
[output cut]
\end{verbatim}

Notice (and remember!) you can use the \texttt{show\ ip\ bgp\ neighbors}
command to see the hold time on two BGP peers, and in the above example
from the ISP to R1, the holdtime is 180 seconds.

\subsection[Summary]{\texorpdfstring{\protect\hypertarget{c21.xhtmlux5cux23c21-sec-35}{}{}Summary}{Summary}}

In this chapter, you learned the difference between the following WAN
services: cable, DSL, HDLC, PPP, and PPPoE. You also learned that you
can use a VPN once any of those services are up and running as well as
create and verify a tunnel interface.

It's so important for you to understand High-Level Data-Link Control
(HDLC) and how to verify with the \texttt{show\ interface} command that
HDLC is enabled! You've been provided with some really important HDLC
information as well as information on how the Point-to-Point Protocol
(PPP) is used if you need more features than HDLC offers or if you're
using two different brands of routers. You now know that this is because
various versions of HDLC are proprietary and won't work between two
different vendors' routers.

When we went through the section on PPP, I discussed the various LCP
options as well as the two types of authentication that can be used: PAP
and CHAP.

We then discussed virtual private networks, IPsec, and encryption, and I
explained GRE and how to configure the tunnel and then verify it.

We finished up the chapter with a discussion on BGP.

\subsection[Exam
Essentials]{\texorpdfstring{\protect\hypertarget{c21.xhtmlux5cux23c21-sec-36}{}{}Exam
Essentials}{Exam Essentials}}

\textbf{Remember the default serial encapsulation on Cisco routers.}
Cisco routers use a proprietary High-Level Data-Link Control (HDLC)
encapsulation on all their serial links by default.

\textbf{Remember the PPP Data Link layer protocols.} The three Data Link
layer protocols are Network Control Protocol (NCP), which defines the
Network layer protocols; Link Control Protocol (LCP), a method of
establishing, configuring, maintaining, and terminating the
point-to-point connection; and High-Level Data-Link Control (HDLC), the
MAC layer protocol that encapsulates the packets.

\textbf{Be able to troubleshoot a PPP link.} Understand that a PPP link
between two routers will show up and a ping would even work between the
router if the layer 3 addresses are wrong.

\protect\hypertarget{c21.xhtmlux5cux23Page_935}{}{}\textbf{Remember the
various types of serial WAN connections.} The serial WAN connections
that are most widely used are HDLC, PPP, and Frame Relay.

\textbf{Understand the term \emph{virtual private network}.} You need to
understand why and how to use a VPN between two sites and the purpose
that IPsec serves with VPNs.

\textbf{Understand how to configure and verify a GRE tunnel.} To
configure GRE, first configure the logical tunnel with the
\texttt{interface\ tunnel\ number} command. Configure the mode and
transport, if needed, with the \texttt{tunnel\ mode\ mode\ protocol}
command, then configure the IP addresses on the tunnel interfaces, the
tunnel source and tunnel destination addresses, and your physical
interfaces with global addresses. Verify with the
\texttt{show\ interface\ tunnel} command as well as the Ping protocol.

\subsection[Written Lab
21]{\texorpdfstring{\protect\hypertarget{c21.xhtmlux5cux23c21-sec-37}{}{}Written
Lab 21}{Written Lab 21}}

You can find the answers to this lab in Appendix A, ``Answers to Written
Labs.''

Write the answers to the following WAN questions:

\begin{enumerate}
\tightlist
\item
  True/False: The IWAN allows transport-independent connectivity.
\item
  True/False: BGP runs between two peers in the same autonomous system
  (AS). It is referred to as External BGP (EBGP).
\item
  TCP port 179 is used for which protocol?
\item
  Which command can you use to know the hold time on the two BGP peers?
\item
  Which command will not tell you if the GRE tunnel is in up/up state?
\item
  True/False: A GRE tunnel is considered secure.
\item
  What protocol would you use if you were running xDSL and needed
  authentication?
\item
  What are the three protocols specified in PPP?
\item
  List two technologies that are examples of layer 2 MPLS VPN
  technologies.
\item
  List two VPNs that are examples of VPNs managed by service providers.
\end{enumerate}

\subsection[Hands-on
Labs]{\texorpdfstring{\protect\hypertarget{c21.xhtmlux5cux23c21-sec-38}{}{}Hands-on
Labs}{Hands-on Labs}}

In this section, you will configure Cisco routers in three different WAN
labs using the figure supplied in each lab. (These labs are included for
use with real Cisco routers but work perfectly with the LammleSim IOS
version simulator and with Cisco's Packet Tracer program.)

\begin{enumerate}
\tightlist
\item
  Lab 21.1: Configuring PPP Encapsulation and Authentication
\item
  Lab 21.2: Configuring and Monitoring HDLC
\item
  Lab 21.3: Configuring a GRE Tunnel
\end{enumerate}

\protect\hypertarget{c21.xhtmlux5cux23Page_936}{}{}

\subsubsection[Hands-on Lab 21.1: Configuring PPP Encapsulation and
Authentication]{\texorpdfstring{\protect\hypertarget{c21.xhtmlux5cux23c21-sec-39}{}{}Hands-on
Lab 21.1: Configuring PPP Encapsulation and
Authentication}{Hands-on Lab 21.1: Configuring PPP Encapsulation and Authentication}}

By default, Cisco routers use High-Level Data-Link Control (HDLC) as a
point-to-point encapsulation method on serial links. If you are
connecting to non-Cisco equipment, then you can use the PPP
encapsulation method to communicate.

Labs 21.1 and 21.2 will have you configure the network in the following
diagram.

\begin{figure}
\centering
\includegraphics{images/c21uf001.jpg}
\caption{}
\end{figure}

\begin{enumerate}
\item
  Type \texttt{sh\ int\ s0/0} on RouterA and RouterB to see the
  encapsulation method.
\item
  Make sure each router has the hostname assigned.

\begin{verbatim}
RouterA#config t
RouterA(config)#hostname RouterA
\end{verbatim}

\begin{verbatim}
RouterB#config t
RouterB(config)#hostname RouterB
\end{verbatim}
\item
  To change the default HDLC encapsulation method to PPP on both
  routers, use the \texttt{encapsulation} command at interface
  configuration. Both ends of the link must run the same encapsulation
  method.

\begin{verbatim}
RouterA#Config t
RouterA(config)#int s0
RouterA(config-if)#encap ppp
\end{verbatim}
\item
  Now go to RouterB and set serial 0/0 to PPP encapsulation.

\begin{verbatim}
RouterB#config t
RouterB(config)#int s0
RouterB(config-if)#encap ppp
\end{verbatim}
\item
  Verify the configuration by typing \texttt{sh\ int\ s0/0} on both
  routers.
\item
  \protect\hypertarget{c21.xhtmlux5cux23Page_937}{}{}Notice the IPCP and
  CDPCP (assuming the interface is up). This is the information used to
  transmit the upper-layer (Network layer) information across the HDLC
  at the MAC sublayer.
\item
  Define a username and password on each router. Notice that the
  username is the name of the remote router. Also, the password must be
  the same.

\begin{verbatim}
RouterA#config t
RouterA(config)#username RouterB password todd
\end{verbatim}

\begin{verbatim}
RouterB#config t
RouterB(config)#username RouterA password todd
\end{verbatim}
\item
  Enable CHAP or PAP authentication on each interface.

\begin{verbatim}
RouterA(config)#int s0
RouterA(config-if)#ppp authentication chap
\end{verbatim}

\begin{verbatim}
RouterB(config)#int s0
RouterB(config-if)#ppp authentication chap
\end{verbatim}
\item
  Verify the PPP configuration on each router by using these commands.

\begin{verbatim}
RouterB(config-if)#shut
RouterB(config-if)#debug ppp authentication
RouterB(config-if)#no shut
\end{verbatim}
\end{enumerate}

\subsubsection[Hands-on Lab 21.2: Configuring and Monitoring
HDLC]{\texorpdfstring{\protect\hypertarget{c21.xhtmlux5cux23c21-sec-40}{}{}Hands-on
Lab 21.2: Configuring and Monitoring
HDLC}{Hands-on Lab 21.2: Configuring and Monitoring HDLC}}

There really is no configuration required for HDLC (as it is the default
configuration on Cisco serial interfaces), but if you completed Lab
21.1, then the PPP encapsulation would be set on both routers. This is
why I put the PPP lab first. This lab allows you to actually configure
HDLC encapsulation on a router.

\begin{center}\rule{0.5\linewidth}{0.5pt}\end{center}

\includegraphics{images/note.png}For this second lab, you will use the
same configuration you used for Lab 21.1.

\begin{center}\rule{0.5\linewidth}{0.5pt}\end{center}

\begin{enumerate}
\item
  Set the encapsulation for each serial interface by using the
  \texttt{encapsulation\ hdlc} command.

\begin{verbatim}
RouterA#config t
RouterA(config)#int s0
RouterA(config-if)#encapsulation hdlc
\end{verbatim}

\begin{verbatim}
RouterB#config t
RouterB(config)#int s0
RouterB(config-if)#encapsulation hdlc
\end{verbatim}
\item
  Verify the HDLC encapsulation by using the
  \texttt{show\ interface\ s0} command on each router.
\end{enumerate}

\subsubsection[Hands-on Lab 21.3: Configuring a GRE
Tunnel]{\texorpdfstring{\protect\hypertarget{c21.xhtmlux5cux23c21-sec-41}{}{}Hands-on
Lab 21.3: Configuring a GRE
Tunnel}{Hands-on Lab 21.3: Configuring a GRE Tunnel}}

In this lab you will configure two point-to-point routers with a simple
IP GRE tunnel. You can use a real router, LammleSim IOS version, or
Packet Tracer to do this lab.

\begin{enumerate}
\item
  First, configure the logical tunnel with the
  \texttt{interface\ tunnel\ number} command.

\begin{verbatim}
Corp(config)#int s0/0/0
Corp(config-if)#ip address 63.1.1.2 255.255.255.252
Corp(config)#int tunnel ?
  <0-2147483647>  Tunnel interface number
Corp(config)#int tunnel 0
*Jan  5 16:58:22.719: %LINEPROTO-5-UPDOWN: Line protocol
on Interface Tunnel0, changed state to down
\end{verbatim}
\item
  Once you have configured your interface and created the logical
  tunnel, you need to configure the mode and then the transport
  protocol.

\begin{verbatim}
Corp(config-if)#tunnel mode ?
  aurp    AURP TunnelTalk AppleTalk encapsulation
  cayman  Cayman TunnelTalk AppleTalk encapsulation
  dvmrp   DVMRP multicast tunnel
  eon     EON compatible CLNS tunnel
  gre     generic route encapsulation protocol
  ipip    IP over IP encapsulation
  ipsec   IPSec tunnel encapsulation
  iptalk  Apple IPTalk encapsulation
  ipv6    Generic packet tunneling in IPv6
  ipv6ip  IPv6 over IP encapsulation
  nos     IP over IP encapsulation (KA9Q/NOS compatible)
  rbscp   RBSCP in IP tunnel
Corp(config-if)#tunnel mode gre ?
  ip          over IP
  ipv6        over IPv6
  multipoint  over IP (multipoint)
\end{verbatim}

\begin{verbatim}
Corp(config-if)#tunnel mode gre ip
\end{verbatim}
\item
  \protect\hypertarget{c21.xhtmlux5cux23Page_939}{}{}Now that you have
  created the tunnel interface, the type, and the transport protocol,
  you need to configure your IP addresses. Of course, you need to use
  your actual interface IP for the tunnel, but you also need to
  configure the tunnel source and tunnel destination addresses.

\begin{verbatim}
Corp(config-if)#int t0
Corp(config-if)#ip address 192.168.10.1 255.255.255.0
Corp(config-if)#tunnel source 63.1.1.1
Corp(config-if)#tunnel destination 63.1.1.2
\end{verbatim}

\begin{verbatim}
Corp#sho run interface tunnel 0
Building configuration...
\end{verbatim}

\begin{verbatim}
Current configuration : 117 bytes
!
interface Tunnel0
 ip address 192.168.10.1 255.255.255.0
 tunnel source 63.1.1.1
 tunnel destination 63.1.1.2
end
\end{verbatim}
\item
  Now configure the other end of the serial link and watch the tunnel
  pop up!

\begin{verbatim}
SF(config)#int s0/0/0
SF(config-if)#ip address 63.1.1.2 255.255.255.252
SF(config-if)#int t0
SF(config-if)#ip address 192.168.10.2 255.255.255.0
SF(config-if)#tunnel source 63.1.1.2
SF(config-if)#tun destination 63.1.1.1
*May 19 22:46:37.099: %LINEPROTO-5-UPDOWN: Line protocol on Interface Tunnel0, changed state to up
\end{verbatim}

  Remember, you don't need to configure your tunnel mode and transport
  protocol because GRE and IP are the defaults. It's really important
  that you remember to configure the tunnel interface with the actual
  source and destination IP addresses to use or the tunnel won't come
  up. In my example, 63.1.1.2 was the source and 63.1.1.1 was the
  destination.
\item
  Verify with the following commands:

\begin{verbatim}
Corp#sh ip int brief
\end{verbatim}

  You should see that the tunnel interface is now showing as an
  interface on your router. The IP address of the tunnel interface and
  the physical and data link status shows as up/up.

\begin{verbatim}
Corp#sh int tun 0
\end{verbatim}

  \protect\hypertarget{c21.xhtmlux5cux23Page_940}{}{}The
  \texttt{show\ interfaces} command shows the configuration settings and
  the interface status as well as the IP address and tunnel source and
  destination address.

\begin{verbatim}
Corp#sh ip route
\end{verbatim}

  The tunnel0 interface shows up as a directly connected interface, and
  although it's a logical interface, the router treats it as a physical
  interface just like serial0/0 in the routing table.
\end{enumerate}

\protect\hypertarget{c21.xhtmlux5cux23Page_941}{}{}

\subsection[Review
Questions]{\texorpdfstring{\protect\hypertarget{c21.xhtmlux5cux23c21-sec-42}{}{}Review
Questions}{Review Questions}}

\begin{center}\rule{0.5\linewidth}{0.5pt}\end{center}

\includegraphics{images/note.png}The following questions are designed to
test your understanding of this chapter's material. For more information
on how to get additional questions, please see
\href{http://www.lammle.com/ccna}{www.lammle.com/ccna}.

\begin{center}\rule{0.5\linewidth}{0.5pt}\end{center}

You can find the answers to these questions in Appendix B, ``Answers to
Review Questions.''

\begin{enumerate}
\item
  Which command will display the CHAP authentication process as it
  occurs between two routers in the network?

  \begin{enumerate}
  \tightlist
  \item
    \texttt{show\ chap\ authentication}
  \item
    \texttt{show\ interface\ serial\ 0}
  \item
    \texttt{debug\ ppp\ authentication}
  \item
    \texttt{debug\ chap\ authentication}
  \end{enumerate}
\item
  Which of the following are true regarding the following command?
  (Choose two.)

  \texttt{R1(config-router)\#neighbor\ 10.10.200.1\ remote-as\ 6200}

  \begin{enumerate}
  \tightlist
  \item
    The local router R1 uses AS 6200.
  \item
    The remote router uses AS 6200.
  \item
    The local interface of R1 is 10.10.200.1.
  \item
    The neighbor IP address is 10.10.200.1.
  \item
    The neighbor's loopback interface is 10.10.200.1.
  \end{enumerate}
\item
  BGP uses which Transport layer protocol and port number?

  \begin{enumerate}
  \tightlist
  \item
    UDP/123
  \item
    TCP/123
  \item
    UDP/179
  \item
    TCP/179
  \item
    UDP/169
  \item
    TCP/169
  \end{enumerate}
\item
  Which command can you use to know the hold time on the two BGP peers?

  \begin{enumerate}
  \tightlist
  \item
    \texttt{show\ ip\ bgp}
  \item
    \texttt{show\ ip\ bgp\ summary}
  \item
    \texttt{show\ ip\ bgp\ all}
  \item
    \texttt{show\ ip\ bgp\ neighbor}
  \end{enumerate}
\item
  \protect\hypertarget{c21.xhtmlux5cux23Page_942}{}{}What does a next
  hop of 0.0.0.0 mean in the \texttt{show\ ip\ bgp} command output?

\begin{verbatim}
    Network          Next Hop            Metric LocPrf Weight Path
 *> 10.1.1.0/24      0.0.0.0                  0         32768 ?
 *> 10.13.13.0/24    0.0.0.0                  0         32768 ?
\end{verbatim}

  \begin{enumerate}
  \tightlist
  \item
    The router does not know the next hop.
  \item
    The network is locally originated via the network command in BGP.
  \item
    It is not a valid network.
  \item
    The next hop is not reachable.
  \end{enumerate}
\item
  Which two of the following are GRE characteristics? (Choose two.)

  \begin{enumerate}
  \tightlist
  \item
    GRE encapsulation uses a protocol-type field in the GRE header to
    support the encapsulation of any OSI layer 3 protocol.
  \item
    GRE itself is stateful. It includes flow-control mechanisms, by
    default.
  \item
    GRE includes strong security mechanisms to protect its payload.
  \item
    The GRE header, together with the tunneling IP header, creates at
    least 24 bytes of additional overhead for tunneled packets.
  \end{enumerate}
\item
  A GRE tunnel is flapping with the following error message:

\begin{verbatim}
07:11:49: %LINEPROTO-5-UPDOWN:
          Line protocol on Interface Tunnel0, changed state to up
07:11:55: %TUN-5-RECURDOWN:
          Tunnel0 temporarily disabled due to recursive routing
07:11:59: %LINEPROTO-5-UPDOWN:
          Line protocol on Interface Tunnel0, changed state to down
07:12:59: %LINEPROTO-5-UPDOWN:
\end{verbatim}

  What could be the reason for the tunnel flapping?

  \begin{enumerate}
  \tightlist
  \item
    IP routing has not been enabled on tunnel interface.
  \item
    There's an MTU issue on the tunnel interface.
  \item
    The router is trying to route to the tunnel destination address
    using the tunnel interface itself.
  \item
    An access list is blocking traffic on the tunnel interface.
  \end{enumerate}
\item
  Which of the following commands will not tell you if the GRE tunnel 0
  is in up/up state?

  \begin{enumerate}
  \tightlist
  \item
    \texttt{show\ ip\ interface\ brief}
  \item
    \texttt{show\ interface\ tunnel\ 0}
  \item
    \texttt{show\ ip\ interface\ tunnel\ 0}
  \item
    \texttt{show\ run\ interface\ tunnel\ 0}
  \end{enumerate}
\item
  \protect\hypertarget{c21.xhtmlux5cux23Page_943}{}{}Which of the
  following PPP authentication protocols authenticates a device on the
  other end of a link with an encrypted password?

  \begin{enumerate}
  \tightlist
  \item
    MD5
  \item
    PAP
  \item
    CHAP
  \item
    DES
  \end{enumerate}
\item
  Which of the following encapsulates PPP frames in Ethernet frames and
  uses common PPP features like authentication, encryption, and
  compression?

  \begin{enumerate}
  \tightlist
  \item
    PPP
  \item
    PPPoA
  \item
    PPPoE
  \item
    Token Ring
  \end{enumerate}
\item
  Shown is the output of a \texttt{show\ interfaces} command on an
  interface that is configured to use PPP. A ping of the IP address on
  the other end of the link fails. Which two of the following could be
  the reason for the problem? (Choose two.)

\begin{verbatim}
R1#show interfaces serial 0/0/1
Serial0/0/0 is up, line protocol is down
  Hardware is GT96K Serial
Internet address is 10.0.1.1/30
\end{verbatim}

  \begin{enumerate}
  \tightlist
  \item
    The CSU/DSU connected to the other router is not powered on.
  \item
    The IP address on the router at the other end of the link is not in
    subnet 192.168.2.0/24.
  \item
    CHAP authentication failed.
  \item
    The router on the other end of the link has been configured to use
    HDLC.
  \end{enumerate}
\item
  You have configured a serial interface with GRE IP commands on a
  corporate router with a point-to-point link to a remote office. What
  command will show you the IP addresses and tunnel source and
  destination addresses of the interfaces?

  \begin{enumerate}
  \tightlist
  \item
    \texttt{show\ int\ serial\ 0/0}
  \item
    \texttt{show\ ip\ int\ brief}
  \item
    \texttt{show\ interface\ tunnel\ 0}
  \item
    \texttt{show\ tunnel\ ip\ status}
  \item
    \texttt{debug\ ip\ interface\ tunnel}
  \end{enumerate}
\item
  Which of the following is true regarding WAN technologies? (Choose
  three.)

  \begin{enumerate}
  \tightlist
  \item
    You must use PPP on a link connecting two routers using a
    point-to-point lease line.
  \item
    You can use a T1 to connect a customer site to the ISP.
  \item
    You can use a T1 to connect a Frame Relay connection to the ISP.
  \item
    \protect\hypertarget{c21.xhtmlux5cux23Page_944}{}{}You can use
    Ethernet as a WAN service by using EoMPLS.
  \item
    When using an Ethernet WAN, you must configure the DLCI.
  \end{enumerate}
\item
  You want to allow remote users to send protected packets to the
  corporate site, but you don't want to install software on the remote
  client machines. What is the best solution that you could implement?

  \begin{enumerate}
  \tightlist
  \item
    GRE tunnel
  \item
    Web VPN
  \item
    VPN Anywhere
  \item
    IPsec
  \end{enumerate}
\item
  Why won't the serial link between the Corp router and the Remote
  router come up?

\begin{verbatim}
Corp#sh int s0/0
Serial0/0 is up, line protocol is down
  Hardware is PowerQUICC Serial
  Internet address is 10.0.1.1/24
  MTU 1500 bytes, BW 1544 Kbit, DLY 20000 usec,
     reliability 254/255, txload 1/255, rxload 1/255
  Encapsulation PPP, loopback not set
Remote#sh int s0/0
Serial0/0 is up, line protocol is down
  Hardware is PowerQUICC Serial
  Internet address is 10.0.1.2/24
  MTU 1500 bytes, BW 1544 Kbit, DLY 20000 usec,
     reliability 254/255, txload 1/255, rxload 1/255
  Encapsulation HDLC, loopback not set
\end{verbatim}

  \begin{enumerate}
  \tightlist
  \item
    The serial cable is faulty.
  \item
    The IP addresses are not in the same subnet.
  \item
    The subnet masks are not correct.
  \item
    The keepalive settings are not correct.
  \item
    The layer 2 frame types are not compatible.
  \end{enumerate}
\item
  Which of the following are benefits of using a VPN in your
  internetwork? (Choose three.)

  \begin{enumerate}
  \tightlist
  \item
    Security
  \item
    Private high-bandwidth links
  \item
    Cost savings
  \item
    Incompatibility with broadband technologies
  \item
    Scalability
  \end{enumerate}
\item
  \protect\hypertarget{c21.xhtmlux5cux23Page_945}{}{}Which two
  technologies are examples of layer 2 MPLS VPN technologies? (Choose
  two.)

  \begin{enumerate}
  \tightlist
  \item
    VPLS
  \item
    DMVPM
  \item
    GETVPN
  \item
    VPWS
  \end{enumerate}
\item
  Which of the following is an industry-wide standard suite of protocols
  and algorithms that allows for secure data transmission over an
  IP-based network that functions at the layer 3 (Network layer) of the
  OSI model?

  \begin{enumerate}
  \tightlist
  \item
    HDLC
  \item
    Cable
  \item
    VPN
  \item
    IPsec
  \item
    xDSL
  \end{enumerate}
\item
  Which of the following describes the creation of private networks
  across the Internet, enabling privacy and tunneling of non-TCP/IP
  protocols?

  \begin{enumerate}
  \tightlist
  \item
    HDLC
  \item
    Cable
  \item
    VPN
  \item
    IPsec
  \item
    xDSL
  \end{enumerate}
\item
  Which two VPNs are examples of service provider--managed VPNs? (Choose
  two.)

  \begin{enumerate}
  \tightlist
  \item
    Remote-access VPN
  \item
    Layer 2 MPLS VPN
  \item
    Layer 3 MPLS VPN
  \item
    DMVPN
  \end{enumerate}
\end{enumerate}

\protect\hypertarget{c22.xhtml}{}{}

\section[{Chapter 22}\\
{Evolution of Intelligent
Networks}]{\texorpdfstring{\protect\hypertarget{c22.xhtmlux5cux23c22}{}{}\protect\hypertarget{c22.xhtmlux5cux23Page_947}{}{}{Chapter
22}\\
{Evolution of Intelligent
Networks}}{Chapter 22 Evolution of Intelligent Networks}}

\subsection{THE FOLLOWING ICND2 EXAM TOPICS ARE COVERED IN THIS
CHAPTER:}

\begin{enumerate}
\tightlist
\item
  \includegraphics{images/tick.png} \textbf{1.6 Describe the benefits of
  switch stacking and chassis aggregation}
\item
  \includegraphics{images/tick.png} \textbf{4.2 Describe the effects of
  cloud resources on enterprise network architecture}

  \begin{enumerate}
  \tightlist
  \item
    \includegraphics{images/square1.png} 4.2.a Traffic path to internal
    and external cloud services
  \item
    \includegraphics{images/square1.png} 4.2.b Virtual services
  \item
    \includegraphics{images/square1.png} 4.2.c Basic virtual network
    infrastructure
  \end{enumerate}
\item
  \includegraphics{images/tick.png} \textbf{4.3 Describe basic QoS
  concepts}

  \begin{enumerate}
  \tightlist
  \item
    \includegraphics{images/square1.png} 4.3.a Marking
  \item
    \includegraphics{images/square1.png} 4.3.b Device trust
  \item
    \includegraphics{images/square1.png} 4.3.c Prioritization
  \item
    \includegraphics{images/square1.png} 4.3.c.
  \item
    \includegraphics{images/square1.png} (i) Voice 4.3.c.
  \item
    \includegraphics{images/square1.png} (ii) Video 4.3.c.
  \item
    \includegraphics{images/square1.png} (iii) Data
  \item
    \includegraphics{images/square1.png} 4.3.d Shaping
  \item
    \includegraphics{images/square1.png} 4.3.e Policing
  \item
    \includegraphics{images/square1.png} 4.3.f Congestion management
  \end{enumerate}
\item
  \includegraphics{images/tick.png} \textbf{4.5 Verify ACLs using the
  APIC-EM Path Trace ACL analysis tool}
\item
  \includegraphics{images/tick.png} \textbf{5.5 Describe network
  programmability in enterprise network architecture}

  \begin{enumerate}
  \tightlist
  \item
    \includegraphics{images/square1.png} 5.5.a Function of a controller
  \item
    \includegraphics{images/square1.png} 5.5.b Separation of control
    plane and data plane
  \item
    \includegraphics{images/square1.png} 5.5.c Northbound and southbound
    APIs
  \end{enumerate}
\end{enumerate}

\protect\hypertarget{c22.xhtmlux5cux23Page_948}{}{}\includegraphics{images/intro.png}
This all-new chapter is totally focused on the CCNA objectives for
intelligent networks. I'll start by covering switch stacking using
StackWise and then move on to discuss the important realm of cloud
computing and its effect on the enterprise network.

I'm going to stick really close to the objectives on the more difficult
subjects to help you tune in specifically to the content that's
important for the CCNA, including the following: Software Defined
Networking (SDN), application programming interfaces (APIs), Cisco's
Application Policy Infrastructure Controller Enterprise Module
(APIC-EM), Intelligent WAN, and finally, quality of service (QoS). While
it's good to understand the cloud and SDN because they're certainly
objectives for the CCNA, just know that they aren't as critical for the
objectives as the QoS section found in this chapter.

In this chapter, I really only have the space to introduce the concepts
of network programmability and SDN because the topic is simply too large
in scope. Plus, this chapter is already super challenging because it's a
foundational chapter containing no configurations. Just remember that
I'm going to spotlight the objectives in this chapter to make this
chapter and its vital content as potent but painless as possible! We'll
check off every exam objective by the time we're through.

\begin{center}\rule{0.5\linewidth}{0.5pt}\end{center}

\includegraphics{images/note.png}To find up-to-the-minute updates for
this chapter, please see
\href{http://www.lammle.com/ccna}{www.lammle.com/ccna} or the book's web
page at \href{http://www.sybex.com/go/ccna}{www.sybex.com/go/ccna}.

\begin{center}\rule{0.5\linewidth}{0.5pt}\end{center}

\subsection[Switch
Stacking]{\texorpdfstring{\protect\hypertarget{c22.xhtmlux5cux23c22-sec-1}{}{}Switch
Stacking}{Switch Stacking}}

It's hard to believe that Cisco is using switch stacking to start their
``Evolution of Intelligent Networks'' objectives because switch stacking
has been around since the word \emph{cloud} meant 420 in my home town of
Boulder, but I digress.

A typical access closet contains access switches placed next to each
other in the same rack and uses high-speed redundant links with copper,
or more typically fiber, to the distribution layer switches.

Here are three big drawbacks to a typical switch topology:

\begin{enumerate}
\tightlist
\item
  Management overhead is high.
\item
  STP will block half of the uplinks.
\item
  There is no direct communication between switches.
\end{enumerate}

Cisco StackWise technology connects switches that are mounted in the
same rack together so they basically become one larger switch. By doing
this, you can incrementally
\protect\hypertarget{c22.xhtmlux5cux23Page_949}{}{}add more access ports
for each closet while avoiding the cost of upgrading to a bigger switch.
So you're adding ports as you grow your company instead of front loading
the investment into a pricier, larger switch all at once. And since
these stacks are managed as a single unit, it reduces the management in
your network.

All switches in a stack share configuration and routing information, so
you can easily add or remove switches at any time without disrupting
your network or affecting its performance.

\protect\hyperlink{c22.xhtmlux5cux23figure22-1}{Figure 22.1} shows a
typical access layer StackWise unit.

\begin{figure}
\centering
\includegraphics{images/c22f001.jpg}
\caption{{\protect\hyperlink{c22.xhtmlux5cux23figureanchor22-1}{\textbf{FIGURE
22.1}} Switch stacking}}
\end{figure}

To create a StackWise unit, you combine individual switches into a
single, logical unit using special stack interconnect cables as shown in
\protect\hyperlink{c22.xhtmlux5cux23figure22-1}{Figure 22.1}. This
creates a bidirectional, closed-loop path in the stack.

Here are some other features of StackWise:

\begin{enumerate}
\tightlist
\item
  Any changes to the network topology or routing information are updated
  continuously through the stack interconnect.
\item
  A master switch manages the stack as a single unit. The master switch
  is elected from one of the stack member switches.
\item
  You can join up to nine separate switches in a stack.
\item
  Each stack of switches has only a single IP address, and the stack is
  managed as a single object. You'll use this single IP address for
  managing the stack, including fault detection, VLAN database updates,
  security, and QoS controls. Each stack has only one configuration
  file, which is distributed to each switch in the StackWise.
\item
  Using Cisco StackWise will produce some management overhead, but at
  the same time, multiple switches in a stack can create an EtherChannel
  connection, eliminating the need for STP.
\end{enumerate}

These are the benefits to using StackWise technology, specifically
mapped to the CCNA objectives to memorize:

\begin{enumerate}
\tightlist
\item
  StackWise provides a method to join multiple physical switches into a
  single logical switching unit.
\item
  Switches are united by special interconnect cables.
\item
  \protect\hypertarget{c22.xhtmlux5cux23Page_950}{}{}The master switch
  is elected.
\item
  The stack is managed as a single object and has a single management IP
  address.
\item
  Management overhead is reduced.
\item
  STP is no longer needed if you use EtherChannel.
\item
  Up to nine switches can be in a StackWise unit.
\end{enumerate}

One more very cool thing\ldots When you add a new switch to the stack,
the master switch automatically configures the unit with the currently
running IOS image as well as the configuration of the stack. So you
don't have to do anything to bring up the switch before its ready to
operate\ldots nice!

\subsection[Cloud Computing and Its Effect on the Enterprise
Network]{\texorpdfstring{\protect\hypertarget{c22.xhtmlux5cux23c22-sec-2}{}{}Cloud
Computing and Its Effect on the Enterprise
Network}{Cloud Computing and Its Effect on the Enterprise Network}}

Cloud computing is by far one of the hottest topics in today's IT world.
Basically, cloud computing can provide virtualized processing, storage,
and computing resources to users remotely, making the resources
transparently available regardless of the user connection. To put it
simply, some people just refer to the cloud as ``someone else's hard
drive.'' This is true, of course, but the cloud is much more than just
storage.

The history of the consolidation and virtualization of our servers tells
us that this has become the de facto way of implementing servers because
of basic resource efficiency. Two physical servers will use twice the
amount of electricity as one server, but through virtualization, one
physical server can host two virtual machines, hence the main thrust
toward virtualization. With it, network components can simply be shared
more efficiently.

Users connecting to a cloud provider's network, whether it be for
storage or applications, really don't care about the underlying
infrastructure because as computing becomes a service rather than a
product, it's then considered an on-demand resource, described in
\protect\hyperlink{c22.xhtmlux5cux23figure22-2}{Figure 22.2}.

\begin{figure}
\centering
\includegraphics{images/c22f002.jpg}
\caption{{\protect\hyperlink{c22.xhtmlux5cux23figureanchor22-2}{\textbf{FIGURE
22.2}} Cloud computing is on-demand.}}
\end{figure}

\protect\hypertarget{c22.xhtmlux5cux23Page_951}{}{}Centralization/consolidation
of resources, automation of services, virtualization, and
standardization are just a few of the big benefits cloud services offer.
Let's take a look in
\protect\hyperlink{c22.xhtmlux5cux23figure22-3}{Figure 22.3}.

\begin{figure}
\centering
\includegraphics{images/c22f003.jpg}
\caption{{\protect\hyperlink{c22.xhtmlux5cux23figureanchor22-3}{\textbf{FIGURE
22.3}} Advantages of cloud computing}}
\end{figure}

Cloud computing has several advantages over the traditional use of
computer resources. Following are advantages to the provider and to the
cloud user.

Here are the advantages to a cloud service builder or provider:

\begin{enumerate}
\tightlist
\item
  Cost reduction, standardization, and automation
\item
  High utilization through virtualized, shared resources
\item
  Easier administration
\item
  Fall-in-place operations model
\end{enumerate}

Here are the advantages to cloud users:

\begin{enumerate}
\tightlist
\item
  On-demand, self-service resource provisioning
\item
  Fast deployment cycles
\item
  Cost effective
\item
  Centralized appearance of resources
\item
  Highly available, horizontally scaled application architectures
\item
  No local backups
\end{enumerate}

Having centralized resources is critical for today's workforce. For
example, if you have your documents stored locally on your laptop and
your laptop gets stolen, you're pretty much screwed unless you're doing
constant local backups. That is so 2005!

After I lost my laptop and all the files for the book I was writing at
the time, I swore (yes, I did that too) to never have my files stored
locally again. I started using only Google
\protect\hypertarget{c22.xhtmlux5cux23Page_952}{}{}Drive, OneDrive, and
Dropbox for all my files, and they became my best backup friends. If I
lose my laptop now, I just need to log in from any computer from
anywhere to my service provider's logical drives and presto, I have all
my files again. This is clearly a simple example of using cloud
computing, specifically SaaS (which is discussed next), and it's
wonderful!

So cloud computing provides for the sharing of resources, lower cost
operations passed to the cloud consumer, computing scaling, and the
ability to dynamically add new servers without going through the
procurement and deployment process.

\subsubsection[Service
Models]{\texorpdfstring{\protect\hypertarget{c22.xhtmlux5cux23c22-sec-3}{}{}Service
Models}{Service Models}}

Cloud providers can offer you different available resources based on
your needs and budget. You can choose just a vitalized network platform
or go all in with the network, OS, and application resources.

\protect\hyperlink{c22.xhtmlux5cux23figure22-4}{Figure 22.4} shows the
three service models available depending on the type of service you
choose to get from a cloud.

\begin{figure}
\centering
\includegraphics{images/c22f004.jpg}
\caption{{\protect\hyperlink{c22.xhtmlux5cux23figureanchor22-4}{\textbf{FIGURE
22.4}} Cloud computing service}}
\end{figure}

You can see that IaaS allows the customer to manage most of the network,
whereas SaaS doesn't allow any management by the customer, and PaaS is
somewhere in the middle of the two. Clearly, choices can be cost driven,
so the most important thing is that the customer pays only for the
services or infrastructure they use.

Let's take a look at each service:

\textbf{Infrastructure as a Service (IaaS): Provides only the network.}
Delivers computer infrastructure---a platform virtualization
environment---where the customer has the most control and management
capability.

\protect\hypertarget{c22.xhtmlux5cux23Page_953}{}{}\textbf{Platform as a
Service (PaaS): Provides the operating system and the network.} Delivers
a computing platform and solution stack, allowing customers to develop,
run, and manage applications without the complexity of building and
maintaining the infrastructure typically associated with developing and
launching an application. An example is Windows Azure.

\textbf{Software as a Service (SaaS): Provides the required software,
operating system, and network.} SaaS is common application software such
as databases, web servers, and email software that's hosted by the SaaS
vendor. The customer accesses this software over the Internet. Instead
of having users install software on their computers or servers, the SaaS
vendor owns the software and runs it on computers in its data center.
Microsoft Office 365 and many Amazon Web Services (AWS) offerings are
perfect examples of SaaS.

So depending on your business requirements and budget, cloud service
providers market a very broad offering of cloud computing products from
highly specialized offerings to a large selection of services.

What's nice here is that you're is offered a fixed price for each
service that you use, which allows you to easily budget wisely for the
future. It's true---at first, you'll have to spend a little cash on
staff training, but with automation you can do more with less staff
because administration will be easier and less complex. All of this
works to free up the company resources to work on new business
requirements and be more agile and innovative in the long run.

\subsection[Overview of Network Programmability in Enterprise
Network]{\texorpdfstring{\protect\hypertarget{c22.xhtmlux5cux23c22-sec-4}{}{}Overview
of Network Programmability in Enterprise
Network}{Overview of Network Programmability in Enterprise Network}}

Right now in our current, traditional networks, our router and switch
ports are the only devices that are not virtualized. So this is what
we're really trying to do here---virtualize our physical ports.

First, understand that our current routers and switches run an operating
system, such as Cisco IOS, that provides network functionality. This has
worked well for us for 25 years or so, but it is way too cumbersome now
to configure, implement, and troubleshoot these autonomous devices in
today's large, complicated networks. Before you even get started, you
have to understand the business requirements and then push that out to
all the devices. This can take weeks or even months since each device is
configured, maintained, and monitored separately.

Before we can talk about the new way to network our ports, you need to
understand how our current networks forward data, which happens via
these two planes:

\textbf{Data plane} This plane, also referred to as the forwarding
plane, is physically responsible for forwarding frames of packets from
its ingress to egress interfaces using protocols managed in the control
plane. Here, data is received, the destination interface is looked up,
and the forwarding of frames and packets happens, so the data plane
relies completely on the control plane to provide solid information.

\protect\hypertarget{c22.xhtmlux5cux23Page_954}{}{}\textbf{Control
plane} This plane is responsible for managing and controlling any
forwarding table that the data plane uses. For example, routing
protocols such as OSPF, EIGRP, RIP, and BGP as well as IPv4 ARP, IPv6
NDP, switch MAC address learning, and STP are all managed by the control
plane.

Now that you understand that there are two planes used to forward
traffic in our current or legacy network, let's take a look at the
future of networking.

\subsection[Application Programming Interfaces
(APIs)]{\texorpdfstring{\protect\hypertarget{c22.xhtmlux5cux23c22-sec-5}{}{}Application
Programming Interfaces
(APIs)}{Application Programming Interfaces (APIs)}}

If you have worked on any enterprise Wi-Fi installations in the last
decade, you would have designed your physical network and then
configured a type of network controller that managed all the wireless
APs in the network. It's hard to imagine that anyone would install a
wireless network today without some type of controller in an enterprise
network, where the access points (APs) receive their directions from the
controller on how to manage the wireless frames and the APs have no
operating system or brains to make many decisions on their own.

The same is now true for our physical router and switch ports, and it's
precisely this centralized management of network frames and packets that
Software Defined Networking (SDN) provides to us.

SDN removes the control plane intelligence from the network devices by
having a central controller manage the network instead of having a full
operating system (Cisco IOS, for example) on the devices. In turn, the
controller manages the network by separating the control and data
(forwarding) planes, which automates configuration and the remediation
of all devices.

So instead of the network devices each having individual control planes,
we now have a centralized control plane, which consolidates all network
operations in the SDN controller. APIs allow for applications to control
and configure the network without human intervention. The APIs are
another type of configuration interface just like the CLI, SNMP, or GUI
interfaces, which facilitate machine-to-machine operations.

The SDN architecture slightly differs from the architecture of
traditional networks by adding a third layer, the application plane, as
described here and shown in
\protect\hyperlink{c22.xhtmlux5cux23figure22-5}{Figure 22.5}:

\textbf{Data (or forwarding) plane} Contains network elements, meaning
any physical or virtual device that deals with data traffic.

\textbf{Control plane} Usually a software solution, the SDN controllers
reside here to provide centralized control of the router and switches
that populate the data plane, removing the control plane from individual
devices.

\textbf{Application plane} This new layer contains the applications that
communicate their network requirements toward the controller using APIs.

\protect\hypertarget{c22.xhtmlux5cux23Page_955}{}{}

\begin{figure}
\centering
\includegraphics{images/c22f005.jpg}
\caption{{\protect\hyperlink{c22.xhtmlux5cux23figureanchor22-5}{\textbf{FIGURE
22.5}} The SDN architecture}}
\end{figure}

SDN is pretty cool because your applications tell the network what to do
based on business needs instead of you having to do it. Then the
controller uses the APIs to pass instructions on to your routers,
switches, or other network gear. So instead of taking weeks or months to
push out a business requirement, the solution now only takes minutes.

There are two sets of APIs that SDN uses and they are very different. As
you already know, the SDN controller uses APIs to communicate with both
the application and data plane. Communication with the data plane is
defined with southbound interfaces, while services are offered to the
application plane using the northbound interface. Let's take a deeper
look at this oh-so-vital CCNA objective.

\subsubsection[Southbound
APIs]{\texorpdfstring{\protect\hypertarget{c22.xhtmlux5cux23c22-sec-6}{}{}Southbound
APIs}{Southbound APIs}}

Logical southbound interface (SBI) APIs (or device-to-control-plane
interfaces) are used for communication between the controllers and
network devices. They allow the two devices to communicate so that the
controller can program the data plane forwarding tables of your routers
and switches. SBIs are pictured in
\protect\hyperlink{c22.xhtmlux5cux23figure22-6}{Figure 22.6}.

\begin{figure}
\centering
\includegraphics{images/c22f006.jpg}
\caption{{\protect\hyperlink{c22.xhtmlux5cux23figureanchor22-6}{\textbf{FIGURE
22.6}} Southbound interfaces}}
\end{figure}

Since all the network drawings had the network gear below the
controller, the APIs that talked to the devices became known as
southbound, meaning, ``out the southbound interface of the controller.''
And don't forget that with SDN, the term \emph{interface} is no longer
referring to a physical interface!

\protect\hypertarget{c22.xhtmlux5cux23Page_956}{}{}Unlike northbound
APIs, southbound APIs have many standards, and you absolutely must know
them well for the objectives. Let's talk about them now:

\textbf{OpenFlow} Describes an industry-standard API, which the ONF
(\texttt{opennetworking.org}) defines. It configures white label
switches, meaning that they are nonproprietary, and as a result defines
the flow path through the network. All the configuration is done through
NETCONF.

\textbf{NETCONF} Although not all devices support NETCONF yet, what this
provides is a network management protocol standardized by the IETF.
Using RPC, you can install, manipulate, and delete the configuration of
network devices using XML.

\begin{center}\rule{0.5\linewidth}{0.5pt}\end{center}

\includegraphics{images/note.png}NETCONF is a protocol that allows you
to modify the configuration of a networking device, but if you want to
modify the device's forwarding table, then the OpenFlow protocol is the
way to go.

\begin{center}\rule{0.5\linewidth}{0.5pt}\end{center}

\textbf{onePK} A Cisco proprietary SBI that allows you to inspect or
modify the network element configuration without hardware upgrades. This
makes life easier for developers by providing software development kits
for Java, C, and Python.

\textbf{OpFlex} The name of the southbound API in the Cisco ACI world is
OpFlex, an open-standard, distributed control system. Understand that
OpenFlow first sends detailed and complex instructions to the control
plane of the network elements in order to implement a new application
policy---something called an imperative SDN model. On the other hand,
OpFlex uses a declarative SDN model because the controller, which Cisco
calls the APIC, sends a more abstract, ``summary policy'' to the network
elements. The summary policy makes the controller believe that the
network elements will implement the required changes using their own
control planes, since the devices will use a partially centralized
control plane.

\subsubsection[Northbound
APIs]{\texorpdfstring{\protect\hypertarget{c22.xhtmlux5cux23c22-sec-7}{}{}Northbound
APIs}{Northbound APIs}}

To communicate from the SDN controller and the applications running over
the network, you'll use northbound interfaces (NBIs), pictured in
\protect\hyperlink{c22.xhtmlux5cux23figure22-7}{Figure 22.7}.

\begin{figure}
\centering
\includegraphics{images/c22f007.jpg}
\caption{{\protect\hyperlink{c22.xhtmlux5cux23figureanchor22-7}{\textbf{FIGURE
22.7}} Northbound interfaces}}
\end{figure}

By setting up a framework that allows the application to demand the
network setup with the configuration that it needs, the NBIs allow your
applications to manage and control the
\protect\hypertarget{c22.xhtmlux5cux23Page_957}{}{}network. This is
priceless for saving time because you no longer need to adjust and tweak
your network to get a service or application running correctly.

The NBI applications include a wide variety of automated network
services, from network virtualization and dynamic virtual network
provisioning to more granular firewall monitoring, user identity
management, and access policy control. This allows for cloud
orchestration applications that tie together, for server provisioning,
storage, and networking that enables a complete rollout of new cloud
services in minutes instead of weeks!

Sadly, at this writing there is no single northbound interface that you
can use for communication between the controller and all applications.
So instead, you use various and sundry northbound APIs, with each one
working only with a specific set of applications.

Most of the time, applications used by NBIs will be on the same system
as the APIC controller, so the APIs don't need to send messages over the
network since both programs run on the same system. However, if they
don't reside on the same system, REST (Representational State Transfer)
comes into play; it uses HTTP messages to transfer data over the API for
applications that sit on different hosts.

\subsection[Cisco
APIC-EM]{\texorpdfstring{\protect\hypertarget{c22.xhtmlux5cux23c22-sec-8}{}{}Cisco
APIC-EM}{Cisco APIC-EM}}

Cisco Application Policy Infrastructure Controller Enterprise Module
(APIC-EM) is a Cisco SDN controller, which uses the previously mentioned
open APIs for policy-based management and security through a single
controller, abstracting the network and making network services simpler.
APIC-EM provides centralized automation of policy-based application
profiles, and the APIC-EM northbound interface is the only API that
you'll need to control your network programmatically. Through this
programmability, automated network control helps IT to respond quickly
to new business opportunities. The APIC-EM also includes support for
greenfield (new installations) and brownfield (current or old
installations) deployments, which allows you to implement
programmability and automation with the infrastructure that you already
have.

APIC-EM is pretty cool, easy to use (that's up for debate), and
automates the tasks that network engineers have been doing for well over
20 years. At first glance this seems like it will replace our current
jobs, and in some circumstances, people resistant to change will
certainly be replaced. But you don't have to be one of them if you start
planning now. \protect\hyperlink{c22.xhtmlux5cux23figure22-8}{Figure
22.8} demonstrates exactly where the APCI-EM controller sits in the SDN
stack.

\begin{figure}
\centering
\includegraphics{images/c22f008.jpg}
\caption{{\protect\hyperlink{c22.xhtmlux5cux23figureanchor22-8}{\textbf{FIGURE
22.8}} Where APIC-EM fits in the SDN stack}}
\end{figure}

\protect\hypertarget{c22.xhtmlux5cux23Page_958}{}{}Cisco APIC-EM's
northbound interface is only an API, but southbound interfaces are
implemented with something called a service abstraction layer (SAL),
which talks to the network elements via SNMP and CLI. Using the SNMP and
the CLI allows APIC-EM to work with legacy Cisco products, and soon
APIC-EM will be able to use NETCONF too.

The network devices can be either physical or virtual, including Nexus
data center switches, ASA firewalls, ASR routers, or even third-party
load balancers. The managed devices must be specific to ACI; in other
words, a special NX-OS or ASR IOS version is required to add the
southbound APIs required to communicate with the APIC controller.

The APIC-EM API is REST based, which as you know allows you to discover
and control your network using HTTP by using the GET, POST, PUT, and
DELETE options along with JSON (JavaScript Object Notation) and XML
(eXtensible Markup Language) syntax.

Here are some important features of Cisco APIC-EM that are covered in
the CCNA objectives, and shown in
\protect\hyperlink{c22.xhtmlux5cux23figure22-9}{Figure 22.9}:

\begin{figure}
\centering
\includegraphics{images/c22f009.jpg}
\caption{{\protect\hyperlink{c22.xhtmlux5cux23figureanchor22-9}{\textbf{FIGURE
22.9}} APIC-Enterprise Module}}
\end{figure}

On the left of the screen you can see the Discover button. This provides
the network information database, which scans the network and provides
the inventory, including all network devices; the network devices are
also shown in the Device Inventory.

\protect\hypertarget{c22.xhtmlux5cux23Page_959}{}{}There's also a
network topology visualization, which reveals the physical topology
discovered, as shown in the layout on
\protect\hyperlink{c22.xhtmlux5cux23figure22-9}{Figure 22.9}. This
auto-discovers and maps network devices to a physical topology with
detailed device-level data, including the discovered hosts. And take
note of that IWAN button. It provides the provisioning of IWAN network
profiles with simple business policies. Plus there is a Path Trace
button, which I will talk about next.

But wait\ldots{} Before I move on to the Path Trace functionality of
APIC-EM, let me go over a few more promising features with you. In the
APIC-EM is the Zero-Touch Deployment feature, which finds a new device
via the controller scanner and automatically configures it. You can
track user identities and endpoints by exchanging the information with
the Cisco Identity Service Engine (Cisco ISE) via the Identity Manager.
You can also quickly set and enforce QoS priority policies with the QoS
deployment and change management feature and accelerate ACL management
by querying and analyzing ACLs on each network device. This means you
can quickly identify ACL misconfiguration using the ACL analysis! And
last but not least, using the Policy Manager, the controller translates
a business policy into a network-device-level policy. The Policy Manager
can enforce the policy for a specific user at various times of the day,
across wired and wireless networks.

Now let's take a look at the very vital path tracing feature of APIC-EM.

\subsubsection[Using APIC-EM for Path
Tracing]{\texorpdfstring{\protect\hypertarget{c22.xhtmlux5cux23c22-sec-9}{}{}Using
APIC-EM for Path Tracing}{Using APIC-EM for Path Tracing}}

An important objective in this intelligent networks chapter is the path
tracing ability in the APIC-EM. Matter of fact, I've mostly been using
the APIC-EM for just this feature in order to help me troubleshoot my
virtual networks, and it does it very well. And it looks cool too!

Pushing toward staying tight to the CCNA objectives here, you really
want to know that you can use the path trace service of APIC-EM for ACL
analysis. Doing this allows you to examine the path that a specific type
of packet travels as it makes its way across the network from a source
to a destination node, and it can use IP and TCP/UDP ports when
diagnosing an application issue. If there is an ACL on an interface that
blocks a certain application, you'll see this in a GUI output. I cannot
tell you how extremely helpful this is in the day-to-day administration
of your networks!

The result of a path trace will be a GUI representation of the path that
a packet takes across all the devices and links between the source and
destination, and you can choose to provide the output of a reverse path
as well. Although I could easily do this same work manually, it would
certainly be a whole lot more time consuming if I did! APIC-EM's Path
Trace app actually does the work for you with just a few clicks at the
user interface.

Once you fill in the fields for the source, destination, and optionally,
the application, the path trace is initiated. You'll then see the path
between hosts, plus the list of each device along the path, illustrated
in \protect\hyperlink{c22.xhtmlux5cux23figure22-10}{Figure 22.10}.

\protect\hypertarget{c22.xhtmlux5cux23Page_960}{}{}

\begin{figure}
\centering
\includegraphics{images/c22f010.jpg}
\caption{{\protect\hyperlink{c22.xhtmlux5cux23figureanchor22-10}{\textbf{FIGURE
22.10}} APIC-Enterprise Module path trace sample}}
\end{figure}

Okay, here you can see the complete path taken from host A to host B. I
chose to view the reverse path as well. In this particular case, we
weren't being blocked by an ACL, but if a packet actually was being
blocked for a certain application, we'd see the exact interface where
the application was blocked and why. Here is more detail on how my trace
occurred.

First, the APIC-EM Discovery finds the network topology. At this point I
can now choose the source and destination address of a packet and,
alternately, port numbers and the application that can be used. The MAC
address table, IP routing tables, and so on are used by the APIC-EM to
find the path of the packet through the network. Finally, the GUI will
show you the path, complete with device and interface information.

Last point: The APIC-EM is free, and most of the applications off the
NBI are built in and included, but there are some solution applications
that need a license. So if you have a VM with at least 64 gigs of RAM,
you're set!

\subsection[Cisco Intelligent
WAN]{\texorpdfstring{\protect\hypertarget{c22.xhtmlux5cux23c22-sec-10}{}{}Cisco
Intelligent WAN}{Cisco Intelligent WAN}}

This topic was covered in the chapter ``Wide Area Networks,'' but it's
important to at least touch on it here since it's included in the
``Evolution of Intelligent Networks'' CCNA objectives. We'll also take a
peek at the APIC-EM for IWAN in this section. What Cisco's IWAN solution
provides is a way to take advantage of cheaper bandwidth at remote
locations, without compromising application performance, availability,
or even security and in an easy-to-configure manner---nice!

Clearly, this allows us to use low-cost or inexpensive Internet
connections, which have become more reliable and cost effective compared
to the dedicated WAN links we used to use, and it means that we can now
take advantage of this low-cost technology with Cisco's new Cisco
Intelligent WAN (Cisco IWAN). Add in the failover and redundancy
features that IWAN provides and you'll definitely see the reason large
enterprises are deploying
\protect\hypertarget{c22.xhtmlux5cux23Page_961}{}{}Cisco's IWAN. The
downside? Well, nothing to you and me, because we always use Cisco's
gear from end to end in all our networks, right?

IWAN can provide great long-distance connections for your organization
by dynamically routing based on the application service-level agreement
(SLA) while paying attention to network conditions. And it can do this
over any type of network!

\protect\hyperlink{c22.xhtmlux5cux23figure22-11}{Figure 22.11} pictures
the IWAN Aggregation Site tab on the IWAN screen.

A feature of the APCI-EM is the IWAN discovery and configuration.

\begin{figure}
\centering
\includegraphics{images/c22f011.jpg}
\caption{{\protect\hyperlink{c22.xhtmlux5cux23figureanchor22-11}{\textbf{FIGURE
22.11}} APIC-Enterprise Module IWAN}}
\end{figure}

There are four components of Cisco's IWAN:

\textbf{Transport-independent connectivity} Cisco IWAN provides a type
of advanced VPN connection across all available routes to remote
locations, providing one network with a single routing domain. This
allows you to easily multihome the network across different types of
connections, including MPLS, broadband, and cellular.

\textbf{Intelligent path control} By using Cisco Performance Routing
(Cisco PfR), Cisco IWAN improves delivery and WAN efficiency of
applications.

\textbf{Application optimization} Via Cisco's Application Visibility and
Control (Cisco AVC), as well as Cisco's Wide Area Application Services
(Cisco WAAS), you can now optimize application performance over WAN
links.

\textbf{Highly secure connectivity} Using VPNs, firewalls, network
segmentation, and security features, Cisco IWAN helps ensure that these
solutions actually provide the security you need over the public
Internet.

\protect\hypertarget{c22.xhtmlux5cux23Page_962}{}{}

\subsection[Quality of
Service]{\texorpdfstring{\protect\hypertarget{c22.xhtmlux5cux23c22-sec-11}{}{}Quality
of Service}{Quality of Service}}

\emph{Quality of service (QoS)} refers to the way the resources are
controlled so that the quality of services is maintained. It's basically
the ability to provide a different priority to one or more types of
traffic over other levels for different applications, data flows, or
users so that they can be guaranteed a certain performance level. QoS is
used to manage contention for network resources for better end-user
experience.

QoS methods focus on problems that can affect data as it traverses
network cable:

\textbf{Delay} Data can run into congested lines or take a
less-than-ideal route to the destination, and delays like these can make
some applications, such as VoIP, fail. This is the best reason to
implement QoS when real-time applications are in use in the network---to
prioritize delay-sensitive traffic.

\textbf{Dropped Packets} Some routers will drop packets if they receive
a packet while their buffers are full. If the receiving application is
waiting for the packets but doesn't get them, it will usually request
that the packets be retransmitted---another common cause of a service(s)
delay. With QoS, when there is contention on a link, less important
traffic is delayed or dropped in favor of delay-sensitive
business-important traffic.

\textbf{Error} Packets can be corrupted in transit and arrive at the
destination in an unacceptable format, again requiring retransmission
and resulting in delays such as video and voice.

\textbf{Jitter} Not every packet takes the same route to the
destination, so some will be more delayed than others if they travel
through a slower or busier network connection. The variation in packet
delay is called \emph{jitter}, and this can have a nastily negative
impact on programs that communicate in real time.

\textbf{Out-of-Order Delivery} Out-of-order delivery is also a result of
packets taking different paths through the network to their
destinations. The application at the receiving end needs to put them
back together in the right order for the message to be completed. So if
there are significant delays, or the packets are reassembled out of
order, users will probably notice degradation of an application's
quality.

QoS can ensure that applications with a required level of predictability
will receive the necessary bandwidth to work properly. Clearly, on
networks with excess bandwidth, this is not a factor, but the more
limited your bandwidth is, the more important a concept like this
becomes!

\subsubsection[Traffic
Characteristics]{\texorpdfstring{\protect\hypertarget{c22.xhtmlux5cux23c22-sec-12}{}{}Traffic
Characteristics}{Traffic Characteristics}}

In today's networks, you will find a mix of data, voice, and video
traffic. Each traffic type has different properties.

\protect\hyperlink{c22.xhtmlux5cux23figure22-12}{Figure 22.12} shows the
traffic characteristics found in today's network for data, voice, and
video.

\protect\hypertarget{c22.xhtmlux5cux23Page_963}{}{}

\begin{figure}
\centering
\includegraphics{images/c22f012.jpg}
\caption{{\protect\hyperlink{c22.xhtmlux5cux23figureanchor22-12}{\textbf{FIGURE
22.12}} Traffic characteristics}}
\end{figure}

Data traffic is not real-time traffic, and includes data packets
comprising of bursty (or unpredictable) traffic and widely varying
packet arrival times.

The following are data characteristics on a network, as shown in
\protect\hyperlink{c22.xhtmlux5cux23figure22-12}{Figure 22.12}:

\begin{enumerate}
\tightlist
\item
  Smooth/bursty
\item
  Benign/greedy
\item
  Drop insensitive
\item
  Delay insensitive
\item
  TCP retransmits
\end{enumerate}

Data traffic doesn't really require special handling in today's network,
especially if TCP is used. Voice traffic is real-time traffic with
constant, predictable bandwidth and known packet arrival times.

The following are voice characteristics on a network:

\begin{enumerate}
\tightlist
\item
  Smooth traffic
\item
  Benign
\item
  Drop insensitive
\item
  Delay sensate insensitive
\item
  UDP priority
\end{enumerate}

One-way voice traffic needs the following:

\begin{enumerate}
\tightlist
\item
  Latency of less than or equal to 150 milliseconds
\item
  Jitter of less than or equal to 30 milliseconds
\item
  \protect\hypertarget{c22.xhtmlux5cux23Page_964}{}{}Loss of less than
  or equal to 1\%
\item
  Bandwidth of only 30--128k Kbps
\end{enumerate}

There are several types of video traffic, and a lot of the traffic on
the Internet today is video traffic, with Netflix, Hulu, etc. Video
traffic can include streaming video, real-time interactive video, and
video conferences.

One-way video traffic needs the following:

\begin{enumerate}
\tightlist
\item
  Latency of less than or equal to 200--400 milliseconds
\item
  Jitter of less than or equal to 30--50 milliseconds
\item
  Loss of less than or equal to 0.1\%--1\%
\item
  Bandwidth of 384 Kbps to 20 Mbps or greater
\end{enumerate}

\subsection[Trust
Boundary]{\texorpdfstring{\protect\hypertarget{c22.xhtmlux5cux23c22-sec-13}{}{}Trust
Boundary}{Trust Boundary}}

The trust boundary is a point in the network where packet markings
(which identify traffic such as voice, video, or data) are not
necessarily trusted. You can create, remove, or rewrite markings at that
point. The borders of a trust domain are the network locations where
packet markings are accepted and acted upon.
\protect\hyperlink{c22.xhtmlux5cux23figure22-13}{Figure 22.13} shows
some typical trust boundaries.

\begin{figure}
\centering
\includegraphics{images/c22f013.jpg}
\caption{{\protect\hyperlink{c22.xhtmlux5cux23figureanchor22-13}{\textbf{FIGURE
22.13}} Trust boundaries}}
\end{figure}

The figure shows that IP phones and router interfaces are typically
trusted, but beyond those points are not. Here are some things you need
to remember for the exam objectives:

\textbf{Untrusted domain} This is the part of the network that you are
not managing, such as PC, printers, etc.

\textbf{Trusted domain} This is part of the network with only
administrator-managed devices such as switches, routers, etc.

\textbf{Trust boundary} This is where packets are classified and marked.
For example, the trust boundary would be IP phones and the boundary
between the ISP and enterprise network. In an enterprise campus network,
the trust boundary is almost always at the edge switch.

\protect\hypertarget{c22.xhtmlux5cux23Page_965}{}{}Traffic at the trust
boundary is classified and marked before being forwarded to the trusted
domain. Markings on traffic coming from an untrusted domain are usually
ignored to prevent end-user-controlled markings from taking unfair
advantage of the network QoS configuration.

\subsection[QoS
Mechanisms]{\texorpdfstring{\protect\hypertarget{c22.xhtmlux5cux23c22-sec-14}{}{}QoS
Mechanisms}{QoS Mechanisms}}

In this section we'll be covering these important mechanisms:

\begin{enumerate}
\tightlist
\item
  Classification and marking tools
\item
  Policing, shaping, and re-marking tools
\item
  Congestion management (or scheduling) tools
\item
  Link-specific tools
\end{enumerate}

So let's take a deeper look at each mechanism now.

\subsubsection[Classification and
Marking]{\texorpdfstring{\protect\hypertarget{c22.xhtmlux5cux23c22-sec-15}{}{}Classification
and Marking}{Classification and Marking}}

A classifier is an IOS tool that inspects packets within a field to
identify the type of traffic that they are carrying. This is so that QoS
can determine which traffic class they belong to and determine how they
should be treated. It's important that this isn't a constant cycle for
traffic because it does take up time and resources. Traffic is then
directed to a policy-enforcement mechanism, referred to as policing, for
its specific type.

Policy enforcement mechanisms include marking, queuing, policing, and
shaping, and there are various layer 2 and layer 3 fields in a frame and
packet for marking traffic. You are definitely going to have to
understand these marking techniques to meet the objectives, so here we
go:

\textbf{Class of Service (CoS)} An Ethernet frame marking at layer 2,
which contains 3 bits. This is called the Priority Code Point (PCP)
within an Ethernet frame header when VLAN tagged frames as defined by
IEEE 802.1Q are used.

\textbf{Type of Service (ToS)} ToS comprises of 8 bits, 3 of which are
designated as the IP precedence field in an IPv4 packet header. The IPv6
header field is called Traffic Class.

\textbf{Differentiated Services Code Point (DSCP or DiffServ)} One of
the methods we can use for classifying and managing network traffic and
providing quality of service (QoS) on modern IP networks is DSCP. This
technology uses a 6-bit differentiated services code point in the 8-bit
Differentiated Services field (DS field) in the IP header for packet
classification. DSCP allows for the creation of traffic classes that can
be used to assign priorities. While IP precedence is the old way to mark
ToS, DSCP is the new way. DSCP is backward compatible with IP
precedence.

Layer 3 packet marking with IP precedence and DSCP is the most widely
deployed marking option because layer 3 packet markings have end-to-end
significance.

\protect\hypertarget{c22.xhtmlux5cux23Page_966}{}{}\textbf{Class
Selector} Class Selector uses the same 3 bits of the field as IP
precedence and is used to indicate a 3-bit subset of DSCP values.

\textbf{Traffic Identifier (TID)} TID, used in wireless frames, describe
a 3-bit field in the QoS control field in 802.11. It's very similar to
CoS, so just remember CoS is wired Ethernet and TID is wireless.

\paragraph{Classification Marking Tools}

As discussed in the previous section, classification of traffic
determines which type of traffic the packets or frames belong to, which
then allows you to apply policies to it by marking, shaping, and
policing. Always try to mark traffic as close to the trust boundary as
possible.

To classify traffic, we generally use three ways:

\textbf{Markings} This looks at header informant on existing layer 2 or
3 settings, and classification is based on existing markings.

\textbf{Addressing} This classification technique looks at header
information using source and destinations of interfaces, layer 2 and 3
addresses, and layer 4 port numbers. You can group traffic with devices
using IP and traffic by type using port numbers.

\textbf{Application signatures} This is the way to look at the
information in the payload, and this classification technique is called
deep packet inspection.

Let's dive deeper into deep packet inspection by discussing something
called Network Based Application Recognition (NBAR).

NBAR is a classifier that provides deep-packet inspection on layer 4 to
7 on a packet, however, know that using NBAR is the most CPU intensive
technique compared to using addresses (IP or ports) or access control
lists (ACLs).

Since it's not always possible to identify applications by looking at
just layers 3 and 4, NBAR looks deep into the packet payload and
compares the payload content against its signature database called a
Packet Description Language Model (PDLM).

There are two different modes of operation used with NBAR:

\textbf{Passive mode} Using passive mode will give you real-time
statistics on applications by protocol or interface as well as the bit
rate, packet, and byte counts.

\textbf{Active mode} Classifies applications for traffic marking so that
QoS policies can be applied.

\subsubsection[Policing, Shaping, and
Re-Marking]{\texorpdfstring{\protect\hypertarget{c22.xhtmlux5cux23c22-sec-16}{}{}Policing,
Shaping, and Re-Marking}{Policing, Shaping, and Re-Marking}}

Okay---now that we've identified and marked traffic, it's time to put
some action on our packet. We do this with bandwidth assignments,
policing, shaping, queuing, or dropping. For example, if some traffic
exceeds bandwidth, it might be delayed, dropped, or even re-marked in
order to avoid congestion.

Policers and shapers are two tools that identify and respond to traffic
problems and are both rate limiters.
\protect\hyperlink{c22.xhtmlux5cux23figure22-14}{Figure 22.14} shows how
they differ.

\protect\hypertarget{c22.xhtmlux5cux23Page_967}{}{}

\begin{figure}
\centering
\includegraphics{images/c22f014.jpg}
\caption{{\protect\hyperlink{c22.xhtmlux5cux23figureanchor22-14}{\textbf{FIGURE
22.14}} Policing and shaping rate limiters}}
\end{figure}

Policers and shapers identify traffic violations in a similar manner,
but they differ in their response:

\textbf{Policers} Since the policers make instant decisions you want to
deploy them on the ingress if possible. This is because you want to drop
traffic as soon as you receive it if it's going to be dropped anyway.
Even so, you can still place them on an egress to control the amount of
traffic per class. When traffic is exceeded, policers don't delay it,
which means they do not introduce jitter or delay, they just check the
traffic and can drop it or re-mark it. Just know that this means there's
a higher drop probability, it can cause a significant amount of TCP
resends.

\textbf{Shapers} Shapers are usually deployed between an enterprise
network, on the egress side, and the service provider network to make
sure you stay within the carrier's contract rate. If the traffic does
exceed the rate, it will get policed by the provider and dropped. This
allows the traffic to meet the SLA and means there will be fewer TCP
resends than with policers. Be aware that shaping does introduce jitter
and delay.

\begin{center}\rule{0.5\linewidth}{0.5pt}\end{center}

\includegraphics{images/tip.png}Just remember that policers drop traffic
and shapers delay it. Policers have significant TCP resends and shapers
do not. Shapers introduce delay and jitter, but policers do not.

\begin{center}\rule{0.5\linewidth}{0.5pt}\end{center}

\subsubsection[Tools for Managing
Congestion]{\texorpdfstring{\protect\hypertarget{c22.xhtmlux5cux23c22-sec-17}{}{}Tools
for Managing Congestion}{Tools for Managing Congestion}}

This section and the next section on congestion avoidance will cover
congestion issues. If traffic exceeds network resource (always), the
traffic gets queued, which is basically the temporary storage of
backed-up packets. You perform queuing in order to avoid dropping
packets. This isn't a bad thing. It's actually a good thing or all
traffic would immediately be dropped if packets couldn't get processed
immediately. However, traffic classes like
\protect\hypertarget{c22.xhtmlux5cux23Page_968}{}{}VoIP would actually
be better off just being immediately dropped unless you can somehow
guarantee delay-free bandwidth for that traffic.

When congestion occurs, the congestion management tools are activated.
There are two types, as shown in
\protect\hyperlink{c22.xhtmlux5cux23figure22-15}{Figure 22.15}.

\begin{figure}
\centering
\includegraphics{images/c22f015.jpg}
\caption{{\protect\hyperlink{c22.xhtmlux5cux23figureanchor22-15}{\textbf{FIGURE
22.15}} Congestion management}}
\end{figure}

Let's take a closer look at congestion management:

\textbf{Queuing (or buffering)} Buffering is the logic of ordering
packets in output buffers. It is activated only when congestion occurs.
When queues fill up, packets can be reordered so that the
higher-priority packets can be sent out of the exit interface sooner
than the lower-priority ones.

\textbf{Scheduling} This is the process of deciding which packet should
be sent out next and occurs whether or not there is congestion on the
link.

Staying with scheduling for another minute, know that there are some
schedule mechanisms that exist you really need to be familiar with.
We'll go over those, and then I'll head back over to a detailed look at
queuing:

\begin{enumerate}
\tightlist
\item
  \textbf{Strict priority scheduling} Low-priority queues are only
  serviced once the high-priority queues are empty. This is great if you
  are the one sending high-priority traffic, but it's possible that
  low-priority queues will never be processed. We call this traffic or
  queue starvation.
\item
  \textbf{Round-robin scheduling} This is a rather fair technique
  because queues are serviced in a set sequence. You won't have starving
  queues here, but real-time traffic suffers greatly.
\item
  \textbf{Weighted fair scheduling} By weighing the queues, the
  scheduling process will service some queues more often than others,
  which is an upgrade over round-robin. You won't have any starvation
  here either, but unlike with round-robin, you can give priority to
  real-time traffic. It does not, however, provide bandwidth guarantees.
\end{enumerate}

Okay, let's run back over and finish queueing. Queuing typically is a
layer 3 process, but some queueing can occur at layer 2 or even layer 1.
Interestingly, if a layer 2 queue fills up, the data can be pushed into
layer 3 queues, and at layer 1 (called the transmit ring or TX-ring
queue), when that fills up, the data will be pushed to layer 2 and 3
queues. This is when QoS becomes active on the device.

There are many different queuing mechanisms, with only two typically
used today, but let's take a look at the legacy queuing methods first:

\protect\hypertarget{c22.xhtmlux5cux23Page_969}{}{}\textbf{First in,
first out (FIFO)} A single queue with packets being processed in the
exact order in which they arrived.

\textbf{Priority queuing (PQ)} This is not really a good queuing method
because lower-priority queues are served only when the higher-priority
queues are empty. There are only four queues, and low-priority traffic
many never be sent.

\textbf{Custom queueing (CQ)} With up to 16 queues and round-robin
scheduling, CQ prevents low-level queue starvation and provides traffic
guarantees. But it doesn't provide strict priority for real-time
traffic, so your VoIP traffic could end up being dropped.

\textbf{Weighted fair queuing (WFQ)} This was actually a pretty popular
way of queuing for a long time because it divided up the bandwidth by
the number of flows, which provided bandwidth for all applications. This
was great for real-time traffic, but it doesn't offer any guarantees for
a particular flow.

Now that you know about all the not so good queuing methods to use,
let's take a look at the two newer queuing mechanisms that are
recommended for today's rich-media networks, detailed in
\protect\hyperlink{c22.xhtmlux5cux23figure22-16}{Figure 22.16}).

\begin{figure}
\centering
\includegraphics{images/c22f016.jpg}
\caption{{\protect\hyperlink{c22.xhtmlux5cux23figureanchor22-16}{\textbf{FIGURE
22.16}} Queuing mechanisms}}
\end{figure}

The two new and improved queuing mechanisms you should now use in
today's network are class-based weighted fair queuing and low latency
queuing:

\textbf{Class-based weighted fair queuing (CBWFQ)} Provides fairness and
bandwidth guarantees for all traffic, but it does not provide latency
guarantees and is typically only used for data traffic management.

\textbf{Low latency queuing (LLQ):} LLQ is really the same thing as
CBWFQ but with stricter priorities for real-time traffic. LLQ is great
for both data and real-time traffic because it provides both latency and
bandwidth guarantees.

In \protect\hyperlink{c22.xhtmlux5cux23figure22-16}{Figure 22.16}, you
can see the LLQ queuing mechanism, which is suitable for networks with
real-time traffic. If you remove the low-latency queue (at the top),
you're then left with CBWFQ, which is only used for data-traffic
networks.

\protect\hypertarget{c22.xhtmlux5cux23Page_970}{}{}

\subsubsection[Tools for Congestion
Avoidance]{\texorpdfstring{\protect\hypertarget{c22.xhtmlux5cux23c22-sec-18}{}{}Tools
for Congestion Avoidance}{Tools for Congestion Avoidance}}

TCP changed our networking world when it introduced sliding windows as a
flow-control mechanism in the mid-1990s. Flow control is a way for the
receiving device to control the amount of traffic from a transmitting
device.

If a problem occurred during a data transmission (always), the previous
flow control methods used by TCP and other layer 4 protocols like SPX,
that we used before sliding windows, the transmission rate would be in
half, and left there at the same rate, or lower, for the duration of the
connection. This was certainly a point of contention with users!

TCP actually does cut transmission rates drastically if a flow control
issue occurs, but it increases the transmission rate once the missing
segments are resolved or the packets are finally processed. Because of
this behavior, and although it was awesome at the time, this method can
result in what we call tail drop. Tail drop is definitely suboptimal for
today's networks because using it, we're not utilizing the bandwidth
effectively.

Just to clarify, tail drop refers to the dropping of packets as they
arrive when the queues on the receiving interface are full. This is a
waste of precious bandwidth since TCP will just keep resending the data
until it's happy again (meaning an ACK has been received). So now this
brings up another new term, \emph{TCP global synchronization}, where
senders will reduce their transmission rate at the same time when packet
loss occurs.

Congestion avoidance starts dropping packets before a queue fills, and
it drops the packets by using traffic weights instead just randomness.
Cisco uses something called weighted random early detection (WRED),
which is a queuing method that ensures that high-precedence traffic has
lower loss rates than other traffic during congestion. This allows more
important traffic, like VoIP, to be prioritized and dropped over what
you'd consider less important traffic such as, for example, a connection
to Facebook.

\begin{center}\rule{0.5\linewidth}{0.5pt}\end{center}

\includegraphics{images/tip.png}Queuing algorithms manage the front of
the queue and congestion mechanisms manage the back of the queue.

\begin{center}\rule{0.5\linewidth}{0.5pt}\end{center}

\protect\hyperlink{c22.xhtmlux5cux23figure22-17}{Figure 22.17}
demonstrates how congestion avoidance works.

\begin{figure}
\centering
\includegraphics{images/c22f017.jpg}
\caption{{\protect\hyperlink{c22.xhtmlux5cux23figureanchor22-17}{\textbf{FIGURE
22.17}} Congestion avoidance}}
\end{figure}

\protect\hypertarget{c22.xhtmlux5cux23Page_971}{}{}If three traffic
flows start at different times, as shown in the example in
\protect\hyperlink{c22.xhtmlux5cux23figure22-17}{Figure 22.17}, and
congestion occurs, using TCP could first cause tail drop, which drops
the traffic as soon as it is received if the buffers are full. At that
point TCP would start another traffic flow, synchronizing the TCP flows
in waves, which would then leave much of the bandwidth unused.

\subsection[Summary]{\texorpdfstring{\protect\hypertarget{c22.xhtmlux5cux23c22-sec-19}{}{}Summary}{Summary}}

This all-new chapter was totally focused on the CCNA objectives for
intelligent networks. I started the chapter by covering switch stacking
using StackWise and then moved on to discuss the important realm of
cloud computing and what effect it has on the enterprise network.

Although this chapter had a lot of material, I stuck really close to the
objectives on the more difficult subjects to help you tune in
specifically to the content that's important for the CCNA, including the
following: Software Defined Networking (SDN), application programming
interfaces (APIs), Cisco's Application Policy Infrastructure Controller
Enterprise Module (APIC-EM), Intelligent WAN, and finally, quality of
service (QoS).

\subsection[Exam
Essentials]{\texorpdfstring{\protect\hypertarget{c22.xhtmlux5cux23c22-sec-20}{}{}Exam
Essentials}{Exam Essentials}}

\textbf{Understand switch stacking and StackWise.} You can connect up to
nine individual switches together to create a StackWise.

\textbf{Understand basic cloud technology.} Understand cloud services
such as SaaS and others and how virtualization works.

\textbf{Have a deep understanding of QoS.} You must understand QoS,
specifically marking; device trust; prioritization for voice, video, and
data; shaping; policing; and congestion management in detail.

\textbf{Understand APIC-EM and the path trace.} Read through the APIC-EM
section as well as the APIC-EM path trace section, which cover the CCNA
objectives fully.

\textbf{Understand SDN.} Understand how a controller works, and
especially the control and data plane, as well as the northbound and
southbound APIs.

\subsection[Written Lab
22]{\texorpdfstring{\protect\hypertarget{c22.xhtmlux5cux23c22-sec-21}{}{}Written
Lab 22}{Written Lab 22}}

You can find the answers to this lab in Appendix A, ``Answers to Written
Labs.''

Write the answers to the following questions:

\begin{enumerate}
\tightlist
\item
  Which QoS mechanism is a 6-bit value that is used to describe the
  meaning of the layer 3 IPv4 ToS field?
\item
  \protect\hypertarget{c22.xhtmlux5cux23Page_972}{}{}Southbound SDN
  interfaces are used between which two planes?
\item
  Which QoS mechanism is a term that is used to describe a 3-bit field
  in the QoS control field of wireless frames?
\item
  What are the three general ways to classify traffic?
\item
  CoS is a layer 2 QoS \_\_\_\_\_\_\_\_\_\_?
\item
  A session is using more bandwidth than allocated. Which QoS mechanism
  will drop the traffic?
\item
  What are the three SDN layers?
\item
  What are two examples of newer queuing mechanisms that are recommended
  for rich-media networks?
\item
  What is a layer 4 to 7 deep-packet inspection classifier that is more
  CPU intensive than marking?
\item
  \_\_\_\_\_\_\_\_\_\_APIs are responsible for the communication between
  the SDN controller and the services running over the network.
\end{enumerate}

\protect\hypertarget{c22.xhtmlux5cux23Page_973}{}{}

\subsection[Review
Questions]{\texorpdfstring{\protect\hypertarget{c22.xhtmlux5cux23c22-sec-22}{}{}Review
Questions}{Review Questions}}

\begin{center}\rule{0.5\linewidth}{0.5pt}\end{center}

\includegraphics{images/note.png}The following questions are designed to
test your understanding of this chapter's material. For more information
on how to get additional questions, please see
\href{http://www.lammle.com/ccna}{www.lammle.com/ccna}.

\begin{center}\rule{0.5\linewidth}{0.5pt}\end{center}

You can find the answers to these questions in Appendix B, ``Answers to
Review Questions.''

\begin{enumerate}
\tightlist
\item
  Which of the following is a congestion-avoidance mechanism?

  \begin{enumerate}
  \tightlist
  \item
    LMI
  \item
    WRED
  \item
    QPM
  \item
    QoS
  \end{enumerate}
\item
  Which of the following are true regarding StackWise? (Choose two.)

  \begin{enumerate}
  \tightlist
  \item
    A StackWise interconnect cable is used to connect the switches to
    create a bidirectional, closed-loop path.
  \item
    A StackWise interconnect cable is used to connect the switches to
    create a unidirectional, closed-loop path.
  \item
    StackWise can connect up to nine individual switches joined in a
    single logical switching unit.
  \item
    StackWise can connect up to nine individual switches joined into
    multiple logical switching units and managed by one IP address.
  \end{enumerate}
\item
  Which of the following is the best definition of cloud computing?

  \begin{enumerate}
  \tightlist
  \item
    UCS data center
  \item
    Computing model with all your data at the service provider
  \item
    On-demand computing model
  \item
    Computing model with all your data in your local data center
  \end{enumerate}
\item
  Which three features are properties and one-way requirements for voice
  traffic? (Choose three.)

  \begin{enumerate}
  \tightlist
  \item
    Bursty voice traffic.
  \item
    Smooth voice traffic.
  \item
    Latency should be below 400 ms.
  \item
    Latency should be below 150 ms.
  \item
    Bandwidth is roughly between 30 and 128 kbps.
  \item
    Bandwidth is roughly between 0.5 and 20 Mbps.
  \end{enumerate}
\item
  \protect\hypertarget{c22.xhtmlux5cux23Page_974}{}{}On which SDN
  architecture layer does Cisco APIC-EM reside?

  \begin{enumerate}
  \tightlist
  \item
    Data
  \item
    Control
  \item
    Presentation
  \item
    Application
  \end{enumerate}
\item
  In which cloud service model is the customer responsible for managing
  the operating system, software, platforms, and applications?

  \begin{enumerate}
  \tightlist
  \item
    IaaS
  \item
    SaaS
  \item
    PaaS
  \item
    APIC-EM
  \end{enumerate}
\item
  Which statement about QoS trust boundaries or domains is true?

  \begin{enumerate}
  \tightlist
  \item
    The trust boundary is always a router.
  \item
    PCs, printers, and tablets are usually part of a trusted domain.
  \item
    An IP phone is a common trust boundary.
  \item
    Routing will not work unless the service provider and the enterprise
    network are one single trust domain.
  \end{enumerate}
\item
  Which statement about IWAN is correct?

  \begin{enumerate}
  \tightlist
  \item
    The IWAN allows transport-independent connectivity.
  \item
    The IWAN allows only static routing.
  \item
    The IWAN does not provide application visibility because only
    encrypted traffic is transported.
  \item
    The IWAN needs special encrypting devices to provide an acceptable
    security level.
  \end{enumerate}
\item
  Which advanced classification tool can be used to classify data
  applications?

  \begin{enumerate}
  \tightlist
  \item
    NBAR
  \item
    MPLS
  \item
    APIC-EM
  \item
    ToS
  \end{enumerate}
\item
  The DSCP field constitutes how many bits in the IP header?

  \begin{enumerate}
  \tightlist
  \item
    3 bits
  \item
    4 bits
  \item
    6 bits
  \item
    8 bits
  \end{enumerate}
\item
  Between which two planes are SDN southbound interfaces used?

  \begin{enumerate}
  \tightlist
  \item
    Control
  \item
    Data
  \item
    \protect\hypertarget{c22.xhtmlux5cux23Page_975}{}{}Routing
  \item
    Application
  \end{enumerate}
\item
  Which option is a layer 2 QoS marking?

  \begin{enumerate}
  \tightlist
  \item
    EXP
  \item
    QoS group
  \item
    DSCP
  \item
    CoS
  \end{enumerate}
\item
  You are starting to use SDN in your network. What does this mean?

  \begin{enumerate}
  \tightlist
  \item
    You no longer have to work anymore, but you'll get paid more.
  \item
    You'll need to upgrade all your applications.
  \item
    You'll need to get rid of all Cisco switches.
  \item
    You now have more time to react faster when you receive a new
    business requirement.
  \end{enumerate}
\item
  Which QoS mechanism will drop traffic if a session uses more than the
  allotted bandwidth?

  \begin{enumerate}
  \tightlist
  \item
    Congestion management
  \item
    Shaping
  \item
    Policing
  \item
    Marking
  \end{enumerate}
\item
  Which three layers are part of the SDN architecture? (Choose three.)

  \begin{enumerate}
  \tightlist
  \item
    Network
  \item
    Data Link
  \item
    Control
  \item
    Data
  \item
    Transport
  \item
    Application
  \end{enumerate}
\item
  Which of the following is NOT true about APIC-EM ACL analysis?

  \begin{enumerate}
  \tightlist
  \item
    Fast comparison of ACLs between devices to visualize difference and
    identify misconfigurations
  \item
    Inspection, interrogation, and analysis of network access control
    policies
  \item
    Ability to provide layer 4 to layer 7 deep-packet inspection
  \item
    Ability to trace application-specific paths between end devices to
    quickly identify ACLs and other problem areas
  \end{enumerate}
\item
  Which two of the following are not part of APIC-EM?

  \begin{enumerate}
  \tightlist
  \item
    Southbound APIs are used for communication between the controllers
    and network devices.
  \item
    Northbound APIs are used for communication between the controllers
    and network devices.
  \item
    \protect\hypertarget{c22.xhtmlux5cux23Page_976}{}{}OnePK is Cisco
    proprietary.
  \item
    The control plane is responsible for the forwarding of frames or
    packets.
  \end{enumerate}
\item
  When stacking switches, which is true? (Choose two.)

  \begin{enumerate}
  \tightlist
  \item
    The stack is managed as multiple objects and has a single management
    IP address.
  \item
    The stack is managed as a single object and has a single management
    IP address.
  \item
    The master switch is chosen when you configure the first switch's
    master algorithm to on.
  \item
    The master switch is elected from one of the stack member switches.
  \end{enumerate}
\item
  Which of the following services provides the operating system and the
  network?

  \begin{enumerate}
  \tightlist
  \item
    IaaS
  \item
    PaaS
  \item
    SaaS
  \item
    None of the above
  \end{enumerate}
\item
  Which of the following services provides the required software, the
  operating system, and the network?

  \begin{enumerate}
  \tightlist
  \item
    IaaS
  \item
    PaaS
  \item
    SaaS
  \item
    None of the above
  \end{enumerate}
\item
  Which of the following is NOT a benefit of cloud computing for a cloud
  user?

  \begin{enumerate}
  \tightlist
  \item
    On-demand, self-service resource provisioning
  \item
    Centralized appearance of resources
  \item
    Local backups
  \item
    Highly available, horizontally scaled application architectures
  \end{enumerate}
\end{enumerate}

\protect\hypertarget{b01.xhtml}{}{}

\section[{Appendix A}\\
{Answers to Written
Labs}]{\texorpdfstring{\protect\hypertarget{b01.xhtmlux5cux23bapp01}{}{}\protect\hypertarget{b01.xhtmlux5cux23Page_977}{}{}{Appendix
A}\\
{Answers to Written Labs}}{Appendix A Answers to Written Labs}}

\subsection[Chapter 1:
Internetworking]{\texorpdfstring{\protect\hypertarget{b01.xhtmlux5cux23bapp01-sec-1}{}{}\protect\hypertarget{b01.xhtmlux5cux23Page_978}{}{}Chapter
1: Internetworking}{Chapter 1: Internetworking}}

\subsubsection[Written Lab 1.1: OSI
Questions]{\texorpdfstring{\protect\hypertarget{b01.xhtmlux5cux23bapp01-sec-2}{}{}Written
Lab 1.1: OSI Questions}{Written Lab 1.1: OSI Questions}}

\begin{enumerate}
\tightlist
\item
  The Application layer is responsible for finding the network resources
  broadcast from a server and adding flow control and error control (if
  the application developer chooses).
\item
  The Physical layer takes frames from the Data Link layer and encodes
  the 1s and 0s into a digital or analog (Ethernet or wireless) signal
  for transmission on the network medium.
\item
  The Network layer provides routing through an internetwork and logical
  addressing.
\item
  The Presentation layer makes sure that data is in a readable format
  for the Application layer.
\item
  The Session layer sets up, maintains, and terminates sessions between
  applications.
\item
  PDUs at the Data Link layer are called frames and provide physical
  addressing plus other options to place packets on the network medium.
\item
  The Transport layer uses virtual circuits to create a reliable
  connection between two hosts.
\item
  The Network layer provides logical addressing, typically IP addressing
  and routing.
\item
  The Physical layer is responsible for the electrical and mechanical
  connections between devices.
\item
  The Data Link layer is responsible for the framing of data packets.
\item
  The Session layer creates sessions between different hosts'
  applications.
\item
  The Data Link layer frames packets received from the Network layer.
\item
  The Transport layer segments user data.
\item
  The Network layer creates packets out of segments handed down from the
  Transport layer.
\item
  The Physical layer is responsible for transporting 1s and 0s (bits) in
  a digital signal.
\item
  Segments, packets, frames, bits
\item
  Transport
\item
  Data Link
\item
  Network
\item
  48 bits (6 bytes) expressed as a hexadecimal number
\end{enumerate}

\subsubsection[Written Lab 1.2: Defining the OSI Layers and
Devices]{\texorpdfstring{\protect\hypertarget{b01.xhtmlux5cux23bapp01-sec-3}{}{}\protect\hypertarget{b01.xhtmlux5cux23Page_979}{}{}Written
Lab 1.2: Defining the OSI Layers and
Devices}{Written Lab 1.2: Defining the OSI Layers and Devices}}

\begin{longtable}[]{@{}ll@{}}
\toprule
Description & Device or OSI Layer\tabularnewline
\midrule
\endhead
This device sends and receives information about the Network layer. &
Router\tabularnewline
This layer creates a virtual circuit before transmitting between two end
stations. & Transport\tabularnewline
This device uses hardware addresses to filter a network. & Bridge or
switch\tabularnewline
Ethernet is defined at these layers. & Data Link and
Physical\tabularnewline
This layer supports flow control, sequencing, and acknowledgments. &
Transport\tabularnewline
This device can measure the distance to a remote network. &
Router\tabularnewline
Logical addressing is used at this layer. & Network\tabularnewline
Hardware addresses are defined at this layer. & Data Link (MAC
sublayer)\tabularnewline
This device creates one collision domain and one broadcast domain. &
Hub\tabularnewline
This device creates many smaller collision domains, but the network is
still one large broadcast domain. & Switch or bridge\tabularnewline
This device can never run full-duplex. & Hub\tabularnewline
This device breaks up collision domains and broadcast domains. &
Router\tabularnewline
\bottomrule
\end{longtable}

\subsubsection[Written Lab 1.3: Identifying Collision and
Broadcast~Domains]{\texorpdfstring{\protect\hypertarget{b01.xhtmlux5cux23bapp01-sec-4}{}{}Written
Lab 1.3: Identifying Collision and
Broadcast~Domains}{Written Lab 1.3: Identifying Collision and Broadcast~Domains}}

\begin{enumerate}
\def\labelenumi{\Alph{enumi}.}
\tightlist
\item
  Hub: One collision domain, one broadcast domain
\item
  Bridge: Two collision domains, one broadcast domain
\item
  Switch: Four collision domains, one broadcast domain
\item
  Router: Three collision domains, three broadcast domains
\end{enumerate}

\subsection[Chapter 2: Ethernet Networking and Data
Encapsulation]{\texorpdfstring{\protect\hypertarget{b01.xhtmlux5cux23bapp01-sec-5}{}{}\protect\hypertarget{b01.xhtmlux5cux23Page_980}{}{}Chapter
2: Ethernet Networking and Data
Encapsulation}{Chapter 2: Ethernet Networking and Data Encapsulation}}

\subsubsection[Written Lab 2.1: Binary/Decimal/Hexadecimal
Conversion]{\texorpdfstring{\protect\hypertarget{b01.xhtmlux5cux23bapp01-sec-6}{}{}Written
Lab 2.1: Binary/Decimal/Hexadecimal
Conversion}{Written Lab 2.1: Binary/Decimal/Hexadecimal Conversion}}

\begin{enumerate}
\item
  \begin{longtable}[]{@{}llllllllll@{}}
  \toprule
  Decimal & 128 & 64 & 32 & 16 & 8 & 4 & 2 & 1 & Binary\tabularnewline
  \midrule
  \endhead
  192 & 1 & 1 & 0 & 0 & 0 & 0 & 0 & 0 & 11000000\tabularnewline
  168 & 1 & 0 & 1 & 0 & 1 & 0 & 0 & 0 & 10101000\tabularnewline
  10 & 0 & 0 & 0 & 0 & 1 & 0 & 1 & 0 & 00001010\tabularnewline
  15 & 0 & 0 & 0 & 0 & 1 & 1 & 1 & 1 & 00001111\tabularnewline
  Decimal & 128 & 64 & 32 & 16 & 8 & 4 & 2 & 1 & Binary\tabularnewline
  172 & 1 & 0 & 1 & 0 & 1 & 1 & 0 & 0 & 10101100\tabularnewline
  16 & 0 & 0 & 0 & 1 & 0 & 0 & 0 & 0 & 00010000\tabularnewline
  20 & 0 & 0 & 0 & 1 & 0 & 1 & 0 & 0 & 00010100\tabularnewline
  55 & 0 & 0 & 1 & 1 & 0 & 1 & 1 & 1 & 00110111\tabularnewline
  Decimal & 128 & 64 & 32 & 16 & 8 & 4 & 2 & 1 & Binary\tabularnewline
  10 & 0 & 0 & 0 & 0 & 1 & 0 & 1 & 0 & 00001010\tabularnewline
  11 & 0 & 0 & 0 & 0 & 1 & 0 & 1 & 1 & 00001011\tabularnewline
  12 & 0 & 0 & 0 & 0 & 1 & 1 & 0 & 0 & 00001100\tabularnewline
  99 & 0 & 1 & 1 & 0 & 0 & 0 & 1 & 1 & 01100011\tabularnewline
  \bottomrule
  \end{longtable}
\item
  \begin{longtable}[]{@{}llllllllll@{}}
  \toprule
  \endhead
  Binary & 128 & 64 & 32 & 16 & 8 & 4 & 2 & 1 & Decimal\tabularnewline
  11001100 & 1 & 1 & 0 & 0 & 1 & 1 & 0 & 0 & 204\tabularnewline
  00110011 & 0 & 0 & 1 & 1 & 0 & 0 & 1 & 1 & 51\tabularnewline
  10101010 & 1 & 0 & 1 & 0 & 1 & 0 & 1 & 0 & 170\tabularnewline
  01010101 & 0 & 1 & 0 & 1 & 0 & 1 & 0 & 1 & 85\tabularnewline
  \protect\hypertarget{b01.xhtmlux5cux23Page_981}{}{}Binary & 128 & 64 &
  32 & 16 & 8 & 4 & 2 & 1 & Decimal\tabularnewline
  11000110 & 1 & 1 & 0 & 0 & 0 & 1 & 1 & 0 & 198\tabularnewline
  11010011 & 1 & 1 & 0 & 1 & 0 & 0 & 1 & 1 & 211\tabularnewline
  00111001 & 0 & 0 & 1 & 1 & 1 & 0 & 0 & 1 & 57\tabularnewline
  11010001 & 1 & 1 & 0 & 1 & 0 & 0 & 0 & 1 & 209\tabularnewline
  Binary & 128 & 64 & 32 & 16 & 8 & 4 & 2 & 1 & Decimal\tabularnewline
  10000100 & 1 & 0 & 0 & 0 & 0 & 1 & 0 & 0 & 132\tabularnewline
  11010010 & 1 & 1 & 0 & 1 & 0 & 0 & 1 & 0 & 210\tabularnewline
  10111000 & 1 & 0 & 1 & 1 & 1 & 0 & 0 & 0 & 184\tabularnewline
  10100110 & 1 & 0 & 1 & 0 & 0 & 1 & 1 & 0 & 166\tabularnewline
  \bottomrule
  \end{longtable}
\item
  \begin{longtable}[]{@{}llllllllll@{}}
  \toprule
  \endhead
  Binary & 128 & 64 & 32 & 16 & 8 & 4 & 2 & 1 &
  Hexadecimal\tabularnewline
  11011000 & 1 & 1 & 0 & 1 & 1 & 0 & 0 & 0 & D8\tabularnewline
  00011011 & 0 & 0 & 0 & 1 & 1 & 0 & 1 & 1 & 1B\tabularnewline
  00111101 & 0 & 0 & 1 & 1 & 1 & 1 & 0 & 1 & 3D\tabularnewline
  01110110 & 0 & 1 & 1 & 1 & 0 & 1 & 1 & 0 & 76\tabularnewline
  Binary & 128 & 6 & 32 & 16 & 8 & 4 & 2 & 1 &
  Hexadecimal\tabularnewline
  11001010 & 1 & 1 & 0 & 0 & 1 & 0 & 1 & 0 & CA\tabularnewline
  11110101 & 1 & 1 & 1 & 1 & 0 & 1 & 0 & 1 & F5\tabularnewline
  10000011 & 1 & 0 & 0 & 0 & 0 & 0 & 1 & 1 & 83\tabularnewline
  11101011 & 1 & 1 & 1 & 0 & 1 & 0 & 1 & 1 & EB\tabularnewline
  Binary & 128 & 64 & 32 & 16 & 8 & 4 & 2 & 1 &
  Hexadecimal\tabularnewline
  10000100 & 1 & 0 & 0 & 0 & 0 & 1 & 0 & 0 & 84\tabularnewline
  11010010 & 1 & 1 & 0 & 1 & 0 & 0 & 1 & 0 & D2\tabularnewline
  01000011 & 0 & 1 & 0 & 0 & 0 & 0 & 1 & 1 & 43\tabularnewline
  10110011 & 1 & 0 & 1 & 1 & 0 & 0 & 1 & 1 & B3\tabularnewline
  \bottomrule
  \end{longtable}
\end{enumerate}

\subsubsection[Written Lab 2.2: CSMA/CD
Operations]{\texorpdfstring{\protect\hypertarget{b01.xhtmlux5cux23bapp01-sec-7}{}{}\protect\hypertarget{b01.xhtmlux5cux23Page_982}{}{}Written
Lab 2.2: CSMA/CD Operations}{Written Lab 2.2: CSMA/CD Operations}}

When a collision occurs on an Ethernet LAN, the following happens:

\begin{enumerate}
\tightlist
\item
  A jam signal informs all devices that a collision occurred.
\item
  The collision invokes a random backoff algorithm.
\item
  Each device on the Ethernet segment stops transmitting for a short
  time until the timers expire.
\item
  All hosts have equal priority to transmit after the timers have
  expired.
\end{enumerate}

\subsubsection[Written Lab 2.3:
Cabling]{\texorpdfstring{\protect\hypertarget{b01.xhtmlux5cux23bapp01-sec-8}{}{}Written
Lab 2.3: Cabling}{Written Lab 2.3: Cabling}}

\begin{enumerate}
\tightlist
\item
  Crossover
\item
  Straight-through
\item
  Crossover
\item
  Crossover
\item
  Straight-through
\item
  Crossover
\item
  Crossover
\item
  Rolled
\end{enumerate}

\subsubsection[Written Lab 2.4:
Encapsulation]{\texorpdfstring{\protect\hypertarget{b01.xhtmlux5cux23bapp01-sec-9}{}{}Written
Lab 2.4: Encapsulation}{Written Lab 2.4: Encapsulation}}

At a transmitting device, the data encapsulation method works like this:

\begin{enumerate}
\tightlist
\item
  User information is converted to data for transmission on the network.
\item
  Data is converted to segments, and a reliable connection is set up
  between the transmitting and receiving hosts.
\item
  Segments are converted to packets or datagrams, and a logical address
  is placed in the header so each packet can be routed through an
  internetwork.
\item
  Packets or datagrams are converted to frames for transmission on the
  local network. Hardware (Ethernet) addresses are used to uniquely
  identify hosts on a local network segment.
\item
  Frames are converted to bits, and a digital encoding and clocking
  scheme is used.
\end{enumerate}

\subsection[Chapter 3: Introduction to
TCP/IP]{\texorpdfstring{\protect\hypertarget{b01.xhtmlux5cux23bapp01-sec-10}{}{}\protect\hypertarget{b01.xhtmlux5cux23Page_983}{}{}Chapter
3: Introduction to TCP/IP}{Chapter 3: Introduction to TCP/IP}}

\subsubsection[Written Lab 3.1:
TCP/IP]{\texorpdfstring{\protect\hypertarget{b01.xhtmlux5cux23bapp01-sec-11}{}{}Written
Lab 3.1: TCP/IP}{Written Lab 3.1: TCP/IP}}

\begin{enumerate}
\tightlist
\item
  192 through 223, 110\emph{xxxxx}
\item
  Host-to-Host or Transport
\item
  1 through 126
\item
  Loopback or diagnostics
\item
  Turn all host bits off.
\item
  Turn all host bits on.
\item
  10.0.0.0 through 10.255.255.255
\item
  172.16.0.0 through 172.31.255.255
\item
  192.168.0.0 through 192.168.255.255
\item
  0 through 9 and \emph{A}, \emph{B}, \emph{C}, \emph{D}, \emph{E}, and
  \emph{F}
\end{enumerate}

\subsubsection[Written Lab 3.2: Mapping Applications to the DoD
Model]{\texorpdfstring{\protect\hypertarget{b01.xhtmlux5cux23bapp01-sec-12}{}{}Written
Lab 3.2: Mapping Applications to the DoD
Model}{Written Lab 3.2: Mapping Applications to the DoD Model}}

\begin{enumerate}
\tightlist
\item
  Internet
\item
  Process/Application
\item
  Process/Application
\item
  Process/Application
\item
  Process/Application
\item
  Internet
\item
  Process/Application
\item
  Host-to-host/Transport
\item
  Process/Application
\item
  Host-to-host/Transport
\item
  Process/Application
\item
  Internet
\item
  \protect\hypertarget{b01.xhtmlux5cux23Page_984}{}{}Internet
\item
  Internet
\item
  Process/Application
\item
  Process/Application
\item
  Process/Application
\end{enumerate}

\subsection[Chapter 4: Easy
Subnetting]{\texorpdfstring{\protect\hypertarget{b01.xhtmlux5cux23bapp01-sec-13}{}{}Chapter
4: Easy Subnetting}{Chapter 4: Easy Subnetting}}

\subsubsection[Written Lab 4.1: Written Subnet Practice
\#1]{\texorpdfstring{\protect\hypertarget{b01.xhtmlux5cux23bapp01-sec-14}{}{}Written
Lab 4.1: Written Subnet Practice
\#1}{Written Lab 4.1: Written Subnet Practice \#1}}

\begin{enumerate}
\tightlist
\item
  192.168.100.25/30. A /30 is 255.255.255.252. The valid subnet is
  192.168.100.24, broadcast is 192.168.100.27, and valid hosts are
  192.168.100.25 and 26.
\item
  192.168.100.37/28. A /28 is 255.255.255.240. The fourth octet is a
  block size of 16. Just count by 16s until you pass 37. 0, 16, 32, 48.
  The host is in the 32 subnet, with a broadcast address of 47. Valid
  hosts 33--46.
\item
  A /27 is 255.255.255.224. The fourth octet is a block size of 32.
  Count by 32s until you pass the host address of 66. 0, 32, 64, 96. The
  host is in the 64 subnet, and the broadcast address is 95. Valid host
  range is 65--94.
\item
  192.168.100.17/29. A /29 is 255.255.255.248. The fourth octet is a
  block size of 8. 0, 8, 16, 24. The host is in the 16 subnet, broadcast
  of 23. Valid hosts 17--22.
\item
  192.168.100.99/26. A /26 is 255.255.255.192. The fourth octet has a
  block size of 64. 0, 64, 128. The host is in the 64 subnet, broadcast
  of 127. Valid hosts 65--126.
\item
  192.168.100.99/25. A /25 is 255.255.255.128. The fourth octet is a
  block size of 128. 0, 128. The host is in the 0 subnet, broadcast of
  127. Valid hosts 1--126.
\item
  A default Class B is 255.255.0.0. A Class B 255.255.255.0 mask is 256
  subnets, each with 254 hosts. We need fewer subnets. If we used
  255.255.240.0, this provides 16 subnets. Let's add one more subnet
  bit. 255.255.248.0. This is 5 bits of subnetting, which provides 32
  subnets. This is our best answer, a /21.
\item
  A /29 is 255.255.255.248. This is a block size of 8 in the fourth
  octet. 0, 8, 16. The host is in the 8 subnet, broadcast is 15.
\item
  A /29 is 255.255.255.248, which is 5 subnet bits and 3 host bits. This
  is only 6 hosts per subnet.
\item
  A /23 is 255.255.254.0. The third octet is a block size of 2. 0, 2, 4.
  The subnet is in the 16.2.0 subnet; the broadcast address is 16.3.255.
\end{enumerate}

\subsubsection[Written Lab 4.2: Written Subnet Practice
\#2]{\texorpdfstring{\protect\hypertarget{b01.xhtmlux5cux23bapp01-sec-15}{}{}\protect\hypertarget{b01.xhtmlux5cux23Page_985}{}{}Written
Lab 4.2: Written Subnet Practice
\#2}{Written Lab 4.2: Written Subnet Practice \#2}}

\begin{longtable}[]{@{}lll@{}}
\toprule
\endhead
Classful Address & Subnet Mask & Number of Hosts per Subnet
(2\textsuperscript{\emph{x}} -- 2)\tabularnewline
/16 & 255.255.0.0 & 65,534\tabularnewline
/17 & 255.255.128.0 & 32,766\tabularnewline
/18 & 255.255.192.0 & 16,382\tabularnewline
/19 & 255.255.224.0 & 8,190\tabularnewline
/20 & 255.255.240.0 & 4,094\tabularnewline
/21 & 255.255.248.0 & 2,046\tabularnewline
/22 & 255.255.252.0 & 1,022\tabularnewline
/23 & 255.255.254.0 & 510\tabularnewline
/24 & 255.255.255.0 & 254\tabularnewline
/25 & 255.255.255.128 & 126\tabularnewline
/26 & 255.255.255.192 & 62\tabularnewline
/27 & 255.255.255.224 & 30\tabularnewline
/28 & 255.255.255.240 & 14\tabularnewline
/29 & 255.255.255.248 & 6\tabularnewline
/30 & 255.255.255.252 & 2\tabularnewline
\bottomrule
\end{longtable}

\subsubsection[Written Lab 4.3: Written Subnet Practice
\#3]{\texorpdfstring{\protect\hypertarget{b01.xhtmlux5cux23bapp01-sec-16}{}{}Written
Lab 4.3: Written Subnet Practice
\#3}{Written Lab 4.3: Written Subnet Practice \#3}}

\begin{longtable}[]{@{}lllll@{}}
\toprule
\endhead
Decimal IP Address & Address Class & Number of Subnet and Host Bits &
Number of Subnets (2\textsuperscript{\emph{x}}) & Number of Hosts
(2\textsuperscript{\emph{x}} -- 2)\tabularnewline
10.25.66.154/23 & A & 15/9 & 32,768 & 510\tabularnewline
172.31.254.12/24 & B & 8/8 & 256 & 254\tabularnewline
192.168.20.123/28 & C & 4/4 & 16 & 14\tabularnewline
63.24.89.21/18 & A & 10/14 & 1,024 & 16,382\tabularnewline
128.1.1.254/20 & B & 4/12 & 16 & 4,094\tabularnewline
208.100.54.209/30 & C & 6/2 & 64 & 2\tabularnewline
\bottomrule
\end{longtable}

\subsection[Chapter 5: VLSMs, Summarization and Troubleshooting
TCP/IP]{\texorpdfstring{\protect\hypertarget{b01.xhtmlux5cux23bapp01-sec-17}{}{}\protect\hypertarget{b01.xhtmlux5cux23Page_986}{}{}Chapter
5: VLSMs, Summarization and Troubleshooting
TCP/IP}{Chapter 5: VLSMs, Summarization and Troubleshooting TCP/IP}}

\begin{enumerate}
\tightlist
\item
  192.168.0.0/20
\item
  172.144.0.0 255.240.0.0
\item
  192.168.32.0 255.255.224.0
\item
  192.168.96.0 255.255.240.0
\item
  66.66.0.0 255.255.240.0
\item
  192.168.0.0/17
\item
  172.16.0.0 255.255.248.0
\item
  192.168.128.0 255.255.192.0
\item
  53.60.96.0 255.255.224.0
\item
  172.16.0.0 255.255.192.0
\end{enumerate}

\subsection[Chapter 6: Cisco's Internetworking Operating System
(IOS)]{\texorpdfstring{\protect\hypertarget{b01.xhtmlux5cux23bapp01-sec-18}{}{}Chapter
6: Cisco's Internetworking Operating System
(IOS)}{Chapter 6: Cisco's Internetworking Operating System (IOS)}}

\subsubsection[Written Lab 6: Cisco
IOS]{\texorpdfstring{\protect\hypertarget{b01.xhtmlux5cux23bapp01-sec-19}{}{}Written
Lab 6: Cisco IOS}{Written Lab 6: Cisco IOS}}

\begin{enumerate}
\item
  \texttt{Router(config)\#\ clock\ rate\ 1000000}
\item
  \texttt{Switch\#\ config\ t}

\begin{verbatim}
switch config)# line vty 0 15
switch(config-line)# no login
\end{verbatim}
\item
  \texttt{Switch\#\ config\ t}

\begin{verbatim}
Switch(config)# int f0/1
Switch(config-if)# no shutdown
\end{verbatim}
\item
  \texttt{Switch\#\ erase\ startup-config}
\item
  \texttt{Switch\#\ config\ t}

\begin{verbatim}
Switch(config)# line console 0
Switch(config-line)# password todd
Switch(config-line)# login
\end{verbatim}
\item
  \texttt{Switch\#\ config\ t}

\begin{verbatim}
Switch(config)# enable secret cisco
\end{verbatim}
\item
  \texttt{Router\#\ show\ controllers\ serial\ 0/2}
\item
  \texttt{Switch\#\ show\ terminal}
\item
  \texttt{Switch\#\ reload}
\item
  \texttt{Switch\#\ config\ t}

\begin{verbatim}
Switch(config)# hostname Sales
\end{verbatim}
\end{enumerate}

\subsection[Chapter 7: Managing a Cisco
Internetwork]{\texorpdfstring{\protect\hypertarget{b01.xhtmlux5cux23bapp01-sec-20}{}{}Chapter
7: Managing a Cisco
Internetwork}{Chapter 7: Managing a Cisco Internetwork}}

\subsubsection[Written Lab 7.1: IOS
Management]{\texorpdfstring{\protect\hypertarget{b01.xhtmlux5cux23bapp01-sec-21}{}{}Written
Lab 7.1: IOS Management}{Written Lab 7.1: IOS Management}}

\begin{enumerate}
\tightlist
\item
  \texttt{copy\ start\ run}
\item
  \texttt{show\ cdp\ neighbor\ detail} or \texttt{show\ cdp\ entry\ *}
\item
  \texttt{show\ cdp\ neighbor}
\item
  Ctrl+Shift+6, then X
\item
  \texttt{show\ sessions}
\item
  Either \texttt{copy\ tftp\ run} or \texttt{copy\ start\ run}
\item
  NTP
\item
  \texttt{ip\ helper-address}
\item
  \texttt{ntp\ server\ ip\_address\ version\ 4}
\item
  \texttt{show\ ntp\ status}
\end{enumerate}

\subsubsection[Written Lab 7.2: Router
Memory]{\texorpdfstring{\protect\hypertarget{b01.xhtmlux5cux23bapp01-sec-22}{}{}Written
Lab 7.2: Router Memory}{Written Lab 7.2: Router Memory}}

\begin{enumerate}
\tightlist
\item
  Flash memory
\item
  ROM
\item
  \protect\hypertarget{b01.xhtmlux5cux23Page_989}{}{}NVRAM
\item
  ROM
\item
  RAM
\item
  RAM
\item
  ROM
\item
  ROM
\item
  RAM
\item
  RAM
\end{enumerate}

\subsection[Chapter 8: Managing Cisco
Devices]{\texorpdfstring{\protect\hypertarget{b01.xhtmlux5cux23bapp01-sec-23}{}{}Chapter
8: Managing Cisco Devices}{Chapter 8: Managing Cisco Devices}}

\subsubsection[Written Lab 8.1: IOS
Management]{\texorpdfstring{\protect\hypertarget{b01.xhtmlux5cux23bapp01-sec-24}{}{}Written
Lab 8.1: IOS Management}{Written Lab 8.1: IOS Management}}

\begin{enumerate}
\tightlist
\item
  \texttt{copy\ flash\ tftp}
\item
  0x2101
\item
  0x2102
\item
  0x2100
\item
  UDI
\item
  0x2142
\item
  \texttt{boot\ system}
\item
  POST test
\item
  \texttt{copy\ tftp\ flash}
\item
  \texttt{show\ license}
\end{enumerate}

\subsection[Chapter 9: IP
Routing]{\texorpdfstring{\protect\hypertarget{b01.xhtmlux5cux23bapp01-sec-25}{}{}Chapter
9: IP Routing}{Chapter 9: IP Routing}}

\begin{enumerate}
\item
  \texttt{router(config)\#\ ip\ route\ 172.16.10.0\ 255.255.255.0\ 172.16.20.1\ 150}
\item
  It will use the gateway interface MAC at L2 and the actual destination
  IP at L3.
\item
  \texttt{router(config)\#\ ip\ route\ 0.0.0.0\ 0.0.0.0\ 172.16.40.1}
\item
  \protect\hypertarget{b01.xhtmlux5cux23Page_990}{}{}Stub network
\item
  \texttt{Router\#\ show\ ip\ route}
\item
  Exit interface
\item
  False. The MAC address would be the local router interface, not the
  remote host.
\item
  True
\item
  \texttt{router(config)\ \#router\ rip}

\begin{verbatim}
router(config-router) #passive-interface S1
\end{verbatim}
\item
  True
\end{enumerate}

\subsection[Chapter 10: Layer 2
Switching]{\texorpdfstring{\protect\hypertarget{b01.xhtmlux5cux23bapp01-sec-26}{}{}Chapter
10: Layer 2 Switching}{Chapter 10: Layer 2 Switching}}

\begin{enumerate}
\tightlist
\item
  \texttt{show\ mac\ address-table}
\item
  Flood the frame out all ports except the port on which it was received
\item
  Address learning, forward/filter decisions, and loop avoidance
\item
  It will add the source MAC address in the forward/filter table and
  associate it with the port on which the frame was received.
\item
  Maximum 1, violation shutdown
\item
  Restrict and shutdown
\item
  Restrict
\item
  The addition of dynamically learned addresses to the
  running-configuration
\item
  \texttt{Show\ port-security\ interface\ fastethernet\ 0/12} and
  \texttt{show\ running-config}
\item
  False
\end{enumerate}

\subsection[Chapter 11: VLANs and InterVLAN
Routing]{\texorpdfstring{\protect\hypertarget{b01.xhtmlux5cux23bapp01-sec-27}{}{}Chapter
11: VLANs and InterVLAN
Routing}{Chapter 11: VLANs and InterVLAN Routing}}

\begin{enumerate}
\tightlist
\item
  False! You do not provide an IP address under any physical port.
\item
  STP
\item
  Broadcast
\item
  \protect\hypertarget{b01.xhtmlux5cux23Page_991}{}{}VLAN 1 is the
  default VLAN and cannot be changed, renamed, or deleted. VLANs
  1002--1005 are reserved, and VLANs 1006--4094 are extended VLANs and
  can only be configured if you are in VTP transparent mode. You can
  only configure VLANs 2--1001 by default.
\item
  \texttt{switchport\ trunk\ encapsulation\ dot1q}
\item
  Trunking sends information about all or many VLANs across a single
  link.
\item
  1000 (2 to 1001). VLAN 1 is the default VLAN and cannot be changed,
  renamed, or deleted. VLANs 1002--1005 are reserved, and VLANs
  1006--4094 are extended VLANs and can only be configured if you are in
  VTP transparent mode.
\item
  True
\item
  Access link
\item
  \texttt{switchport\ trunk\ native\ vlan\ 4}
\end{enumerate}

\subsection[Chapter 12:
Security]{\texorpdfstring{\protect\hypertarget{b01.xhtmlux5cux23bapp01-sec-28}{}{}Chapter
12: Security}{Chapter 12: Security}}

\begin{enumerate}
\item
  \texttt{access-list\ 10\ deny\ 172.16.0.0\ 0.0.255.255}

\begin{verbatim}
access-list 10 permit any
\end{verbatim}
\item
  \texttt{ip\ access-group\ 10\ out}
\item
  \texttt{access-list\ 10\ deny\ host\ 192.168.15.5}

\begin{verbatim}
access-list 10 permit any
\end{verbatim}
\item
  \texttt{show\ access-lists}
\item
  IDS, IPS
\item
  \texttt{access-list\ 110\ deny\ tcp\ host}

\begin{verbatim}
172.16.10.1 host  172.16.30.5 eq 23
access-list 110 permit ip any any
\end{verbatim}
\item
  \texttt{line\ vty\ 0\ 4}

\begin{verbatim}
access-class 110 in
\end{verbatim}
\item
  \texttt{ip\ access-list\ standard\ No172Net}

\begin{verbatim}
deny 172.16.0.0 0.0.255.255
permit any
\end{verbatim}
\item
  \texttt{ip\ access-group\ No172Net\ out}
\item
  \texttt{show\ ip\ interfaces}
\end{enumerate}

\subsection[Chapter 13: Network Address Translation
(NAT)]{\texorpdfstring{\protect\hypertarget{b01.xhtmlux5cux23bapp01-sec-29}{}{}\protect\hypertarget{b01.xhtmlux5cux23Page_992}{}{}Chapter
13: Network Address Translation
(NAT)}{Chapter 13: Network Address Translation (NAT)}}

\begin{enumerate}
\tightlist
\item
  Port Address Translation (PAT), also called NAT Overload
\item
  \texttt{debug\ ip\ nat}
\item
  \texttt{show\ ip\ nat\ translations}
\item
  \texttt{clear\ ip\ nat\ translations\ *}
\item
  Before
\item
  After
\item
  \texttt{show\ ip\ nat\ statistics}
\item
  The \texttt{ip\ nat\ inside} and \texttt{ip\ nat\ outside} commands
\item
  Dynamic NAT
\item
  \texttt{prefix-length}
\end{enumerate}

\subsection[Chapter 14: Internet Protocol Version 6
(IPv6)]{\texorpdfstring{\protect\hypertarget{b01.xhtmlux5cux23bapp01-sec-30}{}{}Chapter
14: Internet Protocol Version 6
(IPv6)}{Chapter 14: Internet Protocol Version 6 (IPv6)}}

\subsubsection[Written Lab 14.1: IPv6
Foundation]{\texorpdfstring{\protect\hypertarget{b01.xhtmlux5cux23bapp01-sec-31}{}{}Written
Lab 14.1: IPv6 Foundation}{Written Lab 14.1: IPv6 Foundation}}

\begin{enumerate}
\tightlist
\item
  128 and 129
\item
  33-33-FF-17-FC-0F
\item
  Link-local
\item
  Link-local
\item
  Multicast
\item
  Anycast
\item
  OSPFv3
\item
  ::1
\item
  FE80::/10
\item
  FF02::2
\end{enumerate}

\subsubsection[Written Lab 14.2: EUI-64
Format]{\texorpdfstring{\protect\hypertarget{b01.xhtmlux5cux23bapp01-sec-32}{}{}Written
Lab 14.2: EUI-64 Format}{Written Lab 14.2: EUI-64 Format}}

\begin{enumerate}
\tightlist
\item
  2001:db8:1:1:090c:abff:fecd:1234
\item
  2001:db8:1:1:040c:32ff:fef1:a4d2
\item
  2001:db8:1:1:12:abff:fecd:1234
\item
  2001:db8:1:1:0f01:3aff:fe2f:1234
\item
  2001:db8:1:1:080c:abff:feac:caba
\end{enumerate}

\subsection[Chapter 15: Enhanced Switched
Technologies]{\texorpdfstring{\protect\hypertarget{b01.xhtmlux5cux23bapp01-sec-33}{}{}Chapter
15: Enhanced Switched
Technologies}{Chapter 15: Enhanced Switched Technologies}}

\subsubsection[Written Lab
15]{\texorpdfstring{\protect\hypertarget{b01.xhtmlux5cux23bapp01-sec-34}{}{}Written
Lab 15}{Written Lab 15}}

\begin{enumerate}
\tightlist
\item
  PAgP
\item
  \texttt{show\ spanning-tree\ summary}
\item
  802.1w
\item
  STP
\item
  BPDU Guard
\item
  \texttt{(config-if)\#\ spanning-tree\ portfast}
\item
  \texttt{Switch\#\ show\ etherchannel\ port-channel}
\item
  \texttt{Switch(config)\#\ spanning-tree\ vlan\ 3\ root\ primary}
\item
  \texttt{show\ spanning-tree}, then follow the root port that connects
  to the root bridge using CDP, or
  \texttt{show\ spanning-tree\ summary}.
\item
  Active and passive
\end{enumerate}

\subsection[Chapter 16: Network Device Management and
Security]{\texorpdfstring{\protect\hypertarget{b01.xhtmlux5cux23bapp01-sec-35}{}{}\protect\hypertarget{b01.xhtmlux5cux23Page_993}{}{}Chapter
16: Network Device Management and
Security}{Chapter 16: Network Device Management and Security}}

\subsubsection[Written Lab
16]{\texorpdfstring{\protect\hypertarget{b01.xhtmlux5cux23bapp01-sec-36}{}{}Written
Lab 16}{Written Lab 16}}

\begin{enumerate}
\tightlist
\item
  INFORM
\item
  GET
\item
  TRAP
\item
  SET
\item
  WALK
\item
  If the active router fails, the standby router takes over with a
  different virtual IP address, and different to the one configured as
  the default-gateway address for end devices, so your hosts stop
  working which defeats the purpose of a FHRP.
\item
  You'll start receiving duplicate IP address warnings.
\item
  You'll start receiving duplicate IP address warnings.
\item
  In version 1, HSRP messages are sent to the multicast IP address
  224.0.0.2 and UDP port 1985. HSRP version 2 uses the multicast IP
  address 224.0.0.102 and UDP port 1985.
\item
  RADIUS and TACACS+, with TACACS+ being proprietary.
\end{enumerate}

\subsection[Chapter 17: Enhanced
IGRP]{\texorpdfstring{\protect\hypertarget{b01.xhtmlux5cux23bapp01-sec-37}{}{}Chapter
17: Enhanced IGRP}{Chapter 17: Enhanced IGRP}}

\subsubsection[Written Lab
17]{\texorpdfstring{\protect\hypertarget{b01.xhtmlux5cux23bapp01-sec-38}{}{}Written
Lab 17}{Written Lab 17}}

\begin{enumerate}
\tightlist
\item
  \texttt{ipv6\ router\ eigrp}\emph{as}
\item
  FF02::A
\item
  False
\item
  The routers will not form an adjacency.
\item
  Passive interface
\item
  A backup route, stored in the topology table
\end{enumerate}

\subsection[Chapter 18: Open Shortest Path First
(OSPF)]{\texorpdfstring{\protect\hypertarget{b01.xhtmlux5cux23bapp01-sec-39}{}{}\protect\hypertarget{b01.xhtmlux5cux23Page_994}{}{}Chapter
18: Open Shortest Path First
(OSPF)}{Chapter 18: Open Shortest Path First (OSPF)}}

\subsubsection[Written Lab
18]{\texorpdfstring{\protect\hypertarget{b01.xhtmlux5cux23bapp01-sec-40}{}{}Written
Lab 18}{Written Lab 18}}

\begin{enumerate}
\tightlist
\item
  \texttt{router\ ospf\ 101}
\item
  \texttt{show\ ip\ ospf}
\item
  \texttt{show\ ip\ ospf\ interface}
\item
  \texttt{show\ ip\ ospf\ neighbor}
\item
  \texttt{show\ ip\ route\ ospf}
\item
  Cost (bandwidth)
\item
  Areas don't match. The routers are not in same subnet. RIDs are the
  same. Hello and Dead timers don't match.
\item
  \texttt{show\ ip\ ospf\ database}
\item
  110
\item
  10 and 40
\end{enumerate}

\subsection[Chapter 19: Multi-Area
OSPF]{\texorpdfstring{\protect\hypertarget{b01.xhtmlux5cux23bapp01-sec-41}{}{}Chapter
19: Multi-Area OSPF}{Chapter 19: Multi-Area OSPF}}

\subsubsection[Written Lab
19]{\texorpdfstring{\protect\hypertarget{b01.xhtmlux5cux23bapp01-sec-42}{}{}Written
Lab 19}{Written Lab 19}}

\begin{enumerate}
\tightlist
\item
  Type 5 or Type 7
\item
  2WAY
\item
  Type 3, and possibly Type 4 and 5
\item
  When all LSAs have synchronized with a neighbor on a point-to-point
  link
\item
  True
\item
  EXCHANGE
\item
  Type 1
\item
  \texttt{ipv6\ ospf\ 1\ area\ 0}
\item
  \protect\hypertarget{b01.xhtmlux5cux23Page_995}{}{}OSPFv2 and v3 use
  the same items when forming an adjacency; Hello and Dead timers,
  subnet info, and area ID all must match. Authentication must also
  match if configured.
\item
  \texttt{show\ ip\ protocols} and \texttt{show\ ipv6\ protocols}
\end{enumerate}

\subsection[Chapter 20: Troubleshooting IP, IPv6, and
VLANs]{\texorpdfstring{\protect\hypertarget{b01.xhtmlux5cux23bapp01-sec-43}{}{}Chapter
20: Troubleshooting IP, IPv6, and
VLANs}{Chapter 20: Troubleshooting IP, IPv6, and VLANs}}

\subsubsection[Written Lab
20]{\texorpdfstring{\protect\hypertarget{b01.xhtmlux5cux23bapp01-sec-44}{}{}Written
Lab 20}{Written Lab 20}}

\begin{enumerate}
\tightlist
\item
  The INCMP is an incomplete message, which means a neighbor
  solicitation message has been sent but the neighbor message has not
  yet been received.
\item
  \texttt{switchport\ trunk\ native\ vlan\ 66}
\item
  Access, auto, desirable, nonegotiate, and trunk (on)
\item
  Verify that the default gateway is correct. Verify that name
  resolution settings are correct. Verify that there are no ACLs
  blocking traffic.
\item
  \texttt{ping\ ::1}
\end{enumerate}

\subsection[Chapter 21: Wide Area
Networks]{\texorpdfstring{\protect\hypertarget{b01.xhtmlux5cux23bapp01-sec-45}{}{}Chapter
21: Wide Area Networks}{Chapter 21: Wide Area Networks}}

\subsubsection[Written Lab
21]{\texorpdfstring{\protect\hypertarget{b01.xhtmlux5cux23bapp01-sec-46}{}{}Written
Lab 21}{Written Lab 21}}

\begin{enumerate}
\tightlist
\item
  True
\item
  False
\item
  BGP
\item
  \texttt{show\ ip\ bgp\ neighbor}
\item
  \texttt{show\ run\ interface\ tunnel\ tunnel\_number}
\item
  False
\item
  PPPoE or PPPoA
\item
  HDLC, LCP, and NCP
\item
  VPLS, VPWS
\item
  Layer 2 MPLS VPN, Layer 3 MPLS VPN
\end{enumerate}

\subsection[Chapter 22: Evolution of Intelligent
Networks]{\texorpdfstring{\protect\hypertarget{b01.xhtmlux5cux23bapp01-sec-47}{}{}\protect\hypertarget{b01.xhtmlux5cux23Page_996}{}{}Chapter
22: Evolution of Intelligent
Networks}{Chapter 22: Evolution of Intelligent Networks}}

\subsubsection[Written Lab
22]{\texorpdfstring{\protect\hypertarget{b01.xhtmlux5cux23bapp01-sec-48}{}{}Written
Lab 22}{Written Lab 22}}

\begin{enumerate}
\tightlist
\item
  DSCP
\item
  Control, data
\item
  TID
\item
  Markings, addressing, and application signatures
\item
  Marking
\item
  Policing
\item
  Data, control, application
\item
  CBWFQ, LLW
\item
  NBAR
\item
  Northbound
\end{enumerate}

\protect\hypertarget{b02.xhtml}{}{}

\section[{Appendix B}\\
{Answers to Review
Questions}]{\texorpdfstring{\protect\hypertarget{b02.xhtmlux5cux23bapp02}{}{}\protect\hypertarget{b02.xhtmlux5cux23Page_997}{}{}{Appendix
B}\\
{Answers to Review Questions}}{Appendix B Answers to Review Questions}}

\subsection[Chapter 1:
Internetworking]{\texorpdfstring{\protect\hypertarget{b02.xhtmlux5cux23bapp02-sec-1}{}{}\protect\hypertarget{b02.xhtmlux5cux23Page_998}{}{}Chapter
1: Internetworking}{Chapter 1: Internetworking}}

\begin{enumerate}
\item
  A. The device shown is a hub and hubs place all ports in the same
  broadcast domain and the same collision domain.
\item
  B. The contents of a protocol data unit (PDU) depend on the PDU
  because they are created in a specific order and their contents are
  based on that order. A packet will contain IP addresses but not MAC
  addresses because MAC addresses are not present until the PDU becomes
  a frame.
\item
  C. You should select a router to connect the two groups. When
  computers are in different subnets, as these two groups are, you will
  require a device that can make decisions based on IP addresses.
  Routers operate at layer 3 of the Open Systems Interconnect (OSI)
  model and make data-forwarding decisions based on layer 3 networking
  information, which are IP addresses. They create routing tables that
  guide them in forwarding traffic out of the proper interface to the
  proper subnet.
\item
  C. Replacing the hub with a switch would reduce collisions and
  retransmissions, which would have the most impact on reducing
  congestion.
\item
  Answer:

  \begin{figure}
  \centering
  \includegraphics{images/a02f001.jpg}
  \caption{}
  \end{figure}

  The given layers of the OSI model use the PDUs shown in the above
  diagram.
\item
  B. Wireless LAN Controllers are used to manage anywhere from a few
  access points to thousands. The AP's are completely managed from the
  controller and are considered lightweight or dumb AP's as they have no
  configuration on the AP itself.
\item
  B. You should use a switch to accomplish the task in this scenario. A
  switch is used to provide dedicated bandwidth to each node by
  eliminating the possibility of collisions on the switch port where the
  node resides. Switches work at layer 2 in the Open Systems
  Interconnection (OSI) model and perform the function of separating
  collision domains.
\item
  \begin{figure}
  \centering
  \includegraphics{images/a02f002.jpg}
  \caption{}
  \end{figure}

  The listed layers of the OSI model have the functions shown in the
  diagram above.
\item
  \protect\hypertarget{b02.xhtmlux5cux23Page_999}{}{}C. Firewalls are
  used to connect our trusted internal network such as the DMZ, to the
  untrusted outside network---typically the internet.
\item
  D. The Application layer is responsible for identifying and
  establishing the availability of the intended communication partner
  and determining whether sufficient resources for the intended
  communication exist.
\item
  A, D. The Transport layer segments data into smaller pieces for
  transport. Each segment is assigned a sequence number so that the
  receiving device can reassemble the data on arrival. The Network layer
  (layer 3) has two key responsibilities. First, this layer controls the
  logical addressing of devices. Second, the Network layer determines
  the best path to a particular destination network and routes the data
  appropriately.
\item
  C. The IEEE Ethernet Data Link layer has two sublayers, the Media
  Access Control (MAC) layer and the Logical Link Control (LLC) layer.
\item
  C. Wireless AP's are very popular today and will be going away about
  the same time that rock n' roll does. The idea behind these devices
  (which are layer 2 bridge devices) is to connect wireless products to
  the wired Ethernet network. The wireless AP will create a single
  collision domain and is typically its own dedicated broadcast domain
  as well.
\item
  A. Hubs operate on the Physical Layer as they have no intelligence and
  send all traffic in all directions.
\item
  C. While it is true that the OSI model's primary purpose is to allow
  different vendors' networks to interoperate, there is no requirement
  that vendors follow the model.
\item
  A. Routers by default do NOT forward broadcasts.
\item
  C. Switches create separate collision domains within a single
  broadcast domain. Routers provide a separate broadcast domain for each
  interface.
\item
  B. The all-hub network at the bottom is one collision domain; the
  bridge network on top equals three collision domains. Add in the
  switch network of five collision domains---one for each switch
  port---and you get a total of nine.
\item
  A. The top three layers define how the applications within the end
  stations will communicate with each other as well as with users.
\item
  A. The following network devices operate at all seven layers of the
  OSI model: network management stations (NMSs), gateways (not default
  gateways), servers, and network hosts.
\end{enumerate}

\subsection[Chapter 2: Ethernet Networking and Data
Encapsulation]{\texorpdfstring{\protect\hypertarget{b02.xhtmlux5cux23bapp02-sec-2}{}{}\protect\hypertarget{b02.xhtmlux5cux23Page_1000}{}{}Chapter
2: Ethernet Networking and Data
Encapsulation}{Chapter 2: Ethernet Networking and Data Encapsulation}}

\begin{enumerate}
\item
  D. The organizationally unique identifier (OUI) is assigned by the
  IEEE to an organization composed of 24 bits, or 3 bytes, which in turn
  assigns a globally administered address also comprising 24 bits, or 3
  bytes, that's supposedly unique to each and every adapter it
  manufactures.
\item
  A. Backoff on an Ethernet network is the retransmission delay that's
  enforced when a collision occurs. When that happens, a host will only
  resume transmission after the forced time delay has expired. Keep in
  mind that after the backoff has elapsed, all stations have equal
  priority to transmit data.
\item
  A. When using a hub, all ports are in the same collision domain, which
  will introduce collisions as shown between devices connected to the
  same hub.
\item
  B. FCS is a field at the end of the frame that's used to store the
  cyclic redundancy check (CRC) answer. The CRC is a mathematical
  algorithm that's based on the data in the frame and run when each
  frame is built. When a receiving host receives the frame and runs the
  CRC, the answer should be the same. If not, the frame is discarded,
  assuming errors have occurred.
\item
  C. Half-duplex Ethernet networking uses a protocol called Carrier
  Sense Multiple Access with Collision Detection (CSMA/CD), which helps
  devices share the bandwidth evenly while preventing two devices from
  transmitting simultaneously on the same network medium.
\item
  A, E. Physical addresses or MAC addresses are used to identify devices
  at layer 2. MAC addresses are only used to communicate on the same
  network. To communicate on different network, we have to use layer 3
  addresses (IP addresses).
\item
  D. The cable shown is a straight-through cable, which is used between
  dissimilar devices.
\item
  C, D. An Ethernet network is a shared environment, so all devices have
  the right to access the medium. If more than one device transmits
  simultaneously, the signals collide and cannot reach the destination.
  If a device detects another device is sending, it will wait for a
  specified amount of time before attempting to transmit.

  When there is no traffic detected, a device will transmit its message.
  While this transmission is occurring, the device continues to listen
  for traffic or collisions on the LAN. After the message is sent, the
  device returns to its default listening mode.
\item
  \protect\hypertarget{b02.xhtmlux5cux23Page_1001}{}{}B. In creating the
  gigabit crossover cable, you'd still cross 1 to 3 and 2 to 6, but you
  would add 4 to 7 and 5 to 8.
\item
  D. When you set up the connection, use these settings:

  \begin{enumerate}
  \tightlist
  \item
    Bits per sec: 9600
  \item
    Data bits: 8
  \item
    Parity: None
  \item
    Stop bits: 1
  \item
    Flow control: None
  \end{enumerate}
\item
  D. When set to 0, this bit represents a globally administered address,
  as specified by the IEEE, but when it's a 1, it represents a locally
  governed and administered address.
\item
  B. You can use a rolled Ethernet cable to connect a host EIA-TIA 232
  interface to a router console serial communication (COM) port.
\item
  B. The collision will invoke a backoff algorithm on all systems, not
  just the ones involved in the collision.
\item
  A. There are no collisions in full-duplex mode.
\item
  B. The connection between the two switches requires a crossover and
  the connection from the hosts to the switches requires a
  straight-through.
\item
  The given cable types are matched with their standards in the
  following table.

  \begin{longtable}[]{@{}ll@{}}
  \toprule
  \endhead
  IEEE 802.3u & 100Base-Tx\tabularnewline
  IEEE 802.3 & 10Base-T\tabularnewline
  IEEE 802.3ab & 1000Base-T\tabularnewline
  IEEE 802.3z & 1000Base-SX\tabularnewline
  \bottomrule
  \end{longtable}
\item
  B. Although rolled cable isn't used to connect any Ethernet
  connections together, you can use a rolled Ethernet cable to connect a
  host EIA-TIA 232 interface to a router console serial communication
  (COM) port.
\item
  B. If you're using TCP, the virtual circuit is defined by the source
  and destination port number plus the source and destination IP address
  and called a \emph{socket}.
\item
  A. The hex value 1c is converted as 28 in decimal.
\item
  A. Fiber-optic cables are the only ones that have a core surrounded by
  a material called cladding.
\end{enumerate}

\subsection[Chapter 3: Introduction to
TCP/IP]{\texorpdfstring{\protect\hypertarget{b02.xhtmlux5cux23bapp02-sec-3}{}{}\protect\hypertarget{b02.xhtmlux5cux23Page_1002}{}{}Chapter
3: Introduction to TCP/IP}{Chapter 3: Introduction to TCP/IP}}

\begin{enumerate}
\item
  C. If a DHCP conflict is detected, either by the server sending a ping
  and getting a response or by a host using a gratuitous ARP (arp'ing
  for its own IP address and seeing if a host responds), then the server
  will hold that address and not use it again until it is fixed by an
  administrator.
\item
  B. Secure Shell (SSH) protocol sets up a secure session that's similar
  to Telnet over a standard TCP/IP connection and is employed for doing
  things like logging into systems, running programs on remote systems,
  and moving files from one system to another.
\item
  C. A host uses something called a gratuitous ARP to help avoid a
  possible duplicate address. The DHCP client sends an ARP broadcast out
  on the local LAN or VLAN using its newly assigned address to help
  solve conflicts before they occur.
\item
  B. Address Resolution Protocol (ARP) is used to find the hardware
  address from a known IP address.
\item
  A, C, D. The listed answers are from the OSI model and the question
  asked about the TCP/IP protocol stack (DoD model). Yes, it is normal
  for the objectives to have this type of question. However, let's just
  look for what is wrong. First, the Session layer is not in the TCP/IP
  model; neither are the Data Link and Physical layers. This leaves us
  with the Transport layer (Host-to-Host in the DoD model), Internet
  layer (Network layer in the OSI), and Application layer
  (Application/Process in the DoD). Remember, the CCENT objectives can
  list the layers as OSI layers or DoD layers at any time, regardless of
  what the question is asking.
\item
  C. A Class C network address has only 8 bits for defining hosts:
  2\textsuperscript{8} -- 2 = 256.
\item
  A, B. A client that sends out a DHCP Discover message in order to
  receive an IP address sends out a broadcast at both layer 2 and layer
  3. The layer 2 broadcast is all \emph{F}s in hex, or
  FF:FF:FF:FF:FF:FF. The layer 3 broadcast is 255.255.255.255, which
  means any networks and all hosts. DHCP is connectionless, which means
  it uses User Datagram Protocol (UDP) at the Transport layer, also
  called the Host-to-Host layer.
\item
  B. Although Telnet does use TCP and IP (TCP/IP), the question
  specifically asks about layer 4, and IP works at layer 3. Telnet uses
  TCP at layer 4.
\item
  RFC 1918. These addresses can be used on a private network, but
  they're not routable through the Internet.
\item
  B, D, E. SMTP, FTP, and HTTP use TCP.
\item
  C. Class C addresses devote 24 bits to the network portion and 8 bits
  to the host portion.
\item
  C. The range of multicast addresses starts with 224.0.0.0 and goes
  through 239.255.255.255.
\item
  C. First, you should know easily that only TCP and UDP work at the
  Transport layer, so now you have a 50/50 shot. However, since the
  header has sequencing, acknowledgment, and window numbers, the answer
  can only be TCP.
\item
  \protect\hypertarget{b02.xhtmlux5cux23Page_1003}{}{} A. Both FTP and
  Telnet use TCP at the Transport layer; however, they both are
  Application layer protocols, so the Application layer is the best
  answer for this question.
\item
  C. The four layers of the DoD model are Application/Process,
  Host-to-Host, Internet, and Network Access. The Internet layer is
  equivalent to the Network layer of the OSI model.
\item
  C, E. The Class A private address range is 10.0.0.0 through
  10.255.255.255. The Class B private address range is 172.16.0.0
  through 172.31.255.255, and the Class C private address range is
  192.168.0.0 through 192.168.255.255.
\item
  B. The four layers of the TCP/IP stack (also called the DoD model) are
  Application/Process, Host-to-Host (also called Transport on the
  objectives), Internet, and Network Access/Link. The Host-to-Host layer
  is equivalent to the Transport layer of the OSI model.
\item
  B, C. ICMP is used for diagnostics and destination unreachable
  messages. ICMP is encapsulated within IP datagrams, and because it is
  used for diagnostics, it will provide hosts with information about
  network problems.
\item
  C. The range of a Class B network address is 128--191. This makes our
  binary range 10\emph{xxxxxx}.
\item
  \begin{longtable}[]{@{}l@{}}
  \toprule
  \endhead
  Answer\tabularnewline
  DHCPDiscover\tabularnewline
  DHCPOffer\tabularnewline
  DHCPRequest\tabularnewline
  DHCPAck\tabularnewline
  \bottomrule
  \end{longtable}

  The steps are as shown in the answer diagram.
\end{enumerate}

\subsection[Chapter 4: Easy
Subnetting]{\texorpdfstring{\protect\hypertarget{b02.xhtmlux5cux23bapp02-sec-4}{}{}Chapter
4: Easy Subnetting}{Chapter 4: Easy Subnetting}}

\begin{enumerate}
\tightlist
\item
  D. A /27 (255.255.255.224) is 3 bits on and 5 bits off. This provides
  8 subnets, each with 30 hosts. Does it matter if this mask is used
  with a Class A, B, or C network address? Not at all. The number of
  subnet bits would never change.
\item
  D. A 240 mask is 4 subnet bits and provides 16 subnets, each with 14
  hosts. We need more subnets, so let's add subnet bits. One more subnet
  bit would be a 248 mask. This provides 5 subnet bits (32 subnets) with
  3 host bits (6 hosts per subnet). This is the best answer.
\item
  C. This is a pretty simple question. A /28 is 255.255.255.240, which
  means that our block size is 16 in the fourth octet. 0, 16, 32, 48,
  64, 80, etc. The host is in the 64 subnet.
\item
  C. A CIDR address of /19 is 255.255.224.0. This is a Class B address,
  so that is only 3 subnet bits, but it provides 13 host bits, or 8
  subnets, each with 8,190 hosts.
\item
  \protect\hypertarget{b02.xhtmlux5cux23Page_1004}{}{}B, D. The mask
  255.255.254.0 (/23) used with a Class A address means that there are
  15 subnet bits and 9 host bits. The block size in the third octet is 2
  (256 -- 254). So this makes the subnets in the interesting octet 0, 2,
  4, 6, etc., all the way to 254. The host 10.16.3.65 is in the 2.0
  subnet. The next subnet is 4.0, so the broadcast address for the 2.0
  subnet is 3.255. The valid host addresses are 2.1 through 3.254.
\item
  D. A /30, regardless of the class of address, has a 252 in the fourth
  octet. This means we have a block size of 4 and our subnets are 0, 4,
  8, 12, 16, etc. Address 14 is obviously in the 12 subnet.
\item
  D. A point-to-point link uses only two hosts. A /30, or
  255.255.255.252, mask provides two hosts per subnet.
\item
  C. A /21 is 255.255.248.0, which means we have a block size of 8 in
  the third octet, so we just count by 8 until we reach 66. The subnet
  in this question is 64.0. The next subnet is 72.0, so the broadcast
  address of the 64 subnet is 71.255.
\item
  A. A /29 (255.255.255.248), regardless of the class of address, has
  only 3 host bits. Six is the maximum number of hosts on this LAN,
  including the router interface.
\item
  C. A /29 is 255.255.255.248, which is a block size of 8 in the fourth
  octet. The subnets are 0, 8, 16, 24, 32, 40, etc. 192.168.19.24 is the
  24 subnet, and since 32 is the next subnet, the broadcast address for
  the 24 subnet is 31. 192.168.19.26 is the only correct answer.
\item
  A. A /29 (255.255.255.248) has a block size of 8 in the fourth octet.
  This means the subnets are 0, 8, 16, 24, etc. 10 is in the 8 subnet.
  The next subnet is 16, so 15 is the broadcast address.
\item
  B. You need 5 subnets, each with at least 16 hosts. The mask
  255.255.255.240 provides 16 subnets with 14 hosts---this will not
  work. The mask 255.255.255.224 provides 8 subnets, each with 30 hosts.
  This is the best answer.
\item
  C. First, you cannot answer this question if you can't subnet. The
  192.168.10.62 with a mask of 255.255.255.192 is a block size of 64 in
  the fourth octet. The host 192.168.10.62 is in the zero subnet, and
  the error occurred because \texttt{ip\ subnet-zero} is not enabled on
  the router.
\item
  A. A /25 mask is 255.255.255.128. Used with a Class B network, the
  third and fourth octets are used for subnetting with a total of 9
  subnet bits, 8 bits in the third octet and 1 bit in the fourth octet.
  Since there is only 1 bit in the fourth octet, the bit is either off
  or on---which is a value of 0 or 128. The host in the question is in
  the 0 subnet, which has a broadcast address of 127 since 112.128 is
  the next subnet.
\item
  A. A /28 is a 255.255.255.240 mask. Let's count to the ninth subnet
  (we need to find the broadcast address of the eighth subnet, so we
  need to count to the ninth subnet). Starting at 16 (remember, the
  question stated that we will not use subnet zero, so we start at 16,
  not 0), we have 16, 32, 48, 64, 80, 96, 112, 128, 144, etc. The eighth
  subnet is 128 and the next subnet is 144, so our broadcast address of
  the 128 subnet is 143. This makes the host range 129--142. 142 is the
  last valid host.
\item
  \protect\hypertarget{b02.xhtmlux5cux23Page_1005}{}{}C. A /28 is a
  255.255.255.240 mask. The first subnet is 16 (remember that the
  question stated not to use subnet zero) and the next subnet is 32, so
  our broadcast address is 31. This makes our host range 17--30. 30 is
  the last valid host.
\item
  B. We need 9 host bits to answer this question, which is a /23.
\item
  E. A Class B network ID with a /22 mask is 255.255.252.0, with a block
  size of 4 in the third octet. The network address in the question is
  in subnet 172.16.16.0 with a broadcast address of 172.16.19.255. Only
  option E has the correct subnet mask listed, and 172.16.18.255 is a
  valid host.
\item
  D, E. The router's IP address on the E0 interface is 172.16.2.1/23,
  which is 255.255.254.0. This makes the third octet a block size of 2.
  The router's interface is in the 2.0 subnet, and the broadcast address
  is 3.255 because the next subnet is 4.0. The valid host range is 2.1
  through 3.254. The router is using the first valid host address in the
  range.
\item
  A. For this example, the network range is 172.16.16.1 to
  172.16.31.254, the network address is 172.16.16.0, and the broadcast
  IP address is 172.16.31.255.
\end{enumerate}

\subsection[Chapter 5: VLSMs, Summarization, and Troubleshooting
TCP/IP]{\texorpdfstring{\protect\hypertarget{b02.xhtmlux5cux23bapp02-sec-5}{}{}Chapter
5: VLSMs, Summarization, and Troubleshooting
TCP/IP}{Chapter 5: VLSMs, Summarization, and Troubleshooting TCP/IP}}

\begin{enumerate}
\item
  D. A point-to-point link uses only two hosts. A /30, or
  255.255.255.252, mask provides two hosts per subnet.
\item
  C. Using a /28 mask, there are 4 bits available for hosts.
  Two-to-the-fourth power minus 2 = 14, or block size −2.
\item
  D. For 6 hosts we need to leave 3 bits in the host portion since 2 to
  the third power = 8 and 8 minus 2 is 6. With 3 bits for the host
  portion, that leaves 29 bits for the mask, or /29.
\item
  C. To use VLSM, the routing protocols in use possess the capability to
  transmit subnet mask information.
\item
  D. In a question like this, you need to look for an interesting octet
  where you can combine networks. In this example, the third octet has
  all our subnets, so we just need to find our block size now. If we
  used a block of 8 starting at 172.16.0.0/19, then we cover 172.16.0.0
  through 172.16.7.255. However, if we used 172.16.0.0/20, then we'd
  cover a block of 16, which would be from 172.16.0.0 through
  172.16.15.255, which is the best answer.
\item
  C. The IP address of the station and the gateway are not in the same
  network. Since the address of the gateway is correct on the station,
  it is \emph{most likely} the IP address of the station is incorrect.
\item
  B. With an incorrect gateway, Host A will not be able to communicate
  with the router or beyond the router but will be able to communicate
  within the subnet.
\item
  \protect\hypertarget{b02.xhtmlux5cux23Page_1006}{}{}A. Pinging the
  remote computer would fail if any of the other steps fail.
\item
  C. When a ping to the local host IP address fails, you can assume the
  NIC is not functional.
\item
  C, D. If a ping to the local host succeeds, you can rule out IP stack
  or NIC failure.
\item
  E. A /29 mask yields only 6 addresses, so none of the networks could
  use it.
\item
  A. The most likely problem if you can ping a computer by IP address
  but not by name is a failure of DNS.
\item
  D. When you issue the \texttt{ping} command, you are using the ICMP
  protocol.
\item
  B. The \texttt{traceroute} command displays the networks traversed on
  a path to a network destination.
\item
  C. The \texttt{ping} command tests connectivity to another station.
  The full command is shown below.

\begin{verbatim}
C:\>ping 172.16.10.2
Pinging 172.16.10.2 with 32 bytes of data:
Reply from 172.16.10.2: bytes=32 time<1ms TTL=128
Reply from 172.16.10.2: bytes=32 time<1ms TTL=128
Reply from 172.16.10.2: bytes=32 time<1ms TTL=128
Reply from 172.16.10.2: bytes=32 time<1ms TTL=128
Ping statistics for 172.16.10.2:
    Packets: Sent = 4, Received = 4, Lost = 0 (0% loss),
Approximate round trip times in milli-seconds:
    Minimum = 0ms, Maximum = 0ms, Average = 0ms
\end{verbatim}
\item
  \begin{longtable}[]{@{}ll@{}}
  \toprule
  \endhead
  traceroute & Displays the list of routers on a path to a network
  destination\tabularnewline
  arp -a & Displays IP-to-MAC-address mappings on a Windows
  PC\tabularnewline
  show ip arp & Displays the ARP table on a Cisco router\tabularnewline
  ipconfig /all & Shows you the PC network configuration\tabularnewline
  \bottomrule
  \end{longtable}

  The commands use the functions described in the answer table.
\item
  C. The interesting octet in this example is the second octet, and it
  is a block size of four starting at 10.0.0.0. By using a 255.252.0.0
  mask, we are telling the summary to use a block size of four in the
  second octet. This will cover 10.0.0.0 through 10.3.255.255. This is
  the best answer.
\item
  A. The command that displays the ARP table on a Cisco router is
  \texttt{show\ ip\ arp}.
\item
  C. The \texttt{/all} switch must be added to the \texttt{ipconfig}
  command on a PC to verify DNS configuration.
\item
  \protect\hypertarget{b02.xhtmlux5cux23Page_1007}{}{}C. If you start at
  192.168.128.0 and go through 192.168.159.0, you can see this is a
  block of 32 in the third octet. Since the network address is always
  the first one in the range, the summary address is 192.168.128.0. What
  mask provides a block of 32 in the third octet? The answer is
  255.255.224.0, or /19.
\end{enumerate}

\subsection[Chapter 6: Cisco's Internetworking Operating System
(IOS)]{\texorpdfstring{\protect\hypertarget{b02.xhtmlux5cux23bapp02-sec-6}{}{}Chapter
6: Cisco's Internetworking Operating System
(IOS)}{Chapter 6: Cisco's Internetworking Operating System (IOS)}}

\begin{enumerate}
\item
  D. Typically, we'd see the input errors and CRC statistics increase
  with a duplex error, but it could be another Physical layer issue such
  as the cable might be receiving excessive interference or the network
  interface cards might have a failure. Typically, you can tell if it is
  interference when the CRC and input errors output grow but the
  collision counters do not, which is the case with this question.
\item
  C. Once the IOS is loaded and up and running, the startup-config will
  be copied from NVRAM into RAM and from then on, referred to as the
  running-config.
\item
  C, D. To configure SSH on your router, you need to set the username
  command, the \texttt{ip\ domain-name}, \texttt{login\ local}, and the
  \texttt{transport\ input\ ssh} under the VTY lines and the
  \texttt{crypto\ key} command. However, SSH version 2 is suggested but
  not required.
\item
  C. The \texttt{show\ controllers\ serial\ 0/0} command will show you
  whether either a DTE or DCE cable is connected to the interface. If it
  is a DCE connection, you need to add clocking with the
  \texttt{clock\ rate} command.
\item
  \begin{figure}
  \centering
  \includegraphics{images/a02f005.jpg}
  \caption{}
  \end{figure}

  User exec mode is limited to basic monitoring commands; privileged
  exec mode provides access to all other router commands. Specific
  configuration modes include the commands that affect a specific
  interface or process, while global configuration mode allows commands
  that affect the entire system. Setup mode is where you access the
  interactive configuration dialog.
\item
  B. The bandwidth shown is 100000 kbits a second, which is a
  FastEthernet port, or 100 Mbs.
\item
  \protect\hypertarget{b02.xhtmlux5cux23Page_1008}{}{}B. From global
  configuration mode, use the \texttt{line\ vty\ 0\ 4} command to set
  all five default VTY lines. However, you would typically always set
  all lines, not just the defaults.
\item
  C. The enable secret password is case sensitive, so the second option
  is wrong. To set the enable secret password, use the
  \texttt{enable\ secret\ password} command from global configuration
  mode. This password is automatically encrypted.
\item
  C. The banner motd sets a message of the day for administrators when
  they login to a switch or router.
\item
  C. The prompts offered as options indicate the following modes:

\begin{verbatim}
Switch(config)# is global configuration mode.
Switch> is user mode.
Switch# is privileged mode.
Switch(config-if)# is interface configuration mode.
\end{verbatim}
\item
  D. To copy the running-config to NVRAM so that it will be used if the
  router is restarted, use the
  \texttt{copy\ running-config\ startup-config} command in privileged
  mode (\texttt{copy\ run\ start} for short).
\item
  D. To allow a VTY (Telnet) session into your router, you must set the
  VTY password. Option C is wrong because it is setting the password on
  the wrong router. Notice that you have to set the password before you
  set the login command.
\item
  C. Wireless AP's are very popular today and will be going away about
  the same time that rock n' roll does. The idea behind these devices
  (which are layer 2 bridge devices) is to connect wireless products to
  the wired Ethernet network. The wireless AP will create a single
  collision domain and is typically its own dedicated broadcast domain
  as well.
\item
  B. If an interface is shut down, the \texttt{show\ interface} command
  will show the interface as administratively down. (It is possible that
  no cable is attached, but you can't tell that from this message.)
\item
  C. With the \texttt{show\ interfaces} command, you can view the
  configurable parameters, get statistics for the interfaces on the
  switch, check for input and CRC errors, and verify if the interfaces
  are shut down.
\item
  C. If you delete the startup-config and reload the switch, the device
  will automatically enter setup mode. You can also type \texttt{setup}
  from privileged mode at any time.
\item
  D. You can view the interface statistics from user mode, but the
  command is \texttt{show\ interface\ fastethernet\ 0/0}.
\item
  B. The \texttt{\%\ ambiguous\ command} error means that there is more
  than one possible \texttt{show} command that starts with \emph{r}. Use
  a question mark to find the correct command.
\item
  B, D. The commands \texttt{show\ interfaces} and
  \texttt{show\ ip\ interface} will show you the layer 1 and 2 status
  and the IP addresses of your router's interfaces.
\item
  \protect\hypertarget{b02.xhtmlux5cux23Page_1009}{}{}A. If you see that
  a serial interface and the protocol are both down, then you have a
  Physical layer problem. If you see
  \texttt{serial1\ is\ up,\ \ line\ protocol\ is\ down}, then you are
  not receiving (Data Link) keepalives from the remote end.
\end{enumerate}

\subsection[Chapter 7: Managing a Cisco
Internetwork]{\texorpdfstring{\protect\hypertarget{b02.xhtmlux5cux23bapp02-sec-7}{}{}Chapter
7: Managing a Cisco
Internetwork}{Chapter 7: Managing a Cisco Internetwork}}

\begin{enumerate}
\tightlist
\item
  B. The IEEE created a new standardized discovery protocol called
  802.1AB for Station and Media Access Control Connectivity Discovery.
  We'll just call it Link Layer Discovery Protocol (LLDP).
\item
  C. The \texttt{show\ processes} (or \texttt{show\ processes\ cpu}) is
  a good tool for determining a given router's CPU utilization. When it
  is high, it is not a good time to execute a debug command.
\item
  B. The command \texttt{traceroute} (\texttt{trace} for short), which
  can be issued from user mode or privileged mode, is used to find the
  path a packet takes through an internetwork and will also show you
  where the packet stops because of an error on a router.
\item
  C. Since the configuration looks correct, you probably didn't screw up
  the copy job. However, when you perform a copy from a network host to
  a router, the interfaces are automatically shut down and need to be
  manually enabled with the \texttt{no\ shutdown} command.
\item
  D. Specifying the address of the DHCP server allows the router to
  relay broadcast traffic destined for a DHCP server to that server.
\item
  C. Before you start to configure the router, you should erase the
  NVRAM with the \texttt{erase\ startup-config} command and then reload
  the router using the \texttt{reload} command.
\item
  C. This command can be run on both routers and switches and it
  displays detailed information about each device connected to the
  device you're running the command on, including the IP address.
\item
  C. The Port ID column describes the interfaces on the remote device
  end of the connection.
\item
  B. Syslog levels range from 0--7, and level 7 (known as Debugging or
  local7) is the default if you were to use the
  \texttt{logging\ ip\_address} command from global config.
\item
  C. If you save a configuration and reload the router and it comes up
  either in setup mode or as a blank configuration, chances are the
  configuration register setting is incorrect.
\item
  D. To keep open one or more Telnet sessions, use the Ctrl+Shift+6 and
  then X keystroke combination.
\item
  B, D. The best answers, the ones you need to remember, are that either
  an access control list is filtering the Telnet session or the VTY
  password is not set on the remote device.
\item
  \protect\hypertarget{b02.xhtmlux5cux23Page_1010}{}{}A, D. The
  \texttt{show\ hosts} command provides information on temporary DNS
  entries and permanent name-to-address mappings created using the
  \texttt{ip\ host} command.
\item
  A, B, D. The \texttt{tracert} command is a Windows command and will
  not work on a router or switch! IOS uses the \texttt{traceroute}
  command.
\item
  D. By default, Cisco IOS devices use facility local7. Moreover, most
  Cisco devices provide options to change the facility level from their
  default value.
\item
  C. To see console messages through your Telnet session, you must enter
  the \texttt{terminal\ monitor} command.
\item
  C, D, F. There are significantly more syslog messages available within
  IOS as compared to SNMP Trap messages. System logging is a method of
  collecting messages from devices to a server running a syslog daemon.
  Logging to a central syslog server helps in aggregation of logs and
  alerts.
\item
  E. Although option A is certainly the ``best'' answer, unfortunately
  option E will work just fine and your boss would probably prefer you
  to use the \texttt{show\ cdp\ neighbors\ detail} command.
\item
  D. To enable a device to be an NTP client, use the
  \texttt{ntp\ server\ IP\_address\ version\ number} command at global
  configuration mode. That's all there is to it! Assuming your NTP
  server is working of course.
\item
  B, D, F. If you specify a level with the ``logging trap \emph{level}''
  command, that level and all the higher levels will be logged. For
  example, by using the \texttt{logging\ trap\ 3} command, emergencies,
  alerts, critical, and error messages will be logged. Only three of
  these were listed as possible options.
\end{enumerate}

\subsection[Chapter 8: Managing Cisco
Devices]{\texorpdfstring{\protect\hypertarget{b02.xhtmlux5cux23bapp02-sec-8}{}{}Chapter
8: Managing Cisco Devices}{Chapter 8: Managing Cisco Devices}}

\begin{enumerate}
\tightlist
\item
  B. The default configuration setting is 0x2102, which tells the router
  to load the IOS from flash and the configuration from NVRAM. 0x2142
  tells the router to bypass the configuration in NVRAM so that you can
  perform password recovery.
\item
  E. To copy the IOS to a backup host, which is stored in flash memory
  by default, use the \texttt{copy\ flash\ tftp} command.
\item
  B. To install a new license on an ISR G2 router, use the
  \texttt{license\ install\ url} command.
\item
  C. The configuration register provides the boot commands, and 0x2101
  tells the router to boot the mini-IOS, if found, and not to load a
  file from flash memory. Many newer routers do not have a mini-IOS, so
  as an alternative, the router would end up in ROM monitor mode if the
  mini-IOS is not found. However, option C is the best answer for this
  question.
\item
  B. The \texttt{show\ flash} command will provide you with the current
  IOS name and size and the size of flash memory.
\item
  \protect\hypertarget{b02.xhtmlux5cux23Page_1011}{}{}C. Before you
  start to configure the router, you should erase the NVRAM with the
  \texttt{erase\ startup-config} command and then reload the router
  using the \texttt{reload} command.
\item
  D. The command \texttt{copy\ tftp\ flash} will allow you to copy a new
  IOS into flash memory on your router.
\item
  C. The best answer is \texttt{show\ version}, which shows you the IOS
  file running currently on your router. The \texttt{show\ flash}
  command shows you the contents of flash memory, not which file is
  running.
\item
  C. All Cisco routers have a default configuration register setting of
  0x2102, which tells the router to load the IOS from flash memory and
  the configuration from NVRAM.
\item
  C. If you save a configuration and reload the router and it comes up
  either in setup mode or as a blank configuration, chances are the
  configuration register setting is incorrect.
\item
  D. The \texttt{license\ boot\ module} command installs a Right-To-Use
  license feature on a router.
\item
  A. The \texttt{show\ license} command determines the licenses that are
  active on your system. It also displays a group of lines for each
  feature in the currently running IOS image along with several status
  variables related to software activation and licensing, both licensed
  and unlicensed features.
\item
  B. The \texttt{show\ license\ feature} command allows you to view the
  technology package licenses and feature licenses that are supported on
  your router along with several status variables related to software
  activation and licensing, both licensed and unlicensed features.
\item
  C. The \texttt{show\ license\ udi} command displays the unique device
  identifier (UDI) of the router, which comprises the product ID (PID)
  and serial number of the router.
\item
  D. The \texttt{show\ version} command displays various pieces of
  information about the current IOS version, including the licensing
  details at the end of the command's output.
\item
  C. The \texttt{license\ save\ flash} command allows you to back up
  your license to flash memory.
\item
  C. The \texttt{show\ version} command provides you with the current
  configuration register setting.
\item
  C, D. The two steps to remove a license are to first disable the
  technology package and then clear the license.
\item
  B, D, E. Before you back up an IOS image to a laptop directly
  connected to a router's Ethernet port, make sure that the TFTP server
  software is running on your laptop, that the Ethernet cable is a
  ``crossover,'' and that the laptop is in the same subnet as the
  router's Ethernet port, and then you can use the
  \texttt{copy\ flash\ tftp} command from your laptop.
\item
  C. The default configuration setting of 0x2102 tells the router to
  look in NVRAM for the boot sequence.
\end{enumerate}

\subsection[Chapter 9: IP
Routing]{\texorpdfstring{\protect\hypertarget{b02.xhtmlux5cux23bapp02-sec-9}{}{}\protect\hypertarget{b02.xhtmlux5cux23Page_1012}{}{}Chapter
9: IP Routing}{Chapter 9: IP Routing}}

\begin{enumerate}
\item
  \texttt{show\ ip\ route}

  The \texttt{ip\ route} command is used to display the routing table of
  a router.
\item
  B. In the new 15 IOS code, Cisco defines a different route called a
  local route. Each has a /32 prefix defining a route just for the one
  address, which is the router's interface.
\item
  A, B. Although option D almost seems right, it is not; the mask option
  is the mask used on the remote network, not the source network. Since
  there is no number at the end of the static route, it is using the
  default administrative distance of 1.
\item
  C, F. The switches are not used as either a default gateway or other
  destination. Switches have nothing to do with routing. It is very
  important to remember that the destination MAC address will always be
  the router's interface. The destination address of a frame, from
  HostA, will be the MAC address of the Fa0/0 interface of RouterA. The
  destination address of a packet will be the IP address of the network
  interface card (NIC) of the HTTPS server. The destination port number
  in the segment header will have a value of 443 (HTTPS).
\item
  B. This mapping was learned dynamically, which means it was learned
  through ARP.
\item
  B. Hybrid protocols use aspects of both distance vector and link
  state---for example, EIGRP. Be advised, however, that Cisco typically
  just calls EIGRP an advanced distance-vector routing protocol. Do not
  be misled by the way the question is worded. Yes, I know that MAC
  addresses are not in a packet. You must read the question to
  understand of what it is really asking.
\item
  A. Since the destination MAC address is different at each hop, it must
  keep changing. The IP address, which is used for the routing process,
  does not.
\item
  B, E. Classful routing means that all hosts in the internetwork use
  the same mask and that only default masks are in use. Classless
  routing means that you can use variable length subnet masks (VLSMs).
\item
  B, C. The distance-vector routing protocol sends its complete routing
  table out of all active interfaces at periodic time intervals.
  Link-state routing protocols send updates containing the state of
  their own links to all routers in the internetwork.
\item
  C. This is how most people see routers, and certainly they could do
  this type of plain ol' packet switching in 1990 when Cisco released
  their very first router and traffic was seriously slow, but not in
  today's networks! This process involves looking up every destination
  in the routing table and finding the exit interface for every packet.
\item
  A, C. The \texttt{S*} shows that this is a candidate for default route
  and that it was configured manually.
\item
  B. RIP has an administrative distance (AD) of 120, while EIGRP has an
  administrative distance of 90, so the router will discard any route
  with a higher AD than 90 to that same network.
\item
  \protect\hypertarget{b02.xhtmlux5cux23Page_1013}{}{}D. Recovery from a
  lost route requires manual intervention by a human to replace the lost
  route.
\item
  A. RIPv1 and RIPv2 only use the lowest hop count to determine the best
  path to a remote network.
\item
  A. Since the routing table shows no route to the 192.168.22.0 network,
  the router will discard the packet and send an ICMP destination
  unreachable message out of interface FastEthernet 0/0, which is the
  source LAN from which the packet originated.
\item
  C. Static routes have an administrative distance of 1 by default.
  Unless you change this, a static route will always be used over any
  other dynamically learned route. EIGRP has an administrative distance
  of 90, and RIP has an administrative distance of 120, by default.
\item
  C. BGP is the only EGP listed.
\item
  A, B, C. Recovery from a lost route requires manual intervention by a
  human to replace the lost route. The advantages are less overhead on
  the router and network as well as more security.
\item
  C. The \texttt{show\ ip\ interface\ brief} command displays a concise
  summary of the interfaces.
\item
  B. The 150 at the end changes the default administrative distance (AD)
  of 1 to 150.
\end{enumerate}

\subsection[Chapter 10: Layer 2
Switching]{\texorpdfstring{\protect\hypertarget{b02.xhtmlux5cux23bapp02-sec-10}{}{}Chapter
10: Layer 2 Switching}{Chapter 10: Layer 2 Switching}}

\begin{enumerate}
\item
  A. Layer 2 switches and bridges are faster than routers because they
  don't take up time looking at the Network Layer header information.
  They do make use of the Data Link layer information.
\item
  \texttt{mac\ address-table\ static\ aaaa.bbbb.cccc\ vlan\ 1\ int\ fa0/7}

  You can set a static MAC address in the MAC address table, and when
  done, it will appear as a static entry in the table.
\item
  B, D, E. Since the MAC address is not present in the table, it will
  send the frame out of all ports in the same VLAN with the exception of
  the port on which it was received.
\item
  \texttt{show\ mac\ address-table}

  This command displays the forward filter table, also called a Content
  Addressable Memory (CAM) table.
\item
  \begin{figure}
  \centering
  \includegraphics{images/a02f006.jpg}
  \caption{}
  \end{figure}

  The three functions are address learning, forward/filter decisions,
  and loop avoidance.
\item
  \protect\hypertarget{b02.xhtmlux5cux23Page_1014}{}{}A, D. In the
  output shown, you can see that the port is in \texttt{Secure-shutdown}
  mode and the light for the port would be amber. To enable the port
  again, you'd need to do the following:

\begin{verbatim}
S3(config-if)#shutdown
S3(config-if)#no shutdown
\end{verbatim}
\item
  \texttt{switchport\ port-security\ maximum\ 2}

  The maximum setting of 2 means only two MAC addresses can be used on
  that port; if the user tries to add another host on that segment, the
  switch port will take the action specified. In the
  \texttt{port-security\ violation}command.
\item
  B. The \texttt{switchport\ port-security} command enables port
  security, which is a prerequisite for the other commands to function.
\item
  B. Gateway redundancy is not an issue addressed by STP.
\item
  A. If no loop avoidance schemes are put in place, the switches will
  flood broadcasts endlessly throughout the internetwork. This is
  sometimes referred to as a broadcast storm.
\item
  B, C. Shutdown and protect mode will alert you via SNMP that a
  violation has occurred on a port.
\item
  Spanning Tree Protocol (STP) STP is a switching loop avoidance scheme
  use by switches.
\item
  \texttt{ip\ default-gateway}

  If you want to manage your switches from outside your LAN, you need to
  set a default gateway on the switches, just as you would with a host.
\item
  C. The IP address is configured under a logical interface, called a
  management domain or VLAN 1.
\item
  B. The \texttt{show\ port-security\ interface} command displays the
  current port security and status of a switch port, as in this sample
  output:

\begin{verbatim}
Switch# show port-security interface fastethernet0/1
Port Security: Enabled
Port status: SecureUp
Violation mode: Shutdown
Maximum MAC Addresses: 2
Total MAC Addresses: 2
Configured MAC Addresses: 2
Aging Time: 30 mins
Aging Type: Inactivity
SecureStatic address aging: Enabled
Security Violation count: 0
\end{verbatim}
\item
  \texttt{switchport\ port-security\ mac-address\ sticky}

  Issuing the \texttt{switchport\ port-security\ mac-address\ sticky}
  command will allow a switch to save a dynamically learned MAC address
  in the running-configuration of the switch, which prevents the
  administrator from having to document or configure specific MAC
  addresses.
\item
  \protect\hypertarget{b02.xhtmlux5cux23Page_1015}{}{} B, D. To limit
  connections to a specific host, you should configure the MAC address
  of the host as a static entry associated with the port, although be
  aware that this host can still connect to any other port, but no other
  port can connect to F0/3, in this example. Another solution would be
  to configure port security to accept traffic only from the MAC address
  of the host. By default, an unlimited number of MAC addresses can be
  learned on a single switch port, whether it is configured as an access
  port or a trunk port. Switch ports can be secured by defining one or
  more specific MAC addresses that should be allowed to connect and by
  defining violation policies (such as disabling the port) to be enacted
  if additional hosts try to gain a connection.
\item
  D. The command statically defines the MAC address of 00c0.35F0.8301 as
  an allowed host on the switch port. By default, an unlimited number of
  MAC addresses can be learned on a single switch port, whether it is
  configured as an access port or a trunk port. Switch ports can be
  secured by defining one or more specific MAC addresses that should be
  allowed to connect, and violation policies (such as disabling the
  port) if additional hosts try to gain a connection.
\item
  D. You would not make the port a trunk. In this example, this
  switchport is a member of one VLAN. However, you can configure port
  security on a trunk port, but again, that's not valid for this
  question.
\item
  \texttt{switchport\ port-security\ violation\ shutdown}

  This command is used to set the reaction of the switch to a port
  violation of shutdown.
\end{enumerate}

\subsection[Chapter 11: VLANs and InterVLAN
Routing]{\texorpdfstring{\protect\hypertarget{b02.xhtmlux5cux23bapp02-sec-11}{}{}Chapter
11: VLANs and InterVLAN
Routing}{Chapter 11: VLANs and InterVLAN Routing}}

\begin{enumerate}
\item
  D. Here's a list of ways VLANs simplify network management:

  \begin{enumerate}
  \tightlist
  \item
    Network adds, moves, and changes are achieved with ease by just
    configuring a port into the appropriate VLAN.
  \item
    A group of users that need an unusually high level of security can
    be put into its own VLAN so that users outside of the VLAN can't
    communicate with them.
  \item
    As a logical grouping of users by function, VLANs can be considered
    independent from their physical or geographic locations.
  \item
    VLANs greatly enhance network security if implemented correctly.
  \item
    VLANs increase the number of broadcast domains while decreasing
    their size.
  \end{enumerate}
\item
  \texttt{ip\ routing}

  Routing must be enabled on the layer 3 switch.
\item
  C. VLANs can span across multiple switches by using trunk links, which
  carry traffic for multiple VLANs.
\item
  \protect\hypertarget{b02.xhtmlux5cux23Page_1016}{}{}B. While in all
  other cases access ports can be a member of only one VLAN, most
  switches will allow you to add a second VLAN to an access port on a
  switch port for your voice traffic; it's called the voice VLAN. The
  voice VLAN used to be called the auxiliary VLAN, which allowed it to
  be overlaid on top of the data VLAN, enabling both types of traffic
  through the same port.
\item
  A. Yes, you have to do a \texttt{no\ shutdown} on the VLAN interface.
\item
  C. Unlike ISL which encapsulates the frame with control information,
  802.1q inserts an 802.1q field along with tag control information.
\item
  D. Instead of using a router interface for each VLAN, you can use one
  FastEthernet interface and run ISL or 802.1q trunking. This allows all
  VLANs to communicate through one interface. Cisco calls this a
  ``router on a stick.''
\item
  \texttt{switchport\ access\ vlan\ 2}

  This command is executed under the interface (switch port) that is
  being placed in the VLAN.
\item
  \texttt{show\ vlan}

  After you create the VLANs that you want, you can use the
  \texttt{show\ vlan} command to check them out.
\item
  B. The \texttt{encapsulation} command specifying the VLAN for the
  subinterface must be present under both subinterfaces.
\item
  A. With a multilayer switch, enable IP routing and create one logical
  interface for each VLAN using the \texttt{interface\ vlan\ number}
  command and you're now doing inter-VLAN routing on the backplane of
  the switch!
\item
  A. Ports Fa0/15--18 are not present in any VLANs. They are trunk
  ports.
\item
  C. Untagged frames are members of the native VLAN, which by default is
  VLAN 1.
\item
  \texttt{sh\ interfaces\ fastEthernet\ 0/15\ switchport}

  This \texttt{show\ interfaces\ interface\ switchport} command shows us
  the administrative mode of dynamic desirable and that the port is a
  trunk port, DTP was used to negotiate the frame tagging method of ISL,
  and the native VLAN is the default of 1
\item
  C. A VLAN is a broadcast domain on a layer 2 switch. You need a
  separate address space (subnet) for each VLAN. There are four VLANs,
  so that means four broadcast domains/subnets.
\item
  B. The host's default gateway should be set to the IP address of the
  subinterface that is associated with the VLAN of which the host is a
  member, in this case VLAN 2.
\item
  C. Frame tagging is used when VLAN traffic travels over a trunk link.
  Trunk links carry frames for multiple VLANs. Therefore, frame tags are
  used for identification of frames from different VLANs.
\item
  \texttt{vlan\ 2}

  To configure VLANs on a Cisco Catalyst switch, use the global config
  \texttt{vlan} command.
\item
  B. 802.1q uses the native VLAN.
\item
  \texttt{switchport\ nonegotiate}

  You can use this command only when the interface switchport mode is
  access or trunk. You must manually configure the neighboring interface
  as a trunk interface to establish a trunk link.
\end{enumerate}

\subsection[Chapter 12:
Security]{\texorpdfstring{\protect\hypertarget{b02.xhtmlux5cux23bapp02-sec-12}{}{}\protect\hypertarget{b02.xhtmlux5cux23Page_1017}{}{}Chapter
12: Security}{Chapter 12: Security}}

\begin{enumerate}
\item
  D. It's compared with lines of the access list only until a match is
  made. Once the packet matches the condition on a line of the access
  list, the packet is acted upon and no further comparisons take place.
\item
  C. The range of 192.168.160.0 to 192.168.191.0 is a block size of 32.
  The network address is 192.168.160.0 and the mask would be
  255.255.224.0, which for an access list must be a wildcard format of
  0.0.31.255. The 31 is used for a block size of 32. The wildcard is
  always one less than the block size.
\item
  C. Using a named access list just replaces the number used when
  applying the list to the router's interface.
  \texttt{ip\ access-group\ Blocksales\ in} is correct.
\item
  B. The list must specify TCP as the Transport layer protocol and use a
  correct wildcard mask (in this case 0.0.0.255), and it must specify
  the destination port (80). It also should specify \texttt{any} as the
  set of computers allowed to have this access.
\item
  A. The first thing to check in a question like this is the access-list
  number. Right away, you can see that the second option is wrong
  because it is using a standard IP access-list number. The second thing
  to check is the protocol. If you are filtering by upper-layer
  protocol, then you must be using either UDP or TCP; this eliminates
  the fourth option. The third and last answers have the wrong syntax.
\item
  C. Of the available choices, only the \texttt{show\ ip\ interface}
  command will tell you which interfaces have access lists applied.
  \texttt{show\ access-lists} will not show you which interfaces have an
  access list applied.
\item
  \begin{figure}
  \centering
  \includegraphics{images/a02f007.jpg}
  \caption{}
  \end{figure}

  The command \texttt{show\ access-list} displays all access lists and
  their parameters configured on the router; it does not show you which
  interface the list is set on. \texttt{show\ access-list\ 110} shows
  only the parameters for the access list 110 and, again, does not tell
  you which interface the list is set on. \texttt{show\ ip\ access-list}
  reveals only the IP access lists configured on the router. Finally,
  \texttt{show\ ip\ interface} shows which interfaces have access lists
  set.

  The functions of each command are as shown in the solution graphic.
\item
  \protect\hypertarget{b02.xhtmlux5cux23Page_1018}{}{} C. The extended
  access list ranges are 100--199 and 2000--2699, so the access-list
  number of 100 is valid. Telnet uses TCP, so the protocol TCP is valid.
  Now you just need to look for the source and destination address. Only
  the third option has the correct sequence of parameters. Option B may
  work, but the question specifically states ``only'' to network
  192.168.10.0, and the wildcard in option B is too broad.
\item
  D. Extended IP access lists use numbers 100--199 and 2000--2699 and
  filter based on source and destination IP address, protocol number,
  and port number. The last option is correct because of the second line
  that specifies \texttt{permit\ ip\ any\ any}. (I used
  \texttt{0.0.0.0\ 255.255.255.255}, which is the same as the
  \texttt{any} option.) The third option does not have this, so it would
  deny access but not allow everything else.
\item
  D. First, you must know that a /20 is 255.255.240.0, which is a block
  size of 16 in the third octet. Counting by 16s, this makes our subnet
  48 in the third octet, and the wildcard for the third octet would be
  15 since the wildcard is always one less than the block size.
\item
  B. To find the wildcard (inverse) version of this mask, the zero and
  one bits are simply reversed as follows:

  111111111111111.11111111.11100000 (27 one bits, or /27)

  00000000.00000000.00000000.00011111 (wildcard/inverse mask)
\item
  A. First, you must know that a /19 is 255.255.224.0, which is a block
  size of 32 in the third octet. Counting by 32s, this makes our subnet
  192 in the third octet, and the wildcard for the third octet would be
  31 since the wildcard is always one less than the block size.
\item
  B, D. The scope of an access list is determined by the wildcard mask
  and the network address to which it is applied. For example, in this
  case the starting point of the list of addresses affected by the mask
  is the network ID 192.111.16.32. The wildcard mask is 0.0.0.31. Adding
  the value of the last octet in the mask to the network address (32 +
  31 = 63) tells you where the effects of the access list ends, which is
  199.111.16.63. Therefore, all addresses in the range
  199.111.16.32--199.111.16.63 will be denied by this list.
\item
  C. To place an access list on an interface, use the
  \texttt{ip\ access-group} command in interface configuration mode.
\item
  B. With no permit statement, the ACL will deny all traffic.
\item
  D. If you add an access list to an interface and you do not have at
  least one permit statement, then you will effectively shut down the
  interface because of the implicit \texttt{deny\ any} at the end of
  every list.
\item
  C. Telnet access to the router is restricted by using either a
  standard or extended IP access list inbound on the VTY lines of the
  router. The command \texttt{access-class} is used to apply the access
  list to the VTY lines.
\item
  C. A Cisco router has rules regarding the placement of access lists on
  a router interface. You can place one access list per direction for
  each layer 3 protocol configured on an interface.
\item
  C. The most common attack on a network today is a denial of service
  (DoS) because it is the easiest attack to achieve.
\item
  \protect\hypertarget{b02.xhtmlux5cux23Page_1019}{}{}C. Implementing
  intrusion detection services and intrusion prevention services will
  help notify you and stop attacks in real time.
\end{enumerate}

\subsection[Chapter 13: Network Address Translation
(NAT)]{\texorpdfstring{\protect\hypertarget{b02.xhtmlux5cux23bapp02-sec-13}{}{}Chapter
13: Network Address Translation
(NAT)}{Chapter 13: Network Address Translation (NAT)}}

\begin{enumerate}
\tightlist
\item
  A, C, E. NAT is not perfect and can cause some issues in some
  networks, but most networks work just fine. NAT can cause delays and
  troubleshooting problems, and some applications just won't work.
\item
  B, D, F. NAT is not perfect, but there are some advantages. It
  conserves global addresses, which allow us to add millions of hosts to
  the Internet without ``real'' IP addresses. This provides flexibility
  in our corporate networks. NAT can also allow you to use the same
  subnet more than once in the same network without overlapping
  networks.
\item
  C. The command \texttt{debug\ ip\ nat} will show you in real time the
  translations occurring on your router.
\item
  A. The command \texttt{show\ ip\ nat\ translations} will show you the
  translation table containing all the active NAT entries.
\item
  D. The command \texttt{clear\ ip\ nat\ translations\ *} will clear all
  the active NAT entries in your translation table.
\item
  B. The \texttt{show\ ip\ nat\ statistics} command displays a summary
  of the NAT configuration as well as counts of active translation
  types, hits to an existing mapping, misses (an attempt to create a
  mapping), and expired translations.
\item
  B. The command \texttt{ip\ nat\ pool\ name} creates the pool that
  hosts can use to get onto the global Internet. What makes option B
  correct is that the range 171.16.10.65 through 171.16.10.94 includes
  30 hosts, but the mask has to match 30 hosts as well, and that mask is
  255.255.255.224. Option C is wrong because there is a lowercase t in
  the pool name. Pool names are case sensitive.
\item
  A, C, E. You can configure NAT three ways on a Cisco router: static,
  dynamic, and NAT Overload (PAT).
\item
  B. Instead of the \texttt{netmask} command, you can use the
  \texttt{prefix-length\ length} statement.
\item
  C. In order for NAT to provide translation services, you must have
  \texttt{ip\ nat\ inside} and \texttt{ip\ nat\ outside} configured on
  your router's interfaces.
\item
  A, B, D. The most popular use of NAT is if you want to connect to the
  Internet and you don't want hosts to have global (real) IP addresses,
  but options B and D are correct as well.
\item
  \protect\hypertarget{b02.xhtmlux5cux23Page_1020}{}{}C. An inside
  global address is considered to be the IP address of the host on the
  private network after translation.
\item
  A. An inside local address is considered to be the IP address of the
  host on the private network before translation.
\item
  D. What we need to figure out for this question is only the inside
  global pool. Basically we start at 1.1.128.1 and end at 1.1.135.174;
  our block size is 8 in the third octet, or /21. Always look for your
  block size and the interesting octet and you can find your answer
  every time.
\item
  B. Once you create your pool, the command
  \texttt{ip\ nat\ inside\ source} must be used to say which inside
  locals are allowed to use the pool. In this question we need to see if
  access-list 100 is configured correctly, if at all, so
  \texttt{show\ access-list} is the best answer.
\item
  A. You must configure your interfaces before NAT will provide any
  translations. On the inside network interfaces, you would use the
  command \texttt{ip\ nat\ inside}. On the outside network interfaces,
  you will use the command \texttt{ip\ nat\ outside}.
\item
  B. You must configure your interfaces before NAT will provide any
  translations. On the inside networks you would use the command
  \texttt{ip\ nat\ inside}. On the outside network interfaces, you will
  use the command \texttt{ip\ nat\ outside}.
\item
  C. Another term for Port Address Translation is \emph{NAT Overload}
  because that is the keyword used to enable port address translation.
\item
  B. Fast-switching is used on Cisco routers to create a type of route
  cache in order to quickly forward packets through a router without
  having to parse the routing table for every packet. As packets are
  processed-switched (looked up in the routing table), this information
  is stored in the cache for later use if needed for faster routing
  processing.
\item
  B. Once you create a pool for the inside locals to use to get out to
  the global Internet, you must configure the command to allow them
  access to the pool. The
  \texttt{ip\ nat\ inside\ source\ list\ number\ pool-name\ overload}
  command has the correct sequence for this question.
\end{enumerate}

\subsection[Chapter 14: Internet Protocol Version 6
(IPv6)]{\texorpdfstring{\protect\hypertarget{b02.xhtmlux5cux23bapp02-sec-14}{}{}Chapter
14: Internet Protocol Version 6
(IPv6)}{Chapter 14: Internet Protocol Version 6 (IPv6)}}

\begin{enumerate}
\item
  D. The modified EUI-64 format interface identifier is derived from the
  48-bit link-layer (MAC) address by inserting the hexadecimal number
  FFFE between the upper 3 bytes (OUI field) and the lower 3 bytes
  (serial number) of the link layer address.
\item
  D. An IPv6 address is represented as eight groups of four hexadecimal
  digits, each group representing 16 bits (two octets). The groups are
  separated by colons (:). Option A has two double colons, B doesn't
  have 8 fields, and option C has invalid hex characters.
\item
  \protect\hypertarget{b02.xhtmlux5cux23Page_1021}{}{} A, B, C. This
  question is easier to answer if you just take out the wrong options.
  First, the loopback is only ::1, so that makes option D wrong. Link
  local is FE80::/10, not /8 and there are no broadcasts..
\item
  A, C, D. Several methods are used in terms of migration, including
  tunneling, translators, and dual-stack. Tunnels are used to carry one
  protocol inside another, while translators simply translate IPv6
  packets into IPv4 packets. Dual-stack uses a combination of both
  native IPv4 and IPv6. With dual-stack, devices are able to run IPv4
  and IPv6 together, and if IPv6 communication is possible, that is the
  preferred protocol. Hosts can simultaneously reach IPv4 and IPv6
  content.
\item
  A, B. ICMPv6 router advertisements use type 134 and must be at least
  64 bits in length.
\item
  B, E, F. Anycast addresses identify multiple interfaces, which is
  somewhat similar to multicast addresses; however, the big difference
  is that the anycast packet is only delivered to one address, the first
  one it finds defined in terms of routing distance. This address can
  also be called one-to-one-of-many, or one-to-nearest.
\item
  C. The loopback address with IPv4 is 127.0.0.1. With IPv6, that
  address is ::1.
\item
  B, C, E. An important feature of IPv6 is that it allows the
  plug-and-play option to the network devices by allowing them to
  configure themselves independently. It is possible to plug a node into
  an IPv6 network without requiring any human intervention. IPv6 does
  not implement traditional IP broadcasts.
\item
  A, D. The loopback address is ::1, link-local starts with FE80::/10,
  site-local addresses start with FEC0::/10, global addresses start with
  200::/3, and multicast addresses start with FF00::/8.
\item
  C. A router solicitation is sent out using the all-routers multicast
  address of FF02::2. The router can send a router advertisement to all
  hosts using the FF02::1 multicast address.
\item
  A, E. IPv6 does not use broadcasts, and autoconfiguration is a feature
  of IPV6 that allows for hosts to automatically obtain an IPv6 address.
\item
  A. The NDP neighbor advertisement (NA) contains the MAC address. A
  neighbor solicitation (NS) was initially sent asking for the MAC
  address.
\item
  B. IPv6 anycast addresses are used for one-to-nearest communication,
  meaning an anycast address is used by a device to send data to one
  specific recipient (interface) that is the closest out of a group of
  recipients (interfaces).
\item
  B, D. To shorten the written length of an IPv6 address, successive
  fields of zeros may be replaced by double colons. In trying to shorten
  the address further, leading zeros may also be removed. Just as with
  IPv4, a single device's interface can have more than one address; with
  IPv6 there are more types of addresses and the same rule applies.
  There can be link-local, global unicast, multicast, and anycast
  addresses all assigned to the same interface.
\item
  A, B, C. The Internet Header Length field was removed because it is no
  longer required. Unlike the variable-length IPv4 header, the IPv6
  header is fixed at 40 bytes. Fragmentation is processed differently in
  IPv6 and does not need the Flags field in the basic IPv4 header.

  \protect\hypertarget{b02.xhtmlux5cux23Page_1022}{}{}In IPv6, routers
  no longer process fragmentation; the host is responsible for
  fragmentation. The Header Checksum field at the IP layer was removed
  because most Data Link layer technologies already perform checksum and
  error control, which forces formerly optional upper-layer checksums
  (UDP, for example) to become mandatory.
\item
  B. There are no broadcasts with IPv6. Unicast, multicast, anycast,
  global, and link-local unicast are used.
\item
  D. This question asked how many bits in a field, not how many bits in
  an IPv6 address. There are 16 bits (four hex characters) in an IPv6
  field and there are eight fields.
\item
  A, D. Global addresses start with 2000::/3, link-locals start with
  FE80::/10, loopback is ::1, and unspecified is just two colons (::).
  Each interface will have a loopback address automatically configured.
\item
  B, C. If you verify your IP configuration on your host, you'll see
  that you have multiple IPv6 addresses, including a loopback address.
  The last 64 bits represent the dynamically created interface ID, and
  leading zeros are not mandatory in a 16-bit IPv6 field.
\item
  C. To enable IPv6 routing on the Cisco router, use the following
  command from global config:

\begin{verbatim}
ipv6 unicast-routing
\end{verbatim}

  If this command is not recognized, your version of IOS does not
  support IPv
\end{enumerate}

\subsection[Chapter 15: Enhanced Switched
Technologies]{\texorpdfstring{\protect\hypertarget{b02.xhtmlux5cux23bapp02-sec-15}{}{}Chapter
15: Enhanced Switched
Technologies}{Chapter 15: Enhanced Switched Technologies}}

\begin{enumerate}
\tightlist
\item
  B, D. The switch is not the root bridge for VLAN 1 or the output would
  tell us exactly that. The root bridge for VLAN 1 is off of interface
  G1/2 with a cost of 4, meaning it is directly connected. Use the
  command \texttt{show\ cdp\ nei} to find your root bridge at this
  point. Also, the switch is running RSTP (802.1w), not STP.
\item
  D. Option A seems like the best answer, and had switches not been
  configured with the primary and secondary command, then the switch
  configured with priority 4096 would have been root. However, since the
  primary and secondary both had a priority of 16384, then the tertiary
  switch would be a switch with a higher priority in this case.
\item
  A, D. It is important that you can find your root bridge, and the
  \texttt{show\ spanning-tree} command will help you do this. To quickly
  find out which VLANs your switch is the root bridge for, use the
  \texttt{show\ spanning-tree\ summary} command.
\item
  A. 802.1w is the also called Rapid Spanning Tree Protocol. It is not
  enabled by default on Cisco switches, but it is a better STP to run
  because it has all the fixes that the Cisco extensions provide with
  802.1d. Remember, Cisco runs RSTP PVST+, not just RSTP.
\item
  \protect\hypertarget{b02.xhtmlux5cux23Page_1023}{}{}B. The Spanning
  Tree Protocol is used to stop switching loops in a layer 2 switched
  network with redundant paths.
\item
  C. Convergence occurs when all ports on bridges and switches have
  transitioned to either the forwarding or blocking states. No data is
  forwarded until convergence is complete. Before data can be forwarded
  again, all devices must be updated.
\item
  C, E. There are two types of EtherChannel: Cisco's PAgP and the IEEE's
  LACP. They are basically the same, and there is little difference to
  configuring them. For PAgP, use auto or desirable mode, and with LACP
  use passive or active. These modes decide which method you are using,
  and they must be configured the same on both sides of the EtherChannel
  bundle.
\item
  A, B, F. RSTP helps with convergence issues that plague traditional
  STP. Rapid PVST+ is based on the 802.1w standard in the same way that
  PVST+ is based on 802.1d. The operation of Rapid PVST+ is simply a
  separate instance of 802.1w for each VLAN.
\item
  D. BPDU Guard is used when a port is configured for PortFast, or it
  should be used, because if that port receives a BPDU from another
  switch, BPDU Guard will shut that port down to stop a loop from
  occurring.
\item
  C. To allow for the PVST+ to operate, there's a field inserted into
  the BPDU to accommodate the extended system ID so that PVST+ can have
  a root bridge configured on a per-STP instance. The extended system ID
  (VLAN ID) is a 12-bit field, and we can even see what this field is
  carrying via the \texttt{show\ spanning-tree} command output.
\item
  C. PortFast and BPDU Guard allow a port to transition to the
  forwarding state quickly, which is great for a switch port but not for
  load balancing. You can somewhat load balance with RSTP, but that is
  out of the scope of our objectives, and although you can use PPP to
  configure multilink (bundle links), this is performed on asynchronous
  or synchronous serial links. Cisco's EtherChannel can bundle up to
  eight ports between switches.
\item
  D. If the Spanning Tree Protocol is not running on your switches and
  you connect them together with redundant links, you will have
  broadcast storms and multiple frame copies being received by the same
  destination device.
\item
  B, C, E. All the ports on both sides of every link must be configured
  exactly the same or it will not work. Speed, duplex, and allowed VLANs
  must match.
\item
  D, F. There are two types of EtherChannel: Cisco's PAgP and the IEEE's
  LACP. They are basically the same, and there is little difference to
  configure them. For PAgP, use the auto or desirable mode, and with
  LACP use the passive or active mode. These modes decide which method
  you are using, and they must be configured the same on both sides of
  the EtherChannel bundle.
\item
  D. You can't answer this question if you don't know who the root
  bridge is. SC has a bridge priority of 4,096, so that is the root
  bridge. The cost for SB was 4, with the direct link, but that link
  went down. If SB goes through SA to SC, the cost would be 4 + 19, or
  23. If SB goes to SA to SD to SC, the cost is 4 + 4 + 4 = 12.
\item
  \protect\hypertarget{b02.xhtmlux5cux23Page_1024}{}{}A, D. To configure
  EtherChannel, create the port channel from global configuration mode,
  and then assign the group number on each interface using the active
  mode to enable LACP. Just configuring the \texttt{channel-group}
  command under your interfaces will enable the bundle, but options A
  and D are the best Cisco objective answers.
\item
  A, D. You can set the priority to any value from 0 through 61,440 in
  increments of 4,096. Setting it to zero (0) means that the switch will
  always be a root as long as it has a lower MAC than another switch
  with its bridge ID also set to 0. You can also force a switch to be a
  root for a VLAN with the \texttt{spanning-tree\ vlan\ vlan\ primary}
  command.
\item
  A. By using per-VLAN spanning tree, the root bridge can be placed in
  the center of where all the resources are for a particular VLAN, which
  enables optimal path determination.
\item
  A, C, D, E. Each 802.1d port transitions through blocking, listening,
  learning, and finally forwarding after 50 seconds, by default. RSTP
  uses discarding, learning, and forwarding only.
\item
  A, C, D, E, F. The roles a switch port can play in STP are root,
  non-root, designated, non-designated, forwarding, and blocking.
  Discarding is used in RSTP, and disabled could be a role, but it's not
  listed as a possible answer.
\end{enumerate}

\subsection[Chapter 16: Network Device Management and
Security]{\texorpdfstring{\protect\hypertarget{b02.xhtmlux5cux23bapp02-sec-16}{}{}Chapter
16: Network Device Management and
Security}{Chapter 16: Network Device Management and Security}}

\begin{enumerate}
\tightlist
\item
  B. You can enter the ACL directly in the SNMP configuration to provide
  security, using either a number or a name.
\item
  C. 100. By setting a higher number then the default on a router, you
  are making that router the active router. Setting preempt would assure
  that if the active router went down, it would become the active router
  again when it comes back up.
\item
  D. To enable the AAA commands on a router or switch, use the global
  configuration command \texttt{aaa\ new-model}.
\item
  A, C. To mitigate access layer threats, use port security, DHCP
  snooping, dynamic ARP inspection, and identity based networking.
\item
  D. DHCP snooping validates DHCP messages, builds and maintains the
  DHCP snooping binding database, and rate-limits DHCP traffic for
  trusted and untrusted source.
\item
  A, D. TACACS+ uses TCP, is Cisco proprietary, and offers multiprotocol
  support as well as separated AAA services.
\item
  B. Unlike with TACACS+, separating AAA services is not an option when
  configuring RADIUS.
\item
  A, D. With a read-only community string, no changes can be made to the
  router. However, SNMPv2c can use GETBULK to create and return multiple
  requests at once.
\item
  \protect\hypertarget{b02.xhtmlux5cux23Page_1025}{}{}C. The idea of a
  first hop redundancy protocol is to provide redundancy for a default
  gateway.
\item
  A, B. A router interface can be in many states with HSRP; the states
  are shown in Table 2.1.
\item
  A. Only option A has the correct sequence to enable HSRP on an
  interface.
\item
  D. This is a question that I used in a lot of job interviews on
  prospects. The \texttt{show\ standby} command is your friend when
  dealing with HSRP.
\item
  D. There is nothing wrong with leaving the priorities at the defaults
  of 100. The first router up will be the active router.
\item
  C. In version 1, HSRP messages are sent to the multicast IP address
  224.0.0.2 and UDP port 1985. HSRP version 2 uses the multicast IP
  address 224.0.0.102 and UDP port 1985.
\item
  B, C. If HSRP1 is configured to preempt, then it will become active
  because of the higher priority; if not, HSRP2 will stay the active
  router.
\item
  C. In version 1, HSRP messages are sent to the multicast IP address
  224.0.0.2 and UDP port 1985. HSRP version 2 uses and the multicast IP
  address 224.0.0.102 and UDP port 1985.
\item
  C, D. SNMPv2c introduced the GETBULK and INFORM SNMP messages but
  didn't have any different security than SNMPv1. SNMPv3 uses TCP and
  provides encryption and authentication.
\item
  D. The correct answer is option D. Take your newly created RADIUS
  group and use it for authentication, and be sure to use the keyword
  \texttt{local} at the end.
\item
  B. DAI, used with DHCP snooping, tracks IP-to-MAC bindings from DHCP
  transactions to protect against ARP poisoning. DHCP snooping is
  required in order to build the MAC-to-IP bindings for DAI validation.
\item
  A, D, E. There are three roles involved in using client/server access
  control for identity-based networking on wired and wireless hosts: The
  client, also referred to as a supplicant, is software that runs on a
  client and is 802.1x compliant. The authenticator is typically a
  switch that controls physical access to the network and is a proxy
  between the client and the authentication server. The authentication
  server (RADIUS) is a server that authenticates each client before it
  can access any services.
\end{enumerate}

\subsection[Chapter 17: Enhanced
IGRP]{\texorpdfstring{\protect\hypertarget{b02.xhtmlux5cux23bapp02-sec-17}{}{}Chapter
17: Enhanced IGRP}{Chapter 17: Enhanced IGRP}}

\begin{enumerate}
\tightlist
\item
  B. Only the EIGRP routes will be placed in the routing table because
  it has the lowest administrative distance (AD), and that is always
  used before metrics.
\item
  A, C. EIGRP maintains three tables in RAM: neighbor, topology, and
  routing. The neighbor and topology tables are built and maintained
  with the use of Hello and update packets.
\item
  B. EIGRP does use reported distance, or advertised distance (AD), to
  tell neighbor routers the cost to get to a remote network. This router
  will send the FD to the neighbor router and the neighbor router will
  add the cost to get to this router plus the AD to find the true FD.
\item
  \protect\hypertarget{b02.xhtmlux5cux23Page_1026}{}{}E. Successor
  routes are going to be in the routing table since they are the best
  path to a remote network. However, the topology table has a link to
  each and every network, so the best answer is topology table and
  routing table. Any secondary route to a remote network is considered a
  feasible successor, and those routes are found only in the topology
  table and used as backup routes in case of primary route failure.
\item
  C. Any secondary route to a remote network is considered a feasible
  successor, and those routes are found only in the topology table and
  used as backup routes in case of primary route failure. You can see
  the topology table with the \texttt{show\ ip\ eigrp\ topology}
  command.
\item
  B, C, E. EIGRP and EIGRPv6 routers can use the same RID, unlike OSPF,
  and this can be set with the \texttt{eigrp\ router-id} command. Also a
  \texttt{variance} can be set to provide unequal-cost load balancing,
  along with the \texttt{maximum-paths} command to set the amount of
  load-balanced paths.
\item
  C. There were two successor routes, so by default, EIGRP was
  load-balancing out s0/0 and s0/1. When s0/0 goes down, EIGRP will just
  keep forwarding traffic out the second link s0/1. s0/0 will be removed
  from the routing table.
\item
  D. To enable EIGRPv6 on a router interface, use the command
  \texttt{ipv6\ eigrp\ as} on individual interfaces that will be part of
  the EIGRPv6 process.
\item
  C. The path to network 10.10.50.0 out serial0/0 is more than two times
  the current FD, so I used a \texttt{variance\ 3} command to
  load-balance unequal-cost links three times the FD.
\item
  B, C. First, a maximum hop count of 16 only is associated with RIP,
  and EIGRP never broadcasts, so we can eliminate A and D as options.
  Feasible successors are backup routes and stored in the topology
  table, so that is correct, and if no feasible successor is located,
  the EIGRP will flood its neighbors asking for a new path to network
  10.10.10.0.
\item
  D. The \texttt{show\ ip\ eigrp\ neighbors} command allows you to check
  the IP addresses as well as the retransmit interval and queue counts
  for the neighbors that have established an adjacency.
\item
  C, E. For EIGRP to form an adjacency with a neighbor, the AS numbers
  must match, and the metric K values must match as well. Also, option F
  could cause the problem; we can see if it is causing a problem from
  the output given.
\item
  A, D. Successor routes are the routes picked from the topology table
  as the best route to a remote network, so these are the routes that IP
  uses in the routing table to forward traffic to a remote destination.
  The topology table contains any route that is not as good as the
  successor route and is considered a feasible successor, or backup
  route. Remember that all routes are in the topology table, even
  successor routes.
\item
  A, B. Option A will work because the router will change the network
  statement to 10.0.0.0 since EIGRP uses classful addresses by default.
  Therefore, it isn't technically a wrong answer, but please understand
  why it is correct for this question. The 10.255.255.64/27 subnet
  address can be configured with wildcards just as we use with OSPF and
  ACLs. The /27 is a block of 32, so the wildcard in the fourth octet
  will be 31. The wildcard of 0.0.0.0 is wrong because this is a network
  address, not a host address, and the 0.0.0.15 is wrong because that is
  only a block of 16 and would only work if the mask was a /28.
\item
  \protect\hypertarget{b02.xhtmlux5cux23Page_1027}{}{}C. To troubleshoot
  adjacencies, you need to check the AS numbers, the K values, networks,
  passive interfaces, and ACLs.
\item
  C. EIGRP and EIGRPv6 will load-balance across 4 equal-cost paths by
  default but can be configured to load-balance across equal- and
  unequal-cost paths, up to 32 with IOS 15.0 code.
\item
  B, E. EIGRP must be enabled with an AS number from global
  configuration mode with the \texttt{ipv6\ router\ eigrp\ as} command
  if you need to set the RID or other global parameters. Instead of
  configuring EIGRP with the network command as with EIGRP, EIGRPv6 is
  configured on a per-interface basis with the \texttt{ipv6\ eigrp\ as}
  command.
\item
  C. There isn't a lot to go on from with the output, but that might
  make this easier than if there were a whole page of output. Since
  s0/0/2 has the lowest FD and AD, that would become the successor
  route. For a route to become a feasible successor, its reported
  distance must be lower than the feasible distance of the current
  successor route, so C is our best answer based on what we can see.
\item
  C. The network in the diagram is considered a discontiguous network
  because you have one classful address subnetted and separated by
  another classful address. Only RIPv2, OSPF, and EIGRP can work with
  discontiguous networks, but RIPv2 and EIGRP won't work by default
  (except for routers running the new 15.0 code). You must use the
  \texttt{no\ auto-summary} command under the routing protocol
  configuration. There is a passive interface on RouterB, but this is
  not on an interface between RouterA and RouterB and won't stop an
  adjacency.
\item
  A, B, C, D. Here are the documented issues that Cisco says to check
  when you have an adjacency issue:

  \begin{enumerate}
  \tightlist
  \item
    Interfaces between the devices are down.
  \item
    The two routers have mismatching EIGRP autonomous system numbers.
  \item
    Proper interfaces are not enabled for the EIGRP process.
  \item
    An interface is configured as passive.
  \item
    K values are mismatched.
  \item
    EIGRP authentication is misconfigured.
  \end{enumerate}
\end{enumerate}

\subsection[Chapter 18: Open Shortest Path First
(OSPF)]{\texorpdfstring{\protect\hypertarget{b02.xhtmlux5cux23bapp02-sec-18}{}{}Chapter
18: Open Shortest Path First
(OSPF)}{Chapter 18: Open Shortest Path First (OSPF)}}

\begin{enumerate}
\item
  B. Only the EIGRP routes will be placed in the routing table because
  it has the lowest administrative distance (AD), and that is always
  used before metrics.
\item
  A, B, C. Any router that is a member of two areas must be an area
  border router or ABR.
\item
  \protect\hypertarget{b02.xhtmlux5cux23Page_1028}{}{}A, C. The process
  ID for OSPF on a router is only locally significant, and you can use
  the same number on each router, or each router can have a different
  number---it just doesn't matter. The numbers you can use are from 1 to
  65,535. Don't get this confused with area numbers, which can be from 0
  to 4.2 billion.
\item
  B. The router ID (RID) is an IP address used to identify the router.
  It need not and should not match.
\item
  C. The router ID (RID) is an IP address used to identify the router.
  Cisco chooses the router ID by using the highest IP address of all
  configured loopback interfaces. If no loopback interfaces are
  configured with addresses, OSPF will choose the highest IP address of
  all active physical interfaces.
\item
  A. The administrator typed in the wrong wildcard mask configuration.
  The wildcard should have been 0.0.0.255 or even 0.255.255.255.
\item
  A. A dash (-) in the State column indicates no DR election, because
  they are not required on a point-to-point link such as a serial
  connection.
\item
  D. By default, the administrative distance of OSPF is 110.
\item
  A. Hello packets are addressed to multicast address 224.0.0.5.
\item
  A. The \texttt{show\ ip\ ospf\ neighbor} command displays all
  interface-related neighbor information. This output shows the DR and
  BDR (unless your router is the DR or BDR), the RID of all directly
  connected neighbors, and the IP address and name of the directly
  connected interface.
\item
  A. 224.0.0.6 is used on broadcast networks to reach the DR and BDR.
\item
  D. The Hello and Dead timers must be set the same on two routers on
  the same link or they will not form an adjacency (relationship). The
  default timers for OSPF are 10 seconds for the Hello timer and 40
  seconds for the Dead timer.
\item
  \begin{figure}
  \centering
  \includegraphics{images/a02f008.jpg}
  \caption{}
  \end{figure}

  A designated router is elected on broadcast networks. Each OSPF router
  maintains an identical database describing the AS topology. A Hello
  protocol provides dynamic neighbor discovery. A routing table contains
  only the best routes.
\item
  \texttt{passive-interface\ fastEthernet\ 0/1}

  The command \texttt{passive-interface\ fastEthernet\ 0/1} will disable
  OSPF on the specified interface only.
\item
  \protect\hypertarget{b02.xhtmlux5cux23Page_1029}{}{}B, G. To enable
  OSPF, you must first start OSPF using a process ID. The number is
  irrelevant; just choose a number from 1 to 65,535 and you're good to
  go. After you start the OSPF process, you must configure interfaces on
  which to activate OSPF using the network command with wildcards and
  specification of an area. Option F is wrong because there must be a
  space after the parameter area and before you list the area number.
\item
  A. The default OSPF interface priority is 1, and the highest interface
  priority determines the designated router (DR) for a subnet. The
  output indicates that the router with a router ID of 192.168.45.2 is
  currently the backup designated router (BDR) for the segment, which
  indicates that another router became the DR. It can be then be assumed
  that the DR router has an interface priority higher than 2. (The
  router serving the DR function is not present in the truncated sample
  output.)
\item
  A, B, C. OSPF is created in a hierarchical design, not a flat design
  like RIP. This decreases routing overhead, speeds up convergence, and
  confines network instability to a single area of the network.
\item
  \texttt{show\ ip\ ospf\ interface}

  The \texttt{show\ ip\ ospf\ interface} command displays all
  interface-related OSPF information. Data is displayed about OSPF
  information for all OSPF-enabled interfaces or for specified
  interfaces.
\item
  A. LSA packets are used to update and maintain the topological
  database.
\item
  B. At the moment of OSPF process startup, the highest IP address on
  any active interface will be the router ID (RID) of the router. If you
  have a loopback interface configured (logical interface), then that
  will override the interface IP address and become the RID of the
  router automatically.
\end{enumerate}

\subsection[Chapter 19: Multi-Area
OSPF]{\texorpdfstring{\protect\hypertarget{b02.xhtmlux5cux23bapp02-sec-19}{}{}Chapter
19: Multi-Area OSPF}{Chapter 19: Multi-Area OSPF}}

\begin{enumerate}
\tightlist
\item
  A, B, D. As the size of a single-area OSPF network grows, so does the
  size of the routing table and OSPF database that have to be
  maintained. Also, if there is a change in network topology, the OSPF
  algorithm has to be rerun for the entire network.
\item
  B. An autonomous system boundary router (ASBR) is any OSPF router that
  is connected to an external routing process (another AS). An ABR, on
  the other hand, connects one (or more) OSPF areas together to area 0.
\item
  B, D, E. In order for two OSPF routers to create an adjacency, the
  Hello and Dead timers must match, and they must both be configured
  into the same area as well as being in the same subnet. Also, if
  authentication is configured, that info must match as well.
\item
  C. The process starts by sending out Hello packets. Every listening
  router will then add the originating router to the neighbor database.
  The responding routers will reply with all of their Hello information
  so that the originating router can add them to its own neighbor table.
  At this point, we will have reached the 2WAY state---only certain
  routers will advance beyond this to establish adjacencies.
\item
  \protect\hypertarget{b02.xhtmlux5cux23Page_1030}{}{}D. If you have
  multiple links to the same network, you can change the default cost of
  a link so OSPF will prefer that link over another with the
  \texttt{ip\ ospf\ cost\ cost} command.
\item
  B. In the FULL state, all LSA information is synchronized among
  adjacent neighbors. OSPF routing can begin only after the FULL state
  has been reached. The FULL state occurs after the LOADING state
  finishes.
\item
  B, D, E. Configuring OSPFv3 is pretty simple, as long as you know what
  interfaces you are using on your router. There are no network
  statements; OSPFv3 is configured on a per-interface basis. OSPFv2 and
  OSPFv3 both use a 32-bit RID, and if you have an IPv4 address
  configured on at least one interface, you do not need to manually set
  a RID when configuring EIGRPv3.
\item
  B. When OSPF adjacency is formed, a router goes through several state
  changes before it becomes fully adjacent with its neighbor. The states
  are (in order) DOWN, ATTEMPT, INIT, 2WAY, EXSTART, EXCHANGE, LOADING,
  and FULL.
\item
  B. Referred to as a network link advertisement (NLA), Type 2 LSAs are
  generated by designated routers (DRs). Remember that a designated
  router is elected to represent other routers in its network, and it
  establishes adjacencies with them. The DR uses a Type 2 LSA to send
  out information about the state of other routers that are part of the
  same network.
\item
  C. Referred to as summary link advertisements (SLAs), Type 3 LSAs are
  generated by area border routers. These ABRs send Type 3 LSAs toward
  the area external to the one where they were generated. The Type 3 LSA
  advertises networks, and these LSAs advertise inter-area routes to the
  backbone area (area 0).
\item
  D. To see all LSAs a router has learned from its neighbors, you need
  to see the OSPF LSDB, and you can see this with the
  \texttt{show\ ip\ ospf\ database} command.
\item
  B. Based on the information in the question, the cost from R1 to R2 is
  4, the cost from R2 to R3 is 15, and the cost from R3 to R5 is 4. 15 +
  4 + 4 = 23. Pretty simple.
\item
  B, D. Since R3 is connected to area 1 and R1 is connected to area 2
  and area 0, the routes advertised from R3 would show as \texttt{OI},
  or inter-area routes.
\item
  A, D, E, F, G. For two OSPF routers to form an adjacency, they must be
  in the same area, must be in the same subnet, and the authentication
  information must match, if configured. You need to also check if an
  ACL is set and if a passive interface is configured, and every OSPF
  router must use a different RID.
\item
  C. The IOS command \texttt{show\ ip\ ospf\ neighbor} shows neighbor
  router information, such as neighbor ID and the state of adjacency
  with the neighboring router.
\item
  D. The command \texttt{show\ ip\ ospf\ interface} on a default
  broadcast multi-access network will show you DRs and BDRs on that
  network.
\item
  A, C, D, F. It's hard to tell from this single output what is causing
  the problem with the adjacency, but we need to check the ACL 10 to see
  what that is doing, verify that the routers are in the same area and
  in the same subnet, and see if passive interface is configured with
  the interface we're using.
\item
  \protect\hypertarget{b02.xhtmlux5cux23Page_1031}{}{}B, D, G. The
  default reference bandwidth is 100 by default, and you can change it
  under the OSPF process with the
  \texttt{auto-cost\ reference\ bandwidth\ number} command, but if you
  do, you need to configure this command on all routers in your AS.
\item
  A, D. An OSPF RID will be used as source of Type 1 LSA, and the router
  will chose the highest loopback interface as its OSPF router ID (if
  available).
\item
  B, C. With single area OSPF you'd use only a couple LSA types, which
  can save on bandwidth. Also, you wouldn't need virtual links, which is
  a configuration that allows you to connect an area to another area
  that is not area 0.
\end{enumerate}

\subsection[Chapter 20: Troubleshooting IP, IPv6, and
VLANs]{\texorpdfstring{\protect\hypertarget{b02.xhtmlux5cux23bapp02-sec-20}{}{}Chapter
20: Troubleshooting IP, IPv6, and
VLANs}{Chapter 20: Troubleshooting IP, IPv6, and VLANs}}

\begin{enumerate}
\tightlist
\item
  D. Positive confirmation has been received confirming that the path to
  the neighbor is functioning correctly. REACH is good!
\item
  B. The most common cause of interface errors is a mismatched duplex
  mode between two ends of an Ethernet link. If they have mismatched
  duplex settings, you'll receive a legion of errors, which cause nasty
  slow performance issues, intermittent connectivity, and massive
  collisions---even total loss of communication!
\item
  D. You can verify the DTP status of an interface with the
  \texttt{sh\ dtp\ interface\ interface} command.
\item
  A. No DTP frames are generated from the interface. Nonegotiate can be
  used only if the neighbor interface is manually set as trunk or
  access.
\item
  D. The command \texttt{show\ ipv6\ neighbors} provides the ARP cache
  on a router.
\item
  B. The state is STALE when the interface has not communicated within
  the neighbor reachable time frame. The next time the neighbor
  communicates, the state will change back to REACH.
\item
  B. There is no IPv6 default gateway, which will be the link-local
  address of the router interface, sent to the host as a router
  advertisement. Until this host receives the router address, the host
  will communicate with IPv6 only on the local subnet.
\item
  D. This host is using IPv4 to communicate on the network, and without
  an IPv6 global address, the host will be able to communicate to only
  remote networks with IPv4. The IPv4 address and default gateway are
  not configured into the same subnet.
\item
  B, C. The commands \texttt{show\ interface\ trunk} and
  \texttt{show\ interface\ interface\ switchport} will show you
  statistics of ports, which includes native VLAN information.
\item
  \protect\hypertarget{b02.xhtmlux5cux23Page_1032}{}{}A. Most Cisco
  switches ship with a default port mode of auto, meaning that they will
  automatically trunk if they connect to a port that is on or desirable.
  Remember that not all switches are shipped as mode auto, but many are,
  and you need to set one side to either on or desirable in order to
  trunk between switches.
\end{enumerate}

\subsection[Chapter 21: Wide Area
Networks]{\texorpdfstring{\protect\hypertarget{b02.xhtmlux5cux23bapp02-sec-21}{}{}Chapter
21: Wide Area Networks}{Chapter 21: Wide Area Networks}}

\begin{enumerate}
\tightlist
\item
  C. The command \texttt{debug\ ppp\ authentication} will show you the
  authentication process that PPP uses across point-to-point
  connections.
\item
  B, D. Since BGP does not automatically discover neighbors like other
  routing protocols do, you have to explicitly configure them using the
  \texttt{neighbor\ peer-ip-address\ remote-as\ peer-as-number} command.
\item
  D. BGP uses TCP as the transport mechanism, which provides reliable
  connection-oriented delivery. BGP uses TCP port 179. Two routers that
  are using BGP form a TCP connection with one another. These two BGP
  routers are called ``peer routers,'' or ``neighbors.''
\item
  D. The \texttt{show\ ip\ bgp\ neighbor} command is used to see the
  hold time on two BGP peers.
\item
  B. The 0.0.0.0 in the next hop field output of the show ip bgp command
  means that the network was locally entered on the router with the
  network command into BGP.
\item
  A, D. GRE tunnels have the following characteristics: GRE uses a
  protocol-type field in the GRE header so any layer 3 protocol can be
  used through the tunnel, GRE is stateless and has no flow control, GRE
  offers no security, and GRE creates additional overhead for tunneled
  packets---at least 24 bytes.
\item
  C. If you receive this flapping message when you configure your GRE
  tunnel, this means that you used your tunnel interface address instead
  of the tunnel destination address.
\item
  D. The \texttt{show\ running-config\ interface\ tunnel\ 0} command
  will show you the configuration of the interface, not the status of
  the tunnel.
\item
  C. PPP uses PAP and CHAP as authentication protocols. PAP is clear
  text, CHAP uses an MD5 hash.
\item
  C. PPPoE encapsulates PPP frames in Ethernet frames and uses common
  PPP features like authentication, encryption, and compression. PPPoA
  is used for ATM.
\item
  C, D. S0/0/0 is up, meaning the s0/0/0 is talking to the CSU/DSU, so
  that isn't the problem. If the authentication failed or the other end
  has a different encapsulation than either one of those reasons would
  be why a data link is not established.
\item
  C. The show interfaces command shows the configuration settings and
  the interface status as well as the IP address and tunnel source and
  destination address.
\item
  B, C, D. This is just a basic WAN question to test your understanding
  of connections. PPP does not need to be used, so option A is not
  valid. You can use any type of connection to
  \protect\hypertarget{b02.xhtmlux5cux23Page_1033}{}{}connect to a
  customer site, so option B is a valid answer. You can also use any
  type of connection to get to the Frame Relay switch, as long as the
  ISP supports it, and T1 is valid, so option C is okay. Ethernet as a
  WAN can be used with Ethernet over MPLS (EoMPLS); however, you don't
  need to configure a DLCI unless you're using Frame Relay, so E is not
  a valid answer for this question.
\item
  B. All web browsers support Secure Sockets Layer (SSL), and SSL VPNs
  are known as Web VPNs. Remote users can use their browser to create an
  encrypted connection and they don't need to install any software. GRE
  doesn't encrypt the data.
\item
  E. This is an easy question because the Remote router is using the
  default HDLC serial encapsulation and the Corp router is using the PPP
  serial encapsulation. You should go to the Remote router and set that
  encapsulation to PPP or change the Corp router back to the default of
  HDLC by typing \texttt{no\ encapsulation} under the interface.
\item
  A, C, E. VPNs can provide very good security by using advanced
  encryption and authentication protocols, which will help protect your
  network from unauthorized access. By connecting the corporate remote
  offices to their closest Internet provider and then creating a VPN
  tunnel with encryption and authentication, you'll gain a huge savings
  over opting for traditional leased point-to-point lines. VPNs scale
  very well to quickly bring up new offices or have mobile users connect
  securely while traveling or when connecting from home. VPNs are very
  compatible with broadband technologies.
\item
  A, D. Internet providers who have an existing Layer 2 network may
  choose to use layer 2 VPNs instead of the other common layer 3 MPLS
  VPN. Virtual Pricate Lan Switch (VPLS) and Virtual Private Wire
  Service (VPWS) are two technologies that provide layer 2 MPLS VPNs.
\item
  D. IPsec is an industry-wide standard suite of protocols and
  algorithms that allows for secure data transmission over an IP-based
  network that functions at the layer 3 Network layer of the OSI model.
\item
  C. A VPN allows or describes the creation of private networks across
  the Internet, enabling privacy and tunneling of TCP/IP protocols. A
  VPN can be set up across any type of link.
\item
  B, C. Layer 2 MPLS VPNs and the more popular Layer 3 MPLS VPN are
  service provided to customers and managed by the provider.
\end{enumerate}

\subsection[Chapter 22: Evolution of Intelligent
Networks]{\texorpdfstring{\protect\hypertarget{b02.xhtmlux5cux23bapp02-sec-22}{}{}Chapter
22: Evolution of Intelligent
Networks}{Chapter 22: Evolution of Intelligent Networks}}

\begin{enumerate}
\tightlist
\item
  B. Dropping packets as they arrive is called tail drop. Selective
  dropping of packets during the time queues are filling up is called
  congestion avoidance (CA). Cisco uses weighted random early detection
  (WRED) as a CA scheme to monitor the buffer depth and performs early
  discards (drops) on random packets when the minimum defined queue
  threshold is exceeded.
\item
  \protect\hypertarget{b02.xhtmlux5cux23Page_1034}{}{}A, C. You unite
  switches into a single logical unit using special stack interconnect
  cables that create a bidirectional closed-loop path. The network
  topology and routing information are updated continuously through the
  stack interconnect.
\item
  C. A more efficient use of resources has a cost benefit because less
  physical equipment means less cost. What minimizes the spending is the
  fact that the customer pays only for the services or infrastructure
  that the customer uses.
\item
  B, D, E. Voice traffic is real-time traffic and comprises constant and
  predictable bandwidth and packet arrival times. One-way requirements
  incudes latency \textless{} 150 ms, jitter \textless30 ms, and loss
  \textless{} 1\%, and bandwidth needs to be 30 to 128 Kbps.
\item
  B. The control plane represents the core layer of the SDN architecture
  and is where the Cisco APIC-EM resides.
\item
  A. Infrastructure as a Service (IaaS) provides only the network and
  delivers the computer infrastructure (platform virtualization
  environment).
\item
  C. A trust boundary is where packets are classified and marked. IP
  phones and the boundary between the ISP and enterprise network are
  common examples of trust boundaries.
\item
  A. The IWAN provides transport-independent connectivity, intelligent
  path control, application optimization, and highly secure
  connectivity.
\item
  A. NBAR is a layer 4 to layer 7 deep-packet inspection classifier.
  NBAR is more CPU intensive than marking and uses the existing
  markings, addresses, or ACLs.
\item
  C. DSCP is a set of 6-bit values that are used to describe the meaning
  of the layer 3 IPv4 ToS field. While IP precedence is the old way to
  mark ToS, DSCP is the new way and is backward compatible with IP
  precedence.
\item
  A, B. Southbound APIs (or device-to-control-plane interfaces) are used
  for communication between the controllers and network devices, which
  puts these interfaces between the control and data planes.
\item
  D. Class of Service (CoS) is a term to describe designated fields in a
  frame or packet header. How devices treat packets in your network
  depends on the field values. CoS is usually used with Ethernet frames
  and contains 3 bits.
\item
  D. Although option A is the best answer by far, it is unfortunately
  false. You will save time working on autonomous devices, which in turn
  will allow you more time to work on new business requirements.
\item
  C. When traffic exceeds the allocated rate, the policer can take one
  of two actions. It can either drop traffic or re-mark it to another
  class of service. The new class usually has a higher drop probability.
\item
  C, D, F. The SDN architecture slightly differs from the architecture
  of traditional networks. It comprises three stacked layers: data,
  control, and application.
\item
  C. NBAR is a layer 4 to layer 7 deep-packet inspection classifier.
\item
  \protect\hypertarget{b02.xhtmlux5cux23Page_1035}{}{}B, D. Southbound
  APIs (or device-to-control-plane interfaces) are used for
  communication between the controllers and network devices. Northbound
  APIs, or northbound interfaces, are responsible for the communication
  between the SDN controller and the services running over the network.
  With onePK, Cisco attempting to provide a high-level proprietary API
  that allows you to inspect or modify the network element configuration
  without hardware upgrades. The data plane is responsible for the
  forwarding of frames or packets.
\item
  B, D. Each stack of switches has a single IP address and is managed as
  a single object. This single IP management applies to activities such
  as fault detection, VLAN creation and modification, security, and QoS
  controls. Each stack has only one configuration file, which is
  distributed to each member in the stack. When you add a new switch to
  the stack, the master switch automatically configures the unit with
  the currently running IOS image and the configuration of the stack.
  You do not have to do anything to bring up the switch before it is
  ready to operate.
\item
  B. Platform as a Service (PaaS) provides the operating system and the
  network by delivering a computing platform and solution stack.
\item
  C. Software as a Service (SaaS) provides the required software,
  operating system, and network by providing ready-to-use applications
  or software.
\item
  C. All data that the cloud stores will always be available. This
  availability means that users do not need to back up their data.
  Before the cloud, users could lose important documents because of an
  accidental deletion, misplacement, or computer breakdown.
\end{enumerate}

\protect\hypertarget{b03.xhtml}{}{}

\section[{Appendix C}\\
{Disabling and Configuring Network
Services}]{\texorpdfstring{\protect\hypertarget{b03.xhtmlux5cux23bapp03}{}{}\protect\hypertarget{b03.xhtmlux5cux23Page_1037}{}{}{Appendix
C}\\
{Disabling and Configuring Network
Services}}{Appendix C Disabling and Configuring Network Services}}

\protect\hypertarget{b03.xhtmlux5cux23Page_1038}{}{}\includegraphics{images/c01inline001.png}
By default, the Cisco IOS runs some services that are unnecessary to its
normal operation, and if you don't disable them, they can be easy
targets for denial-of-service (DoS) attacks and break-in attempts.

DoS attacks are the most common attacks because they are the easiest to
perform. Using software and/or hardware tools such as an intrusion
detection system (IDS) and intrusion prevention system (IPS) tools can
both warn and stop these simple, but harmful, attacks. However, if we
can't implement IDS/IPS, there are some basic commands we can use on our
router to make them more safe. Keep in mind, though, that nothing will
make you completely safe in today's networks.

Let's take a look at the basic services we should disable on our
routers.

\subsection[Blocking SNMP
Packets]{\texorpdfstring{\protect\hypertarget{b03.xhtmlux5cux23bapp03-sec-1}{}{}Blocking
SNMP Packets}{Blocking SNMP Packets}}

The Cisco IOS default configurations permit remote access from any
source, so unless you're either way too trusting or insane, it should be
totally obvious to you that those configurations need a bit of
attention. You've got to restrict them. If you don't, the router will be
a pretty easy target for an attacker who wants to log in to it. This is
where access lists come into the game---they can really protect you.

If you place the following command on the serial0/0 interface of the
perimeter router, it'll stop any SNMP packets from entering the router
or the DMZ. (You'd also need to have a \texttt{permit} command along
with this list to really make it work, but this is just an example.)

\begin{verbatim}
Lab_B(config)#access-list 110 deny udp any any eq snmp
Lab_B(config)#interface s0/0
Lab_B(config-if)#access-group 110 in
\end{verbatim}

\subsection[Disabling
Echo]{\texorpdfstring{\protect\hypertarget{b03.xhtmlux5cux23bapp03-sec-2}{}{}Disabling
Echo}{Disabling Echo}}

In case you don't know this already, small services are servers
(daemons) running in the router that are quite useful for diagnostics.
And here we go again---by default, the Cisco router has a series of
diagnostic ports enabled for certain UDP and TCP services, including
echo, chargen, and discard.

\protect\hypertarget{b03.xhtmlux5cux23Page_1039}{}{}When a host attaches
to those ports, a small amount of CPU is consumed to service these
requests. All a single attacking device needs to do is send a whole slew
of requests with different, random, phony source IP addresses to
overwhelm the router, making it slow down or even fail. You can use the
\texttt{no} version of these commands to stop a chargen attack:

\begin{verbatim}
Lab_B(config)#no service tcp-small-servers
Lab_B(config)#no service udp-small-servers
\end{verbatim}

Finger is a utility program designed to allow users of Unix hosts on the
Internet to get information about each other:

\begin{verbatim}
Lab_B(config)#no service finger
\end{verbatim}

This matters because the \texttt{finger} command can be used to find
information about all users on the network and/or the router. It's also
why you should disable it. The \texttt{finger} command is the remote
equivalent to issuing the \texttt{show\ users} command on the router.

Here are the TCP small services:

\textbf{Echo} Echoes back whatever you type. Type the command
\texttt{telnet\ x.x.x.x\ echo\ ?} to see the options.

\textbf{Chargen} Generates a stream of ASCII data. Type the command
\texttt{telnet\ x.x.x.x\ chargen\ ?} to see the options.

\textbf{Discard} Throws away whatever you type. Type the command
\texttt{telnet\ x.x.x.x\ discard\ ?} to see the options.

\textbf{Daytime} Returns the system date and time, if correct. It is
correct if you are running NTP or have set the date and time manually
from the EXEC level. Type the command
\texttt{telnet\ x.x.x.x\ daytime\ ?} to see the options.

The UDP small services are as follows:

\textbf{Echo} Echoes the payload of the datagram you send.

\textbf{Discard} Silently pitches the datagram you send.

\textbf{Chargen} Pitches the datagram you send and responds with a
72-character string of ASCII characters terminated with a CR+LF.

\subsection[Turning off BootP and
Auto-Config]{\texorpdfstring{\protect\hypertarget{b03.xhtmlux5cux23bapp03-sec-3}{}{}Turning
off BootP and Auto-Config}{Turning off BootP and Auto-Config}}

Again, by default, the Cisco router also offers the BootP service as
well as remote auto-configuration. To disable these functions on your
Cisco router, use the following commands:

\begin{verbatim}
Lab_B(config)#no ip boot server
Lab_B(config)#no service config
\end{verbatim}

\subsection[Disabling the HTTP
Interface]{\texorpdfstring{\protect\hypertarget{b03.xhtmlux5cux23bapp03-sec-4}{}{}\protect\hypertarget{b03.xhtmlux5cux23Page_1040}{}{}Disabling
the HTTP Interface}{Disabling the HTTP Interface}}

The \texttt{ip\ http\ server} command may be useful for configuring and
monitoring the router, but the cleartext nature of HTTP can obviously be
a security risk. To disable the HTTP process on your router, use the
following command:

\begin{verbatim}
Lab_B(config)#no ip http server
\end{verbatim}

To enable an HTTP server on a router for AAA, use the global
configuration command \texttt{ip\ http\ server}.

\subsection[Disabling IP Source
Routing]{\texorpdfstring{\protect\hypertarget{b03.xhtmlux5cux23bapp03-sec-5}{}{}Disabling
IP Source Routing}{Disabling IP Source Routing}}

The IP header source-route option allows the source IP host to set a
packet's route through the IP network. With IP source routing enabled,
packets containing the source-route option are forwarded to the router
addresses specified in the header. Use the following command to disable
any processing of packets with source-routing header options:

\begin{verbatim}
Lab_B(config)#no ip source-route
\end{verbatim}

\subsection[Disabling Proxy
ARP]{\texorpdfstring{\protect\hypertarget{b03.xhtmlux5cux23bapp03-sec-6}{}{}Disabling
Proxy ARP}{Disabling Proxy ARP}}

Proxy ARP is the technique in which one host---usually a
router---answers ARP requests intended for another machine. By
``faking'' its identity, the router accepts responsibility for getting
those packets to the ``real'' destination. Proxy ARP can help machines
on a subnet reach remote subnets without configuring routing or a
default gateway. The following command disables proxy ARP:

\begin{verbatim}
Lab_B(config)#interface fa0/0
Lab_B(config-if)#no ip proxy-arp
\end{verbatim}

Apply this command to all your router's LAN interfaces.

\subsection[Disabling Redirect
Messages]{\texorpdfstring{\protect\hypertarget{b03.xhtmlux5cux23bapp03-sec-7}{}{}Disabling
Redirect Messages}{Disabling Redirect Messages}}

ICMP redirect messages are used by routers to notify hosts on the data
link that a better route is available for a particular destination. To
disable the redirect messages so bad people can't draw out your network
topology with this information, use the following command:

\begin{verbatim}
Lab_B(config)#interface s0/0
Lab_B(config-if)#no ip redirects
\end{verbatim}

\protect\hypertarget{b03.xhtmlux5cux23Page_1041}{}{}Apply this command
to all your router's interfaces. However, just understand that if this
is configured, legitimate user traffic may end up taking a suboptimal
route. Use caution when disabling this command.

\subsection[Disabling the Generation of ICMP Unreachable
Messages]{\texorpdfstring{\protect\hypertarget{b03.xhtmlux5cux23bapp03-sec-8}{}{}Disabling
the Generation of ICMP Unreachable
Messages}{Disabling the Generation of ICMP Unreachable Messages}}

The \texttt{no\ ip\ unreachables} command prevents the perimeter router
from divulging topology information by telling external hosts which
subnets are not configured. This command is used on a router's interface
that is connected to an outside network:

\begin{verbatim}
Lab_B(config)#interface s0/0
Lab_B(config-if)#no ip unreachables
\end{verbatim}

Again, apply this to all the interfaces of your router that connect to
the outside world.

\subsection[Disabling Multicast Route
Caching]{\texorpdfstring{\protect\hypertarget{b03.xhtmlux5cux23bapp03-sec-9}{}{}Disabling
Multicast Route Caching}{Disabling Multicast Route Caching}}

The multicast route cache lists multicast routing cache entries. These
packets can be read, and so they create a security problem. To disable
the multicast route caching, use the following command:

\begin{verbatim}
Lab_B(config)#interface s0/0
Lab_B(config-if)#no ip mroute-cache
\end{verbatim}

Apply this command to all the interfaces of the router. However, use
caution when disabling this command because it may slow legitimate
multicast traffic.

\subsection[Disabling the Maintenance Operation Protocol
(MOP)]{\texorpdfstring{\protect\hypertarget{b03.xhtmlux5cux23bapp03-sec-10}{}{}Disabling
the Maintenance Operation Protocol
(MOP)}{Disabling the Maintenance Operation Protocol (MOP)}}

The Maintenance Operation Protocol (MOP) works at the Data Link and
Network layers in the DECnet protocol suite and is used for utility
services like uploading and downloading system software, remote testing,
and problem diagnosis. So, who uses DECnet? Anyone with their hands up?
I didn't think so. To disable this service, use the following command:

\begin{verbatim}
Lab_B(config)#interface s0/0
Lab_B(config-if)#no mop enabled
\end{verbatim}

Apply this command to all the interfaces of the router.

\subsection[Turning Off the X.25 PAD
Service]{\texorpdfstring{\protect\hypertarget{b03.xhtmlux5cux23bapp03-sec-11}{}{}\protect\hypertarget{b03.xhtmlux5cux23Page_1042}{}{}Turning
Off the X.25 PAD Service}{Turning Off the X.25 PAD Service}}

Packet assembler/disassembler (PAD) connects asynchronous devices like
terminals and computers to public/private X.25 networks. Since every
computer in the world is pretty much IP savvy, and X.25 has gone the way
of the dodo bird, there is no reason to leave this service running. Use
the following command to disable the PAD service:

\begin{verbatim}
Lab_B(config)#no service pad
\end{verbatim}

\subsection[Enabling the Nagle TCP Congestion
Algorithm]{\texorpdfstring{\protect\hypertarget{b03.xhtmlux5cux23bapp03-sec-12}{}{}Enabling
the Nagle TCP Congestion
Algorithm}{Enabling the Nagle TCP Congestion Algorithm}}

The Nagle TCP congestion algorithm is useful for small packet
congestion, but if you're using a higher setting than the default MTU of
1,500 bytes, it can create an above-average traffic load. To enable this
service, use the following command:

\begin{verbatim}
Lab_B(config)#service nagle
\end{verbatim}

It is important to understand that the Nagle congestion service can
break X Window connections to an X server, so don't use it if you're
using X Window.

\subsection[Logging Every
Event]{\texorpdfstring{\protect\hypertarget{b03.xhtmlux5cux23bapp03-sec-13}{}{}Logging
Every Event}{Logging Every Event}}

Used as a syslog server, the Cisco ACS server can log events for you to
verify. Use the \texttt{logging\ trap\ debugging}
or\texttt{\ logging\ trap} \emph{level} command and the \texttt{logging}
\emph{ip\_address} command to turn this feature on:

\begin{verbatim}
Lab_B(config)#logging trap debugging
Lab_B(config)#logging 192.168.254.251
Lab_B(config)#exit
Lab_B#sh logging
Syslog logging: enabled (0 messages dropped, 0 flushes, 0 overruns)
    Console logging: level debugging, 15 messages logged
    Monitor logging: level debugging, 0 messages logged
    Buffer logging: disabled
    Trap logging: level debugging, 19 message lines logged
        Logging to 192.168.254.251, 1 message lines logged
\end{verbatim}

\protect\hypertarget{b03.xhtmlux5cux23Page_1043}{}{}The
\texttt{show\ logging} command provides you with statistics of the
logging configuration on the router.

\subsection[Disabling Cisco Discovery
Protocol]{\texorpdfstring{\protect\hypertarget{b03.xhtmlux5cux23bapp03-sec-14}{}{}Disabling
Cisco Discovery Protocol}{Disabling Cisco Discovery Protocol}}

Cisco Discovery Protocol (CDP) does just that---it's a Cisco proprietary
protocol that discovers directly connected Cisco devices on the network.
But because it's a Data Link layer protocol, it can't find Cisco devices
on the other side of a router. Plus, by default, Cisco switches don't
forward CDP packets, so you can't see Cisco devices attached to any
other port on a switch.

When you are bringing up your network for the first time, CDP can be a
really helpful protocol for verifying it. But since you're going to be
thorough and document your network, you don't need the CDP after that.
And because CDP does discover Cisco routers and switches on your
network, you should disable it. You do that in global configuration
mode, which turns off CDP completely for your router or switch:

\begin{verbatim}
Lab_B(config)#no cdp run
\end{verbatim}

Or, you can turn off CDP on each individual interface using the
following command:

\begin{verbatim}
Lab_B(config-if)#no cdp enable
\end{verbatim}

\subsection[Disabling the Default Forwarded UDP
Protocols]{\texorpdfstring{\protect\hypertarget{b03.xhtmlux5cux23bapp03-sec-15}{}{}Disabling
the Default Forwarded UDP
Protocols}{Disabling the Default Forwarded UDP Protocols}}

When you use the \texttt{ip\ helper-address} command as follows on an
interface, your router will forward UDP broadcasts to the listed server
or servers:

\begin{verbatim}
Lab_B(config)#interface f0/0
Lab_B(config-if)#ip helper-address 192.168.254.251
\end{verbatim}

You would generally use the \texttt{ip\ helper-address} command when you
want to forward DHCP client requests to a DHCP server. The problem is
that not only does this forward port 67 (BootP server request), it
forwards seven other ports by default as well. To disable the unused
ports, use the following commands:

\begin{verbatim}
Lab_B(config)#no ip forward-protocol udp 69
Lab_B(config)#no ip forward-protocol udp 53
Lab_B(config)#no ip forward-protocol udp 37
Lab_B(config)#no ip forward-protocol udp 137
Lab_B(config)#no ip forward-protocol udp 138
Lab_B(config)#no ip forward-protocol udp 68
Lab_B(config)#no ip forward-protocol udp 49
\end{verbatim}

Now, only the BootP server request (67) will be forwarded to the DHCP
server. If you want to forward a certain port---say, TACACS+, for
example---use the following command:

\begin{verbatim}
Lab_B(config)#ip forward-protocol udp 49
\end{verbatim}

\subsection[Cisco's \emph{auto
secure}]{\texorpdfstring{\protect\hypertarget{b03.xhtmlux5cux23bapp03-sec-16}{}{}Cisco's
\emph{auto secure}}{Cisco's auto secure}}

Okay, so ACLs seem like a lot of work and so does turning off all those
services I just discussed. But you do want to secure your router with
ACLs, especially on your interface connected to the Internet. However,
you are just not sure what the best approach should be, or maybe you
just don't want to miss happy hour with your buddies because you're
creating ACLs and turning off default services all night long.

Either way, Cisco has a solution that is a good start, and it's darn
easy to implement. The command is called \texttt{auto\ secure}, and you
just run it from privileged mode as shown:

\begin{verbatim}
R1#auto secure
                --- AutoSecure Configuration ---
*** AutoSecure configuration enhances the security of
the router, but it will not make it absolutely resistant
to all security attacks ***

AutoSecure will modify the configuration of your device.
All configuration changes will be shown. For a detailed
explanation of how the configuration changes enhance
security and any possible side effects, please refer to Cisco.com
for Autosecure documentation.
At any prompt you may enter '?' for help.
Use ctrl-c to abort this session at any prompt.

Gathering information about the router for AutoSecure
Is this router connected to internet? [no]: yes
Enter the number of interfaces facing the internet [1]: [enter]
Interface             IP-Address      OK? Method Status                Protocol
FastEthernet0/0       10.10.10.1      YES NVRAM  up                    up
Serial0/0             1.1.1.1         YES NVRAM  down                  down
FastEthernet0/1       unassigned      YES NVRAM  administratively down down
Serial0/1             unassigned      YES NVRAM  administratively down down
Enter the interface name that is facing the internet: serial0/0
Securing Management plane services...

Disabling service finger
Disabling service pad
Disabling udp & tcp small servers
Enabling service password encryption
Enabling service tcp-keepalives-in
Enabling service tcp-keepalives-out
Disabling the cdp protocol

Disabling the bootp server
Disabling the http server
Disabling the finger service
Disabling source routing
Disabling gratuitous arp

Here is a sample Security Banner to be shown
at every access to device. Modify it to suit your
enterprise requirements.

Authorized Access only
  This system is the property of So-&-So-Enterprise.
  UNAUTHORIZED ACCESS TO THIS DEVICE IS PROHIBITED.
  You must have explicit permission to access this
  device. All activities performed on this device
  are logged. Any violations of access policy will result
  in disciplinary action.

Enter the security banner {Put the banner between
k and k, where k is any character}:
#
If you are not part of the www.globalnettc.com domain, disconnect now!
#
Enable secret is either not configured or
 is the same as enable password
Enter the new enable secret: [password not shown]
% Password too short - must be at least 6 characters. Password configuration
failed
Enter the new enable secret: [password not shown]
Confirm the enable secret : [password not shown]
Enter the new enable password: [password not shown]
Confirm the enable password: [password not shown]
Configuration of local user database
Enter the username: Todd
Enter the password: [password not shown]
Confirm the password: [password not shown]
Configuring AAA local authentication
Configuring Console, Aux and VTY lines for
local authentication, exec-timeout, and transport
Securing device against Login Attacks
Configure the following parameters
Blocking Period when Login Attack detected: ?
% A decimal number between 1 and 32767.
Blocking Period when Login Attack detected: 100
Maximum Login failures with the device: 5
Maximum time period for crossing the failed login attempts: 10
Configure SSH server? [yes]: [enter to take default of yes]
Enter the domain-name: lammle.com
Configuring interface specific AutoSecure services
Disabling the following ip services on all interfaces:

 no ip redirects
 no ip proxy-arp
 no ip unreachables
 no ip directed-broadcast
 no ip mask-reply
Disabling mop on Ethernet interfaces

Securing Forwarding plane services...

Enabling CEF (This might impact the memory requirements for your platform)
Enabling unicast rpf on all interfaces connected
to internet

Configure CBAC Firewall feature? [yes/no]:
Configure CBAC Firewall feature? [yes/no]: no
Tcp intercept feature is used prevent tcp syn attack
on the servers in the network. Create autosec_tcp_intercept_list
to form the list of servers to which the tcp traffic is to
be observed

Enable tcp intercept feature? [yes/no]: yes
\end{verbatim}

And that's it---all the services I mentioned earlier are disabled, plus
some! By saving the configuration that the \texttt{auto\ secure} command
created, you can then take a look at your running-config to see your new
configuration. It's a long one!

Although it is tempting to run out to happy hour right now, you still
need to verify your security and add your internal access-list
configurations to your intranet.

\protect\hypertarget{b04.xhtml}{}{}

\hfill\break

\section[{Comprehensive Online Learning
Environment}]{\texorpdfstring{\protect\hypertarget{b04.xhtmlux5cux23bapp01}{}{}\protect\hypertarget{b04.xhtmlux5cux23Page_1047}{}{}{Comprehensive
Online Learning
Environment}}{Comprehensive Online Learning Environment}}

Register on Sybex.com to gain access to the comprehensive online
interactive learning environment and test bank to help you study for
your CCNA Routing and Switching exams.

The online test bank includes the following:

\begin{itemize}
\tightlist
\item
  Assessment Test to help you focus your study to specific objectives
\item
  Chapter Tests to reinforce what you've learned
\item
  Practice Exams to test your knowledge of the material
\item
  Digital Flashcards to reinforce your learning and provide last-minute
  test prep before the exam
\item
  Searchable Glossary to define the key terms you'll need to know for
  the exam
\item
  21 hours of CCENT/CCNA instruction and 1 FREE Switch lab from the
  subject matter experts at ITProTV
\end{itemize}

Go to
\href{http://http://www.wiley.com/go/sybextestprep}{http://www.wiley.com/go/sybextestprep}
to register and gain access to this comprehensive study tool package.

\hfill\break
\hfill\break
\hfill\break
\hfill\break

\hfill\break
\hfill\break
\hfill\break
\hfill\break

\includegraphics{images/advert.jpg}

\protect\hypertarget{eula.xhtml}{}{}

\section[\textbf{WILEY END USER LICENSE
AGREEMENT}]{\texorpdfstring{\protect\hypertarget{eula.xhtmlux5cux23eula}{}{}\textbf{WILEY
END USER LICENSE AGREEMENT}}{WILEY END USER LICENSE AGREEMENT}}

Go to \href{http://www.wiley.com/go/eula}{www.wiley.com/go/eula} to
access Wiley's ebook EULA.
