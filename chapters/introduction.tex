\Chapter{Introduction}

\section{Routing schemes}
\label{sec:routing-schemes}
\mode<article>{
Routing schemes differ in how they deliver messages.
Unicast is the dominant form of message delivery on the Internet.
}

\Paragraph{unicast}
\mode<article>{
Unicast is a one-to-one transmission from one point in the network to another point; that is, one sender and one receiver, each identified by a network address.
}

\Paragraph{broadcast}
\mode<article>{
Broadcasting is a method of transferring a message to all recipients simultaneously.
Broadcasting can be performed as a high-level operation in a program, or it may be a low-level networking operation, for example broadcasting on Ethernet.
}

\Paragraph{multicast}
\mode<article>{
Multicast is group communication where data transmission is addressed to a group of destination computers simultaneously.
Multicast can be one-to-many or many-to-many distribution.
}

\Paragraph{anycast}
\mode<article>{
Anycast is a network addressing and routing methodology in which a single destination IP address is shared by devices (generally servers) in multiple locations.
Routers direct packets addressed to this destination to the location nearest the sender, using their normal decision-making algorithms, typically the lowest number of BGP network hops.
Anycast routing is widely used by \emph{content delivery networks} (CDN) such as web and DNS hosts, to bring their content closer to end users.
}
