\Chapter{Transport layer protocols}
\label{chap:transport-layer}

\Section{User Datagram Protocol}
\label{sec:udp}

\Paragraph{port numbers}
\mode<article>{
In computer networking, a port is a communication endpoint.
At the software level, within an operating system, a port is a logical construct that identifies a specific process or a type of network service.
A port is identified for each transport protocol and address combination by a 16-bit unsigned number, known as the \emph{port number}.

A port number is always associated with an IP address of a host and the type of transport protocol used for communication.
It completes the destination or origination network address of a message.
Specific port numbers are reserved to identify specific services so that an arriving packet can be easily forwarded to a running application.
For this purpose, port numbers lower than 1024 identify the historically most commonly used services and are called the \emph{well-known port numbers}.
Higher-numbered ports are available for general use by applications and are known as \emph{ephemeral ports}.

\Vref{tab:port-numbers} lists a few well-known port numbers while \vref{tab:ephemeral-ports} lists the range of port numbers used by some operating systems.

\begin{table}
    \centering
    \begin{tabular}{rl}
    \textbf{port} & \textbf{application} \\[1ex]
     22 & SSH \\
     25 & SMTP \\
     53 & DNS \\
     67 & DHCP (server) \\
     68 & DHCP (client) \\
     80 & HTTP \\
    110 & POP3 \\
    143 & IMAP \\
    443 & HTTPS \\
    \end{tabular}
    \caption{A few well-known ports}
    \label{tab:port-numbers}
\end{table}

\begin{table}
    \centering
    \begin{tabular}{lr@{--}l}
    \textbf{operating system} & \multicolumn{2}{l}{\textbf{port range}} \\[1ex]
    Many Linux kernels & \num{32768} & \num{60999} \\
    FreeBSD            & \num{49152} & \num{65535} \\
    Windows XP         & \num{1025} & \num{5000} \\
    Windows 7          & \num{49152} & \num{65535} \\
    \end{tabular}
    \caption{Port ranges used by operating systems as ephemeral ports}
    \label{tab:ephemeral-ports}
\end{table}
}
\slide{well-known port numbers}
\slide{ephemeral port numbers}


\Paragraph{checksum}
\mode<article>{
A checksum is a small-sized block of data derived from another block of digital data for the purpose of detecting errors that may have been introduced during its transmission or storage.
By themselves, checksums are often used to verify data integrity but are not relied upon to verify data authenticity.
}

\Paragraph{``unreliable''}
\mode<article>{
UDP does not keep track of which packets have been sent and received correctly.
It sends data and forgets about it.
Should the application require reliability, it itself is responsible for this function.
This lack of reliability makes UDP a very fast protocol.
}

\Section{Transmission Control Protocol}
\label{sec:tcp}


\Paragraph{port numbers and checksum}
\mode<article>{
See \cref{sec:udp}.
}

\Paragraph{``reliable''}
\mode<article>{
TCP makes sure all data that is sent, is also received correctly by the destination.
It uses a combination of \emph{sequence numbers} and \emph{acknowledgements} to achieve this reliability.
}