\chapter{Routing}
\label{chap:routing}

A \emph{router} connects multiple networks together.
\Vref{fig:routing-basic} shows a router in\-ter\-con\-nect\-ing two networks.
When a host of one network wants to communicate with a host on the other network, it can send the traffic to the router.
The router can forward the traffic on to the other network.
Things are easy for this router as there are only two networks and the router knows about both of them as it has an interface in each network.


\begin{figure}
   \centering
   A ROUTER CONNECTING TWO NETWORKS TOGETHER
   \caption{A router connecting two networks together}
   \label{fig:routing-basic}
\end{figure}

Things become a bit more complex in \vref{fig:routing-two-routers}.
Router \hostname{blue} knows about three of the four networks as it has an interface in these three networks.
Router \hostname{red} only knows about two of the four networks.
For \hostname{blue} to be able to send traffic to the fourth network, we must teach \hostname{blue} how to reach this network.
In this case, things are relatively easy: we can tell \hostname{blue} to forward all traffic destined to this fourth network to \hostname{red}, who already knows about this network.
Similarly we need to teach \hostname{red} how to reach the two networks that it does not know about.


\begin{figure}
   \centering
   \caption{Two routers interconnecting a total of four networks}
   \label{fig:routing-two-routers}
\end{figure}


We could manually configure both \hostname{blue} and \hostname{red} but this is repetitive work, error-prone and it does not react well to changes in the network, for example, when a link fails or gets added.
The industry thus developed several protocols for the routers to exchange this information dynamically.
Each router tells its neighbouring routers the networks it knows about.
This way, eventually all routers learn about all networks and they can calculate the best path to reach each network.

In this chapter will first consider static routing in more detail.
Next we will discuss the Routing Information Protocol, a simple yet still widely used protocol for dynamically exchanging network prefixes.
We conclude the chapter with a brief introduction to both Open Shortest Path First and the Border Gateway Protocol.



\section{The router's routing table}

Every host which has an \abbr{IP} address, also has a routing table which tells the device how to forward traffic.
Depending on your operating system you can view the local routing table with \cmd{netstat} or \cmd{route}.
Below is an example for a server running FreeBSD version 13.0-\textsc{release}-p8.
\begin{verbatim}
$ netstat -rnf inet
Routing tables

Internet:
Destination        Gateway            Flags     Netif Expire
default            198.19.208.1       UGS      vtnet0
10.18.0.0/16       link#1             U        vtnet0
10.18.0.5          link#1             UHS         lo0
10.133.0.0/16      link#2             U        vtnet1
10.133.19.35       link#2             UHS         lo0
127.0.0.1          link#3             UH          lo0
198.19.208.0/21    link#1             U        vtnet0
198.19.210.27      link#1             UHS         lo0
\end{verbatim}
I have given three flags to the command.
The first flag (``-r'') tells \cmd{netstat} to list the routing table.
The second flag (``-n'') tells \cmd{netstat} to not resolve hostnames.
Omitting this flag results in ``127.0.0.1'' being shown as ``localhost'' and ``198.19.210.27'' -- which is this host's local IP address -- to be replaced by the hostname of this host.
The third and final option (``-f inet'') makes \cmd{netstat} only display the \abbr{IP} version~4 or \emph{inet4} routing table.

The first column lists the prefix and prefix length with the exception of a \emph{host route} or single \abbr{IP} address, which has a prefix length of 32 bits, and the default route.
The world ``default'' means ``0.0.0.0/0'' or ``no bits have to match,'' i.e.~it accept all possible \abbr{IP} addresses.

The second column lists the \emph{gateway} or router to which to send the packets destined for remote networks.
Only the default route has a gateway listed, which is also sometimes called the \emph{gateway of last resort}.
The other networks are known to the computer as it has an interface in those networks so there is no need for a gateway.

We will skip the flags in the third column.
You can look those up in the documentation.
Finally the last column lists the interface from which to send the packets to their destination.

Now let's take a look at the first two entries -- namely 0.0.0.0/0 and 10.18.0.0/16 -- and assume we want to send a packet destined to 10.18.20.7.
It matches the default route as all packets match this default, but it also matches the first sixteen bits of the 10.18.0.0 prefix.
Which route must it take then?
This brings us to the first important rule about routing tables:

\begin{quote}
   When selecting a route from the routing table, the longest match always wins.
\end{quote}

As the \abbr{IP} address matches the prefix 10.18.0.0/16 with sixteen bits and the prefix 0.0.0.0/0 with zero bits, the packet is sent out on the local network connected to vtnet0.
This happens to be the same interface as the default route is pointed out to but remember the \emph{address resolution protocol} (\abbr{ARP})!
When you send out traffic destined for a directly connected network you send out an \abbr{ARP} request for the destination \abbr{IP} address while when you need to forward the packet to a gateway, you send out an \abbr{ARP} request for that gateway's \abbr{MAC} address (see \vref{sec:arp}).

Before we move on to static routing, let's take a quick look at an example of a routing table on Windows~10.
Note that the command is not \cmd{netstat} but \cmd{route print}.
I have omitted the interface list which is printed at the top as this is not relevant for this example.

\begin{verbatim}
C:\>route print -4
...
Active Routes:
===============================================================
Network Destination        Netmask  Gateway   Interface  Metric
          0.0.0.0          0.0.0.0  10.0.0.1  10.0.0.75      35
         10.0.0.0        255.0.0.0  On-link   10.0.0.75     290
        10.0.0.75  255.255.255.255  On-link   10.0.0.75     290
       10.0.0.255  255.255.255.255  On-link   10.0.0.75     290
        127.0.0.0        255.0.0.0  On-link   127.0.0.1     330
        127.0.0.1  255.255.255.255  On-link   127.0.0.1     330
  127.255.255.255  255.255.255.255  On-link   127.0.0.1     330
        224.0.0.0        240.0.0.0  On-link   127.0.0.1     330
        224.0.0.0        240.0.0.0  On-link   10.0.0.75     290
  255.255.255.255  255.255.255.255  On-link   127.0.0.1     330
  255.255.255.255  255.255.255.255  On-link   10.0.0.75     290
===============================================================
Persistent Routs:
  None
\end{verbatim}




\section{Static routes}
\label{sec:routing-static}

Although I made them sound as something terrible and to-be avoided, static routes are very useful and are still used extensively in companies both small and large.
Somtimes there is only one way out and then it does not make sense to run a routing protocol.

Let's start with the example in \vref{fig:routing-home-network}.
The only way out for your home router is via the ISP router.
A simple static default route will thus suffice.
A \emph{default route} is the name given to the prefix 0.0.0.0/0 meaning that zero bits (due to the prefix length being zero) have to match the given prefix 0.0.0.0.
This matches all possible IP addresses.


\begin{figure}
   \centering
   Cloud -- ISP router -- home router -- local network (switch) with two computers
   \caption{A home router connecting your local network to the Internet}
   \label{fig:routing-home-network}
\end{figure}


\Vref{fig:routing-static-vpn} shows a second example for static routing.
Two offices are connected to the Internet and have an IPsec \abbr{VPN} tunnel configured between them.
For the devices in the first office to reach the devices in the other office through the tunnel, you can configure a static routing directing the network traffic for the other network through the tunnel interface.


\begin{figure}
   \centering
   Two networks connecting to the Internet and a \abbr{VPN} tunnel between them
   \caption{Static routing is often used to direct traffic through a \abbr{VPN} tunnel}
   \label{fig:routing-static-vpn}
\end{figure}


The third example is one from the big enterprises and service providers.
They tend to have a separate management network (\vref{fig:routing-static-management}).
Servers have their default gateway pointing out the public interface but need to send all management traffic out their secondary interface designated for management.


\begin{figure}
   \centering
   Server connecting to a ``public'' network and a management network
   \caption{Large enterprises and service providers tend to have separate management networks requiring static routing to be configured on the servers}
   \label{fig:routing-static-management}
\end{figure}



\section{Dynamic routing protocols}
\label{sec:routing-dynamic}

In specific simple network topologies static routing is perfectly suitable to provide connectivity between routers.
However, most networks are not this simplistic and a dynamic routing protocol then makes life a lot easier.

There are two kinds of dynamic routing protocols: \emph{distance-vector} and \emph{link-state} protocols.
A distance-vector routing protocol calculates the best routes to reach every network that it knows and then sends this information to its neighbours as a vector containing distance and direction.
You can compare these vectors to the fingerposts you see on crossroads.
They list the cities you can reach, the direction to go to and the distance you have to travel before reaching your destination.
A router using a distance-vector routing protocol does not know the detailed information on the different paths available.
It only knows that the best (shortest) route is a given direction.
This is also sometimes called \emph{routing by rumour}.

Link-state routing protocols are comparable to a street map.
Every router sends updates throughout the network indicating which networks it is connected to.
A router can use the updates from all other routers to construct a complete network map.
It knows exactly which router connects to which and where each network is located.
It will then calcuate the best route based on this information.

The Routing Information Protocol (\abbr{RIP}) and the Enhanced Interior Gateway Routing Protocol (\abbr{EIGRP}) are both examples of distance-vector routing protocols.
Open Shortest Path First (\abbr{OSPF}) and Intermediate System to Intermediate System (\abbr{IS-IS}) are two link-state routing protocols.

Each routing protocol uses a \emph{metric} to calcuate the best path but each protocol can decide what this metric must be.
It could mean the shortest path (least amount of routers it must pass) or it could mean the fastest path (using the fastest links).
It can also be a combination of different factors.

\section{Routing Information Protocol}
\label{sec:rip}

The Routing Information Protocol (\abbr{RIP}) is the simplest distance-vector routing protocol and uses a \emph{hop count} to calculate the metric.
The hop count corresponds to the number of routers you have to pass to reach the destination.
The hop count for a given prefix is unique per router.
Each router determines its cost to reach the given prefix.


\begin{figure}
   \caption{For router \hostname{blue} to reach network 10.0.0.0/24, traffic must pass two more routers, making the hop count a value of two}
   \label{fig:routing-rip-hop-count}
\end{figure}


\abbr{RIP} is only interesting for smaller, stable networks as it is slow to converge and it has a maximum hop count of fifteen.
The value sixteen means a network is unreachable.
It implements \emph{split horizong}, \emph{route poisoning} and \emph{holddown} mechanisms to prevent incorrect routing information from being propagated.

\begin{description}
\item[split horizon]
This mechanism prevents a router from advertising prefixes it receives on an interface back out on that same interface.
\item[route poisoning]
When a network becomes unavailable, instead of not advertising it in subsequent routing updates, advertise it with a maximum metric value to indicate it is unavailable.
The prefix is thus explicitly advertised as being unavailable instead of implicitly by its absence from the updates.
\item[holddown timer]
When a network becomes unavailable the router starts a holddown timer.
During this period it will not accept updates from routers that this network is in fact reachable.
This can be stated as ``bad news travels fast, good news travels slow.''
\end{description}


\section{Open Shortest Path First}
\label{sec:ospf}

\section{Border Gateway Protocol}
\label{sec:bgp}

