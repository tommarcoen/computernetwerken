\Chapter{Ethernet}
\label{chap:ethernet}

\Paragraph{\acs{MAC} addresses}
\mode<article>{
A \ac{MAC} address is a unique identifier that is assigned to a \ac{NIC} for use as an address in communication within a network segment.
\ac{MAC} addresses are primarily assigned by device manufacturers and stored in read-only memory.
As such they are often referred to as the \emph{burned-in address}, or as a \emph{hardware address}, or \emph{physical address}.
Nowadays it is very often possible to change the \acs{MAC} adddress used on an interface.
}

\Paragraph{48 bits}
\mode<article>{
A \ac{MAC} address consists of 48~bits, most often displayed as six groups of two hexadecimal digits, separated by hyphens or colons (e.g.~02:\-00:\-33:\-aa:\-bb:\-cc).
Something the address is displayed as three groups of four hexadecimal digits, separated by dots, e.g. 0200.33aa.bbcc.%
   \footnote{It is important to note that hexadecimal numbers are not case-sensitive, meaning that the value of 4F is equivalent to that of 4f.}
}

\Paragraph{\acf{OUI}}
\mode<article>{
The first 24 bits of a \ac{MAC} address identify the vendor or manufacturer of the network card.
\Acp{OUI} are purchased from the \acs{IEEE} registration authority by the vendor.
They are used to make sure that -- in theory at least -- every \acs{MAC} address is unique.
}

\Paragraph{1522 bytes}
\mode<article>{
Ethernet frames have a maximum size of 1522~bytes and a minimum frame size of 64~bytes.
Twenty-two bytes are used by the Ethernet header (18~bytes) and trailer (4~bytes) so that the encapsulated \acs{IP} packet has a maximum size of 1500~bytes.
}

\Paragraph{Ethernet header}
\mode<article>{
The standard length of the Ethernet header is 14~bytes, however, in the case of utilizing a \acs{VLAN} tag, it is extended to 18~bytes.
It contains the source and destination \acs{MAC} addresses, a field designated for either the \emph{ethertype} or the length of the data payload, and an optional \abbr{802.1Q} tag containg \acs{VLAN} and \acs{CoS} information.

The ethertype field is a two-byte field in the Ethernet header that is used to identify the type of data that is being transmitted in the Ethernet frame.
It is used to indicate the protocol that is encapsulated in the payload of the Ethernet frame.
The ethertype field is located immediately after the destination and source \acs{MAC} addresses in the Ethernet header.
% TODO: Expand on the ethertype and the 802.1Q tag.
% TODO: Explain the difference between the length and ethertype and how the contents or the length is calculated when not provided.
}

\Paragraph{Ethernet trailer}
% TODO: Explain

\Paragraph{jumbo frames} 
\mode<article>{
Some implementations of Gigabit Ethernet and other higher-speed variants of Ethernet support larger frames, known as jumbo frames.
These frames can be up to 9022 bytes in size.
Jumbo frames are mainly used in data centres as the use of these larger frames require all intermediate devices to support them which is impossible to require from devices on the internet.
}


\Section{Ethernet switch}
\label{sec:ethernet-switch}

\Paragraph{hub}
\mode<article>{
An Ethernet hub has multiple \ac{IO} ports, in which a signal introduced at the input of any port appears at the output of every port except the original incoming.
A hub works at the physical layer (layer 1) of the \ac{OSI} model.
}

\Paragraph{\acs{MAC}-address table}
\mode<article>{
A network switch learns the identities of connected devices from incoming frames and then only forwards data to the port connected to the device to which it is addressed.
These learnt \acs{MAC} addresses are stored in the \ac{MAC}-address table.
}

\Paragraph{time to live?}
\mode<article>{
Ethernet has no \acl{TTL} field so when a loop gets created in the network, the frames will keep traversing the network, bringing the entire network down.
}

\Paragraph{\acf{STP}}
\mode<article>{
The \acl{STP} is a network protocol that builds a loop-free logical topology for Ethernet networks.
The basic function of \acs{STP} is to prevent bridge loops and the broadcast radiation that results from them.
\acs{STP} also allows a network design to include backup links providing fault tolerance if an active link fails.
}

\Paragraph{Flavours of \acs{STP}}
% TODO: Explain

\Paragraph{\acf{LAG}}
\mode<article>{
Link aggregation is the combining (aggregating) of multiple network connections in parallel by any of several methods, in order to increase throughput beyond what a single connection could sustain, to provide redundancy in case one of the links should fail, or both.
A \ac{LAG} is the combined collection of physical ports.
Other umbrella terms used to describe the concept include trunking%
   \footnote{Not to be confused with \acs{VLAN} trunking which is something entirely different that link aggregation trunking.}%
, bundling, bonding, channeling or teaming.
}

\Paragraph{managed switch}
\mode<article>{
Managed switches have one or more methods to modify the operation of the switch.
Common management methods include: a \ac{CLI} accessed via serial console, Telnet or \ac{SSH} (see \vref{chap:applications}), an embedded \ac{SNMP} agent allowing management from a remote console or management station, or a web interface for management from a web browser.
Examples of configuration changes that one can do from a managed switch include: enabling features such as \ac{STP} or port mirroring, setting port bandwidth, creating or modifying \acp{VLAN}, etc.

We will take a quick look at the following features of managed switches:
\begin{inlinelist}
\item \aclp{VLAN},
\item port security and \abbr{802.1X},
\item \acs{DHCP} snooping, and
\item \acl{DAI}.
\end{inlinelist}
}

\Paragraph{\acs{VLAN}}
\mode<article>{
   A \acf{VLAN} is any broadcast domain that is partitioned and isolated in a computer network at the data link layer.
   \acp{VLAN} work by applying tags to network frames and handling these tags in networking systems -- creating the appearance and functionality of network traffic that is physically on a single network but acts as if it is split between separate networks.
   In this way, \acp{VLAN} can keep network applications separate despite being connected to the same physical network, and without requiring multiple sets of cabling and networking devices to be deployed.
}

\Paragraph{port security}
\mode<article>{
   \ac{MAC} \emph{filtering} is a security access control method whereby the \acs{MAC} address assigned to each network card is used to determine access to the network.
   \ac{MAC} filtering on a network permits and denies network access to specific devices through the use of blacklists and whitelists.
   Many devices that support \acs{MAC} filtering do so on a device basis.
   Whitelisted \ac{MAC} addresses are allowed through any port on the device and blacklisted \acs{MAC} addresses are blocked on all ports.
   Other devices, such as Cisco Catalyst switches, support \acs{MAC} filtering on a port-by-port basis.
   This is referred to as port security.
}

\Paragraph{\abbr{802.1X}}
\mode<article>{
\acs{IEEE} \abbr{802.1X} is an \acs{IEEE} standard for \ac{PNAC}.
It provides an authentication mechanism to devices wishing to attach to a \ac{LAN} or \ac{WAN}.

\abbr{802.1X} authentication involves three parties: a supplicant, an authenticator, and an authentication server.
The supplicant is a client device (such as a laptop) that wishes to attach to the \ac{LAN} or \ac{WAN}.
The authenticator is a network device that provides a data link between the client and the network and can allow or block network traffic between the two, such as an Ethernet switch or wireless access point.
The authentication server is typically a trusted server that can receive and respond to requests for network access, and can tell the authenticator whether the connection is to be allowed, and various settings that should apply to that client's connection.
Authentication servers typically run software supporting the \ac{RADIUS} and \ac{EAP} protocols.
}

\Paragraph{\acs{DHCP} snooping}
\mode<article>{
\ac{DHCP} snooping is a series of techniques applied to improve the security of a \ac{DHCP} infrastructure.
\ac{DHCP} snooping can be configured on \ac{LAN} switches to exclude rogue \ac{DHCP} servers and remove malicious or malformed \ac{DHCP} traffic.
In addition, information on hosts which have successfully completed a \ac{DHCP} transaction is accrued in a database of bindings which may then be used by other security or accounting features, such as \acl{DAI}.

The \ac{DHCP} snooping feature performs the following activities:
\begin{itemize}
\item it validates \acs{DHCP} messages received from untrusted sources and filters out invalid messages.
\item it rate-limits \acs{DHCP} traffic both from trusted and untrusted sources.
\item it builds and maintains the \acs{DHCP} snooping database, which contains information about untrusted hosts with leased \acs{IP} addresses.
\item it utilises the \acs{DHCP} snooping binding database to validate subsequent requests from untrusted hosts.
\end{itemize}
}

\Paragraph{\acl{DAI}}
\mode<article>{
\Acf{DAI} is a security feature that validates \acs{ARP} packets in a network.
\Acl{DAI} allows a network administrator to intercept, log, and discard \ac{ARP} packets with invalid \acs{MAC}-address-to-\acs{IP}-address bindings.
This capability protects the network from certain man-in-the-middle attacks.
}
\slide{\acf{DAI}}

\Paragraph{layer 3 switch}
\mode<article>{
As engineers continue to advance in their field, they have developed the ability to create larger and more complex chips.
This advancement allows for the integration of more complex logic into network switches.
Layer-3 switches, in particular, have the capability to route packets in hardware based on \acs{IP} addresses, effectively functioning as routers.
As a result, it has become increasingly convenient to implement a layer-3 switch into network infrastructure, as it combines the functionality of both routers and traditional switches.
}