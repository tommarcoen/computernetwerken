\Chapter{Ethernet}
\label{chap:ethernet}

\Paragraph{MAC addresses}
\mode<article>{
A media access control (MAC) address is a unique identifier assigned to a network interface controller (NIC) for use as a network address in communications within a network segment.
MAC addresses are primarily assigned by device manufacturers, and are therefore often referred to as the \emph{burned-in address}, or as a \emph{hardware address}, or \emph{physical address}.
}

\Paragraph{48 bits}
\mode<article>{
A MAC address consists of 48~bits, most often displayed as six groups of two hexadecimal digits, separated by hyphens or colons (e.g.~02:\-00:\-33:\-AA:\-BB:\-CC).
Something the address is displayed as three groups of four hexadecimal digits, separated by dots, e.g. 0200.33aa.bbcc.
}

\Paragraph{organisationally unique identifier}
\mode<article>{
The first 24 bits of a MAC address identify the vendor or manufacturer of the network card.
OUIs are purchased from the IEEE registration authority by the vendor.
They are used to uniquely identify a particular piece of equipment through derived identifiers such as the MAC addresses.
}

\Paragraph{1522 bytes}
\mode<article>{
Ethernet frames have a maximum size of 1522~bytes and a minimum frame size of 64~bytes.}

\Paragraph{jumbo frames} 
\mode<article>{
Some implementations of Gigabit Ethernet and other higher-speed variants of Ethernet support larger frames, known as jumbo frames.
These frames can be up to 9022 bytes in size.
}


\Section{Ethernet switch}
\label{sec:ethernet-switch}

\Paragraph{hub}
\mode<article>{
An Ethernet hub has multiple input/output (I/O) ports, in which a signal introduced at the input of any port appears at the output of every port except the original incoming.
A hub works at the physical layer (layer 1) of the OSI model.
}

\Paragraph{MAC-address table}
\mode<article>{
A network switch learns the identities of connected devices and then only forwards data to the port connected to the device to which it is addressed.
These learnt MAC addresses are stored in the MAC-address table.
}

\Paragraph{time to live?}
\mode<article>{
Ethernet has to time-to-live field so when a loop gets created in the network, the frames will keep traversing the network, bringing the entire network down.
}

\Paragraph{Spanning-Tree Protocol}
\mode<article>{
The Spanning-Tree Protocol (STP) is a network protocol that builds a loop-free logical topology for Ethernet networks.
The basic function of STP is to prevent bridge loops and the broadcast radiation that results from them.
STP also allows a network design to include backup links providing fault tolerance if an active link fails.
}

\Paragraph{link aggregation groups}
\mode<article>{
Link aggregation is the combining (aggregating) of multiple network connections in parallel by any of several methods, in order to increase throughput beyond what a single connection could sustain, to provide redundancy in case one of the links should fail, or both.
A link aggregation group (LAG) is the combined collection of physical ports.
Other umbrella terms used to describe the concept include trunking, bundling, bonding, channeling or teaming.
}

\Paragraph{managed switch}
\mode<article>{
Managed switches have one or more methods to modify the operation of the switch.
Common management methods include: a command-line interface (CLI) accessed via serial console, telnet or Secure Shell (see \vref{chap:applications}), an embedded Simple Network Management Protocol (SNMP) agent allowing management from a remote console or management station, or a web interface for management from a web browser.
Examples of configuration changes that one can do from a managed switch include: enabling features such as Spanning-Tree Protocol or port mirroring, setting port bandwidth, creating or modifying VLANs (virtual LANs), etc.
\begin{description}
\item[VLAN]
   A virtual local area network (VLAN) is any broadcast domain that is partitioned and isolated in a computer network at the data link layer.
   VLANs work by applying tags to network frames and handling these tags in networking systems --- creating the appearance and functionality of network traffic that is physically on a single network but acts as if it is split between separate networks.
   In this way, VLANs can keep network applications separate despite being connected to the same physical network, and without requiring multiple sets of cabling and networking devices to be deployed.
\item[port security]
   MAC \emph{filtering} is a security access control method whereby the MAC address assigned to each network card is used to determine access to the network.
   MAC filtering on a network permits and denies network access to specific devices through the use of blacklists and whitelists.
   Many devices that support MAC filtering do so on a device basis.
   Whitelisted MAC addresses are allowed through any port on the device and blacklisted MAC addresses are blocked on all ports.
   Other devices, such as Cisco Catalyst switches, support MAC filtering on a port-by-port basis.
   This is referred to as port security.
\item[802.1X]
   IEEE 802.1X is an IEEE standard for port-based network access control (PNAC).
   It provides an authentication mechanism to devices wishing to attach to a LAN or WAN.
   
   802.1X authentication involves three parties: a supplicant, an authenticator, and an authentication server.
   The supplicant is a client device (such as a laptop) that wishes to attach to the LAN or WAN.
   The authenticator is a network device that provides a data link between the client and the network and can allow or block network traffic between the two, such as an Ethernet switch or wireless access point; and the authentication server is typically a trusted server that can receive and respond to requests for network access, and can tell the authenticator if the connection is to be allowed, and various settings that should apply to that client's connection or setting.
   Authentication servers typically run software supporting the RADIUS and EAP protocols.
\item[DHCP snooping]
   DHCP snooping is a series of techniques applied to improve the security of a DHCP infrastructure.
   DHCP servers allocate IP addresses to clients on a LAN.
   DHCP snooping can be configured on LAN switches to exclude rogue DHCP servers and remove malicious or malformed DHCP traffic.
   In addition, information on hosts which have successfully completed a DHCP transaction is accrued in a database of bindings which may then be used by other security or accounting features.
   
   The DHCP snooping feature performs the following activities:
   \begin{itemize}
   \item validates DHCP messages received from untrusted sources and filters out invalid messages.
   \item rate-limits DHCP traffic both from trusted and untrusted sources.
   \item builds and maintains the DHCP snooping database, which contains information about untrusted hosts with leased IP addresses.
   \item Utilises the DHCP snooping binding database to validate subsequent requests from untrusted hosts.
   \end{itemize}
\item[dynamic ARP inspecition]
   Dynamic ARP inspection (DAI) is a security feature that validates Address Resolution Protocol (ARP) packets in a network.
   DAI allows a network administrator to intercept, log, and discard ARP packets with invalid MAC-address-to-IP-address bindings.
   This capability protects the network from certain ``man-in-the-middle'' attacks.
\end{description}
}
\slide{VLAN}
\slide{port security}
\slide{802.1X}
\slide{DHCP snooping}
\slide{dynamic ARP inspection}