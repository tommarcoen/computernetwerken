\chapter{Ethernet}
\label{chap:ethernet}

In dit eerste hoofdstuk gaan we dieper in op Ethernet, het dominante protocol op de datalinklaag.
Twee netwerktoestellen die met elkaar willen communiceren, versturen \emph{frames} naar elkaar.
Dit frame bevat een aantal belangrijke velden (\cref{fig:ethernet-frame}).
\begin{description}
    \item[destination address]
    het \abbr{MAC}-adres van het toestel voor wie het frame bestemd is.
    
    \item[source address]
    het \abbr{MAC}-adres van het toestel dat het frame verstuurt.
    
    \item[802.1q tag]
    dit veld wordt gebruikt om virtuele netwerken te maken (\vref{sec:vlan}).
    
    \item[ethertype]
    dit veld bevat een code die aangeeft welk soort data zich in het frame bevindt; bijvoorbeeld 0x0800 geeft aan dat de data een \abbr{IPv4}-pakket is, terwijl 0x86dd aangeeft dat de inhoud een \abbr{IPv6}-pakket is.
    
    \item[\abbr{CRC}]
    een eenvoudige \emph{checksum} die gebruikt wordt door de ontvanger om te controleren dat de data intact is aangekomen.
\end{description}

Data die binnen komt op een interface van een repeater of een hub (ook wel een multiport repeater genoemd), wordt langs iedere andere interface weer naar buiten gestuurd.
Hubs zijn dus domme elektrische aparaten zonder enige intelligentie.

Als een frame daarentegen aankomt op een switch interface, voert de switch enkele acties uit.
\begin{enumerate}
    \item
    De switch berekent de \abbr{CRC} en controleert deze met de waarde achteraan in het frame. Als beide waardes niet kloppen, wordt het frame verwijderd.
    
    \item
    Als de switch het source \abbr{MAC}-adres nog niet kent, wordt deze opgeslagen in de \abbr{MAC} \emph{address table}, een tabel die bijhoudt op welke interface een \abbr{MAC}-adres gezien werd.
    
    \item
    De switch zoekt het destination \abbr{MAC}-adres op in dezelfde \abbr{MAC} address table.
    Als de switch het \abbr{MAC}-adres terugvindt, stuurt de switch het frame enkel via de bijhorende interface naar buiten.
    Als de switch het \abbr{MAC}-adres niet kent, stuurt deze het frame langs alle interfaces naar buiten.
    We spreken dan van een \emph{unknown unicast}.
    
    \item
    Multicast- en broadcastverkeer worden steeds langs alle interfaces naar buiten gestuurd.%
    \footnote{Dit klopt niet voor multicastverkeer als de \abbr{IGMP} \emph{snooping} feature geconfigureerd wordt op de switch. Deze feature wordt verder niet besproken in deze cursus.}
\end{enumerate}

\begin{figure}
    \centering
    \begin{bytefield}[bitwidth=1.1em]{16}
\bitheader{0-15} \\
\wordbox{3}{destination \abbr{MAC} address} \\
\wordbox{3}{source \abbr{MAC} address} \\
\wordbox{2}{\abbr{802.1Q} tag} \\
\wordbox{1}{ethertype} \\
\wordbox{3}{payload} \\
\wordbox{2}{\abbr{CRC}} \\
\end{bytefield}
    \caption{802.3 Ethernetframe}
    \label{fig:ethernet-frame}
\end{figure}



\section{Lussen in het netwerk}
\label{sec:stp}

Er schuilt een subtiel gevaar in de manier waarop switches werken.
Neem bijvoorbeeld de opstelling in \vref{fig:drie-switches}.
In het netwerk bestaande uit drie switches, werd een lus gemaakt door de eerste en de derde switch met elkaar te verbinden via een netwerkkabel.
Op deze manier blijft netwerkconnectiviteit behouden ongeacht welk van de drie kabels defect gaat.


\begin{figure}
    \centering
    \begin{tikzpicture}

\node [switch,label={[label distance=1mm]180:$a$}] (a) at (-10mm,0) {};
\node [switch,label={[label distance=1mm]0:$b$}] (b) at (10mm,0) {};
\node [switch,label={[label distance=1mm]0:$c$}] (c) at (0,15mm) {};

\draw[thick] (a) -- (b) -- (c) -- (a);
\end{tikzpicture}
    \caption{Een eenvoudig netwerk met redundantie\ldots\ en het gevaar voor \emph{network loops}}
    \label{fig:drie-switches}
\end{figure}


We zullen twee situaties bespreken.
In de eerste situatie is een computer $A$ verbonden met switch $a$ en een computer $B$ met switch $b$.
Verder kennen alle switches de \abbr{MAC}-adressen van beide computers.
In de tweede situatie zijn dezelfde computers verbonden met dezelfde switches, maar kennen de switches de \abbr{MAC}-adressen niet.


\section{Virtuele switches}
\label{sec:vlan}
Het is wenselijk om bepaalde toestellen van elkaar gescheiden te houden.
Zo wil je bijvoorbeeld niet dat bezoekers op het gastennetwerk kunnen printen of aan je boekhouding kunnen.
Het toepassen van \emph{firewall} regels gaat nagenoeg altijd gepaard met het maken van verschillende netwerken, dus met de introductie van een \emph{router} (zie later).
\sidenote{\begin{tikzpicture}

\node [router] (r) at (0,15mm) {};
\node [switch] (s1) at (-15mm,0) {};
\node [switch] (s2) at (0,0) {};
\node [switch] (s3) at (15mm,0) {};
\node[draw=thered,dashed,rounded corners,inner sep=2mm,fit=(s1) (s2) (s3)] {};

% LAN above router
%\coordinate (a) at ($(r) + (-1.5,1.2)$);
%\coordinate (b) at ($(r) + (1.5,1.2)$);
%\draw [thick] (a) -- (b);
%\draw [fill=white,draw=black,thick] ($(a) - (.07,.07)$) rectangle ($(a) + (.07,.07)$);
%\draw [fill=white,draw=black,thick] ($(b) - (.07,.07)$) rectangle ($(b) + (.07,.07)$);

% LAN below router
%\coordinate (c) at ($(r) + (-1.5,-1.2)$);
%\coordinate (d) at ($(r) + (1.5,-1.2)$);
%\draw [thick] (c) -- (d);
%\draw [fill=white,draw=black,thick] ($(c) - (.07,.07)$) rectangle ($(c) + (.07,.07)$);
%\draw [fill=white,draw=black,thick] ($(d) - (.07,.07)$) rectangle ($(d) + (.07,.07)$);

% Connect router to LANs
\draw[thick] (s1) -- (r) -- (s2);
\draw[thick] (s3) -- (r);
%\draw[thick] (r) -- ($(a)!0.5!(b)$);
%\draw[thick] (r) -- ($(c)!0.5!(d)$);

\end{tikzpicture}\par Hier komt een hele uitleg over bovenstaande tekening en hoe deze relevant is voor de hoofdtekst op deze pagina.}

Het is echter niet wenselijk om voor elk netwerk een aparte switch te gebruiken.
Stel dat je maar vier printers hebt, zou dat zonde zijn van de overige vierenveertig poorten op de switch die onbenut blijven.
Om dit op te lossen, werd het mogelijk gemaakt om virtuele switches of \abbr{VLAN}'s te definiëren op een switch.
Vervolgens kan je elke interfaces aan een bepaald \abbr{VLAN} koppelen.

\begin{iosconfig}{Het maken van \abbr{VLAN}'s op een Cisco switch}
vlan 100
   name DMZ
vlan 200
   name printers
vlan 300
   name boekhouding
\end{iosconfig}

\begin{iosconfig}{Het toekennen van een \abbr{VLAN} aan een interface}
interface GigabitEthernet1/1
   description Printer eerste verdieping
   switchport mode access
   switchport access vlan 200
!
\end{iosconfig}

Eerst geven we aan welke interface we wensen te configureren.
Vervolgens geven we een omschrijving mee.
Dit is louter informatief.
De derde regel plaatst de interface in de modus \emph{access}; de andere mogelijkheid is \emph{trunk}.
Als laatste geven we aan in welk virtueel netwerk we deze interface wensen te plaatsen.
In dit geval is het \abbr{VLAN}~200, het printer-\abbr{VLAN}.

\subsection{Trunkpoorten}