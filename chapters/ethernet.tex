\chapter{Ethernet}
\label{chap:ethernet}

In dit eerste hoofdstuk gaan we dieper in op Ethernet, het dominante protocol op de datalinklaag.
Twee netwerktoestellen die met elkaar willen communiceren, versturen \emph{frames} naar elkaar.
Dit frame bevat een aantal belangrijke velden (\cref{fig:ethernet-frame}).
\begin{description}
    \item[destination address]
    het \abbr{MAC}-adres van het toestel voor wie het frame bestemd is.
    
    \item[source address]
    het \abbr{MAC}-adres van het toestel dat het frame verstuurt.
    
    \item[802.1q tag]
    dit veld wordt gebruikt om virtuele netwerken te maken (\vref{sec:vlan}).
    
    \item[ethertype]
    dit veld bevat een code die aangeeft welk soort data zich in het frame bevindt; bijvoorbeeld 0x0800 geeft aan dat de data een \abbr{IPv4}-pakket is, terwijl 0x86dd aangeeft dat de inhoud een \abbr{IPv6}-pakket is.
    
    \item[\abbr{CRC}]
    een eenvoudige \emph{checksum} die gebruikt wordt door de ontvanger om te controleren dat de data intact is aangekomen.
\end{description}

Data die binnen komt op een interface van een repeater of een hub (ook wel een multiport repeater genoemd), wordt langs iedere andere interface weer naar buiten gestuurd.
Hubs zijn dus domme elektrische aparaten zonder enige intelligentie.

Als een frame daarentegen aankomt op een switch interface, voert de switch enkele acties uit.
\begin{enumerate}
    \item
    De switch berekent de \abbr{CRC} en controleert deze met de waarde achteraan in het frame. Als beide waardes niet kloppen, wordt het frame verwijderd.
    
    \item
    Als de switch het source \abbr{MAC}-adres nog niet kent, wordt deze opgeslagen in de \abbr{MAC} \emph{address table}, een tabel die bijhoudt op welke interface een \abbr{MAC}-adres gezien werd.
    
    \item
    De switch zoekt het destination \abbr{MAC}-adres op in dezelfde \abbr{MAC} address table.
    Als de switch het \abbr{MAC}-adres terugvindt, stuurt de switch het frame enkel via de bijhorende interface naar buiten.
    Als de switch het \abbr{MAC}-adres niet kent, stuurt deze het frame langs alle interfaces naar buiten.
    We spreken dan van een \emph{unknown unicast}.
\end{enumerate}

\begin{figure}
    \centering
    \begin{bytefield}[bitwidth=1.1em]{16}
\bitheader{0-15} \\
\wordbox{3}{destination \abbr{MAC} address} \\
\wordbox{3}{source \abbr{MAC} address} \\
\wordbox{2}{\abbr{802.1Q} tag} \\
\wordbox{1}{ethertype} \\
\wordbox{3}{payload} \\
\wordbox{2}{\abbr{CRC}} \\
\end{bytefield}
    \caption{802.3 Ethernetframe}
    \label{fig:ethernet-frame}
\end{figure}



\section{Lussen in het netwerk}
\label{sec:stp}


\section{Virtuele netwerken}
\label{sec:vlan}