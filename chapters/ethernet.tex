\section{Ethernet}
\label{sec:ethernet}

In \vref{sec:arp} hadden we gezien hoe een computer het destination MAC-adres bekomt als hij een frame wilt versturen over het netwerk.
In dit hoofdstuk leren we hoe switches deze frames naar hun bestemming sturen en welke gevaren er schuil gaan in een netwerk op laag~2.


\begin{figure}[hbp]
    \centering
    \includegraphics[width=\textwidth]{images/Ethernet_frame.png}
    \caption{An Ethernet frame inside an Ethernet packet, with SFD marking the end of the packet preamble and indicating the beginning of the frame}
    \label{fig:ethernet-frame}
\end{figure}


\Vref{fig:ethernet-frame} toont de velden in een Ethernet frame.
Eerst komen de \emph{preamble} en de \emph{start of frame delimiter} (SFD).
Deze dienen om het kloksignaal van de ontvanger af te stemmen op het kloksignaal van de zender zodat beide machines weten hoe lang het signaal van een 0 of een 1 duurt.

Hierna begint het eigenlijke Ethernet frame met het destination MAC-adres, gevolgd door het MAC-adres van de zender.
De \emph{EtherType} vertelt de ontvanger wat voor data er wordt verstuurd.
Enkele voorbeelden van geldige EtherTypes worden gegeven in \vref{tab:ethertype}.
Hierna komt de eigenlijke data of de \emph{payload} en ten slotte komt de \emph{frame check sequence} (FCS).
Dit is een code die gebruikt wordt om te detecteren of er fouten in het frame zitten.
Als het frame corrupt is, wordt deze gewoon verwijderd.
De ontvanger verwittigt de zender van het frame niet dat het frame niet goed is aangekomen.

\begin{table}[htp]
    \centering
    \begin{tabular}{ll}
    \textit{EtherType} & \textit{protocol} \\[1ex]
    0x0800 & IPv4 \\
    0x0806 & address resolution protocol (ARP) \\
    0x8100 & VLAN-tagged frame \\
    0x86DD & IPv6 \\
    \end{tabular}
    \caption{Enkele mogelijke waarden van de EtherType in een Ethernet frame}
    \label{tab:ethertype}
\end{table}


\subsection{Ethernetswitches}

Een Ethernetswitch ziet er niet anders uit dan een hub.
Een hub werkt echter heel anders dan een switch.
De interfaces van een hub zijn rechtstreeks met elkaar verbonden door middel van koperdraadjes zodat een elektrisch signaal dat binnenkomt op een van de interfaces, doorgestuurd wordt naar alle andere interfaces van de hub.
Een hub bevat geen intelligentie en kan niet geconfigureerd worden.

Een switch daarentegen is een toestel met heel wat intelligentie.
Er bestaan twee soorten switches: \emph{managed} en \emph{unmanaged} switches.
Unmanaged switches hebben de nodige intelligentie maar kunnen verder niet geconfigureerd worden.
Managed switches daarentegen beschikken over een volledig besturingssysteem en kunnen over zeer veel geavanceerde features beschikken.

\Vref{fig:ethernetswitch} toont een Juniper switch met 48~poorten of interfaces.
Rechts onder de display bevinden zich nog vier ``gaten'' in de switch.
Dit zijn sloten waar SFP-modules geplaatst kunnen worden (zie \vref{sec:sfp-modules}).

\begin{figure}
    \centering
    \includegraphics[width=\textwidth]{images/ethernetswitch.jpg}
    \caption{Een Ethernetswitch met 48 poorten en 4 openingen voor SFP-modules}
    \label{fig:ethernetswitch}
\end{figure}


In \vref{fig:simple-lan} zijn drie computers met elkaar verbonden via een switch.
Computer~1 wil communiceren met computer~2.
We gaan er even van uit dat beide computers elkaars IP-adres en MAC-adres kennen.
Als de switch de eerste frame ontvangt van computer~1, weet het nog niets over het netwerk of de toestellen die verbonden zijn met de switch, dus stuurt het het frame via alle interfaces naar buiten.
Computers~2 en~3 ontvangen dus dit eerste frame.
Computer~3 ziet het destination MAC-adres van het frame en, omdat dit niet zijn MAC-adres is, gooit het frame weg.
Computer~2 herkent wel zijn eigen MAC-adres in het frame en verwerkt de inhoud en stuurt uiteindelijk een antwoord terug.

De switch stuurt niet alleen het frame langs alle interfaces naar buiten, maar maakt ook notitie van het source MAC-adres en noteert dit in de MAC-tabel samen met de interface waarop hij het MAC-adres geleerd heeft.
\begin{center}
\begin{tabular}{cc}
\textit{MAC-adres} & \textit{interface} \\[1ex]
0200.aaaa.0001 & fa0/1\\
\end{tabular}
\end{center}
Als computer~2 nu een antwoord stuurt naar computer~1, komt dit frame aan op fa0/2 van de switch.
Weer noteert de switch het source MAC-adres in de MAC-tabel samen met de interface waarop de switch het frame ontvangt.
Vervolgens zoekt de switch het destination MAC-adres op in de MAC-tabel, vindt de bijhorende interface, en verstuurt het frame enkel langs deze interface naar buiten.
Computer~3 ziet dit antwoord -- en alle verder communicatie tussen computers~1 en~2 -- niet meer.

\begin{figure}
    \centering
    \documentclass[tikz]{standalone}
\usepackage{ifthen}
\usepackage{contour}
%%
%% Medtronic color palette
%%

\usepackage{xcolor}

% Primary blue color palette
\definecolor{mdtnavyblue}{cmyk}{1,.94,.47,.43}
\definecolor{mdtmedtronicblue}{cmyk}{.99,.74,.17,.04}
\definecolor{mdtcobaltblue}{cmyk}{.81,.35,0,0}
\definecolor{mdtmediumblue}{cmyk}{.73,.12,0,0}
\definecolor{mdtskyblue}{cmyk}{.55,.04,.04,0}
\definecolor{mdtlightblue}{cmyk}{.29,.05,.05,0}

% Primary neutral color palette
\definecolor{mdtcharcoal}{rgb}{.83,.86,.90}
\definecolor{mdtbluegray}{cmyk}{.68,.40,.28,.08}
\definecolor{mdtdarkgray}{cmyk}{0,0,0,.55}
\definecolor{mdtlightgray}{cmyk}{0,0,0,.30}
\definecolor{mdtpalegray}{cmyk}{.08,.06,.06,0}

% Accent colors
\definecolor{mdtyellow}{cmyk}{0,.18,.93,0}
\definecolor{mdtlightorange}{cmyk}{.02,.38,.96,0}
\definecolor{mdtorange}{cmyk}{.04,.78,1,0}
\definecolor{mdtpurple}{cmyk}{.38,.98,0,0}
\definecolor{mdtgreen}{cmyk}{.57,.02,1,0}
\definecolor{mdtturquoise}{cmyk}{.70,0,.38,0}

% TikZ/PGF
\usepackage{tikz}
\usetikzlibrary{backgrounds}
\usetikzlibrary{calc}
\usetikzlibrary{decorations.pathreplacing}
\usetikzlibrary{fit}
\usetikzlibrary{matrix}
\usetikzlibrary{patterns}
\usetikzlibrary{positioning}
\usetikzlibrary{shapes.geometric}
\usetikzlibrary{arrows}

%%
%% Styles
%%

% Network nodes
\tikzset{network node image/.style={
                inner sep=0mm,
                %anchor=south west
                }}
\tikzset{network node/.style={
                network node image,
                thick,
                text=mdtmedtronicblue,
                draw=mdtmedtronicblue,
                fill=white,
                minimum size=10mm}}
\tikzset{network node circle/.style={
                network node,
                circle}}
\tikzset{network node square/.style={
                network node,
                rectangle,
                rounded corners}}
\tikzset{network node rectangle/.style={
                network node,
                rectangle,
                minimum height=15mm,
                rounded corners}}
\tikzset{server rack/.style={
                align=left,
                text width=8mm,
                inner sep=0pt,
                anchor=north west,
                minimum height=7.5mm-.8pt,
                minimum width=10.7mm-.8pt,
                thick,
                text=mdtmedtronicblue,
                draw=mdtmedtronicblue,
                fill=white}}
\tikzset{network rack/.style={
                align=left,
                text width=9.3mm,
                inner sep=0pt,
                anchor=north west,
                minimum height=7.5mm-.8pt,
                minimum width=12mm-.8pt,
                %minimum height=7.5mm,
                %minimum width=12mm,
                thick,
                text=mdtmedtronicblue,
                draw=mdtmedtronicblue,
                fill=white}}
\tikzset{patch rack/.style={
                align=left,
                text width=9.3mm,
                inner sep=0pt,
                anchor=north west,
                minimum height=7.5mm-.8pt,
                minimum width=12mm-.8pt,
                thick,
                text=mdtmedtronicblue,
                draw=mdtmedtronicblue,
                fill=white}}

% Labels
\tikzset{interface label/.style={
                font=\footnotesize,
                inner sep=2pt},
                fill=white}
\tikzset{node label/.style={
                font=\small,
                align=center,
                inner sep=2pt,
                thick,
                rounded corners}}


% Floor tiles
\newcommand{\dcfloortiles}[2]{%
    \draw[step=6mm,gray,very thin] (0,0) grid ($.6*(#1,#2)$);
}
\newcommand{\dccoldaisle}[2]{%
    \draw[pattern=north west lines, pattern color=mdtlightgray] ($.6*(#1)-(.6,.6)$) rectangle ($.6*(#1)+.6*(#2)-(.6,.6)$);
}
% Markings for different types of racks
\newcommand{\caserack}[1]{%
    \draw[very thick,mdtorange] ($(#1.north east)-(.1,.1)$) -- ($(#1.south east)+(-.1,.1)$);
}
\newcommand{\storagerack}[1]{%
    \draw[very thick,mdtgreen] ($(#1.north east)-(.2,.1)$) -- ($(#1.south east)+(-.2,.1)$);
}
\newcommand{\networkrack}[1]{%
    \draw[very thick,mdtmediumblue] ($(#1.north east)-(.15,.1)$) -- ($(#1.south east)+(-.15,.1)$);
}
\newcommand{\mitgrack}[1]{%
    \draw[very thick,mdtpurple] ($(#1.north east)-(.25,.1)$) -- ($(#1.south east)+(-.25,.1)$);
}
\newcommand{\unixrack}[1]{%
    \draw[very thick,mdtdarkgray] ($(#1.north east)-(.3,.1)$) -- ($(#1.south east)+(-.3,.1)$);
}
\newcommand{\computerack}[1]{%
    \draw[very thick,mdtturquoise] ($(#1.north east)-(.35,.1)$) -- ($(#1.south east)+(-.35,.1)$);
}
\newcommand{\racklabels}[1]{%
    \matrix (racklabels) at (#1) [matrix of nodes,
        nodes={
            anchor=west,
            align=left
            }
        ] {
            |[mdtorange]| CASE \\
            |[mdtturquoise]| Compute (CORE) \\
            |[mdtpurple]| MITG \\
            |[mdtmediumblue]| Network \\
            |[mdtgreen]| Storage \\
            |[mdtdarkgray]| UNIX \\
    };
}
%
% layers.tex
%
% This document contains code to hide a PGF layer if so desired,
% e.g. you can hide the layer with all the labels on it.
%

% Define the layers to be used
\pgfdeclarelayer{connections}
\pgfdeclarelayer{interfaces}
\pgfdeclarelayer{hostnames}
\pgfdeclarelayer{labels}
\pgfdeclarelayer{arrows}

% Make the background white so we can hide the labels by swapping the order of the layers
\tikzset{show background rectangle,background rectangle/.style={rounded corners,fill=white}}

% \showalllayers - Show the nodes, connections, and detailed labels, including interface labels
% The left most layer is the bottom layer; everything left from the background will be hidden.
\newcommand\showalllayers{      \pgfsetlayers{hostnames,arrows,background,connections,main,interfaces,labels}}
\newcommand\showarrows{         \pgfsetlayers{hostnames,background,connections,main,interfaces,labels,arrows}}
% \hidelabels - Show only the nodes and the connections
\newcommand\hidelabels{         \pgfsetlayers{hostnames,arrows,interfaces,labels,background,connections,main}}
% \hideinterfaces - Hide only the interface labels
\newcommand\hideinterfaces{     \pgfsetlayers{hostnames,arrows,interfaces,background,connections,main,labels}}
% \showonlyhostnames - Hide everything but show the hostnames only
\newcommand\showonlyhostnames{  \pgfsetlayers{arrows,interfaces,labels,background,connections,main,hostnames}}
\newcommand\showhostnamesinterfaces{ \pgfsetlayers{arrows,labels,background,connections,main,hostnames,interfaces}}
% Default setting is to show all layers
\showalllayers
%
% connections.tex
%
% This document contains all the commands needed to create connections between
% different network nodes, e.g.
%  - a normal connection, with our without labels.
%  - a vPC connection between 3 or 4 nodes.
%

%
% Syntax:
% \connect[<color>]{<node1>}{<intf1>}{<node2>}{<intf2>}
\newcommand{\connect}[5][black]{

    % Save the coordinates for the interface labels
    \path (#2) -- (#4) coordinate[very near start] (i1) coordinate[midway] (im) coordinate[very near end] (i2);
    
    % Draw the connection between the two devices on the correct layer.
    \begin{pgfonlayer}{connections}
      \draw[color=#1] (#2) -- (#4);
    \end{pgfonlayer}
    
    % Draw the interface labels on the correct layer.
    \begin{pgfonlayer}{interfaces}
        \ifthenelse{\equal{#3}{}}{}{\node[interface label,fill=white] at (i1) {\contour{white}{#3}};}
        \ifthenelse{\equal{#5}{}}{}{\node[interface label,fill=white] at (i2) {\contour{white}{#5}};}
    \end{pgfonlayer}
    
    %\ifthenelse{\equal{#5}{}}{
    %    % Parameter 5 is empty:
    %    \ifthenelse{\equal{#3}{}}{
    %        % Parameter 3 is empty: no interfaces have been given.
    %        \draw[color=#1] (#2) -- (#4);
    %    }{
    %        % Both devices share the same interface name.
    %        \draw[color=#1] (#2) -- (#4)
    %            node [midway,interface] {#3};
    %    }
    %}{
    %    % Parameter 5 is not empty: both interfaces are given.
    %    \draw[color=#1] (#2) -- (#4)
    %        node [on background layer,near start,interface] {#3}
    %        node [near end,interface] {#5};
    %}
}

%
% Syntax:
% \connectClusters[<straight/cross>]{cluster1}{cluster2}
\newcommand{\connectClusters}[3][S]{
    \message{connectClusters 2-1: #2-1^^J}
    \message{connectClusters 2-2: #2-2^^J}
    \message{connectClusters 3-1: #3-1^^J}
    \message{connectClusters 3-2: #3-2^^J}
    \ifthenelse{\equal{#1}{S}}{
        \connect{#2-1}{}{#3-1}{}
        \connect{#2-2}{}{#3-2}{}
    }{
        \connect{#2-1}{}{#3-2}{}
        \connect{#2-2}{}{#3-1}{}
    }
}

%
% Syntax:
% \arrowcoords[<position>]{<coordinates start>}{<coordinates end>}{<text>}
\newcommand{\arrowcoords}[4][midway]{
    \draw[mdtgreen, arrows={-latex'}, shorten <= 0.25cm, shorten >= 0.25cm, very thick, distance=2cm] (#2) to[out=230,in=170] (#3) node[#1] {#4};
}

%
%
%

\def\iconset{3015}


%
% Display only the icons
%
% Syntax: \icon<type>{<hostname>}{<coordinates>}
%

\newcommand{\iconcloud}[2]{
    \node[network node image] (#1) at (#2) {\includegraphics[width=30mm]{images//cloud.png}};
    \node at ([yshift=-2mm]#2) {#1};
}
\newcommand{\iconfirewall}[2]{
    \ifthenelse{\equal{\iconset}{basic}}{
        \node[network node circle,draw=mdtorange] (#1) at (#2) {FW};
    }{
        \node[network node image] (#1) at (#2) {\includegraphics[width=10mm]{images/\iconset/firewall.eps}};
    }
}
\newcommand{\iconips}[2]{
    \ifthenelse{\equal{\iconset}{basic}}{
        \node[network node square] (#1) at (#2) {IPS};
    }{
        \node[network node image] (#1) at (#2) {\includegraphics[width=10mm]{images/\iconset/ips.eps}};
    }
}
\newcommand{\iconrouter}[2]{
    \ifthenelse{\equal{\iconset}{basic}}{
        \node[network node circle] (#1) at (#2) {RTR};
    }{
        \node[network node image] (#1) at (#2) {\includegraphics[width=10mm]{images/\iconset/router.eps}};
    }
    \message{iconrouter: #1^^J}
}
\newcommand{\iconswitch}[2]{
    \ifthenelse{\equal{\iconset}{basic}}{
        \node[network node square] (#1) at (#2) {SW};
    }{
        \node[network node image] (#1) at (#2) {\includegraphics[width=13mm]{images/\iconset/switch.eps}};
    }
}
\newcommand{\icondistswitch}[2]{
    \ifthenelse{\equal{\iconset}{basic}}{
        \node[network node square] (#1) at (#2) {SW};
    }{
        \node[network node image] (#1) at (#2) {\includegraphics[width=13mm]{images/\iconset/l3switch.eps}};
    }
}
\newcommand{\iconcoreswitch}[2]{
    \ifthenelse{\equal{\iconset}{basic}}{
        \node[network node square] (#1) at (#2) {SW};
    }{
        \node[network node image] (#1) at (#2) {\includegraphics[width=13mm]{images/\iconset/coreswitch.eps}};
    }
}
\newcommand{\iconhost}[2]{
   \ifthenelse{\equal{\iconset}{basic}}{
      \node[network node square] (#1) at (#2) {PC};
   }{
      \node[network node square] (#1) at (#2) {PC};
   }
}



%
% Display the node: icon + labels
%
%
% Syntax: \node<type>[<label position>]{<name>}{<coordinates>}{<type>}{<IP address>}
%
% Where:
%  - <label position> : top, bottom, left, right (default)
%  - <type> : hardware type, e.g. Catalyst 3850
%

\newcommand{\nodecloud}[5][right]{
    \iconcloud{#2}{#3}
}
\newcommand{\nodefirewall}[5][right]{
    \iconfirewall{#2}{#3}
    \nodelabel[#1]{#3}{#2}{#4}{#5}
}
\newcommand{\nodeips}[5][right]{
    \iconips{#2}{#3}
    \nodelabel[#1]{#3}{#2}{#4}{#5}
}
\newcommand{\noderouter}[5][right]{
    \iconrouter{#2}{#3}
    \nodelabel[#1]{#3}{#2}{#4}{#5}
}
\newcommand{\nodedistswitch}[5][right]{
    \icondistswitch{#2}{#3}
    \nodelabel[#1]{#3}{#2}{#4}{#5}
}
\newcommand{\nodeswitch}[5][right]{    
    \iconswitch{#2}{#3}
    \nodelabel[#1]{#3}{#2}{#4}{#5}
}
\newcommand{\nodecoreswitch}[5][right]{    
    \iconcoreswitch{#2}{#3}
    \nodelabel[#1]{$(#3)-(0,.4)$}{#2}{#4}{#5}
}
\newcommand{\nodehost}[5][right]{
   \iconhost{#2}{#3}
   \nodelabel[#1]{#3}{#2}{#4}{#5}
}



%
% Clusters of nodes
%
% Syntax: \cluster<type>[<label position>]{<name>}{<coordinates>}{<type>}{<IP address>}
%

\newcommand{\clustercloud}[3][bottom]{
    \draw[network node,draw=mdtpurple,fill=mdtpalegray] (#3) ellipse (15mm and 10mm);
    \node (#2) at (#3) {#2};
    \ifthenelse{\equal{#1}{top}}{
        \coordinate (a) at ($(#3)-(10mm,-7mm)$);
        \coordinate (b) at ($(#3)+(10mm,7mm)$);
    }{
        \coordinate (a) at ($(#3)-(10mm,7mm)$);
        \coordinate (b) at ($(#3)+(10mm,-7mm)$);
    }
    \message{clustercloud: #2-1^^J}
    \iconrouter{#2-1}{a}
    \message{clustercloud: #2-2^^J}
    \iconrouter{#2-2}{b}
}

\newcommand{\clusterfirewall}[5][bottom]{
    \coordinate (left) at ($(#3)-(1,0)$);
    \coordinate (right) at ($(#3)+(1,0)$);
    \iconfirewall{#2-1}{left}
    \iconfirewall{#2-2}{right}
    
    \begin{pgfonlayer}{connections}
        \node (#2) [draw=mdtmedtronicblue,rounded corners,inner sep=2mm,dashed,fit=(#2-1) (#2-2)] {};
    \end{pgfonlayer}
    \connect{#2-1}{}{#2-2}{}
    
    \nodelabel[#1]{#3}{#2}{#4}{#5}
}




%
% Syntax:
% \nodelabel[<position>]{<coordinates>}{<hostname>}{<type>}{<IP address>}
\newcommand{\nodelabel}[5][right]{%
    \def\text{%
        \ifthenelse{\equal{#4}{}}{%
            \ifthenelse{\equal{#5}{}}{%
                \contour{white}{\textbf{#3}}%
            }{%
                \contour{white}{\textbf{#3}}\\\contour{white}{\emph{#4}}%
            }%
        }{%
            \contour{white}{\textbf{#3}}\\\contour{white}{\emph{#4}}\\\contour{white}{#5}%
        }%
    }

    \begin{pgfonlayer}{labels}
    
    % On the right
    \ifthenelse{\equal{#1}{right}}{
        %\node[node label,anchor=west] at ($(#2)+(1,0)$) {\contour{white}{\textbf{#3}\\\emph{#4}\\#5}};
        \node[node label,anchor=west] at ($(#2)+(1,0)$) {\text};
        
    % On the left
    }{\ifthenelse{\equal{#1}{left}}{
        \node[node label,anchor=east] at ($(#2)-(1,0)$) {\text};
    
    % At the top
    }{\ifthenelse{\equal{#1}{top}}{
        \node[node label,anchor=south] at ($(#2)+(0,1)$) {\text};
    
    % At the bottom
    }{\ifthenelse{\equal{#1}{bottom}}{
        \node[node label,anchor=north] at ($(#2)-(0,1)$) {\text};
    
    }{}}}}
    
    \end{pgfonlayer}
    
    \begin{pgfonlayer}{hostnames}
        \node[node label,anchor=north] at ($(#2)-(0,.6)$) {\contour{white}{#3}};
    \end{pgfonlayer}
}

%
% Syntax:
% \vlaninformationR[<xshift>]{<node name>}{<relative location>}
%\newcommand{\vlaninformationR}[4][5cm]{
%    \matrix (#2) [matrix of nodes,
%        right=of #3,
%        xshift=#1,
%        left delimiter=\{,
%        nodes={
%            anchor=west,
%            align=left
%            }
%        ] {
%            #4
%    };
%}
\begin{document}
\begin{tikzpicture}[framed]
\def\iconset{basic}
%\showhostnamesinterfaces
\nodeswitch[bottom]{switch}{3,0}{Cisco 2960}{}
\nodehost[left]{pc 1}{0,0}{FreeBSD}{0200.aaaa.0001}
\nodehost{pc 2}{6,0}{FreeBSD}{0200.aaaa.0002}
\nodehost{pc 3}{3,3}{Linux}{0200.aaaa.0003}
\connect{switch}{fa0/1}{pc 1}{em0}
\connect{switch}{fa0/2}{pc 2}{em0}
\connect{switch}{fa0/3}{pc 3}{eth0}


\end{tikzpicture}
\end{document}
    \caption{Een eenvoudig netwerk met twee computers die met elkaar verbonden zijn via een switch}
    \label{fig:simple-lan}
\end{figure}

Een MAC-adres kan slechts aan één interface gekoppeld zijn.
Als de switch een MAC-adres dat het al kent, opnieuw leert via een andere interface, dan update het de bestaande entry in de MAC-tabel door de oude interface te vervangen door de nieuwe interface.

Het is wel mogelijk dat meerdere MAC-adressen op dezelfde interface geleerd worden.
Dit is mogelijk als een computer met meerdere virtuele machines, verbonden wordt met de switch.
Elke virtuele machine heeft immers een eigen MAC-adres.
Het is ook mogelijk meerdere MAC-adressen te leren op een interface als deze interface met een andere switch verbindt.
De eerste switch leert dan de MAC-adressen van alle computers die verbinden met de tweede switch via deze interface.


\subsection{Lussen in het netwerk}
\label{sec:stp}

Als één switch niet meer voldoende is om alle computers en printers met elkaar te verbinden, verbinden we een tweede switch met de eerste switch.
Als beide switch niet meer voldoende zijn, kunnen we nog een derde switch verbinden met het netwerk, en vervolgens nog een vierde en vijfde switch (zie \vref{fig:daisy-chain}).


\begin{figure}
    \centering
    \documentclass[tikz]{standalone}
\usepackage{ifthen}
\usepackage{contour}
%%
%% Medtronic color palette
%%

\usepackage{xcolor}

% Primary blue color palette
\definecolor{mdtnavyblue}{cmyk}{1,.94,.47,.43}
\definecolor{mdtmedtronicblue}{cmyk}{.99,.74,.17,.04}
\definecolor{mdtcobaltblue}{cmyk}{.81,.35,0,0}
\definecolor{mdtmediumblue}{cmyk}{.73,.12,0,0}
\definecolor{mdtskyblue}{cmyk}{.55,.04,.04,0}
\definecolor{mdtlightblue}{cmyk}{.29,.05,.05,0}

% Primary neutral color palette
\definecolor{mdtcharcoal}{rgb}{.83,.86,.90}
\definecolor{mdtbluegray}{cmyk}{.68,.40,.28,.08}
\definecolor{mdtdarkgray}{cmyk}{0,0,0,.55}
\definecolor{mdtlightgray}{cmyk}{0,0,0,.30}
\definecolor{mdtpalegray}{cmyk}{.08,.06,.06,0}

% Accent colors
\definecolor{mdtyellow}{cmyk}{0,.18,.93,0}
\definecolor{mdtlightorange}{cmyk}{.02,.38,.96,0}
\definecolor{mdtorange}{cmyk}{.04,.78,1,0}
\definecolor{mdtpurple}{cmyk}{.38,.98,0,0}
\definecolor{mdtgreen}{cmyk}{.57,.02,1,0}
\definecolor{mdtturquoise}{cmyk}{.70,0,.38,0}

% TikZ/PGF
\usepackage{tikz}
\usetikzlibrary{backgrounds}
\usetikzlibrary{calc}
\usetikzlibrary{decorations.pathreplacing}
\usetikzlibrary{fit}
\usetikzlibrary{matrix}
\usetikzlibrary{patterns}
\usetikzlibrary{positioning}
\usetikzlibrary{shapes.geometric}
\usetikzlibrary{arrows}

%%
%% Styles
%%

% Network nodes
\tikzset{network node image/.style={
                inner sep=0mm,
                %anchor=south west
                }}
\tikzset{network node/.style={
                network node image,
                thick,
                text=mdtmedtronicblue,
                draw=mdtmedtronicblue,
                fill=white,
                minimum size=10mm}}
\tikzset{network node circle/.style={
                network node,
                circle}}
\tikzset{network node square/.style={
                network node,
                rectangle,
                rounded corners}}
\tikzset{network node rectangle/.style={
                network node,
                rectangle,
                minimum height=15mm,
                rounded corners}}
\tikzset{server rack/.style={
                align=left,
                text width=8mm,
                inner sep=0pt,
                anchor=north west,
                minimum height=7.5mm-.8pt,
                minimum width=10.7mm-.8pt,
                thick,
                text=mdtmedtronicblue,
                draw=mdtmedtronicblue,
                fill=white}}
\tikzset{network rack/.style={
                align=left,
                text width=9.3mm,
                inner sep=0pt,
                anchor=north west,
                minimum height=7.5mm-.8pt,
                minimum width=12mm-.8pt,
                %minimum height=7.5mm,
                %minimum width=12mm,
                thick,
                text=mdtmedtronicblue,
                draw=mdtmedtronicblue,
                fill=white}}
\tikzset{patch rack/.style={
                align=left,
                text width=9.3mm,
                inner sep=0pt,
                anchor=north west,
                minimum height=7.5mm-.8pt,
                minimum width=12mm-.8pt,
                thick,
                text=mdtmedtronicblue,
                draw=mdtmedtronicblue,
                fill=white}}

% Labels
\tikzset{interface label/.style={
                font=\footnotesize,
                inner sep=2pt},
                fill=white}
\tikzset{node label/.style={
                font=\small,
                align=center,
                inner sep=2pt,
                thick,
                rounded corners}}


% Floor tiles
\newcommand{\dcfloortiles}[2]{%
    \draw[step=6mm,gray,very thin] (0,0) grid ($.6*(#1,#2)$);
}
\newcommand{\dccoldaisle}[2]{%
    \draw[pattern=north west lines, pattern color=mdtlightgray] ($.6*(#1)-(.6,.6)$) rectangle ($.6*(#1)+.6*(#2)-(.6,.6)$);
}
% Markings for different types of racks
\newcommand{\caserack}[1]{%
    \draw[very thick,mdtorange] ($(#1.north east)-(.1,.1)$) -- ($(#1.south east)+(-.1,.1)$);
}
\newcommand{\storagerack}[1]{%
    \draw[very thick,mdtgreen] ($(#1.north east)-(.2,.1)$) -- ($(#1.south east)+(-.2,.1)$);
}
\newcommand{\networkrack}[1]{%
    \draw[very thick,mdtmediumblue] ($(#1.north east)-(.15,.1)$) -- ($(#1.south east)+(-.15,.1)$);
}
\newcommand{\mitgrack}[1]{%
    \draw[very thick,mdtpurple] ($(#1.north east)-(.25,.1)$) -- ($(#1.south east)+(-.25,.1)$);
}
\newcommand{\unixrack}[1]{%
    \draw[very thick,mdtdarkgray] ($(#1.north east)-(.3,.1)$) -- ($(#1.south east)+(-.3,.1)$);
}
\newcommand{\computerack}[1]{%
    \draw[very thick,mdtturquoise] ($(#1.north east)-(.35,.1)$) -- ($(#1.south east)+(-.35,.1)$);
}
\newcommand{\racklabels}[1]{%
    \matrix (racklabels) at (#1) [matrix of nodes,
        nodes={
            anchor=west,
            align=left
            }
        ] {
            |[mdtorange]| CASE \\
            |[mdtturquoise]| Compute (CORE) \\
            |[mdtpurple]| MITG \\
            |[mdtmediumblue]| Network \\
            |[mdtgreen]| Storage \\
            |[mdtdarkgray]| UNIX \\
    };
}
%
% layers.tex
%
% This document contains code to hide a PGF layer if so desired,
% e.g. you can hide the layer with all the labels on it.
%

% Define the layers to be used
\pgfdeclarelayer{connections}
\pgfdeclarelayer{interfaces}
\pgfdeclarelayer{hostnames}
\pgfdeclarelayer{labels}
\pgfdeclarelayer{arrows}

% Make the background white so we can hide the labels by swapping the order of the layers
\tikzset{show background rectangle,background rectangle/.style={rounded corners,fill=white}}

% \showalllayers - Show the nodes, connections, and detailed labels, including interface labels
% The left most layer is the bottom layer; everything left from the background will be hidden.
\newcommand\showalllayers{      \pgfsetlayers{hostnames,arrows,background,connections,main,interfaces,labels}}
\newcommand\showarrows{         \pgfsetlayers{hostnames,background,connections,main,interfaces,labels,arrows}}
% \hidelabels - Show only the nodes and the connections
\newcommand\hidelabels{         \pgfsetlayers{hostnames,arrows,interfaces,labels,background,connections,main}}
% \hideinterfaces - Hide only the interface labels
\newcommand\hideinterfaces{     \pgfsetlayers{hostnames,arrows,interfaces,background,connections,main,labels}}
% \showonlyhostnames - Hide everything but show the hostnames only
\newcommand\showonlyhostnames{  \pgfsetlayers{arrows,interfaces,labels,background,connections,main,hostnames}}
\newcommand\showhostnamesinterfaces{ \pgfsetlayers{arrows,labels,background,connections,main,hostnames,interfaces}}
% Default setting is to show all layers
\showalllayers
%
% connections.tex
%
% This document contains all the commands needed to create connections between
% different network nodes, e.g.
%  - a normal connection, with our without labels.
%  - a vPC connection between 3 or 4 nodes.
%

%
% Syntax:
% \connect[<color>]{<node1>}{<intf1>}{<node2>}{<intf2>}
\newcommand{\connect}[5][black]{

    % Save the coordinates for the interface labels
    \path (#2) -- (#4) coordinate[very near start] (i1) coordinate[midway] (im) coordinate[very near end] (i2);
    
    % Draw the connection between the two devices on the correct layer.
    \begin{pgfonlayer}{connections}
      \draw[color=#1] (#2) -- (#4);
    \end{pgfonlayer}
    
    % Draw the interface labels on the correct layer.
    \begin{pgfonlayer}{interfaces}
        \ifthenelse{\equal{#3}{}}{}{\node[interface label,fill=white] at (i1) {\contour{white}{#3}};}
        \ifthenelse{\equal{#5}{}}{}{\node[interface label,fill=white] at (i2) {\contour{white}{#5}};}
    \end{pgfonlayer}
    
    %\ifthenelse{\equal{#5}{}}{
    %    % Parameter 5 is empty:
    %    \ifthenelse{\equal{#3}{}}{
    %        % Parameter 3 is empty: no interfaces have been given.
    %        \draw[color=#1] (#2) -- (#4);
    %    }{
    %        % Both devices share the same interface name.
    %        \draw[color=#1] (#2) -- (#4)
    %            node [midway,interface] {#3};
    %    }
    %}{
    %    % Parameter 5 is not empty: both interfaces are given.
    %    \draw[color=#1] (#2) -- (#4)
    %        node [on background layer,near start,interface] {#3}
    %        node [near end,interface] {#5};
    %}
}

%
% Syntax:
% \connectClusters[<straight/cross>]{cluster1}{cluster2}
\newcommand{\connectClusters}[3][S]{
    \message{connectClusters 2-1: #2-1^^J}
    \message{connectClusters 2-2: #2-2^^J}
    \message{connectClusters 3-1: #3-1^^J}
    \message{connectClusters 3-2: #3-2^^J}
    \ifthenelse{\equal{#1}{S}}{
        \connect{#2-1}{}{#3-1}{}
        \connect{#2-2}{}{#3-2}{}
    }{
        \connect{#2-1}{}{#3-2}{}
        \connect{#2-2}{}{#3-1}{}
    }
}

%
% Syntax:
% \arrowcoords[<position>]{<coordinates start>}{<coordinates end>}{<text>}
\newcommand{\arrowcoords}[4][midway]{
    \draw[mdtgreen, arrows={-latex'}, shorten <= 0.25cm, shorten >= 0.25cm, very thick, distance=2cm] (#2) to[out=230,in=170] (#3) node[#1] {#4};
}

%
%
%

\def\iconset{3015}


%
% Display only the icons
%
% Syntax: \icon<type>{<hostname>}{<coordinates>}
%

\newcommand{\iconcloud}[2]{
    \node[network node image] (#1) at (#2) {\includegraphics[width=30mm]{images//cloud.png}};
    \node at ([yshift=-2mm]#2) {#1};
}
\newcommand{\iconfirewall}[2]{
    \ifthenelse{\equal{\iconset}{basic}}{
        \node[network node circle,draw=mdtorange] (#1) at (#2) {FW};
    }{
        \node[network node image] (#1) at (#2) {\includegraphics[width=10mm]{images/\iconset/firewall.eps}};
    }
}
\newcommand{\iconips}[2]{
    \ifthenelse{\equal{\iconset}{basic}}{
        \node[network node square] (#1) at (#2) {IPS};
    }{
        \node[network node image] (#1) at (#2) {\includegraphics[width=10mm]{images/\iconset/ips.eps}};
    }
}
\newcommand{\iconrouter}[2]{
    \ifthenelse{\equal{\iconset}{basic}}{
        \node[network node circle] (#1) at (#2) {RTR};
    }{
        \node[network node image] (#1) at (#2) {\includegraphics[width=10mm]{images/\iconset/router.eps}};
    }
    \message{iconrouter: #1^^J}
}
\newcommand{\iconswitch}[2]{
    \ifthenelse{\equal{\iconset}{basic}}{
        \node[network node square] (#1) at (#2) {SW};
    }{
        \node[network node image] (#1) at (#2) {\includegraphics[width=13mm]{images/\iconset/switch.eps}};
    }
}
\newcommand{\icondistswitch}[2]{
    \ifthenelse{\equal{\iconset}{basic}}{
        \node[network node square] (#1) at (#2) {SW};
    }{
        \node[network node image] (#1) at (#2) {\includegraphics[width=13mm]{images/\iconset/l3switch.eps}};
    }
}
\newcommand{\iconcoreswitch}[2]{
    \ifthenelse{\equal{\iconset}{basic}}{
        \node[network node square] (#1) at (#2) {SW};
    }{
        \node[network node image] (#1) at (#2) {\includegraphics[width=13mm]{images/\iconset/coreswitch.eps}};
    }
}
\newcommand{\iconhost}[2]{
   \ifthenelse{\equal{\iconset}{basic}}{
      \node[network node square] (#1) at (#2) {PC};
   }{
      \node[network node square] (#1) at (#2) {PC};
   }
}



%
% Display the node: icon + labels
%
%
% Syntax: \node<type>[<label position>]{<name>}{<coordinates>}{<type>}{<IP address>}
%
% Where:
%  - <label position> : top, bottom, left, right (default)
%  - <type> : hardware type, e.g. Catalyst 3850
%

\newcommand{\nodecloud}[5][right]{
    \iconcloud{#2}{#3}
}
\newcommand{\nodefirewall}[5][right]{
    \iconfirewall{#2}{#3}
    \nodelabel[#1]{#3}{#2}{#4}{#5}
}
\newcommand{\nodeips}[5][right]{
    \iconips{#2}{#3}
    \nodelabel[#1]{#3}{#2}{#4}{#5}
}
\newcommand{\noderouter}[5][right]{
    \iconrouter{#2}{#3}
    \nodelabel[#1]{#3}{#2}{#4}{#5}
}
\newcommand{\nodedistswitch}[5][right]{
    \icondistswitch{#2}{#3}
    \nodelabel[#1]{#3}{#2}{#4}{#5}
}
\newcommand{\nodeswitch}[5][right]{    
    \iconswitch{#2}{#3}
    \nodelabel[#1]{#3}{#2}{#4}{#5}
}
\newcommand{\nodecoreswitch}[5][right]{    
    \iconcoreswitch{#2}{#3}
    \nodelabel[#1]{$(#3)-(0,.4)$}{#2}{#4}{#5}
}
\newcommand{\nodehost}[5][right]{
   \iconhost{#2}{#3}
   \nodelabel[#1]{#3}{#2}{#4}{#5}
}



%
% Clusters of nodes
%
% Syntax: \cluster<type>[<label position>]{<name>}{<coordinates>}{<type>}{<IP address>}
%

\newcommand{\clustercloud}[3][bottom]{
    \draw[network node,draw=mdtpurple,fill=mdtpalegray] (#3) ellipse (15mm and 10mm);
    \node (#2) at (#3) {#2};
    \ifthenelse{\equal{#1}{top}}{
        \coordinate (a) at ($(#3)-(10mm,-7mm)$);
        \coordinate (b) at ($(#3)+(10mm,7mm)$);
    }{
        \coordinate (a) at ($(#3)-(10mm,7mm)$);
        \coordinate (b) at ($(#3)+(10mm,-7mm)$);
    }
    \message{clustercloud: #2-1^^J}
    \iconrouter{#2-1}{a}
    \message{clustercloud: #2-2^^J}
    \iconrouter{#2-2}{b}
}

\newcommand{\clusterfirewall}[5][bottom]{
    \coordinate (left) at ($(#3)-(1,0)$);
    \coordinate (right) at ($(#3)+(1,0)$);
    \iconfirewall{#2-1}{left}
    \iconfirewall{#2-2}{right}
    
    \begin{pgfonlayer}{connections}
        \node (#2) [draw=mdtmedtronicblue,rounded corners,inner sep=2mm,dashed,fit=(#2-1) (#2-2)] {};
    \end{pgfonlayer}
    \connect{#2-1}{}{#2-2}{}
    
    \nodelabel[#1]{#3}{#2}{#4}{#5}
}




%
% Syntax:
% \nodelabel[<position>]{<coordinates>}{<hostname>}{<type>}{<IP address>}
\newcommand{\nodelabel}[5][right]{%
    \def\text{%
        \ifthenelse{\equal{#4}{}}{%
            \ifthenelse{\equal{#5}{}}{%
                \contour{white}{\textbf{#3}}%
            }{%
                \contour{white}{\textbf{#3}}\\\contour{white}{\emph{#4}}%
            }%
        }{%
            \contour{white}{\textbf{#3}}\\\contour{white}{\emph{#4}}\\\contour{white}{#5}%
        }%
    }

    \begin{pgfonlayer}{labels}
    
    % On the right
    \ifthenelse{\equal{#1}{right}}{
        %\node[node label,anchor=west] at ($(#2)+(1,0)$) {\contour{white}{\textbf{#3}\\\emph{#4}\\#5}};
        \node[node label,anchor=west] at ($(#2)+(1,0)$) {\text};
        
    % On the left
    }{\ifthenelse{\equal{#1}{left}}{
        \node[node label,anchor=east] at ($(#2)-(1,0)$) {\text};
    
    % At the top
    }{\ifthenelse{\equal{#1}{top}}{
        \node[node label,anchor=south] at ($(#2)+(0,1)$) {\text};
    
    % At the bottom
    }{\ifthenelse{\equal{#1}{bottom}}{
        \node[node label,anchor=north] at ($(#2)-(0,1)$) {\text};
    
    }{}}}}
    
    \end{pgfonlayer}
    
    \begin{pgfonlayer}{hostnames}
        \node[node label,anchor=north] at ($(#2)-(0,.6)$) {\contour{white}{#3}};
    \end{pgfonlayer}
}

%
% Syntax:
% \vlaninformationR[<xshift>]{<node name>}{<relative location>}
%\newcommand{\vlaninformationR}[4][5cm]{
%    \matrix (#2) [matrix of nodes,
%        right=of #3,
%        xshift=#1,
%        left delimiter=\{,
%        nodes={
%            anchor=west,
%            align=left
%            }
%        ] {
%            #4
%    };
%}
\begin{document}
\begin{tikzpicture}[framed]
\def\iconset{basic}
\nodecloud{Internet}{0,4}{}{}
\nodefirewall{firewall}{0,2}{}{}
\nodeswitch[bottom]{sw1}{0,0}{}{}
\nodeswitch[bottom]{sw2}{2,0}{}{}
\nodeswitch[bottom]{sw3}{4,0}{}{}
\nodeswitch[bottom]{sw4}{6,0}{}{}
\nodeswitch[bottom]{sw5}{8,0}{}{}
\connect{sw1}{}{sw2}{}
\connect{sw2}{}{sw3}{}
\connect{sw3}{}{sw4}{}
\connect{sw4}{}{sw5}{}
\end{tikzpicture}
\end{document}
    \caption{Meerdere switches worden met elkaar verbonden in een lange lijn. Dit noemt \emph{daisy chaining}.}
    \label{fig:daisy-chain}
\end{figure}


\subsection{Virtuele switches}


\subsection{Virtuele kabels}

% Ik heb geen idee wat ik bedoelde met virtuele kabels...