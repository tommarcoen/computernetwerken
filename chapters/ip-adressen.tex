\section{IP-adressen}

\begin{frame}
\begin{center}
\includegraphics<presentation>[width=.8\textwidth]{images/client-server-2.jpeg}
\includegraphics<article>[width=.65\textwidth]{images/client-server-2.jpeg}
\end{center}
\end{frame}

\mode<article>{
Als we de situatie uit de inleiding in meer detail bekijken, merken we dat zowel de client als de server deel uitmaken van een netwerk en dat deze netwerk met elkaar verbonden worden met behulp van \emph{routers}.
}



\begin{frame}{Netwerkadressen}
\begin{itemize}
\item<1-> Elke computer heeft een IP-adres
    \begin{itemize}
    \item<2-> Mogelijk meerdere IP-adressen
    \end{itemize}
\item<3-> Een IP-adres bestaat uit twee delen:
    \begin{enumerate}
    \item Netwerkdeel
    \item Hostdeel        
    \end{enumerate}
    \begin{exampleblock}<4>{Voorbeeld: 192.0.2.1/24}
    \begin{itemize}
    \item Netwerkdeel: 192.0.2
    \item Hostdeel: 1
    \end{itemize}
    \end{exampleblock}
\item<5-> Elk netwerk heeft een eigen, uniek, netwerkdeel
\end{itemize}
\end{frame}



\begin{frame}{Subnetmasker en prefix-length}
\begin{itemize}
\item Het netwerkgedeelte wordt bepaald door de prefix-length
\item Deze kan je omzetten naar een subnetmasker
\item Subnetmaskers zijn de oude manier van noteren
\end{itemize}
\end{frame}

\subsection{Hiërarchische adressen}

\begin{frame}
\end{frame}

\subsection{Variabele groottes van subnetten}

\begin{frame}
\end{frame}

\subsection{Dynamisch toekennen van adressen}

\begin{frame}
\end{frame}

\subsection{Network address translation}

\begin{frame}
\end{frame}

\subsection{IP versie 6}

\begin{frame}
\end{frame}

