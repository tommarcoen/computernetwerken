\section{IP-adressen}

\begin{frame}{Netwerken}
\begin{center}
\includegraphics<presentation>[width=.65\textwidth]{images/client-server-2.png}
\includegraphics<article>[width=.65\textwidth]{images/client-server-2.png}
\end{center}
\begin{itemize}
   \item De client en server maken deel uit van een netwerk
   \item Netwerken worden met elkaar verbonden via routers
\end{itemize}
\end{frame}

\mode<article>{
Als we de situatie uit de inleiding in meer detail bekijken, merken we dat zowel de client als de server deel uitmaken van een netwerk en dat deze netwerken met elkaar verbonden worden met behulp van \emph{routers}.
De werking van routers bestuderen we in een later hoofdstuk.
In dit hoofdstuk focussen we op de IP-adressen.
}



\begin{frame}{Netwerkadressen}
\begin{itemize}
\item<1-> Elke computer heeft een IP-adres
    \begin{itemize}
    \item<2-> Mogelijk meerdere IP-adressen
    \end{itemize}
\item<3-> Een IP-adres bestaat uit twee delen:
    \begin{enumerate}
    \item Netwerkdeel
    \item Hostdeel        
    \end{enumerate}
\item<5-> Elk netwerk heeft een eigen, uniek, netwerkdeel
\end{itemize}
\uncover<4>{
   \begin{exampleblock}{Voorbeeld: 192.0.2.1/24}
   \begin{itemize}
   \item Netwerkdeel: 192.0.2
   \item Hostdeel: 1
   \end{itemize}
   \end{exampleblock}
 }
\end{frame}



\begin{frame}{Netwerkadressen: een voorbeeld}
   \begin{center}
      \includegraphics<presentation>[width=.65\textwidth]{images/client-server-3.png}
      \includegraphics<article>[width=.65\textwidth]{images/client-server-3.png}
      \end{center}
\end{frame}



\begin{frame}{IP-adressen}
\begin{itemize}
\item<1->Een IP-adres bestaat uit 32 bits
\item<2->De computer ziet het adres als één getal van 0 tot \num{4294967295}
\item<3->Mensen vinden de \emph{dotted decimal} notatie duidelijker
\item<5->Vier blokjes (\emph{octet}) van 8 bits, gescheiden door een punt
\item<6->Elk octet is een getal van 0 tot 255
\end{itemize}
\uncover<4>{
   \begin{exampleblock}{Voorbeeld}
      \begin{itemize}
      \item 192.0.2.1
      \item 224.0.0.2
      \item 198.15.67.32
      \end{itemize}
      \end{exampleblock}
}
\end{frame}



\begin{frame}{Subnetmasker en prefix-length}
\begin{itemize}
\item<1-> De grootte van het netwerkgedeelte wordt bepaald door de prefix-length
\item<2-> Vanaf links te tellen bepaalt deze het netwerkgedeelte
\item<3-> Deze kan je omzetten naar een subnetmasker
\item<4-> Subnetmaskers zijn de oude manier van noteren
\item<5-> $32-p$ bits vanaf rechts bepalen het hostgedeelte
\end{itemize}
\end{frame}



\begin{frame}{Subnetmaskers}
\begin{exampleblock}{Voorbeelden}
\begin{itemize}
\item<1-> 192.168.23.7/25 \hfill 255.255.255.128 \hspace*{.3\textwidth}\null
\item<2-> 203.0.113.154/26 \hfill 255.255.255.192 \hspace*{.3\textwidth}\null
\item<3-> 172.17.3.22/12 \hfill 255.255.240.0 \hspace*{.3\textwidth}\null
\end{itemize}
\end{exampleblock}
\end{frame}



\begin{frame}{Speciale adressen}
\begin{itemize}
\item Het eerste en laatste adres in een netwerk zijn gereserveerd
\item Het eerste adres is het \alert{netwerkadres}
\item Het laatste adres is het \alert{broadcastadres}
\end{itemize}
\end{frame}


\subsection{Hiërarchische adressen}

\begin{frame}
\end{frame}

\subsection{Variabele groottes van subnetten}

\begin{frame}
\end{frame}



\subsection{Dynamically assigning IP addresses}

\begin{frame}{Static configuration}
\begin{itemize}[<+->]
\item A lot of work
\item Difficult
\item Not only IP address and subnet mask need reconfiguration
\end{itemize}

\only<1>{
   \begin{example}
      Configuring IP settings for thousands of computers in a company
   \end{example}
}
\only<2>{
   \begin{example}
      If you take your laptop to a coffeeshop you do not know their IP settings
   \end{example}
}
\only<3->{
   \begin{example}
      \begin{itemize}[<+->]
      \item DNS settings
      \item TFTP settings (VoIP)
      \end{itemize}
   \end{example}
}
\end{frame}



\begin{frame}{Dynamic Host Configuration Protocol}
\begin{itemize}[<+->]
\item Uses \alert{broadcasts} to exchange packets
\item 
\end{itemize}
\end{frame}

\subsection{Network address translation}

\begin{frame}
\end{frame}
\subsection{IP versie 6}

\begin{frame}
\end{frame}

