\chapter{Preface}

This manual serves as a comprehensive summary and companion guide to the course titled `Computernetwerken met \acs{TCP}/\acs{IP}'\footnote{This title translates to \emph{Computer networks with} \acs{TCP}/\acs{IP}.} which I teach at Syntra Bizz.
Although I teach that course in Dutch, this companion guide is written in English for two reasons.
First, it allows for a wider audience to access this manual.
Second, it is challenging to write a technology book in Dutch, as nearly all technical terms are in English.
I found it too difficult to decide which terms to translate from English to Dutch and which to retain in English.
It is impossible to translate every technical term and translating none of them -- which is really the best option so that it would be easier to search for the terms online to find more resources on them -- would make the text a real mix of Dutch and English.
I decided it is better to write the complete text in English.

I teach the accompanying course over a period of three days consisting of six hours each for a total of eighteen hours of in-class training.
%Additionally I expect the students to make lab exercises at home.
%Should you consider doing these labs in class also, the course would take about forty hours.
A rough schedule for the course is given in \vref{tab:course-schedule}.

\begin{table}
\begin{sidecaption}{In-class course schedule}[tab:course-schedule]
\centering
\begin{tabular}{rlll}
               & {day 1}                    & {day 2}              & {day 3}             \\
\midrule
\textit{a.m.}  & \cref{chap:introduction,chap:ip}  & \cref{chap:transport} & \cref{chap:wires-wireless} \\
\textit{p.m.}  & \cref{chap:ip} \emph{(cont.)}     & \cref{chap:ethernet}        & \cref{chap:applications}   \\
\end{tabular}
\end{sidecaption}
\end{table}

Some topics have the symbol of a university\marginsymbol\ or greek temple in the margin.
This means those topics are a bit more advances and are not covered in class.
I do suggest though that you read them at home to broaden your understanding of computer networks.

At the end of each chapter I try to add the following four sections:
\begin{description}
\item[Review questions] Questions to test whether you understood the topics covered in the chapter.
\item[Guided exercises] I walk you through a pratical excercise for which you need Cisco Packet Tracer, eve-ng, some other virtualisation software, or real equipment.
\item[Practice questions] Exercises which are similar to the guided exercises but without any explanations.
\item[Further reading] A few book suggestions for when you want to study the topics from the chapter in more depth.
\end{description}

This manual is a constant work in progress and I am currently working on adding exercises to the different chapters as well as adding a section on load balancers and a chapter on firewalls.
I welcome constructive feedback at \href{mailto:info@marcoen.net}{info@\-marcoen.net} or create an issue on the \href{https://github.com/tommarcoen/computernetwerken}{github page}.
The \acs{URL} is \url{https://github.com/tommarcoen/computernetwerken}.
I also host a website on which I try to maintain a blog with useful information and updates regarding the courses that I teach.
You can find it on \url{https://student.marcoen.net/}.


%This manual consists of two parts.
%The first part covers the topics that I teach in class and consists of chapters~1--6.
%The second part contains examples and exercises, starting with \vref{chap:cisco-pt}.
%I assume the students -- and thus also the reader -- have access to Cisco Packet Tracer but really any virtualisation or simulation software will do.
%Other possible candidates include \href{https://www.eve-ng.net/}{eve-ng} and \href{https://www.gns3.com/}{\abbr{GNS3}}.%
%\footnote{These software packages can be found at \href{https://www.eve-ng.net/}{www.eve-ng.net} and \href{https://www.gns3.com/}{www.gns3.com}.}

%Most of the examples given are using Cisco's \gls{IOS} syntax but there are also examples for Juniper Networks' Junos.
%Example configuration snippets for other products, such as HP ProCurve or the newer Aruba switches might be added in the future.
%, Aruba's ArubaOS and HP ProCurve.
