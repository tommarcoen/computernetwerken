\chapter{Inleiding}
\label{chap:inleiding}

De \TCPIP\ \emph{protocol stack} is een verzameling van communicatieprotocollen die er voor zorgen dat het Internet werkt.
Een gedegen kennis van \TCPIP\ is een must voor elke \abbr{IT}'er die professionele computernetwerken wilt opzetten en beheren.
Deze cursus is dan ook een uitstekende introductie voor zij die de opleiding netwerk- en systeembeheer willen aanvatten.

Tijdens%
\sidenote{Eventueel worden sommige oefeningen gemaakt met behulp van de eve-ng virtualisatieomgeving.}
deze training bespreken we de evolutie en ontwikkelingen in \TCPIP-netwerken en focussen we op de interne werking van het protocol.
In de praktijklessen gaan we aan de slag met Packet Tracer,
het visuele simulatieprogramma van Cisco.

\section{Wat is een netwerk?}

Een computernetwerk is een systeem voor communicatie tussen twee of meer computers.
De communicatie verloopt via netwerkkabels of via een draadloos netwerk.
In de netwerktopologie worden fysieke en logische topologieën onderscheiden.
Men spreekt van een \abbr{LAN} in het geval van lokale plaatsgebonden bekabeling waarop computers binnen één gebouw of een campus aangesloten worden en een \abbr{WAN} wanneer er sprake is van verbindingen over grotere afstanden.



\section{De evolutie van computernetwerken}









\section{Het \abbr{OSI}-model}


\subsection{Physical layer (fysieke laag)}

De \emph{physical layer} is de eerste of onderste laag uit het OSI-model.
Deze bevat de elektrische en mechanische definities van het transportmedium en het signaal.
De fysieke laag beschrijft
\begin{inlinelist}
\item het vertalen van binaire informatie naar een elektrisch signaal en weer terug;
\item de definitie van de mechanische karakteristieken van connector en kabel, inclusief bijvoorbeeld de maximale lengte;
\item de definitie van de signaalkarakteristieken zoals het elektrisch signaal bij transport over koper, het optische signaal bij transport over glasvezel, of het radiosignaal bij transport door de lucht.
\end{inlinelist}
Het meest bekende voorbeeld is Ethernet, maar ook protocollen zoals DSL, RS-232 of GSM bevinden zich in deze laag.

Een hub of repeater en een Wi-Fi access point bevinden zich op deze laag.

\subsection{Data link layer (datalinklaag)}

De \emph{data link layer} beschrijft het transport op het lokale netwerk.
De bekendste voorbeelden hiervan zijn IEEE~802.3 (Ethernet) en IEEE~802.11 (Wi-Fi).
Beide protocollen gebruiken MAC-adressen voor het versturen van \emph{frames} doorheen het netwerk en switches om het netwerk op te bouwen.


\subsection{Network layer (netwerklaag)}
\subsection{Transport layer (transportlaag)}
\subsection{Lagen 5--7}
