\section*{Inleiding}


\begin{frame}{Communicatie tussen computers}
\begin{center}
\includegraphics<presentation>[width=\textwidth]{images/http-request.png}
\includegraphics<article>[width=.65\textwidth]{images/http-request.png}
\end{center}
\end{frame}

\mode<article>{
We beginnen met het voorbeeld dat een computer, de \emph{client}, een webpagina wilt ophalen bij een webserver.
De client verstuurt een HTTP \emph{request} en de server antwoordt met een HTTP \emph{response}.
}



\begin{frame}[fragile]{HTTP: client request}
\usebeamerfont{verbatim}
\begin{semiverbatim}
GET /index.html HTTP/1.1
Host: www.example.com
\end{semiverbatim}
\end{frame}



\begin{frame}[fragile=singleslide]{HTTP: server response}
\usebeamerfont{verbatim}
\begin{verbatim}
HTTP/1.1 200 OK
Content-Type: text/html; charset=UTF-8
Content-Length: 155
Server: Apache/1.3.3.7 (Unix) (Red-Hat/Linux)

<html>
  <head>
    <title>An Example Page</title>
  </head>
  <body>
    <p>Hello World!</p>
  </body>
</html>
\end{verbatim}
\end{frame}



\begin{frame}{Enkele vragen}
\begin{enumerate}
\item<1-> Hoe kunnen we de server bereiken?
\item<2-> Voor welke applicatie is de data bestemd?
\item<3-> Hoe versturen we heel veel data?
\item<4-> Wat als er dingen misgaan?
\end{enumerate}
\end{frame}

\mode<article>{
Maar wat is er allemaal nodig om deze request van de client tot bij de server te krijgen en het antwoord bij de client?
}



\begin{frame}{Hoe de server bereiken?}
\begin{itemize}
\item<1-> DNS-naam
    \begin{itemize}
    \item Adresbalk: \url{http://www.example.com/}
    \item Hostheader: www.example.com
    \end{itemize}
\item<2-> IP-adres
\end{itemize}
\end{frame}



\begin{frame}{Voor welke applicatie is de data bestemd?}
\begin{itemize}
\item Elke serverapplicatie luistert op een bepaalde ``poort''
\item Computerstandaarden bepalen poortnummers
\item De client gebruikt een willekeurige poort
\end{itemize}
\end{frame}



\begin{frame}{Hoe versturen we heel veel data?}
\begin{itemize}
\item Fragmentatie: opsplitsen in kleine pakketjes
\item Volgorde van de fragmenten
\end{itemize}
\end{frame}



\begin{frame}{Wat als er dingen mis gaan?}
\begin{itemize}
\item Pakketjes kunnen verloren gaan
\item Er kunnen fouten optreden in de data
\end{itemize}
\end{frame}



\subsection*{Verschillende lagen en modellen}

\begin{frame}{Verschillende lagen}
\begin{itemize}
\item Structuur brengt orde
\item Vereenvoudigt troubleshooten
\item Maakt protocollen vervangbaar
    \begin{itemize}
    \item bv IPv4 wordt IPv6
    \end{itemize}
\end{itemize}
\end{frame}

\mode<article>{
In deze cursus zoeken we het antwoord op deze vragen en ontdekken we hoe computernetwerk opgebouwd zijn en hoe je kan communiceren tussen verschillende netwerken.

De industrie heeft een aantal verschillende maar gelijkaardige modellen in het leven geroepen om stuctuur te brengen in de verschillende protocollen.
De theorie en geschiedenis zijn niet belangrijk, maar een basiskennis van deze protocollen helpt wel om alles te plaatsen en om netwerkproblemen op te sporen en op te lossen.
}



\begin{frame}{Verschillende modellen (1/2)}
\begin{itemize}
\item OSI-model (7 lagen)
\item TCP/IP-model of het DoD-model (4 lagen)
\item hybride model (5 lagen)
\end{itemize}
\end{frame}

\mode<article>{
Er bestaan twee verschillende modellen, het OSI-model en het TCP/IP-model.
Dit laatste model wordt ook wel het DoD-model genoemd.
Zelf verkies ik een hybride model te gebruiken, gebaseerd op beide modellen.
}



\begin{frame}{Verschillende modellen (2/2)}
\begin{table}
    \centering
    \small\sffamily
    \begin{tabular}{rlll}
      & \textbf{OSI}         & \textbf{TCP/IP}                   & \textbf{hybride}       \\[2ex]
    \textit{7} & Applicatie  & \multirow{3}{*}{Applicatie}       & \multirow{3}{*}{Applicatie} \\
    \textit{6} & Presentatie &                                   &                             \\
    \textit{5} & Sessie      &                                   &                        \\[2ex]
    \textit{4} & Transport   & Transport                         & Transport              \\[2ex]
    \textit{3} & Netwerk     & Internet                          & Netwerk                \\[2ex]
    \textit{2} & Datalink    & \multirow{2}{*}{Netwerkinterface} & Datalink                    \\
    \textit{1} & Fysiek      &                                   & Fysiek                      \\
    \end{tabular}
    \caption{Drie verschillende modellen}
    \label{tab:models-layers}
\end{table}
\end{frame}

\mode<article>{
Het is voor netwerbeheerders niet interessant wat exact thuishoort in de applicatielaag of in de sessielaag.
Dit groeperen we allemaal in één applicatilaag, zoals bij het TCP/IP-model.
Het is daarentegen wel interessant om een onderscheid te maken tussen de datalinklaag waar switches en MAC-adressen thuishoren, en de fysieke laag waar we bijvoorbeeld de \emph{tranceivers} en DWDM-apparaten terugvinden.
Deze twee lagen houden we dus gescheiden, zoals in het OSI-model.
}



\begin{frame}{De lagen toegepast}
\begin{itemize}
\item Applicatielaag: HTTP client request
\item Transportlaag: TCP-poort 80 voor HTTP
\item Netwerklaag: IP-adres van client en server
\item Datalinklaag: MAC-adressen
\item Fysieke laag: stroom of licht op de kabel
\end{itemize}
\end{frame}



\subsection*{Netwerktopologieën}

\begin{frame}{Netwerktopologieën}
\begin{itemize}
\item point-to-point
\item daisy chain (linear or ring)
\item bus
\item ster
\item mesh
\end{itemize}
\end{frame}

\mode<article>{
Deze verschillende topologieën komen we tijdens de cursus tegen dus benoemen we deze eerst kort. 
}