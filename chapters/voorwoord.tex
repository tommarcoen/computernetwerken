\section{Voorwoord}

Dit handboek werd geschreven als leidraad voor de cursus \emph{Computernetwerken met TCP/IP} dewelke ik doceer voor Syntra Bizz in Antwerpen en Mechelen.
Deze cursus wordt gegeven in drie dagen van elk zes uur.
Het programma van deze cursus ziet er in grote lijnen als volgt uit.

\begin{center}
   \begin{tabular}{rlll}
                       & \textit{dag 1} & \textit{dag 2}      & \textit{dag 3}     \\[1ex]
   \textit{voormiddag} & IP-adressen    & TCP/UDP en Ethernet & bekabeling en wifi \\
   \textit{namiddag}   & inleiding PT   & VLAN's en STP       & applicaties        \\
   \end{tabular}
\end{center}

Dit handboek is opgedeeld in twee delen.
Het eerste deel behandelt de theorie; het tweede deel bevat oefeningen en voorbeelden.
Voor deze oefeningen kan gebruik worden gemaakt van Cisco Packet Tracer, van een andere virtuele omgeving (zoals bijvoorbeeld eve-ng) of van echte hardware.
De focus ligt op Cisco maar er worden ook voorbeelden gegeven voor HP en Juniper.