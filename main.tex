\documentclass[a4paper,12pt]{memoir}
\usepackage{geometry}
\geometry{a4paper,top=30mm,bottom=65mm,left=30mm,right=65mm,footskip=30mm}
\usepackage[utf8]{inputenc}
\usepackage{csquotes}
\usepackage[dutch]{babel}
\usepackage{libertine}
\usepackage{microtype}
\pagestyle{ruled}
%\usepackage{setspace}
%\setstretch{1.05}

\RequirePackage{enumitem}
\newlist{inlinelist}{enumerate*}{1}
\setlist*[inlinelist,1]{%
  label=(\roman*),
}

\title{Computernetwerken met TCP/IP}
\author{Tom Marcoen}

\begin{document}
\frontmatter
\maketitle
\thispagestyle{empty}

\chapter{Inleiding}

De TCP/IP \emph{protocol stack} is een verzameling van communicatieprotocollen die er voor zorgen dat het Internet werkt.
Een gedegen kennis van TCP/IP is een must voor elke IT'er die professionele computernetwerken wilt opzetten en beheren.
Deze cursus is dan ook een uitstekende introductie voor zij die de opleiding netwerk- en systeembeheer willen aanvatten.

Tijdens deze training bespreken we de evolutie en ontwikkelingen in TCP/IP-netwerken en focussen we op de interne werking van het protocol.
In de praktijklessen gaan we aan de slag met Packet Tracer, het visuele simulatieprogramma van Cisco.

\section{Wat is een netwerk?}


\section{De evolutie van computernetwerken}

\section{Het OSI-model}

\clearforchapter
\tableofcontents

\mainmatter
\chapter{Physical layer (fysieke laag)}

De \emph{physical layer} is de eerste of onderste laag uit het OSI-model.
Deze bevat de elektrische en mechanische definities van het transportmedium en het signaal.
De fysieke laag beschrijft
\begin{inlinelist}
\item het vertalen van binaire informatie naar een elektrisch signaal en weer terug;
\item de definitie van de mechanische karakteristieken van connector en kabel, inclusief bijvoorbeeld de maximale lengte;
\item de definitie van de signaalkarakteristieken zoals het elektrisch signaal bij transport over koper, het optische signaal bij transport over glasvezel, of het radiosignaal bij transport door de lucht.
\end{inlinelist}
Het meest bekende voorbeeld is Ethernet, maar ook protocollen zoals DSL, RS-232 of GSM bevinden zich in deze laag.

Een hub of repeater en een Wi-Fi access point bevinden zich op deze laag.

\chapter{Data link layer (datalinklaag)}

De \emph{data link layer} beschrijft het transport op het lokale netwerk.
De bekendste voorbeelden hiervan zijn IEEE~802.3 (Ethernet) en IEEE~802.11 (Wi-Fi).
Beide protocollen gebruiken MAC-adressen voor het versturen van \emph{frames} doorheen het netwerk en switches om het netwerk op te bouwen.

\chapter{Network layer (netwerklaag)}
\chapter{Transport layer (transportlaag)}
\chapter{Lagen 5--7}

\end{document}
