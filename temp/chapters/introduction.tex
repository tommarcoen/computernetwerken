\chapter{Introduction}
\label{chap:introduction}

This course is focused on the subject of computer networks.
It is safe to presume that all participants have a basic understanding of what a computer is, despite the fact that many devices contain a computer without the user being aware of it.
%For example, the \SC{ATM}%
%   \footnote{Here \SC{ATM} stands for automated teller machine; in the rest of the notes it stands for \acl{ATM}.}
%used to withdraw money from a bank account, the computer systems present in modern vehicles, and even studio mixers%
%   \footnote{For example, \href{https://www.presonus.com/learn/technical-articles/How-To-Network-Studiolive-Digital-Mixers-for-Remote-Control}{this article} explains how to connect a PreSonus StudioLive 16 to the network for remote control. But it is also possible to send the audio over the network to the mixer.}
%, which often feature a network interface, are all examples of computers.
% Dante for AV systems: https://www.audinate.com/meet-dante/what-is-dante
% Art-Net for lighting systems: https://art-net.org.uk
    
When you connect these computers together using network cables or using wireless network adapters, they form a network.
A computer network is used to exchange information between different computers.


\section{Exercises}

\begin{exercise}
This is our very first question.
It is an easy question and thus only worth one point.
\end{exercise}