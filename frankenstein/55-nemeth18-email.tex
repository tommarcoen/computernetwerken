\chapter{Electronic mail}
\label{chap:email}

Decades ago, cooking a chicken dinner involved not just frying the
chicken, but selecting a tender young chicken out of the coop,
terminating it with a kill signal, plucking the feathers, etc. Today,
most of us just buy a package of chicken at the grocery store or butcher
shop and skip the mess.

Email has evolved in a similar way. Ages ago, it was common for
organizations to hand-craft their email infrastructure, sometimes to the
point of predetermining exact mail routing. Today, many organizations
use packaged, cloud-hosted email services such as
\protect\hypertarget{part0026_split_000.htmlux5cux23_idIndexMarker2382}{}{}Google
Gmail or
\protect\hypertarget{part0026_split_000.htmlux5cux23_idIndexMarker2383}{}{}Microsoft
Office 365.

Even if your email system runs in the cloud, you will still have
occasion to understand, support, and interact with it as an
administrator. If your site uses local email servers, the workload
expands even further to include configuration, monitoring, and testing
chores.

If you find yourself in one of these more hands-on scenarios, this
chapter is for you. Otherwise, skip this material and spend your email
administration time responding to messages from wealthy foreigners who
need help moving millions of dollars in exchange for a large reward.
(Just kidding, of course.)



\section{Mail system architecture}

A mail system consists of several distinct components:

\begin{itemize}
\item
  A
  ``\protect\hypertarget{part0026_split_001.htmlux5cux23_idIndexMarker2385}{}{}\protect\hypertarget{part0026_split_001.htmlux5cux23_idIndexMarker2386}{}{}\protect\hypertarget{part0026_split_001.htmlux5cux23_idIndexMarker2387}{}{}mail
  user agent'' (MUA or UA) that lets users read and compose mail
\item
  A
  ``\protect\hypertarget{part0026_split_001.htmlux5cux23_idIndexMarker2388}{}{}\protect\hypertarget{part0026_split_001.htmlux5cux23_idIndexMarker2389}{}{}mail
  submission agent'' (MSA) that accepts outgoing mail from an MUA,
  grooms it, and submits it to the transport system
\item
  A
  ``\protect\hypertarget{part0026_split_001.htmlux5cux23_idIndexMarker2390}{}{}\protect\hypertarget{part0026_split_001.htmlux5cux23_idIndexMarker2391}{}{}mail
  transport agent'' (MTA) that routes messages among machines
\item
  A
  ``\protect\hypertarget{part0026_split_001.htmlux5cux23_idIndexMarker2392}{}{}\protect\hypertarget{part0026_split_001.htmlux5cux23_idIndexMarker2393}{}{}delivery
  agent'' (DA) that places messages in a local message store (the
  receiving users' mailboxes or, sometimes, a database)
\item
  An optional
  ``\protect\hypertarget{part0026_split_001.htmlux5cux23_idIndexMarker2394}{}{}\protect\hypertarget{part0026_split_001.htmlux5cux23_idIndexMarker2395}{}{}access
  agent'' (AA) that connects the user agent to the message store (e.g.,
  through the IMAP or POP protocol)
\end{itemize}

Note that these functional divisions are somewhat abstract. Real-world
mail systems break out these roles into somewhat different packages.

Attached to some of these functions are tools for recognizing spam,
viruses, and (outbound) internal company secrets.
\protect\hyperlink{part0026_split_001.htmlux5cux23_idTextAnchor1003}{Exhibit
A} illustrates how the various pieces fit together as a message winds
its way from sender to
receiver.\protect\hypertarget{part0026_split_001.htmlux5cux23_idIndexMarker2396}{}{}

\paragraph[{Exhibit A: }Mail system components]{\texorpdfstring{{Exhibit
A:
}\protect\hypertarget{part0026_split_001.htmlux5cux23_idTextAnchor1003}{}{}Mail
system components}{Exhibit A: Mail system components}}

%\includegraphics{images/00784.jpeg}

\protect\hypertarget{part0026_split_002.html}{}{}

\hypertarget{part0026_split_002.htmlux5cux23_idContainer1247}{}
\hypertarget{part0026_split_002.htmlux5cux23calibre_pb_1}{%
\subsection[User
agents]{\texorpdfstring{\protect\hypertarget{part0026_split_002.htmlux5cux23_idTextAnchor1004}{}{}\protect\hypertarget{part0026_split_002.htmlux5cux23_idIndexMarker2397}{}{}User
agents}{User agents}}\label{part0026_split_002.htmlux5cux23calibre_pb_1}}

\protect\hypertarget{part0026_split_002.htmlux5cux23_idIndexMarker2398}{}{}\protect\hypertarget{part0026_split_002.htmlux5cux23_idIndexMarker2399}{}{}\protect\hypertarget{part0026_split_002.htmlux5cux23_idIndexMarker2400}{}{}\protect\hypertarget{part0026_split_002.htmlux5cux23_idIndexMarker2401}{}{}Email
users run a user agent (sometimes called an email client) to read and
compose messages. Email messages originally consisted only of text, but
a standard known as Multipurpose Internet Mail Extensions (MIME) now
encodes text formats and attachments (including viruses) into email. It
is supported by most user agents. Since MIME generally does not affect
the addressing or transport of mail, we do not discuss it further.

\protect\hypertarget{part0026_split_002.htmlux5cux23_idIndexMarker2402}{}{}{/bin/mail}
was the original user agent, and it remains the ``good ol' standby'' for
reading text email messages at a shell prompt. Since email on the
Internet has moved far beyond the text era, text-based user agents are
no longer practical for most users. But we shouldn't throw {/bin/mail}
away; it's still a handy interface for scripts and other programs.

One of the elegant features illustrated in
\protect\hyperlink{part0026_split_001.htmlux5cux23_idTextAnchor1003}{Exhibit
A} is that a user agent doesn't necessarily need to be running on the
same system---or even on the same platform---as the rest of your mail
system. Users can reach their email from a Windows laptop or smartphone
through access agent protocols such as IMAP and POP.

\protect\hypertarget{part0026_split_003.html}{}{}

\hypertarget{part0026_split_003.htmlux5cux23_idContainer1247}{}
\hypertarget{part0026_split_003.htmlux5cux23calibre_pb_2}{%
\subsection[Submission
agents]{\texorpdfstring{\protect\hypertarget{part0026_split_003.htmlux5cux23_idTextAnchor1005}{}{}Submission
agents}{Submission agents}}\label{part0026_split_003.htmlux5cux23calibre_pb_2}}

\protect\hypertarget{part0026_split_003.htmlux5cux23_idIndexMarker2403}{}{}\protect\hypertarget{part0026_split_003.htmlux5cux23_idIndexMarker2404}{}{}\protect\hypertarget{part0026_split_003.htmlux5cux23_idIndexMarker2405}{}{}MSAs,
a late addition to the email pantheon, were invented to offload some of
the computational tasks of MTAs. MSAs make it easy for mail hub servers
to distinguish incoming from outbound email (when making decisions about
allowing relaying, for example) and give user agents a uniform and
simple configuration for outbound mail.

The MSA is a sort of ``receptionist'' for new messages being injected
into the system by local user agents. An MSA sits between the user agent
and the transport agent and takes over several functions that were
formerly a part of the MTA's job. An MSA implements secure (encrypted
and authenticated) communication with user agents and often does minor
header rewriting and cleanup on incoming messages. In many cases, the
MSA is really just the MTA listening on a different port with a
different configuration applied.

MSAs speak the same mail transfer protocol used by MTAs, so they appear
to be MTAs from the perspective of user agents. However, they typically
listen for connections on port 587 rather than port 25, the MTA
standard. For this scheme to work, user agents must connect on port 587
instead of port 25. If your user agents cannot be taught to use port
587, you can still run an MSA on port 25, but you must do so on a system
other than the one that runs your MTA; only one process at a time can
listen on a particular port.

If you use an MSA, be sure to configure your transport agent so that it
doesn't duplicate any of the rewriting or header fix-up work done by the
MSA. Duplicate processing won't affect the correctness of mail handling,
but it does represent useless extra work.

\leavevmode\hypertarget{part0026_split_003.htmlux5cux23_idContainer1103}{}%
See
\protect\hyperlink{part0026_split_012.htmlux5cux23_idTextAnchor1018}{this
page} for more information about SMTP authentication.

Since your MSA uses your MTA to relay messages, the MSA and MTA must use
SMTP-AUTH to authenticate each other. Otherwise, you create a so-called
open relay that spammers can exploit and that other sites will blacklist
you for.

\protect\hypertarget{part0026_split_004.html}{}{}

\hypertarget{part0026_split_004.htmlux5cux23_idContainer1247}{}
\hypertarget{part0026_split_004.htmlux5cux23calibre_pb_3}{%
\subsection[Transport
agents]{\texorpdfstring{\protect\hypertarget{part0026_split_004.htmlux5cux23_idTextAnchor1006}{}{}Transport
agents}{Transport agents}}\label{part0026_split_004.htmlux5cux23calibre_pb_3}}

\protect\hypertarget{part0026_split_004.htmlux5cux23_idIndexMarker2406}{}{}\protect\hypertarget{part0026_split_004.htmlux5cux23_idIndexMarker2407}{}{}\protect\hypertarget{part0026_split_004.htmlux5cux23_idIndexMarker2408}{}{}A
transport agent must accept mail from a user agent or submission agent,
understand the recipients' addresses, and somehow get the mail to the
correct hosts for delivery. Transport agents speak the Simple Mail
Transport Protocol (SMTP), which was originally defined in RFC821 but
has now been superseded and extended by RFC5321. The extended version is
called
\protect\hypertarget{part0026_split_004.htmlux5cux23_idIndexMarker2409}{}{}ESMTP.

An MTA's list of chores, as both a mail sender and receiver, includes

\begin{itemize}
\item
  Receiving email messages from remote mail servers
\item
  Understanding the recipients' addresses
\item
  Rewriting addresses to a form understood by the delivery agent
\item
  Forwarding the message to the next responsible mail server or passing
  it to a local delivery agent to be saved to a user's mailbox
\end{itemize}

The bulk of the work involved in setting up a mail system relates to the
configuration of the MTA. In this book, we cover three open source MTAs:
{sendmail}, Exim, and Postfix.

\protect\hypertarget{part0026_split_005.html}{}{}

\hypertarget{part0026_split_005.htmlux5cux23_idContainer1247}{}
\hypertarget{part0026_split_005.htmlux5cux23calibre_pb_4}{%
\subsection[Local delivery
agents]{\texorpdfstring{\protect\hypertarget{part0026_split_005.htmlux5cux23_idTextAnchor1007}{}{}Local
delivery
agents}{Local delivery agents}}\label{part0026_split_005.htmlux5cux23calibre_pb_4}}

\protect\hypertarget{part0026_split_005.htmlux5cux23_idIndexMarker2410}{}{}\protect\hypertarget{part0026_split_005.htmlux5cux23_idIndexMarker2411}{}{}\protect\hypertarget{part0026_split_005.htmlux5cux23_idIndexMarker2412}{}{}A
delivery agent, sometimes called a local delivery agent (LDA), accepts
mail from a transport agent and delivers it to the appropriate
recipients' mailboxes on the local machine. As originally specified,
email can be delivered to a person, to a mailing list, to a file, or
even to a program. However, the last two types of recipients can weaken
the security and safety of your system.

MTAs usually include a built-in local delivery agent for easy
deliveries.
\protect\hypertarget{part0026_split_005.htmlux5cux23_idIndexMarker2413}{}{}{procmail}
(procmail.org) and
\protect\hypertarget{part0026_split_005.htmlux5cux23_idIndexMarker2414}{}{}Maildrop
(\href{http://courier-mta.org/maildrop}{courier-mta.org/maildrop}) are
LDAs that can filter or sort mail before delivering it. Some access
agents (AAs) also have built-in LDAs that do both delivery and local
housekeeping chores.

\protect\hypertarget{part0026_split_006.html}{}{}

\hypertarget{part0026_split_006.htmlux5cux23_idContainer1247}{}
\hypertarget{part0026_split_006.htmlux5cux23calibre_pb_5}{%
\subsection[Message
stores]{\texorpdfstring{\protect\hypertarget{part0026_split_006.htmlux5cux23_idTextAnchor1008}{}{}Message
stores}{Message stores}}\label{part0026_split_006.htmlux5cux23calibre_pb_5}}

A message store is the final resting place of an email message once it
has completed its journey across the Internet and been delivered to
recipients.

Mail has traditionally been stored in either
\protect\hypertarget{part0026_split_006.htmlux5cux23_idIndexMarker2415}{}{}\protect\hypertarget{part0026_split_006.htmlux5cux23_idIndexMarker2416}{}{}{mbox}
format or
\protect\hypertarget{part0026_split_006.htmlux5cux23_idIndexMarker2417}{}{}\protect\hypertarget{part0026_split_006.htmlux5cux23_idIndexMarker2418}{}{}{Maildir}
format. The former stores all mail in a single file, typically
{/var/mail/}{username}, with individual messages separated by a special
From line. {Maildir} format stores each message in a separate file. A
file for each message is more convenient but creates directories with
many, many small files; some filesystems may not be amused.

Flat files in {mbox} or {Maildir} format are still widely used, but ISPs
with thousands or millions of email clients have typically migrated to
other technologies for their message stores, usually databases.
Unfortunately, that means that message stores are becoming more opaque.

\protect\hypertarget{part0026_split_007.html}{}{}

\hypertarget{part0026_split_007.htmlux5cux23_idContainer1247}{}
\hypertarget{part0026_split_007.htmlux5cux23calibre_pb_6}{%
\subsection[Access
agents]{\texorpdfstring{\protect\hypertarget{part0026_split_007.htmlux5cux23_idTextAnchor1009}{}{}Access
agents}{Access agents}}\label{part0026_split_007.htmlux5cux23calibre_pb_6}}

\protect\hypertarget{part0026_split_007.htmlux5cux23_idIndexMarker2419}{}{}\protect\hypertarget{part0026_split_007.htmlux5cux23_idIndexMarker2420}{}{}\protect\hypertarget{part0026_split_007.htmlux5cux23_idIndexMarker2421}{}{}Two
protocols access message stores and download email messages to a local
device (workstation, laptop, smartphone, etc.): Internet Message Access
Protocol {version 4} (IMAP4) and Post Office Protocol version 3 (POP3).
Earlier versions of these protocols had security issues. Be sure to use
a version (IMAPS or POP3S) that incorporates SSL encryption and hence
does not transmit passwords in cleartext over the Internet.

IMAP is significantly better than POP. It delivers your mail one message
at a time rather than all at once, which is kinder to the network
(especially on slow links) and better for someone who travels from
location to location. IMAP is especially good at dealing with the giant
attachments that some folks like to send: you can browse the headers of
your messages and not download the attachments until you are ready to
deal with them.



\section{Anatomy of a mail message}

A mail message has three distinct parts:

\begin{itemize}
\item
  Envelope
\item
  Headers
\item
  Body of the message
\end{itemize}

\protect\hypertarget{part0026_split_008.htmlux5cux23_idIndexMarker2423}{}{}The
envelope determines where the message will be delivered or, if the
message can't be delivered, to whom it should be returned. The envelope
is invisible to users and is not part of the message itself; it's used
internally by the MTA.

\protect\hypertarget{part0026_split_008.htmlux5cux23_idIndexMarker2424}{}{}\protect\hypertarget{part0026_split_008.htmlux5cux23_idIndexMarker2425}{}{}Envelope
addresses generally agree with the From and To lines of the header when
the sender and recipient are individuals. The envelope and headers might
not agree if the message was sent to a mailing list or was generated by
a spammer who is trying to conceal his identity.

Headers are a collection of property/value pairs as specified in RFC5322
(updated by RFC6854). They record all kinds of information about the
message, such as the date and time it was sent, the transport agents
through which it passed on its journey, and who it is to and from. The
headers are a bona fide part of the mail message, but user agents
typically hide the less interesting ones when displaying messages for
the user.

The body of the message is the content to be sent. It usually consists
of plain text, although that text often represents a mail-safe encoding
for various types of binary or rich-text content.

Dissecting mail headers to locate problems within the mail system is an
essential sysadmin skill. Many user agents hide the headers, but there
is usually a way to see them, even if you have to use an editor on the
message store.

Below are most of the headers (with occasional truncations indicated by
\ldots) from a typical nonspam message. We removed another half page of
headers that Gmail uses as part of its spam filtering. (In memory of
Evi, who originally owned this chapter, this historical example has been
kept intact.)

%\includegraphics{images/00785.gif}

To decode this beast, start reading the Received lines, but start from
the bottom (sender side). This message went from David Schweikert's home
machine in the schweikert.ch domain to his mail server
(mail.schweikert.ch), where it was scanned for viruses. It was then
forwarded to the recipient evi@atrust.com. However, the receiving host
mail-relay.atrust.com sent it on to sailingevi@gmail.com, where it
entered Evi's mailbox.

\leavevmode\hypertarget{part0026_split_008.htmlux5cux23_idContainer1105}{}%
See
\protect\hyperlink{part0026_split_015.htmlux5cux23_idTextAnchor1024}{this
page} for more information about SPF.

Midway through the headers, you see an
\protect\hypertarget{part0026_split_008.htmlux5cux23_idIndexMarker2426}{}{}\protect\hypertarget{part0026_split_008.htmlux5cux23_idIndexMarker2427}{}{}\protect\hypertarget{part0026_split_008.htmlux5cux23_idIndexMarker2428}{}{}SPF
(Sender Policy Framework) validation failure, an indication that the
message has been flagged as spam. This failure happened because Google
checked the IP address of mail-relay.atrust.com and compared it with the
SPF record at {schweikert.ch}; of course, it doesn't match. This is an
inherent weakness of relying on SPF records to identify forgeries---they
don't work for mail that has been relayed.

You can often see the MTAs that were used (Postfix at schweikert.ch,
{sendmail} 8.12 at atrust.com), and in this case, you can also see that
virus scanning was performed through {amavisd-new} on port 10,024 on a
machine running Debian Linux. You can follow the progress of the message
from the Central European Summer Time zone (CEST +0200), to Colorado
(-0600), and on to the Gmail server (PDT -0700); the numbers are the
differences between local time and UTC, Coordinated Universal Time. A
lot of info is stashed in the headers!

Here are the headers, again truncated, from a spam message:

%\includegraphics{images/00786.gif}

According to the From header, this message's sender is alert@atrust.com.
But according to the Return-Path header, which contains a copy of the
envelope sender, the originator was smotheringl39@sherman.dp.ua, an
address in the Ukraine. The first MTA that handled the message is at IP
address 187.10.167.249, which is in Brazil. Sneaky spammers\ldots{} It's
important to note that many of the lines in the header, including the
Received lines, may have been forged. Use this data with extreme
caution.

The SPF check at Google fails again, this time with a ``neutral'' result
because the domain sherman.dp.ua does not have an SPF record with which
to compare the IP address of mail-relay.atrust.com.

The recipient information is also at least partially untrue. The To
header says the message is addressed to ned@atrust.com. However, the
envelope recipient addresses must have included evi@atrust.com in order
for the message to be forwarded to sailingevi@gmail.com for delivery.


\section{The SMTP protocol}

The
\protect\hypertarget{part0026_split_009.htmlux5cux23_idIndexMarker2429}{}{}Simple
Mail Transport Protocol (SMTP) and its extended version,
\protect\hypertarget{part0026_split_009.htmlux5cux23_idIndexMarker2430}{}{}ESMTP,
have been standardized in the RFC series (RFC5321, updated by RFC7504)
and are used for most message hand-offs among the various pieces of the
mail system:

\begin{itemize}
\item
  UA-to-MSA or -MTA as a message is injected into the mail system
\item
  MSA-to-MTA as the message starts its delivery journey
\item
  MTA- or MSA-to-antivirus or -antispam scanning programs
\item
  MTA-to-MTA as a message is forwarded from one site to another
\item
  MTA-to-DA as a message is delivered to the local message store
\end{itemize}

Because the format of messages and the transfer protocol are both
standardized, my MTA and your MTA don't have to be the same or even know
each other's identity; they just have to both speak SMTP or ESMTP. Your
various mail servers can run different MTAs and interoperate just fine.

True to its name, SMTP is\ldots simple. An MTA connects to your mail
server and says, ``Here's a message; please deliver it to
user@your.domain.'' Your MTA says ``OK.''

Requiring strict adherence to the SMTP protocol has become a technique
for fighting spam and malware, so it's important for mail administrators
to be somewhat familiar with the protocol. The language has only a few
commands;
\protect\hyperlink{part0026_split_009.htmlux5cux23_idTextAnchor1012}{Table
18.1} shows the most important ones.

\paragraph[{Table 18.1: }SMTP commands]{\texorpdfstring{{Table 18.1:
}\protect\hypertarget{part0026_split_009.htmlux5cux23_idIndexMarker2431}{}{}\protect\hypertarget{part0026_split_009.htmlux5cux23_idTextAnchor1012}{}{}\protect\hypertarget{part0026_split_009.htmlux5cux23_idTextAnchor1013}{}{}SMTP
commands}{Table 18.1: SMTP commands}}

%\includegraphics{images/00787.gif}

\protect\hypertarget{part0026_split_010.html}{}{}

\hypertarget{part0026_split_010.htmlux5cux23_idContainer1247}{}
\hypertarget{part0026_split_010.htmlux5cux23calibre_pb_9}{%
\subsection[You had me at
EHLO]{\texorpdfstring{\protect\hypertarget{part0026_split_010.htmlux5cux23_idTextAnchor1014}{}{}You
had me at
EHLO}{You had me at EHLO}}\label{part0026_split_010.htmlux5cux23calibre_pb_9}}

ESMTP speakers start conversations with EHLO instead of HELO. If the
process at the other end understands and responds with an OK, then the
participants negotiate supported extensions and agree on a lowest common
denominator for the exchange. If the peer returns an error in response
to the EHLO, then the ESMTP speaker falls back to SMTP. But today,
almost everything uses ESMTP.

A typical SMTP conversation to deliver an email message goes as follows:
HELO or EHLO, MAIL FROM:, RCPT TO:, DATA, and QUIT. The sender does most
of the talking, with the recipient contributing error codes and
acknowledgments.

\protect\hypertarget{part0026_split_010.htmlux5cux23_idIndexMarker2432}{}{}\protect\hypertarget{part0026_split_010.htmlux5cux23_idIndexMarker2433}{}{}SMTP
and ESMTP are both text-based protocols, so you can use them directly
when debugging the mail system. Just {telnet} to TCP port 25 or 587 and
start entering SMTP commands. See the example
\protect\hyperlink{part0026_split_012.htmlux5cux23_idTextAnchor1019}{here}.

\protect\hypertarget{part0026_split_011.html}{}{}

\hypertarget{part0026_split_011.htmlux5cux23_idContainer1247}{}
\hypertarget{part0026_split_011.htmlux5cux23calibre_pb_10}{%
\subsection[SMTP error
codes]{\texorpdfstring{\protect\hypertarget{part0026_split_011.htmlux5cux23_idTextAnchor1015}{}{}SMTP
error
codes}{SMTP error codes}}\label{part0026_split_011.htmlux5cux23calibre_pb_10}}

\protect\hypertarget{part0026_split_011.htmlux5cux23_idIndexMarker2434}{}{}Also
specified in the RFCs that define SMTP are a set of temporary and
permanent error codes. These were originally three-digit codes (e.g.,
550), with each digit being interpreted separately. A first digit of 2
indicated success, a 4 signified a temporary error, and a 5 indicated a
permanent error.

The three-digit error code system did not scale, so RFC3463 (updated by
RFCs 3886, 4468, 4865, 4954, and 5248) restructured it to create more
flexibility. It defined an expanded error code format known as a
delivery status notification or DSN. DSNs have the format X.X.X instead
of the old XXX, and each of the individual Xs can be a multidigit
number. The initial X must still be 2, 4, or 5. The second digit
specifies a topic, and the third provides the details. The new system
uses the second number to distinguish host errors from mailbox errors.
\protect\hyperlink{part0026_split_011.htmlux5cux23_idTextAnchor1016}{Table
18.2} lists a few of the DSN codes. RFC3463's Appendix A shows them all.

\paragraph[{Table 18.2: }RFC3463 delivery status
notifications]{\texorpdfstring{{Table 18.2:
}\protect\hypertarget{part0026_split_011.htmlux5cux23_idIndexMarker2435}{}{}\protect\hypertarget{part0026_split_011.htmlux5cux23_idTextAnchor1016}{}{}\protect\hypertarget{part0026_split_011.htmlux5cux23_idTextAnchor1017}{}{}RFC3463
delivery status
notifications}{Table 18.2: RFC3463 delivery status notifications}}

%\includegraphics{images/00788.gif}

\protect\hypertarget{part0026_split_012.html}{}{}

\hypertarget{part0026_split_012.htmlux5cux23_idContainer1247}{}
\hypertarget{part0026_split_012.htmlux5cux23calibre_pb_11}{%
\subsection[SMTP
authentication]{\texorpdfstring{\protect\hypertarget{part0026_split_012.htmlux5cux23_idTextAnchor1018}{}{}SMTP
authentication}{SMTP authentication}}\label{part0026_split_012.htmlux5cux23calibre_pb_11}}

\protect\hypertarget{part0026_split_012.htmlux5cux23_idIndexMarker2436}{}{}RFC4954
(updated by RFC5248) defines an extension to the original SMTP protocol
that allows an SMTP client to identify and
\protect\hypertarget{part0026_split_012.htmlux5cux23_idIndexMarker2437}{}{}\protect\hypertarget{part0026_split_012.htmlux5cux23_idIndexMarker2438}{}{}\protect\hypertarget{part0026_split_012.htmlux5cux23_idIndexMarker2439}{}{}\protect\hypertarget{part0026_split_012.htmlux5cux23_idIndexMarker2440}{}{}\protect\hypertarget{part0026_split_012.htmlux5cux23_idIndexMarker2441}{}{}authenticate
itself to a mail server. The server might then let the client relay mail
through it. The protocol supports several different authentication
mechanisms. The exchange is as follows:

{1.}The client says EHLO, announcing that it speaks ESMTP.

{2.}The server responds and advertises its authentication mechanisms.

{3.}The client says AUTH and names a specific mechanism that it wants to
use, optionally including its authentication data.

{4.}The server accepts the data sent with AUTH or starts a challenge and
response sequence with the client.

{5.}The server either accepts or denies the authentication attempt.

\protect\hypertarget{part0026_split_012.htmlux5cux23_idTextAnchor1019}{}{}To
see what authentication mechanisms a server supports, you can {telnet}
to port 25 and say EHLO. For example, here is a truncated conversation
with the mail server mail-relay.atrust.com (the commands we typed are in
bold):

%\includegraphics{images/00789.gif}

In this case, the mail server supports the LOGIN and PLAIN
authentication mechanisms. {sendmail}, Exim, and Postfix all support
SMTP authentication; details of configuration are covered
\protect\hyperlink{part0026_split_038.htmlux5cux23_idTextAnchor1100}{here},
\protect\hyperlink{part0026_split_049.htmlux5cux23_idTextAnchor1144}{here},
and
\protect\hyperlink{part0026_split_063.htmlux5cux23_idTextAnchor1193}{here},
respectively.


\section{Spam and malware}

\protect\hypertarget{part0026_split_013.htmlux5cux23_idIndexMarker2442}{}{}Spam
is the jargon word for junk mail, also known as unsolicited commercial
email or UCE. It is one of the most universally hated aspects of the
Internet. Once upon a time, system administrators spent many hours each
week hand-tuning block lists and adjusting decision weights in
home-grown spam filtering tools. Unfortunately, spammers have become so
crafty and commercialized that these measures are no longer an effective
use of system administrators' time.

In this section we cover the basic antispam features of each MTA.
However, there's a certain futility to any attempt to fight spam as a
lone vigilante. You should really pay for a
\protect\hypertarget{part0026_split_013.htmlux5cux23_idIndexMarker2443}{}{}cloud-based
spam-fighting service (such as
\protect\hypertarget{part0026_split_013.htmlux5cux23_idIndexMarker2444}{}{}McAfee
SaaS Email Protection,
\protect\hypertarget{part0026_split_013.htmlux5cux23_idIndexMarker2445}{}{}Google
G Suite, or
\protect\hypertarget{part0026_split_013.htmlux5cux23_idIndexMarker2446}{}{}Barracuda)
and leave the spam fighting to the professionals who love that stuff.
They have better intelligence about the state of the global emailsphere
and can react far more quickly to new information than you can.

Spam has become a serious problem because although the absolute response
rate is low, the responses per dollar spent is high. (A list of 30
million email addresses costs about \$20.) If it didn't work for the
spammers, it wouldn't be such a problem. Surveys show that 95\%--98\% of
all mail is spam.

There are even venture-capital-funded companies whose entire mission is
to deliver spam less expensively and more efficiently (although they
typically call it ``marketing email'' rather than spam). If you work at
or buy services from one of these companies, we're not sure how you
sleep at night.

In all cases, advise your users to simply delete the spam they receive.
Many spam messages contain instructions that purport to explain how
recipients can be removed from the mailing list. If you follow those
instructions, however, the spammers may remove you from the current
list, but they immediately add you to several other lists with the
annotation ``reaches a real human who reads the message.'' Your email
address is then worth even more.

\protect\hypertarget{part0026_split_014.html}{}{}

\hypertarget{part0026_split_014.htmlux5cux23_idContainer1247}{}
\hypertarget{part0026_split_014.htmlux5cux23calibre_pb_13}{%
\subsection[Forgeries]{\texorpdfstring{\protect\hypertarget{part0026_split_014.htmlux5cux23_idTextAnchor1021}{}{}Forgeries}{Forgeries}}\label{part0026_split_014.htmlux5cux23calibre_pb_13}}

\protect\hypertarget{part0026_split_014.htmlux5cux23_idIndexMarker2447}{}{}Forging
email is trivial; many user agents let you fill in the sender's address
with anything you want. MTAs can use SMTP authentication between local
servers, but that doesn't scale to Internet sizes. Some MTAs add warning
headers to outgoing local messages that they think might be forged.

Any user can be impersonated in mail messages. Be careful if email is
your organization's authorization vehicle for things like door keys,
access cards, and money. The practice of targeting users with forged
email is commonly called ``phishing.'' You should warn administrative
users of this fact and suggest that if they see suspicious mail that
appears to come from a person in authority, they should verify the
validity of the message. Caution is doubly appropriate if the message
asks that unreasonable privileges be given to an unusual person.

\protect\hypertarget{part0026_split_015.html}{}{}

\hypertarget{part0026_split_015.htmlux5cux23_idContainer1247}{}
\hypertarget{part0026_split_015.htmlux5cux23calibre_pb_14}{%
\subsection[SPF and Sender
ID]{\texorpdfstring{\protect\hypertarget{part0026_split_015.htmlux5cux23_idTextAnchor1022}{}{}\protect\hypertarget{part0026_split_015.htmlux5cux23_idTextAnchor1023}{}{}\protect\hypertarget{part0026_split_015.htmlux5cux23_idTextAnchor1024}{}{}SPF
and Sender
ID}{SPF and Sender ID}}\label{part0026_split_015.htmlux5cux23calibre_pb_14}}

\protect\hypertarget{part0026_split_015.htmlux5cux23_idIndexMarker2448}{}{}\protect\hypertarget{part0026_split_015.htmlux5cux23_idIndexMarker2449}{}{}\protect\hypertarget{part0026_split_015.htmlux5cux23_idIndexMarker2450}{}{}\protect\hypertarget{part0026_split_015.htmlux5cux23_idIndexMarker2451}{}{}The
best way to fight spam is to stop it at its source. This sounds simple
and easy, but in reality it's almost an impossible challenge. The
structure of the Internet makes it difficult to track the real source of
a message and to verify its authenticity. The community needs a
sure-fire way to verify that the entity sending an email is actually who
or what it claims to be. Many proposals have addressed this problem, but
SPF and Sender ID have achieved the most traction.

SPF, or Sender Policy Framework, has been described by the IETF in
RFC7208. SPF defines a set of DNS records through which an organization
can identify its official outbound mail servers. MTAs can then refuse
email purporting to be from that organization's domain if the email does
not originate from one of these official sources. Of course, the system
only works well if the majority of organizations publish SPF records.

Sender ID and SPF are virtually identical in form and function. However,
key parts of Sender ID are patented by Microsoft, and hence it has been
the subject of much controversy. As of this writing (2017), Microsoft is
still trying to strong-arm the industry into adopting its proprietary
standards. The IETF chose not to choose and published RFC4406 on Sender
ID and RFC7208 on SPF. Organizations that implement this type of spam
avoidance strategy typically use SPF.

Messages that are relayed break both SPF and Sender ID, which is a
serious flaw in these systems. The receiver consults the SPF record for
the original sender to discover its list of authorized servers. However,
those addresses won't match any relay machines that were involved in
transporting the message. Be careful what decisions you make in response
to SPF failures.

\protect\hypertarget{part0026_split_016.html}{}{}

\hypertarget{part0026_split_016.htmlux5cux23_idContainer1247}{}
\hypertarget{part0026_split_016.htmlux5cux23calibre_pb_15}{%
\subsection[DKIM]{\texorpdfstring{\protect\hypertarget{part0026_split_016.htmlux5cux23_idTextAnchor1025}{}{}DKIM}{DKIM}}\label{part0026_split_016.htmlux5cux23calibre_pb_15}}

\protect\hypertarget{part0026_split_016.htmlux5cux23_idIndexMarker2452}{}{}DKIM
(DomainKeys Identified Mail) is a cryptographic signature system for
email messages. It lets the receiver verify not only the sender's
identity but also the fact that a message has not been tampered with in
transit. The system uses DNS records to publish a domain's cryptographic
keys and message-signing policy. DKIM is supported by all the MTAs
described in this chapter, but real-world deployment has been extremely
rare.

\section{Message privacy and encryption}

\protect\hypertarget{part0026_split_017.htmlux5cux23_idIndexMarker2453}{}{}\protect\hypertarget{part0026_split_017.htmlux5cux23_idIndexMarker2454}{}{}By
default, all mail is sent unencrypted. Educate your users that they
should never send sensitive data through email unless they make use of
an external encryption package or your organization has provided a
centralized encryption solution for email. Even {with} encryption,
electronic communication can never be guaranteed to be 100\% secure. You
pays your money and you takes your chances.

Historically, the most common external encryption packages have been
\protect\hypertarget{part0026_split_017.htmlux5cux23_idIndexMarker2455}{}{}\protect\hypertarget{part0026_split_017.htmlux5cux23_idIndexMarker2456}{}{}Pretty
Good Privacy (PGP), its GNUified clone
\protect\hypertarget{part0026_split_017.htmlux5cux23_idIndexMarker2457}{}{}GPG,
and
\protect\hypertarget{part0026_split_017.htmlux5cux23_idIndexMarker2458}{}{}S/MIME.
Both S/MIME and PGP are documented in the RFC series, with S/MIME being
on the standards track. Most common user agents support plug-ins for
both solutions.

These standards offer a basis for email confidentiality, authentication,
message integrity assurance, and nonrepudiation of origin.{ }But
although PGP/GPG and {S/MIME} are potentially viable solutions for
tech-savvy users who care about privacy, they have proved too cumbersome
for unsophisticated users. Both require some facility with cryptographic
key management and an understanding of the underlying encryption
strategy. (Pro tip: If you use PGP/GPG or S/MIME, you can increase your
odds of remaining secure by ensuring that your public key or certificate
is expired and replaced frequently. Long-term use of a key increases the
likelihood that it will be compromised without your awareness.)

\protect\hypertarget{part0026_split_017.htmlux5cux23_idTextAnchor1028}{}{}Most
organizations that handle sensitive data in email (especially ones that
communicate with the public, such as health care institutions) opt for a
centralized service that uses proprietary technology to encrypt
messages. Such systems can use either on-premises solutions (such as
\protect\hypertarget{part0026_split_017.htmlux5cux23_idIndexMarker2459}{}{}Cisco's
IronPort) that you deploy in your data center or cloud-based services
(such as
\protect\hypertarget{part0026_split_017.htmlux5cux23_idIndexMarker2460}{}{}Zix,
zixcorp.com) that can be configured to encrypt outbound messages
according to their contents or other rules. Centralized email encryption
is one category of service for which it's best to use a commercial
solution rather than rolling your own.

At least in the email realm,
\protect\hypertarget{part0026_split_017.htmlux5cux23_idIndexMarker2461}{}{}\protect\hypertarget{part0026_split_017.htmlux5cux23_idIndexMarker2462}{}{}\protect\hypertarget{part0026_split_017.htmlux5cux23_idIndexMarker2463}{}{}data
loss prevention (DLP) is a kissing cousin to centralized encryption. DLP
systems seek to avoid---or at least, detect---the leakage of proprietary
information into the stream of email leaving your organization. They
scan outbound email for potentially sensitive content. Suspicious
messages can be flagged, blocked, or returned to their senders. Our
recommendation is that you choose a centralized encryption platform that
also includes DLP capability; it's one less platform to manage.

\leavevmode\hypertarget{part0026_split_017.htmlux5cux23_idContainer1110}{}%
See
\protect\hyperlink{part0037_split_040.htmlux5cux23_idTextAnchor1727}{this
page} for more information about TLS.

In addition to encrypting transport between MTAs, it's important to
ensure that user-agent-to-access-agent communication is always
encrypted, especially because this channel typically employs some form
of user credentials to connect. Make sure that only the secure,
TLS-using versions of the IMAP and POP protocols are allowed by access
agents. (These are known as
\protect\hypertarget{part0026_split_017.htmlux5cux23_idIndexMarker2464}{}{}IMAPS
and
\protect\hypertarget{part0026_split_017.htmlux5cux23_idIndexMarker2465}{}{}POP3S,
respectively.)


\section{Mail aliases}

\protect\hypertarget{part0026_split_018.htmlux5cux23_idIndexMarker2467}{}{}Another
concept that is common to all MTAs is the use of aliases. Aliases allow
mail to be rerouted either by the system administrator or by individual
users.

Aliases can define mailing lists, forward mail among machines, or allow
users to be referred to by more than one name. Alias processing is
recursive, so it's legal for an alias to point to other destinations
that are themselves aliases.

Technically, aliases are configured only by sysadmins. A user's control
of mail routing through the use of a {.forward} file is not really
aliasing, but we have lumped them together here.

Sysadmins often use role or functional aliases (e.g.,
printers@example.com) to route email about a particular issue to
whatever person is currently handling that issue. Other examples might
include an alias that receives the results of a nightly security scan or
an alias for the postmaster in charge of email.

The most common method for configuring aliases is to use a simple flat
file such as the
\protect\hypertarget{part0026_split_018.htmlux5cux23_idIndexMarker2468}{}{}{/etc/mail/aliases}
file discussed later in this section. This method was originally
introduced by {sendmail}, but Exim and Postfix support it, too.

Most user agents also provide some sort of ``aliasing'' feature (usually
called ``my groups,'' ``my mailing lists,'' or something of that
nature). However, the user agent expands such aliases before mail ever
reaches an MSA or MTA. These aliases are internal to the user agent and
don't require support from the rest of the mail system.

\protect\hypertarget{part0026_split_018.htmlux5cux23_idIndexMarker2469}{}{}Aliases
can also be defined in a forwarding file in the home directory of each
user, usually
\protect\hypertarget{part0026_split_018.htmlux5cux23_idIndexMarker2470}{}{}{\textasciitilde/.forward}.
These aliases, which use a slightly nonstandard syntax, apply to all
mail delivered to that particular user. They're often used to forward
mail to a different account or to implement automatic ``I'm on
vacation'' responses.

MTAs look for aliases in the global {aliases} file ({/etc/mail/aliases}
or {/etc/aliases}) and then in recipients' forwarding files. Aliasing is
applied only to messages that the transport agent considers to be local.

The format of an entry in the {aliases} file is

%\includegraphics{images/00790.gif}

where {local-name} is the original address to be matched against
incoming messages and the recipient list contains either recipient
addresses or the names of other aliases. Indented lines are considered
continuations of the preceding lines.

From mail's point of view, the {aliases} file supersedes {/etc/passwd},
so the entry

%\includegraphics{images/00791.gif}

would prevent the local user david from ever receiving any mail.
Therefore, administrators and {adduser} tools should check both the
{passwd} file and the {aliases} file when selecting new usernames.

\protect\hypertarget{part0026_split_018.htmlux5cux23_idIndexMarker2471}{}{}\protect\hypertarget{part0026_split_018.htmlux5cux23_idIndexMarker2472}{}{}The
{aliases} file should always contain an alias named
``\protect\hypertarget{part0026_split_018.htmlux5cux23_idIndexMarker2473}{}{}postmaster''
that forwards mail to whoever maintains the mail system. Similarly, an
alias for ``abuse'' is appropriate in case someone outside your
organization needs to contact you regarding spam or suspicious network
behavior that originates at your site. An alias for automatic messages
from the MTA must also be present; it's usually called Mailer-Daemon and
is often aliased to postmaster.

Sadly, the mail system is so commonly abused these days that some sites
configure their standard contact addresses to throw mail away instead of
forwarding it to a human user. Entries such as

%\includegraphics{images/00792.gif}

are common. We don't recommend this practice, because humans who are
having trouble reaching your site by email do sometimes write to the
postmaster address.

A better paradigm might
be\protect\hypertarget{part0026_split_018.htmlux5cux23_idIndexMarker2474}{}{}

%\includegraphics{images/00793.gif}

You should redirect root's mail to your site's sysadmins or to someone
who logs in every day. The bin, sys, daemon, nobody, and hostmaster
accounts (and any other site-specific pseudo-user accounts you set up)
should all have similar aliases.

In addition to a list of users, aliases can refer to

\begin{itemize}
\item
  A file containing a list of addresses
\item
  A file to which messages should be appended
\item
  A command to which messages should be given as input
\end{itemize}

These last two targets should push your ``What about security?'' button,
because the sender of a message totally determines its content. Being
able to append that content to a file or deliver it as input to a
command sounds pretty scary. Many MTAs either disallow these alias
targets or severely limit the commands and file permissions that are
acceptable.

Aliases can cause
\protect\hypertarget{part0026_split_018.htmlux5cux23_idIndexMarker2475}{}{}mail
loops. MTAs try to detect loops that would cause mail to be forwarded
back and forth forever and return the errant messages to the sender. To
determine when mail is looping, an MTA can count the number of Received
lines in a message's header and stop forwarding it when the count
reaches a preset limit (usually 25). Each visit to a new machine is
called a ``hop'' in email jargon; returning a message to the sender is
known as ``bouncing'' it. So a more typically jargonized
\protect\hypertarget{part0026_split_018.htmlux5cux23_idIndexMarker2476}{}{}summary
of loop handling would be, ``Mail bounces after 25 hops.''

Another way MTAs can detect mail loops is by adding a Delivered-To
header for each host to which a message is forwarded. If an MTA finds
itself wanting to send a message to a host that's already mentioned in a
Delivered-To header, it knows the message has traveled in a loop.

In this chapter, we sometimes call a returned message a ``bounce'' and
sometimes call it an ``error.'' What we really mean is that a delivery
status notification (DSN, a specially formatted email message) has been
generated. Such a notification usually means that a message was
undeliverable and is therefore being returned to the sender.

\protect\hypertarget{part0026_split_019.html}{}{}

\hypertarget{part0026_split_019.htmlux5cux23_idContainer1247}{}
\hypertarget{part0026_split_019.htmlux5cux23calibre_pb_18}{%
\subsection[Getting aliases from
files]{\texorpdfstring{\protect\hypertarget{part0026_split_019.htmlux5cux23_idTextAnchor1032}{}{}Getting
aliases from
files}{Getting aliases from files}}\label{part0026_split_019.htmlux5cux23calibre_pb_18}}

The {:include:} directive in the {aliases} file (or a user's {.forward}
file) allows the list of
\protect\hypertarget{part0026_split_019.htmlux5cux23_idIndexMarker2477}{}{}targets
for the alias to be taken from the specified file. It is a great way to
let users manage their own local mailing lists. The included file can be
owned by the user and changed without involving a system administrator.
However, such an alias can also become a tasty and effective spam
expander, so don't let email from outside your site be directed there.

When setting up a list to use {:include:}, the sysadmin must enter the
alias into the global {aliases} file, create the included file, and
{chown} the included file to the user that is maintaining the mailing
list. For example, the {aliases} file might contain

%\includegraphics{images/00794.gif}

The file {ulsah.authors} should be on a local filesystem and should be
writable only by its owner. To be complete, we should also include
aliases for the mailing list's owner so that errors (bounces) are sent
to the owner of the list and not to the sender of a message addressed to
the list:

%\includegraphics{images/00795.gif}

\protect\hypertarget{part0026_split_020.html}{}{}

\hypertarget{part0026_split_020.htmlux5cux23_idContainer1247}{}
\hypertarget{part0026_split_020.htmlux5cux23calibre_pb_19}{%
\subsection[Mailing to
files]{\texorpdfstring{\protect\hypertarget{part0026_split_020.htmlux5cux23_idTextAnchor1033}{}{}Mailing
to
files}{Mailing to files}}\label{part0026_split_020.htmlux5cux23calibre_pb_19}}

\protect\hypertarget{part0026_split_020.htmlux5cux23_idIndexMarker2478}{}{}If
the target of an alias is an absolute pathname, messages are appended to
the specified file. The file must already exist. For example:

%\includegraphics{images/00796.gif}

If the pathname includes special characters, it must be enclosed in
double quotes.

It's useful to be able to send mail to files, but this feature arouses
the interest of the security police and is therefore restricted. This
syntax is only valid in the {aliases} file and in a user's {.forward}
file (or in a file that's interpolated into one of these files with the
{:include:} directive). A filename is not understood as a normal
address, so mail addressed to /etc/passwd@example.com would bounce.

If the destination file is referenced from the {aliases} file, it must
be world-writable (not advisable), setuid but not executable, or owned
by the MTA's default user. The identity of the default user is set in
the MTA's configuration file.

If the file is referenced in a {.forward} file, it must be owned and
writable by the original message recipient, who must be a valid user
with an entry in the {passwd} file and a valid shell that's listed in
{/etc/shells}. For files owned by root, use mode 4644 or 4600, setuid
but not executable.

\protect\hypertarget{part0026_split_021.html}{}{}

\hypertarget{part0026_split_021.htmlux5cux23_idContainer1247}{}
\hypertarget{part0026_split_021.htmlux5cux23calibre_pb_20}{%
\subsection[Mailing to
programs]{\texorpdfstring{\protect\hypertarget{part0026_split_021.htmlux5cux23_idTextAnchor1034}{}{}Mailing
to
programs}{Mailing to programs}}\label{part0026_split_021.htmlux5cux23calibre_pb_20}}

An alias can also route mail to the standard input of a program. This
behavior is
\protect\hypertarget{part0026_split_021.htmlux5cux23_idIndexMarker2479}{}{}specified
with a line such as

%\includegraphics{images/00797.gif}

It's even easier to create security holes with this feature than with
mailing to a file, so once again it is only permitted in {aliases},
{.forward}, or {:include:} files, and often requires the use of a
restricted shell.

\protect\hypertarget{part0026_split_022.html}{}{}

\hypertarget{part0026_split_022.htmlux5cux23_idContainer1247}{}
\hypertarget{part0026_split_022.htmlux5cux23calibre_pb_21}{%
\subsection[Building the hashed alias
database]{\texorpdfstring{\protect\hypertarget{part0026_split_022.htmlux5cux23_idTextAnchor1035}{}{}Building
the hashed alias
database}{Building the hashed alias database}}\label{part0026_split_022.htmlux5cux23calibre_pb_21}}

\protect\hypertarget{part0026_split_022.htmlux5cux23_idIndexMarker2480}{}{}Since
entries in the {aliases} file are unordered, it would be inefficient for
the MTA to search this file directly. Instead, a hashed version is
constructed with the Berkeley DB system. Hashing significantly speeds
alias lookups, especially when the file gets large.

\protect\hypertarget{part0026_split_022.htmlux5cux23_idTextAnchor1036}{}{}The
file derived
fro\protect\hypertarget{part0026_split_022.htmlux5cux23_idTextAnchor1037}{}{}m
{/etc/mail/aliases} is called {aliases.db}. If you are running Postfix
or {sendmail}, you must rebuild the hashed database with the
\protect\hypertarget{part0026_split_022.htmlux5cux23_idIndexMarker2481}{}{}{newaliases}
command every time you change the {aliases} file. Exim detects changes
to the {aliases }file automatically. Save the error output if you run
{newaliases} automatically---you might have introduced formatting errors
in the {aliases} file.



\section{Email configuration}

{\protect\hypertarget{part0026_split_023.htmlux5cux23_idIndexMarker2482}{}{}}\protect\hypertarget{part0026_split_023.htmlux5cux23_idTextAnchor1040}{}{}The
heart of an email system is its MTA, or mail transport agent.
\protect\hypertarget{part0026_split_023.htmlux5cux23_idIndexMarker2483}{}{}{sendmail}
is the original UNIX MTA, written by
\protect\hypertarget{part0026_split_023.htmlux5cux23_idIndexMarker2484}{}{}Eric
Allman while he was a graduate student many years ago. Since then, a
host of other MTAs have been developed. Some of them are commercial
products and some are open source implementations. In this chapter, we
cover three open source mail-transport agents: {sendmail},
\protect\hypertarget{part0026_split_023.htmlux5cux23_idIndexMarker2485}{}{}Postfix
by
\protect\hypertarget{part0026_split_023.htmlux5cux23_idIndexMarker2486}{}{}Wietse
Venema of
\protect\hypertarget{part0026_split_023.htmlux5cux23_idIndexMarker2487}{}{}IBM
Research, and
\protect\hypertarget{part0026_split_023.htmlux5cux23_idIndexMarker2488}{}{}Exim
by
\protect\hypertarget{part0026_split_023.htmlux5cux23_idIndexMarker2489}{}{}Philip
Hazel of the
\protect\hypertarget{part0026_split_023.htmlux5cux23_idIndexMarker2490}{}{}University
of Cambridge.

Configuration of the MTA can be a significant sysadmin chore.
Fortunately, the default or sample configurations that ship with MTAs
are often close to what the average site needs. You need not start from
scratch when configuring your MTA.

\protect\hypertarget{part0026_split_023.htmlux5cux23_idIndexMarker2491}{}{}SecuritySpace
(securityspace.com) does a survey monthly to determine the market share
of the various MTAs. In their June 2017 survey, 1.7 million out of 2
million MTAs surveyed replied with a banner that identified the MTA
software in use.
\protect\hyperlink{part0026_split_023.htmlux5cux23_idTextAnchor1041}{Table
18.3} shows these results, as well as the SecuritySpace results for 2009
and some 2001 values from a different survey.

\paragraph[{Table 18.3: }Mail transport agent market
share]{\texorpdfstring{{Table 18.3:
}\protect\hypertarget{part0026_split_023.htmlux5cux23_idTextAnchor1041}{}{}\protect\hypertarget{part0026_split_023.htmlux5cux23_idTextAnchor1042}{}{}Mail
transport agent market
share}{Table 18.3: Mail transport agent market share}}

%\includegraphics{images/00798.gif}

\protect\hypertarget{part0026_split_023.htmlux5cux23_idIndexMarker2492}{}{}The
trend is clearly away from {sendmail} and toward Exim and Postfix, with
Microsoft dropping to almost nothing. Keep in mind that this data
includes only MTAs that are directly exposed to the Internet.

For each of the MTAs we cover, we include details on the common areas of
interest:

\begin{itemize}
\item
  Configuration of simple clients
\item
  Configuration of an Internet-facing mail server
\item
  Control of both inbound and outbound mail routing
\item
  Stamping of mail as coming from a central server or the domain itself
\item
  Security
\item
  Debugging
\end{itemize}

If you are implementing a mail system from scratch and have no site
politics or biases to deal with, you may find it hard to choose an MTA.
{sendmail} is largely out of vogue, with the possible exception of pure
FreeBSD sites. Exim is powerful and highly configurable but suffers in
complexity. Postfix is simpler, faster, and was designed with security
as a primary goal. If your site or your sysadmins have a history with a
particular MTA, it's probably not worth switching unless you need
features that are not available from your old MTA.




\section{Postfix}
\label{sec:postfix}

Postfix is another popular alternative to {sendmail}.
Wietse Venema started the Postfix project when he spent a sabbatical year at IBM's T.J.\ Watson Research Center in 1996, and he is still actively developing it.
Postfix's design goals included not only security (first and foremost!), but also an open source distribution policy, speedy performance, robustness, and flexibility.
All major Linux distributions include Postfix, and since version 10.3, macOS has shipped Postfix instead of {sendmail} as its default mail system.

\leavevmode\hypertarget{part0026_split_057.htmlux5cux23_idContainer1213}{}%
See
\protect\hyperlink{part0014_split_023.htmlux5cux23_idTextAnchor367}{this
page} for more information about regular expressions.

The most important things to know about Postfix are, first, that it
works almost out of the box (the simplest config files are only a line
or two long), and second, that it leverages regular expression maps to
filter email effectively, especially in conjunction with the
\protect\hypertarget{part0026_split_057.htmlux5cux23_idIndexMarker2681}{}{}PCRE
(Perl-Compatible Regular Expression) library. Postfix is compatible with
{sendmail} in the sense that Postfix's {aliases} and {.forward} files
have the same format and semantics as those of {sendmail}.

Postfix speaks ESMTP. Virtual domains and spam filtering are both
supported. For address rewriting, Postfix relies on table lookups from
flat files, Berkeley DB, DBM, LDAP, NetInfo, or SQL databases.

\protect\hypertarget{part0026_split_058.html}{}{}

\hypertarget{part0026_split_058.htmlux5cux23_idContainer1247}{}
\hypertarget{part0026_split_058.htmlux5cux23calibre_pb_57}{%
\subsection[Postfix
architecture]{\texorpdfstring{\protect\hypertarget{part0026_split_058.htmlux5cux23_idTextAnchor1165}{}{}Postfix
architecture}{Postfix architecture}}\label{part0026_split_058.htmlux5cux23calibre_pb_57}}

\protect\hypertarget{part0026_split_058.htmlux5cux23_idIndexMarker2682}{}{}Postfix
comprises several small, cooperating programs that send network
messages, receive messages, deliver email locally, etc. Communication
among them is performed through local domain sockets or FIFOs. This
architecture is quite different from that of {sendmail} and Exim,
wherein a single large program does most of the work.

The {master} program starts and monitors all Postfix processes. Its
configuration file, {master.cf}, lists the subsidiary programs along
with information about how they should be started. The default values
set in that file cover most needs; in general, no tweaking is necessary.
One common change is to comment out a program, for example, {smtpd},
when a client should not listen on the SMTP port.

The most important server programs involved in the delivery of email are
shown in
\protect\hyperlink{part0026_split_058.htmlux5cux23_idTextAnchor1166}{Exhibit
B}.

\paragraph[{Exhibit B: }Postfix server
programs]{\texorpdfstring{{Exhibit B:
}\protect\hypertarget{part0026_split_058.htmlux5cux23_idIndexMarker2683}{}{}\protect\hypertarget{part0026_split_058.htmlux5cux23_idTextAnchor1166}{}{}Postfix
server programs}{Exhibit B: Postfix server programs}}

%\includegraphics{images/00885.jpeg}

\subsubsection[Receiving
mail]{\texorpdfstring{\protect\hypertarget{part0026_split_058.htmlux5cux23_idTextAnchor1167}{}{}Receiving
mail}{Receiving mail}}

\protect\hypertarget{part0026_split_058.htmlux5cux23_idIndexMarker2684}{}{}\protect\hypertarget{part0026_split_058.htmlux5cux23_idIndexMarker2685}{}{}{smtpd}
receives mail entering the system through SMTP. It also verifies that
the connecting clients are authorized to send the mail they are trying
to deliver. When email is sent locally through the {/usr/lib/sendmail}
compatibility program, a file is written to the
\protect\hypertarget{part0026_split_058.htmlux5cux23_idIndexMarker2686}{}{}{/var/spool/postfix/maildrop}
directory. That directory is periodically scanned by the
\protect\hypertarget{part0026_split_058.htmlux5cux23_idIndexMarker2687}{}{}{pickup}
program, which processes any new files it finds.

All incoming email passes through
\protect\hypertarget{part0026_split_058.htmlux5cux23_idIndexMarker2688}{}{}{cleanup},
which adds missing headers and rewrites addresses according to the
{canonical} and {virtual} maps. Before inserting mail into the
{incoming} queue, {cleanup} passes it through {trivial-rewrite}, which
does minor fixing of the addresses, such as appending a mail domain to
addresses that are not fully qualified.

\subsubsection[Managing mail-waiting
queues]{\texorpdfstring{\protect\hypertarget{part0026_split_058.htmlux5cux23_idTextAnchor1168}{}{}Managing
mail-waiting queues}{Managing mail-waiting queues}}

\protect\hypertarget{part0026_split_058.htmlux5cux23_idIndexMarker2689}{}{}\protect\hypertarget{part0026_split_058.htmlux5cux23_idIndexMarker2690}{}{}{qmgr}
manages five queues that contain mail waiting to be delivered:

\begin{itemize}
\item
  {incoming} -- mail that is arriving
\item
  {active} -- mail that is being delivered
\item
  {deferred} -- mail for which delivery has failed in the past
\item
  {hold} -- mail blocked in the queue by the administrator
\item
  {corrupt} -- mail that can't be read or parsed
\end{itemize}

The queue manager generally follows a simple FIFO strategy to select the
next message to process, but it also supports a complex preemption
algorithm that prefers messages with few recipients over bulk mail.

To avoid overwhelming a receiving host, especially one that has been
down, Postfix uses a slow-start algorithm to control how fast it tries
to deliver email. Deferred messages are given a try-again time stamp
that exponentially backs off so as not to waste resources on
undeliverable messages. A status cache of unreachable destinations
avoids unnecessary delivery attempts.

\subsubsection[Sending
mail]{\texorpdfstring{\protect\hypertarget{part0026_split_058.htmlux5cux23_idTextAnchor1169}{}{}Sending
mail}{Sending mail}}

\protect\hypertarget{part0026_split_058.htmlux5cux23_idIndexMarker2691}{}{}{qmgr},
aided by {trivial-rewrite}, decides where a message should be sent. The
routing decision made by {trivial-rewrite }can be overridden through
lookup tables ({transport\_maps}).

Delivery to remote hosts via the SMTP protocol is performed by the
{smtp} program.
\protect\hypertarget{part0026_split_058.htmlux5cux23_idIndexMarker2692}{}{}{lmtp}
delivers mail with
\protect\hypertarget{part0026_split_058.htmlux5cux23_idIndexMarker2693}{}{}\protect\hypertarget{part0026_split_058.htmlux5cux23_idIndexMarker2694}{}{}LMTP,
the Local Mail Transfer Protocol defined in RFC2033. LMTP is derived
from SMTP, but the protocol has been modified so that the mail server is
not required to manage a mail queue. This mailer is particularly useful
for delivering email to mailbox servers such as the Cyrus IMAP suite.

\protect\hypertarget{part0026_split_058.htmlux5cux23_idIndexMarker2695}{}{}{local}'s
job is to deliver email locally. It resolves addresses in the
\protect\hypertarget{part0026_split_058.htmlux5cux23_idIndexMarker2696}{}{}{aliases}
table and follows instructions found in recipients' {.forward} files.
Messages are forwarded to another address, passed to an external program
for processing, or stored in users' mail folders.

The {virtual} program delivers email to ``virtual mailboxes''; that is,
mailboxes that are not related to a local UNIX account but that still
represent valid email destinations. Finally, {pipe} implements delivery
through external programs.

\protect\hypertarget{part0026_split_059.html}{}{}

\hypertarget{part0026_split_059.htmlux5cux23_idContainer1247}{}
\hypertarget{part0026_split_059.htmlux5cux23calibre_pb_58}{%
\subsection[Security]{\texorpdfstring{Securi\protect\hypertarget{part0026_split_059.htmlux5cux23_idTextAnchor1170}{}{}ty}{Security}}\label{part0026_split_059.htmlux5cux23calibre_pb_58}}

\protect\hypertarget{part0026_split_059.htmlux5cux23_idIndexMarker2697}{}{}\protect\hypertarget{part0026_split_059.htmlux5cux23_idIndexMarker2698}{}{}Postfix
implements security at several levels. Most of the Postfix server
programs can run in a {chroot}ed environment. They are separate programs
with no {parent/child} relationship. None of them are setuid. The mail
drop directory is group-writable by the postdrop group, to which the
{postdrop}
\protect\hypertarget{part0026_split_059.htmlux5cux23_idIndexMarker2699}{}{}program
is setgid.

\protect\hypertarget{part0026_split_060.html}{}{}

\hypertarget{part0026_split_060.htmlux5cux23_idContainer1247}{}
\hypertarget{part0026_split_060.htmlux5cux23calibre_pb_59}{%
\subsection[Postfix commands and
documentation]{\texorpdfstring{\protect\hypertarget{part0026_split_060.htmlux5cux23_idTextAnchor1171}{}{}Postfix
commands and
documentation}{Postfix commands and documentation}}\label{part0026_split_060.htmlux5cux23calibre_pb_59}}

\protect\hypertarget{part0026_split_060.htmlux5cux23_idIndexMarker2700}{}{}Several
command-line utilities permit user interaction with the mail system:

\begin{itemize}
\item
  \protect\hypertarget{part0026_split_060.htmlux5cux23_idIndexMarker2701}{}{}{postalias}
  -- builds, modifies, and queries alias tables
\item
  \protect\hypertarget{part0026_split_060.htmlux5cux23_idIndexMarker2702}{}{}{postcat}
  -- prints the contents of queue files
\item
  \protect\hypertarget{part0026_split_060.htmlux5cux23_idIndexMarker2703}{}{}{postconf}
  -- displays and edits the main configuration file, {main.cf}
\item
  \protect\hypertarget{part0026_split_060.htmlux5cux23_idIndexMarker2704}{}{}{postfix}
  -- starts and stops the mail system (must be run as root)
\item
  \protect\hypertarget{part0026_split_060.htmlux5cux23_idIndexMarker2705}{}{}{postmap}
  -- builds, modifies, or queries lookup tables
\item
  \protect\hypertarget{part0026_split_060.htmlux5cux23_idIndexMarker2706}{}{}{postsuper}
  -- manages mail queues
\item
  {sendmail}, {mailq}, {newaliases} -- are {sendmail}-compatible
  replacements
\end{itemize}

The Postfix distribution includes a set of man pages that describe all
the programs and their options. On-line documents at postfix.org explain
how to configure and manage various aspects of Postfix. These documents
are also included in the Postfix distribution in the {README\_FILES}
directory.

\protect\hypertarget{part0026_split_061.html}{}{}

\hypertarget{part0026_split_061.htmlux5cux23_idContainer1247}{}
\hypertarget{part0026_split_061.htmlux5cux23calibre_pb_60}{%
\subsection[Postfix
configuration]{\texorpdfstring{\protect\hypertarget{part0026_split_061.htmlux5cux23_idTextAnchor1172}{}{}Postfix
configuration}{Postfix configuration}}\label{part0026_split_061.htmlux5cux23calibre_pb_60}}

\protect\hypertarget{part0026_split_061.htmlux5cux23_idIndexMarker2707}{}{}The
\protect\hypertarget{part0026_split_061.htmlux5cux23_idIndexMarker2708}{}{}{main.cf}
file is Postfix's principal configuration file. The
\protect\hypertarget{part0026_split_061.htmlux5cux23_idIndexMarker2709}{}{}{master.cf}
file configures the server programs. It also defines various lookup
tables that are referenced from {main.cf} and that provide different
types of service mappings.

The {postconf}(5) man page describes every parameter you can set in the
{main.cf} file. There is also a {postconf} program, so if you just type
{man postconf}, you'll get the man page for that instead of
{postconf}(5). Use {man -s 5 postconf} to get the right version.

The Postfix configuration language looks a bit like a series of {sh}
comments and assignment statements. Variables can be referenced in the
definition of other variables by being prefixed with a {\$}. Variable
definitions are stored just as they appear in the config file; they are
not expanded until they are used, and any substitutions occur at that
time.

You can create new variables by assigning values to them. Be careful to
choose names that do not conflict with existing configuration variables.

All Postfix configuration files, including the lookup tables, consider
lines starting with whitespace to be continuation lines. This convention
results in readable configuration files, but you must start new lines in
column one.

\subsubsection[What to put in
{main.cf}]{\texorpdfstring{\protect\hypertarget{part0026_split_061.htmlux5cux23_idTextAnchor1173}{}{}What
to put in {main.cf}}{What to put in main.cf}}

More than 500 parameters can be specified in the {main.cf} file.
However, just a few of them need to be set at an average site. The
author of Postfix strongly recommends that only parameters with
nondefault values be included in your configuration. That way, if the
default value of a parameter changes in the future, your configuration
will automatically adopt the new value.

The sample {main.cf} file that comes with the distribution includes many
{commented}-out example parameters, along with some brief documentation.
The original version is best left alone as a reference. Start with an
empty file for your own configuration so that your settings do not
become lost in a sea of comments.

\subsubsection[Basic
settings]{\texorpdfstring{\protect\hypertarget{part0026_split_061.htmlux5cux23_idTextAnchor1174}{}{}Basic
settings}{Basic settings}}

The simplest possible Postfix configuration is an empty file.
Surprisingly, this is a perfectly reasonable setup. It results in a mail
server that delivers email locally within the same domain as the local
hostname and that sends any messages directed to nonlocal addresses
directly to the appropriate remote servers.

\subsubsection[Null
client]{\texorpdfstring{\protect\hypertarget{part0026_split_061.htmlux5cux23_idTextAnchor1175}{}{}Null
client}{Null client}}

\protect\hypertarget{part0026_split_061.htmlux5cux23_idIndexMarker2710}{}{}Another
simple configuration is a ``null client''; that is, a system that
doesn't deliver email locally but rather forwards outbound mail to a
designated central server. To implement this configuration, you define
several parameters, starting with {mydomain}, which defines the domain
part of the hostname, and {myorigin}, which is the mail domain appended
to unqualified email addresses. If these two parameters are the same,
you can write something like this:

%\includegraphics{images/00886.gif}

Another parameter you should set is {mydestination}, which specifies the
mail domains that are local. If the recipient address of a message has
{mydestination} as its mail domain, the message is delivered through the
{local} program to the corresponding user (assuming that no relevant
alias or {.forward} file is found). If more than one mail domain is
included in{ mydestination}, these domains are all considered aliases
for the same domain.

For a null client, you want no local delivery, so leave this parameter
empty:

%\includegraphics{images/00887.gif}

Finally, the {relayhost} parameter tells Postfix to send all nonlocal
messages to a specified host instead of sending them directly to their
apparent destinations:

%\includegraphics{images/00888.gif}

The square brackets tell Postfix to treat the specified string as a
hostname (DNS A record) instead of a mail domain name (DNS MX record).

Since null clients should not receive mail from other systems, the last
thing to do in a null client configuration is to comment out the {smtpd}
line in the {master.cf} file. This change prevents Postfix from running
{smtpd} at all. With just these few lines, you've defined a fully
functional null client!

For a ``real'' mail server, you'll need a few more configuration options
as well as some mapping tables. We cover these in the next few sections.

\subsubsection[Use of
{postconf}]{\texorpdfstring{\protect\hypertarget{part0026_split_061.htmlux5cux23_idTextAnchor1176}{}{}Use
of {postconf}}{Use of postconf}}

{\protect\hypertarget{part0026_split_061.htmlux5cux23_idIndexMarker2711}{}{}}{postconf
}is a handy tool that helps you configure Postfix. When run without
arguments, it prints all the parameters as they are currently
configured. If you name a specific parameter as an argument, {postconf}
prints the value of that parameter. The {-d} option makes {postconf}
print the defaults instead of the currently configured values. For
example:

%\includegraphics{images/00889.gif}

Another useful option is {-n}, which tells {postconf} to print only the
parameters that differ from the default. If you ask for help on the
Postfix mailing list, that's the configuration information you should
put in your email.

\subsubsection[Lookup
tables]{\texorpdfstring{\protect\hypertarget{part0026_split_061.htmlux5cux23_idTextAnchor1177}{}{}Lookup
tables}{Lookup tables}}

\protect\hypertarget{part0026_split_061.htmlux5cux23_idIndexMarker2712}{}{}Many
aspects of Postfix's behavior are shaped through the use of lookup
tables, which can map keys to values or implement simple lists. For
example, the default setting for the {alias\_maps} table is

%\includegraphics{images/00890.gif}

Data sources are specified with the notation {type:path}. Multiple
values can be separated by commas, spaces, or both.
\protect\hyperlink{part0026_split_061.htmlux5cux23_idTextAnchor1178}{Table
18.19} lists the available data sources; {postconf -m} shows this
information as well.

\paragraph[{Table 18.19: }Information sources for Postfix lookup
tables]{\texorpdfstring{{Table 18.19:
}\protect\hypertarget{part0026_split_061.htmlux5cux23_idTextAnchor1178}{}{}\protect\hypertarget{part0026_split_061.htmlux5cux23_idTextAnchor1179}{}{}Information
sources for Postfix lookup
tables}{Table 18.19: Information sources for Postfix lookup tables}}

%\includegraphics{images/00891.gif}

The {dbm} and {sdbm} types are only for compatibility with the
traditional {sendmail} alias table. Berkeley DB ({hash}) is a more
modern implementation; it's safer and faster. If compatibility is not a
problem, then go
with\protect\hypertarget{part0026_split_061.htmlux5cux23_idIndexMarker2713}{}{}\protect\hypertarget{part0026_split_061.htmlux5cux23_idIndexMarker2714}{}{}

%\includegraphics{images/00892.gif}

The {alias\_database} specifies the table that is rebuilt by
\protect\hypertarget{part0026_split_061.htmlux5cux23_idIndexMarker2715}{}{}{newaliases}
and should correspond to the table that you specify in {alias\_maps}.
The two parameters are separate because {alias\_maps} might include
non-DB sources such as {mysql} that never need to be rebuilt.

All DB-class tables ({dbm}, {sdbm}, {hash}, and {btree}) compile a text
file to an {efficiently} searchable binary format. The syntax for these
text files is similar to that of the configuration files with respect to
comments and continuation lines. Entries are specified as simple
key/value pairs separated by whitespace, except for alias tables, which
use a colon after the key to retain {sendmail} compatibility. For
example, the following lines are appropriate for an alias table:

%\includegraphics{images/00893.gif}

As another example, here's an access table for relaying mail from any
client with a hostname ending in cs.colorado.edu.

%\includegraphics{images/00894.gif}

Text files are compiled to their binary formats with the {postmap}
command for normal tables and the {postalias} command for alias tables.
The table specification (including the type) must be given as the first
argument. For example:

%\includegraphics{images/00895.gif}

{postmap} can also query values in a lookup table (no match = no
output):

%\includegraphics{images/00896.gif}

\subsubsection[Local
delivery]{\texorpdfstring{\protect\hypertarget{part0026_split_061.htmlux5cux23_idTextAnchor1180}{}{}Local
delivery}{Local delivery}}

The
\protect\hypertarget{part0026_split_061.htmlux5cux23_idIndexMarker2716}{}{}{local}
program delivers mail to local recipients. It also handles local
aliasing. For example, if {mydestination }is set to cs.colorado.edu and
email arrives for the recipient evi@cs.colorado.edu, {local} first
consults the {alias\_maps} tables and then substitutes any matching
entries recursively.

If no aliases match, {local} looks for a
\protect\hypertarget{part0026_split_061.htmlux5cux23_idIndexMarker2717}{}{}{.forward}
file in user evi's home directory and follows the instructions in this
file if it exists. (The syntax is the same as for the right side of an
alias map.) Finally, if no {.forward} file is found, the email is
delivered to evi's local mailbox.

By default, {local} writes to standard {mbox}-format files under
{/var/mail}. You can change that behavior with the parameters shown in
\protect\hyperlink{part0026_split_061.htmlux5cux23_idTextAnchor1181}{Table
18.20}.

\paragraph[{Table 18.20: }Parameters for local mailbox delivery (set in
{main.cf})]{\texorpdfstring{{Table 18.20:
}\protect\hypertarget{part0026_split_061.htmlux5cux23_idTextAnchor1181}{}{}\protect\hypertarget{part0026_split_061.htmlux5cux23_idTextAnchor1182}{}{}Parameters
for local mailbox delivery (set in
{main.cf}){\protect\hypertarget{part0026_split_061.htmlux5cux23_idIndexMarker2718}{}{}\protect\hypertarget{part0026_split_061.htmlux5cux23_idIndexMarker2719}{}{}\protect\hypertarget{part0026_split_061.htmlux5cux23_idIndexMarker2720}{}{}\protect\hypertarget{part0026_split_061.htmlux5cux23_idIndexMarker2721}{}{}\protect\hypertarget{part0026_split_061.htmlux5cux23_idIndexMarker2722}{}{}}}{Table 18.20: Parameters for local mailbox delivery (set in main.cf)}}

%\includegraphics{images/00897.gif}

The {mail\_spool\_directory} and {home\_mailbox} options normally
generate {mbox}-format mailboxes, but they can also produce {Maildir}
mailboxes. To request this behavior, add a slash to the end of the
pathname.

If {recipient\_delimiter} is {+}, mail addressed to
evi+{whatever}@cs.colorado.edu is accepted for delivery to the evi
account. With this facility, users can create special-purpose addresses
and sort their mail by destination address. Postfix first attempts
lookups on the full address, and only if that fails does it strip the
extended components and fall back to the base address. Postfix also
looks for a corresponding forwarding file, {.forward+}{whatever}, for
further aliasing.

\protect\hypertarget{part0026_split_062.html}{}{}

\hypertarget{part0026_split_062.htmlux5cux23_idContainer1247}{}
\hypertarget{part0026_split_062.htmlux5cux23calibre_pb_61}{%
\subsection[Virtual
domains]{\texorpdfstring{\protect\hypertarget{part0026_split_062.htmlux5cux23_idTextAnchor1183}{}{}\protect\hypertarget{part0026_split_062.htmlux5cux23_idTextAnchor1184}{}{}Virtual
domains}{Virtual domains}}\label{part0026_split_062.htmlux5cux23calibre_pb_61}}

\protect\hypertarget{part0026_split_062.htmlux5cux23_idIndexMarker2723}{}{}To
host a mail domain on your Postfix mail server, you have three choices:

\begin{itemize}
\item
  List the domain in {mydestination}. Delivery is performed as described
  above: aliases are expanded and mail is delivered to the corresponding
  accounts.
\item
  List the domain in the {virtual\_alias\_domains} parameter. This
  option gives the domain its own addressing namespace that is
  independent of the system's user accounts. All addresses within the
  domain must be resolvable (through mapping) to real addresses outside
  of it.
\item
  List the domain in the {virtual\_mailbox\_domains} parameter. As with
  the {virtual\_alias\_domains} option, the domain has its own
  namespace. All mailboxes must live beneath a specified directory.
\end{itemize}

List the domain in only one of these three places. Choose carefully,
because many configuration elements depend on that choice. We have
already reviewed the handling of the {mydestination} method. The other
options are discussed below.

\subsubsection[Virtual alias
domains]{\texorpdfstring{\protect\hypertarget{part0026_split_062.htmlux5cux23_idTextAnchor1185}{}{}Virtual
alias domains}{Virtual alias domains}}

If a domain is listed as a value of the {virtual\_alias\_domains}
parameter, mail to that domain is accepted by Postfix and must be
forwarded to an actual recipient either on the local machine or
elsewhere.

The forwarding for addresses in the virtual domain must be defined in a
lookup table included in the {virtual\_alias\_maps} parameter. Entries
in the table have the address in the virtual domain on the left side and
the actual destination address on the right. An unqualified name on the
right is interpreted as a local username.

Consider the following example from {main.cf}:

%\includegraphics{images/00898.gif}

In {/etc/mail/admin.com/virtual} we could then have the lines

%\includegraphics{images/00899.gif}

Mail for evi@admin.com would be redirected to evi@cs.colorado.edu
({myorigin} is appended) and would ultimately be delivered to the
mailbox of user evi because cs.colorado.edu is included in
{mydestination}.

Definitions can be recursive: the right hand side can contain addresses
that are further defined on the left hand side. Note that the right hand
side can only be a list of addresses. To execute an external program or
to use {:include:} files, redirect the email to an alias, which can then
be expanded according to your needs.

To keep everything in one file, set {virtual\_alias\_domains} to the
same lookup table as {virtual\_alias\_maps} and put a special entry in
the table to mark it as a virtual alias domain. In
{main.cf}:\protect\hypertarget{part0026_split_062.htmlux5cux23_idIndexMarker2724}{}{}

%\includegraphics{images/00900.gif}

In {/etc/mail/admin.com/virtual}:

%\includegraphics{images/00901.gif}

The right hand side of the entry for the mail domain (admin.com) is
never actually used; admin.com's existence in the table as an
independent entry is enough to make Postfix consider it a virtual alias
domain.

\subsubsection[Virtual mailbox
domains]{\texorpdfstring{\protect\hypertarget{part0026_split_062.htmlux5cux23_idTextAnchor1186}{}{}Virtual
mailbox domains}{Virtual mailbox domains}}

Domains listed under {virtual\_mailbox\_domains} are similar to local
domains, but the list of users and their corresponding mailboxes must be
managed independently of the system's user accounts.

The parameter {virtual\_mailbox\_maps} points to a table that lists all
valid users in the domain. The map format is

%\includegraphics{images/00902.gif}

If the path ends with a slash, the mailboxes are stored in {Maildir}
format. The value of {virtual\_mailbox\_base} is always prefixed to the
specified paths.

You often want to alias some of the addresses in the virtual mailbox
domain. A {virtual\_alias\_map} will do that for you. Here is a complete
example. In
{main.cf}:\protect\hypertarget{part0026_split_062.htmlux5cux23_idIndexMarker2725}{}{}

%\includegraphics{images/00903.gif}

{/etc/mail/admin.com/vmailboxes} might contain entries like these:

%\includegraphics{images/00904.gif}

{/etc/mail/admin.com/valiases} might contain:

%\includegraphics{images/00905.gif}

You can use virtual alias maps even on addresses that are not within
virtual alias domains. Virtual alias maps let you redirect any address
from any domain, independently of the type of the domain (canonical,
virtual alias, or virtual mailbox). Since mailbox paths can only be put
on the right hand side of the virtual mailbox map, this mechanism is the
only way to set up aliases in that domain.

\protect\hypertarget{part0026_split_063.html}{}{}

\hypertarget{part0026_split_063.htmlux5cux23_idContainer1247}{}
\hypertarget{part0026_split_063.htmlux5cux23calibre_pb_62}{%
\subsection[Access
control]{\texorpdfstring{\protect\hypertarget{part0026_split_063.htmlux5cux23_idTextAnchor1187}{}{}Access
control}{Access control}}\label{part0026_split_063.htmlux5cux23calibre_pb_62}}

\protect\hypertarget{part0026_split_063.htmlux5cux23_idIndexMarker2726}{}{}Mail
servers should relay mail for third parties only on behalf of trusted
clients. If a mail server forwards mail from unknown clients to other
servers, it is a so-called open relay, which is bad. See
\protect\hyperlink{part0026_split_037.htmlux5cux23_idTextAnchor1093}{this
page} for more details.

Fortunately, Postfix doesn't act as an open relay by default. In fact,
its defaults are quite restrictive; you are more likely to need to
liberalize the permissions than to tighten them. Access control for SMTP
transactions is configured in Postfix through ``access restriction
lists.'' The parameters shown in
\protect\hyperlink{part0026_split_063.htmlux5cux23_idTextAnchor1188}{Table
18.21} control what should be checked during the different phases of an
SMTP session.

\paragraph[{Table 18.21: }Postfix parameters for SMTP access
restriction]{\texorpdfstring{{Table 18.21:
}\protect\hypertarget{part0026_split_063.htmlux5cux23_idTextAnchor1188}{}{}Postfix
parameters for SMTP access
restriction{\protect\hypertarget{part0026_split_063.htmlux5cux23_idIndexMarker2727}{}{}}}{Table 18.21: Postfix parameters for SMTP access restriction}}

%\includegraphics{images/00906.gif}

The most important parameter is {smtpd\_recipient\_restrictions}. That's
because access control is most easily performed when the recipient
address is known and can be identified as being local or not. All other
parameters in
\protect\hyperlink{part0026_split_063.htmlux5cux23_idTextAnchor1188}{Table
18.21} are empty in the default configuration. The default value is

%\includegraphics{images/00907.gif}

Each of the specified restrictions is tested in turn until a definitive
decision about what to do with the mail is reached.
\protect\hyperlink{part0026_split_063.htmlux5cux23_idTextAnchor1189}{Table
18.22} shows the common restrictions.

\paragraph[{Table 18.22: }Common Postfix access
restrictions]{\texorpdfstring{{Table 18.22:
}\protect\hypertarget{part0026_split_063.htmlux5cux23_idTextAnchor1189}{}{}\protect\hypertarget{part0026_split_063.htmlux5cux23_idTextAnchor1190}{}{}Common
Postfix access
restrictions{\protect\hypertarget{part0026_split_063.htmlux5cux23_idIndexMarker2728}{}{}\protect\hypertarget{part0026_split_063.htmlux5cux23_idIndexMarker2729}{}{}\protect\hypertarget{part0026_split_063.htmlux5cux23_idIndexMarker2730}{}{}}}{Table 18.22: Common Postfix access restrictions}}

%\includegraphics{images/00908.gif}

Everything can be tested in these restrictions, not just specific
information like the sender address in the
{smtpd\_sender\_restrictions}. Therefore, for simplicity, you might want
to put all the restrictions under a single parameter. Make that
{smtpd\_recipient\_restrictions }because it is the only one that can
test everything (except the DATA part).

{smtpd\_recipient\_restrictions} and {smtpd\_relay\_restrictions} are
where mail relaying is tested. Keep the {reject\_unauth\_destination}
restriction and carefully choose the ``permit'' restrictions before it.

\subsubsection[Access
tables]{\texorpdfstring{\protect\hypertarget{part0026_split_063.htmlux5cux23_idTextAnchor1191}{}{}Access
tables}{Access tables}}

Each restriction returns one of the actions shown in
\protect\hyperlink{part0026_split_063.htmlux5cux23_idTextAnchor1192}{Table
18.23}. Access tables are used in restrictions such as
{check\_client\_access} and {check\_recipient\_access} to select an
action according to the client host address or recipient address,
respectively.

\paragraph[{Table 18.23: }Actions for access
tables]{\texorpdfstring{{Table 18.23:
}\protect\hypertarget{part0026_split_063.htmlux5cux23_idTextAnchor1192}{}{}Actions
for access tables}{Table 18.23: Actions for access tables}}

%\includegraphics{images/00909.gif}

For example, suppose you wanted to allow relaying for all machines
within the {cs.colorado.edu} domain and that you wanted to allow only
trusted clients to post to the internal mailing list
{newsletter@cs.colorado.edu}. You could implement these policies with
the following lines in
{main.cf}:\protect\hypertarget{part0026_split_063.htmlux5cux23_idIndexMarker2731}{}{}

%\includegraphics{images/00910.gif}

Note that commas are optional when the list of values for a parameter is
specified.

In
{/}{\protect\hypertarget{part0026_split_063.htmlux5cux23_idIndexMarker2732}{}{}}{etc/postfix/relaying\_access}:

%\includegraphics{images/00911.gif}

In
\protect\hypertarget{part0026_split_063.htmlux5cux23_idIndexMarker2733}{}{}{/etc/postfix/restricted\_recipients}:

%\includegraphics{images/00912.gif}

The text after {REJECT} is an optional string that is sent to the client
along with the error code. It tells the sender why the mail was
rejected.

\subsubsection[Authentication of clients and
encryption]{\texorpdfstring{\protect\hypertarget{part0026_split_063.htmlux5cux23_idTextAnchor1193}{}{}Authentication
of clients and encryption}{Authentication of clients and encryption}}

\protect\hypertarget{part0026_split_063.htmlux5cux23_idIndexMarker2734}{}{}For
users sending mail from home, it is usually easiest to route outgoing
mail through the home ISP's mail server, regardless of the sender
address that appears on that mail. Most ISPs trust their direct clients
and allow relaying. If this configuration isn't possible or if you are
using a system such as Sender ID or SPF, ensure that mobile users
outside your network can be authorized to submit messages to your
{smtpd}.

The solution to this problem is to have the SMTP AUTH mechanism
authenticate directly at the SMTP level. Postfix must be compiled with
support for the SASL library to make this work. You can then configure
the feature like
this:\protect\hypertarget{part0026_split_063.htmlux5cux23_idIndexMarker2735}{}{}\protect\hypertarget{part0026_split_063.htmlux5cux23_idIndexMarker2736}{}{}

%\includegraphics{images/00913.gif}

You also need to support encrypted connections to avoid sending
passwords in clear text. {Add lines like the following to
}{main.cf}{:}\protect\hypertarget{part0026_split_063.htmlux5cux23_idIndexMarker2737}{}{}

%\includegraphics{images/00914.gif}

You need to put a properly signed certificate in {/etc/certs/smtp.pem}.
It's also a good idea to turn on encryption on outgoing SMTP
connections:

%\includegraphics{images/00915.gif}

\protect\hypertarget{part0026_split_064.html}{}{}

\hypertarget{part0026_split_064.htmlux5cux23_idContainer1247}{}
\hypertarget{part0026_split_064.htmlux5cux23calibre_pb_63}{%
\subsection[Debugging]{\texorpdfstring{\protect\hypertarget{part0026_split_064.htmlux5cux23_idTextAnchor1194}{}{}\protect\hypertarget{part0026_split_064.htmlux5cux23_idIndexMarker2738}{}{}\protect\hypertarget{part0026_split_064.htmlux5cux23_idIndexMarker2739}{}{}\protect\hypertarget{part0026_split_064.htmlux5cux23_idTextAnchor1195}{}{}Debugging}{Debugging}}\label{part0026_split_064.htmlux5cux23calibre_pb_63}}

When you have a problem with Postfix, first check the log files. The
answers to your questions are most likely there; it's just a question of
finding them. Every Postfix program normally issues a log entry for
every message it processes. For example, the trail of an outbound
message might look like this:

%\includegraphics{images/00916.gif}

As you can see, the interesting information is spread over many lines.
Note that the identifier 0E4A93688 is common to every line: Postfix
assigns a queue ID as soon as a message enters the mail system and never
changes it. Therefore, when searching the logs for the history of a
message, first concentrate on determining the message's queue ID. Once
you know that, it's easy to {grep} the logs for all the relevant
entries.

Postfix is good at logging helpful messages about problems that it
notices. However, it's sometimes difficult to spot the important lines
among the thousands of normal status messages. This is a good place to
consider using some of the tools discussed in the section
\protect\hyperlink{part0017_split_020.htmlux5cux23_idTextAnchor533}{{Management
of logs at scale}}.

\subsubsection[Looking at the
queue]{\texorpdfstring{\protect\hypertarget{part0026_split_064.htmlux5cux23_idTextAnchor1196}{}{}Looking
at the queue}{Looking at the queue}}

Another place to look for problems is the mail queue. As in the
{sendmail} system, a
\protect\hypertarget{part0026_split_064.htmlux5cux23_idIndexMarker2740}{}{}{mailq}
command prints the contents of a queue. You can use it to see if and why
a message has become stuck.

Another helpful tool is the {qshape} script that's shipped with recent
versions of Postfix. It shows summary statistics about the contents of a
queue. The output looks like this:

%\includegraphics{images/00917.gif}

{\protect\hypertarget{part0026_split_064.htmlux5cux23_idIndexMarker2741}{}{}}{qshape}
summarizes the given queue (here, the deferred queue), sorted by
recipient domain. The columns report the number of minutes the relevant
messages have been in the queue. For example, you can see that 25
messages bound for expn.com have been in the queue longer than 1,280
minutes. All the destinations in this example are suggestive of messages
having been sent from vacation scripts in response to spam.

{qshape} can also summarize by sender domain with the {-s} flag.



\subsubsection[Soft-bouncing]{\texorpdfstring{\protect\hypertarget{part0026_split_064.htmlux5cux23_idTextAnchor1197}{}{}Soft-bouncing}{Soft-bouncing}}

\protect\hypertarget{part0026_split_064.htmlux5cux23_idIndexMarker2742}{}{}If
\protect\hypertarget{part0026_split_064.htmlux5cux23_idIndexMarker2743}{}{}{soft\_bounce}
is set to {yes}, Postfix sends temporary error messages whenever it
would normally send permanent error messages such as ``user unknown'' or
``relaying denied.'' This is a great testing feature; it lets you
monitor the disposition of messages after a configuration change without
the risk of permanently losing legitimate email. Anything you reject
will eventually come back for another try. Don't forget to turn off this
feature when you are done testing or you will have to deal with every
rejected message over and over again.



\section{Recommended reading}

\subsection{Postfix references}

{Dent, Kyle D}. {Postfix: The Definitive Guide}. Sebastopol, CA: O'Reilly Media, 2003.

{Hildebrandt, Ralf, and Patrick Koetter. }{The Book of Postfix: State of
the Art Message Transport.} San Francisco, CA: No Starch Press, 2005.

This book is the best; it guides you through all the details of Postfix
configuration, even for complex environments. The authors are active in
the Postfix community and participate regularly on the postfix-users
mailing list. The book is unfortunately out of print, but used copies
are readily available.


\subsection{RFCs}

RFCs 5321 (updated by 7504) and 5322 (updated by 6854) are the current
versions of RFCs 821 and 822. They define the SMTP protocol and the
formats of messages and addresses for Internet email. RFCs 6531 and 6532
cover extensions for internationalized email addresses. There are
currently almost 90 email-related RFCs, too many to list here. See the
general RFC search engine at rfc-editor.org for more.
