\chapter{Electronic mail}
\label{chap:email}

Decades ago, cooking a chicken dinner involved not just frying the
chicken, but selecting a tender young chicken out of the coop,
terminating it with a kill signal, plucking the feathers, etc. Today,
most of us just buy a package of chicken at the grocery store or butcher
shop and skip the mess.

Email has evolved in a similar way. Ages ago, it was common for
organizations to hand-craft their email infrastructure, sometimes to the
point of predetermining exact mail routing. Today, many organizations
use packaged, cloud-hosted email services such as
\protect\hypertarget{part0026_split_000.htmlux5cux23_idIndexMarker2382}{}{}Google
Gmail or
\protect\hypertarget{part0026_split_000.htmlux5cux23_idIndexMarker2383}{}{}Microsoft
Office 365.

Even if your email system runs in the cloud, you will still have
occasion to understand, support, and interact with it as an
administrator. If your site uses local email servers, the workload
expands even further to include configuration, monitoring, and testing
chores.

If you find yourself in one of these more hands-on scenarios, this
chapter is for you. Otherwise, skip this material and spend your email
administration time responding to messages from wealthy foreigners who
need help moving millions of dollars in exchange for a large reward.
(Just kidding, of course.)



\section{Mail system architecture}

A mail system consists of several distinct components:

\begin{itemize}
\item
  A
  ``\protect\hypertarget{part0026_split_001.htmlux5cux23_idIndexMarker2385}{}{}\protect\hypertarget{part0026_split_001.htmlux5cux23_idIndexMarker2386}{}{}\protect\hypertarget{part0026_split_001.htmlux5cux23_idIndexMarker2387}{}{}mail
  user agent'' (MUA or UA) that lets users read and compose mail
\item
  A
  ``\protect\hypertarget{part0026_split_001.htmlux5cux23_idIndexMarker2388}{}{}\protect\hypertarget{part0026_split_001.htmlux5cux23_idIndexMarker2389}{}{}mail
  submission agent'' (MSA) that accepts outgoing mail from an MUA,
  grooms it, and submits it to the transport system
\item
  A
  ``\protect\hypertarget{part0026_split_001.htmlux5cux23_idIndexMarker2390}{}{}\protect\hypertarget{part0026_split_001.htmlux5cux23_idIndexMarker2391}{}{}mail
  transport agent'' (MTA) that routes messages among machines
\item
  A
  ``\protect\hypertarget{part0026_split_001.htmlux5cux23_idIndexMarker2392}{}{}\protect\hypertarget{part0026_split_001.htmlux5cux23_idIndexMarker2393}{}{}delivery
  agent'' (DA) that places messages in a local message store (the
  receiving users' mailboxes or, sometimes, a database)
\item
  An optional
  ``\protect\hypertarget{part0026_split_001.htmlux5cux23_idIndexMarker2394}{}{}\protect\hypertarget{part0026_split_001.htmlux5cux23_idIndexMarker2395}{}{}access
  agent'' (AA) that connects the user agent to the message store (e.g.,
  through the IMAP or POP protocol)
\end{itemize}

Note that these functional divisions are somewhat abstract. Real-world
mail systems break out these roles into somewhat different packages.

Attached to some of these functions are tools for recognizing spam,
viruses, and (outbound) internal company secrets.
\protect\hyperlink{part0026_split_001.htmlux5cux23_idTextAnchor1003}{Exhibit
A} illustrates how the various pieces fit together as a message winds
its way from sender to
receiver.\protect\hypertarget{part0026_split_001.htmlux5cux23_idIndexMarker2396}{}{}

\paragraph[{Exhibit A: }Mail system components]{\texorpdfstring{{Exhibit
A:
}\protect\hypertarget{part0026_split_001.htmlux5cux23_idTextAnchor1003}{}{}Mail
system components}{Exhibit A: Mail system components}}

%\includegraphics{images/00784.jpeg}

\protect\hypertarget{part0026_split_002.html}{}{}

\hypertarget{part0026_split_002.htmlux5cux23_idContainer1247}{}
\hypertarget{part0026_split_002.htmlux5cux23calibre_pb_1}{%
\subsection[User
agents]{\texorpdfstring{\protect\hypertarget{part0026_split_002.htmlux5cux23_idTextAnchor1004}{}{}\protect\hypertarget{part0026_split_002.htmlux5cux23_idIndexMarker2397}{}{}User
agents}{User agents}}\label{part0026_split_002.htmlux5cux23calibre_pb_1}}

\protect\hypertarget{part0026_split_002.htmlux5cux23_idIndexMarker2398}{}{}\protect\hypertarget{part0026_split_002.htmlux5cux23_idIndexMarker2399}{}{}\protect\hypertarget{part0026_split_002.htmlux5cux23_idIndexMarker2400}{}{}\protect\hypertarget{part0026_split_002.htmlux5cux23_idIndexMarker2401}{}{}Email
users run a user agent (sometimes called an email client) to read and
compose messages. Email messages originally consisted only of text, but
a standard known as Multipurpose Internet Mail Extensions (MIME) now
encodes text formats and attachments (including viruses) into email. It
is supported by most user agents. Since MIME generally does not affect
the addressing or transport of mail, we do not discuss it further.

\protect\hypertarget{part0026_split_002.htmlux5cux23_idIndexMarker2402}{}{}{/bin/mail}
was the original user agent, and it remains the ``good ol' standby'' for
reading text email messages at a shell prompt. Since email on the
Internet has moved far beyond the text era, text-based user agents are
no longer practical for most users. But we shouldn't throw {/bin/mail}
away; it's still a handy interface for scripts and other programs.

One of the elegant features illustrated in
\protect\hyperlink{part0026_split_001.htmlux5cux23_idTextAnchor1003}{Exhibit
A} is that a user agent doesn't necessarily need to be running on the
same system---or even on the same platform---as the rest of your mail
system. Users can reach their email from a Windows laptop or smartphone
through access agent protocols such as IMAP and POP.

\protect\hypertarget{part0026_split_003.html}{}{}

\hypertarget{part0026_split_003.htmlux5cux23_idContainer1247}{}
\hypertarget{part0026_split_003.htmlux5cux23calibre_pb_2}{%
\subsection[Submission
agents]{\texorpdfstring{\protect\hypertarget{part0026_split_003.htmlux5cux23_idTextAnchor1005}{}{}Submission
agents}{Submission agents}}\label{part0026_split_003.htmlux5cux23calibre_pb_2}}

\protect\hypertarget{part0026_split_003.htmlux5cux23_idIndexMarker2403}{}{}\protect\hypertarget{part0026_split_003.htmlux5cux23_idIndexMarker2404}{}{}\protect\hypertarget{part0026_split_003.htmlux5cux23_idIndexMarker2405}{}{}MSAs,
a late addition to the email pantheon, were invented to offload some of
the computational tasks of MTAs. MSAs make it easy for mail hub servers
to distinguish incoming from outbound email (when making decisions about
allowing relaying, for example) and give user agents a uniform and
simple configuration for outbound mail.

The MSA is a sort of ``receptionist'' for new messages being injected
into the system by local user agents. An MSA sits between the user agent
and the transport agent and takes over several functions that were
formerly a part of the MTA's job. An MSA implements secure (encrypted
and authenticated) communication with user agents and often does minor
header rewriting and cleanup on incoming messages. In many cases, the
MSA is really just the MTA listening on a different port with a
different configuration applied.

MSAs speak the same mail transfer protocol used by MTAs, so they appear
to be MTAs from the perspective of user agents. However, they typically
listen for connections on port 587 rather than port 25, the MTA
standard. For this scheme to work, user agents must connect on port 587
instead of port 25. If your user agents cannot be taught to use port
587, you can still run an MSA on port 25, but you must do so on a system
other than the one that runs your MTA; only one process at a time can
listen on a particular port.

If you use an MSA, be sure to configure your transport agent so that it
doesn't duplicate any of the rewriting or header fix-up work done by the
MSA. Duplicate processing won't affect the correctness of mail handling,
but it does represent useless extra work.

\leavevmode\hypertarget{part0026_split_003.htmlux5cux23_idContainer1103}{}%
See
\protect\hyperlink{part0026_split_012.htmlux5cux23_idTextAnchor1018}{this
page} for more information about SMTP authentication.

Since your MSA uses your MTA to relay messages, the MSA and MTA must use
SMTP-AUTH to authenticate each other. Otherwise, you create a so-called
open relay that spammers can exploit and that other sites will blacklist
you for.

\protect\hypertarget{part0026_split_004.html}{}{}

\hypertarget{part0026_split_004.htmlux5cux23_idContainer1247}{}
\hypertarget{part0026_split_004.htmlux5cux23calibre_pb_3}{%
\subsection[Transport
agents]{\texorpdfstring{\protect\hypertarget{part0026_split_004.htmlux5cux23_idTextAnchor1006}{}{}Transport
agents}{Transport agents}}\label{part0026_split_004.htmlux5cux23calibre_pb_3}}

\protect\hypertarget{part0026_split_004.htmlux5cux23_idIndexMarker2406}{}{}\protect\hypertarget{part0026_split_004.htmlux5cux23_idIndexMarker2407}{}{}\protect\hypertarget{part0026_split_004.htmlux5cux23_idIndexMarker2408}{}{}A
transport agent must accept mail from a user agent or submission agent,
understand the recipients' addresses, and somehow get the mail to the
correct hosts for delivery. Transport agents speak the Simple Mail
Transport Protocol (SMTP), which was originally defined in RFC821 but
has now been superseded and extended by RFC5321. The extended version is
called
\protect\hypertarget{part0026_split_004.htmlux5cux23_idIndexMarker2409}{}{}ESMTP.

An MTA's list of chores, as both a mail sender and receiver, includes

\begin{itemize}
\item
  Receiving email messages from remote mail servers
\item
  Understanding the recipients' addresses
\item
  Rewriting addresses to a form understood by the delivery agent
\item
  Forwarding the message to the next responsible mail server or passing
  it to a local delivery agent to be saved to a user's mailbox
\end{itemize}

The bulk of the work involved in setting up a mail system relates to the
configuration of the MTA. In this book, we cover three open source MTAs:
{sendmail}, Exim, and Postfix.

\protect\hypertarget{part0026_split_005.html}{}{}

\hypertarget{part0026_split_005.htmlux5cux23_idContainer1247}{}
\hypertarget{part0026_split_005.htmlux5cux23calibre_pb_4}{%
\subsection[Local delivery
agents]{\texorpdfstring{\protect\hypertarget{part0026_split_005.htmlux5cux23_idTextAnchor1007}{}{}Local
delivery
agents}{Local delivery agents}}\label{part0026_split_005.htmlux5cux23calibre_pb_4}}

\protect\hypertarget{part0026_split_005.htmlux5cux23_idIndexMarker2410}{}{}\protect\hypertarget{part0026_split_005.htmlux5cux23_idIndexMarker2411}{}{}\protect\hypertarget{part0026_split_005.htmlux5cux23_idIndexMarker2412}{}{}A
delivery agent, sometimes called a local delivery agent (LDA), accepts
mail from a transport agent and delivers it to the appropriate
recipients' mailboxes on the local machine. As originally specified,
email can be delivered to a person, to a mailing list, to a file, or
even to a program. However, the last two types of recipients can weaken
the security and safety of your system.

MTAs usually include a built-in local delivery agent for easy
deliveries.
\protect\hypertarget{part0026_split_005.htmlux5cux23_idIndexMarker2413}{}{}{procmail}
(procmail.org) and
\protect\hypertarget{part0026_split_005.htmlux5cux23_idIndexMarker2414}{}{}Maildrop
(\href{http://courier-mta.org/maildrop}{courier-mta.org/maildrop}) are
LDAs that can filter or sort mail before delivering it. Some access
agents (AAs) also have built-in LDAs that do both delivery and local
housekeeping chores.

\protect\hypertarget{part0026_split_006.html}{}{}

\hypertarget{part0026_split_006.htmlux5cux23_idContainer1247}{}
\hypertarget{part0026_split_006.htmlux5cux23calibre_pb_5}{%
\subsection[Message
stores]{\texorpdfstring{\protect\hypertarget{part0026_split_006.htmlux5cux23_idTextAnchor1008}{}{}Message
stores}{Message stores}}\label{part0026_split_006.htmlux5cux23calibre_pb_5}}

A message store is the final resting place of an email message once it
has completed its journey across the Internet and been delivered to
recipients.

Mail has traditionally been stored in either
\protect\hypertarget{part0026_split_006.htmlux5cux23_idIndexMarker2415}{}{}\protect\hypertarget{part0026_split_006.htmlux5cux23_idIndexMarker2416}{}{}{mbox}
format or
\protect\hypertarget{part0026_split_006.htmlux5cux23_idIndexMarker2417}{}{}\protect\hypertarget{part0026_split_006.htmlux5cux23_idIndexMarker2418}{}{}{Maildir}
format. The former stores all mail in a single file, typically
{/var/mail/}{username}, with individual messages separated by a special
From line. {Maildir} format stores each message in a separate file. A
file for each message is more convenient but creates directories with
many, many small files; some filesystems may not be amused.

Flat files in {mbox} or {Maildir} format are still widely used, but ISPs
with thousands or millions of email clients have typically migrated to
other technologies for their message stores, usually databases.
Unfortunately, that means that message stores are becoming more opaque.

\protect\hypertarget{part0026_split_007.html}{}{}

\hypertarget{part0026_split_007.htmlux5cux23_idContainer1247}{}
\hypertarget{part0026_split_007.htmlux5cux23calibre_pb_6}{%
\subsection[Access
agents]{\texorpdfstring{\protect\hypertarget{part0026_split_007.htmlux5cux23_idTextAnchor1009}{}{}Access
agents}{Access agents}}\label{part0026_split_007.htmlux5cux23calibre_pb_6}}

\protect\hypertarget{part0026_split_007.htmlux5cux23_idIndexMarker2419}{}{}\protect\hypertarget{part0026_split_007.htmlux5cux23_idIndexMarker2420}{}{}\protect\hypertarget{part0026_split_007.htmlux5cux23_idIndexMarker2421}{}{}Two
protocols access message stores and download email messages to a local
device (workstation, laptop, smartphone, etc.): Internet Message Access
Protocol {version 4} (IMAP4) and Post Office Protocol version 3 (POP3).
Earlier versions of these protocols had security issues. Be sure to use
a version (IMAPS or POP3S) that incorporates SSL encryption and hence
does not transmit passwords in cleartext over the Internet.

IMAP is significantly better than POP. It delivers your mail one message
at a time rather than all at once, which is kinder to the network
(especially on slow links) and better for someone who travels from
location to location. IMAP is especially good at dealing with the giant
attachments that some folks like to send: you can browse the headers of
your messages and not download the attachments until you are ready to
deal with them.



\section{Anatomy of a mail message}

A mail message has three distinct parts:

\begin{itemize}
\item
  Envelope
\item
  Headers
\item
  Body of the message
\end{itemize}

\protect\hypertarget{part0026_split_008.htmlux5cux23_idIndexMarker2423}{}{}The
envelope determines where the message will be delivered or, if the
message can't be delivered, to whom it should be returned. The envelope
is invisible to users and is not part of the message itself; it's used
internally by the MTA.

\protect\hypertarget{part0026_split_008.htmlux5cux23_idIndexMarker2424}{}{}\protect\hypertarget{part0026_split_008.htmlux5cux23_idIndexMarker2425}{}{}Envelope
addresses generally agree with the From and To lines of the header when
the sender and recipient are individuals. The envelope and headers might
not agree if the message was sent to a mailing list or was generated by
a spammer who is trying to conceal his identity.

Headers are a collection of property/value pairs as specified in RFC5322
(updated by RFC6854). They record all kinds of information about the
message, such as the date and time it was sent, the transport agents
through which it passed on its journey, and who it is to and from. The
headers are a bona fide part of the mail message, but user agents
typically hide the less interesting ones when displaying messages for
the user.

The body of the message is the content to be sent. It usually consists
of plain text, although that text often represents a mail-safe encoding
for various types of binary or rich-text content.

Dissecting mail headers to locate problems within the mail system is an
essential sysadmin skill. Many user agents hide the headers, but there
is usually a way to see them, even if you have to use an editor on the
message store.

Below are most of the headers (with occasional truncations indicated by
\ldots) from a typical nonspam message. We removed another half page of
headers that Gmail uses as part of its spam filtering. (In memory of
Evi, who originally owned this chapter, this historical example has been
kept intact.)

%\includegraphics{images/00785.gif}

To decode this beast, start reading the Received lines, but start from
the bottom (sender side). This message went from David Schweikert's home
machine in the schweikert.ch domain to his mail server
(mail.schweikert.ch), where it was scanned for viruses. It was then
forwarded to the recipient evi@atrust.com. However, the receiving host
mail-relay.atrust.com sent it on to sailingevi@gmail.com, where it
entered Evi's mailbox.

\leavevmode\hypertarget{part0026_split_008.htmlux5cux23_idContainer1105}{}%
See
\protect\hyperlink{part0026_split_015.htmlux5cux23_idTextAnchor1024}{this
page} for more information about SPF.

Midway through the headers, you see an
\protect\hypertarget{part0026_split_008.htmlux5cux23_idIndexMarker2426}{}{}\protect\hypertarget{part0026_split_008.htmlux5cux23_idIndexMarker2427}{}{}\protect\hypertarget{part0026_split_008.htmlux5cux23_idIndexMarker2428}{}{}SPF
(Sender Policy Framework) validation failure, an indication that the
message has been flagged as spam. This failure happened because Google
checked the IP address of mail-relay.atrust.com and compared it with the
SPF record at {schweikert.ch}; of course, it doesn't match. This is an
inherent weakness of relying on SPF records to identify forgeries---they
don't work for mail that has been relayed.

You can often see the MTAs that were used (Postfix at schweikert.ch,
{sendmail} 8.12 at atrust.com), and in this case, you can also see that
virus scanning was performed through {amavisd-new} on port 10,024 on a
machine running Debian Linux. You can follow the progress of the message
from the Central European Summer Time zone (CEST +0200), to Colorado
(-0600), and on to the Gmail server (PDT -0700); the numbers are the
differences between local time and UTC, Coordinated Universal Time. A
lot of info is stashed in the headers!

Here are the headers, again truncated, from a spam message:

%\includegraphics{images/00786.gif}

According to the From header, this message's sender is alert@atrust.com.
But according to the Return-Path header, which contains a copy of the
envelope sender, the originator was smotheringl39@sherman.dp.ua, an
address in the Ukraine. The first MTA that handled the message is at IP
address 187.10.167.249, which is in Brazil. Sneaky spammers\ldots{} It's
important to note that many of the lines in the header, including the
Received lines, may have been forged. Use this data with extreme
caution.

The SPF check at Google fails again, this time with a ``neutral'' result
because the domain sherman.dp.ua does not have an SPF record with which
to compare the IP address of mail-relay.atrust.com.

The recipient information is also at least partially untrue. The To
header says the message is addressed to ned@atrust.com. However, the
envelope recipient addresses must have included evi@atrust.com in order
for the message to be forwarded to sailingevi@gmail.com for delivery.


\section{The SMTP protocol}

The
\protect\hypertarget{part0026_split_009.htmlux5cux23_idIndexMarker2429}{}{}Simple
Mail Transport Protocol (SMTP) and its extended version,
\protect\hypertarget{part0026_split_009.htmlux5cux23_idIndexMarker2430}{}{}ESMTP,
have been standardized in the RFC series (RFC5321, updated by RFC7504)
and are used for most message hand-offs among the various pieces of the
mail system:

\begin{itemize}
\item
  UA-to-MSA or -MTA as a message is injected into the mail system
\item
  MSA-to-MTA as the message starts its delivery journey
\item
  MTA- or MSA-to-antivirus or -antispam scanning programs
\item
  MTA-to-MTA as a message is forwarded from one site to another
\item
  MTA-to-DA as a message is delivered to the local message store
\end{itemize}

Because the format of messages and the transfer protocol are both
standardized, my MTA and your MTA don't have to be the same or even know
each other's identity; they just have to both speak SMTP or ESMTP. Your
various mail servers can run different MTAs and interoperate just fine.

True to its name, SMTP is\ldots simple. An MTA connects to your mail
server and says, ``Here's a message; please deliver it to
user@your.domain.'' Your MTA says ``OK.''

Requiring strict adherence to the SMTP protocol has become a technique
for fighting spam and malware, so it's important for mail administrators
to be somewhat familiar with the protocol. The language has only a few
commands;
\protect\hyperlink{part0026_split_009.htmlux5cux23_idTextAnchor1012}{Table
18.1} shows the most important ones.

\paragraph[{Table 18.1: }SMTP commands]{\texorpdfstring{{Table 18.1:
}\protect\hypertarget{part0026_split_009.htmlux5cux23_idIndexMarker2431}{}{}\protect\hypertarget{part0026_split_009.htmlux5cux23_idTextAnchor1012}{}{}\protect\hypertarget{part0026_split_009.htmlux5cux23_idTextAnchor1013}{}{}SMTP
commands}{Table 18.1: SMTP commands}}

%\includegraphics{images/00787.gif}

\protect\hypertarget{part0026_split_010.html}{}{}

\hypertarget{part0026_split_010.htmlux5cux23_idContainer1247}{}
\hypertarget{part0026_split_010.htmlux5cux23calibre_pb_9}{%
\subsection[You had me at
EHLO]{\texorpdfstring{\protect\hypertarget{part0026_split_010.htmlux5cux23_idTextAnchor1014}{}{}You
had me at
EHLO}{You had me at EHLO}}\label{part0026_split_010.htmlux5cux23calibre_pb_9}}

ESMTP speakers start conversations with EHLO instead of HELO. If the
process at the other end understands and responds with an OK, then the
participants negotiate supported extensions and agree on a lowest common
denominator for the exchange. If the peer returns an error in response
to the EHLO, then the ESMTP speaker falls back to SMTP. But today,
almost everything uses ESMTP.

A typical SMTP conversation to deliver an email message goes as follows:
HELO or EHLO, MAIL FROM:, RCPT TO:, DATA, and QUIT. The sender does most
of the talking, with the recipient contributing error codes and
acknowledgments.

\protect\hypertarget{part0026_split_010.htmlux5cux23_idIndexMarker2432}{}{}\protect\hypertarget{part0026_split_010.htmlux5cux23_idIndexMarker2433}{}{}SMTP
and ESMTP are both text-based protocols, so you can use them directly
when debugging the mail system. Just {telnet} to TCP port 25 or 587 and
start entering SMTP commands. See the example
\protect\hyperlink{part0026_split_012.htmlux5cux23_idTextAnchor1019}{here}.

\protect\hypertarget{part0026_split_011.html}{}{}

\hypertarget{part0026_split_011.htmlux5cux23_idContainer1247}{}
\hypertarget{part0026_split_011.htmlux5cux23calibre_pb_10}{%
\subsection[SMTP error
codes]{\texorpdfstring{\protect\hypertarget{part0026_split_011.htmlux5cux23_idTextAnchor1015}{}{}SMTP
error
codes}{SMTP error codes}}\label{part0026_split_011.htmlux5cux23calibre_pb_10}}

\protect\hypertarget{part0026_split_011.htmlux5cux23_idIndexMarker2434}{}{}Also
specified in the RFCs that define SMTP are a set of temporary and
permanent error codes. These were originally three-digit codes (e.g.,
550), with each digit being interpreted separately. A first digit of 2
indicated success, a 4 signified a temporary error, and a 5 indicated a
permanent error.

The three-digit error code system did not scale, so RFC3463 (updated by
RFCs 3886, 4468, 4865, 4954, and 5248) restructured it to create more
flexibility. It defined an expanded error code format known as a
delivery status notification or DSN. DSNs have the format X.X.X instead
of the old XXX, and each of the individual Xs can be a multidigit
number. The initial X must still be 2, 4, or 5. The second digit
specifies a topic, and the third provides the details. The new system
uses the second number to distinguish host errors from mailbox errors.
\protect\hyperlink{part0026_split_011.htmlux5cux23_idTextAnchor1016}{Table
18.2} lists a few of the DSN codes. RFC3463's Appendix A shows them all.

\paragraph[{Table 18.2: }RFC3463 delivery status
notifications]{\texorpdfstring{{Table 18.2:
}\protect\hypertarget{part0026_split_011.htmlux5cux23_idIndexMarker2435}{}{}\protect\hypertarget{part0026_split_011.htmlux5cux23_idTextAnchor1016}{}{}\protect\hypertarget{part0026_split_011.htmlux5cux23_idTextAnchor1017}{}{}RFC3463
delivery status
notifications}{Table 18.2: RFC3463 delivery status notifications}}

%\includegraphics{images/00788.gif}

\protect\hypertarget{part0026_split_012.html}{}{}

\hypertarget{part0026_split_012.htmlux5cux23_idContainer1247}{}
\hypertarget{part0026_split_012.htmlux5cux23calibre_pb_11}{%
\subsection[SMTP
authentication]{\texorpdfstring{\protect\hypertarget{part0026_split_012.htmlux5cux23_idTextAnchor1018}{}{}SMTP
authentication}{SMTP authentication}}\label{part0026_split_012.htmlux5cux23calibre_pb_11}}

\protect\hypertarget{part0026_split_012.htmlux5cux23_idIndexMarker2436}{}{}RFC4954
(updated by RFC5248) defines an extension to the original SMTP protocol
that allows an SMTP client to identify and
\protect\hypertarget{part0026_split_012.htmlux5cux23_idIndexMarker2437}{}{}\protect\hypertarget{part0026_split_012.htmlux5cux23_idIndexMarker2438}{}{}\protect\hypertarget{part0026_split_012.htmlux5cux23_idIndexMarker2439}{}{}\protect\hypertarget{part0026_split_012.htmlux5cux23_idIndexMarker2440}{}{}\protect\hypertarget{part0026_split_012.htmlux5cux23_idIndexMarker2441}{}{}authenticate
itself to a mail server. The server might then let the client relay mail
through it. The protocol supports several different authentication
mechanisms. The exchange is as follows:

{1.}The client says EHLO, announcing that it speaks ESMTP.

{2.}The server responds and advertises its authentication mechanisms.

{3.}The client says AUTH and names a specific mechanism that it wants to
use, optionally including its authentication data.

{4.}The server accepts the data sent with AUTH or starts a challenge and
response sequence with the client.

{5.}The server either accepts or denies the authentication attempt.

\protect\hypertarget{part0026_split_012.htmlux5cux23_idTextAnchor1019}{}{}To
see what authentication mechanisms a server supports, you can {telnet}
to port 25 and say EHLO. For example, here is a truncated conversation
with the mail server mail-relay.atrust.com (the commands we typed are in
bold):

%\includegraphics{images/00789.gif}

In this case, the mail server supports the LOGIN and PLAIN
authentication mechanisms. {sendmail}, Exim, and Postfix all support
SMTP authentication; details of configuration are covered
\protect\hyperlink{part0026_split_038.htmlux5cux23_idTextAnchor1100}{here},
\protect\hyperlink{part0026_split_049.htmlux5cux23_idTextAnchor1144}{here},
and
\protect\hyperlink{part0026_split_063.htmlux5cux23_idTextAnchor1193}{here},
respectively.


\section{Spam and malware}

\protect\hypertarget{part0026_split_013.htmlux5cux23_idIndexMarker2442}{}{}Spam
is the jargon word for junk mail, also known as unsolicited commercial
email or UCE. It is one of the most universally hated aspects of the
Internet. Once upon a time, system administrators spent many hours each
week hand-tuning block lists and adjusting decision weights in
home-grown spam filtering tools. Unfortunately, spammers have become so
crafty and commercialized that these measures are no longer an effective
use of system administrators' time.

In this section we cover the basic antispam features of each MTA.
However, there's a certain futility to any attempt to fight spam as a
lone vigilante. You should really pay for a
\protect\hypertarget{part0026_split_013.htmlux5cux23_idIndexMarker2443}{}{}cloud-based
spam-fighting service (such as
\protect\hypertarget{part0026_split_013.htmlux5cux23_idIndexMarker2444}{}{}McAfee
SaaS Email Protection,
\protect\hypertarget{part0026_split_013.htmlux5cux23_idIndexMarker2445}{}{}Google
G Suite, or
\protect\hypertarget{part0026_split_013.htmlux5cux23_idIndexMarker2446}{}{}Barracuda)
and leave the spam fighting to the professionals who love that stuff.
They have better intelligence about the state of the global emailsphere
and can react far more quickly to new information than you can.

Spam has become a serious problem because although the absolute response
rate is low, the responses per dollar spent is high. (A list of 30
million email addresses costs about \$20.) If it didn't work for the
spammers, it wouldn't be such a problem. Surveys show that 95\%--98\% of
all mail is spam.

There are even venture-capital-funded companies whose entire mission is
to deliver spam less expensively and more efficiently (although they
typically call it ``marketing email'' rather than spam). If you work at
or buy services from one of these companies, we're not sure how you
sleep at night.

In all cases, advise your users to simply delete the spam they receive.
Many spam messages contain instructions that purport to explain how
recipients can be removed from the mailing list. If you follow those
instructions, however, the spammers may remove you from the current
list, but they immediately add you to several other lists with the
annotation ``reaches a real human who reads the message.'' Your email
address is then worth even more.

\protect\hypertarget{part0026_split_014.html}{}{}

\hypertarget{part0026_split_014.htmlux5cux23_idContainer1247}{}
\hypertarget{part0026_split_014.htmlux5cux23calibre_pb_13}{%
\subsection[Forgeries]{\texorpdfstring{\protect\hypertarget{part0026_split_014.htmlux5cux23_idTextAnchor1021}{}{}Forgeries}{Forgeries}}\label{part0026_split_014.htmlux5cux23calibre_pb_13}}

\protect\hypertarget{part0026_split_014.htmlux5cux23_idIndexMarker2447}{}{}Forging
email is trivial; many user agents let you fill in the sender's address
with anything you want. MTAs can use SMTP authentication between local
servers, but that doesn't scale to Internet sizes. Some MTAs add warning
headers to outgoing local messages that they think might be forged.

Any user can be impersonated in mail messages. Be careful if email is
your organization's authorization vehicle for things like door keys,
access cards, and money. The practice of targeting users with forged
email is commonly called ``phishing.'' You should warn administrative
users of this fact and suggest that if they see suspicious mail that
appears to come from a person in authority, they should verify the
validity of the message. Caution is doubly appropriate if the message
asks that unreasonable privileges be given to an unusual person.

\protect\hypertarget{part0026_split_015.html}{}{}

\hypertarget{part0026_split_015.htmlux5cux23_idContainer1247}{}
\hypertarget{part0026_split_015.htmlux5cux23calibre_pb_14}{%
\subsection[SPF and Sender
ID]{\texorpdfstring{\protect\hypertarget{part0026_split_015.htmlux5cux23_idTextAnchor1022}{}{}\protect\hypertarget{part0026_split_015.htmlux5cux23_idTextAnchor1023}{}{}\protect\hypertarget{part0026_split_015.htmlux5cux23_idTextAnchor1024}{}{}SPF
and Sender
ID}{SPF and Sender ID}}\label{part0026_split_015.htmlux5cux23calibre_pb_14}}

\protect\hypertarget{part0026_split_015.htmlux5cux23_idIndexMarker2448}{}{}\protect\hypertarget{part0026_split_015.htmlux5cux23_idIndexMarker2449}{}{}\protect\hypertarget{part0026_split_015.htmlux5cux23_idIndexMarker2450}{}{}\protect\hypertarget{part0026_split_015.htmlux5cux23_idIndexMarker2451}{}{}The
best way to fight spam is to stop it at its source. This sounds simple
and easy, but in reality it's almost an impossible challenge. The
structure of the Internet makes it difficult to track the real source of
a message and to verify its authenticity. The community needs a
sure-fire way to verify that the entity sending an email is actually who
or what it claims to be. Many proposals have addressed this problem, but
SPF and Sender ID have achieved the most traction.

SPF, or Sender Policy Framework, has been described by the IETF in
RFC7208. SPF defines a set of DNS records through which an organization
can identify its official outbound mail servers. MTAs can then refuse
email purporting to be from that organization's domain if the email does
not originate from one of these official sources. Of course, the system
only works well if the majority of organizations publish SPF records.

Sender ID and SPF are virtually identical in form and function. However,
key parts of Sender ID are patented by Microsoft, and hence it has been
the subject of much controversy. As of this writing (2017), Microsoft is
still trying to strong-arm the industry into adopting its proprietary
standards. The IETF chose not to choose and published RFC4406 on Sender
ID and RFC7208 on SPF. Organizations that implement this type of spam
avoidance strategy typically use SPF.

Messages that are relayed break both SPF and Sender ID, which is a
serious flaw in these systems. The receiver consults the SPF record for
the original sender to discover its list of authorized servers. However,
those addresses won't match any relay machines that were involved in
transporting the message. Be careful what decisions you make in response
to SPF failures.

\protect\hypertarget{part0026_split_016.html}{}{}

\hypertarget{part0026_split_016.htmlux5cux23_idContainer1247}{}
\hypertarget{part0026_split_016.htmlux5cux23calibre_pb_15}{%
\subsection[DKIM]{\texorpdfstring{\protect\hypertarget{part0026_split_016.htmlux5cux23_idTextAnchor1025}{}{}DKIM}{DKIM}}\label{part0026_split_016.htmlux5cux23calibre_pb_15}}

\protect\hypertarget{part0026_split_016.htmlux5cux23_idIndexMarker2452}{}{}DKIM
(DomainKeys Identified Mail) is a cryptographic signature system for
email messages. It lets the receiver verify not only the sender's
identity but also the fact that a message has not been tampered with in
transit. The system uses DNS records to publish a domain's cryptographic
keys and message-signing policy. DKIM is supported by all the MTAs
described in this chapter, but real-world deployment has been extremely
rare.

\section{Message privacy and encryption}

\protect\hypertarget{part0026_split_017.htmlux5cux23_idIndexMarker2453}{}{}\protect\hypertarget{part0026_split_017.htmlux5cux23_idIndexMarker2454}{}{}By
default, all mail is sent unencrypted. Educate your users that they
should never send sensitive data through email unless they make use of
an external encryption package or your organization has provided a
centralized encryption solution for email. Even {with} encryption,
electronic communication can never be guaranteed to be 100\% secure. You
pays your money and you takes your chances.

Historically, the most common external encryption packages have been
\protect\hypertarget{part0026_split_017.htmlux5cux23_idIndexMarker2455}{}{}\protect\hypertarget{part0026_split_017.htmlux5cux23_idIndexMarker2456}{}{}Pretty
Good Privacy (PGP), its GNUified clone
\protect\hypertarget{part0026_split_017.htmlux5cux23_idIndexMarker2457}{}{}GPG,
and
\protect\hypertarget{part0026_split_017.htmlux5cux23_idIndexMarker2458}{}{}S/MIME.
Both S/MIME and PGP are documented in the RFC series, with S/MIME being
on the standards track. Most common user agents support plug-ins for
both solutions.

These standards offer a basis for email confidentiality, authentication,
message integrity assurance, and nonrepudiation of origin.{ }But
although PGP/GPG and {S/MIME} are potentially viable solutions for
tech-savvy users who care about privacy, they have proved too cumbersome
for unsophisticated users. Both require some facility with cryptographic
key management and an understanding of the underlying encryption
strategy. (Pro tip: If you use PGP/GPG or S/MIME, you can increase your
odds of remaining secure by ensuring that your public key or certificate
is expired and replaced frequently. Long-term use of a key increases the
likelihood that it will be compromised without your awareness.)

\protect\hypertarget{part0026_split_017.htmlux5cux23_idTextAnchor1028}{}{}Most
organizations that handle sensitive data in email (especially ones that
communicate with the public, such as health care institutions) opt for a
centralized service that uses proprietary technology to encrypt
messages. Such systems can use either on-premises solutions (such as
\protect\hypertarget{part0026_split_017.htmlux5cux23_idIndexMarker2459}{}{}Cisco's
IronPort) that you deploy in your data center or cloud-based services
(such as
\protect\hypertarget{part0026_split_017.htmlux5cux23_idIndexMarker2460}{}{}Zix,
zixcorp.com) that can be configured to encrypt outbound messages
according to their contents or other rules. Centralized email encryption
is one category of service for which it's best to use a commercial
solution rather than rolling your own.

At least in the email realm,
\protect\hypertarget{part0026_split_017.htmlux5cux23_idIndexMarker2461}{}{}\protect\hypertarget{part0026_split_017.htmlux5cux23_idIndexMarker2462}{}{}\protect\hypertarget{part0026_split_017.htmlux5cux23_idIndexMarker2463}{}{}data
loss prevention (DLP) is a kissing cousin to centralized encryption. DLP
systems seek to avoid---or at least, detect---the leakage of proprietary
information into the stream of email leaving your organization. They
scan outbound email for potentially sensitive content. Suspicious
messages can be flagged, blocked, or returned to their senders. Our
recommendation is that you choose a centralized encryption platform that
also includes DLP capability; it's one less platform to manage.

\leavevmode\hypertarget{part0026_split_017.htmlux5cux23_idContainer1110}{}%
See
\protect\hyperlink{part0037_split_040.htmlux5cux23_idTextAnchor1727}{this
page} for more information about TLS.

In addition to encrypting transport between MTAs, it's important to
ensure that user-agent-to-access-agent communication is always
encrypted, especially because this channel typically employs some form
of user credentials to connect. Make sure that only the secure,
TLS-using versions of the IMAP and POP protocols are allowed by access
agents. (These are known as
\protect\hypertarget{part0026_split_017.htmlux5cux23_idIndexMarker2464}{}{}IMAPS
and
\protect\hypertarget{part0026_split_017.htmlux5cux23_idIndexMarker2465}{}{}POP3S,
respectively.)


\section{Mail aliases}

\protect\hypertarget{part0026_split_018.htmlux5cux23_idIndexMarker2467}{}{}Another
concept that is common to all MTAs is the use of aliases. Aliases allow
mail to be rerouted either by the system administrator or by individual
users.

Aliases can define mailing lists, forward mail among machines, or allow
users to be referred to by more than one name. Alias processing is
recursive, so it's legal for an alias to point to other destinations
that are themselves aliases.

Technically, aliases are configured only by sysadmins. A user's control
of mail routing through the use of a {.forward} file is not really
aliasing, but we have lumped them together here.

Sysadmins often use role or functional aliases (e.g.,
printers@example.com) to route email about a particular issue to
whatever person is currently handling that issue. Other examples might
include an alias that receives the results of a nightly security scan or
an alias for the postmaster in charge of email.

The most common method for configuring aliases is to use a simple flat
file such as the
\protect\hypertarget{part0026_split_018.htmlux5cux23_idIndexMarker2468}{}{}{/etc/mail/aliases}
file discussed later in this section. This method was originally
introduced by {sendmail}, but Exim and Postfix support it, too.

Most user agents also provide some sort of ``aliasing'' feature (usually
called ``my groups,'' ``my mailing lists,'' or something of that
nature). However, the user agent expands such aliases before mail ever
reaches an MSA or MTA. These aliases are internal to the user agent and
don't require support from the rest of the mail system.

\protect\hypertarget{part0026_split_018.htmlux5cux23_idIndexMarker2469}{}{}Aliases
can also be defined in a forwarding file in the home directory of each
user, usually
\protect\hypertarget{part0026_split_018.htmlux5cux23_idIndexMarker2470}{}{}{\textasciitilde/.forward}.
These aliases, which use a slightly nonstandard syntax, apply to all
mail delivered to that particular user. They're often used to forward
mail to a different account or to implement automatic ``I'm on
vacation'' responses.

MTAs look for aliases in the global {aliases} file ({/etc/mail/aliases}
or {/etc/aliases}) and then in recipients' forwarding files. Aliasing is
applied only to messages that the transport agent considers to be local.

The format of an entry in the {aliases} file is

%\includegraphics{images/00790.gif}

where {local-name} is the original address to be matched against
incoming messages and the recipient list contains either recipient
addresses or the names of other aliases. Indented lines are considered
continuations of the preceding lines.

From mail's point of view, the {aliases} file supersedes {/etc/passwd},
so the entry

%\includegraphics{images/00791.gif}

would prevent the local user david from ever receiving any mail.
Therefore, administrators and {adduser} tools should check both the
{passwd} file and the {aliases} file when selecting new usernames.

\protect\hypertarget{part0026_split_018.htmlux5cux23_idIndexMarker2471}{}{}\protect\hypertarget{part0026_split_018.htmlux5cux23_idIndexMarker2472}{}{}The
{aliases} file should always contain an alias named
``\protect\hypertarget{part0026_split_018.htmlux5cux23_idIndexMarker2473}{}{}postmaster''
that forwards mail to whoever maintains the mail system. Similarly, an
alias for ``abuse'' is appropriate in case someone outside your
organization needs to contact you regarding spam or suspicious network
behavior that originates at your site. An alias for automatic messages
from the MTA must also be present; it's usually called Mailer-Daemon and
is often aliased to postmaster.

Sadly, the mail system is so commonly abused these days that some sites
configure their standard contact addresses to throw mail away instead of
forwarding it to a human user. Entries such as

%\includegraphics{images/00792.gif}

are common. We don't recommend this practice, because humans who are
having trouble reaching your site by email do sometimes write to the
postmaster address.

A better paradigm might
be\protect\hypertarget{part0026_split_018.htmlux5cux23_idIndexMarker2474}{}{}

%\includegraphics{images/00793.gif}

You should redirect root's mail to your site's sysadmins or to someone
who logs in every day. The bin, sys, daemon, nobody, and hostmaster
accounts (and any other site-specific pseudo-user accounts you set up)
should all have similar aliases.

In addition to a list of users, aliases can refer to

\begin{itemize}
\item
  A file containing a list of addresses
\item
  A file to which messages should be appended
\item
  A command to which messages should be given as input
\end{itemize}

These last two targets should push your ``What about security?'' button,
because the sender of a message totally determines its content. Being
able to append that content to a file or deliver it as input to a
command sounds pretty scary. Many MTAs either disallow these alias
targets or severely limit the commands and file permissions that are
acceptable.

Aliases can cause
\protect\hypertarget{part0026_split_018.htmlux5cux23_idIndexMarker2475}{}{}mail
loops. MTAs try to detect loops that would cause mail to be forwarded
back and forth forever and return the errant messages to the sender. To
determine when mail is looping, an MTA can count the number of Received
lines in a message's header and stop forwarding it when the count
reaches a preset limit (usually 25). Each visit to a new machine is
called a ``hop'' in email jargon; returning a message to the sender is
known as ``bouncing'' it. So a more typically jargonized
\protect\hypertarget{part0026_split_018.htmlux5cux23_idIndexMarker2476}{}{}summary
of loop handling would be, ``Mail bounces after 25 hops.''

Another way MTAs can detect mail loops is by adding a Delivered-To
header for each host to which a message is forwarded. If an MTA finds
itself wanting to send a message to a host that's already mentioned in a
Delivered-To header, it knows the message has traveled in a loop.

In this chapter, we sometimes call a returned message a ``bounce'' and
sometimes call it an ``error.'' What we really mean is that a delivery
status notification (DSN, a specially formatted email message) has been
generated. Such a notification usually means that a message was
undeliverable and is therefore being returned to the sender.

\protect\hypertarget{part0026_split_019.html}{}{}

\hypertarget{part0026_split_019.htmlux5cux23_idContainer1247}{}
\hypertarget{part0026_split_019.htmlux5cux23calibre_pb_18}{%
\subsection[Getting aliases from
files]{\texorpdfstring{\protect\hypertarget{part0026_split_019.htmlux5cux23_idTextAnchor1032}{}{}Getting
aliases from
files}{Getting aliases from files}}\label{part0026_split_019.htmlux5cux23calibre_pb_18}}

The {:include:} directive in the {aliases} file (or a user's {.forward}
file) allows the list of
\protect\hypertarget{part0026_split_019.htmlux5cux23_idIndexMarker2477}{}{}targets
for the alias to be taken from the specified file. It is a great way to
let users manage their own local mailing lists. The included file can be
owned by the user and changed without involving a system administrator.
However, such an alias can also become a tasty and effective spam
expander, so don't let email from outside your site be directed there.

When setting up a list to use {:include:}, the sysadmin must enter the
alias into the global {aliases} file, create the included file, and
{chown} the included file to the user that is maintaining the mailing
list. For example, the {aliases} file might contain

%\includegraphics{images/00794.gif}

The file {ulsah.authors} should be on a local filesystem and should be
writable only by its owner. To be complete, we should also include
aliases for the mailing list's owner so that errors (bounces) are sent
to the owner of the list and not to the sender of a message addressed to
the list:

%\includegraphics{images/00795.gif}

\protect\hypertarget{part0026_split_020.html}{}{}

\hypertarget{part0026_split_020.htmlux5cux23_idContainer1247}{}
\hypertarget{part0026_split_020.htmlux5cux23calibre_pb_19}{%
\subsection[Mailing to
files]{\texorpdfstring{\protect\hypertarget{part0026_split_020.htmlux5cux23_idTextAnchor1033}{}{}Mailing
to
files}{Mailing to files}}\label{part0026_split_020.htmlux5cux23calibre_pb_19}}

\protect\hypertarget{part0026_split_020.htmlux5cux23_idIndexMarker2478}{}{}If
the target of an alias is an absolute pathname, messages are appended to
the specified file. The file must already exist. For example:

%\includegraphics{images/00796.gif}

If the pathname includes special characters, it must be enclosed in
double quotes.

It's useful to be able to send mail to files, but this feature arouses
the interest of the security police and is therefore restricted. This
syntax is only valid in the {aliases} file and in a user's {.forward}
file (or in a file that's interpolated into one of these files with the
{:include:} directive). A filename is not understood as a normal
address, so mail addressed to /etc/passwd@example.com would bounce.

If the destination file is referenced from the {aliases} file, it must
be world-writable (not advisable), setuid but not executable, or owned
by the MTA's default user. The identity of the default user is set in
the MTA's configuration file.

If the file is referenced in a {.forward} file, it must be owned and
writable by the original message recipient, who must be a valid user
with an entry in the {passwd} file and a valid shell that's listed in
{/etc/shells}. For files owned by root, use mode 4644 or 4600, setuid
but not executable.

\protect\hypertarget{part0026_split_021.html}{}{}

\hypertarget{part0026_split_021.htmlux5cux23_idContainer1247}{}
\hypertarget{part0026_split_021.htmlux5cux23calibre_pb_20}{%
\subsection[Mailing to
programs]{\texorpdfstring{\protect\hypertarget{part0026_split_021.htmlux5cux23_idTextAnchor1034}{}{}Mailing
to
programs}{Mailing to programs}}\label{part0026_split_021.htmlux5cux23calibre_pb_20}}

An alias can also route mail to the standard input of a program. This
behavior is
\protect\hypertarget{part0026_split_021.htmlux5cux23_idIndexMarker2479}{}{}specified
with a line such as

%\includegraphics{images/00797.gif}

It's even easier to create security holes with this feature than with
mailing to a file, so once again it is only permitted in {aliases},
{.forward}, or {:include:} files, and often requires the use of a
restricted shell.

\protect\hypertarget{part0026_split_022.html}{}{}

\hypertarget{part0026_split_022.htmlux5cux23_idContainer1247}{}
\hypertarget{part0026_split_022.htmlux5cux23calibre_pb_21}{%
\subsection[Building the hashed alias
database]{\texorpdfstring{\protect\hypertarget{part0026_split_022.htmlux5cux23_idTextAnchor1035}{}{}Building
the hashed alias
database}{Building the hashed alias database}}\label{part0026_split_022.htmlux5cux23calibre_pb_21}}

\protect\hypertarget{part0026_split_022.htmlux5cux23_idIndexMarker2480}{}{}Since
entries in the {aliases} file are unordered, it would be inefficient for
the MTA to search this file directly. Instead, a hashed version is
constructed with the Berkeley DB system. Hashing significantly speeds
alias lookups, especially when the file gets large.

\protect\hypertarget{part0026_split_022.htmlux5cux23_idTextAnchor1036}{}{}The
file derived
fro\protect\hypertarget{part0026_split_022.htmlux5cux23_idTextAnchor1037}{}{}m
{/etc/mail/aliases} is called {aliases.db}. If you are running Postfix
or {sendmail}, you must rebuild the hashed database with the
\protect\hypertarget{part0026_split_022.htmlux5cux23_idIndexMarker2481}{}{}{newaliases}
command every time you change the {aliases} file. Exim detects changes
to the {aliases }file automatically. Save the error output if you run
{newaliases} automatically---you might have introduced formatting errors
in the {aliases} file.



\section{Email configuration}

{\protect\hypertarget{part0026_split_023.htmlux5cux23_idIndexMarker2482}{}{}}\protect\hypertarget{part0026_split_023.htmlux5cux23_idTextAnchor1040}{}{}The
heart of an email system is its MTA, or mail transport agent.
\protect\hypertarget{part0026_split_023.htmlux5cux23_idIndexMarker2483}{}{}{sendmail}
is the original UNIX MTA, written by
\protect\hypertarget{part0026_split_023.htmlux5cux23_idIndexMarker2484}{}{}Eric
Allman while he was a graduate student many years ago. Since then, a
host of other MTAs have been developed. Some of them are commercial
products and some are open source implementations. In this chapter, we
cover three open source mail-transport agents: {sendmail},
\protect\hypertarget{part0026_split_023.htmlux5cux23_idIndexMarker2485}{}{}Postfix
by
\protect\hypertarget{part0026_split_023.htmlux5cux23_idIndexMarker2486}{}{}Wietse
Venema of
\protect\hypertarget{part0026_split_023.htmlux5cux23_idIndexMarker2487}{}{}IBM
Research, and
\protect\hypertarget{part0026_split_023.htmlux5cux23_idIndexMarker2488}{}{}Exim
by
\protect\hypertarget{part0026_split_023.htmlux5cux23_idIndexMarker2489}{}{}Philip
Hazel of the
\protect\hypertarget{part0026_split_023.htmlux5cux23_idIndexMarker2490}{}{}University
of Cambridge.

Configuration of the MTA can be a significant sysadmin chore.
Fortunately, the default or sample configurations that ship with MTAs
are often close to what the average site needs. You need not start from
scratch when configuring your MTA.

\protect\hypertarget{part0026_split_023.htmlux5cux23_idIndexMarker2491}{}{}SecuritySpace
(securityspace.com) does a survey monthly to determine the market share
of the various MTAs. In their June 2017 survey, 1.7 million out of 2
million MTAs surveyed replied with a banner that identified the MTA
software in use.
\protect\hyperlink{part0026_split_023.htmlux5cux23_idTextAnchor1041}{Table
18.3} shows these results, as well as the SecuritySpace results for 2009
and some 2001 values from a different survey.

\paragraph[{Table 18.3: }Mail transport agent market
share]{\texorpdfstring{{Table 18.3:
}\protect\hypertarget{part0026_split_023.htmlux5cux23_idTextAnchor1041}{}{}\protect\hypertarget{part0026_split_023.htmlux5cux23_idTextAnchor1042}{}{}Mail
transport agent market
share}{Table 18.3: Mail transport agent market share}}

%\includegraphics{images/00798.gif}

\protect\hypertarget{part0026_split_023.htmlux5cux23_idIndexMarker2492}{}{}The
trend is clearly away from {sendmail} and toward Exim and Postfix, with
Microsoft dropping to almost nothing. Keep in mind that this data
includes only MTAs that are directly exposed to the Internet.

For each of the MTAs we cover, we include details on the common areas of
interest:

\begin{itemize}
\item
  Configuration of simple clients
\item
  Configuration of an Internet-facing mail server
\item
  Control of both inbound and outbound mail routing
\item
  Stamping of mail as coming from a central server or the domain itself
\item
  Security
\item
  Debugging
\end{itemize}

If you are implementing a mail system from scratch and have no site
politics or biases to deal with, you may find it hard to choose an MTA.
{sendmail} is largely out of vogue, with the possible exception of pure
FreeBSD sites. Exim is powerful and highly configurable but suffers in
complexity. Postfix is simpler, faster, and was designed with security
as a primary goal. If your site or your sysadmins have a history with a
particular MTA, it's probably not worth switching unless you need
features that are not available from your old MTA.

{sendmail} configuration is covered in the next section. Exim
configuration begins
\protect\hyperlink{part0026_split_040.htmlux5cux23_idTextAnchor1125}{here},
and Postfix configuration
\protect\hyperlink{part0026_split_057.htmlux5cux23_idTextAnchor1163}{here}.


\section{sendmail}

{\protect\hypertarget{part0026_split_024.htmlux5cux23_idIndexMarker2493}{}{}\protect\hypertarget{part0026_split_024.htmlux5cux23_idIndexMarker2494}{}{}}The
{sendmail} distribution is available in source form from sendmail.org,
but it's rarely necessary to build {sendmail} from scratch these days.
If you must do so, refer to the top-level {INSTALL} file for
instructions. To tweak some of the build defaults, look up {sendmail}'s
assumptions in {devtools/OS/}{your-OS-name.} Add features by editing
{devtools/Site/site.config.m4}. As of October 2013, {sendmail} is
supported and distributed by Proofpoint, Inc., a public company.

{sendmail} uses the
\protect\hypertarget{part0026_split_024.htmlux5cux23_idIndexMarker2495}{}{}{m4}
macro preprocessor not only for compilation but also for configuration.
An {m4} configuration file is usually named {hostname}{.mc} and is then
translated from a slightly user-friendly syntax into a totally
inscrutable low-level language in the file {hostname}{.cf}, which is in
turn installed as {/etc/mail/sendmail.cf}.

To see what version of {sendmail} is installed on your system and how it
was compiled, try the following command:

%\includegraphics{images/00799.gif}

This command puts {sendmail} in address test mode ({-bt}) and debug mode
({-d0.1}) but gives it no addresses to test ({\textless/dev/null}). A
side effect is that {sendmail} tells us its version and the compiler
flags it was built with. Once you know the version number, you can look
at the sendmail.org web site to see if any known security
vulnerabilities are associated with that release.

To find the {sendmail} files on your system, look at the beginning of
the installed {/etc/mail/sendmail.cf} file. The comments there mention
the directory in which the configuration was built. That directory
should in turn lead you to the {.mc} file that is the original source of
the configuration.

\protect\hypertarget{part0026_split_024.htmlux5cux23_idIndexMarker2496}{}{}Most
vendors that ship {sendmail} include not only the binary but also the
{cf} directory from the distribution tree, which they hide somewhere
among the operating system files.
\protect\hyperlink{part0026_split_024.htmlux5cux23_idTextAnchor1045}{Table
18.4} will help you find it.

\paragraph[{Table 18.4: }Config directory
locations]{\texorpdfstring{{Table 18.4:
}\protect\hypertarget{part0026_split_024.htmlux5cux23_idTextAnchor1045}{}{}Config
directory locations}{Table 18.4: Config directory locations}}

%\includegraphics{images/00800.gif}

\protect\hypertarget{part0026_split_025.html}{}{}

\hypertarget{part0026_split_025.htmlux5cux23_idContainer1247}{}
\hypertarget{part0026_split_025.htmlux5cux23calibre_pb_24}{%
\subsection[The switch
file]{\texorpdfstring{\protect\hypertarget{part0026_split_025.htmlux5cux23_idTextAnchor1046}{}{}The
switch
file}{The switch file}}\label{part0026_split_025.htmlux5cux23calibre_pb_24}}

\leavevmode\hypertarget{part0026_split_025.htmlux5cux23_idContainer1122}{}%
The service switch is covered in more detail starting
\protect\hyperlink{part0025_split_012.htmlux5cux23_idTextAnchor990}{here}.

\protect\hypertarget{part0026_split_025.htmlux5cux23_idIndexMarker2497}{}{}Most
systems have a ``service switch'' configuration file,
\protect\hypertarget{part0026_split_025.htmlux5cux23_idIndexMarker2498}{}{}{/etc/nsswitch.conf},
that enumerates the methods that can satisfy various standard queries
such as user and host lookups. If more than one resolution method is
listed for a given type of query, the service switch file also
determines the order in which the various methods are consulted.

The existence of the service switch is normally transparent to software.
However, {sendmail} likes to exert fine-grained control over its
lookups, so it currently ignores the system switch file and instead uses
its own internal service configuration file
({/etc/mail/service.switch}).

Two fields in the switch file impact the mail system: {aliases} and
{hosts}. The possible values for the hosts service are {dns}, {nis},
{nisplus}, and {files}. For aliases, the possible values are {files},
{nis}, {nisplus}, and {ldap}. Support for the mechanisms you use (except
{files}) must be compiled into {sendmail} before the service can be
used.

\protect\hypertarget{part0026_split_026.html}{}{}

\hypertarget{part0026_split_026.htmlux5cux23_idContainer1247}{}
\hypertarget{part0026_split_026.htmlux5cux23calibre_pb_25}{%
\subsection[Starting
{sendmail}]{\texorpdfstring{\protect\hypertarget{part0026_split_026.htmlux5cux23_idTextAnchor1047}{}{}Starting
{sendmail}}{Starting sendmail}}\label{part0026_split_026.htmlux5cux23calibre_pb_25}}

\protect\hypertarget{part0026_split_026.htmlux5cux23_idIndexMarker2499}{}{}{sendmail}
should not be controlled by {inetd} or {systemd}, so it must be
explicitly started at boot time. See
\protect\hyperlink{part0009_split_000.htmlux5cux23_idTextAnchor065}{Chapter
2, {Booting and System Management Daemons}}, for startup details.

The flags that {sendmail} is started with determine its behavior. You
can run it in several different modes, selected with the {-b} flag. {-b}
stands for ``be'' or ``become'' and is always used with another flag
that determines the role {sendmail} will play.
\protect\hyperlink{part0026_split_026.htmlux5cux23_idTextAnchor1048}{Table
18.5} lists the legal values and also includes the {-A} flag, which
selects between MTA and MSA behavior.

\paragraph[{Table 18.5: }Command-line flags for {sendmail}'s major
modes]{\texorpdfstring{{Table 18.5:
}\protect\hypertarget{part0026_split_026.htmlux5cux23_idIndexMarker2500}{}{}\protect\hypertarget{part0026_split_026.htmlux5cux23_idTextAnchor1048}{}{}\protect\hypertarget{part0026_split_026.htmlux5cux23_idTextAnchor1049}{}{}Command-line
flags for {sendmail}'s major
modes}{Table 18.5: Command-line flags for sendmail's major modes}}

%\includegraphics{images/00801.gif}

If you are configuring a server that will accept incoming mail from the
Internet, run
\protect\hypertarget{part0026_split_026.htmlux5cux23_idIndexMarker2501}{}{}{sendmail}
in daemon mode ({-bd}). In this mode, {sendmail} listens on network port
25 and waits for work. (The ports that {sendmail} listens on are
determined by {DAEMON\_OPTIONS}; port 25 is the default.)

You will usually specify the {-q} flag, too---it sets the interval
\protect\hypertarget{part0026_split_026.htmlux5cux23_idIndexMarker2502}{}{}at
which {sendmail} processes the mail queue. For example, {-q30m} runs the
queue every thirty minutes and {-q1h} runs it every hour.

{sendmail} normally tries to deliver messages immediately, saving them
in the queue only momentarily to guarantee reliability. But if your host
is too busy or the destination machine is unreachable, {sendmail} queues
messages and tries to send them again later. {sendmail} uses persistent
queue runners that are usually started at boot time. It does locking, so
multiple, simultaneous queue runs are safe. You can use the ``queue
groups'' configuration feature to facilitate delivery of large mailing
lists and queues.

{sendmail} reads its configuration file,
\protect\hypertarget{part0026_split_026.htmlux5cux23_idIndexMarker2503}{}{}{sendmail.cf},
only when it starts up. Therefore, you must either kill and restart
{sendmail} or send it a HUP signal when you change the config file.
{sendmail} creates a {sendmail.pid} file that contains its process ID
and the command that started it. You should start {sendmail} with an
absolute path because it re-{exec}s itself on receipt of the HUP signal.
The {sendmail.pid} file allows the process to be HUPed with the command

%\includegraphics{images/00802.gif}

The location of the PID file is OS dependent. It's usually
{/var/run/sendmail.pid} or {/etc/mail/sendmail.pid} but can be set in
the config file with the {confPID\_FILE} option:

%\includegraphics{images/00803.gif}

\protect\hypertarget{part0026_split_027.html}{}{}

\hypertarget{part0026_split_027.htmlux5cux23_idContainer1247}{}
\hypertarget{part0026_split_027.htmlux5cux23calibre_pb_26}{%
\subsection[Mail
queues]{\texorpdfstring{\protect\hypertarget{part0026_split_027.htmlux5cux23_idTextAnchor1050}{}{}Mail
queues}{Mail queues}}\label{part0026_split_027.htmlux5cux23calibre_pb_26}}

\protect\hypertarget{part0026_split_027.htmlux5cux23_idIndexMarker2504}{}{}{sendmail}
uses at least two queues:
\protect\hypertarget{part0026_split_027.htmlux5cux23_idIndexMarker2505}{}{}{/var/spool/mqueue}
when acting as an MTA on port 25, and
\protect\hypertarget{part0026_split_027.htmlux5cux23_idIndexMarker2506}{}{}{/var/spool/clientmqueue}
when acting as an MSA on port 587. {sendmail} can use multiple queues
beneath {mqueue} to increase performance. All messages make at least a
brief stop in the queue before being sent on their way.

A queued message is saved in pieces in several different files.
\protect\hyperlink{part0026_split_027.htmlux5cux23_idTextAnchor1051}{Table
18.6} shows the six possible pieces. Each filename has a two-letter
prefix that identifies the piece, followed by a random ID built from
{sendmail}'s process ID.

\paragraph[{Table 18.6: }Prefixes for files in the mail
queue]{\texorpdfstring{{Table 18.6:
}\protect\hypertarget{part0026_split_027.htmlux5cux23_idTextAnchor1051}{}{}\protect\hypertarget{part0026_split_027.htmlux5cux23_idTextAnchor1052}{}{}Prefixes
for files in the mail
queue}{Table 18.6: Prefixes for files in the mail queue}}

%\includegraphics{images/00804.gif}

If subdirectories {qf}, {df}, or {xf} exist in a queue directory, then
those pieces of the message are put in the proper subdirectory. The {qf}
file contains not only the message header but also the envelope
addresses, the date at which the message should be returned as
undeliverable, the message's priority in the queue, and the reason the
message is in the queue. Each line begins with a single-letter code that
identifies the rest of the line.

Each message that is queued must have a {qf} and {df} file. All the
other prefixes are used by {sendmail} during attempted delivery. When a
machine crashes and reboots, the startup sequence for {sendmail} should
delete the {tf}, {xf}, and {Tf} files from each queue. If you are the
sysadmin responsible for mail, check occasionally for {Qf} files in case
local configuration is causing the bounces. An occasional glance at the
queue directories lets you spot problems before they become disasters.

The mail queue opens up several opportunities for things to go wrong.
For example, the filesystem can fill up (avoid putting
{/var/spool/mqueue} and {/var/log} on the same partition), the queue can
become clogged, or orphaned mail messages can get stuck in the queue.
{sendmail} has configuration options to help with performance on busy
machines.

\protect\hypertarget{part0026_split_028.html}{}{}

\hypertarget{part0026_split_028.htmlux5cux23_idContainer1247}{}
\hypertarget{part0026_split_028.htmlux5cux23calibre_pb_27}{%
\subsection[
configuration]{\texorpdfstring{{\protect\hypertarget{part0026_split_028.htmlux5cux23_idTextAnchor1053}{}{}sendmail}
configuration}{sendmail configuration}}\label{part0026_split_028.htmlux5cux23calibre_pb_27}}

\protect\hypertarget{part0026_split_028.htmlux5cux23_idIndexMarker2507}{}{}{sendmail}
is controlled by a single configuration file, typically called
{/etc/mail/}{\protect\hypertarget{part0026_split_028.htmlux5cux23_idIndexMarker2508}{}{}}{sendmail.cf}
for a {sendmail} running as an MTA or
{/etc/mail/}{\protect\hypertarget{part0026_split_028.htmlux5cux23_idIndexMarker2509}{}{}}{submit.cf}
for a {sendmail} acting as an MSA. The flags with which {sendmail} is
started determine which config file it uses: {-bm}, {-bs}, and {-bt} use
{submit.cf }if it exists, and all other modes use {sendmail.cf}. You can
change these names with command-line flags or config file options, but
it is best not to.

The raw config file format was designed to be easy to parse by machines,
not humans. The {m4} source ({.mc}) file from which the{ .cf} file is
generated is an improvement, but its picky and rigid syntax isn't going
to win any awards for user friendliness either. Fortunately, many of the
paradigms you might want to set up have already been hammered out by
others with similar needs and are supplied in the distribution as
prepackaged features.

{sendmail} configuration involves several steps:

{1.}Determine the role of the machine you are configuring: client,
server, {Internet}-facing mail receiver, etc.

{2.}Choose the features needed to implement that role and build an {.mc}
file for the configuration

{3.}Compile the {.mc} file with {m4} to produce a {.cf} config file

We cover the features commonly used for site-wide, Internet-facing
servers and for little desktop clients. For more detailed coverage, we
refer you to two key pieces of documentation on the care and feeding of
{sendmail}: the O'Reilly book {sendmail} by Bryan Costales et al. and
the file {cf/README} from the distribution.

\protect\hypertarget{part0026_split_029.html}{}{}

\hypertarget{part0026_split_029.htmlux5cux23_idContainer1247}{}
\hypertarget{part0026_split_029.htmlux5cux23calibre_pb_28}{%
\subsection[The {m4}
preprocessor]{\texorpdfstring{Th\protect\hypertarget{part0026_split_029.htmlux5cux23_idTextAnchor1054}{}{}e
{m4}
preprocessor}{The m4 preprocessor}}\label{part0026_split_029.htmlux5cux23calibre_pb_28}}

\protect\hypertarget{part0026_split_029.htmlux5cux23_idIndexMarker2510}{}{}{m4},
originally intended as a front end for programming languages, lets users
write more readable (or perhaps more cryptic) programs. {m4} is powerful
enough to be useful in many input transformation situations, and it
works nicely for {sendmail} configuration files.

{m4} macros have the form

%\includegraphics{images/00805.gif}

There cannot be any space between the name and the opening parenthesis.
Left and right single quotes (that is, backticks and ``normal'' single
quotes) designate strings as arguments. {m4}'s quote conventions are
weird, since the left and right quotes are different characters. Quotes
nest, too.

{m4} has some built-in macros, and users can also define their own.
\protect\hyperlink{part0026_split_029.htmlux5cux23_idTextAnchor1055}{Table
18.7} lists the most common built-in macros that are used in {sendmail}
configuration.

\paragraph[{Table 18.7: } macros commonly used with
{sendmail}]{\texorpdfstring{{Table 18.7:
}{\protect\hypertarget{part0026_split_029.htmlux5cux23_idTextAnchor1055}{}{}\protect\hypertarget{part0026_split_029.htmlux5cux23_idTextAnchor1056}{}{}m4}
macros commonly used with
{sendmail}}{Table 18.7: m4 macros commonly used with sendmail}}

%\includegraphics{images/00806.gif}

\protect\hypertarget{part0026_split_030.html}{}{}

\hypertarget{part0026_split_030.htmlux5cux23_idContainer1247}{}
\hypertarget{part0026_split_030.htmlux5cux23calibre_pb_29}{%
\subsection[The {sendmail} configuration
pieces]{\texorpdfstring{\protect\hypertarget{part0026_split_030.htmlux5cux23_idTextAnchor1057}{}{}The
{sendmail} configuration
pieces}{The sendmail configuration pieces}}\label{part0026_split_030.htmlux5cux23calibre_pb_29}}

The {sendmail} distribution includes a {cf} subdirectory beneath which
are all the pieces necessary for {m4} configuration.
\protect\hyperlink{part0026_split_024.htmlux5cux23_idTextAnchor1045}{Table
18.4} shows the location of the {cf} directory if you did not install
the {sendmail} source but relied on your vendor. The {README} file found
in the {cf} directory is {sendmail}'s configuration documentation. The
subdirectories, listed in
\protect\hyperlink{part0026_split_030.htmlux5cux23_idTextAnchor1058}{Table
18.8}, contain examples and snippets you can include in your own
configuration.

\paragraph[{Table 18.8: } configuration
subdirectories]{\texorpdfstring{{Table 18.8:
}{\protect\hypertarget{part0026_split_030.htmlux5cux23_idTextAnchor1058}{}{}\protect\hypertarget{part0026_split_030.htmlux5cux23_idTextAnchor1059}{}{}sendmail}
configuration
subdirectories}{Table 18.8: sendmail configuration subdirectories}}

%\includegraphics{images/00807.gif}

The {cf/cf} directory contains examples of {.mc} files. In fact, it
contains so many examples that yours may get lost in the clutter. We
recommend that you keep your own {.mc} files separate from those in the
distributed {cf} directory. Either create a new directory named for your
site ({cf/}{sitename}) or move the {cf} directory aside to {cf.examples}
and create a new {cf} directory. If you do this, copy the {Makefile} and
{Build} script over to your new directory so the instructions in the
{README} file still work. Alternatively, you can copy all your own
configuration {.mc} files to a central location rather than leaving them
inside the {sendmail} distribution. The {Build} script uses relative
pathnames, so you'll have to modify it if you want to build a {.cf} file
from an {.mc} file and are not in the {sendmail} distribution hierarchy.

The files in the {cf/ostype} directory configure {sendmail} for each
specific operating system. Many are predefined, but if you have moved
things around on your system, you might have to modify one or create a
new one. Copy one that is close to reality for your system and give it a
new name.

The {cf/feature} directory is where you shop for any configuration
pieces you might need. There is a feature for just about anything that
any site running {sendmail} has found useful.

The other directories beneath {cf} are pretty much boilerplate and do
not need to be tweaked or even understood---just use them.

\protect\hypertarget{part0026_split_031.html}{}{}

\hypertarget{part0026_split_031.htmlux5cux23_idContainer1247}{}
\hypertarget{part0026_split_031.htmlux5cux23calibre_pb_30}{%
\subsection[A configuration file built from a sample {.mc}
file]{\texorpdfstring{\protect\hypertarget{part0026_split_031.htmlux5cux23_idTextAnchor1060}{}{}A
configuration file built from a sample {.mc}
file}{A configuration file built from a sample .mc file}}\label{part0026_split_031.htmlux5cux23calibre_pb_30}}

\protect\hypertarget{part0026_split_031.htmlux5cux23_idIndexMarker2511}{}{}Before
we take off into the wilds of the various configuration macros,
features, and options you might use in a {sendmail} configuration, we
shall put the cart before the horse and devise a ``no frills''
configuration to illustrate the general process. Our example is for a
leaf node, myhost.example.com; the master configuration file is called
{myhost.mc}. Here's the complete {.mc} file:

%\includegraphics{images/00808.gif}

Except for the diversions and comments, each line invokes a prepackaged
macro. The first four lines are boilerplate; they insert comments in the
compiled file to note the version of {sendmail}, the directory the
configuration was built in, etc. The {OSTYPE} macro includes the
{../ostype/linux.m4} file. The {MAILER} lines allow for local delivery
(to users with accounts on myhost.example.com) and for delivery to
Internet sites.

To build the real configuration file, just run the {Build} command you
copied over to the new {cf} directory:

%\includegraphics{images/00809.gif}

Finally, install {myhost.cf} in the right spot---normally
{/etc/mail/sendmail.cf}, but some vendors move it. Favorite vendor
hiding places are {/etc} and {/usr/lib}.

At a larger site, you might want to create a separate {m4} file to hold
site-wide defaults; put it in the {cf/domain} directory. Individual
hosts can then include the contents of this file with the {DOMAIN}
macro. Not every host needs a separate config file, but each group of
similar hosts (same architecture and same role: server, client, etc.)
will probably need its own configuration.

The order of the macros in the {.mc} file is not arbitrary. It should be

%\includegraphics{images/00810.gif}

Even with {sendmail}'s easy {m4} configuration system, you still have to
make several configuration decisions for your site. As you read about
the features described below, think about how they might fit into your
site's organization. A small site will probably have only a hub node and
leaf nodes and thus will need only two versions of the config file. A
larger site might need separate hubs for incoming and outgoing mail and,
perhaps, a separate POP/IMAP server.

Whatever the complexity of your site and whatever face it shows to the
outside world (exposed, behind a firewall, or on a virtual private
network, for example), it's likely that the {cf} directory contains some
appropriate ready-made configuration snippets just waiting to be
customized and put to work.

\protect\hypertarget{part0026_split_032.html}{}{}

\hypertarget{part0026_split_032.htmlux5cux23_idContainer1247}{}
\hypertarget{part0026_split_032.htmlux5cux23calibre_pb_31}{%
\subsection[Configuration
primitives]{\texorpdfstring{\protect\hypertarget{part0026_split_032.htmlux5cux23_idTextAnchor1061}{}{}Configuration
primitives}{Configuration primitives}}\label{part0026_split_032.htmlux5cux23calibre_pb_31}}

{sendmail} configuration commands are case sensitive. By convention, the
names of predefined macros are all caps (e.g., {OSTYPE}), {m4} commands
are all lower case (e.g., {define}), and configurable option names
usually start with lowercase {conf} and end with an all-caps variable
name (e.g., {confFAST\_SPLIT}). Macros usually refer to an {m4} file
called {../}{macroname}{/}{arg1}{.m4}. For example, the reference
{OSTYPE(`linux')} causes the file {../ostype/linux.m4} to be included.

\protect\hypertarget{part0026_split_033.html}{}{}

\hypertarget{part0026_split_033.htmlux5cux23_idContainer1247}{}
\hypertarget{part0026_split_033.htmlux5cux23calibre_pb_32}{%
\subsection[Tables and
databases]{\texorpdfstring{\protect\hypertarget{part0026_split_033.htmlux5cux23_idTextAnchor1062}{}{}Tables
and
databases}{Tables and databases}}\label{part0026_split_033.htmlux5cux23calibre_pb_32}}

\protect\hypertarget{part0026_split_033.htmlux5cux23_idIndexMarker2512}{}{}Before
we plunge into specific configuration primitives, we must first discuss
tables (sometimes called maps or databases), which {sendmail} can use to
perform mail routing or address rewriting. Most are used in conjunction
with the {FEATURE} macro.

A table is a cache (usually a text file) of routing, aliasing, policy,
or other information that is converted to a database format with the
{makemap} command and then used as an information source for one or more
of {sendmail}'s various lookup operations. Although the data usually
starts as a text file, data for {sendmail} tables can come from DNS,
LDAP, or other sources. The use of a centralized IMAP server relieves
{sendmail} of the chore of chasing down users and obsoletes some of its
tables.

{sendmail} defines three
da\protect\hypertarget{part0026_split_033.htmlux5cux23_idTextAnchor1063}{}{}tabase
map types:

\begin{itemize}
\item
  {dbm} -- legacy; uses an extensible hashing algorithm ({dbm}/{ndbm})
\item
  {hash} -- uses a standard hashing scheme (DB)
\item
  {btree} -- uses a B-tree data structure (DB)
\end{itemize}

For most table applications in {sendmail}, the {hash} database
type---the default---is the best. Use the
\protect\hypertarget{part0026_split_033.htmlux5cux23_idIndexMarker2513}{}{}{makemap}
command to build the database file from a text file; you specify the
database type and the output file base name. The text version of the
database should appear on {makemap}'s standard input. For example:

%\includegraphics{images/00811.gif}

At first glance this command looks like a mistake that would cause the
input file to be overwritten by an empty output file. However, {makemap}
tacks on an appropriate suffix, so the actual output file is
{/etc/mail/access.db} and in fact no conflict occurs. Each time the text
file is changed, the database file must be rebuilt with {makemap} (but
{sendmail} need not be HUP'd).

Comments can appear in the text files from which maps are produced. They
begin with {\#} and continue until the end of the line.

In most circumstances, the longest possible match is used for database
keys. As with any hashed data structure, the order of entries in the
input text file is not significant. Some {FEATURE}s expect a database
file as a parameter; they default to {hash} as the database type and
{/etc/mail/}{tablename}{.db} as the filename for the database.

\protect\hypertarget{part0026_split_034.html}{}{}

\hypertarget{part0026_split_034.htmlux5cux23_idContainer1247}{}
\hypertarget{part0026_split_034.htmlux5cux23calibre_pb_33}{%
\subsection[Generic macros and
features]{\texorpdfstring{\protect\hypertarget{part0026_split_034.htmlux5cux23_idTextAnchor1064}{}{}Generic
macros and
features}{Generic macros and features}}\label{part0026_split_034.htmlux5cux23calibre_pb_33}}

\protect\hyperlink{part0026_split_034.htmlux5cux23_idTextAnchor1065}{Table
18.9} lists common configuration primitives, whether they are typically
used (yes, no, maybe), and a brief description of what they do.

\paragraph[{Table 18.9: } generic configuration
primitives]{\texorpdfstring{{Table 18.9:
}{\protect\hypertarget{part0026_split_034.htmlux5cux23_idIndexMarker2514}{}{}}{\protect\hypertarget{part0026_split_034.htmlux5cux23_idTextAnchor1065}{}{}\protect\hypertarget{part0026_split_034.htmlux5cux23_idTextAnchor1066}{}{}sendmail}
generic configuration
primitives}{Table 18.9: sendmail generic configuration primitives}}

%\includegraphics{images/00812.gif}

\subsubsection[{OSTYPE}
macro]{\texorpdfstring{\protect\hypertarget{part0026_split_034.htmlux5cux23_idTextAnchor1067}{}{}\protect\hypertarget{part0026_split_034.htmlux5cux23_idIndexMarker2515}{}{}{OSTYPE}
macro}{OSTYPE macro}}

An {OSTYPE} file packages a variety of vendor-specific information, such
as the expected locations of mail-related files, paths to commands that
{sendmail} needs, flags to mailer programs, etc. See {cf/README} for a
list of all the variables that can be defined in an {OSTYPE} file.

So where is the {OSTYPE} macro itself defined? In a file in the {cf/m4}
directory, which is magically prepended to your config file when you run
the {Build} script.

\subsubsection[{DOMAIN}
macro]{\texorpdfstring{\protect\hypertarget{part0026_split_034.htmlux5cux23_idTextAnchor1068}{}{}\protect\hypertarget{part0026_split_034.htmlux5cux23_idIndexMarker2516}{}{}{DOMAIN}
macro}{DOMAIN macro}}

The {DOMAIN} directive lets you specify site-wide generic information in
one place ({cf/domain/}{filename}{.m4}) and then include it in each
host's config file with

%\includegraphics{images/00813.gif}

\subsubsection[
macro]{\texorpdfstring{{MAI\protect\hypertarget{part0026_split_034.htmlux5cux23_idTextAnchor1069}{}{}LER}
macro}{MAILER macro}}

You must include a {MAILER} macro for every delivery agent you want to
enable. You'll find a complete list of supported mailers in the
directory {cf/mailers}, but typically you need only {local }and{ smtp}.
{MAILER} lines are generally the last thing in the {.mc} file.

\subsubsection[
macro]{\texorpdfstring{\protect\hypertarget{part0026_split_034.htmlux5cux23_idTextAnchor1070}{}{}\protect\hypertarget{part0026_split_034.htmlux5cux23_idIndexMarker2517}{}{}{FEAT\protect\hypertarget{part0026_split_034.htmlux5cux23_idTextAnchor1071}{}{}URE}
macro}{FEATURE macro}}

The {FEATURE} macro enables a whole host of common scenarios (56 at last
count!) by including {m4} files from the {feature} directory. The syntax
is

%\includegraphics{images/00814.gif}

where {keyword} corresponds to a file {keyword}{.m4} in the {cf/feature}
directory and the {args} are passed to it. There can be at most nine
arguments to a feature.

\subsubsection[
feature]{\texorpdfstring{\protect\hypertarget{part0026_split_034.htmlux5cux23_idTextAnchor1072}{}{}\protect\hypertarget{part0026_split_034.htmlux5cux23_idIndexMarker2518}{}{}{\protect\hypertarget{part0026_split_034.htmlux5cux23_idTextAnchor1073}{}{}use\_\protect\hypertarget{part0026_split_034.htmlux5cux23_idTextAnchor1074}{}{}cw\_file}
feature}{use\_cw\_file feature}}

The {sendmail} internal class {w} (hence the name {cw}) contains the
names of all local hosts for which this host accepts and delivers mail.
This feature specifies that mail be accepted for the hosts listed, one
per line, in
\protect\hypertarget{part0026_split_034.htmlux5cux23_idIndexMarker2519}{}{}{/etc/mail/local-host-names}.
The configuration line

%\includegraphics{images/00815.gif}

invokes the feature. A client machine does not really need this feature
unless it has nicknames, but your incoming mail hub machine does. The
{local-host-names} file should include any local hosts and virtual
domains for which you accept email, including sites whose backup
\protect\hypertarget{part0026_split_034.htmlux5cux23_idIndexMarker2520}{}{}MX
records (see
\protect\hyperlink{part0024_split_027.htmlux5cux23_idTextAnchor882}{this
page}) point to you.

Without this feature, {sendmail} delivers mail locally only if it is
addressed to the machine on which {sendmail} is running.

If you add a new host at your site, you must add it to the
{local-host-names} file and send a HUP signal to {sendmail} to make your
changes take effect.

\subsubsection[{redirect}
feature]{\texorpdfstring{\protect\hypertarget{part0026_split_034.htmlux5cux23_idTextAnchor1075}{}{}\protect\hypertarget{part0026_split_034.htmlux5cux23_idIndexMarker2521}{}{}{redirect}
feature}{redirect feature}}

When people leave your organization, you usually either forward their
mail or let mail to them bounce back to the sender with an error. The
{redirect} feature provides support for a more elegant way of bouncing
mail.

If Joe Smith has graduated from oldsite.edu (login smithj) to
newsite.com (login joe), then enabling {redirect} with

%\includegraphics{images/00816.gif}

and adding the line

%\includegraphics{images/00817.gif}

to the {aliases} file at oldsite.edu causes mail to smithj to be
returned to the sender with an error message suggesting that the sender
try the address joe@newsite.com instead. The message itself is not
automatically forwarded.

\subsubsection[{always\_add\_domain}
feature]{\texorpdfstring{\protect\hypertarget{part0026_split_034.htmlux5cux23_idTextAnchor1076}{}{}\protect\hypertarget{part0026_split_034.htmlux5cux23_idIndexMarker2522}{}{}{always\_add\_domain}
feature}{always\_add\_domain feature}}

The {always\_add\_domain} feature makes all email addresses fully
qualified. It should always be used.

\subsubsection[
feature]{\texorpdfstring{\protect\hypertarget{part0026_split_034.htmlux5cux23_idTextAnchor1077}{}{}\protect\hypertarget{part0026_split_034.htmlux5cux23_idIndexMarker2523}{}{}{\protect\hypertarget{part0026_split_034.htmlux5cux23_idTextAnchor1078}{}{}access\_db}
feature}{access\_db feature}}

The {access\_db} feature controls relaying and other policy issues.
Typically, the raw data that drives this feature either comes from LDAP
or is kept in a text file called
\protect\hypertarget{part0026_split_034.htmlux5cux23_idIndexMarker2524}{}{}{/etc/mail/access}.
In the latter case, the text file must be converted to some kind of
indexed format with the {makemap} command, as described
\protect\hyperlink{part0026_split_033.htmlux5cux23_idTextAnchor1062}{here}.
To use the flat file, use {FEATURE(`access\_db')} in the configuration
file; for the LDAP version, use {FEATURE(`access\_db', `LDAP'). }The
LDAP form uses the default schema defined in the file
{cf/sendmail.schema}; if you want a different schema file, use
additional arguments in your {FEATURE} statement.

The key field in the access database is an IP network or a domain name
with an optional tag such as {Connect:}, {To:,} or {From:}. The value
field specifies what to do with the message.

The most common values are {OK} to accept the message, {RELAY} to allow
it to be relayed, {REJECT} to reject it with a generic error indication,
or {ERROR:"}{error code and message}{"} to reject it with a specific
message. Other possible values allow for finer-grained control. Here is
a snippet from a sample {/etc/mail/access} file:

%\includegraphics{images/00818.gif}

\subsubsection[{virtusertable}
feature]{\texorpdfstring{\protect\hypertarget{part0026_split_034.htmlux5cux23_idTextAnchor1079}{}{}\protect\hypertarget{part0026_split_034.htmlux5cux23_idIndexMarker2525}{}{}{virtusertable}
feature}{virtusertable feature}}

\protect\hypertarget{part0026_split_034.htmlux5cux23_idIndexMarker2526}{}{}The
{virtusertable} feature supports domain aliasing for incoming mail
through a map stored in {/etc/mail/virtusertable}. This feature lets one
machine host multiple virtual domains and is used frequently at
web-hosting sites. The key field of the table contains either an email
address ({user@host.domain}) or a domain specification ({@domain}). The
value field is a local or external email address. If the key is a
domain, the value can either pass the {user} field along as the variable
{\%1} or route the mail to a different user. Here are some examples:

%\includegraphics{images/00819.gif}

All the host keys on the left side of the data mappings must be listed
in the {cw} file, {/etc/mail/local-host-names}, or be included in the
{VIRTUSER\_DOMAIN} list. If they are not, {sendmail} will not know to
accept the mail locally and will try to find the destination host on the
Internet. But DNS MX records will point {sendmail} back to this same
server and you will get a ``local configuration error'' message in the
resulting bounce message. Unfortunately, {sendmail} cannot tell that the
error message for this instance should in fact be ``virtusertable key
not in cw file.''

\subsubsection[
feature]{\texorpdfstring{\protect\hypertarget{part0026_split_034.htmlux5cux23_idTextAnchor1080}{}{}\protect\hypertarget{part0026_split_034.htmlux5cux23_idIndexMarker2527}{}{}\protect\hypertarget{part0026_split_034.htmlux5cux23_idIndexMarker2528}{}{}{ldap\_rou\protect\hypertarget{part0026_split_034.htmlux5cux23_idTextAnchor1081}{}{}ting}
feature}{ldap\_routing feature}}

\protect\hypertarget{part0026_split_034.htmlux5cux23_idIndexMarker2529}{}{}LDAP,
the Lightweight Directory Access Protocol, can be a source of data for
aliases or mail routing information as well as general tabular data as
described earlier. The {cf/README} file has a long section on LDAP with
lots of examples.

\leavevmode\hypertarget{part0026_split_034.htmlux5cux23_idContainer1142}{}%
See
\protect\hyperlink{part0025_split_002.htmlux5cux23_idTextAnchor974}{this
page} for general information about LDAP.

To use LDAP in this way, you must have built {sendmail} to include LDAP
support. In your {.mc} file, add the lines

%\includegraphics{images/00820.gif}

Those lines tell {sendmail} that you want to use an LDAP database to
route incoming mail addressed to the specified domain. The
{LDAP\_DEFAULT\_SPEC} option identifies the LDAP server and the LDAP
basename for searches. LDAP uses port 389 unless you specify a different
port by adding {-p} {ldap\_port} to the {define}.

{sendmail} uses the values of two tags in the LDAP database:

\begin{itemize}
\item
  {mailLocalAddress} for the addressee on incoming mail
\item
  {mailRoutingAddress} for the destination to which email should be sent
\end{itemize}

{sendmail} also supports the tag {mailHost}, which if present routes
mail to the MX-designated mail handler for the specified host. The
recipient address remains the value of the {mailRoutingAddress} tag.

LDAP database entries support a wild card entry, {@domain}, that
reroutes mail addressed to anyone at the specified domain (as was done
in the {virtusertable}).

By default, mail addressed to user@host1.mydomain would first trigger a
lookup on user@host1.mydomain. If that failed, {sendmail} would try
@host1.mydomain but not user@mydomain. Including the line

%\includegraphics{images/00821.gif}

would also try the keys user@mydomain and @mydomain. This feature
enables a single database to route mail at a complex site. You can also
take the entries for the {LDAPROUTE\_EQUIVALENT} clauses from a file,
which makes the feature quite usable. The syntax for that form is

%\includegraphics{images/00822.gif}

Additional arguments to the {ldap\_routing} feature let you specify more
details about the LDAP schema and control the handling of addressee
names that have a {+detail} part. As always, see the {cf/README} file
for exact details.

\subsubsection[Masquerading
features]{\texorpdfstring{Ma\protect\hypertarget{part0026_split_034.htmlux5cux23_idTextAnchor1082}{}{}squerading
features}{Masquerading features}}

\protect\hypertarget{part0026_split_034.htmlux5cux23_idIndexMarker2530}{}{}An
email address is usually made up of a username, a host, and a domain,
but many sites do not want the names of their internal hosts exposed on
the Internet. The
\protect\hypertarget{part0026_split_034.htmlux5cux23_idIndexMarker2531}{}{}{MASQUERADE\_AS}
macro lets you specify a single identity for other machines to hide
behind. All mail appears to emanate from the designated machine or
domain. This is fine for regular users, but for debugging purposes,
system users such as root should be excluded from the masquerade.

For example, the
sequence\protect\hypertarget{part0026_split_034.htmlux5cux23_idIndexMarker2532}{}{}

%\includegraphics{images/00823.gif}

would stamp mail as coming from user@atrust.com unless it was sent by
root or the mail system; in these cases, the mail would carry the name
of the originating host.

{MASQUERADE\_AS} is actually just the tip of a vast masquerading iceberg
that extends downward through a dozen variations and exceptions. The
{allmasquerade} and {masquerade\_envelope} features (in combination with
{MASQUERADE\_AS}) hide just the right amount of local info. See the
{cf/README} for details.

\subsubsection[ and {SMART\_HOST}
macros]{\texorpdfstring{\protect\hypertarget{part0026_split_034.htmlux5cux23_idTextAnchor1083}{}{}\protect\hypertarget{part0026_split_034.htmlux5cux23_idIndexMarker2533}{}{}{MA\protect\hypertarget{part0026_split_034.htmlux5cux23_idTextAnchor1084}{}{}IL\_HUB}
and
\protect\hypertarget{part0026_split_034.htmlux5cux23_idIndexMarker2534}{}{}{SMART\_HOST}
macros}{MAIL\_HUB and SMART\_HOST macros}}

Masquerading makes all mail {appear} to come from a single host or
domain by rewriting the headers and, optionally, the envelope. But most
sites want all mail to {actually} come from (or go to) a single machine
so that they can control the flow of viruses, spam, and company secrets.
You can achieve this control with a combination of
\protect\hypertarget{part0026_split_034.htmlux5cux23_idIndexMarker2535}{}{}MX
records in DNS, the {MAIL\_HUB} macro for incoming mail, and the
{SMART\_HOST} macro for outgoing mail.

\leavevmode\hypertarget{part0026_split_034.htmlux5cux23_idContainer1147}{}%
See
\protect\hyperlink{part0024_split_027.htmlux5cux23_idTextAnchor882}{this
page} for more information about DNS MX records.

For example, in a structured email implementation, MX records would
direct incoming email from the Internet to an MTA in the network's
demilitarized zone. After verification that the received email was free
of viruses and spam and was directed to valid local users, the mail
could be relayed, with the following {define}, to the internal routing
MTA for delivery:

%\includegraphics{images/00824.gif}

\leavevmode\hypertarget{part0026_split_034.htmlux5cux23_idContainer1149}{}%
See the next section for more about {nullclient}.

Likewise, client machines would relay their mail to the {SMART\_HOST}
designated in the {nullclient} feature in their configuration. The
{SMART\_HOST} could then filter for viruses and spam so that mail from
your site did not pollute the Internet.

The syntax of {SMART\_HOST} parallels that of {MAIL\_HUB}, and the
default delivery agent is again {relay}. For example:

%\includegraphics{images/00825.gif}

You can use the same machine as the server for both incoming and
outgoing mail. Both the {SMART\_HOST} and the {MAIL\_HUB} must allow
relaying, the first from clients inside your domain and the second from
the MTA in the DMZ.

\protect\hypertarget{part0026_split_035.html}{}{}

\hypertarget{part0026_split_035.htmlux5cux23_idContainer1247}{}
\hypertarget{part0026_split_035.htmlux5cux23calibre_pb_34}{%
\subsection[Client
configuration]{\texorpdfstring{\protect\hypertarget{part0026_split_035.htmlux5cux23_idTextAnchor1085}{}{}Client
configuration}{Client configuration}}\label{part0026_split_035.htmlux5cux23calibre_pb_34}}

Most of your site's machines should be configured as clients who just
submit outgoing mail generated by users and don't receive mail at all.
One of {sendmail's} {FEATURE}s, {nullclient}, is just right for this
situation. It creates a config file that forwards all mail to a central
hub over SMTP. The entire config file, after the {VERSIONID} and
{OSTYPE} lines, would be simply

%\includegraphics{images/00826.gif}

where {mailserver} is the name of your central hub. The {nocanonify}
feature tells {sendmail} not to do DNS lookups or rewrite addresses with
fully qualified domain names. All that work will be done by the
{mailserver} host. This feature is similar to {SMART\_HOST} and assumes
that the client will {MASQUERADE\_AS} {mailserver}. The {EXPOSED\_USER}
clause exempts root from the masquerading and so facilitates debugging.

The {mailserver} machine must allow relaying from its null clients. That
permission is granted in the {access\_db}, described
\protect\hyperlink{part0026_split_034.htmlux5cux23_idTextAnchor1078}{here}.
The null client must have an associated MX record that points to
{mailserver} and must also be included in the {mailserver}'s {cw} file
(usually {/etc/mail/local-host-names}). These settings allow the
{mailserver} to accept mail for the client.

{sendmail} should run as an MSA (without the {-bd} flag) if the user
agents on the client machine can be taught to use port 587 for
submitting mail. If not, you can run {sendmail} in daemon mode ({-bd}),
but set the {DAEMON\_OPTIONS} configuration option to listen for
connections only on the loopback interface.

\protect\hypertarget{part0026_split_036.html}{}{}

\hypertarget{part0026_split_036.htmlux5cux23_idContainer1247}{}
\hypertarget{part0026_split_036.htmlux5cux23calibre_pb_35}{%
\subsection[ configuration
options]{\texorpdfstring{{\protect\hypertarget{part0026_split_036.htmlux5cux23_idTextAnchor1086}{}{}m4}
configuration
options}{m4 configuration options}}\label{part0026_split_036.htmlux5cux23calibre_pb_35}}

You set config file options with the {m4} {define} command. A complete
list of options that are accessible as {m4} variables (along with their
default values) is given in the {cf/README} file.

The defaults are OK for a typical site that is not too paranoid about
security and not too concerned with performance. The defaults try to
protect you from spam by turning off relaying, by requiring addresses to
be fully qualified, and by requiring that senders' domains resolve to an
IP address. If your mail hub machine is busy and services a lot of
mailing lists, you might need to tweak some of the performance values.

\protect\hyperlink{part0026_split_036.htmlux5cux23_idTextAnchor1087}{Table
18.10} lists some options that you might need to adjust (about 10\% of
over 175 configuration options). Their default values are shown in
parentheses. To save space, the option names are shown without their
{conf} prefix; for example, the {FAST\_SPLIT} option is actually named
{confFAST\_SPLIT}. We divided the table into subsections that identify
the kind of issue the variable addresses: resource management,
performance, security and spam abatement, and miscellaneous options.
Some options fit in more than one category, but we have listed them only
once.

\paragraph[{Table 18.10: }Basic {sendmail} configuration
options]{\texorpdfstring{{Table 18.10:
}\protect\hypertarget{part0026_split_036.htmlux5cux23_idTextAnchor1087}{}{}\protect\hypertarget{part0026_split_036.htmlux5cux23_idTextAnchor1088}{}{}Basic
{sendmail} configuration
options\protect\hypertarget{part0026_split_036.htmlux5cux23_idIndexMarker2536}{}{}\protect\hypertarget{part0026_split_036.htmlux5cux23_idIndexMarker2537}{}{}\protect\hypertarget{part0026_split_036.htmlux5cux23_idIndexMarker2538}{}{}\protect\hypertarget{part0026_split_036.htmlux5cux23_idIndexMarker2539}{}{}\protect\hypertarget{part0026_split_036.htmlux5cux23_idIndexMarker2540}{}{}\protect\hypertarget{part0026_split_036.htmlux5cux23_idIndexMarker2541}{}{}\protect\hypertarget{part0026_split_036.htmlux5cux23_idIndexMarker2542}{}{}\protect\hypertarget{part0026_split_036.htmlux5cux23_idIndexMarker2543}{}{}\protect\hypertarget{part0026_split_036.htmlux5cux23_idIndexMarker2544}{}{}\protect\hypertarget{part0026_split_036.htmlux5cux23_idIndexMarker2545}{}{}\protect\hypertarget{part0026_split_036.htmlux5cux23_idIndexMarker2546}{}{}\protect\hypertarget{part0026_split_036.htmlux5cux23_idIndexMarker2547}{}{}\protect\hypertarget{part0026_split_036.htmlux5cux23_idIndexMarker2548}{}{}\protect\hypertarget{part0026_split_036.htmlux5cux23_idIndexMarker2549}{}{}\protect\hypertarget{part0026_split_036.htmlux5cux23_idIndexMarker2550}{}{}\protect\hypertarget{part0026_split_036.htmlux5cux23_idIndexMarker2551}{}{}\protect\hypertarget{part0026_split_036.htmlux5cux23_idIndexMarker2552}{}{}\protect\hypertarget{part0026_split_036.htmlux5cux23_idIndexMarker2553}{}{}\protect\hypertarget{part0026_split_036.htmlux5cux23_idIndexMarker2554}{}{}\protect\hypertarget{part0026_split_036.htmlux5cux23_idIndexMarker2555}{}{}\protect\hypertarget{part0026_split_036.htmlux5cux23_idIndexMarker2556}{}{}\protect\hypertarget{part0026_split_036.htmlux5cux23_idIndexMarker2557}{}{}\protect\hypertarget{part0026_split_036.htmlux5cux23_idTextAnchor1089}{}{}\protect\hypertarget{part0026_split_036.htmlux5cux23_idIndexMarker2558}{}{}\protect\hypertarget{part0026_split_036.htmlux5cux23_idTextAnchor1090}{}{}}{Table 18.10: Basic sendmail configuration options}}

%\includegraphics{images/00827.gif}

\protect\hypertarget{part0026_split_037.html}{}{}

\hypertarget{part0026_split_037.htmlux5cux23_idContainer1247}{}
\hypertarget{part0026_split_037.htmlux5cux23calibre_pb_36}{%
\subsection[Spam-related features in
{sendmail}]{\texorpdfstring{\protect\hypertarget{part0026_split_037.htmlux5cux23_idTextAnchor1091}{}{}\protect\hypertarget{part0026_split_037.htmlux5cux23_idTextAnchor1092}{}{}Spam-related
features in
{sendmail}}{Spam-related features in sendmail}}\label{part0026_split_037.htmlux5cux23calibre_pb_36}}

{\protect\hypertarget{part0026_split_037.htmlux5cux23_idIndexMarker2559}{}{}}{sendmail}
has a variety of features and configuration options that can help you
control spam and viruses:

\begin{itemize}
\item
  Rules that control third party
  (\protect\hypertarget{part0026_split_037.htmlux5cux23_idIndexMarker2560}{}{}\protect\hypertarget{part0026_split_037.htmlux5cux23_idIndexMarker2561}{}{}aka
  promiscuous, aka open) relaying; that is, the use of your mail server
  by one off-site user to send mail to another off-site user. Spammers
  often use relaying to mask the true source of their mail and thereby
  avoid detection by ISPs. Relaying also lets spammers use {your} cycles
  and save their own.
\item
  The access database for filtering recipient addresses. This feature is
  rather like a firewall for email.
\item
  Blacklists that catalog open relays and known spam-friendly sites that
  {sendmail} can check against.
\item
  Throttles that can slow down mail acceptance when certain types of bad
  behavior are detected.
\item
  Header checking and input mail filtering by means of a generic mail
  filtering interface called {libmilter}. It allows arbitrary scanning
  of message headers and content and lets you reject messages that match
  a particular profile. Milters are plentiful and powerful; see
  milter.org.
\end{itemize}

\subsubsection[Relay
control]{\texorpdfstring{\protect\hypertarget{part0026_split_037.htmlux5cux23_idTextAnchor1093}{}{}Relay
control}{Relay control}}

\protect\hypertarget{part0026_split_037.htmlux5cux23_idIndexMarker2562}{}{}\protect\hypertarget{part0026_split_037.htmlux5cux23_idIndexMarker2563}{}{}{sendmail}
accepts incoming mail, looks at the envelope addresses, decides where
the mail should go, and then passes the message along to an appropriate
destination. That destination can be local or it can be another
transport agent farther along in the delivery chain. When an incoming
message has no local recipients, the transport agent that handles it is
said to be acting as a relay.

Only hosts that are tagged with {RELAY} in the access database (see
\protect\hyperlink{part0026_split_034.htmlux5cux23_idTextAnchor1078}{this
page}) or that are listed in
\protect\hypertarget{part0026_split_037.htmlux5cux23_idIndexMarker2564}{}{}{/etc/mail/relay-domains}
are allowed to submit mail for relaying. Some types of relaying are
useful and legitimate. How can you tell which messages to relay and
which to reject? Relaying is actually necessary in only three
situations:

\begin{itemize}
\item
  When the transport agent acts as a gateway for hosts that are not
  reachable in any other way; for example, hosts that are not always
  turned on (laptops, Windows PCs) and virtual hosts. In this situation,
  all the recipients for which you want to relay lie within the same
  domain.
\item
  When the transport agent is the outgoing mail server for other,
  not-so-smart hosts. In this case, all the senders' hostnames or IP
  addresses are local (or at least enumerable).
\item
  When you have agreed to be a backup MX destination for another site.
\end{itemize}

Any other situation that appears to require relaying is probably just an
indication of bad design (with the possible exception of support for
mobile users). You can obviate the first use of relaying (above) by
designating a centralized server to receive mail, with POP or IMAP being
used for client access. The second case should always be allowed, but
only for your own hosts. You can check IP addresses or hostnames. In the
third case, you can list the other site in your access database and
allow relaying just for that site's IP address blocks.

Although {sendmail} comes with relaying turned off by default, several
features can turn relaying back on, either fully or in a limited and
controlled way. These features are listed below for completeness, but
our recommendation is that you be careful about opening things up too
much. The {access\_db} feature is the safest way to allow limited
relaying.

\begin{itemize}
\item
  {FEATURE(`}{\protect\hypertarget{part0026_split_037.htmlux5cux23_idIndexMarker2565}{}{}}{relay\_entire\_domain')}
  -- allows relaying for just your domain
\item
  \protect\hypertarget{part0026_split_037.htmlux5cux23_idIndexMarker2566}{}{}{RELAY\_DOMAIN(`}{domain}{,
  ...')} -- adds more domains to be relayed
\item
  {RELAY\_DOMAIN\_FILE(`}{filename}{')} -- same; takes domain list from
  a file
\item
  {FEATURE(`}{\protect\hypertarget{part0026_split_037.htmlux5cux23_idIndexMarker2567}{}{}}{relay\_hosts\_only')}
  -- affects {RELAY\_DOMAIN}, {accessdb}
\end{itemize}

You need to make an exception if you use the {SMART\_HOST} or
{MAIL\_HUB} designations to route mail through a particular mail server
machine. That server must be set up to relay mail from local hosts.
Configure it with

%\includegraphics{images/00828.gif}

If you consider turning on relaying in some form, consult the {sendmail}
documentation in {cf/README} to be sure you don't inadvertently become a
friend of spammers. When you are done, have one of the relay-checking
sites verify that you did not inadvertently create an open relay---try
spamhelp.org.

\subsubsection[User or site
blacklisting]{\texorpdfstring{\protect\hypertarget{part0026_split_037.htmlux5cux23_idTextAnchor1094}{}{}\protect\hypertarget{part0026_split_037.htmlux5cux23_idTextAnchor1095}{}{}U\protect\hypertarget{part0026_split_037.htmlux5cux23_idTextAnchor1096}{}{}ser
or site blacklisting}{User or site blacklisting}}

\protect\hypertarget{part0026_split_037.htmlux5cux23_idIndexMarker2568}{}{}\protect\hypertarget{part0026_split_037.htmlux5cux23_idIndexMarker2569}{}{}\protect\hypertarget{part0026_split_037.htmlux5cux23_idIndexMarker2570}{}{}If
you have local users or hosts to which you want to block mail,
use\protect\hypertarget{part0026_split_037.htmlux5cux23_idIndexMarker2571}{}{}

%\includegraphics{images/00829.gif}

It supports the following types of entries in your access file:

%\includegraphics{images/00830.gif}

These lines block incoming mail to user nobody on any host, to host
printer, and to a particular user's address on one machine. The use of
the {To:} tag lets these users send messages, just not receive them;
some printers have that capability.

To include a DNS-style blacklist for incoming email, use the {dnsbl}
feature:

%\includegraphics{images/00831.gif}

This feature makes {sendmail} reject mail from any site whose IP address
is in any of the three blacklists of known spammers (SBL, XBL, and PBL)
maintained at {spamhaus.org}. Other lists catalog sites that run open
relays and blocks of addresses that are known to be havens for spammers.
These blacklists are distributed through a clever tweak of the DNS
system; hence the name {dnsbl}.

You can pass a third argument to the {dnsbl} feature to specify the
error message you would like returned. If you omit this argument,
{sendmail} returns a fixed error message from the DNS database that
contains the records.

You can include the {dnsbl} feature several times to check multiple
lists of abusers.

\subsubsection[Throttles, rates, and connection
limits]{\texorpdfstring{\protect\hypertarget{part0026_split_037.htmlux5cux23_idTextAnchor1097}{}{}Throttles,
rates, and connection limits}{Throttles, rates, and connection limits}}

\protect\hyperlink{part0026_split_037.htmlux5cux23_idTextAnchor1098}{Table
18.11} lists several {sendmail} controls that can slow down mail
processing when clients' behavior appears suspicious.

\paragraph[{Table 18.11: }'s ``slow down'' configuration
primitives]{\texorpdfstring{{Table 18.11:
}{\protect\hypertarget{part0026_split_037.htmlux5cux23_idTextAnchor1098}{}{}\protect\hypertarget{part0026_split_037.htmlux5cux23_idTextAnchor1099}{}{}sendmail}'s
``slow down'' configuration
primitives{\protect\hypertarget{part0026_split_037.htmlux5cux23_idIndexMarker2572}{}{}\protect\hypertarget{part0026_split_037.htmlux5cux23_idIndexMarker2573}{}{}\protect\hypertarget{part0026_split_037.htmlux5cux23_idIndexMarker2574}{}{}\protect\hypertarget{part0026_split_037.htmlux5cux23_idIndexMarker2575}{}{}\protect\hypertarget{part0026_split_037.htmlux5cux23_idIndexMarker2576}{}{}}}{Table 18.11: sendmail's ``slow down'' configuration primitives}}

%\includegraphics{images/00832.gif}

After the no-such-login count reaches the limit set in the
{BAD\_RCPT\_THROTTLE} option, {sendmail} sleeps for one second after
each rejected {RCPT} command, slowing a spammer's address harvesting to
a crawl. To set that threshold to 3, use

%\includegraphics{images/00833.gif}

Setting the {MAX\_RCPTS\_PER\_MESSAGE} option causes the sender to queue
extra recipients for later. This is a cheap form of greylisting for
messages that have a suspiciously large number of recipients.

The {ratecontrol} and {conncontrol} features allow per-host or per-net
limits on the rate at which incoming connections are accepted and the
number of simultaneous connections, respectively. Both use the
{/etc/mail/access} file to specify the limits and the domains to which
they should apply, the first with the tag {ClientRate:} in the key field
and the second with tag {ClientConn:}. To enable rate controls, insert
lines like these in your {.mc} file:

%\includegraphics{images/00834.gif}

Then, add to your {/etc/mail/access} file the list of hosts or nets to
be controlled and their restriction thresholds. For example, the lines

%\includegraphics{images/00835.gif}

limit the hosts 192.168.6.17 and 170.65.3.4 to two new connections per
minute and ten new connections per minute, respectively. The lines

%\includegraphics{images/00836.gif}

set limits of two simultaneous connections for 192.168.2.8, seven for
175.14.4.1, and ten simultaneous connections for all other hosts.

Another nifty feature is {greet\_pause}. When a remote transport agent
connects to your {sendmail} server, the SMTP protocol mandates that it
wait for your server's welcome greeting before speaking. However, it's
common for spam mailers to blurt out an EHLO/HELO command immediately.
This behavior is partially explainable as poor implementation of the
SMTP protocol in spam-sending tools, but it may also be a feature that
aims to save time on the spammer's behalf. Whatever the cause, this
behavior is suspicious and is known as ``slamming.''

The {greet\_pause} feature makes {sendmail} wait for a specified period
of time at the beginning of the connection before greeting its newfound
friend. If the remote MTA does not wait to be properly greeted and
proceeds with an EHLO or HELO command during the planned awkward moment,
{sendmail} logs an error and refuses subsequent commands from the remote
MTA.

You can enable greeting pauses with this entry in the {.mc} file:

%\includegraphics{images/00837.gif}

This line causes a 700 millisecond delay at the beginning of every new
connection. You can set per-host or per-net delays with a {GreetPause:}
prefix in the access database, but most sites use a blanket value for
this feature.

\protect\hypertarget{part0026_split_038.html}{}{}

\hypertarget{part0026_split_038.htmlux5cux23_idContainer1247}{}
\hypertarget{part0026_split_038.htmlux5cux23calibre_pb_37}{%
\subsection[Security and
{sendmail}]{\texorpdfstring{\protect\hypertarget{part0026_split_038.htmlux5cux23_idTextAnchor1100}{}{}\protect\hypertarget{part0026_split_038.htmlux5cux23_idTextAnchor1101}{}{}Security
and
{sendmail}}{Security and sendmail}}\label{part0026_split_038.htmlux5cux23calibre_pb_37}}

{\protect\hypertarget{part0026_split_038.htmlux5cux23_idIndexMarker2577}{}{}\protect\hypertarget{part0026_split_038.htmlux5cux23_idIndexMarker2578}{}{}}With
the explosive growth of the Internet, programs such as {sendmail} that
accept arbitrary user-supplied input and deliver it to local users,
files, or shells have frequently provided an avenue of attack for
hackers. {sendmail}, along with DNS and even IP, is flirting with
authentication and encryption as a built-in solution to some of these
fundamental security issues.

{sendmail} supports both SMTP authentication and encryption with
\protect\hypertarget{part0026_split_038.htmlux5cux23_idIndexMarker2579}{}{}\protect\hypertarget{part0026_split_038.htmlux5cux23_idIndexMarker2580}{}{}TLS,
Transport Layer Security (formerly known as SSL, the Secure Sockets
Layer). TLS brought with it six new configuration options for
certificate files and key files. New actions for access database matches
can require that authentication must have succeeded.

{sendmail} carefully inspects file permissions before it believes the
contents of, say, a {.forward} or an {aliases} file. Although this
tightening of security is generally welcome, it's sometimes necessary to
relax the tough policies. To this end, {sendmail} introduced the
\protect\hypertarget{part0026_split_038.htmlux5cux23_idIndexMarker2581}{}{}{DontBlameSendmail}
option, so named in hopes that the name might suggest to sysadmins that
what they are doing is unsafe.

This option has many possible values---55 at last count. The default is
{safe}, the strictest possible. For a complete list of values, see
{doc/op/op.ps }in the {sendmail} distribution or the O'Reilly {sendmail}
book. Or just leave the option set to {safe}.

\subsubsection[Ownerships]{\texorpdfstring{\protect\hypertarget{part0026_split_038.htmlux5cux23_idTextAnchor1102}{}{}Ownerships}{Ownerships}}

Three user accounts are important in the {sendmail} universe: the
{DefaultUser}, the {RunAsUser}, and the {TrustedUser}.

By default, all of {sendmail}'s mailers run as the {DefaultUser} unless
the mailer's flags specify otherwise. If a user mailnull, sendmail, or
daemon exists in the {passwd} file, {DefaultUser} will be that.
Otherwise, it defaults to UID 1 and GID 1. We recommend the use of the
mailnull account and a mailnull group. Add it to {/etc/passwd} with a
star as the password, no valid shell, no home directory, and a default
group of mailnull. You'll have to add the mailnull entry to the {group}
file, too. The mailnull account should not own any files. If {sendmail}
is not running as root, the mailers must be setuid.

If {RunAsUser} is set, {sendmail} ignores the value of {DefaultUser} and
does everything as {RunAsUser}. If you are running {sendmail} setgid,
then the submission {sendmail} just passes messages to the real
{sendmail} through SMTP. The real {sendmail} does not have its setuid
bit set, but it runs as root from the startup files.

The {RunAsUser} is the UID that {sendmail} runs under after opening its
socket connection to port 25. Ports numbered less than 1,024 can be
opened only by the superuser; therefore, {sendmail} must initially run
as root. However, after performing this operation, {sendmail} can switch
to a different UID. Such a switch reduces the risk of damage or access
if {sendmail} is tricked into doing something bad. Don't use the
{RunAsUser} feature on machines that support user accounts or other
services; it is meant for use only on firewalls or bastion hosts
(specially hardened hosts intended to withstand attack when placed in a
DMZ or outside a firewall).

By default, {sendmail} does not switch identities and continues to run
as root. If you change the {RunAsUser} to something other than root, you
must change several other things as well. The {RunAsUser} must own the
mail queue, be able to read all maps and include files, be able to run
programs, etc. Expect to spend a few hours discovering all the file and
directory ownerships that must be changed.

{sendmail}'s {TrustedUser} can own maps and alias files. The
{TrustedUser} is allowed to start the daemon or rebuild the {aliases}
file. This facility exists mostly to support GUI interfaces to
{sendmail} that need to provide limited administrative control to
certain users. If you set {TrustedUser}, be sure to guard the account
that it points to because this account can easily be exploited to gain
root access. The {TrustedUser} is different from the {TRUSTED\_USERS}
class, which determines who can rewrite the From line of messages. (The
{TRUSTED\_USERS} feature is typically used to support mailing list
software.)

\subsubsection[Permissions]{\texorpdfstring{\protect\hypertarget{part0026_split_038.htmlux5cux23_idTextAnchor1103}{}{}Permissions}{Permissions}}

\protect\hypertarget{part0026_split_038.htmlux5cux23_idIndexMarker2582}{}{}File
and directory permissions are important to {sendmail} security. Use the
settings listed in
\protect\hyperlink{part0026_split_038.htmlux5cux23_idTextAnchor1104}{Table
18.12} to be safe.

\paragraph[{Table 18.12: }Owner and permissions for {sendmail}-related
directories]{\texorpdfstring{{Table 18.12:
}\protect\hypertarget{part0026_split_038.htmlux5cux23_idTextAnchor1104}{}{}\protect\hypertarget{part0026_split_038.htmlux5cux23_idTextAnchor1105}{}{}Owner
and permissions for {sendmail}-related
directories}{Table 18.12: Owner and permissions for sendmail-related directories}}

%\includegraphics{images/00838.gif}

{sendmail} no longer reads
\protect\hypertarget{part0026_split_038.htmlux5cux23_idIndexMarker2583}{}{}{.forward}
files that have link counts greater than 1 if the directory paths that
lead to them have lax permissions. This rule bit Evi when one of her
{.forward} files, which she usually hard-linked to either
{.forward.to.boulder} or {.forward.to.sandiego}, silently failed to
forward her mail from a small site at which she did not receive much
mail. It was months before she realized that ``I never got your mail''
was her own fault and not a valid excuse.

You can turn off many of the restrictive file access policies mentioned
above with the {DontBlameSendmail} option. But don't do that.

\subsubsection[Safer mail to files and
programs]{\texorpdfstring{\protect\hypertarget{part0026_split_038.htmlux5cux23_idTextAnchor1106}{}{}Safer
mail to files and programs}{Safer mail to files and programs}}

We recommend that you use
\protect\hypertarget{part0026_split_038.htmlux5cux23_idIndexMarker2584}{}{}{smrsh}
instead of {/bin/sh} as your program mailer and that you use
{mail.local} instead of {/bin/mail} as your local mailer. Both programs
are included in the {sendmail} distribution. To incorporate them into
your configuration, add the lines

%\includegraphics{images/00839.gif}

to your {.mc} file. If you omit the explicit paths, the commands are
assumed to live in {/usr/libexec}. You can use {sendmail}'s
{confEBINDIR} option to change the default location of the binaries to
whatever you want.
\protect\hyperlink{part0026_split_038.htmlux5cux23_idTextAnchor1107}{Table
18.13} helps you find where our friendly vendors have stashed things.

\paragraph[{Table 18.13: }Location of {sendmail}'s restricted delivery
agents]{\texorpdfstring{{Table 18.13:
}\protect\hypertarget{part0026_split_038.htmlux5cux23_idTextAnchor1107}{}{}\protect\hypertarget{part0026_split_038.htmlux5cux23_idTextAnchor1108}{}{}Location
of {sendmail}'s restricted delivery
agents}{Table 18.13: Location of sendmail's restricted delivery agents}}

%\includegraphics{images/00840.gif}

{smrsh} is a restricted shell that executes only the programs contained
in one directory ({/usr/adm/sm.bin} by default). {smrsh} ignores
user-specified paths and tries to find any requested commands in its own
known-safe directory. {smrsh} also blocks the use of certain shell
metacharacters such as \textless, the input redirection symbol. Symbolic
links are allowed in {sm.bin}, so you need not make duplicate copies of
the programs you allow. The {vacation} program is a good candidate for
{sm.bin}. Don't put {procmail} there; it's insecure.

Here are some example shell commands and their possible {smrsh}
interpretations:

%\includegraphics{images/00841.gif}

{sendmail}'s {SafeFileEnvironment} option controls where files can be
written when email is redirected to a file by {aliases} or a {.forward}
file. It causes {sendmail} to execute a {chroot} system call, making the
root of the filesystem no longer {/} but rather {/safe} or whatever path
you specified in the {SafeFileEnvironment} option. An alias that
directed mail to the {/etc/passwd} file, for example, would actually be
written to {/safe/etc/passwd}.

The {SafeFileEnvironment} option also protects device files,
directories, and other special files by allowing writes only to regular
files. Besides increasing security, this option ameliorates the effects
of user mistakes. Some sites set the option to {/home} to allow access
to home directories while keeping system files off-limits.

Mailers can also be run in a {chroot}ed directory.

\subsubsection[Privacy
options]{\texorpdfstring{Priv\protect\hypertarget{part0026_split_038.htmlux5cux23_idTextAnchor1109}{}{}acy
options}{Privacy options}}

\protect\hypertarget{part0026_split_038.htmlux5cux23_idIndexMarker2585}{}{}\protect\hypertarget{part0026_split_038.htmlux5cux23_idIndexMarker2586}{}{}{sendmail}
privacy options also control

\begin{itemize}
\item
  What external folks can determine about your site through SMTP
\item
  What you require of the host on the other end of an SMTP connection
\item
  Whether your users can see or run the mail queue
\end{itemize}

\protect\hyperlink{part0026_split_038.htmlux5cux23_idTextAnchor1110}{Table
18.14} lists the possible values for the privacy options as of this
writing; see the file {doc/op/op.ps} in the distribution for current
information.

\paragraph[{Table 18.14: }Values of the {PrivacyOption}
variable]{\texorpdfstring{{Table 18.14:
}\protect\hypertarget{part0026_split_038.htmlux5cux23_idIndexMarker2587}{}{}\protect\hypertarget{part0026_split_038.htmlux5cux23_idTextAnchor1110}{}{}\protect\hypertarget{part0026_split_038.htmlux5cux23_idTextAnchor1111}{}{}Values
of the {PrivacyOption}
variable}{Table 18.14: Values of the PrivacyOption variable}}

%\includegraphics{images/00842.gif}

We recommend conservatism; in your {.mc} file, use

%\includegraphics{images/00843.gif}

{sendmail}'s default value for the privacy options is {authwarnings};
the above line would reset that value. Notice the double sets of quotes;
some versions of {m4} require them to protect the commas in the list of
privacy option values.

\subsubsection[Running a {chroot}ed {sendmail} (for the truly
paranoid)]{\texorpdfstring{\protect\hypertarget{part0026_split_038.htmlux5cux23_idTextAnchor1112}{}{}Running
a {chroot}ed {sendmail} (for the truly
paranoid)}{Running a chrooted sendmail (for the truly paranoid)}}

\protect\hypertarget{part0026_split_038.htmlux5cux23_idIndexMarker2588}{}{}If
you are worried about the access that {sendmail} has to your filesystem,
you can start it in a {chroot}ed jail. Create a minimal filesystem in
your jail, including things like {/dev/null}, {/etc} essentials
({passwd}, {group}, {resolv.conf}, {sendmail.cf}, any map files,
{mail/*}), the shared libraries that {sendmail} needs, the {sendmail}
binary, the mail queue directory, and any log files. You will probably
have to fiddle with the list to get it just right. Use the {chroot}
command to start a jailed {sendmail}. For example:

%\includegraphics{images/00844.gif}

\subsubsection[Denial of service
attacks]{\texorpdfstring{\protect\hypertarget{part0026_split_038.htmlux5cux23_idTextAnchor1113}{}{}Denial
of service attacks}{Denial of service attacks}}

\protect\hypertarget{part0026_split_038.htmlux5cux23_idIndexMarker2589}{}{}Denial
of service attacks are difficult to prevent because no a priori method
can determine that a message is an attack rather than a valid piece of
email. Attackers can try various nasty things, including flooding the
SMTP port with bogus {connections}, filling disk partitions with giant
messages, clogging outgoing connections, and mail bombing. {sendmail}
has some configuration parameters that can help slow down or limit the
impact of a denial of service attack, but these parameters can also
interfere with the delivery of legitimate mail.

The
\protect\hypertarget{part0026_split_038.htmlux5cux23_idIndexMarker2590}{}{}{MaxDaemonChildren}
option limits the number of {sendmail} processes. It prevents the system
from being overwhelmed with {sendmail} work. However, it also allows an
attacker to easily shut down SMTP service.

The
\protect\hypertarget{part0026_split_038.htmlux5cux23_idIndexMarker2591}{}{}{MaxMessageSize}
option can help prevent the mail queue directory from filling. But if
you set it too low, legitimate mail will bounce. You might mention your
limit to users so that they aren't surprised when their mail bounces. We
recommend a fairly high limit (such as 50MB) anyway, since some
legitimate mail is huge.

The
\protect\hypertarget{part0026_split_038.htmlux5cux23_idIndexMarker2592}{}{}{ConnectionRateThrottle}
option, which limits the number of permitted connections per second, can
slow things down a bit. Finally, setting {MaxRcptsPerMessage}, which
controls the maximum number of recipients allowed on a single message,
may also help.

{sendmail} has always been able to refuse connections (option
\protect\hypertarget{part0026_split_038.htmlux5cux23_idIndexMarker2593}{}{}{REFUSE\_LA})
or queue email
(\protect\hypertarget{part0026_split_038.htmlux5cux23_idIndexMarker2594}{}{}{QUEUE\_LA})
according to the system load average. A variation,
\protect\hypertarget{part0026_split_038.htmlux5cux23_idIndexMarker2595}{}{}{DELAY\_LA},
keeps the mail flowing, but at a reduced rate.

In spite of all these protections for your mail system, someone mail
bombing you will still interfere with legitimate mail. Mail bombing can
be quite nasty.

\subsubsection[TLS: Transport Layer
Security]{\texorpdfstring{\protect\hypertarget{part0026_split_038.htmlux5cux23_idTextAnchor1114}{}{}TLS:
Transport Layer Security}{TLS: Transport Layer Security}}

\leavevmode\hypertarget{part0026_split_038.htmlux5cux23_idContainer1170}{}%
See
\protect\hyperlink{part0037_split_040.htmlux5cux23_idTextAnchor1727}{this
page} for general information about TLS.

\protect\hypertarget{part0026_split_038.htmlux5cux23_idIndexMarker2596}{}{}\protect\hypertarget{part0026_split_038.htmlux5cux23_idIndexMarker2597}{}{}TLS,
a encryption/authentication system, is specified in RFC3207. It is
implemented in {sendmail} as an extension to SMTP called
\protect\hypertarget{part0026_split_038.htmlux5cux23_idIndexMarker2598}{}{}STARTTLS.

Strong authentication can replace a hostname or IP address as the
authorization token for relaying mail or for accepting a connection from
a host in the first place. An entry such as

%\includegraphics{images/00845.gif}

in the {access\_db} indicates that STARTTLS is in use and that email to
the domain secure.example.com must be encrypted with at least 112-bit
encryption keys. Email from a host in the laptop.example.com domain
should be accepted only if the client has authenticated itself.

Although STARTTLS provides strong encryption, note that its protection
covers only the journey to the ``next hop'' MTA. Once the message
arrives at the next hop, it might be forwarded to another MTA that does
not use a secure transport method. If you have control of all possible
MTAs in the path, you can create a secure mail transport network. If
not, you will need to rely on a UA-based encryption package (such as
PGP/GPG) or a centralized email encryption service (see
\protect\hyperlink{part0026_split_017.htmlux5cux23_idTextAnchor1028}{this
page}).

\protect\hypertarget{part0026_split_038.htmlux5cux23_idIndexMarker2599}{}{}Greg
Shapiro and
\protect\hypertarget{part0026_split_038.htmlux5cux23_idIndexMarker2600}{}{}Claus
Assmann of Sendmail, Inc., have stashed some (slightly dated) extra
documentation about security and {sendmail} on the web. It's available
from sendmail.org/\textasciitilde gshapiro and
sendmail.org/\textasciitilde ca. The {index} link in \textasciitilde ca
is especially useful.

\protect\hypertarget{part0026_split_039.html}{}{}

\hypertarget{part0026_split_039.htmlux5cux23_idContainer1247}{}
\hypertarget{part0026_split_039.htmlux5cux23calibre_pb_38}{%
\subsection[ testing and
debugging]{\texorpdfstring{{\protect\hypertarget{part0026_split_039.htmlux5cux23_idTextAnchor1115}{}{}sendmail}
testing and
debugging}{sendmail testing and debugging}}\label{part0026_split_039.htmlux5cux23calibre_pb_38}}

\protect\hypertarget{part0026_split_039.htmlux5cux23_idIndexMarker2601}{}{}\protect\hypertarget{part0026_split_039.htmlux5cux23_idIndexMarker2602}{}{}{\protect\hypertarget{part0026_split_039.htmlux5cux23_idTextAnchor1116}{}{}m4}-based
configurations are to some extent pretested. You probably won't need to
do low-level debugging if you use them. But one thing the debugging
flags cannot
t\protect\hypertarget{part0026_split_039.htmlux5cux23_idTextAnchor1117}{}{}est
is your design.

While researching this chapter, we found errors in several of the
configuration files and designs that we examined. The errors ranged from
invoking a feature without the prerequisite macro (e.g., enabling
{masquerade\_envelope} without having turned on masquerading with
{MASQUERADE\_AS}) to total conflict between the design of the {sendmail}
configuration and the firewall that controlled whether and under what
conditions mail was allowed in.

You cannot design a mail system in a vacuum. You must synchronize it
with (or at least not be in conflict with) your DNS MX records and your
firewall policy.

\subsubsection[Queue
monitoring]{\texorpdfstring{\protect\hypertarget{part0026_split_039.htmlux5cux23_idTextAnchor1118}{}{}Queue
monitoring}{Queue monitoring}}

\protect\hypertarget{part0026_split_039.htmlux5cux23_idIndexMarker2603}{}{}You
can use the
\protect\hypertarget{part0026_split_039.htmlux5cux23_idIndexMarker2604}{}{}{mailq}
command (which is equivalent to {sendmail -bp}) to view the status of
queued messages. Messages are queued while they are being delivered or
when delivery has been attempted but has failed.

{mailq} prints a human-readable summary of the files in
{/var/spool/mqueue} at any given moment. The output is useful for
determining why a message may have been delayed. If it appears that a
mail backlog is developing, you can monitor the status of {sendmail}'s
attempts to clear the jam.

There are two default queues: one for messages received on port 25 and
another for messages received on port 587 (the client submission queue).
You can invoke {mailq -Ac} to see the client queue.

Below, some typical output from {mailq} shows three messages waiting to
be delivered.

%\includegraphics{images/00846.gif}

If you think you understand the situation better than {sendmail} or you
just want {sendmail} to try to redeliver the queued messages
immediately, you can force a queue run with {sendmail -q}. If you use
{sendmail -q -v}, {sendmail} shows the play-by-play results of each
delivery attempt, information that is often useful for debugging. Left
to its own devices, {sendmail} retries delivery every queue run interval
(typically every 30 minutes).

\subsubsection[Logging]{\texorpdfstring{\protect\hypertarget{part0026_split_039.htmlux5cux23_idTextAnchor1119}{}{}Logging}{Logging}}

\leavevmode\hypertarget{part0026_split_039.htmlux5cux23_idContainer1173}{}%
See
\protect\hyperlink{part0017_split_000.htmlux5cux23_idTextAnchor493}{Chapter
10} for more information about syslog.

{\protect\hypertarget{part0026_split_039.htmlux5cux23_idIndexMarker2605}{}{}}{sendmail}
uses syslog to log error and status messages with the syslog facility
``mail'' and levels ``debug'' through ``crit''; messages are tagged with
the string ``sendmail.'' You can override the logging string
``sendmail'' with the {-L} command-line option; this capability is handy
if you are debugging one copy of {sendmail} while other copies are doing
regular email chores.

The
\protect\hypertarget{part0026_split_039.htmlux5cux23_idIndexMarker2606}{}{}{confLOG\_LEVE\protect\hypertarget{part0026_split_039.htmlux5cux23_idTextAnchor1120}{}{}L}
option, specified on the command line or in the config file, determines
the severity level that {sendmail} uses as a threshold for logging.
Hig\protect\hypertarget{part0026_split_039.htmlux5cux23_idTextAnchor1121}{}{}h
values of the log level imply low severity levels and cause more info to
be logged.

\protect\hyperlink{part0026_split_039.htmlux5cux23_idTextAnchor1122}{Table
18.15} gives an approximate mapping between {sendmail} log levels and
syslog severity levels.

\paragraph[{Table 18.15: } log levels (L) vs. syslog
levels]{\texorpdfstring{{Table 18.15:
}{\protect\hypertarget{part0026_split_039.htmlux5cux23_idTextAnchor1122}{}{}\protect\hypertarget{part0026_split_039.htmlux5cux23_idTextAnchor1123}{}{}sendmail}
log levels (L) vs. syslog
levels}{Table 18.15: sendmail log levels (L) vs. syslog levels}}

%\includegraphics{images/00847.gif}

Recall that a message logged to syslog at a particular level is reported
to that level and all those above it. The {/etc/syslog.conf} or
{/etc/rsyslog.conf} file determines the eventual destination of each
message.
\protect\hyperlink{part0026_split_039.htmlux5cux23_idTextAnchor1124}{Table
18.16} shows their default locations.

\paragraph[{Table 18.16: }Default {sendmail} log
locations]{\texorpdfstring{{Table 18.16:
}\protect\hypertarget{part0026_split_039.htmlux5cux23_idTextAnchor1124}{}{}Default
{sendmail} log locations}{Table 18.16: Default sendmail log locations}}

%\includegraphics{images/00848.gif}

Several programs can summarize {sendmail} log files, with the end
products ranging from simple counts and text tables ({mreport}) to fancy
web pages (Yasma). You might need to limit access to this data or at
least inform your users that you are collecting it.



\section{Exim}

\protect\hypertarget{part0026_split_040.htmlux5cux23_idIndexMarker2607}{}{}The
Exim mail transport and submission agent was written in 1995 by
\protect\hypertarget{part0026_split_040.htmlux5cux23_idIndexMarker2608}{}{}Philip
Hazel of the
\protect\hypertarget{part0026_split_040.htmlux5cux23_idIndexMarker2609}{}{}University
of Cambridge and is distributed under the GNU General Public License.
The current release, Exim version 4.89, came out in spring 2017. Tons of
Exim documentation are available on-line, as are a couple of books by
the author of the software.

Googling for Exim questions often seems to lead to old, undated, and
sometimes inappropriate materials, so check the official documentation
first. A 400+ page specification and configuration document
({doc/spec.txt}) is included in the distribution. This document is also
available from exim.org as a PDF file. It's the definitive reference
work for Exim and is updated religiously with each new release.

There are two cultures with respect to Exim configuration: Debian's and
the rest of the world's. Debian runs its own set of mailing lists to
support users; we do not cover the Debian-specific configuration
extensions here.

Exim is like {sendmail} in that it is implemented as a single process
that performs essentially all the ongoing chores associated with email.
However, Exim does not carry all {sendmail}'s historical baggage
(support for ancient address formats, needing to get mail to hosts not
on the Internet, etc.). Many aspects of Exim's behavior are specified at
compile time, the chief examples being Exim's database and message store
formats.

The workhorses in the Exim system are called routers and transports.
Both are included in the general category of ``drivers.'' Routers decide
how messages should be delivered, and transports decide on the mechanics
of making deliveries. Routers are an ordered list of things to try,
whereas transports are an unordered set of delivery methods.

\protect\hypertarget{part0026_split_041.html}{}{}

\hypertarget{part0026_split_041.htmlux5cux23_idContainer1247}{}
\hypertarget{part0026_split_041.htmlux5cux23calibre_pb_40}{%
\subsection[Exim
installation]{\texorpdfstring{\protect\hypertarget{part0026_split_041.htmlux5cux23_idTextAnchor1127}{}{}Exim
installation}{Exim installation}}\label{part0026_split_041.htmlux5cux23calibre_pb_40}}

\protect\hypertarget{part0026_split_041.htmlux5cux23_idIndexMarker2610}{}{}You
can download the latest distribution from exim.org or from your favorite
package repository. Refer to the top-level {README} file and the file
{src/EDITME}, in which you must set installation locations, user IDs,
and other compile-time parameters. {EDITME} is over 1,000 lines long,
but it's mostly comments that lead you through the compilation process;
required changes are well labeled. After your edits, save the file as
{../Local/Makefile }or {../Local/Makefile-}{osname}{ }(if you are
building configurations for several different operating systems from the
same distribution directory) before you run {make}.

\protect\hypertarget{part0026_split_041.htmlux5cux23_idIndexMarker2611}{}{}Here
are a few of the important variables (our opinion) and suggested values
(Exim developers' opinion) from the {EDITME} file. The first five are
required, and the rest are
recommended.\protect\hypertarget{part0026_split_041.htmlux5cux23_idIndexMarker2612}{}{}\protect\hypertarget{part0026_split_041.htmlux5cux23_idIndexMarker2613}{}{}\protect\hypertarget{part0026_split_041.htmlux5cux23_idIndexMarker2614}{}{}\protect\hypertarget{part0026_split_041.htmlux5cux23_idIndexMarker2615}{}{}\protect\hypertarget{part0026_split_041.htmlux5cux23_idIndexMarker2616}{}{}

%\includegraphics{images/00849.gif}

Routers and transports must be compiled into the code if you intend to
use them. In these days of large memories, you might as well leave them
all in. Some default paths are certainly nonstandard: for example, the
binary in {/usr/exim/bin} and the PID file in {/var/spool/exim}. You
might want to tweak these values to match your other installed software.

About ten
dat\protect\hypertarget{part0026_split_041.htmlux5cux23_idTextAnchor1128}{}{}abase
lookup methods are available, including MySQL, Oracle, and LDAP. If you
include LDAP, you must specify the {LDAP\_LIB\_TYPE} variable to tell
Exim which LDAP library you are using. You may also need to specify the
path to LDAP include files and libraries.

The {EDITME} file does a good job of telling you about any dependencies
your database choices might entail. Any entries above that have ``(from
README)'' in their comment line were not listed in {src/EDITME} but
rather in the {README}.

\protect\hypertarget{part0026_split_041.htmlux5cux23_idIndexMarker2617}{}{}\protect\hypertarget{part0026_split_041.htmlux5cux23_idIndexMarker2618}{}{}{EDITME}
has many additional security options that you might want to include,
such as support for SMTP AUTH, TLS, PAM, and options for controlling
file ownerships and permissions. You can disable certain Exim options at
compile time to limit the damage a hacker might cause if the software is
compromised.

It's advisable to read the entire {EDITME} file before you complete the
installation. It gives you a good feel for what you can control at run
time through the configuration file. The top-level {README} file has
lots of detail about OS-specific quirks that you migh need to add to the
{EDITME} file as well.

Once you have modified {EDITME} and installed it as {Local/Makefile},
run {make} at the top of the distribution tree followed by {sudo make
install}. The next step is to test your shiny new {exim} binary and see
if it delivers mail as expected. The {doc/spec.txt} file contains good
testing documentation.

Once you are satisfied that Exim is working properly, link
{/usr/sbin/sendmail} to
\protect\hypertarget{part0026_split_041.htmlux5cux23_idIndexMarker2619}{}{}{exim}
so that Exim can emulate the traditional command-line interface to the
mail system used by many user agents. You must also arrange for {exim}
to be started at boot time.

\protect\hypertarget{part0026_split_042.html}{}{}

\hypertarget{part0026_split_042.htmlux5cux23_idContainer1247}{}
\hypertarget{part0026_split_042.htmlux5cux23calibre_pb_41}{%
\subsection[Exim
startup]{\texorpdfstring{\protect\hypertarget{part0026_split_042.htmlux5cux23_idTextAnchor1129}{}{}Exim
startup}{Exim startup}}\label{part0026_split_042.htmlux5cux23calibre_pb_41}}

On a mail hub machine, {exim} typically starts at boot time in daemon
mode and runs continuously, listening on port 25 and accepting messages
through SMTP. See
\protect\hyperlink{part0009_split_000.htmlux5cux23_idTextAnchor065}{Chapter
2, {Booting and System Management Daemons}}, for startup details for
your operating system.

Like {sendmail}, Exim can wear several hats, and if started with
specific flags or alternative command names, it performs different
functions. Exim's mode flags are similar to those understood by
{sendmail} because {exim} works hard to maintain compatibility when
called by user agents and other tools.
\protect\hyperlink{part0026_split_042.htmlux5cux23_idTextAnchor1130}{Table
18.17} lists a few common flags.

\paragraph[{Table 18.17: }Common {exim} command-line
flags]{\texorpdfstring{{Table 18.17:
}\protect\hypertarget{part0026_split_042.htmlux5cux23_idIndexMarker2620}{}{}\protect\hypertarget{part0026_split_042.htmlux5cux23_idTextAnchor1130}{}{}\protect\hypertarget{part0026_split_042.htmlux5cux23_idTextAnchor1131}{}{}Common
{exim} command-line
flags\protect\hypertarget{part0026_split_042.htmlux5cux23_idIndexMarker2621}{}{}\protect\hypertarget{part0026_split_042.htmlux5cux23_idIndexMarker2622}{}{}}{Table 18.17: Common exim command-line flags}}

%\includegraphics{images/00850.gif}

Any errors in the config file that can be detected at parse time are
caught by {exim -bV}, but some errors can only be caught at run time.
Misplaced braces are a common mistake.

The {exim} man page gives lots of detail on all the nooks and crannies
of {exim}'s command-line flags and options, including extensive
debugging information.

\protect\hypertarget{part0026_split_043.html}{}{}

\hypertarget{part0026_split_043.htmlux5cux23_idContainer1247}{}
\hypertarget{part0026_split_043.htmlux5cux23calibre_pb_42}{%
\subsection[Exim
utilities]{\texorpdfstring{\protect\hypertarget{part0026_split_043.htmlux5cux23_idTextAnchor1132}{}{}Exim
utilities}{Exim utilities}}\label{part0026_split_043.htmlux5cux23calibre_pb_42}}

\protect\hypertarget{part0026_split_043.htmlux5cux23_idIndexMarker2623}{}{}The
Exim distribution includes a bunch of utilities to help you monitor,
debug, and sanity-check your installation. Below is the current list
along with a brief description of each. See the documentation from the
distribution for more detail.

\begin{itemize}
\item
  \protect\hypertarget{part0026_split_043.htmlux5cux23_idIndexMarker2624}{}{}{exicyclog}
  -- rotates log files
\item
  \protect\hypertarget{part0026_split_043.htmlux5cux23_idIndexMarker2625}{}{}{exigrep}
  -- searches the main log
\item
  \protect\hypertarget{part0026_split_043.htmlux5cux23_idIndexMarker2626}{}{}{exilog}
  -- visualizes log files across multiple servers
\item
  \protect\hypertarget{part0026_split_043.htmlux5cux23_idIndexMarker2627}{}{}{exim\_checkaccess}
  -- checks address acceptance from a given IP address
\item
  \protect\hypertarget{part0026_split_043.htmlux5cux23_idIndexMarker2628}{}{}{exim\_dbmbuild}
  -- builds a DBM file
\item
  \protect\hypertarget{part0026_split_043.htmlux5cux23_idIndexMarker2629}{}{}{exim\_dumpdb}
  -- dumps a hints database
\item
  \protect\hypertarget{part0026_split_043.htmlux5cux23_idIndexMarker2630}{}{}{exim\_fixdb}
  -- patches a hints database
\item
  \protect\hypertarget{part0026_split_043.htmlux5cux23_idIndexMarker2631}{}{}{exim\_lock}
  -- locks a mailbox file
\item
  \protect\hypertarget{part0026_split_043.htmlux5cux23_idIndexMarker2632}{}{}{exim\_tidydb}
  -- cleans up a hints database
\item
  \protect\hypertarget{part0026_split_043.htmlux5cux23_idIndexMarker2633}{}{}{eximstats}
  -- extracts statistics from the log
\item
  \protect\hypertarget{part0026_split_043.htmlux5cux23_idIndexMarker2634}{}{}{exinext}
  -- extracts retry information
\item
  \protect\hypertarget{part0026_split_043.htmlux5cux23_idIndexMarker2635}{}{}{exipick}
  -- selects messages according to various criteria
\item
  \protect\hypertarget{part0026_split_043.htmlux5cux23_idIndexMarker2636}{}{}{exiqgrep}
  -- searches the queue
\item
  \protect\hypertarget{part0026_split_043.htmlux5cux23_idIndexMarker2637}{}{}{exiqsumm}
  -- summarizes the queue
\item
  \protect\hypertarget{part0026_split_043.htmlux5cux23_idIndexMarker2638}{}{}{exiwhat}
  -- lists what Exim processes are doing
\end{itemize}

Another utility that is part of the Exim suite is
\protect\hypertarget{part0026_split_043.htmlux5cux23_idIndexMarker2639}{}{}{eximon},
an X Windows application that displays Exim's state, the state of Exim's
queue, and the tail of the log file. As with the main distribution, you
build it by editing a well-commented {EDITME} file in the
{exim\_monitor} directory and running {make}. However, in the case of
{eximon} the defaults are usually fine, so you should not have to do
much configuration to build the application. Some configuration and
queue management can be done from the {eximon} GUI as well.

\protect\hypertarget{part0026_split_044.html}{}{}

\hypertarget{part0026_split_044.htmlux5cux23_idContainer1247}{}
\hypertarget{part0026_split_044.htmlux5cux23calibre_pb_43}{%
\subsection[Exim configuration
language]{\texorpdfstring{\protect\hypertarget{part0026_split_044.htmlux5cux23_idTextAnchor1133}{}{}\protect\hypertarget{part0026_split_044.htmlux5cux23_idIndexMarker2640}{}{}E\protect\hypertarget{part0026_split_044.htmlux5cux23_idTextAnchor1134}{}{}xim
configuration
language}{Exim configuration language}}\label{part0026_split_044.htmlux5cux23calibre_pb_43}}

The Exim configuration language (or more accurately, languages: one for
filters, one for regular expressions, etc.) feels a bit like the ancient
(1970s) language Forth. When first reading an Exim configuration, you
might find it hard to distinguish between keywords and option names
(which are fixed by Exim) and variable names (which are defined by
sysadmins through configuration statements).

Although Exim is advertised as being easy to configure and is
extensively documented, there can be quite a learning curve for new
users. The section ``How Exim receives and delivers mail'' in the
specification document is essential reading for newcomers. It gives a
good feel for the underlying concepts of the system.

When assigned a value, the Exim language's predefined options sometimes
cause an action. The values of about 120 predefined variables may also
change in response to an action. These variables can be included in
conditional statements.

The language for evaluating {if} statements and the like may remind you
of the reverse Polish notation used during the heyday of Hewlett-Packard
calculators. Let's look at a simple example. In the line

%\includegraphics{images/00851.gif}

the {acl\_smtp\_rcpt} option, when set, causes an ACL to be implemented
for each recipient (SMTP RCPT command) in the SMTP exchange. The value
assigned to this option is either {acl\_check\_rcpt} or
{acl\_check\_rcpt\_submit}, depending on whether or not the Exim
variable {\$interface\_port} has value 25.

We do not detail the Exim configuration language in this chapter, but
refer you instead to the extensive documentation. In particular, pay
close attention to the string expansion section of the Exim
specification.

\protect\hypertarget{part0026_split_045.html}{}{}

\hypertarget{part0026_split_045.htmlux5cux23_idContainer1247}{}
\hypertarget{part0026_split_045.htmlux5cux23calibre_pb_44}{%
\subsection[Exim configuration
file]{\texorpdfstring{\protect\hypertarget{part0026_split_045.htmlux5cux23_idTextAnchor1135}{}{}Exim
configuration
file}{Exim configuration file}}\label{part0026_split_045.htmlux5cux23calibre_pb_44}}

Exim's run-time behavior is controlled by a single configuration file,
usually called
\protect\hypertarget{part0026_split_045.htmlux5cux23_idIndexMarker2641}{}{}{/usr/exim/configure}.
Its name is one of the required variables specified in the {EDITME} file
and compiled into the binary.

The supplied default configuration file, {src/configure.default}, is
well commented and is a good starting place for sites just getting set
up with Exim. In fact, we recommend that you don't stray too far from it
until you thoroughly understand the Exim paradigm and need to elaborate
on the default configuration for a specific purpose. Exim works hard to
support common situations and has sensible defaults.

It's also helpful to stick with the variable names used in the default
config file. These naming conventions are assumed by folks on the
exim-users mailing list. Those people are also a good resource to
consult regarding your configuration questions.

{exim} prints a message to stderr and exits if you have a syntax error
in your configuration file. It doesn't catch all syntax errors
immediately, however, because it does not expand variables until it
needs to.

The order of entries in the configuration file is not quite arbitrary:
the global configuration options section must be first and must exist.
All other sections are optional and can appear in any order.

Possible sections include

\begin{itemize}
\item
  Global configuration options (mandatory)
\item
  {acl} -- access control lists that filter addresses and messages
\item
  {authenticators} -- for SMTP AUTH or TLS authentication
\item
  {routers} -- ordered sequence to determine where a message should go
\item
  {transports} -- definitions of the drivers that do the actual delivery
\item
  {retry} -- policy settings for dealing with problem messages
\item
  {rewrite} -- global address rewriting rules
\item
  {local\_scan} -- a hook for fancy flexibility
\end{itemize}

Each section except the first starts with a {begin}{ section-name}
statement---for example, {begin acl}. There is no {end}{ section-name}
statement; the end is signaled by the next section's {begin} statement.
Indentation to show subordination makes the config file easier to read
for humans, but it is not meaningful to Exim.

Some configuration statements name objects that will later be used to
control the flow of messages. Those names must begin with a letter and
contain only letters, numbers, and the underscore character. If the
first non-whitespace character on a line is {\#}, the rest of the line
is treated as a comment. Note that this means you cannot put a comment
on the same line as a statement; it will not be recognized as a comment
because the first character is not {\#}.

Exim lets you include files anywhere in the configuration file. Two
forms of include are used:

%\includegraphics{images/00852.gif}

The first form generates an error if the file does not exist. Although
include files keep your config file tidy, they are read several times
during the life of a message, so it might be best just to include their
contents directly into your configuration.

\protect\hypertarget{part0026_split_046.html}{}{}

\hypertarget{part0026_split_046.htmlux5cux23_idContainer1247}{}
\hypertarget{part0026_split_046.htmlux5cux23calibre_pb_45}{%
\subsection[Global
options]{\texorpdfstring{\protect\hypertarget{part0026_split_046.htmlux5cux23_idTextAnchor1136}{}{}Global
options}{Global options}}\label{part0026_split_046.htmlux5cux23calibre_pb_45}}

\protect\hypertarget{part0026_split_046.htmlux5cux23_idIndexMarker2642}{}{}Lots
of stuff is specified in the global options section, including operating
parameters (limits, sizes, timeouts, properties of the mail server on
this host), list definitions (local hosts, local hosts to relay for,
remote domains to relay for), and macros (hostname, contact, location,
error messages, SMTP banner).

\subsubsection[Options]{\texorpdfstring{\protect\hypertarget{part0026_split_046.htmlux5cux23_idTextAnchor1137}{}{}Options}{Options}}

\protect\hypertarget{part0026_split_046.htmlux5cux23_idIndexMarker2643}{}{}Options
are set with the basic syntax

%\includegraphics{images/00853.gif}

where the {values} can be Booleans, strings, integers, decimal numbers,
or time intervals. Multivalued options are allowed, in which case the
various values are separated by colons.

Use of the colon as a value separator presents a problem when you
express IPv6 addresses, which use colons as part of the address. You can
escape the colons by doubling them, but the easiest and most readable
fix is to redefine the separator character with the {\textless{}}
character as you assign values to the option. For example, both of the
following two lines set the value of the {localhost\_interfaces} option,
which contains the IPv4 and IPv6 localhost
addresses:\protect\hypertarget{part0026_split_046.htmlux5cux23_idIndexMarker2644}{}{}

%\includegraphics{images/00854.gif}

The second form, in which the semicolon has been defined as the
separator, is more readable and less fragile.

There are a zillion options---more than 500 in the options index of the
documentation. And we said {sendmail} was complicated! Most options have
sensible defaults, and all have descriptive names. It's handy to have a
copy of the {doc/spec.txt} file from the distribution in your favorite
text editor when you are researching a new option. We don't cover all
the options below, just the ones that occur in our example configuration
bits.

\subsubsection[Lists]{\texorpdfstring{\protect\hypertarget{part0026_split_046.htmlux5cux23_idTextAnchor1138}{}{}Lists}{Lists}}

\protect\hypertarget{part0026_split_046.htmlux5cux23_idIndexMarker2645}{}{}Exim
has four kinds of lists, introduced by the keywords {hostlist},
{domainlist}, {addresslist}, and {localpartslist}. Here are two examples
that use {hostlist}:

%\includegraphics{images/00855.gif}

Members can be listed in-line or taken from a file. If in-line, they are
separated by colons. There can be up to 16 named lists of each type. In
the in-line example above, we included all machines on a local /24
network and a specific hostname.

The symbol {@} can be a member of a list; it means the name of the local
host and helps you write a single generic configuration file that works
for most nonhub machines at your site. The notation {@{[}{]}} is also
useful and means all IP addresses on which Exim is listening; that is,
all the IP addresses of the local host.

Lists can include references to other lists and the {!} character to
indicate negation. Lists that include references to variables (e.g.,
{\$variable\_name}) make processing slower because Exim cannot cache the
results of evaluating the list, which it otherwise does by default.

To reference a list, just put {+} in front of its name to match members
of the list or {!+} to match nonmembers; for example,
{+my\_relay\_list}. Omit space between the {+} sign and the name of the
list.

\subsubsection[Macros]{\texorpdfstring{\protect\hypertarget{part0026_split_046.htmlux5cux23_idTextAnchor1139}{}{}Macros}{Macros}}

\protect\hypertarget{part0026_split_046.htmlux5cux23_idIndexMarker2646}{}{}You
can use macros to define parameters, error messages, etc. The parsing is
primitive, so you cannot define a macro whose name is a subset of
another macro without unpredictable results.

The syntax is

%\includegraphics{images/00856.gif}

For example, the first of the following lines defines a macro named
{ALIAS\_QUERY} that looks up a user's alias entry in a MySQL database.
The second line shows the use of the macro to perform an actual lookup,
with the result being stored in the variable called {data}.

%\includegraphics{images/00857.gif}

Macro names are not required to be all caps, but they must begin with a
capital letter. However, the all-caps convention aids clarity. The
configuration file can include {ifdef}s that evaluate a macro and use it
to determine whether or not to include a portion of the config file.
Every imaginable form of {ifdef} is supported; they all begin with a
dot.

\protect\hypertarget{part0026_split_047.html}{}{}

\hypertarget{part0026_split_047.htmlux5cux23_idContainer1247}{}
\hypertarget{part0026_split_047.htmlux5cux23calibre_pb_46}{%
\subsection[Access control lists (ACLs)]{\texorpdfstring{Access control
lists
(\protect\hypertarget{part0026_split_047.htmlux5cux23_idTextAnchor1140}{}{}ACLs)}{Access control lists (ACLs)}}\label{part0026_split_047.htmlux5cux23calibre_pb_46}}

\protect\hypertarget{part0026_split_047.htmlux5cux23_idIndexMarker2647}{}{}\protect\hypertarget{part0026_split_047.htmlux5cux23_idIndexMarker2648}{}{}Access
control lists filter the addresses of incoming messages and either
accept or deny them. Exim divides incoming addresses into a local part
that represents the user and a domain part that is the recipient's
domain.

ACLs can be applied at any of the various stages of an SMTP
conversation: HELO, MAIL, RCPT, DATA, etc. Typically, an ACL enforces
strict adherence to the SMTP protocol at the HELO stage, checks the
sender and the sender's domain at the MAIL stage, checks the recipients
at the RCPT stage, and scans the message content at the DATA stage.

A slew of options named {acl\_smtp\_}{command} specify which ACL should
be applied after each {command} in the SMTP protocol. For example, the
{acl\_smtp\_rcpt} option directs the ACL to run on each address that is
a recipient of the message. Another commonly used checkpoint is
{acl\_smtp\_data}, which checks the ACL against the message after it has
been received, for example, to scan content.

You can define ACLs in the {acl} section of the config file, in a file
that is referenced by the {acl\_smtp\_}{command} option or in-line when
the option is defined.

A sample ACL called {my\_acl\_check\_rcpt} is defined below. We would
invoke it by assigning its name to the {acl\_smtp\_rcpt} option in the
global options section of the config file. (If this ACL denies an
address at the level of the RCPT command, the sending server should give
up and not try the address again.)

This is a long ACL specification, so we break it up into digestible
pieces that we can decode individually.

The first portion:

%\includegraphics{images/00858.gif}

The default name for this access control list is {acl\_check\_rcpt}; you
probably should not change its name as we did here. We used a
nonstandard name simply to emphasize that the name is something you
specify, not a keyword that's special to Exim.

The first {accept} line, containing just a colon, is an empty list. The
empty list of remote hosts matches cases in which a local MUA submitted
a message on the MTA's standard input. If the address being tested meets
this condition, the ACL accepts the address and disables DKIM signature
validation, which is turned on by default. If the address does not match
this {address} clause, control drops through to the next clause in the
ACL definition:

%\includegraphics{images/00859.gif}

The first {deny} stanza is intended for messages coming into your local
domains. It rejects any address whose local part (the username) starts
with a dot or contains the special characters {@}, {\%}, {!}, {/}, or
{\textbar{}}. The second {deny} applies to messages being sent out by
your users. It, too, disallows certain special characters and sequences
in the local parts of addresses, in case your users' machines have been
infected with
\protect\hypertarget{part0026_split_047.htmlux5cux23_idIndexMarker2649}{}{}\protect\hypertarget{part0026_split_047.htmlux5cux23_idIndexMarker2650}{}{}a
virus or other malware. In the past, such addresses have been used by
spammers to confuse ACLs or have been associated with other security
problems.

In general, if you are intending to use {\$local\_parts} (supposedly,
the recipient's username) in a directory path (to store mail or look for
a vacation file, for example) be careful that your ACLs have filtered
out any special characters that could cause unwanted behavior. (The
example looks for the sequence /../, which could be problematic if the
username were inserted into a path.)

%\includegraphics{images/00860.gif}

This {accept} stanza guarantees that mail to postmaster always gets
through if it's sent to a local domain, and that can help with
debugging.

%\includegraphics{images/00861.gif}

The {require} line checks to see if a bounce message can be returned;
however, it checks only the sender's domain. ({require} means ``{deny}
if not matched.'') If the sender's username has been forged, a bounce
message could still fail; that is, the bounce message itself could
bounce.You can add more extensive checking here by calling another
program, but some
\protect\hypertarget{part0026_split_047.htmlux5cux23_idIndexMarker2651}{}{}sites
consider such callouts abusive and might add your mail server to a
blacklist or bad-reputation list.

%\includegraphics{images/00862.gif}

The above {accept} stanza checks for hosts that are allowed to relay
through this host, namely, local hosts that are submitting mail into the
system. The {control} line specifies that Exim should act as a mail
submission agent and fix up any header deficiencies as the message
arrives from the user agent. The recipient's address is not checked
because many user agents are confused by error returns. (This part of
the configuration is appropriate only for local machines that relay to a
smart host, not for any external domains you might be willing to relay
for.) DKIM verification is disabled because these messages are outbound
from your users or relay friends.

%\includegraphics{images/00863.gif}

The last {accept} stanza deals with local hosts that authenticate
through SMTP AUTH. Once again, these messages are treated as submissions
from user agents.

%\includegraphics{images/00864.gif}

Here, we check the destination domain to which the message is headed and
require that it be either in our list of {local\_domains} or in our list
of domains to which we allow relaying, {relay\_to\_domains}. (These
domain lists are defined outside the context of the ACL.) Any
destinations not in one of those lists are refused with a customized
error message.

%\includegraphics{images/00865.gif}

Finally, given that all previous requirements have been met but that no
more-{specific} {accept} or {deny} rule has been triggered, we verify
the recipient and accept the message. Most Internet messages to local
users fall into this category.

\protect\hypertarget{part0026_split_047.htmlux5cux23_idTextAnchor1141}{}{}We
haven't included any blacklist scanning in the example above. To access
a blacklist, use one of the examples in the default config file or
something like this:

%\includegraphics{images/00866.gif}

Translated to English, the code specifies that if a message matches
{all} of the following criteria, it is rejected with a custom error
message and logged (also with a custom message):

\begin{itemize}
\item
  It's from an IPv4 address (some lists don't handle IPv6 correctly).
\item
  It's not associated with an authenticated SMTP session.
\item
  It's from a sender not in the local whitelist.
\item
  It's from a sender not in the global (Internet) whitelist.
\item
  It's addressed to a valid local recipient.
\item
  The sending host is on the zen.spamhaus.org blacklist.
\end{itemize}

The variables {dnslist\_text} and {dnslist\_domain} are set by the
assignment to {dnslists}, which triggers the blacklist lookup. This
{deny} clause could be placed right after your checks for unusual
characters in addresses.

Here's another example ACL that rejects mail if the remote side does not
say HELO properly:

%\includegraphics{images/00867.gif}

Exim solves the early talker problem (a more specific case of ``not
saying HELO properly'') with the {smtp\_enforce\_sync} option, which is
turned on by default.

\protect\hypertarget{part0026_split_048.html}{}{}

\hypertarget{part0026_split_048.htmlux5cux23_idContainer1247}{}
\hypertarget{part0026_split_048.htmlux5cux23calibre_pb_47}{%
\subsection[Content scanning at ACL
time]{\texorpdfstring{\protect\hypertarget{part0026_split_048.htmlux5cux23_idTextAnchor1142}{}{}Con\protect\hypertarget{part0026_split_048.htmlux5cux23_idTextAnchor1143}{}{}tent
scanning at ACL
time}{Content scanning at ACL time}}\label{part0026_split_048.htmlux5cux23calibre_pb_47}}

Exim supports powerful content scanning at several points in a message's
traversal of the mail system: at ACL time (after the SMTP DATA command);
at delivery time through the {transport\_filter} option; or with a
{local\_scan} function after all ACL checks have been completed. You
must compile support for content scanning into Exim by setting the
{WITH\_CONTENT\_SCAN} variable in the {EDITME} file; it is commented out
by default. This option endows ACLs with extra power and flexibility and
adds two new configuration options: {spamd\_address} and {av\_scanner}.

Scanning at ACL time allows a message to be rejected in-line with the
MTA's conversation with the sending host. The message is never accepted
for delivery, so it need not be bounced. This way of rejecting the
message is nice because it avoids backscatter spam caused by bounce
messages to forged sender addresses.

\protect\hypertarget{part0026_split_049.html}{}{}

\hypertarget{part0026_split_049.htmlux5cux23_idContainer1247}{}
\hypertarget{part0026_split_049.htmlux5cux23calibre_pb_48}{%
\subsection[Authenticators]{\texorpdfstring{\protect\hypertarget{part0026_split_049.htmlux5cux23_idTextAnchor1144}{}{}Auth\protect\hypertarget{part0026_split_049.htmlux5cux23_idTextAnchor1145}{}{}enticators}{Authenticators}}\label{part0026_split_049.htmlux5cux23calibre_pb_48}}

\protect\hypertarget{part0026_split_049.htmlux5cux23_idIndexMarker2652}{}{}Authenticators
are drivers that interact with the SMTP AUTH command's
challenge-and-response sequence and identify an authentication mechanism
acceptable to both client and server. Exim supports the following
mechanisms:

\begin{itemize}
\item
  {AUTH\_CRAM\_MD5} (RFC2195)
\item
  {AUTH\_PLAINTEXT}, which includes both PLAIN and LOGIN
\item
  {AUTH\_SPA}, which supports Microsoft's Secure Password Authentication
\end{itemize}

If Exim is receiving email, it is acting as an SMTP AUTH server. If it
is sending mail, it is a client. Options that appear in the definitions
of authenticator instances are tagged with a prefix of either {server\_}
or {client\_} to allow for different configurations depending on the
role Exim is playing.

Authenticators are used in access control lists, as in the following
clause in the ACL example from
\protect\hyperlink{part0026_split_047.htmlux5cux23_idTextAnchor1141}{this
page}:

%\includegraphics{images/00868.gif}

Below is an example that shows both the client-side and server-side
LOGIN mechanisms. This simple example uses a fixed username and
password, which is OK for small sites but probably inadvisable for
larger installations.

%\includegraphics{images/00869.gif}

\protect\hypertarget{part0026_split_049.htmlux5cux23_idIndexMarker2653}{}{}Authentication
data can come from many sources: LDAP, PAM, {/etc/passwd}, etc. The
{server\_advertise\_condition} clause above prevents mail clients from
sending passwords in the clear by requiring TLS security (through
STARTTLS or SSL) on connection. If you want the same behavior when Exim
acts as the client system, use the {client\_condition} option in the
client clause, too, again with {tis\_cipher}.

Refer to the Exim documentation for details of all possible
authentication options and for examples.

\protect\hypertarget{part0026_split_050.html}{}{}

\hypertarget{part0026_split_050.htmlux5cux23_idContainer1247}{}
\hypertarget{part0026_split_050.htmlux5cux23calibre_pb_49}{%
\subsection[Routers]{\texorpdfstring{R\protect\hypertarget{part0026_split_050.htmlux5cux23_idTextAnchor1146}{}{}outers}{Routers}}\label{part0026_split_050.htmlux5cux23calibre_pb_49}}

\protect\hypertarget{part0026_split_050.htmlux5cux23_idIndexMarker2654}{}{}Routers
work on recipient email addresses, either by rewriting them or by
assigning them to a transport and sending them on their way. A
particular router can have multiple instances, each with different
options.

You specify a sequence of routers. A message starts with the first
router and progresses through the list until the message is either
accepted or rejected. The accepting router typically hands the message
to a transport driver. Routers handle both incoming and outgoing
messages. They feel a bit like subroutines in a programming language.

A router can return any of the dispositions shown in
\protect\hyperlink{part0026_split_050.htmlux5cux23_idTextAnchor1147}{Table
18.18} for a message.

\paragraph[{Table 18.18: }Exim router statuses]{\texorpdfstring{{Table
18.18:
}\protect\hypertarget{part0026_split_050.htmlux5cux23_idTextAnchor1147}{}{}Exim
router statuses}{Table 18.18: Exim router statuses}}

%\includegraphics{images/00870.gif}

If a message receives a {pass} or {decline} from all the routers in the
sequence, it is unroutable. Exim bounces or rejects such messages,
depending on the context.

If a message meets the preconditions for a router and the router ends
with a {no\_more} statement, then that message will not be presented to
any additional routers, regardless of its disposition by the current
router. For example, if your remote SMTP router has the precondition
{domains = !+local\_domains} and has {no\_more} set, then only messages
to local users (that is, those that would fail the {domains}
precondition) will continue to the next router in the sequence.

Routers have many possible options; some common examples are
preconditions, acceptance or failure conditions, error messages to
return, and transport drivers to use.

The next few sections detail the routers called {accept}, {dnslookup},
{manualroute}, and {redirect}. The example configuration snippets assume
that Exim is running on a local machine in the example.com domain.
They're all pretty straightforward; refer to the documentation if you
want to use some of the fancier routers.

\subsubsection[The {accept}
router]{\texorpdfstring{\protect\hypertarget{part0026_split_050.htmlux5cux23_idTextAnchor1148}{}{}The
{accept} router}{The accept router}}

\protect\hypertarget{part0026_split_050.htmlux5cux23_idTextAnchor1149}{}{}The
\protect\hypertarget{part0026_split_050.htmlux5cux23_idIndexMarker2655}{}{}{accept}
router labels an address as OK and passes the associated message to a
transport driver. Below are examples of {accept} router instances called
{localusers}, for delivering local mail, and {save\_to\_file}, for
appending to an archive.

%\includegraphics{images/00871.gif}

The {localusers} router instance checks that the domain part of the
destination address is example.com and that the local part of the
address is the login name of a local user. If both conditions are met,
the router hands the message to the transport driver instance called
{my\_local\_delivery}, which is defined in the {transports} section. The
{save\_to\_file} instance is designed for dial-up users; it appends the
message to a file specified in the {batchsmtp\_appendfile} transport
definition.

\subsubsection[The {dnslookup}
router]{\texorpdfstring{\protect\hypertarget{part0026_split_050.htmlux5cux23_idTextAnchor1150}{}{}The
{dnslookup} router}{The dnslookup router}}

The
\protect\hypertarget{part0026_split_050.htmlux5cux23_idIndexMarker2656}{}{}{dnslookup}
router typically handles outgoing messages. It looks up the MX record of
the recipient's domain and hands the message to an SMTP transport driver
for delivery. Here is an instance called {remoteusers}:

%\includegraphics{images/00872.gif}

\leavevmode\hypertarget{part0026_split_050.htmlux5cux23_idContainer1200}{}%
See
\protect\hyperlink{part0021_split_021.htmlux5cux23_idTextAnchor657}{this
page} for more information about RFC1918 private address spaces.

The {dnslookup} code looks up the
\protect\hypertarget{part0026_split_050.htmlux5cux23_idIndexMarker2657}{}{}MX
records for the addressee. If no MX records exist, it tries the A
record. A common extension to this router instance prohibits delivery to
certain IP addresses; a prime example is the RFC1918 private addresses
that cannot be routed on the Internet. See the {ignore\_target\_hosts}
option for more information.

\subsubsection[The {manualroute}
router]{\texorpdfstring{\protect\hypertarget{part0026_split_050.htmlux5cux23_idTextAnchor1151}{}{}The
{manualroute} router}{The manualroute router}}

The flexible
\protect\hypertarget{part0026_split_050.htmlux5cux23_idIndexMarker2658}{}{}{manualroute}
driver can pretty much route email in whatever way you want. The routing
information can be a table of rules that match by recipient domain
({route\_list}) or a single rule that applies to all domains
({route\_data}).

Below are two examples of {manualroute} instances. The first example
implements the ``smart host'' concept, in which all outgoing nonlocal
mail is sent to a central (``smart'') host for processing. This instance
is called {smarthost} and applies to all recipients' domains that are
not (the {!} character) in the {local\_domains} list.

%\includegraphics{images/00873.gif}

The router instance below, {firewall}, speaks SMTP to send incoming
messages to hosts inside the firewall (perhaps after scanning them for
spam and viruses). It looks up the routing data for each recipient
domain in a DBM database that contains the names of local hosts.

%\includegraphics{images/00874.gif}

\subsubsection[The {redirect}
router]{\texorpdfstring{\protect\hypertarget{part0026_split_050.htmlux5cux23_idTextAnchor1152}{}{}The
{redirect} router}{The redirect router}}

The
\protect\hypertarget{part0026_split_050.htmlux5cux23_idIndexMarker2659}{}{}{redirect}
driver does address rewriting, such as that called for in the
system-wide {aliases} file or in a user's {\textasciitilde/.forward}
file. It usually does not assign the rewritten address to a transport;
that task is left to other routers in the chain.

The first instance shown below, {system\_aliases}, looks up aliases with
a linear search ({lsearch}) of the {/etc/aliases} file. That's fine for
a small {aliases} file, but if
\protect\hypertarget{part0026_split_050.htmlux5cux23_idIndexMarker2660}{}{}yours
is huge, replace that linear search with a database lookup. The second
instance, {user\_forward}, first verifies that mail is addressed to a
local user, then checks that user's {.forward} file.

%\includegraphics{images/00875.gif}

The {check\_local\_user} option ensures that the recipient is a valid
local user. The {no\_verify} says not to verify the validity of the
address to which the forward file redirects the message; just ship it.

\subsubsection[Per-user filtering through {.forward}
files]{\texorpdfstring{\protect\hypertarget{part0026_split_050.htmlux5cux23_idTextAnchor1153}{}{}Per-user
filtering through {.forward}
files}{Per-user filtering through .forward files}}

Exim not only allows forwarding through {.forward} files but also allows
filtering. It supports its own filtering system as well as the Sieve
filtering that is being standardized by the IETF. If the first line of a
user's {.forward} file is

%\includegraphics{images/00876.gif}

or

%\includegraphics{images/00877.gif}

then the subsequent filtering commands (there are about 15 of them) can
determine where the message should be delivered. Filtering does not
actually deliver messages---it just meddles with the destination. For
example:

%\includegraphics{images/00878.gif}

Lots of options are available to control what users can and cannot do in
their
\protect\hypertarget{part0026_split_050.htmlux5cux23_idIndexMarker2661}{}{}{.forward}
files. The option names begin with {forbid\_} or {allow\_}. They're
important because they can prevent users from running shells, loading
libraries into binaries, or accessing the embedded Perl interpreter when
they shouldn't. Check for new {forbid\_*} options when you upgrade to be
sure your users can't get too fancy in their {.forward} files.

\protect\hypertarget{part0026_split_051.html}{}{}

\hypertarget{part0026_split_051.htmlux5cux23_idContainer1247}{}
\hypertarget{part0026_split_051.htmlux5cux23calibre_pb_50}{%
\subsection[Transports]{\texorpdfstring{\protect\hypertarget{part0026_split_051.htmlux5cux23_idTextAnchor1154}{}{}Transports}{Transports}}\label{part0026_split_051.htmlux5cux23calibre_pb_50}}

\protect\hypertarget{part0026_split_051.htmlux5cux23_idIndexMarker2662}{}{}Rou\protect\hypertarget{part0026_split_051.htmlux5cux23_idTextAnchor1155}{}{}ters
decide where messages should go, and transports actually take them
there. Local transports typically append to a file, pipe to a local
program, or speak the LMTP protocol to an IMAP server. Remote transports
speak SMTP to their counterparts across the Internet.

There are five Exim transports: {appendfile}, {lmtp}, {smtp},
{autoreply}, and {pipe}; we detail {appendfile} and {smtp}. The
\protect\hypertarget{part0026_split_051.htmlux5cux23_idIndexMarker2663}{}{}{autoreply}
transport is typically used to send vacation messages, and the {pipe}
transport hands messages as input to a command through a UNIX pipe. As
with routers, you must define instances of transports, and it's OK to
have multiple instances of the same type of transport. Order is
significant for routers, but not for transports.

\subsubsection[The {appendfile}
transport]{\texorpdfstring{\protect\hypertarget{part0026_split_051.htmlux5cux23_idTextAnchor1156}{}{}The
\protect\hypertarget{part0026_split_051.htmlux5cux23_idIndexMarker2664}{}{}{appendfile}
transport}{The appendfile transport}}

The {appendfile} driver stores messages in {mbox}, {mbx}, {Maildir}, or
{mailstore} format in a specified file or directory. You must have
included the appropriate mailbox formats when you compiled Exim; they
are commented out of the {EDITME} file by default.

The following example defines the {my\_local\_delivery} transport (an
instance of the {appendfile} transport{) }referred to in the
{localusers} router instance definition on
\protect\hyperlink{part0026_split_050.htmlux5cux23_idTextAnchor1149}{this
page}.\protect\hypertarget{part0026_split_051.htmlux5cux23_idIndexMarker2665}{}{}

%\includegraphics{images/00879.gif}

The various *{\_add} lines add headers to the message. The {group} and
{mode} clauses ensure that the transport agent can write to the file.

\subsubsection[The {smtp}
transport]{\texorpdfstring{\protect\hypertarget{part0026_split_051.htmlux5cux23_idTextAnchor1157}{}{}The
\protect\hypertarget{part0026_split_051.htmlux5cux23_idIndexMarker2666}{}{}{smtp}
transport}{The smtp transport}}

The {smtp} transport is the workhorse of any mail system. Here, we
define two instances, one for the standard SMTP port (25) and one for
the mail submission port
(587).\protect\hypertarget{part0026_split_051.htmlux5cux23_idIndexMarker2667}{}{}

%\includegraphics{images/00880.gif}

The second instance, {my\_remote\_delivery\_port587}, specifies the port
and also a header to be added to the message that includes an indication
of the outgoing port. {MACRO\_HEADER} would be defined elsewhere in the
configuration file.

\protect\hypertarget{part0026_split_052.html}{}{}

\hypertarget{part0026_split_052.htmlux5cux23_idContainer1247}{}
\hypertarget{part0026_split_052.htmlux5cux23calibre_pb_51}{%
\subsection[Retry
configuration]{\texorpdfstring{\protect\hypertarget{part0026_split_052.htmlux5cux23_idTextAnchor1158}{}{}Retry
configuration}{Retry configuration}}\label{part0026_split_052.htmlux5cux23calibre_pb_51}}

\protect\hypertarget{part0026_split_052.htmlux5cux23_idIndexMarker2668}{}{}The
{retry} section of the configuration file must exist or Exim will never
attempt redelivery of messages that could not be delivered on the first
attempt. You can specify three time intervals, each less frequent than
the previous one. After the last interval has expired, messages bounce
back to the sender as undeliverable. {retry} statements understand the
suffixes {m}, {h}, {d}, and {w} to indicate minutes, hours, days, and
weeks. You can specify different intervals for different hosts or
domains.

Here's what a {retry} section looks like:

%\includegraphics{images/00881.gif}

This example means, ``For any domain, an address that fails temporarily
should be retried every 15 minutes for 2 hours, then every hour for the
next 24 hours, then every 6 hours for 4 days, and finally, bounced as
undeliverable.''

\protect\hypertarget{part0026_split_053.html}{}{}

\hypertarget{part0026_split_053.htmlux5cux23_idContainer1247}{}
\hypertarget{part0026_split_053.htmlux5cux23calibre_pb_52}{%
\subsection[Rewriting
configuration]{\texorpdfstring{\protect\hypertarget{part0026_split_053.htmlux5cux23_idTextAnchor1159}{}{}Rewriting
configuration}{Rewriting configuration}}\label{part0026_split_053.htmlux5cux23calibre_pb_52}}

\protect\hypertarget{part0026_split_053.htmlux5cux23_idIndexMarker2669}{}{}The
rewriting section of the configuration file starts with {begin rewrite}.
It's used to fix up addresses, not to reroute messages. For example, you
could use it on your outgoing addresses

\begin{itemize}
\item
  To make mail appear to be from your domain, not from individual hosts
\item
  To map usernames to a standard format such as First.Last
\end{itemize}

Do not apply rewriting to addresses in incoming mail.

\protect\hypertarget{part0026_split_054.html}{}{}

\hypertarget{part0026_split_054.htmlux5cux23_idContainer1247}{}
\hypertarget{part0026_split_054.htmlux5cux23calibre_pb_53}{%
\subsection[Local scan
function]{\texorpdfstring{\protect\hypertarget{part0026_split_054.htmlux5cux23_idTextAnchor1160}{}{}Local
scan
function}{Local scan function}}\label{part0026_split_054.htmlux5cux23calibre_pb_53}}

To further customize Exim, for example, to filter for the latest and
greatest virus, you could write a C function that does your scanning and
install it in the {local\_scan} section of the config file. Refer to the
Exim documentation for details and examples of how to do this.

\protect\hypertarget{part0026_split_055.html}{}{}

\hypertarget{part0026_split_055.htmlux5cux23_idContainer1247}{}
\hypertarget{part0026_split_055.htmlux5cux23calibre_pb_54}{%
\subsection[Logging]{\texorpdfstring{\protect\hypertarget{part0026_split_055.htmlux5cux23_idTextAnchor1161}{}{}Logging}{Logging}}\label{part0026_split_055.htmlux5cux23calibre_pb_54}}

\protect\hypertarget{part0026_split_055.htmlux5cux23_idIndexMarker2670}{}{}Exim
by default writes three different log files: a main log, a reject log,
and a panic log. Each log entry includes the time the message was
written. You specify the location of the log files in the {EDITME} file
(before building Exim) or in the run-time config file in the value of
the {log\_file\_path} option. By default, logs are kept in the
\protect\hypertarget{part0026_split_055.htmlux5cux23_idIndexMarker2671}{}{}{/var/spool/exim/log}
directory.

The {log\_file\_path} option accepts up to two colon-separated values.
Each value must be either the keyword {syslog} or an absolute path with
a {\%s} embedded where the names {main}, {reject}, and {panic} can be
substituted. For
example,\protect\hypertarget{part0026_split_055.htmlux5cux23_idIndexMarker2672}{}{}

%\includegraphics{images/00882.gif}

would log both to syslog (with facility ``mail'') and to the separate
files {exim\_main}, {exim\_reject}, and {exim\_panic} in the {/var/log}
directory. Exim submits the {main} log entries to syslog at priority
info, the {reject} entries at priority notice, and the {panic} entries
at priority alert.

The {main} log contains one line for the arrival and delivery of each
message. It can be summarized by the Perl script {eximstats}, which is
included in the Exim distribution.

The {reject} log records information about messages that have been
rejected for policy reasons: malware, spam, etc. It includes the summary
line for the message from the {main} log and also the original headers
of the message that was rejected. If you change your policies, check the
{reject} log to make sure that all is still well.

The
\protect\hypertarget{part0026_split_055.htmlux5cux23_idIndexMarker2673}{}{}{panic}
log is for serious errors in the software; {exim} writes here just
before it gives up. The {panic} log should not exist in the absence of
problems. Ask {cron} to check it for you and if it exists, fix the
problem that caused the panic and then delete the file. {exim} will
re-create it when the next panic-worthy situation arises.

When debugging, you can increase the amount and type of data logged.
Invoke the {log\_selector} option. For
example:\protect\hypertarget{part0026_split_055.htmlux5cux23_idIndexMarker2674}{}{}

%\includegraphics{images/00883.gif}

The logging categories that can be included or excluded by the
{log\_selector} mechanism are listed in the Exim specification, in the
section called ``Log files'' toward the end. About 35 categories are
defined, including {+all}, which will really fill your disks!

{exim} also keeps a temporary log for each message it handles. It is
named with the message ID and lives in
\protect\hypertarget{part0026_split_055.htmlux5cux23_idIndexMarker2675}{}{}{/var/spool/exim/msglog}.
If you are having trouble with a particular destination, check there.

\protect\hypertarget{part0026_split_056.html}{}{}

\hypertarget{part0026_split_056.htmlux5cux23_idContainer1247}{}
\hypertarget{part0026_split_056.htmlux5cux23calibre_pb_55}{%
\subsection[Debugging]{\texorpdfstring{\protect\hypertarget{part0026_split_056.htmlux5cux23_idTextAnchor1162}{}{}Debugging}{Debugging}}\label{part0026_split_056.htmlux5cux23calibre_pb_55}}

\protect\hypertarget{part0026_split_056.htmlux5cux23_idIndexMarker2676}{}{}\protect\hypertarget{part0026_split_056.htmlux5cux23_idIndexMarker2677}{}{}Exim
has powerful debugging aids. You can configure the amount of information
you want to see about each potential debugging topic. {exim -d} tells
{exim} to go into debugging mode, in which it stays in the foreground
and does not detach from the terminal. You can add specific debugging
categories to {-d} with a {+} or {-} in front of them to verbosify or
eliminate a category. For example, {-d+expand+acl} requests regular
debugging output plus extra details regarding string expansions and ACL
interpretation. (These two categories are common problem spots.) You can
tune more than 30 categories of debugging information; see the man page
for a list.

A common technique when debugging mail systems is to start the MTA on a
nonstandard port and then talk to it through {telnet}. For example, to
start {exim} in daemon mode, listening on port 26, with debugging info
turned on, run

%\includegraphics{images/00884.gif}

You can then {telnet} to port 26 and type SMTP commands in an attempt to
reproduce the problem you are debugging.

Alternatively, you can have {swaks} do your SMTP talking for you. It's a
Perl script that makes SMTP debugging faster and easier. {swaks
-\/-help} gets you some documentation, and
\href{http://jetmore.org/john/code/swaks}{jetmore.org/john/code/swaks}
supplies complete details.

If your log files show timeouts of around 30 seconds, that's suggestive
of a DNS issue.


\section{Postfix}

\protect\hypertarget{part0026_split_057.htmlux5cux23_idIndexMarker2678}{}{}Postfix
is another popular alternative to {sendmail}.
\protect\hypertarget{part0026_split_057.htmlux5cux23_idIndexMarker2679}{}{}Wietse
Venema started the Postfix project when he spent a sabbatical year at
\protect\hypertarget{part0026_split_057.htmlux5cux23_idIndexMarker2680}{}{}IBM's
T. J. Watson Research Center in 1996, and he is still actively
developing it. Postfix's design goals included not only security (first
and foremost!), but also an open source distribution policy, speedy
performance, robustness, and flexibility. All major Linux distributions
include Postfix, and since version 10.3, macOS has shipped Postfix
instead of {sendmail} as its default mail system.

\leavevmode\hypertarget{part0026_split_057.htmlux5cux23_idContainer1213}{}%
See
\protect\hyperlink{part0014_split_023.htmlux5cux23_idTextAnchor367}{this
page} for more information about regular expressions.

The most important things to know about Postfix are, first, that it
works almost out of the box (the simplest config files are only a line
or two long), and second, that it leverages regular expression maps to
filter email effectively, especially in conjunction with the
\protect\hypertarget{part0026_split_057.htmlux5cux23_idIndexMarker2681}{}{}PCRE
(Perl-Compatible Regular Expression) library. Postfix is compatible with
{sendmail} in the sense that Postfix's {aliases} and {.forward} files
have the same format and semantics as those of {sendmail}.

Postfix speaks ESMTP. Virtual domains and spam filtering are both
supported. For address rewriting, Postfix relies on table lookups from
flat files, Berkeley DB, DBM, LDAP, NetInfo, or SQL databases.

\protect\hypertarget{part0026_split_058.html}{}{}

\hypertarget{part0026_split_058.htmlux5cux23_idContainer1247}{}
\hypertarget{part0026_split_058.htmlux5cux23calibre_pb_57}{%
\subsection[Postfix
architecture]{\texorpdfstring{\protect\hypertarget{part0026_split_058.htmlux5cux23_idTextAnchor1165}{}{}Postfix
architecture}{Postfix architecture}}\label{part0026_split_058.htmlux5cux23calibre_pb_57}}

\protect\hypertarget{part0026_split_058.htmlux5cux23_idIndexMarker2682}{}{}Postfix
comprises several small, cooperating programs that send network
messages, receive messages, deliver email locally, etc. Communication
among them is performed through local domain sockets or FIFOs. This
architecture is quite different from that of {sendmail} and Exim,
wherein a single large program does most of the work.

The {master} program starts and monitors all Postfix processes. Its
configuration file, {master.cf}, lists the subsidiary programs along
with information about how they should be started. The default values
set in that file cover most needs; in general, no tweaking is necessary.
One common change is to comment out a program, for example, {smtpd},
when a client should not listen on the SMTP port.

The most important server programs involved in the delivery of email are
shown in
\protect\hyperlink{part0026_split_058.htmlux5cux23_idTextAnchor1166}{Exhibit
B}.

\paragraph[{Exhibit B: }Postfix server
programs]{\texorpdfstring{{Exhibit B:
}\protect\hypertarget{part0026_split_058.htmlux5cux23_idIndexMarker2683}{}{}\protect\hypertarget{part0026_split_058.htmlux5cux23_idTextAnchor1166}{}{}Postfix
server programs}{Exhibit B: Postfix server programs}}

%\includegraphics{images/00885.jpeg}

\subsubsection[Receiving
mail]{\texorpdfstring{\protect\hypertarget{part0026_split_058.htmlux5cux23_idTextAnchor1167}{}{}Receiving
mail}{Receiving mail}}

\protect\hypertarget{part0026_split_058.htmlux5cux23_idIndexMarker2684}{}{}\protect\hypertarget{part0026_split_058.htmlux5cux23_idIndexMarker2685}{}{}{smtpd}
receives mail entering the system through SMTP. It also verifies that
the connecting clients are authorized to send the mail they are trying
to deliver. When email is sent locally through the {/usr/lib/sendmail}
compatibility program, a file is written to the
\protect\hypertarget{part0026_split_058.htmlux5cux23_idIndexMarker2686}{}{}{/var/spool/postfix/maildrop}
directory. That directory is periodically scanned by the
\protect\hypertarget{part0026_split_058.htmlux5cux23_idIndexMarker2687}{}{}{pickup}
program, which processes any new files it finds.

All incoming email passes through
\protect\hypertarget{part0026_split_058.htmlux5cux23_idIndexMarker2688}{}{}{cleanup},
which adds missing headers and rewrites addresses according to the
{canonical} and {virtual} maps. Before inserting mail into the
{incoming} queue, {cleanup} passes it through {trivial-rewrite}, which
does minor fixing of the addresses, such as appending a mail domain to
addresses that are not fully qualified.

\subsubsection[Managing mail-waiting
queues]{\texorpdfstring{\protect\hypertarget{part0026_split_058.htmlux5cux23_idTextAnchor1168}{}{}Managing
mail-waiting queues}{Managing mail-waiting queues}}

\protect\hypertarget{part0026_split_058.htmlux5cux23_idIndexMarker2689}{}{}\protect\hypertarget{part0026_split_058.htmlux5cux23_idIndexMarker2690}{}{}{qmgr}
manages five queues that contain mail waiting to be delivered:

\begin{itemize}
\item
  {incoming} -- mail that is arriving
\item
  {active} -- mail that is being delivered
\item
  {deferred} -- mail for which delivery has failed in the past
\item
  {hold} -- mail blocked in the queue by the administrator
\item
  {corrupt} -- mail that can't be read or parsed
\end{itemize}

The queue manager generally follows a simple FIFO strategy to select the
next message to process, but it also supports a complex preemption
algorithm that prefers messages with few recipients over bulk mail.

To avoid overwhelming a receiving host, especially one that has been
down, Postfix uses a slow-start algorithm to control how fast it tries
to deliver email. Deferred messages are given a try-again time stamp
that exponentially backs off so as not to waste resources on
undeliverable messages. A status cache of unreachable destinations
avoids unnecessary delivery attempts.

\subsubsection[Sending
mail]{\texorpdfstring{\protect\hypertarget{part0026_split_058.htmlux5cux23_idTextAnchor1169}{}{}Sending
mail}{Sending mail}}

\protect\hypertarget{part0026_split_058.htmlux5cux23_idIndexMarker2691}{}{}{qmgr},
aided by {trivial-rewrite}, decides where a message should be sent. The
routing decision made by {trivial-rewrite }can be overridden through
lookup tables ({transport\_maps}).

Delivery to remote hosts via the SMTP protocol is performed by the
{smtp} program.
\protect\hypertarget{part0026_split_058.htmlux5cux23_idIndexMarker2692}{}{}{lmtp}
delivers mail with
\protect\hypertarget{part0026_split_058.htmlux5cux23_idIndexMarker2693}{}{}\protect\hypertarget{part0026_split_058.htmlux5cux23_idIndexMarker2694}{}{}LMTP,
the Local Mail Transfer Protocol defined in RFC2033. LMTP is derived
from SMTP, but the protocol has been modified so that the mail server is
not required to manage a mail queue. This mailer is particularly useful
for delivering email to mailbox servers such as the Cyrus IMAP suite.

\protect\hypertarget{part0026_split_058.htmlux5cux23_idIndexMarker2695}{}{}{local}'s
job is to deliver email locally. It resolves addresses in the
\protect\hypertarget{part0026_split_058.htmlux5cux23_idIndexMarker2696}{}{}{aliases}
table and follows instructions found in recipients' {.forward} files.
Messages are forwarded to another address, passed to an external program
for processing, or stored in users' mail folders.

The {virtual} program delivers email to ``virtual mailboxes''; that is,
mailboxes that are not related to a local UNIX account but that still
represent valid email destinations. Finally, {pipe} implements delivery
through external programs.

\protect\hypertarget{part0026_split_059.html}{}{}

\hypertarget{part0026_split_059.htmlux5cux23_idContainer1247}{}
\hypertarget{part0026_split_059.htmlux5cux23calibre_pb_58}{%
\subsection[Security]{\texorpdfstring{Securi\protect\hypertarget{part0026_split_059.htmlux5cux23_idTextAnchor1170}{}{}ty}{Security}}\label{part0026_split_059.htmlux5cux23calibre_pb_58}}

\protect\hypertarget{part0026_split_059.htmlux5cux23_idIndexMarker2697}{}{}\protect\hypertarget{part0026_split_059.htmlux5cux23_idIndexMarker2698}{}{}Postfix
implements security at several levels. Most of the Postfix server
programs can run in a {chroot}ed environment. They are separate programs
with no {parent/child} relationship. None of them are setuid. The mail
drop directory is group-writable by the postdrop group, to which the
{postdrop}
\protect\hypertarget{part0026_split_059.htmlux5cux23_idIndexMarker2699}{}{}program
is setgid.

\protect\hypertarget{part0026_split_060.html}{}{}

\hypertarget{part0026_split_060.htmlux5cux23_idContainer1247}{}
\hypertarget{part0026_split_060.htmlux5cux23calibre_pb_59}{%
\subsection[Postfix commands and
documentation]{\texorpdfstring{\protect\hypertarget{part0026_split_060.htmlux5cux23_idTextAnchor1171}{}{}Postfix
commands and
documentation}{Postfix commands and documentation}}\label{part0026_split_060.htmlux5cux23calibre_pb_59}}

\protect\hypertarget{part0026_split_060.htmlux5cux23_idIndexMarker2700}{}{}Several
command-line utilities permit user interaction with the mail system:

\begin{itemize}
\item
  \protect\hypertarget{part0026_split_060.htmlux5cux23_idIndexMarker2701}{}{}{postalias}
  -- builds, modifies, and queries alias tables
\item
  \protect\hypertarget{part0026_split_060.htmlux5cux23_idIndexMarker2702}{}{}{postcat}
  -- prints the contents of queue files
\item
  \protect\hypertarget{part0026_split_060.htmlux5cux23_idIndexMarker2703}{}{}{postconf}
  -- displays and edits the main configuration file, {main.cf}
\item
  \protect\hypertarget{part0026_split_060.htmlux5cux23_idIndexMarker2704}{}{}{postfix}
  -- starts and stops the mail system (must be run as root)
\item
  \protect\hypertarget{part0026_split_060.htmlux5cux23_idIndexMarker2705}{}{}{postmap}
  -- builds, modifies, or queries lookup tables
\item
  \protect\hypertarget{part0026_split_060.htmlux5cux23_idIndexMarker2706}{}{}{postsuper}
  -- manages mail queues
\item
  {sendmail}, {mailq}, {newaliases} -- are {sendmail}-compatible
  replacements
\end{itemize}

The Postfix distribution includes a set of man pages that describe all
the programs and their options. On-line documents at postfix.org explain
how to configure and manage various aspects of Postfix. These documents
are also included in the Postfix distribution in the {README\_FILES}
directory.

\protect\hypertarget{part0026_split_061.html}{}{}

\hypertarget{part0026_split_061.htmlux5cux23_idContainer1247}{}
\hypertarget{part0026_split_061.htmlux5cux23calibre_pb_60}{%
\subsection[Postfix
configuration]{\texorpdfstring{\protect\hypertarget{part0026_split_061.htmlux5cux23_idTextAnchor1172}{}{}Postfix
configuration}{Postfix configuration}}\label{part0026_split_061.htmlux5cux23calibre_pb_60}}

\protect\hypertarget{part0026_split_061.htmlux5cux23_idIndexMarker2707}{}{}The
\protect\hypertarget{part0026_split_061.htmlux5cux23_idIndexMarker2708}{}{}{main.cf}
file is Postfix's principal configuration file. The
\protect\hypertarget{part0026_split_061.htmlux5cux23_idIndexMarker2709}{}{}{master.cf}
file configures the server programs. It also defines various lookup
tables that are referenced from {main.cf} and that provide different
types of service mappings.

The {postconf}(5) man page describes every parameter you can set in the
{main.cf} file. There is also a {postconf} program, so if you just type
{man postconf}, you'll get the man page for that instead of
{postconf}(5). Use {man -s 5 postconf} to get the right version.

The Postfix configuration language looks a bit like a series of {sh}
comments and assignment statements. Variables can be referenced in the
definition of other variables by being prefixed with a {\$}. Variable
definitions are stored just as they appear in the config file; they are
not expanded until they are used, and any substitutions occur at that
time.

You can create new variables by assigning values to them. Be careful to
choose names that do not conflict with existing configuration variables.

All Postfix configuration files, including the lookup tables, consider
lines starting with whitespace to be continuation lines. This convention
results in readable configuration files, but you must start new lines in
column one.

\subsubsection[What to put in
{main.cf}]{\texorpdfstring{\protect\hypertarget{part0026_split_061.htmlux5cux23_idTextAnchor1173}{}{}What
to put in {main.cf}}{What to put in main.cf}}

More than 500 parameters can be specified in the {main.cf} file.
However, just a few of them need to be set at an average site. The
author of Postfix strongly recommends that only parameters with
nondefault values be included in your configuration. That way, if the
default value of a parameter changes in the future, your configuration
will automatically adopt the new value.

The sample {main.cf} file that comes with the distribution includes many
{commented}-out example parameters, along with some brief documentation.
The original version is best left alone as a reference. Start with an
empty file for your own configuration so that your settings do not
become lost in a sea of comments.

\subsubsection[Basic
settings]{\texorpdfstring{\protect\hypertarget{part0026_split_061.htmlux5cux23_idTextAnchor1174}{}{}Basic
settings}{Basic settings}}

The simplest possible Postfix configuration is an empty file.
Surprisingly, this is a perfectly reasonable setup. It results in a mail
server that delivers email locally within the same domain as the local
hostname and that sends any messages directed to nonlocal addresses
directly to the appropriate remote servers.

\subsubsection[Null
client]{\texorpdfstring{\protect\hypertarget{part0026_split_061.htmlux5cux23_idTextAnchor1175}{}{}Null
client}{Null client}}

\protect\hypertarget{part0026_split_061.htmlux5cux23_idIndexMarker2710}{}{}Another
simple configuration is a ``null client''; that is, a system that
doesn't deliver email locally but rather forwards outbound mail to a
designated central server. To implement this configuration, you define
several parameters, starting with {mydomain}, which defines the domain
part of the hostname, and {myorigin}, which is the mail domain appended
to unqualified email addresses. If these two parameters are the same,
you can write something like this:

%\includegraphics{images/00886.gif}

Another parameter you should set is {mydestination}, which specifies the
mail domains that are local. If the recipient address of a message has
{mydestination} as its mail domain, the message is delivered through the
{local} program to the corresponding user (assuming that no relevant
alias or {.forward} file is found). If more than one mail domain is
included in{ mydestination}, these domains are all considered aliases
for the same domain.

For a null client, you want no local delivery, so leave this parameter
empty:

%\includegraphics{images/00887.gif}

Finally, the {relayhost} parameter tells Postfix to send all nonlocal
messages to a specified host instead of sending them directly to their
apparent destinations:

%\includegraphics{images/00888.gif}

The square brackets tell Postfix to treat the specified string as a
hostname (DNS A record) instead of a mail domain name (DNS MX record).

Since null clients should not receive mail from other systems, the last
thing to do in a null client configuration is to comment out the {smtpd}
line in the {master.cf} file. This change prevents Postfix from running
{smtpd} at all. With just these few lines, you've defined a fully
functional null client!

For a ``real'' mail server, you'll need a few more configuration options
as well as some mapping tables. We cover these in the next few sections.

\subsubsection[Use of
{postconf}]{\texorpdfstring{\protect\hypertarget{part0026_split_061.htmlux5cux23_idTextAnchor1176}{}{}Use
of {postconf}}{Use of postconf}}

{\protect\hypertarget{part0026_split_061.htmlux5cux23_idIndexMarker2711}{}{}}{postconf
}is a handy tool that helps you configure Postfix. When run without
arguments, it prints all the parameters as they are currently
configured. If you name a specific parameter as an argument, {postconf}
prints the value of that parameter. The {-d} option makes {postconf}
print the defaults instead of the currently configured values. For
example:

%\includegraphics{images/00889.gif}

Another useful option is {-n}, which tells {postconf} to print only the
parameters that differ from the default. If you ask for help on the
Postfix mailing list, that's the configuration information you should
put in your email.

\subsubsection[Lookup
tables]{\texorpdfstring{\protect\hypertarget{part0026_split_061.htmlux5cux23_idTextAnchor1177}{}{}Lookup
tables}{Lookup tables}}

\protect\hypertarget{part0026_split_061.htmlux5cux23_idIndexMarker2712}{}{}Many
aspects of Postfix's behavior are shaped through the use of lookup
tables, which can map keys to values or implement simple lists. For
example, the default setting for the {alias\_maps} table is

%\includegraphics{images/00890.gif}

Data sources are specified with the notation {type:path}. Multiple
values can be separated by commas, spaces, or both.
\protect\hyperlink{part0026_split_061.htmlux5cux23_idTextAnchor1178}{Table
18.19} lists the available data sources; {postconf -m} shows this
information as well.

\paragraph[{Table 18.19: }Information sources for Postfix lookup
tables]{\texorpdfstring{{Table 18.19:
}\protect\hypertarget{part0026_split_061.htmlux5cux23_idTextAnchor1178}{}{}\protect\hypertarget{part0026_split_061.htmlux5cux23_idTextAnchor1179}{}{}Information
sources for Postfix lookup
tables}{Table 18.19: Information sources for Postfix lookup tables}}

%\includegraphics{images/00891.gif}

The {dbm} and {sdbm} types are only for compatibility with the
traditional {sendmail} alias table. Berkeley DB ({hash}) is a more
modern implementation; it's safer and faster. If compatibility is not a
problem, then go
with\protect\hypertarget{part0026_split_061.htmlux5cux23_idIndexMarker2713}{}{}\protect\hypertarget{part0026_split_061.htmlux5cux23_idIndexMarker2714}{}{}

%\includegraphics{images/00892.gif}

The {alias\_database} specifies the table that is rebuilt by
\protect\hypertarget{part0026_split_061.htmlux5cux23_idIndexMarker2715}{}{}{newaliases}
and should correspond to the table that you specify in {alias\_maps}.
The two parameters are separate because {alias\_maps} might include
non-DB sources such as {mysql} that never need to be rebuilt.

All DB-class tables ({dbm}, {sdbm}, {hash}, and {btree}) compile a text
file to an {efficiently} searchable binary format. The syntax for these
text files is similar to that of the configuration files with respect to
comments and continuation lines. Entries are specified as simple
key/value pairs separated by whitespace, except for alias tables, which
use a colon after the key to retain {sendmail} compatibility. For
example, the following lines are appropriate for an alias table:

%\includegraphics{images/00893.gif}

As another example, here's an access table for relaying mail from any
client with a hostname ending in cs.colorado.edu.

%\includegraphics{images/00894.gif}

Text files are compiled to their binary formats with the {postmap}
command for normal tables and the {postalias} command for alias tables.
The table specification (including the type) must be given as the first
argument. For example:

%\includegraphics{images/00895.gif}

{postmap} can also query values in a lookup table (no match = no
output):

%\includegraphics{images/00896.gif}

\subsubsection[Local
delivery]{\texorpdfstring{\protect\hypertarget{part0026_split_061.htmlux5cux23_idTextAnchor1180}{}{}Local
delivery}{Local delivery}}

The
\protect\hypertarget{part0026_split_061.htmlux5cux23_idIndexMarker2716}{}{}{local}
program delivers mail to local recipients. It also handles local
aliasing. For example, if {mydestination }is set to cs.colorado.edu and
email arrives for the recipient evi@cs.colorado.edu, {local} first
consults the {alias\_maps} tables and then substitutes any matching
entries recursively.

If no aliases match, {local} looks for a
\protect\hypertarget{part0026_split_061.htmlux5cux23_idIndexMarker2717}{}{}{.forward}
file in user evi's home directory and follows the instructions in this
file if it exists. (The syntax is the same as for the right side of an
alias map.) Finally, if no {.forward} file is found, the email is
delivered to evi's local mailbox.

By default, {local} writes to standard {mbox}-format files under
{/var/mail}. You can change that behavior with the parameters shown in
\protect\hyperlink{part0026_split_061.htmlux5cux23_idTextAnchor1181}{Table
18.20}.

\paragraph[{Table 18.20: }Parameters for local mailbox delivery (set in
{main.cf})]{\texorpdfstring{{Table 18.20:
}\protect\hypertarget{part0026_split_061.htmlux5cux23_idTextAnchor1181}{}{}\protect\hypertarget{part0026_split_061.htmlux5cux23_idTextAnchor1182}{}{}Parameters
for local mailbox delivery (set in
{main.cf}){\protect\hypertarget{part0026_split_061.htmlux5cux23_idIndexMarker2718}{}{}\protect\hypertarget{part0026_split_061.htmlux5cux23_idIndexMarker2719}{}{}\protect\hypertarget{part0026_split_061.htmlux5cux23_idIndexMarker2720}{}{}\protect\hypertarget{part0026_split_061.htmlux5cux23_idIndexMarker2721}{}{}\protect\hypertarget{part0026_split_061.htmlux5cux23_idIndexMarker2722}{}{}}}{Table 18.20: Parameters for local mailbox delivery (set in main.cf)}}

%\includegraphics{images/00897.gif}

The {mail\_spool\_directory} and {home\_mailbox} options normally
generate {mbox}-format mailboxes, but they can also produce {Maildir}
mailboxes. To request this behavior, add a slash to the end of the
pathname.

If {recipient\_delimiter} is {+}, mail addressed to
evi+{whatever}@cs.colorado.edu is accepted for delivery to the evi
account. With this facility, users can create special-purpose addresses
and sort their mail by destination address. Postfix first attempts
lookups on the full address, and only if that fails does it strip the
extended components and fall back to the base address. Postfix also
looks for a corresponding forwarding file, {.forward+}{whatever}, for
further aliasing.

\protect\hypertarget{part0026_split_062.html}{}{}

\hypertarget{part0026_split_062.htmlux5cux23_idContainer1247}{}
\hypertarget{part0026_split_062.htmlux5cux23calibre_pb_61}{%
\subsection[Virtual
domains]{\texorpdfstring{\protect\hypertarget{part0026_split_062.htmlux5cux23_idTextAnchor1183}{}{}\protect\hypertarget{part0026_split_062.htmlux5cux23_idTextAnchor1184}{}{}Virtual
domains}{Virtual domains}}\label{part0026_split_062.htmlux5cux23calibre_pb_61}}

\protect\hypertarget{part0026_split_062.htmlux5cux23_idIndexMarker2723}{}{}To
host a mail domain on your Postfix mail server, you have three choices:

\begin{itemize}
\item
  List the domain in {mydestination}. Delivery is performed as described
  above: aliases are expanded and mail is delivered to the corresponding
  accounts.
\item
  List the domain in the {virtual\_alias\_domains} parameter. This
  option gives the domain its own addressing namespace that is
  independent of the system's user accounts. All addresses within the
  domain must be resolvable (through mapping) to real addresses outside
  of it.
\item
  List the domain in the {virtual\_mailbox\_domains} parameter. As with
  the {virtual\_alias\_domains} option, the domain has its own
  namespace. All mailboxes must live beneath a specified directory.
\end{itemize}

List the domain in only one of these three places. Choose carefully,
because many configuration elements depend on that choice. We have
already reviewed the handling of the {mydestination} method. The other
options are discussed below.

\subsubsection[Virtual alias
domains]{\texorpdfstring{\protect\hypertarget{part0026_split_062.htmlux5cux23_idTextAnchor1185}{}{}Virtual
alias domains}{Virtual alias domains}}

If a domain is listed as a value of the {virtual\_alias\_domains}
parameter, mail to that domain is accepted by Postfix and must be
forwarded to an actual recipient either on the local machine or
elsewhere.

The forwarding for addresses in the virtual domain must be defined in a
lookup table included in the {virtual\_alias\_maps} parameter. Entries
in the table have the address in the virtual domain on the left side and
the actual destination address on the right. An unqualified name on the
right is interpreted as a local username.

Consider the following example from {main.cf}:

%\includegraphics{images/00898.gif}

In {/etc/mail/admin.com/virtual} we could then have the lines

%\includegraphics{images/00899.gif}

Mail for evi@admin.com would be redirected to evi@cs.colorado.edu
({myorigin} is appended) and would ultimately be delivered to the
mailbox of user evi because cs.colorado.edu is included in
{mydestination}.

Definitions can be recursive: the right hand side can contain addresses
that are further defined on the left hand side. Note that the right hand
side can only be a list of addresses. To execute an external program or
to use {:include:} files, redirect the email to an alias, which can then
be expanded according to your needs.

To keep everything in one file, set {virtual\_alias\_domains} to the
same lookup table as {virtual\_alias\_maps} and put a special entry in
the table to mark it as a virtual alias domain. In
{main.cf}:\protect\hypertarget{part0026_split_062.htmlux5cux23_idIndexMarker2724}{}{}

%\includegraphics{images/00900.gif}

In {/etc/mail/admin.com/virtual}:

%\includegraphics{images/00901.gif}

The right hand side of the entry for the mail domain (admin.com) is
never actually used; admin.com's existence in the table as an
independent entry is enough to make Postfix consider it a virtual alias
domain.

\subsubsection[Virtual mailbox
domains]{\texorpdfstring{\protect\hypertarget{part0026_split_062.htmlux5cux23_idTextAnchor1186}{}{}Virtual
mailbox domains}{Virtual mailbox domains}}

Domains listed under {virtual\_mailbox\_domains} are similar to local
domains, but the list of users and their corresponding mailboxes must be
managed independently of the system's user accounts.

The parameter {virtual\_mailbox\_maps} points to a table that lists all
valid users in the domain. The map format is

%\includegraphics{images/00902.gif}

If the path ends with a slash, the mailboxes are stored in {Maildir}
format. The value of {virtual\_mailbox\_base} is always prefixed to the
specified paths.

You often want to alias some of the addresses in the virtual mailbox
domain. A {virtual\_alias\_map} will do that for you. Here is a complete
example. In
{main.cf}:\protect\hypertarget{part0026_split_062.htmlux5cux23_idIndexMarker2725}{}{}

%\includegraphics{images/00903.gif}

{/etc/mail/admin.com/vmailboxes} might contain entries like these:

%\includegraphics{images/00904.gif}

{/etc/mail/admin.com/valiases} might contain:

%\includegraphics{images/00905.gif}

You can use virtual alias maps even on addresses that are not within
virtual alias domains. Virtual alias maps let you redirect any address
from any domain, independently of the type of the domain (canonical,
virtual alias, or virtual mailbox). Since mailbox paths can only be put
on the right hand side of the virtual mailbox map, this mechanism is the
only way to set up aliases in that domain.

\protect\hypertarget{part0026_split_063.html}{}{}

\hypertarget{part0026_split_063.htmlux5cux23_idContainer1247}{}
\hypertarget{part0026_split_063.htmlux5cux23calibre_pb_62}{%
\subsection[Access
control]{\texorpdfstring{\protect\hypertarget{part0026_split_063.htmlux5cux23_idTextAnchor1187}{}{}Access
control}{Access control}}\label{part0026_split_063.htmlux5cux23calibre_pb_62}}

\protect\hypertarget{part0026_split_063.htmlux5cux23_idIndexMarker2726}{}{}Mail
servers should relay mail for third parties only on behalf of trusted
clients. If a mail server forwards mail from unknown clients to other
servers, it is a so-called open relay, which is bad. See
\protect\hyperlink{part0026_split_037.htmlux5cux23_idTextAnchor1093}{this
page} for more details.

Fortunately, Postfix doesn't act as an open relay by default. In fact,
its defaults are quite restrictive; you are more likely to need to
liberalize the permissions than to tighten them. Access control for SMTP
transactions is configured in Postfix through ``access restriction
lists.'' The parameters shown in
\protect\hyperlink{part0026_split_063.htmlux5cux23_idTextAnchor1188}{Table
18.21} control what should be checked during the different phases of an
SMTP session.

\paragraph[{Table 18.21: }Postfix parameters for SMTP access
restriction]{\texorpdfstring{{Table 18.21:
}\protect\hypertarget{part0026_split_063.htmlux5cux23_idTextAnchor1188}{}{}Postfix
parameters for SMTP access
restriction{\protect\hypertarget{part0026_split_063.htmlux5cux23_idIndexMarker2727}{}{}}}{Table 18.21: Postfix parameters for SMTP access restriction}}

%\includegraphics{images/00906.gif}

The most important parameter is {smtpd\_recipient\_restrictions}. That's
because access control is most easily performed when the recipient
address is known and can be identified as being local or not. All other
parameters in
\protect\hyperlink{part0026_split_063.htmlux5cux23_idTextAnchor1188}{Table
18.21} are empty in the default configuration. The default value is

%\includegraphics{images/00907.gif}

Each of the specified restrictions is tested in turn until a definitive
decision about what to do with the mail is reached.
\protect\hyperlink{part0026_split_063.htmlux5cux23_idTextAnchor1189}{Table
18.22} shows the common restrictions.

\paragraph[{Table 18.22: }Common Postfix access
restrictions]{\texorpdfstring{{Table 18.22:
}\protect\hypertarget{part0026_split_063.htmlux5cux23_idTextAnchor1189}{}{}\protect\hypertarget{part0026_split_063.htmlux5cux23_idTextAnchor1190}{}{}Common
Postfix access
restrictions{\protect\hypertarget{part0026_split_063.htmlux5cux23_idIndexMarker2728}{}{}\protect\hypertarget{part0026_split_063.htmlux5cux23_idIndexMarker2729}{}{}\protect\hypertarget{part0026_split_063.htmlux5cux23_idIndexMarker2730}{}{}}}{Table 18.22: Common Postfix access restrictions}}

%\includegraphics{images/00908.gif}

Everything can be tested in these restrictions, not just specific
information like the sender address in the
{smtpd\_sender\_restrictions}. Therefore, for simplicity, you might want
to put all the restrictions under a single parameter. Make that
{smtpd\_recipient\_restrictions }because it is the only one that can
test everything (except the DATA part).

{smtpd\_recipient\_restrictions} and {smtpd\_relay\_restrictions} are
where mail relaying is tested. Keep the {reject\_unauth\_destination}
restriction and carefully choose the ``permit'' restrictions before it.

\subsubsection[Access
tables]{\texorpdfstring{\protect\hypertarget{part0026_split_063.htmlux5cux23_idTextAnchor1191}{}{}Access
tables}{Access tables}}

Each restriction returns one of the actions shown in
\protect\hyperlink{part0026_split_063.htmlux5cux23_idTextAnchor1192}{Table
18.23}. Access tables are used in restrictions such as
{check\_client\_access} and {check\_recipient\_access} to select an
action according to the client host address or recipient address,
respectively.

\paragraph[{Table 18.23: }Actions for access
tables]{\texorpdfstring{{Table 18.23:
}\protect\hypertarget{part0026_split_063.htmlux5cux23_idTextAnchor1192}{}{}Actions
for access tables}{Table 18.23: Actions for access tables}}

%\includegraphics{images/00909.gif}

For example, suppose you wanted to allow relaying for all machines
within the {cs.colorado.edu} domain and that you wanted to allow only
trusted clients to post to the internal mailing list
{newsletter@cs.colorado.edu}. You could implement these policies with
the following lines in
{main.cf}:\protect\hypertarget{part0026_split_063.htmlux5cux23_idIndexMarker2731}{}{}

%\includegraphics{images/00910.gif}

Note that commas are optional when the list of values for a parameter is
specified.

In
{/}{\protect\hypertarget{part0026_split_063.htmlux5cux23_idIndexMarker2732}{}{}}{etc/postfix/relaying\_access}:

%\includegraphics{images/00911.gif}

In
\protect\hypertarget{part0026_split_063.htmlux5cux23_idIndexMarker2733}{}{}{/etc/postfix/restricted\_recipients}:

%\includegraphics{images/00912.gif}

The text after {REJECT} is an optional string that is sent to the client
along with the error code. It tells the sender why the mail was
rejected.

\subsubsection[Authentication of clients and
encryption]{\texorpdfstring{\protect\hypertarget{part0026_split_063.htmlux5cux23_idTextAnchor1193}{}{}Authentication
of clients and encryption}{Authentication of clients and encryption}}

\protect\hypertarget{part0026_split_063.htmlux5cux23_idIndexMarker2734}{}{}For
users sending mail from home, it is usually easiest to route outgoing
mail through the home ISP's mail server, regardless of the sender
address that appears on that mail. Most ISPs trust their direct clients
and allow relaying. If this configuration isn't possible or if you are
using a system such as Sender ID or SPF, ensure that mobile users
outside your network can be authorized to submit messages to your
{smtpd}.

The solution to this problem is to have the SMTP AUTH mechanism
authenticate directly at the SMTP level. Postfix must be compiled with
support for the SASL library to make this work. You can then configure
the feature like
this:\protect\hypertarget{part0026_split_063.htmlux5cux23_idIndexMarker2735}{}{}\protect\hypertarget{part0026_split_063.htmlux5cux23_idIndexMarker2736}{}{}

%\includegraphics{images/00913.gif}

You also need to support encrypted connections to avoid sending
passwords in clear text. {Add lines like the following to
}{main.cf}{:}\protect\hypertarget{part0026_split_063.htmlux5cux23_idIndexMarker2737}{}{}

%\includegraphics{images/00914.gif}

You need to put a properly signed certificate in {/etc/certs/smtp.pem}.
It's also a good idea to turn on encryption on outgoing SMTP
connections:

%\includegraphics{images/00915.gif}

\protect\hypertarget{part0026_split_064.html}{}{}

\hypertarget{part0026_split_064.htmlux5cux23_idContainer1247}{}
\hypertarget{part0026_split_064.htmlux5cux23calibre_pb_63}{%
\subsection[Debugging]{\texorpdfstring{\protect\hypertarget{part0026_split_064.htmlux5cux23_idTextAnchor1194}{}{}\protect\hypertarget{part0026_split_064.htmlux5cux23_idIndexMarker2738}{}{}\protect\hypertarget{part0026_split_064.htmlux5cux23_idIndexMarker2739}{}{}\protect\hypertarget{part0026_split_064.htmlux5cux23_idTextAnchor1195}{}{}Debugging}{Debugging}}\label{part0026_split_064.htmlux5cux23calibre_pb_63}}

When you have a problem with Postfix, first check the log files. The
answers to your questions are most likely there; it's just a question of
finding them. Every Postfix program normally issues a log entry for
every message it processes. For example, the trail of an outbound
message might look like this:

%\includegraphics{images/00916.gif}

As you can see, the interesting information is spread over many lines.
Note that the identifier 0E4A93688 is common to every line: Postfix
assigns a queue ID as soon as a message enters the mail system and never
changes it. Therefore, when searching the logs for the history of a
message, first concentrate on determining the message's queue ID. Once
you know that, it's easy to {grep} the logs for all the relevant
entries.

Postfix is good at logging helpful messages about problems that it
notices. However, it's sometimes difficult to spot the important lines
among the thousands of normal status messages. This is a good place to
consider using some of the tools discussed in the section
\protect\hyperlink{part0017_split_020.htmlux5cux23_idTextAnchor533}{{Management
of logs at scale}}.

\subsubsection[Looking at the
queue]{\texorpdfstring{\protect\hypertarget{part0026_split_064.htmlux5cux23_idTextAnchor1196}{}{}Looking
at the queue}{Looking at the queue}}

Another place to look for problems is the mail queue. As in the
{sendmail} system, a
\protect\hypertarget{part0026_split_064.htmlux5cux23_idIndexMarker2740}{}{}{mailq}
command prints the contents of a queue. You can use it to see if and why
a message has become stuck.

Another helpful tool is the {qshape} script that's shipped with recent
versions of Postfix. It shows summary statistics about the contents of a
queue. The output looks like this:

%\includegraphics{images/00917.gif}

{\protect\hypertarget{part0026_split_064.htmlux5cux23_idIndexMarker2741}{}{}}{qshape}
summarizes the given queue (here, the deferred queue), sorted by
recipient domain. The columns report the number of minutes the relevant
messages have been in the queue. For example, you can see that 25
messages bound for expn.com have been in the queue longer than 1,280
minutes. All the destinations in this example are suggestive of messages
having been sent from vacation scripts in response to spam.

{qshape} can also summarize by sender domain with the {-s} flag.

\subsubsection[Soft-bouncing]{\texorpdfstring{\protect\hypertarget{part0026_split_064.htmlux5cux23_idTextAnchor1197}{}{}Soft-bouncing}{Soft-bouncing}}

\protect\hypertarget{part0026_split_064.htmlux5cux23_idIndexMarker2742}{}{}If
\protect\hypertarget{part0026_split_064.htmlux5cux23_idIndexMarker2743}{}{}{soft\_bounce}
is set to {yes}, Postfix sends temporary error messages whenever it
would normally send permanent error messages such as ``user unknown'' or
``relaying denied.'' This is a great testing feature; it lets you
monitor the disposition of messages after a configuration change without
the risk of permanently losing legitimate email. Anything you reject
will eventually come back for another try. Don't forget to turn off this
feature when you are done testing or you will have to deal with every
rejected message over and over again.

\section{Recommended reading}

Rather than jumble together the references listed here, we've sorted them by MTA and topic.

\protect\hypertarget{part0026_split_066.html}{}{}

\hypertarget{part0026_split_066.htmlux5cux23_idContainer1247}{}
\hypertarget{part0026_split_066.htmlux5cux23calibre_pb_65}{%
\subsection[
references]{\texorpdfstring{{\protect\hypertarget{part0026_split_066.htmlux5cux23_idTextAnchor1203}{}{}sendmail}
references}{sendmail references}}\label{part0026_split_066.htmlux5cux23calibre_pb_65}}

{Costales, Bryan, Claus Assmann, George Jansen, and Gregory Neil
Shapiro}. {sendmail, 4th Edition.} Sebastopol, CA: O'Reilly Media, 2007.

This book is the definitive tome for {sendmail} configuration---1,300
pages' worth. It includes a sysadmin guide as well as a complete
reference section. An electronic edition is available, too. The author
mix includes two key {sendmail} developers (Claus and Greg) who enforce
technical correctness and add insight to the mix.

Installation instructions and a good description of the configuration
file are covered in the {Sendmail Installation and Operation Guide},
which can be found in the {doc/op} subdirectory of the {sendmail}
distribution. This document is quite complete, and in conjunction with
the {README} file in the {cf} directory, gives a good nuts-and-bolts
view of the {sendmail} system.

sendmail.org, sendmail.org/\textasciitilde ca, and
sendmail.org/\textasciitilde gshapiro all contain documents, HOWTOs, and
tutorials related to {sendmail}.

\protect\hypertarget{part0026_split_067.html}{}{}

\hypertarget{part0026_split_067.htmlux5cux23_idContainer1247}{}
\hypertarget{part0026_split_067.htmlux5cux23calibre_pb_66}{%
\subsection[Exim
references]{\texorpdfstring{\protect\hypertarget{part0026_split_067.htmlux5cux23_idTextAnchor1204}{}{}Exim
references}{Exim references}}\label{part0026_split_067.htmlux5cux23calibre_pb_66}}

{Hazel, Philip}. {The Exim SMTP Mail Server: Official Guide for Release
4, 2nd Edition}. Cambridge, UK: User Interface Technologies, Ltd., 2007.

{Hazel, Philip}. {Exim: The Mail Transfer Agent}. Sebastopol, CA:
O'Reilly Media, 2001.

The Exim specification is the defining document for Exim configuration.
It is quite complete and is updated with each new distribution. A text
version is included in the file {doc/spec.txt} in the distribution, and
a PDF version is available from {exim.org}. The web site also includes
several how-to documents.

\protect\hypertarget{part0026_split_068.html}{}{}

\hypertarget{part0026_split_068.htmlux5cux23_idContainer1247}{}
\hypertarget{part0026_split_068.htmlux5cux23calibre_pb_67}{%
\subsection[Postfix
references]{\texorpdfstring{\protect\hypertarget{part0026_split_068.htmlux5cux23_idTextAnchor1205}{}{}Postfix
references}{Postfix references}}\label{part0026_split_068.htmlux5cux23calibre_pb_67}}

{Dent, Kyle D}. {Postfix: The Definitive Guide}. Sebastopol, CA:
O'Reilly Media, 2003.

{Hildebrandt, Ralf, and Patrick Koetter. }{The Book of Postfix: State of
the Art Message Transport.} San Francisco, CA: No Starch Press, 2005.

This book is the best; it guides you through all the details of Postfix
configuration, even for complex environments. The authors are active in
the Postfix community and participate regularly on the postfix-users
mailing list. The book is unfortunately out of print, but used copies
are readily available.

\protect\hypertarget{part0026_split_069.html}{}{}

\hypertarget{part0026_split_069.htmlux5cux23_idContainer1247}{}
\hypertarget{part0026_split_069.htmlux5cux23calibre_pb_68}{%
\subsection[RFCs]{\texorpdfstring{\protect\hypertarget{part0026_split_069.htmlux5cux23_idTextAnchor1206}{}{}RFCs}{RFCs}}\label{part0026_split_069.htmlux5cux23calibre_pb_68}}

RFCs 5321 (updated by 7504) and 5322 (updated by 6854) are the current
versions of RFCs 821 and 822. They define the SMTP protocol and the
formats of messages and addresses for Internet email. RFCs 6531 and 6532
cover extensions for internationalized email addresses. There are
currently almost 90 email-related RFCs, too many to list here. See the
general RFC search engine at rfc-editor.org for more.
